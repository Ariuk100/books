% This LaTeX document needs to be compiled with XeLaTeX.
\documentclass[10pt]{article}
\usepackage[utf8]{inputenc}
\usepackage{ucharclasses}
\usepackage{graphicx}
\usepackage[export]{adjustbox}
\graphicspath{ {./images/} }
\usepackage{hyperref}
\hypersetup{colorlinks=true, linkcolor=blue, filecolor=magenta, urlcolor=cyan,}
\urlstyle{same}
\usepackage{amsmath}
\usepackage{amsfonts}
\usepackage{amssymb}
\usepackage[version=4]{mhchem}
\usepackage{stmaryrd}
\usepackage{caption}
\usepackage{fvextra, csquotes}
\usepackage{multirow}
\usepackage[fallback]{xeCJK}
\usepackage{polyglossia}
\usepackage{fontspec}
\usepackage{newunicodechar}
\IfFontExistsTF{Noto Serif CJK JP}
{\setCJKmainfont{Noto Serif CJK JP}}
{\IfFontExistsTF{STSong}
  {\setCJKmainfont{STSong}}
  {\IfFontExistsTF{Droid Sans Fallback}
    {\setCJKmainfont{Droid Sans Fallback}}
    {\setCJKmainfont{SimSun}}
}}

\setmainlanguage{russian}
\setotherlanguages{greek, hindi}
\IfFontExistsTF{Noto Serif Devanagari}
{\newfontfamily\hindifont{Noto Serif Devanagari}}
{\IfFontExistsTF{Kohinoor Devanagari}
  {\newfontfamily\hindifont{Kohinoor Devanagari}}
  {\IfFontExistsTF{Devanagari MT}
    {\newfontfamily\hindifont{Devanagari MT}}
    {\IfFontExistsTF{Lohit Devanagari}
      {\newfontfamily\hindifont{Lohit Devanagari}}
      {\IfFontExistsTF{FreeSerif}
        {\newfontfamily\hindifont{FreeSerif}}
        {\newfontfamily\hindifont{Arial Unicode MS}}
}}}}
\IfFontExistsTF{CMU Serif}
{\newfontfamily\lgcfont{CMU Serif}}
{\IfFontExistsTF{DejaVu Sans}
  {\newfontfamily\lgcfont{DejaVu Sans}}
  {\newfontfamily\lgcfont{Georgia}}
}
\setDefaultTransitions{\lgcfont}{}
\newfontfamily\cyrillicfont{CMU Serif}
\setDefaultTransitions{\cyrillicfont}{}
\setTransitionsForDevanagari{\hindifont}{\rmfamily}

\title{ЭЛЕКТРОМАГНЕТИЗМ. ОСНОВНЫЕ ЗАКОНЫ }

\author{Ведущие редакторы Б. И. Копылов, Т. Г. Хохлова\\
Художник Н. А. Лозинская\\
Технический редактор Е. В. Денюкова\\
Компьютерная верстка: В. А. Носенко\\
Подписано 11.02.14. Формат \(60 \times 90 / 16\).\\
Усл. печ. л. 19,94.\\
Издательство «БИНОМ. Лаборатория знаний»\\
125167, Москва, проезд Аэропорта, д. 3\\
Телефон: (499) 157-5272\\
e-mail: binom@Lbz.ru, http://www.Lbz.ru}
\date{}


%New command to display footnote whose markers will always be hidden
\let\svthefootnote\thefootnote
\newcommand\blfootnotetext[1]{%
  \let\thefootnote\relax\footnote{#1}%
  \addtocounter{footnote}{-1}%
  \let\thefootnote\svthefootnote%
}

%Overriding the \footnotetext command to hide the marker if its value is `0`
\let\svfootnotetext\footnotetext
\renewcommand\footnotetext[2][?]{%
  \if\relax#1\relax%
    \ifnum\value{footnote}=0\blfootnotetext{#2}\else\svfootnotetext{#2}\fi%
  \else%
    \if?#1\ifnum\value{footnote}=0\blfootnotetext{#2}\else\svfootnotetext{#2}\fi%
    \else\svfootnotetext[#1]{#2}\fi%
  \fi
}

\newunicodechar{⋅}{\ifmmode\cdot\else{$\cdot$}\fi}
\newunicodechar{↑}{\ifmmode\uparrow\else{$\uparrow$}\fi}

\begin{document}
\maketitle
\captionsetup{singlelinecheck=false}
\begin{center}
\includegraphics[max width=\textwidth, alt={}]{3e870b25-6320-483d-9133-cdd5212c2707-001_1885_1211_0_0}
\end{center}

\section*{И.Е. Иродов}
\section*{ЭЛЕКТРОМАГНЕТИЗМ}
\section*{основные законы}
9-е издание (электронное)

Рекомендовано\\
Министерством образования Российской Федерации в качестве учебного пособия для студентов высших учебных заведений

Деривативное электронное издание на основе печатного издания: Электромагнетизм. Основные законы / И. Е. Иродов. -9-е изд. -М. : БИНОМ. Лаборатория знаний, 2014. - 319 с. : ил.

И83 Электромагнетизм. Основные законы [Электронный peсурс] / И. Е. Иродов. -9-е изд. (эл.). -М. : БИНОМ. Лаборатория знаний, 2014. - 319 с. : ил.

ISBN 978-5-9963-2348-7\\
Книга содержит теоретический материал (основные идеи электромагнетизма), а также разбор многочисленных примеров и задач. Задачи тесно связаны с основным текстом и часто являются его развитием и дополнением. Материал книги, насколько возможно, освобожден от излишней математизации - основной акцент перенесен на физическую сторону рассматриваемых явлений.

Для студентов физических специальностей вузов.\\
УДК 004.514\\
ББК 32.973

По вопросам приобретения обращаться:\\
«БИНОМ. Лаборатория знаний»\\
Телефон: (499) 157-5272\\
e-mail: \href{mailto:binom@Lbz.ru}{binom@Lbz.ru}, \href{http://www.Lbz.ru}{http://www.Lbz.ru}

\section*{Содержание}
Предисловие к 4-му изданию ..... 6\\
Принятые обозначения ..... 7\\
Глава 1. Электростатическое поле в вакууме ..... 9\\
§ 1.1. Электрическое поле ..... 9\\
§ 1.2. Теорема Гаусса ..... 14\\
§ 1.3. Применения теоремы Гаусса ..... 17\\
§ 1.4. Теорема Гаусса в дифференциальной форме ..... 22\\
§ 1.5. Циркуляция вектора Е. Потенциал ..... 25\\
§ 1.6. Связь между потенциалом и вектором Е ..... 29\\
§ 1.7. Электрический диполь ..... 33\\
Задачи ..... 38\\
Глава 2. Проводник в электростатическом поле ..... 45\\
§ 2.1. Поле в веществе ..... 45\\
§ 2.2. Поле внутри и снаружи проводника ..... 46\\
§ 2.3. Силы, действующие на поверхность проводника ..... 49\\
§ 2.4. Свойства замкнутой проводящей оболочки ..... 51\\
§ 2.5. Общая задача электростатики. Метод изображений ..... 53\\
§ 2.6. Электроемкость. Конденсаторы ..... 57\\
Задачи ..... 60\\
Глава 3. Электрическое поле в диэлектрике ..... 68\\
§ 3.1. Поляризация диэлектрика ..... 68\\
§ 3.2. Поляризованность Р ..... 71\\
§ 3.3. Свойства поля вектора Р ..... 72\\
§ 3.4. Вектор D ..... 76\\
§ 3.5. Условия на границе ..... 80\\
§ 3.6. Поле в однородном диэлектрике ..... 84\\
Задачи ..... 86\\
Глава 4. Энергия электрического поля ..... 96\\
§ 4.1. Электрическая энергия системы зарядов ..... 96\\
§ 4.2. Энергия заряженных проводника и конденсатора ..... 100\\
§ 4.3. Энергия электрического поля ..... 102\\
§ 4.4. Система двух заряженных тел ..... 106\\
§ 4.5. Силы при наличии диэлектрика ..... 107\\
Задачи ..... 112\\
Глава 5. Постоянный электрический ток ..... 119\\
§ 5.1. Плотность тока. Уравнение непрерывности ..... 119\\
§ 5.2. Закон Ома для однородного проводника ..... 122\\
§ 5.3. Обобщенный закон Ома ..... 125\\
§ 5.4. Разветвленные цепи. Правила Кирхгофа ..... 129\\
§ 5.5. Закон Джоуля-Ленца ..... 132\\
§ 5.6. Переходные процессы в цепи с конденсатором ..... 135\\
Задачи ..... 138\\
Глава 6. Магнитное поле в вакууме ..... 145\\
§ 6.1. Сила Лоренца. Поле В ..... 145\\
§ 6.2. Закон Био-Савара ..... 148\\
§ 6.3. Основные законы магнитного поля ..... 151\\
§ 6.4. Применения теоремы о циркуляции вектора В ..... 154\\
§ 6.5. Дифференциальная форма основных законов маг- нитного поля ..... 157\\
§ 6.6. Сила Ампера ..... 159\\
§ 6.7. Момент сил, действующих на контур с током ..... 163\\
§ 6.8. Работа при перемещении контура с током ..... 165\\
Задачи ..... 167\\
Глава 7. Магнитное поле в веществе ..... 177\\
§ 7.1. Намагничение вещества. Намагниченность J ..... 177\\
§ 7.2. Циркуляция вектора J ..... 180\\
§ 7.3. Вектор H ..... 182\\
§ 7.4. Граничные условия для В и Н ..... 187\\
§ 7.5. Поле в однородном магнетике ..... 190\\
§ 7.6. Ферромагнетизм ..... 193\\
Задачи ..... 197\\
Глава 8. Относительность электрического и маг- нитного полей ..... 204\\
§ 8.1. Электромагнитное поле. Инвариантность заряда ..... 204\\
§ 8.2. Законы преобразования полей Е и В ..... 206\\
§ 8.3. Следствия из законов преобразования полей ..... 212\\
§ 8.4. Инварианты электромагнитного поля ..... 214\\
Задачи ..... 215\\
Глава 9. Электромагнитная индукция ..... 224\\
§ 9.1. Закон электромагнитной индукции. Правило Ленца ..... 224\\
§ 9.2. Природа электромагнитной индукции ..... 227\\
§ 9.3. Явление самоиндукции ..... 233\\
§ 9.4. Взаимная индукция ..... 239\\
§ 9.5. Энергия магнитного поля ..... 243\\
§ 9.6. Магнитная энергия двух контуров с токами ..... 246\\
§ 9.7. Энергия и силы в магнитном поле ..... 249\\
Задачи ..... 253\\
Глава 10. Уравнения Максвелла. Энергия электро- магнитного поля ..... 263\\
§ 10.1. Ток смещения ..... 263\\
§ 10.2. Система уравнений Максвелла ..... 267\\
§ 10.3. Свойства уравнений Максвелла ..... 271\\
§ 10.4. Энергия и поток энергии. Вектор Пойнтинга ..... 274\\
§ 10.5. Импульс электромагнитного поля ..... 278\\
Задачи ..... 281\\
Глава 11. Электрические колебания ..... 288\\
§ 11.1. Уравнение колебательного контура ..... 288\\
§ 11.2. Свободные электрические колебания ..... 291\\
§ 11.3. Вынужденные электрические колебания ..... 296\\
§ 11.4. Переменный ток ..... 301\\
Задачи ..... 304\\
Приложения ..... 311

\begin{enumerate}
  \item Единицы величин в СИ и системе Гаусса ..... 311
  \item Основные формулы электромагнетизма в СИ и системе Гаусса ..... 312
  \item Основные величины и единицы СИ ..... 315
  \item Греческий алфавит ..... 316
  \item Некоторые физические константы ..... 316\\
Предметный указатель ..... 317
\end{enumerate}

\section*{Предисловие к 4-му изданию}
Основная идея предлагаемой книги - органически совместить в одном учебном пособии изложение принципов теории и практику peшения задач. С этой целью в каждой главе сначала излагается теория соответствующего вопроса (с иллюстрацией на конкретных примерах), а затем дается разбор ряда задач, где показывается, как, по мнению автора, надо подходить к их решению. Задачи тесно связаны с основным текстом, часто являются его развитием и дополнением, поэтому работа над ними должна проводиться параллельно с изучением основного материала. Кроме того, предлагаемый набор задач должен, по замыслу автора, дать возможность учащемуся дополнительно обдумать ряд важных вопросов и помочь представить (даже если многие задачи не решать, а просто прочитать) большой диапазон приложения изучаемых идей.

При изложении теоретического материала автор стремился исключить из текста все второстепенное, с тем чтобы сконцентрировать внимание на основных законах электромагнетизма и, в частности, на вопросах наиболее трудных для понимания. Стремление изложить основные идеи кратко, доступно и вместе с тем достаточно корректно побудило автора насколько возможно освободить материал от излишней математизации и перенести основной акцент на физическую сторону рассматриваемых явлений. С этой же целью широко использованы различные модельные представления, упрощающие факторы, частные случаи, соображения симметрии и др.

Изложение ведется в СИ. Вместе с тем, учитывая достаточно широкое использование системы Гаусса, в Приложении дана сводка основных единиц и наиболее важных формул как в СИ, так и в системе Гаусса.

Курсивом выделены важнейшие положения и термины. Петит используется для материала повышенной трудности и относительно громоздких расчетов (этот материал при первом чтении можно безболезненно опустить), а также для примеров и задач.

Книга как учебное пособие рассчитана на студентов вузов с расширенной программой по физике (в рамках общего курса физики). Она может быть также полезной и преподавателям вузов.

В четвертом издании внесены некоторые дополнения и уточнения, а также исправлены неточности и опечатки, замеченные читателями. Этим читателям автор искренне признателен.\\
И. Иродов

\section*{Принятые обозначения}
Векторы обозначены жирным прямым шрифтом (например, \(\mathbf{r}, \mathbf{E}\) ); та же буква светлым шрифтом ( \(r, E\) ) означает модуль вектора.

Средние величины отмечены скобками \(\rangle\), например \(\langle\mathbf{v}\rangle,\langle P\rangle\).\\
Символы перед величинами означают:\\
\(\Delta\) - конечное приращение величины, т. е. разность ее конечного и начального значений, например \(\Delta \mathbf{E}=\mathbf{E}_{2}-\mathbf{E}_{1}, \Delta \varphi=\varphi_{2}-\varphi_{1}\);\\
d - дифференциал (бесконечно малое приращение), \(\mathrm{dE}, \mathrm{d} \varphi\);\\
\(\delta\) - элементарное значение величины, например \(\delta A\);\\
υ - пропорционально, например φ υ \(q\);\\
\(\sim\) - величина порядка... Например \(l \sim 10^{2}\) м.\\
Орты - единичные векторы:\\
\(\mathbf{e}_{x}, \mathbf{e}_{y}, \mathbf{e}_{z}\) (или \(\mathbf{i}, \mathbf{j}, \mathbf{k}\) ) - орты декартовых координат;\\
\(\mathbf{e}_{\rho}, \mathbf{e}_{\varphi}, \mathbf{e}_{z}\) - орты цилиндрических координат \(\rho, \varphi, z\);\\
\(\mathbf{n}\) - орт нормали к элементу поверхности;\\
\(\tau\) - орт касательной к контуру или границе раздела.\\
Производная по времени от произвольной функции \(f\) обозначена \(\partial f / \partial t\) или точкой, стоящей над функцией, \(f^{\dot{\prime}}\).

Интегралы любой кратности обозначены одним единственным знаком \(\int\) и различаются лишь обозначением элемента интегрирования: \(\mathrm{d} V, \mathrm{dS}, \mathrm{d} \mathbf{l}\) - элементы объема, поверхности, контура. Знак \(\oint\) - интегрирование по замкнутому контуру или по замкнутой поверхности.

Векторный оператор \(\nabla\) (набла). Операции с ним обозначены так:\\
\(\nabla \varphi\) - градиент \(\varphi(\operatorname{grad} \varphi)\),\\
\(\nabla \mathbf{E}\) - дивергенция \(\mathbf{E}(\operatorname{div} \mathbf{E})\),\\
\(\boldsymbol{\nabla} \times \mathbf{E}\) - ротор \(\mathbf{E}(\operatorname{rot} \mathbf{E})\).

\section*{Обозначения и названия единиц}
\begin{center}
\begin{tabular}{|l|l|l|}
\hline
A - ампер & Дж - джоуль & c - секунда \\
\hline
В - вольт & Кл - кулон & См - сименс \\
\hline
Вб - вебер & м - метр & ср - стерадиан \\
\hline
Вт - ватт & мин - минута & Тл - тесла \\
\hline
Гн - генри & Мкс - максвелл & Ф - фарад \\
\hline
Гс - гаусс & H - ньютон & ч - час \\
\hline
Гц - герц & Om - ом & Э - эрстед \\
\hline
дин - дина & рад - радиан & эВ - электронвольт \\
\hline
\end{tabular}
\end{center}

\section*{Десятичные приставки к названиям единиц}
\[
\begin{array}{ll}
\Gamma-\text { гига, } 10^{9} & \text { м }- \text { милли, } 10^{-3} \\
\text { M }- \text { мега, } 10^{6} & \text { мк }- \text { микро, } 10^{-6} \\
\text { к }- \text { кило, } 10^{3} & \text { н }- \text { нано, } 10^{-9} \\
\text { с } \text { санти, } 10^{-2} & \text { п }- \text { пико, } 10^{-12}
\end{array}
\]

\section*{Глава 1}
\section*{Электростатическое поле в вакууме}
\section*{§ 1.1. Электрическое поле}
Электрический заряд. В настоящее время известно, что в основе всего разнообразия явлений природы лежат четыре фундаментальных взаимодействия между элементарными частицами - сильное, электромагнитное, слабое и гравитационное. Каждый вид взаимодействия связывается с определенной характеристикой частицы. Например, гравитационное взаимодействие зависит от масс частиц, электромагнитное - от электрических зарядов.

Электрический заряд частицы является одной из основных, первичных ее характеристик. Ему присущи следующие фундаментальные свойства:

\begin{enumerate}
  \item электрический заряд существует в двух видах: как положительный, так и отрицательный;
  \item в любой электрически изолированной системе алгебраическая сумма зарядов не изменяется, это утверждение выражает закон сохранения электрического заряда;
  \item электрический заряд является релятивистски инвариантным: его величина не зависит от системы отсчета, а значит, не зависит от того, движется он или покоится.
\end{enumerate}

Эти фундаментальные свойства электрического заряда имеют, как мы увидим, далеко идущие последствия.

Электрическое поле. Согласно современным представлениям взаимодействие между зарядами осуществляется через поле. Всякий электрический заряд \(q\) изменяет определенным образом свойства окружающего его пространства - создает электрическое поле. Это поле проявляет себя в том, что помещенный в какую-либо его точку другой, «пробный», заряд испытывает действие силы.

Опыт показывает, что сила \(F\), действующая на неподвижный точечный пробный заряд \(q^{\prime}\), всегда может быть представлена как


\begin{equation*}
\mathbf{F}=q^{\prime} \mathbf{E} \tag{1.1}
\end{equation*}


где вектор \(\mathbf{E}\) называют напряженностью электрического поля в данной точке. Вектор Е, как видно из (1.1), можно определить как силу, действующую на единичный положительный неподвижный заряд. Здесь предполагается, что пробный заряд \(q^{\prime}\) должен быть достаточно малым, чтобы его внесение не вызвало заметного искажения интересующего нас поля (вследствие возможного перераспределения создающих поле зарядов).

Поле точечного заряда. Из опыта (закон Кулона) непосредственно следует, что напряженность поля неподвижного точечного заряда \(q\) на расстоянии \(r\) от него можно представить как


\begin{equation*}
\mathbf{E}=\frac{1}{4 \pi \varepsilon_{0}} \frac{q}{r^{2}} \mathbf{e}_{r}, \tag{1.2}
\end{equation*}


где \(\varepsilon_{0}\) - электрическая постоянная; \(\mathbf{e}_{r}\) - орт радиуса-вектора \(\mathbf{r}\), проведенного из центра поля, в котором расположен заряд \(q\), до интересующей нас точки. Формула (1.2) записана в СИ. Здесь коэффициент

\[
1 / 4 \pi \varepsilon_{0}=9 \cdot 10^{9} \mathrm{M} / \Phi,
\]

заряд \(q\) определяют в кулонах (Кл), напряженность поля \(\mathbf{E}\) - в вольтах на метр (В/м). В зависимости от знака заряда \(q\) вектор \(\mathbf{E}\) направлен так же, как и \(\mathbf{r}\), или противоположно ему.

По существу формула (1.2) выражает не что иное, как закон Кулона, но в «полевой» форме. Весьма важно, что напряженность \(\mathbf{E}\) поля точечного заряда обратно пропорциональна квадрату расстояния \(r\). Вся совокупность экспериментальных фактов показывает, что этот закон справедлив для расстояний от \(10^{-13} \mathrm{~cm}\) до нескольких километров, и пока нет никаких оснований ожидать, что этот закон не выполняется и при больших расстояниях.

Заметим еще, что в поле, создаваемом неподвижным точечным зарядом, сила, действующая на пробный заряд, не зависит от того, покоится пробный заряд или движется. Это относится и к системе неподвижных зарядов.

Принцип суперпозиции. Другой опытный факт, кроме закона (1.2), заключается в том, что напряженность поля системы точечных неподвижных зарядов равна векторной сумме напряженностей полей, которые создавали бы каждый из зарядов в отдельности:


\begin{equation*}
\mathbf{E}=\sum \mathbf{E}_{i}=\frac{1}{4 \pi \varepsilon_{0}} \sum \frac{q_{i}}{r_{i}^{2}} \mathbf{e}_{r i}, \tag{1.3}
\end{equation*}


где \(r_{i}\) - расстояние между зарядом \(q_{i}\) и интересующей нас точкой поля.

Это утверждение называют принципом суперпозиции (наложения) электрических полей. Он выражает одно из самых замечательных свойств полей и позволяет вычислять напряженность поля любой системы зарядов, представив ее в виде совокупности точечных зарядов, вклад каждого из которых дается формулой (1.2).

Распределение зарядов. Для упрощения математических расчетов во многих случаях бывает удобно игнорировать тот факт, что заряды имеют дискретную структуру (электроны, ядра), и считать, что они «размазаны» определенным образом в пространстве. Другими словами, удобно заменить истинное распределение точечных дискретных зарядов фиктивным непрерывным распределением. Это позволяет значительно упрощать расчеты, не внося сколько-нибудь значительной ошибки.

При переходе к непрерывному распределению вводят понятие о плотности зарядов - объемной \(\rho\), поверхностной \(\sigma\) и линейной \(\lambda\). По определению,


\begin{equation*}
\rho=\frac{\mathrm{d} q}{\mathrm{~d} V}, \quad \sigma=\frac{\mathrm{d} q}{\mathrm{~d} S}, \quad \lambda=\frac{\mathrm{d} q}{\mathrm{~d} l}, \tag{1.4}
\end{equation*}


где \(\mathrm{d} q\) - заряд, заключенный соответственно в объеме \(\mathrm{d} V\), на поверхности \(\mathrm{d} S\) и на длине \(\mathrm{d} l\).

С учетом этих распределений формула (1.3) может быть представлена в другой форме. Например, если заряд распределен по объему, то надо заменить \(q_{i}\) на \(\mathrm{d} q=\rho \mathrm{d} V\) и \(\sum\) на \(\int\), тогда


\begin{equation*}
\mathbf{E}=\frac{1}{4 \pi \varepsilon_{0}} \int \frac{\rho \mathbf{e}_{r} \mathrm{~d} V}{r^{2}}=\frac{1}{4 \pi \varepsilon_{0}} \int \frac{\rho \mathbf{r} \mathrm{~d} V}{r^{3}}, \tag{1.5}
\end{equation*}


где интегрирование проводится по всему пространству, в котором ρ отлично от нуля.

Таким образом, зная распределение зарядов, мы можем полностью решить задачу о нахождении напряженности электрического поля по формуле (1.3), если распределение дискретно, или по формуле (1.5) и аналогично ей, если распределение непрерывно. В общем случае расчет сопряжен со значительными трудностями (правда, не принципиального характера). Действительно, для нахождения вектора \(\mathbf{E}\) надо вычислить сначала его проекции \(E_{x}, E_{y}, E_{z}\), а это по существу три интеграла типа (1.5). И только в тех случаях, когда система зарядов обладает той или иной симметрией, задача, как правило, значительно облегчается. Приведем два примера.

Пример 1. Поле на оси тонкого равномерно заряженного кольца. Заряд \(q>0\) равномерно распределен по тонкому кольцу радиусом \(a\). Найти напряженность \(E\) электрического поля на оси кольца как функцию расстояния \(z\) от его центра.\\
Легко сообразить, что в данном случае вектор \(\mathbf{E}\) должен быть направлен по оси кольца (рис. 1.1). Выделим на кольце около точки \(A\) элемент \(\mathrm{d} l\). Запишем выражение для составляющей \(\mathrm{d} E_{z}\) от этого элемента в точке \(C\) :

\begin{figure}[h]
\begin{center}
  \includegraphics[alt={},max width=\textwidth]{3e870b25-6320-483d-9133-cdd5212c2707-013_399_335_1355_236}
\captionsetup{labelformat=empty}
\caption{Рис. 1.1}
\end{center}
\end{figure}

\[
\mathrm{d} E_{z}=\frac{1}{4 \pi \varepsilon_{0}} \frac{\lambda \mathrm{~d} l}{r^{2}} \cos \alpha,
\]

где \(\lambda=q / 2 \pi a\). Для всех элементов кольца \(r\) и \(\alpha\) будут одними и теми же, поэтому интегрирование этого выражения сводится просто к замене \(\lambda \mathrm{d} l\) на \(q\). В результате

\[
E=\frac{q}{4 \pi \varepsilon_{0}} \frac{z}{\left(a^{2}+z^{2}\right)^{3 / 2}} .
\]

Видно, что при \(z \gg a\) поле \(E \approx q / 4 \pi \varepsilon_{0} z^{2}\), т. е. на больших расстояниях эта система ведет себя как точечный заряд.

Пример 2. Поле равномерно заряженной прямой нити. Тонкая прямая нить длиной \(2 l\) заряжена равномерно зарядом \(q\). Найти напряженность \(E\) поля в точке, отстоящей на расстояние \(x\) от центра нити и расположенной симметрично относительно ее концов.

Из соображений симметрии ясно, что вектор E должен иметь направление, показанное на рис. 1.2. Это подсказывает, как надо поступить далее: определим составляющую \(\mathrm{d} E_{x}\) от элемента \(\mathrm{d} l\) нити с зарядом \(\mathrm{d} q\) и затем проинтегрируем по всем элементам нити. В нашем случае

\begin{figure}[h]
\begin{center}
  \includegraphics[alt={},max width=\textwidth]{3e870b25-6320-483d-9133-cdd5212c2707-014_504_477_387_679}
\captionsetup{labelformat=empty}
\caption{Рис. 1.2}
\end{center}
\end{figure}

\[
\mathrm{d} E_{x}=\mathrm{d} E \cos \alpha=\frac{1}{4 \pi \varepsilon_{0}} \frac{\lambda \mathrm{~d} l}{r^{2}} \cos \alpha,
\]

где \(\lambda=q / 2 l\) - линейная плотность заряда. Приведем это уравнение к виду, удобному для интегрирования. Из рис. 1.2 видно, что \(\mathrm{d} l \cos \alpha=r \mathrm{~d} \alpha\) и \(r=x / \cos \alpha\), поэтому

\[
\mathrm{d} E_{x}=\frac{1}{4 \pi \varepsilon_{0}} \frac{\lambda r \mathrm{~d} \alpha}{r^{2}}=\frac{\lambda}{4 \pi \varepsilon_{0} x} \cos \alpha \mathrm{~d} \alpha .
\]

Это выражение легко проинтегрировать:

\[
E=\frac{\lambda}{4 \pi \varepsilon_{0} x} 2 \int_{0}^{\alpha_{0}} \cos \alpha \mathrm{~d} \alpha=\frac{\lambda}{4 \pi \varepsilon_{0} x} 2 \sin \alpha_{0},
\]

где \(\alpha_{0}\) - максимальное значение угла \(\alpha, \sin \alpha_{0}=l / \sqrt{l^{2}+x^{2}}\), поэтому

\[
E=\frac{q / 2 l}{4 \pi \varepsilon_{0} x} 2 \frac{l}{\sqrt{l^{2}+x^{2}}}=\frac{1}{4 \pi \varepsilon_{0}} \frac{q}{x \sqrt{l^{2}+x^{2}}} .
\]

И здесь \(E \approx q / 4 \pi \varepsilon_{0} x^{2}\) при \(x \gg l\), как поле точечного заряда.

Геометрическое описание электрического поля. Зная вектор E в каждой точке, можно представить электрическое поле очень наглядно с помощью линий напряженности, или линий вектора Е. Эти линии проводят так, чтобы касательная к ним в каждой точке совпадала с направлением вектора E, а густота линий, т. е. число линий, пронизывающих единичную площадку, перпендикулярную линиям в данной точке, была бы пропорциональна модулю вектора Е. Кроме того, этим линиям приписывают направление, совпадающее с направлением вектора Е. По полученной картине можно легко судить о конфигурации данного электрического поля - о направлении и модуле вектора Е в разных точках поля.

Об общих свойствах поля Е. Определенное выше поле Е обладает, как выяснилось, двумя чрезвычайно важными свойствами, знание которых помогло глубже проникнуть в суть самого понятия поля и сформулировать его законы, а также открыло возможность решить ряд вопросов весьма просто и изящно. Эти свойства - так называемые теорема Гаусса и теорема о циркуляции вектора \(\mathbf{E}\) - связаны с двумя важнейшими математическими характеристиками всех векторных полей: потоком и циркуляцией. Как мы увидим, пользуясь только этими двумя понятиями, можно описать все законы не только электричества, но и магнетизма. Перейдем к последовательному рассмотрению этих свойств.

\section*{§ 1.2. Теорема Гаусса}
Поток вектора Е. Для большей наглядности воспользуемся геометрической картиной описания электрического поля (с помощью линий вектора Е) и еще, для упрощения рассуждений, будем считать, что густота линий \(\mathbf{E}\) равна модулю вектора \(\mathbf{E}\). Тогда число линий, пронизывающих элементарную площадку

\begin{figure}[h]
\begin{center}
  \includegraphics[alt={},max width=\textwidth]{3e870b25-6320-483d-9133-cdd5212c2707-015_223_310_1538_74}
\captionsetup{labelformat=empty}
\caption{Рис. 1.3}
\end{center}
\end{figure}

\(\mathrm{d} S\), нормаль \(\mathbf{n}\) которой составляет угол \(\alpha\) с вектором \(\mathbf{E}\), определяется согласно рис. 1.3 как \(E d S \cos \alpha\). Эта величина и есть поток d Ф вектора Е сквозь площадку dS. В более компактной форме

\[
\mathrm{d} \Phi=E_{n} \mathrm{~d} S=\mathbf{E} \mathrm{d} \mathbf{S},
\]

где \(E_{n}\) - проекция вектора \(\mathbf{E}\) на нормаль \(\mathbf{n}\) к площадке \(\mathrm{d} S\), dS - вектор, модуль которого равен dS , а направление совпадает с нормалью \(\mathbf{n}\) к площадке. Заметим, что выбор направления вектора \(\mathbf{n}\) (а следовательно, и dS) условен, его можно было бы направить и в противоположную сторону.

Если имеется некоторая произвольная поверхность \(S\), то поток вектора \(\mathbf{E}\) сквозь нее


\begin{equation*}
\Phi=\int_{S} \mathbf{E} \mathrm{~d} \mathbf{S} . \tag{1.6}
\end{equation*}


Эта величина алгебраическая: она зависит не только от конфигурации поля \(\mathbf{E}\), но и от выбора направления нормали. В случае замкнутых поверхностей принято нормаль \(\mathbf{n}\) брать наружу области, охватываемой этими поверхностями, т. е. выбирать внешнюю нормаль, что в дальнейшем будет всегда и подразумеваться.

Хотя здесь речь шла о потоке вектора \(\mathbf{E}\), понятие потока в равной степени относится к любому векторному полю.

Теорема Гаусса. Поток вектора E сквозь произвольную замкнутую поверхность \(S\) обладает удивительным и замечательным свойством: он зависит только от алгебраической суммы зарядов, охватываемых этой поверхностью. А именно


\begin{equation*}
\oint \mathbf{E} d \mathbf{S}=\frac{1}{\varepsilon_{0}} q_{\text {внутр }}, \tag{1.7}
\end{equation*}


где кружок у интеграла означает, что интегрирование проводится по замкнутой поверхности.

Это выражение и составляет суть теоремы Гаусса: поток вектора Е сквозь замкнутую поверхность равен алгебраической сумме зарядов внутри этой поверхности, деленной на \(\varepsilon_{0}\).

Доказательство теоремь. Сначала рассмотрим поле одного точечного заряда \(q\). Окружим этот заряд произвольной замкнутой поверхностью \(S\) (рис. 1.4) и найдем поток вектора \(\mathbf{E}\) сквозь элемент dS:


\begin{equation*}
\mathrm{d} \Phi=\mathrm{EdS}=E \mathrm{~d} S \cos \alpha=\frac{1}{4 \pi \varepsilon_{0}} \frac{q}{r^{2}} \mathrm{~d} S \cos \alpha=\frac{q}{4 \pi \varepsilon_{0}} \mathrm{~d} \Omega, \tag{1.8}
\end{equation*}


\begin{figure}[h]
\begin{center}
  \includegraphics[alt={},max width=\textwidth]{3e870b25-6320-483d-9133-cdd5212c2707-017_281_308_149_68}
\captionsetup{labelformat=empty}
\caption{Рис. 1.4}
\end{center}
\end{figure}

где \(\mathrm{d} \Omega\) - телесный угол, опирающийся на элемент поверхности \(d S\), с вершиной в точке расположения заряда \(q\). Интегрирование этого выражения по всей поверхности \(S\) эквивалентно интегрированию по всему телесному углу, т. е. замене \(\mathrm{d} \Omega\) на \(4 \pi\), и мы получим \(\Phi=q / \varepsilon_{0}\), как и требует формула (1.7).

Заметим, что при более сложной форме замкнутой поверхности углы \(\alpha\) могут быть больше \(\pi / 2\), а значит, \(\cos \alpha\) и \(\mathrm{d} \Omega\) в (1.8) принимают, вообще говоря, как положительные, так и отрицательные значения. Итак, \(\mathrm{d} \Omega\) - величина алгебраическая: если \(\mathrm{d} \Omega\) опирается на внутреннюю сторону поверхности \(S\), то \(\mathrm{d} \Omega>0\), если же на внешнюю сторону, то \(\mathrm{d} \Omega<0\).

Отсюда, в частности, следует: если заряд \(q\) расположен вне замкнутой поверхности \(S\), то поток вектора \(\mathbf{E}\) через нее равен

\begin{figure}[h]
\begin{center}
  \includegraphics[alt={},max width=\textwidth]{3e870b25-6320-483d-9133-cdd5212c2707-017_295_306_847_68}
\captionsetup{labelformat=empty}
\caption{Рис. 1.5}
\end{center}
\end{figure}

нулю. Для этого достаточно провести из заряда \(q\) коническую поверхность так, чтобы она оказалась касательной к замкнутой поверхности \(S\). Тогда интегрирование выражения (1.8) по поверхности \(S\) эквивалентно интегрированию по \(\Omega\) (рис. 1.5): внешняя сторона поверхности \(S\) будет видна из точки \(q\) под углом \(\Omega>0\), а внутренняя под углом \(-\Omega\) (оба угла по модулю равны). В сумме получим нуль, и \(\Phi=0\), что также совпадает с утверждением (1.7). На языке линий вектора \(\mathbf{E}\) это означает, что сколько линий входит в объем, ограниченный поверхностью \(S\), столько и выходит.

Теперь обратимся к случаю, когда электрическое поле создается системой точечных зарядов \(q_{1}, q_{2}\) и т. д. В этом случае согласно принципу суперпозиции \(\mathbf{E}=\mathbf{E}_{1}+\mathbf{E}_{2}+\ldots\), где \(\mathbf{E}_{1}\) - поле, создаваемое зарядом \(q_{1}\), и т. д. Тогда поток вектора \(\mathbf{E}\) можно записать так:

\[
\oint \mathbf{E} \mathrm{d} \mathbf{S}=\oint\left(\mathbf{E}_{1}+\mathbf{E}_{2}+\ldots\right) \mathrm{d} \mathbf{S}=\oint \mathbf{E}_{1} \mathrm{~d} \mathbf{S}+\oint \mathbf{E}_{2} \mathrm{~d} \mathbf{S}+\ldots=\Phi_{1}+\Phi_{2}+\ldots
\]

Согласно предыдущему каждый интеграл в правой части равен \(q_{i} / \varepsilon_{0}\), если заряд \(q_{i}\) находится внутри замкнутой поверхно-

сти \(S\), и нулю, если снаружи поверхности \(S\). Поэтому в правой части останется алгебраическая сумма только тех зарядов, которые находятся внутри поверхности \(S\).

Для завершения доказательства теоремы остается учесть случай, когда заряды распределены непрерывно с объемной плотностью, зависящей от координат. В этом случае можно считать, что каждый элементарный объем \(\mathrm{d} V\) содержит «точечный» заряд \(\rho \mathrm{d} V\). Тогда в правой части (1.7)


\begin{equation*}
q_{\text {внутр }}=\int \rho \mathrm{d} V, \tag{1.9}
\end{equation*}


где интегрирование проводится только по объему, заключенному внутри замкнутой поверхности \(S\).

Необходимо обратить внимание на следующее важное обстоятельство: в то время как само поле \(\mathbf{E}\) зависит от конфигурации все \(\boldsymbol{x}\) зарядов, поток вектора \(\mathbf{E}\) сквозь произвольную замкнутую поверхность \(S\) определяется только алгебраической суммой зарядов внутри поверхности \(S\). Это значит, что если передвинуть заряды, то поле \(\mathbf{E}\) изменится всюду, в частности, и на поверхности \(S\); изменится, вообще говоря, и поток вектора \(\mathbf{E}\) через \(S\). Однако если передвижка зарядов произошла без пересечения поверхности \(S\), поток вектора \(\mathbf{E}\) через эту поверхность останется прежним, хотя, повторяем, само поле \(\mathbf{E}\) может измениться, причем весьма существенно. Удивительное свойство электрического поля!

\section*{§ 1.3. Применения теоремы Гаусса}
Поскольку поле Е зависит от конфигурации всех зарядов, теорема Гаусса, вообще говоря, не дает возможности найти это поле. Однако в ряде случаев теорема Гаусса оказывается весьма эффективным аналитическим инструментом: она позволяет получить ответы на некоторые принципиальные вопросы, не решая задачи, а также находить и само поле E, причем чрезвычайно простым путем. Рассмотрим несколько примеров, а затем сформулируем некоторые общие выводы о том, в каких случаях применение теоремы Гаусса оказывается наиболее целесообразным.

Пример 1. О невозможности устойчивого равновесия заряда в электрическом поле. Пусть в вакууме имеется система неподвижных точечных зарядов, находящихся в равновесии. Рассмотрим один из этих зарядов - заряд \(q\). Может ли состояние его равновесия быть устойчивым?

Чтобы ответить на этот вопрос, окружим заряд \(q\) небольшой замкнутой поверхностью \(S\) (рис. 1.6). Допустим, для

\begin{figure}[h]
\begin{center}
  \includegraphics[alt={},max width=\textwidth]{3e870b25-6320-483d-9133-cdd5212c2707-019_209_292_439_235}
\captionsetup{labelformat=empty}
\caption{Рис. 1.6}
\end{center}
\end{figure}

определенности, что \(q>0\). Тогда для того чтобы равновесие заряда \(q\) было устойчивым, необходимо, чтобы во всех точках поверхности \(S\) поле \(\mathbf{E}\), образованное всеми остальньлми зарядами системы, было направлено к заряду \(q\) : только в этом случае при любом малом смещении заряда \(q\) из положения равновесия на него будет действовать возвращающая сила, и положение равновесия действительно будет устойчивым. Но такая конфигурация поля \(\mathbf{E}\) вокруг заряда \(q\) противоречит теореме Гаусса: поток вектора \(\mathbf{E}\) сквозь поверхность \(S\) отрицателен, согласно же теореме Гаусса он должен быть равным нулю, поскольку этот поток создается зарядами, расположенными вне поверхности \(S\). А равенство потока вектора \(\mathbf{E}\) нулю означает, что в каких-то точках поверхности \(S\) вектор \(\mathbf{E}\) направлен внутрь, а в каких-то обязательно наружу. Отсюда и следует, что устойчивое равновесие заряда в любом электростатическом поле невозможно.\\
Пример 2. Поле равномерно заряженной плоскости. Пусть поверхностная плотность заряда равна \(\sigma\). Из симметрии задачи очевидно, что вектор \(\mathbf{E}\) может быть только перпендикулярным заряженной плоскости. Кроме того,

\begin{figure}[h]
\begin{center}
  \includegraphics[alt={},max width=\textwidth]{3e870b25-6320-483d-9133-cdd5212c2707-019_277_358_1333_237}
\captionsetup{labelformat=empty}
\caption{Рис. 1.7}
\end{center}
\end{figure}

ясно, что в симметричных относительно этой плоскости точках вектор \(\mathbf{E}\) одинаков по модулю и противоположен по направлению. Такая конфигурация поля подсказывает, что в качестве замкнутой поверхности следует выбрать прямой цилиндр, расположенный, как на рис. 1.7 , где предполагается \(\sigma>0\).

Поток сквозь боковую поверхность этого цилиндра равен нулю, и поэтому полный поток через всю поверхность ци-

линдра будет \(2 E \Delta S\), где \(\Delta S\) - площадь каждого торца. Внутри цилиндра заключен заряд \(\sigma \Delta S\). Согласно теореме Гаусса \(2 E \Delta S=\sigma \Delta S / \varepsilon_{0}\), откуда \(E=\sigma / 2 \varepsilon_{0}\). Точнее это выражение следует записать так:


\begin{equation*}
E_{n}=\sigma / 2 \varepsilon_{0}, \tag{1.10}
\end{equation*}


где \(E_{n}\) - проекция вектора \(\mathbf{E}\) на нормаль \(\mathbf{n}\) к заряженной плоскости, причем вектор n направлен от этой плоскости. Если \(\sigma>0\), то и \(E_{n}>0\), значит, вектор \(\mathbf{E}\) направлен от заряженной плоскости, как на рис. 1.7 ; если же \(\sigma<0\), то \(E_{n}<0\), значит, вектор \(\mathbf{E}\) направлен к заряженной плоскости. Тот факт, что \(E\) не зависит от расстояния до плоскости, означает, что соответствующее электрическое поле является однородным (как слева, так и справа от плоскости).

Полученный результат справедлив только для бесконечной плоской поверхности, ибо только в этом случае могут быть использованы приведенные соображения симметрии. Однако он приближенно справедлив и для области, прилегающей к средней части конечной равномерно заряженной плоской поверхности, вдали от ее краев.

Пример 3. Поле двух параллельных плоскостей, заряженных равномерно разноименными зарядами с плотностями σ и \(-\sigma\).

Это поле можно легко найти как суперпозицию полей, создаваемых каждой из плоскостей в отдельности (рис. 1.8). Здесь верхние стрелки соответствуют полю от положительно заряженной плоскости, нижние - от отрицательно заряженной плоскости. Между плоскостями напряженности складываемых полей имеют одинаковое направление, поэтому результат (1.10) просто удвоится, и результирующая напряженность

\begin{figure}[h]
\begin{center}
  \includegraphics[alt={},max width=\textwidth]{3e870b25-6320-483d-9133-cdd5212c2707-020_277_303_1133_857}
\captionsetup{labelformat=empty}
\caption{Рис. 1.8}
\end{center}
\end{figure}

поля между плоскостями


\begin{equation*}
E=\sigma / \varepsilon_{0}, \tag{1.11}
\end{equation*}


где под \(\sigma\) подразумевается модуль поверхностной плотности заряда. Вне этой области, как легко видеть, поле равно нулю. Таким образом, поле в данном случае сосредоточено между плоскостями и является однородным в этой области.

Полученный результат приближенно справедлив и для пластин конечных размеров, если только расстояние между пластинами значительно меньше их линейных размеров (плоский конденсатор). Здесь заметные отклонения поля от однородности наблюдаются только вблизи краев пластин (этим при расчетах часто пренебрегают).\\
Пример 4. Поле бесконечного круглого цилиндра, заряженного равномерно по поверхности так, что на единицу его длины приходится заряд \(\lambda\).\\
Из соображений симметрии следует, что поле здесь имеет радиальный характер, т. е. вектор E в каждой точке перпендикулярен оси цилиндра, а модуль вектора E зависит только от расстояния \(r\) до оси цилиндра. Это подсказывает, что замкнутую поверхность здесь надо взять в форме коак-

\begin{figure}[h]
\begin{center}
  \includegraphics[alt={},max width=\textwidth]{3e870b25-6320-483d-9133-cdd5212c2707-021_377_323_711_244}
\captionsetup{labelformat=empty}
\caption{Рис. 1.9}
\end{center}
\end{figure}

сиального прямого цилиндра (рис. 1.9). Тогда поток вектора \(\mathbf{E}\) сквозь торцы этого цилиндра равен нулю, а через боковую поверхность \(E_{r} 2 \pi r h\), где \(E_{r}\) - проекция вектора \(\mathbf{E}\) на ра-диус-вектор \(\mathbf{r}\), совпадающий по направлению с нормалью \(\mathbf{n}\) к боковой поверхности цилиндра радиусом \(r\) и высотой \(h\). По теореме Гаусса для случая \(r>a\) имеем \(E_{r} 2 \pi r h=\lambda h / \varepsilon_{0}\), откуда


\begin{equation*}
E_{r}=\frac{\lambda}{2 \pi \varepsilon_{0} r^{2}}(r>a) . \tag{1.12}
\end{equation*}


При \(\lambda>0\) и \(E_{r}>0\), т. е. вектор \(\mathbf{E}\) направлен от заряженного цилиндра, и наоборот.\\
Если \(r<a\), то замкнутая поверхность не содержит внутри зарядов, поэтому в этой области \(E=0\) независимо от \(r\).\\
Таким образом, внутри равномерно заряженного по поверхности круглого бесконечного цилиндра поля нет.\\
Пример 5. Поле сферической поверхности радиусом \(a\), заряженной равномерно зарядом \(q\).\\
Это поле, очевидно, центрально-симметричное: направление вектора \(\mathbf{E}\) в любой точке проходит через центр сферы, а модуль вектора \(\mathbf{E}\) должен зависеть только от расстояния \(r\) до центра сферы. Ясно, что при такой конфигурации поля в качестве замкнутой поверхности надо взять концентриче-

скую сферу. Пусть ее радиус \(r>a\), тогда по теореме Гаусса \(E_{r} 4 \pi r^{2}=q / \varepsilon_{0}\), откуда


\begin{equation*}
E_{r}=\frac{q}{4 \pi \varepsilon_{0} r^{2}}(r>a), \tag{1.13}
\end{equation*}


где \(E_{r}\) - проекция вектора \(\mathbf{E}\) на радиус-вектор \(\mathbf{r}\), совпадающий по направлению с нормалью \(\mathbf{n}\) к поверхности в каждой ее точке. Знак заряда \(q\) и здесь определяет знак проекции \(E_{r}\), а следовательно, и направление самого вектора \(\mathbf{E}\) : от заряженной сферы (при \(q>0\) ) или к ней (при \(q<0\) ).

Если \(r<a\), то замкнутая поверхность не содержит внутри зарядов, поэтому в этой области всюду \(E=0\), т. е. внутри равномерно заряженной сферической поверхности электрическое поле отсутствует. Вне этой поверхности поле убывает с расстоянием \(r\) по такому же закону, как у точечного заряда.

\section*{Пример 6. Поле равномерно заряженного шара.}
Пусть заряд \(q\) равномерно распределен по шару радиусом \(a\). Поле такой системы, очевидно, также центрально-симметричное, поэтому и здесь для нахождения поля следует в качестве замкнутой поверхности взять концентрическую сферу. Нетрудно сообразить, что для поля вне шара получится тот же результат, что и в предыдущем примере [см. (1.13)]. Внутри же шара выражение для поля будет другим. Сфера радиусом \(r<a\) охватывает заряд \(q^{\prime}=q(r / a)^{3}\), ибо в нашем случае заряды относятся как объемы, а последние как кубы радиусов. Поэтому согласно теореме Гаусса

\[
E_{r} 4 \pi r^{2}=\frac{1}{\varepsilon_{0}} q(r / a)^{3},
\]

откуда


\begin{equation*}
E_{r}=\frac{1}{4 \pi \varepsilon_{0}} \frac{q}{a^{3}} r \quad(r \leq a), \tag{1.14}
\end{equation*}


т. е. внутри равномерно заряженного шара напряженность растет линейно с расстоянием \(r\) от его центра. График зависимости \(E\) от \(r\) показан на рис. 1.10.

\begin{figure}[h]
\begin{center}
  \includegraphics[alt={},max width=\textwidth]{3e870b25-6320-483d-9133-cdd5212c2707-022_313_427_1467_732}
\captionsetup{labelformat=empty}
\caption{Рис. 1.10}
\end{center}
\end{figure}

Общие выводы. Полученные в этих примерах результаты можно было бы найти и непосредственно интегрированием с помощью формулы (1.5). Однако, как можно было убедиться, использование теоремы Гаусса позволило нам решать эти задачи несравненно более простым путем.

Простота, с которой были решены рассмотренные задачи, может создать иллюзорное впечатление о силе метода, основанного на применении теоремы Гаусса, и о возможности находить с помощью этой теоремы решения многих других задач. К сожалению, это не так. Число задач, легко решаемых с помощью теоремы Гаусса, весьма ограничено. Уже при решении задачи о нахождении поля такого симметричного распределения заряда, как у равномерно заряженного диска, теорема Гаусса оказывается бессильной. В этом случае конфигурация поля достаточно сложная, и замкнутой поверхности, обладающей необходимой для простоты вычисления потока вектора \(\mathbf{E}\) формой, здесь нет.

Использование теоремы Гаусса для расчета полей эффективно лишь в тех случаях, где поле обладает специальной симметрией (чаще всего плоской, цилиндрической или сферической). Симметрия, а следовательно, и конфигурация поля должны быть такими, чтобы, во-первых, можно было найти достаточно простую замкнутую поверхность \(S\) и, во-вторых, вычисление потока вектора \(\mathbf{E}\) свести к простому умножению \(E\) (или \(E_{n}\) ) на площадь поверхности \(S\) или ее часть. Если этого нет, задачу о нахождении поля приходится решать или непосредственно с помощью формулы (1.5), или с помощью других методов, с которыми мы познакомимся ниже.

\section*{§ 1.4. Теорема Гаусса в дифференциальной форме}
Замечательное свойство электрического поля, которое выражает собой теорема Гаусса, побуждает представить эту теорему в иной форме, расширяющей ее возможности как инструмента исследования и расчета.

В отличие от формулы (1.7) - ее называют интегральной мы будем искать дифференциальную форму теоремы Гаусса, в которой устанавливается связь между объемной плотностью заряда \(\rho\) и изменениями напряженности \(\mathbf{E}\) в окрестности данной точки пространства.

Для этого представим сначала заряд \(q\) в объеме \(V\), охватываемом замкнутой поверхностью \(S\), как \(q_{\text {внутр }}=\langle\rho\rangle V\), где \(\langle\rho\rangle-\) среднее по объему \(V\) значение объемной плотности заряда. Затем подставим это выражение в уравнение (1.7) и разделим обе части его на \(V\). В результате получим


\begin{equation*}
\frac{1}{V} \oint \mathbf{E d} \mathbf{S}=\langle\rho\rangle / \varepsilon_{0} . \tag{1.15}
\end{equation*}


Теперь устремим объем \(V\) к нулю, стягивая его к интересующей нас точке поля. Очевидно, при этом « \(\rho\) » будет стремиться к значению \(\rho\) в данной точке поля, а значит, отношение в левой части уравнения (1.15) будет стремиться к \(\rho / \varepsilon_{0}\).

Величину, являющуюся пределом отношения \(\oint \mathbf{E} d \mathbf{S} \kappa V\) при \(V \rightarrow 0\), называют дивергенцией поля \(E\) и обозначают \(\operatorname{div} \mathbf{E}\). Таким образом, по определению


\begin{equation*}
\operatorname{div} \mathbf{E}=\lim _{V \rightarrow 0} \frac{1}{V} \oint \mathbf{E} \mathrm{~d} \mathbf{S} . \tag{1.16}
\end{equation*}


Аналогично определяется дивергенция любого другого векторного поля. Из определения (1.16) следует, что дивергенция является скалярной функцией координат.

Чтобы получить выражение для дивергенции поля \(\mathbf{E}\), надо согласно (1.16) взять бесконечно малый объем \(V\), определить поток вектора \(\mathbf{E}\) сквозь замкнутую поверхность, охватывающую этот объем, и найти отношение этого потока к объему. Полученное выражение для дивергенции будет зависеть от выбора системы координат (в разных системах координат оно оказывается разным). Например, в декартовой системе координат


\begin{equation*}
\operatorname{div} \mathbf{E}=\frac{\partial E_{x}}{\partial x}+\frac{\partial E_{y}}{\partial y}+\frac{\partial E_{z}}{\partial z} . \tag{1.17}
\end{equation*}


Итак, мы выяснили, что при \(V \rightarrow 0\) в выражении (1.15) его правая часть стремится \(\kappa \rho / \varepsilon_{0}\), а левая \(-\kappa \operatorname{div} \mathbf{E}\). Следовательно, дивергенция поля Е связана с плотностью заряда в той же точке уравнением


\begin{equation*}
\operatorname{div} \mathbf{E}=\rho / \varepsilon_{0} . \tag{1.18}
\end{equation*}


Это уравнение и выражает теорему Гаусса в дифференциальной форме.

Написание многих формул и действия с ними значительно упрощаются, если ввести векторный дифференциальный оператор \(\nabla\) («набла»). В декартовых координатах он имеет вид


\begin{equation*}
\nabla=\mathbf{i} \frac{\partial}{\partial x}+\mathbf{j} \frac{\partial}{\partial y}+\mathbf{k} \frac{\partial}{\partial z}, \tag{1.19}
\end{equation*}


где \(\mathbf{i}, \mathbf{j}, \mathbf{k}\) - орты осей \(X, Y, Z\). Сам по себе вектор \(\nabla\) смысла не имеет. Он приобретает смысл только в сочетании со скалярной или векторной функцией, на которую символически умножается. Так, например, если вектор \(\nabla\) умножить скалярно на вектор \(\mathbf{E}\), то получим

\[
\nabla \cdot \mathbf{E}=\nabla_{x} E_{x}+\nabla_{y} E_{y}+\nabla_{z} E_{z}=\frac{\partial}{\partial x} E_{x}+\frac{\partial}{\partial y} E_{y}+\frac{\partial}{\partial z} E_{z}
\]

а это есть не что иное, как \(\operatorname{div} \mathbf{E}\), согласно (1.17).\\
Таким образом, дивергенция поля \(E\) может быть записана как \(\operatorname{div} \mathbf{E}\) или \(\boldsymbol{\nabla} \cdot \mathbf{E}\) (в обоих случаях читается как «дивергенция Е»). Мы будем пользоваться вторым, более удобным обозначением. Тогда, например, теорема Гаусса (1.18) будет иметь вид


\begin{equation*}
\nabla \cdot \mathbf{E}=\rho / \varepsilon_{0} . \tag{1.20}
\end{equation*}


В дифференциальной форме теорема Гаусса является локальной теоремой: дивергенция поля \(\mathbf{E}\) в данной точке зависит только от плотности электрического заряда \(\rho\) в той же точке и больше ни от чего. Это одно из замечательных свойств электрического поля. Например, в разных точках поля точечного заряда поле Е отличается друг от друга. Это же относится, вообще говоря, и к пространственным производным \(\partial E_{x} / \partial x, \partial E_{y} / \partial y\), \(\partial E_{z} / \partial z\). Однако, как утверждает теорема Гаусса, сумма этих производных, которая определяет дивергенцию E, оказывается во всех точках поля (вне самого заряда) равной нулю.

В тех точках поля, где дивергенция \(\mathbf{E}\) положительна, мы имеем источники поля (положительные заряды), а в тех точках, где она отрицательна, - стоки (отрицательные заряды). Линии вектора \(\mathbf{E}\) выходят из источников поля, а в местах стоков они заканчиваются.

\section*{§ 1.5. Циркуляция вектора Е. Потенциал}
Теорема о циркуляции вектора Е. Из механики известно, что любое стационарное поле центральных сил является потенциальным, т. е. работа сил этого поля не зависит от пути, а зависит только от положения начальной и конечной точки. Именно таким свойством обладает электростатическое поле - поле, образованное системой неподвижных зарядов. Если в качестве пробного заряда, переносимого из точки 1 заданного поля \(\mathbf{E}\) в точку 2 , взять единичный положительный заряд, то элементарная работа сил поля на перемещении dl равна \(\mathbf{E} \mathrm{dl}\), а вся работа сил поля на пути от точки 1 до точки 2 определяется как


\begin{equation*}
\int_{1}^{2} \mathrm{E} \mathrm{~d} l . \tag{1.21}
\end{equation*}


Этот интеграл берется по некоторой линии (пути), поэтому его называют линейньм.

Как мы сейчас покажем, из независимости линейного интеграла (1.21) от пути между двумя точками следует, что по произвольному замкнутому пути этот интеграл равен нулю. Интеграл (1.21) по замкнутому пути называют циркуляцией вектора Е и обозначают \(\oint\).

Итак, мы утверждаем, что циркуляция вектора \(\mathbf{E}\) в любом электростатическом поле равна нулю, т. е.


\begin{equation*}
\oint \mathbf{E} \mathrm{d} \mathbf{l}=0 . \tag{1.22}
\end{equation*}


Это утверждение и называют теоремой о циркуляции вектора Е.

Для доказательства этой теоремы разобьем произвольный замкнутый путь на две части \(1 a 2\) и \(2 b 1\) (рис. 1.11). Так как линейный интеграл (1.21) - обозначим его \(\int_{12}\) не зависит от пути между точками 1 и 2 , то \(\int_{12}^{(a)}=\int_{12}^{(b)}\). С другой стороны, ясно, что \(\int_{12}^{(b)}=-\int_{21}^{(b)}\),

\begin{figure}[h]
\begin{center}
  \includegraphics[alt={},max width=\textwidth]{3e870b25-6320-483d-9133-cdd5212c2707-026_193_276_1540_887}
\captionsetup{labelformat=empty}
\caption{Рис. 1.11}
\end{center}
\end{figure}

где \(\int_{21}^{(b)}\) - интеграл по тому же участку \(b\), но в обратном направлении. Поэтому

\[
\int_{12}^{(a)}+\int_{21}^{(b)}=\int_{12}^{(a)}-\int_{12}^{(b)}=0,
\]

что и требовалось доказать.\\
Поле, обладающее свойством (1.22), называют потенциальным. Значит, любое электростатическое поле является потенииальным.

Теорема о циркуляции вектора \(\mathbf{E}\) позволяет сделать ряд важных выводов, практически не прибегая к расчетам. Вот два примера.

\section*{Пример 1. Линии электростатического поля Е не могут быть замкнутыми.}
В самом деле, если это не так и какая-то линия вектора \(\mathbf{E}\) замкнута, то взяв циркуляцию вектора \(\mathbf{E}\) вдоль этой линии, мы сразу же придем к противоречию с теоремой (1.22). Значит, действительно, в электростатическом поле замкнутых линий вектора \(\mathbf{E}\) не существует: линии начинаются на положительных зарядах и кончаются на отрицательных (или уходят в бесконечность),

\section*{Пример 2. Возможна ли конфигурация электростатического поля как на рис. 1.12?}
Нет, невозможна. Это сразу станет ясно, если мы применим теорему о циркуляции вектора \(\mathbf{E}\) к замкнутому конту-

\begin{figure}[h]
\begin{center}
  \includegraphics[alt={},max width=\textwidth]{3e870b25-6320-483d-9133-cdd5212c2707-027_220_315_1323_242}
\captionsetup{labelformat=empty}
\caption{Рис. 1.12}
\end{center}
\end{figure}

ру, показанному на рисунке пунктиром. Стрелки на контуре показывают направление обхода. При таком специальном выборе контура вклад в циркуляцию на вертикальных участках его равен нулю: здесь \(\mathbf{E} \perp \mathrm{dl}\) и \(\mathbf{E} \mathrm{dl}=0\); остаются два одинаковых по длине горизонтальных участка. Из рисунка сразу видно, что вклады в циркуляцию на этих участках противоположны по знаку, но не одинаковы по модулю (на верхнем участке больше, ибо линии гуще, а значит, \(E\) больше). Поэтому циркуляция оказывается отличной от нуля, что противоречит (1.22).

Потенциал. До сих пор мы рассматривали описание электрического поля с помощью вектора Е. Существует, однако, и другой адекватный способ описания - с помощью потенциала \(\varphi\) (заметим сразу, что оба эти способа однозначно соответствуют друг другу). Как мы увидим, второй способ обладает рядом существенных преимуществ.

Тот факт, что линейный интеграл (1.21), представляющий собой работу сил поля при перемещении единичного положительного заряда из точки 1 в точку 2 , не зависит от пути между этими точками, позволяет утверждать, что в электрическом поле существует некоторая скалярная функция координат \(\varphi(\mathbf{r})\), убыль которой


\begin{equation*}
\varphi_{1}-\varphi_{2}=\int_{1}^{2} \mathbf{E} \mathrm{~d} \mathbf{l} \tag{1.23}
\end{equation*}


где \(\varphi_{1}\) и \(\varphi_{2}\) - значения функции φ в точках 1 и 2. Так определенная величина \(\varphi(\mathbf{r})\) называется потенциалом поля. Из сопоставления выражения (1.23) с выражением для работы сил потенциального поля (которая равна убыли потенциальной энергии частицы в поле) можно сказать, что потенциал - это величина, численно равная потенциальной энергии единичного положительного заряда в данной точке поля.

Потенциалу какой-либо произвольной точки \(O\) поля можно условно приписать любое значение \(\varphi_{0}\). Тогда потенциалы всех других точек поля определяются согласно (1.23) однозначно. Если изменить \(\varphi_{0}\) на некоторую величину \(\Delta \varphi\), то на такую же величину изменятся и потенциалы во всех других точках поля.

Таким образом, потенциал φ определяется с точностью до произвольной аддитивной постоянной. Значение этой постоянной не играет роли, так как все электрические явления зависят только от напряженности электрического поля. Последняя же определяется, как мы увидим, не самим потенциалом в данной точке поля, а разностью потенциалов в соседних точках поля.

Единицей потенциала является вольт (В).\\
Потенциал поля точечного заряда. Формула (1.23) содержит не только определение потенциала \(\varphi\), но и способ нахождения этой функции. Для этого достаточно вычислить интеграл \(\int \mathbf{E ~ d l}\)

по любому пути между двумя точками и представить затем полученный результат в виде убыли некоторой функции, которая и есть \(\varphi(\mathbf{r})\). Можно поступить и проще. Воспользуемся тем, что формула (1.23) справедлива не только для конечных перемещений, но и для элементарных dl. Тогда согласно этой формуле элементарная убыль потенциала на этом перемещении есть


\begin{equation*}
-\mathrm{d} \varphi=\mathbf{E} \mathrm{d} \mathbf{l} . \tag{1.24}
\end{equation*}


Другими словами, если известно поле \(\mathbf{E}(\mathbf{r})\), то для нахождения φ надо представить \(\mathbf{E}\) dl (путем соответствующих преобразований) как убыль некоторой функции. Эта функция и есть φ.

Найдем таким способом потенциал поля неподвижного точечного заряда:

\[
\mathbf{E} \mathrm{d} \mathbf{l}=\frac{1}{4 \pi \varepsilon_{0}} \frac{q}{r} \mathbf{e}_{r} \mathrm{~d} \mathbf{l}=\frac{q}{4 \pi \varepsilon_{0}} \frac{d r}{r^{2}}=-\mathrm{d}\left(\frac{1}{4 \pi \varepsilon_{0}} \frac{q}{r}+\text { const }\right),
\]

где учтено, что \(\mathbf{e}_{r} \mathrm{~d} \mathbf{l}=1 \cdot(\mathrm{dl})_{r}\), ибо проекция вектора dl на вектор \(\mathbf{e}_{r}\), а значит, и на \(\mathbf{r}\) равна приращению модуля вектора \(\mathbf{r}\), т. е. dr. Величина, стоящая в круглых скобках под знаком дифференциала, и есть \(\varphi(\mathbf{r})\). Так как присутствующая здесь аддитивная константа никакой физической роли не играет, ее обычно опускают, стремясь выражение для φ сделать проще. Таким образом, потенциал поля точечного заряда


\begin{equation*}
\varphi=\frac{1}{4 \pi \varepsilon_{0}} \frac{q}{r} . \tag{1.25}
\end{equation*}


Отсутствие в этом выражении аддитивной константы означает, что мы условно полагаем потенциал на бесконечности ( \(r \rightarrow \infty\) ) равным нулю.

Потенциал поля системы зарядов. Пусть система состоит из неподвижных точечных зарядов \(q_{1}, q_{2}\) „... Согласно принципу суперпозиции в любой точке поля напряженность \(\mathbf{E}=\mathbf{E}_{1}+\mathbf{E}_{2}+ +\ldots\), где \(\mathbf{E}_{1}\) - напряженность поля заряда \(q_{1}\) и т. д. Тогда можно записать, используя формулу (1.24):

\[
\mathbf{E} \mathrm{d} \mathbf{l}=\left(\mathbf{E}_{1}+\mathbf{E}_{2}+\ldots\right) \mathrm{d} \mathbf{l}=\mathbf{E}_{1} \mathrm{~d} \mathbf{l}+\mathbf{E}_{2} \mathrm{~d} \mathbf{l}+\ldots=-\mathrm{d} \varphi_{1}-\mathrm{d} \varphi_{2}-\ldots=-\mathrm{d} \varphi,
\]

где \(\varphi=\sum \varphi_{i}\), т.е. принцип суперпозиции оказывается справедливым и для потенциала. Таким образом, потенциал системы неподвижных точечных зарядов


\begin{equation*}
\varphi=\frac{1}{4 \pi \varepsilon_{0}} \sum \frac{q_{i}}{r_{i}}, \tag{1.26}
\end{equation*}


где \(r_{i}\) - расстояние от точечного заряда \(q_{i}\) до интересующей нас точки поля. Здесь также произвольная постоянная опущена. Это полностью соответствует тому факту, что всякая реальная система зарядов ограничена в пространстве, поэтому ее потенциал на бесконечности можно принять равным нулю.

Если заряды, образующие систему, распределены непрерывно, то, как обычно, мы считаем, что каждый элементарный объем \(\mathrm{d} V\) содержит «точечный» заряд \(\rho \mathrm{d} V\), где \(\rho\) - объемная плотность заряда в месте нахождения объема \(\mathrm{d} V\). С учетом этого формуле (1.26) можно придать иной вид:


\begin{equation*}
\varphi=\frac{1}{4 \pi \varepsilon_{0}} \int \frac{\rho \mathrm{~d} V}{r}, \tag{1.27}
\end{equation*}


где интегрирование проводится или по всему пространству, или по той его части, которая содержит заряды. Если заряды расположены только на поверхности \(S\), то


\begin{equation*}
\varphi=\frac{1}{4 \pi \varepsilon_{0}} \int \frac{\sigma \mathrm{~d} S}{r}, \tag{1.28}
\end{equation*}


где \(\sigma\) - поверхностная плотность заряда; \(d S\) - элемент поверхности \(S\). Аналогичное выражение будет и в том случае, когда заряды распределены линейно.

Итак, зная распределение зарядов (дискретное, непрерывное), мы можем в принципе найти потенциал поля любой системы.

\section*{§ 1.6. Связь между потенциалом и вектором Е}
Электрическое поле, как известно, полностью описывается векторной функцией E(r). Зная ее, мы можем найти силу, дей-

ствующую на интересующий нас заряд в любой точке поля, вычислить работу сил поля при каком угодно перемещении заряда и др. А что дает введение потенциала? Прежде всего, оказывается, зная потенциал \(\varphi(\mathbf{r})\) данного электрического поля, можно достаточно просто восстановить и само поле E(r). Рассмотрим этот вопрос более подробно.

Связь между φ и \(\mathbf{E}\) можно установить с помощью уравнения (1.24). Пусть перемещение dl параллельно оси \(X\), тогда \(\mathrm{d} \mathbf{l}=\mathbf{i} \mathrm{d} x\), где \(\mathbf{i}\) - орт оси \(X, \mathrm{~d} x\) - приращение координаты \(x\),

\[
\mathbf{E} \mathrm{d} \mathbf{l}=\mathbf{E} \mathbf{i} \mathrm{d} x=E_{x} \mathrm{~d} x
\]

где \(E_{x}\) - проекция вектора \(\mathbf{E}\) на орт і (а не на перемещение dl). Сопоставив последнее выражение с формулой (1.24), получим


\begin{equation*}
E_{x}=-\partial \varphi / \partial x \tag{1.29}
\end{equation*}


где символ частной производной подчеркивает, что функцию \(\varphi(x, y, z)\) надо дифференцировать только по \(x\), считая \(y\) и \(z\) при этом постоянными.

Рассуждая аналогично, можно получить соответствующие выражения для проекций \(E_{y}\) и \(E_{z}\). А определив \(E_{x}, E_{y}, E_{z}\), легко найти и сам вектор Е:


\begin{equation*}
\mathbf{E}=-\left(\frac{\partial \varphi}{\partial x} \mathbf{i}+\frac{\partial \varphi}{\partial y} \mathbf{j}+\frac{\partial \varphi}{\partial z} \mathbf{k}\right) . \tag{1.30}
\end{equation*}


Величина, стоящая в скобках, есть не что иное, как градиент потенциала \(\varphi(\operatorname{grad} \varphi\) или \(\nabla \varphi)\). Мы будем пользоваться вторым, более удобным обозначением и рассматривать формально \(\nabla \varphi\) как произведение символического вектора \(\nabla\) на \(\varphi\). Тогда уравнение (1.30) можно представить в более компактной форме:


\begin{equation*}
\mathbf{E}=-\nabla \varphi, \tag{1.31}
\end{equation*}


т. е. напряженность \(\mathbf{E}\) поля равна со знаком минус градиенту потенциала. Это и есть та формула, с помощью которой можно восстановить поле \(\mathbf{E}\), зная функцию \(\varphi(\mathbf{r})\).

Пример. Найти напряженность \(\mathbf{E}\) поля, потенциал которого имеет вид: 1) \(\varphi(x, y)=-a x y, a\) - постоянная, 2) \(\varphi(\mathbf{r})=-\mathbf{a r}\), a - постоянный вектор, \(\mathbf{r}\) - радиус-вектор интересующей нас точки поля.

\begin{enumerate}
  \item Воспользовавшись формулой (1.30), получим \(\mathbf{E}=a(y \mathbf{i}+x \mathbf{j})\).
  \item Представим сначала функцию \(\varphi\) как \(\varphi=-a_{x} x-a_{y} y-a_{z} z\), где \(a_{x}, a_{y}, a_{z}\) - постоянные. После этого с помощью формулы (1.30) найдем \(\mathbf{E}=a_{x} \mathbf{i}+a_{y} \mathbf{j}+a_{z} \mathbf{k}=\mathbf{a}\). Видно, что поле \(\mathbf{E}\) является в данном случае однородным.
\end{enumerate}

Получим еще одну полезную формулу. В соотношении (1.24) запишем правую часть как \(\mathbf{E} \mathrm{d} \mathbf{l}=E_{l} \mathrm{~d} l\), где \(\mathrm{d} l=|\mathrm{d} \mathbf{l}|\) - элементарный путь; \(E_{l}\) - проекция вектора \(\mathbf{E}\) на перемещение dl . Отсюда


\begin{equation*}
E_{l}=-\partial \varphi / \partial l, \tag{1.32}
\end{equation*}


т. е. проекция вектора \(\mathbf{E}\) на направление перемещения dl равна со знаком минус производной потенциала по данному направлению (это подчеркнуто символом частной производной).

Эквипотенциальные поверхности. Введем понятие эквипотенциальной поверхности - поверхности, во всех точках которой потенциал φ имеет одно и то же значение. Убедимся в том, что вектор Е направлен в каждой точке по нормали к эквипотенциальной поверхности в сторону уменьшения потенциала φ. В самом деле, из формулы (1.32) следует, что проекция вектора \(\mathbf{E}\) на любое направление, касательное к эквипотенциальной поверхности в данной точке, равна нулю. А это значит, что вектор \(\mathbf{E}\) нормален к данной поверхности. Далее, возьмем перемещение dl по нормали к поверхности в сторону уменьшения \(\varphi\), тогда \(\partial \varphi<0\) и согласно (1.32) \(E_{l}>0\), т. е. вектор \(\mathbf{E}\) направлен в сторону уменьшения \(\varphi\), или в сторону, противоположную вектору \(\nabla \varphi\).

Эквипотенциальные поверхности наиболее целесообразно проводить так, чтобы разность потенциалов для двух соседних поверхностей была бы одинаковой. Тогда по густоте эквипотенциальных поверхностей можно наглядно судить о значении напряженности поля в разных точках. Там, где эти поверхности расположены гуще («круче потенциальный рельеф»), там напряженность поля больше.

Далее, ввиду того, что вектор \(\mathbf{E}\) всюду нормален к эквипотенциальной поверхности, линии вектора \(\mathbf{E}\) ортогональны этим поверхностям.

На рис. 1.13 показана двухмерная картина электрического поля: пунктиром - эквипотенциали, сплошными линиями -

\begin{figure}[h]
\begin{center}
  \includegraphics[alt={},max width=\textwidth]{3e870b25-6320-483d-9133-cdd5212c2707-033_257_328_383_88}
\captionsetup{labelformat=empty}
\caption{Рис. 1.13}
\end{center}
\end{figure}

линии вектора Е. Такое изображение придает бо́льшую наглядность. Сразу же видно, в какую сторону направлен вектор \(\mathbf{E}\), где напряженность больше, где меньше, где больше крутизна потенциального рельефа. С помощью таких картин можно получить и качественные ответы на ряд вопросов: куда начнет двигаться заряд при помещении его в ту или иную точку, где больше градиент потенциала (по модулю), в какой точке поля на заряд будет действовать большая сила и др.

О преимуществах потенциала. Ранее было отмечено, что электростатическое поле исчерпывающим образом характеризуется векторной функцией E(r). Какая же польза от введения потенциала? Существует несколько весомых причин, убедительно свидетельствующих о том, что потенциал - понятие действительно весьма полезное, и не случайно, что этим понятием широко пользуются не только в физике, но и в технике.

\begin{enumerate}
  \item Зная потенциал \(\varphi(\mathbf{r})\), можно предельно просто вычислить работу сил поля при перемещении точечного заряда \(q^{\prime}\) из точки 1 в точку 2:
\end{enumerate}


\begin{equation*}
A_{12}=q^{\prime}\left(\varphi_{1}-\varphi_{2}\right), \tag{1.33}
\end{equation*}


где \(\varphi_{1}\) и \(\varphi_{2}\) - потенциалы в точках 1 и 2. Значит, искомая работа равна убыли потенциальной энергии заряда \(q^{\prime}\) в поле при перемещении его из точки 1 в точку 2 . Расчет работы сил поля по формуле (1.33) оказывается не только проще, но в некоторых случаях и единственно возможным.\\
Пример. Заряд \(q\) распределен по тонкому кольцу радиусом \(a\). Найти работу сил поля при перемещении точечного заряда \(q^{\prime}\) из центра кольца на бесконечность.\\
Так как неизвестно, как распределен заряд \(q\) по кольцу, то ничего нельзя сказать о напряженности \(\mathbf{E}\) поля этого заряда.

\begin{displayquote}
А это значит, что непосредственно вычислить работу как интеграл \(\int q^{\prime} \mathbf{E ~ d l}\) здесь непросто. С помощью же потенциала эта задача решается элементарно. В самом деле, так как все элементы кольца находятся на одном и том же расстоянии \(a\) от центра кольца, то потенциал в этой точке \(\varphi_{0}=q / 4 \pi \varepsilon_{0} a\). А потенциал на бесконечности \(\varphi=0\). Следовательно, работа \(A= =q^{\prime} \varphi_{0}=q^{\prime} q / 4 \pi \varepsilon_{0} a\).
\end{displayquote}

\begin{enumerate}
  \setcounter{enumi}{1}
  \item Во многих случаях оказывается, что для нахождения напряженности E электрического поля легче сначала подсчитать потенциал φ и затем взять градиент от него, нежели вычислять E непосредственно. Это весьма существенное преимущество потенциала. Действительно, для вычисления φ нужно взять один интеграл, а для вычисления \(\mathbf{E}\) - три (ведь это вектор). Кроме того, обычно интегралы для определения \(\varphi\) проще, чем для \(E_{x}, E_{y}, E_{z}\).
\end{enumerate}

Сразу же заметим, что это не касается сравнительно небольшого числа задач с достаточно хорошей симметрией. В этих случаях нахождение поля \(\mathbf{E}\) непосредственно или с помощью теоремы Гаусса часто оказывается значительно проще.

\section*{§ 1.7. Электрический диполь}
Поле диполя. Электрический диполь - это система из двух одинаковых по модулю разноименных точечных зарядов \(+q\) и \(-q\), находящихся на некотором расстоянии \(l\) друг от друга. Когда говорят о поле диполя, то предполагают сам диполь точечным, т. е. считают расстояния \(r\) от диполя до интересующих нас точек поля значительно больше \(l\).

Поле диполя обладает осевой симметрией, поэтому картина поля в любой плоскости, проходящей через ось диполя, одна и та же и вектор Е лежит в этой плоскости.

Найдем сначала потенциал поля диполя, а затем его напряженность. Согласно (1.25) потенциал поля диполя в точке \(P\) (рис. 1.14, \(a\) ) определяется как

\[
\varphi=\frac{1}{4 \pi \varepsilon_{0}}\left(\frac{q}{r_{+}}-\frac{q}{r_{-}}\right)=\frac{1}{4 \pi \varepsilon_{0}} \frac{q\left(r_{-}-r_{+}\right)}{r_{+} r_{-}} .
\]

Так как \(r \gg l\), то, как видно из рис. 1.14, \(a, r_{-}-r_{+}=l \cos \vartheta\) и \(r_{+} r_{-}=r^{2}\), где \(r\) - расстояние от точки \(P\) до диполя (он точеч-

\begin{figure}[h]
\begin{center}
  \includegraphics[alt={},max width=\textwidth]{3e870b25-6320-483d-9133-cdd5212c2707-035_368_951_145_153}
\captionsetup{labelformat=empty}
\caption{Рис. 1.14}
\end{center}
\end{figure}

ный!). С учетом этого


\begin{equation*}
\varphi=\frac{1}{4 \pi \varepsilon_{0}} \frac{p \cos \vartheta}{r^{2}}, \tag{1.34}
\end{equation*}


где \(p=q l\) - электрический момент диполя. Этой величине сопоставляют вектор, направленный по оси диполя от отрицательного заряда к положительному:


\begin{equation*}
\mathrm{p}=q \mathrm{l}, \tag{1.35}
\end{equation*}


где \(q>0\) и \(\mathbf{l}\) - вектор, направленный в ту же сторону, что и \(\mathbf{p}\).\\
Из формулы (1.34) видно, что поле диполя зависит от его электрического момента р. Как мы увидим далее, и поведение диполя во внешнем поле также зависит от p. Следовательно, p является важной характеристикой диполя.

Следует также обратить внимание на то, что потенциал поля диполя убывает с расстоянием \(r\) быстрее, чем потенциал поля точечного заряда ( \(1 / r^{2}\) вместо \(1 / r\) ).

Для нахождения поля диполя воспользуемся формулой (1.32), вычислив с помощью нее проекции вектора \(\mathbf{E}\) на два взаимно перпендикулярных направления - вдоль ортов \(\mathbf{e}_{r}\) и \(\mathbf{e}_{\vartheta}\) (рис. 1.14, б):


\begin{equation*}
E_{r}=-\frac{\partial \varphi}{\partial r}=\frac{1}{4 \pi \varepsilon_{0}} \frac{2 p \cos \vartheta}{r^{3}}, \quad E_{\vartheta}=-\frac{\partial \varphi}{r \partial \vartheta}=-\frac{1}{4 \pi \varepsilon_{0}} \frac{p \sin \vartheta}{r^{3}} . \tag{1.36}
\end{equation*}


Отсюда модуль вектора \(\mathbf{E}\)


\begin{equation*}
E=\sqrt{E_{r}^{2}+E_{\vartheta}^{2}}=\frac{1}{4 \pi \varepsilon_{0}} \frac{p}{r^{3}} \sqrt{1+3 \cos ^{2} \vartheta} . \tag{1.37}
\end{equation*}


В частности, при \(\vartheta=0\) и \(\vartheta=\pi / 2\) мы получим выражения для напряженности поля соответственно на оси диполя ( \(E_{\|}\)) и перпендикулярно ей ( \(E_{\perp}\) ):


\begin{equation*}
E_{| |}=\frac{1}{4 \pi \varepsilon_{0}} \frac{2 p}{r^{3}}, \quad E_{\perp}=\frac{1}{4 \pi \varepsilon_{0}} \frac{p}{r^{3}}, \tag{1.38}
\end{equation*}


\begin{center}
\includegraphics[max width=\textwidth, alt={}]{3e870b25-6320-483d-9133-cdd5212c2707-036_43_1050_431_68}
\end{center}

\begin{figure}[h]
\begin{center}
  \includegraphics[alt={},max width=\textwidth]{3e870b25-6320-483d-9133-cdd5212c2707-036_325_748_488_233}
\captionsetup{labelformat=empty}
\caption{Рис. 1.15}
\end{center}
\end{figure}

Сила, действующая на диполь. Поместим диполь во внешнее неоднородное электрическое поле. Пусть \(\mathbf{E}_{+}\)и \(\mathbf{E}_{-}\)- напряженности внешнего поля в точках, где расположены положительный и отрицательный заряды диполя. Тогда результирующая сила \(\mathbf{F}\), действующая на диполь, равна (рис. 1.15, \(a\) ):

\[
\mathbf{F}=q \mathbf{E}_{+}-q \mathbf{E}_{-}=q\left(\mathbf{E}_{+}-\mathbf{E}_{-}\right) .
\]

Разность \(\mathbf{E}_{+}-\mathbf{E}_{-}\)- это приращение \(\Delta \mathbf{E}\) вектора \(\mathbf{E}\) на отрезке, равном длине диполя \(l\), в направлении вектора 1 . Вследствие малости этого отрезка можно записать

\[
\Delta \mathbf{E}=\mathbf{E}_{+}-\mathbf{E}_{-}=\frac{\Delta \mathbf{E}}{l} l=\frac{\partial \mathbf{E}}{\partial l} l .
\]

После подстановки этого выражения в формулу для \(\mathbf{F}\) получим, что сила, действующая на диполь:


\begin{equation*}
\mathbf{F}=p \frac{\partial \mathbf{E}}{\partial l}, \tag{1.39}
\end{equation*}


где \(p=q l\) - электрический момент диполя. Входящую в это выражение производную принято называть производной векто-

ра по направлению. Знак частной производной подчеркивает, что эта производная берется по определенному направлению направлению, совпадающему с вектором \(\mathbf{l}\) или p.

Простота формулы (1.39), к сожалению, обманчива: производная \(\partial \mathbf{E} / \partial l\) является довольно сложной математической операцией. Мы не будем останавливаться на этом более подробно, а обратим внимание на существо полученного результата. Прежде всего отметим, что в однородном поле \(\partial \mathbf{E} / \partial l=0\), поэтому и \(\mathbf{F}=0\). Значит, сила действует на диполь, вообще говоря, только в неоднородном поле. Далее, направление вектора \(\mathbf{F}\) в общем случае не совпадает ни с вектором \(\mathbf{E}\), ни с вектором \(\mathbf{p}\). Вектор \(\mathbf{F}\) совпадает по направлению лишь с элементарным приращением вектора \(\mathbf{E}\), взятым в направлении вектора \(\mathbf{l}\) или \(\mathbf{p}\) (рис. 1.15, б).

На рис. 1.16 показаны направления силы \(\mathbf{F}\), действующей

\begin{figure}[h]
\begin{center}
  \includegraphics[alt={},max width=\textwidth]{3e870b25-6320-483d-9133-cdd5212c2707-037_286_265_826_68}
\captionsetup{labelformat=empty}
\caption{Рис. 1.16}
\end{center}
\end{figure}

на диполь в поле положительного точечного заряда \(q\), при трех разных расположениях диполя. Убедиться самостоятельно, что это действительно так.

Если нас интересует проекция силы \(\mathbf{F}\) на некоторое направление \(X\), то достаточно записать равенство (1.39) в проекциях на это направление, и мы получим


\begin{equation*}
F_{x}=p \frac{\partial E_{x}}{\partial l}, \tag{1.40}
\end{equation*}


где \(\partial E_{x} / \partial l\) - производная соответствующей проекции вектора \(\mathbf{E}\) опять же по направлению вектора \(\mathbf{l}\) или \(\mathbf{p}\).

\begin{figure}[h]
\begin{center}
  \includegraphics[alt={},max width=\textwidth]{3e870b25-6320-483d-9133-cdd5212c2707-037_354_342_1395_72}
\captionsetup{labelformat=empty}
\caption{Рис. 1.17}
\end{center}
\end{figure}

Пусть диполь с моментом p расположен вдоль оси симметрии некоторого неоднородного поля Е. Возьмем положительное направление оси \(X\), например, как показано на рис. 1.17. Так как в направлении вектора \(\mathbf{p}\) приращение проекции \(E_{x}\) будет отрицательным, то \(F_{x}<0\), а значит, вектор \(\mathbf{F}\) направлен влево - в сторону, где напряженность поля больше. Если же вектор p на этом рисунке повер-

нуть на \(90^{\circ}\) так, чтобы центр диполя совпадал с осью симметрии поля, то нетрудно сообразить, что в таком положении проекция \(F_{x}=0\).

Момент сил, действующих на диполь. Рассмотрим, как ведет себя электрический диполь p во внешнем электрическом поле Е. Как видно из рис. 1.18, силы, действующие на положительный и отрицательный заряды диполя, образуют пару

\[
\mathbf{F}_{+}=q \mathbf{E} \quad \text { и } \quad \mathbf{F}_{-}=-q \mathbf{E},
\]

\begin{figure}[h]
\begin{center}
  \includegraphics[alt={},max width=\textwidth]{3e870b25-6320-483d-9133-cdd5212c2707-038_278_439_326_713}
\captionsetup{labelformat=empty}
\caption{Рис. 1.18}
\end{center}
\end{figure}

плечо которой равно \(l \sin \alpha\), т. е. зависит от ориентации диполя относительно поля Е. Модуль каждой из этих сил равен \(q E\), и на диполь будет действовать механический момент \(N\), определяемый, как мы знаем, произведением \(q E\) на плечо пары, т. е.

\[
N=q E \cdot l \sin \alpha=p E \sin \alpha,
\]

где \(p=q l\) - электрический момент диполя.\\
Полученную формулу можно представить в векторном виде как


\begin{equation*}
\mathbf{N}=[\mathbf{p E}] . \tag{1.41}
\end{equation*}


Этот момент сил стремится повернуть диполь так, чтобы его электрический момент \(\mathbf{p}\) установился по направлению внешнего поля Е. Такое положение диполя является устойчивым.

В неоднородном электрическом поле диполь будет вести себя следующим образом: под действием момента сил (1.41) диполь будет стремиться установиться по полю ( \(\mathbf{p} \uparrow \uparrow \mathbf{E}\) ), а под действием результирующей силы (1.39) - переместиться в направлении, где \(\mathbf{E}\) по модулю больше. Оба движения будут совершаться одновременно.

Энергия диполя в поле. Мы знаем, что энергия точечного заряда \(q\) во внешнем поле равна \(W=q \varphi\), где \(\varphi\) - потенциал поля в точке нахождения заряда \(q\). Диполь - это система из двух зарядов, поэтому его энергия во внешнем поле

\[
W=q_{+} \varphi_{+}+q_{-} \varphi_{-}=q\left(\varphi_{+}-\varphi_{-}\right),
\]

где \(\varphi_{+}\)и \(\varphi_{-}\)- потенциал внешнего поля в точках расположения зарядов \(+q\) и \(-q\). С точностью до величины второго порядка малости

\[
\varphi_{+}-\varphi_{-}=\frac{\partial \varphi}{\partial l} l,
\]

где \(\partial \varphi / \partial l\) - производная потенциала по направлению вектора 1. Согласно (1.32) \(\partial \varphi / \partial l=-E_{l, \text { поэтому }} \varphi_{+}-\varphi_{-}=-E_{l} l=-\mathbf{E l}\) и


\begin{equation*}
W=-\mathbf{p E} . \tag{1.42}
\end{equation*}


Из этой формулы следует, что минимальную энергию \(\left(W_{\text {мин }}=-p E\right)\) диполь имеет в положении \(\mathbf{p} \uparrow \uparrow \mathbf{E}\) (положение устойчивого равновесия). При отклонении из этого положения возникает момент внешних сил, возвращающий диполь к положению равновесия.

\section*{Задачи}
1.1. Очень тонкий диск равномерно заряжен с поверхностной плотностью \(\sigma>0\). Найти напряженность \(\mathbf{E}\) электрического поля на оси этого диска в точке, из которой диск виден под телесным углом \(\Omega\). Решение. Из соображений симметрии ясно, что вектор \(\mathbf{E}\) на оси диска должен совпадать с направлением этой оси (рис. 1.19). Поэ-

\begin{figure}[h]
\begin{center}
  \includegraphics[alt={},max width=\textwidth]{3e870b25-6320-483d-9133-cdd5212c2707-039_453_372_1129_67}
\captionsetup{labelformat=empty}
\caption{Рис. 1.19}
\end{center}
\end{figure}

тому достаточно найти составляющую \(\mathrm{d} E_{z}\) в точке \(A\) от элемента заряда на площади \(\mathrm{d} S\) и затем проинтегрировать полученное выражение по всей поверхности диска. Нетрудно сообразить (рис. 1.19), что


\begin{equation*}
\mathrm{d} E_{z}=\frac{1}{4 \pi \varepsilon_{0}} \frac{\sigma \mathrm{~d} S}{r^{2}} \cos \vartheta . \tag{1}
\end{equation*}


В данном случае \(\mathrm{d} S \cos \vartheta / r^{2}=\mathrm{d} \Omega\) - телесный угол, под которым площадка \(d S\) видна из точки \(A\), и выражение (1) можно переписать так:

\[
\mathrm{d} E_{z}=\frac{1}{4 \pi \varepsilon_{0}} \sigma \mathrm{~d} \Omega
\]

Отсюда искомая величина

\[
E=\frac{1}{4 \pi \varepsilon_{0}} \sigma \Omega .
\]

Заметим, что на больших расстояниях от диска \(\Omega=S / r^{2}\), где \(S\) площадь диска, и \(E=q / 4 \pi \varepsilon_{0} r^{2}\) - как поле точечного заряда \(q=\sigma S\). В непосредственной же близости от точки \(O\) телесный угол \(\Omega=2 \pi\) и \(E=\sigma / 2 \varepsilon_{0}\).\\
1.2. Тонкое непроводящее кольцо радиусом \(R\) заряжено с линейной плотностью \(\lambda=\lambda_{0} \cos \varphi\), где \(\lambda_{0}\) — положительная постоянная, \(\varphi\) — азимутальный угол. Найти напряженность \(E\) электрического поля в центре кольца.\\
Решение. Заданное распределение заряда показано на рис. 1.20. Из симметрии этого распределения ясно, что вектор E в точке \(O\) направлен вправо и модуль этого вектора равен сумме проекций на направление \(\mathbf{E}\) векторов dE - от элементарных зарядов \(\mathrm{d} q\). Проекция вектоpa dE на вектор E есть

\begin{figure}[h]
\begin{center}
  \includegraphics[alt={},max width=\textwidth]{3e870b25-6320-483d-9133-cdd5212c2707-040_260_384_493_776}
\captionsetup{labelformat=empty}
\caption{Рис. 1.20}
\end{center}
\end{figure}


\begin{equation*}
\mathrm{d} E \cos \varphi=\frac{1}{4 \pi \varepsilon_{0}} \frac{\mathrm{~d} q}{R^{2}} \cos \varphi, \tag{1}
\end{equation*}


где \(\mathrm{d} q=\lambda R \mathrm{~d} \varphi=\lambda_{0} R \cos \varphi \mathrm{~d} \varphi\). Проинтегрировав (1) по φ от 0 до \(2 \pi\), найдем модуль вектора E:

\[
E=\frac{\lambda_{0}}{4 \pi \varepsilon_{0} R} \int_{0}^{2 \pi} \cos ^{2} \varphi \mathrm{~d} \varphi=\frac{\lambda_{0}}{4 \varepsilon_{0} R} .
\]

Заметим, что этот интеграл проще всего вычислить, зная, что \(\left\langle\cos ^{2} \varphi\right\rangle=\frac{1}{2}\). Тогда

\[
\int_{0}^{2 \pi} \cos ^{2} \varphi \mathrm{~d} \varphi=\left\langle\cos ^{2} \varphi\right\rangle 2 \mathrm{p}=\mathrm{p}
\]

1.3. Полубесконечная прямая равномерно заряженная нить имеет заряд \(\lambda\) на единицу длины. Найти модуль и направление напряженности поля в точке, которая отстоит от нити на расстоянии \(r\) и находится на перпендикуляре к нити, проходящем через ее конец.\\
Решение. Задача сводится к нахождению \(E_{x}\) и \(E_{r}\) - проекций вектора \(\mathbf{E}\) (рис. 1.21, где предполагается \(\lambda>0\) ). Начнем с \(E_{x}\). Элемент заряда на участке \(\mathrm{d} x\) нити дает следующий

\begin{figure}[h]
\begin{center}
  \includegraphics[alt={},max width=\textwidth]{3e870b25-6320-483d-9133-cdd5212c2707-040_350_287_1418_875}
\captionsetup{labelformat=empty}
\caption{Рис. 1.21}
\end{center}
\end{figure}

вклад в \(E_{x}\) :


\begin{equation*}
\mathrm{d} E_{x}=\frac{1}{4 \pi \varepsilon_{0}} \frac{\lambda \mathrm{~d} x}{r^{2}} \sin \alpha . \tag{1}
\end{equation*}


Приведем это выражение к виду, удобному для интегрирования. В нашем случае \(\mathrm{d} x=r \mathrm{~d} \alpha / \cos _{\alpha}, r=y / \cos \alpha\). Тогда

\[
\mathrm{d} E_{x}=\frac{\lambda}{4 \pi \varepsilon_{0} y} \sin \alpha \mathrm{~d} \alpha .
\]

Проинтегрировав это выражение по \(\alpha\) от 0 до \(\pi / 2\), найдем

\[
E_{x}=\lambda / 4 \pi \varepsilon_{0} r .
\]

Для нахождения проекции \(E_{r}\) достаточно обратить внимание на то, что \(\mathrm{d} E_{r}\) отличается от \(\mathrm{d} E_{x}\) просто заменой \(\sin _{\alpha}\) в (1) на \(\cos _{\alpha}\). Тогда

\[
d E_{r}=\lambda \cos _{\alpha} \mathrm{d} \alpha / 4 \pi \varepsilon_{0} r \quad \text { и } \quad E_{x}=\lambda / 4 \pi \varepsilon_{0} r .
\]

Мы получили интересный результат: \(E_{x}=E_{r}\) независимо от \(r\), т. е. вектор \(\mathbf{E}\) ориентирован под углом \(45^{\circ}\) к нити. Модуль вектора \(\mathbf{E}\)

\[
E=\sqrt{E_{x}^{2}+E_{r}^{2}}=\lambda \sqrt{2} / 4 \pi \varepsilon_{0} r .
\]

1.4. Теорема Гаусса. Имеется аксиально-симметричное поле, напряженность которого зависит только от расстояния \(r\) до его оси как \(\mathbf{E}=a \mathbf{r} / r^{2}\), где \(a\) - постоянная. Найти заряд в объеме, ограниченном сферой радиусом \(R\) с центром на оси этого поля.

Решение. Искомый заряд равен согласно теореме Гаусса потоку вектора \(\mathbf{E}\) через указанную сферу, умноженному на \(\varepsilon_{0}\). В данном

\begin{figure}[h]
\begin{center}
  \includegraphics[alt={},max width=\textwidth]{3e870b25-6320-483d-9133-cdd5212c2707-041_310_253_1273_138}
\captionsetup{labelformat=empty}
\caption{Рис. 1.22}
\end{center}
\end{figure}

случае для определения потока можно поступить так. Заметив, что поле \(\mathbf{E}\) является осесимметричным (полем заряженной равномерно нити), приходим к выводу, что поток через сферу радиусом \(R\) равен потоку через боковую поверхность цилиндра того же радиуса и высотой \(2 R\), расположенного, как показано на рис. 1.22. Тогда

\[
q=\varepsilon_{0} \oint \mathbf{E d} \mathbf{S}=\varepsilon_{0} E_{r} S,
\]

где \(E_{r}=a / R\) и \(S=2 \pi R \cdot 2 R=4 \pi R^{2}\). И окончательно,

\[
q=\mathbf{4} \pi \varepsilon_{0} a R .
\]

1.5. Система состоит из равномерно заряженной сферы радиусом \(R\) и окружающей среды, заполненной зарядом с объемной плотностью \(\rho=\alpha / r\), где \(\alpha\) - положительная постоянная, \(r\) - расстояние от центра сферы. Найти заряд сферы, при котором напряженность \(E\) электрического поля вне сферы не будет зависеть от \(r\). Чему равно \(E\) ?

Решение. Пусть искомый заряд сферы равен \(q\), тогда, воспользовавшись теоремой Гаусса, запишем для сферической поверхности радиусом \(r\) (снаружи сферы с зарядом \(q\) ):

\[
E \cdot 4 \pi r^{2}=\frac{q}{\varepsilon_{0}}+\frac{1}{\varepsilon_{0}} \int_{R}^{r} \frac{\alpha}{r} 4 \pi r^{2} \mathrm{~d} r .
\]

Проинтегрировав, преобразуем это уравнение к виду

\[
E \cdot 4 \pi r^{2}=\left(q-2 \pi \alpha R^{2}\right) / \varepsilon_{0}+4 \pi \alpha r^{2} / 2 \varepsilon_{0} .
\]

Напряженность \(E\) не зависит от \(r\) при условии, когда выражение в скобках равно нулю. Отсюда

\[
q=2 \pi \alpha R^{2} \text { и } E=\alpha / 2 \varepsilon_{0} .
\]

1.6. Найти напряженность \(\mathbf{E}\) электрического поля в области пересечения двух шаров, равномерно заряженных разноименными по знаку зарядами с объемной плотностью \(\rho\) и - \(\rho\), если расстояние между центрами шаров определяется вектором 1 (рис. 1.23).\\
Решение. С помощью теоремы Гаусса нетрудно показать, что напряженность электрического поля внутри равномерно заряженного шара

\[
\mathbf{E}=\left(\rho / 3 \varepsilon_{0}\right) \mathbf{r},
\]

где \(\mathbf{r}\) - радиус-вектор относительно центра шара. Поле в области пересечения шаров можно рассматривать как суперпозицию по-

\begin{figure}[h]
\begin{center}
  \includegraphics[alt={},max width=\textwidth]{3e870b25-6320-483d-9133-cdd5212c2707-042_332_491_1419_163}
\captionsetup{labelformat=empty}
\caption{Рис. 1.23}
\end{center}
\end{figure}

\begin{figure}[h]
\begin{center}
  \includegraphics[alt={},max width=\textwidth]{3e870b25-6320-483d-9133-cdd5212c2707-042_329_487_1424_666}
\captionsetup{labelformat=empty}
\caption{Рис. 1.24}
\end{center}
\end{figure}

лей двух равномерно заряженных шаров. Тогда в произвольной точке \(A\) (рис. 1.24) этой области

\[
\mathbf{E}=\mathbf{E}_{+}+\mathbf{E}_{-}=\rho\left(\mathbf{r}_{+}-\mathbf{r}_{-}\right) / 3 \varepsilon_{0}=\rho \mathbf{l} / 3 \varepsilon_{0} .
\]

\begin{figure}[h]
\begin{center}
  \includegraphics[alt={},max width=\textwidth]{3e870b25-6320-483d-9133-cdd5212c2707-043_273_269_298_147}
\captionsetup{labelformat=empty}
\caption{Рис. 1.25}
\end{center}
\end{figure}

Таким образом, поле в области пересечения таких шаров является однородным. Этот вывод справедлив независимо от соотношения радиусов шаров и расстояния между их центрами. Он справедлив, в частности, и тогда, когда один шар находится целиком внутри другого, или, другими словами, когда в шаре имеется сферическая полость (рис. 1.25).\\
1.7. Воспользовавшись решением предыдущей задачи, найти напряженность \(\mathbf{E}\) поля внутри сферы, по которой распределен заряд с поверхностной плотностью \(\sigma=\sigma_{0} \cos _{\vartheta}\), где \(\sigma_{0}\) - постоянная, \(\vartheta\) полярный угол.

Решение. Рассмотрим два шара одинакового радиуса, имеющих равномерно распределенные по объему заряды с плотностями \(\rho\) и -ρ. Пусть центры шаров смещены относительно друг друга на расстояние \(l\) (рис. 1.26). Тогда согласно решению предыдущей задачи поле в области пересечения этих шаров будет однородным:


\begin{equation*}
\mathbf{E}=\left(\rho / 3 \varepsilon_{0}\right) \mathbf{l} . \tag{1}
\end{equation*}


\begin{figure}[h]
\begin{center}
  \includegraphics[alt={},max width=\textwidth]{3e870b25-6320-483d-9133-cdd5212c2707-043_334_263_1109_167}
\captionsetup{labelformat=empty}
\caption{Рис. 1.26}
\end{center}
\end{figure}

В нашем случае объемный заряд отличается от нуля только в поверхностном слое. При очень малом \(l\) мы придем к представлению о поверхностной плотности заряда на сфере. Толщина заряженного слоя в точках, определяемых углом \(\vartheta\) (рис. 1.26), равна \(l \cos \vartheta\). Значит, на единицу площади в этом месте приходится заряд \(\sigma=\rho l \cos \vartheta= =\sigma_{0} \cos \vartheta\), где \(\sigma_{0}=\rho l\), и выражение (1) можно представить как

\[
\mathbf{E}=-\left(\sigma_{0} / 3 \varepsilon_{0}\right) \mathbf{k},
\]

где \(\mathbf{k}\) - орт оси \(Z\), от которой отсчитывается угол Э.\\
1.8. Потенциал. Потенциал некоторого электрического поля имеет вид \(\varphi=\alpha\left(x y-z^{2}\right)\). Найти проекцию вектора \(\mathbf{E}\) на направление вектора \(\mathbf{a}=\mathbf{i}+3 \mathbf{k}\) в точке \(M(2,1,-3)\).

Решение. Сначала найдем вектор Е:

\[
\mathbf{E}=-\nabla \varphi=-\alpha(y \mathbf{i}+x \mathbf{j}-2 z \mathbf{k}) .
\]

Искомая проекция

\[
E_{a}=\mathbf{E} \frac{\mathbf{a}}{a}=\frac{-\alpha(y \mathbf{i}+x \mathbf{j}-2 z \mathbf{k})(\mathbf{i}+3 \mathbf{k})}{\sqrt{1+3^{2}}}=\frac{-\alpha(y-6 z)}{\sqrt{10}} .
\]

В точке \(M\)

\[
E_{a}=\frac{-\alpha(1+18)}{\sqrt{10}}=-\frac{19}{\sqrt{10}} \alpha .
\]

1.9. Найти потенциал φ на краю тонкого диска, по одной стороне которого равномерно распределен заряд с поверхностной плотностью \(\sigma\). Радиус диска равен \(R\).

Решение. По определению потенциал в случае поверхностного распределения заряда дается интегралом (1.28). Для упрощения интегрирования выберем в качестве площадки \(\mathrm{d} S\) часть кольца радиусом \(r\) и шириной \(\mathrm{d} r\) (рис. 1.27). Тогда \(\mathrm{d} S=2 \vartheta r \mathrm{~d} r, r=2 R \cos \vartheta\), \(\mathrm{d} r=-2 R \sin \vartheta \mathrm{~d} \vartheta\). После подстановки этих выражений в интеграл (1.28) получим для φ в точке \(O\) :

\begin{figure}[h]
\begin{center}
  \includegraphics[alt={},max width=\textwidth]{3e870b25-6320-483d-9133-cdd5212c2707-044_268_291_746_862}
\captionsetup{labelformat=empty}
\caption{Рис. 1.27}
\end{center}
\end{figure}

\[
\varphi=-\frac{\sigma R}{\pi \varepsilon_{0}} \int_{\pi / 2}^{0} \vartheta \sin \vartheta \mathrm{~d} \vartheta .
\]

Интегрирование проводим по частям, обозначив \(\vartheta=u, \sin \vartheta \mathrm{~d} \vartheta=\mathrm{d} v:\)

\[
\int \vartheta \sin \vartheta \mathrm{d} \vartheta=-\vartheta \cos \vartheta+\int \cos \vartheta \mathrm{d} \vartheta=-\vartheta \cos \vartheta+\sin \vartheta,
\]

что дает после подстановки пределов интегрирования -1 . В результате

\[
\varphi=\sigma R / \pi \varepsilon_{0} .
\]

1.10. Потенциал поля внутри заряженного шара зависит только от расстояния \(r\) до его центра по закону \(\varphi=a r^{2}+b\), где \(a\) и \(b\) - постоянные. Найти распределение объемного заряда \(\rho(r)\) внутри шара.

Решение. Сначала найдем напряженность поля. Согласно (1.32)


\begin{equation*}
E_{r}=-\partial \varphi / \partial r=-2 a r . \tag{1}
\end{equation*}


Затем воспользуемся теоремой Гаусса: \(4 \pi r^{2} E_{r}=q / \varepsilon_{0}\). Дифференциал этого выражения

\[
4 \pi \mathrm{~d}\left(r^{2} E_{r}\right)=\frac{1}{\varepsilon_{0}} \mathrm{~d} q=\frac{1}{\varepsilon_{0}} \rho \cdot 4 \pi r^{2} \mathrm{~d} r
\]

где \(\mathrm{d} q\) - заряд между сферами, радиусы которых \(r\) и \(r+\mathrm{d} r\). Отсюда

\[
r^{2} \mathrm{~d} E_{r}+2 r E_{r} \mathrm{~d} r=\frac{1}{\varepsilon_{0}} \rho r^{2} \mathrm{~d} r, \quad \frac{\partial E_{r}}{\partial r}+\frac{2}{r} E_{r}=\frac{\rho}{\varepsilon_{0}} .
\]

Подставив (1) в последнее уравнение, получим

\[
\rho=-6 \varepsilon_{0} a,
\]

т. е. заряд внутри шара распределен равномерно.\\
1.11. Диполь. Найти силу взаимодействия двух точечных диполей с моментами \(\mathbf{p}_{1}\) и \(\mathbf{p}_{2}\), если векторы \(\mathbf{p}_{1}\) и \(\mathbf{p}_{2}\) направлены вдоль прямой, соединяющей диполи, и расстояние между последними равно \(l\).

Решение. Согласно (1.39)

\[
F=p_{1}|\partial E / \partial l|,
\]

где \(E\) - напряженность поля диполя \(\mathbf{p}_{2}\), определяемая первой из формул (1.38):

\[
E=\frac{1}{4 \pi \varepsilon_{0}} \frac{2 p_{2}}{l^{3}}
\]

Взяв производную последнего выражения по \(l\) и подставив ее в формулу для \(F\), получим

\[
F=\frac{1}{4 \pi \varepsilon_{0}} \frac{6 p_{1} p_{2}}{l^{4}}
\]

Заметим, что диполи будут притягиваться, если \(\mathbf{p}_{1} \uparrow \uparrow \mathbf{p}_{2}\), и отталкиваться, если \(\mathbf{p}_{1} \downarrow \uparrow \mathbf{p}_{2}\).

\section*{Глава 2}
\section*{Проводник в электростатическом поле}
\section*{§ 2.1. Поле в веществе}
Микро- и макрополе. Истинное электрическое поле в любом веществе - его называют микрополем - меняется весьма резко как в пространстве, так и во времени. Оно различно в разных точках атомов и промежутках между ними. Чтобы найти напряженность \(\mathbf{E}\) истинного поля в некоторой точке в данный момент, нужно было бы сложить напряженности полей всех отдельных заряженных частиц вещества - электронов и ядер. Решение этой задачи, очевидно, является совершенно нереальным. Да и сам результат оказался бы настолько сложным, что его просто нельзя было бы использовать. Более того, для решения макроскопических задач такое поле и вовсе не нужно. Для многих целей достаточно более простое и несравненно более грубое описание, которым мы и будем пользоваться в дальнейшем.

Под электрическим полем \(\mathbf{E}\) в веществе - его называют макрополем - мы будем понимать пространственно усредненное микрополе (после пространственного усреднения временное усреднение уже не требуется). Это усреднение проводится по так называемому физически бесконечно малому объему объему, содержащему большое число атомов, но имеющему размеры во много раз меньше, чем те расстояния, на которых макрополе меняется заметно. Усреднение по таким объемам сглаживает все нерегулярные и быстро меняющиеся вариации микрополя на расстояниях порядка атомных, но сохраняет плавные изменения макрополя на макроскопических расстояниях. Итак, поле в веществе


\begin{equation*}
\mathbf{E}=\mathbf{E}_{\text {макро }}=\left\langle\mathbf{E}_{\text {микро }}\right\rangle . \tag{2.1}
\end{equation*}


Влияние вещества на поле. При внесении любого вещества в электрическое поле в веществе происходит смещение положительных и отрицательных зарядов (ядер и электронов), что в свою очередь приводит к частичному разделению этих зарядов. В тех или иных местах вещества появляются нескомпенсированные заряды различного знака. Это явление называют электростатической индукцией, а появившиеся в результате разделения заряды - индуцированными зарядами.

Индуцированные заряды создают дополнительное электрическое поле, которое вместе с исходным (внешним) электрическим полем образует результирующее поле. Зная внешнее поле и распределение индуцированных зарядов, можно при нахождении результирующего поля уже не обращать внимание на наличие самого вещества - его роль уже учтена с помощью индуцированных зарядов.

Таким образом, результирующее поле при наличии вещества определяется просто как суперпозиция внешнего поля и поля индуцированных зарядов. Однако во многих случаях дело усложняется тем, что мы заранее не знаем, как распределяются в пространстве все эти заряды - задача оказывается далеко не такой простой, как могло бы показаться вначале. Как мы увидим далее, распределение индуцированных зарядов в решающей степени зависит от свойств самого вещества - от его физической природы и формы тел. С этими вопросами нам и предстоит ознакомиться более подробно.

\section*{§ 2.2. Поле внутри и снаружи проводника}
Внутри проводника \(\mathbf{E}=\mathbf{0}\). Поместим металлический проводник во внешнее электростатическое поле или сообщим ему ка-кой-нибудь заряд. В обоих случаях на все заряды проводника будет действовать электрическое поле, в результате чего все отрицательные заряды (электроны) сместятся против поля. Такое перемещение зарядов (ток) будет продолжаться до тех пор (практически это происходит в течение малой доли секунды), пока не установится определенное распределение зарядов, при котором электрическое поле во всех точках внутри проводника обратится в нуль. Таким образом, в статическом случае электрическое поле внутри проводника отсутствует ( \(\mathbf{E}=\mathbf{0}\) ).

Далее, поскольку в проводнике всюду \(\mathbf{E}=\mathbf{0}\), то плотность избыточных (нескомпенсированных) зарядов внутри проводника также всюду равна нулю ( \(\rho=0\) ). Это легко понять с помощью теоремы Гаусса. Действительно, так как внутри проводника \(\mathbf{E}=\mathbf{0}\), то и поток вектора \(\mathbf{E}\) сквозь любую замкнутую поверхность внутри проводника также равен нулю. А это и значит, что внутри проводника избыточных зарядов нет.

Избыточные заряды появляются лишь на поверхности проводника с некоторой плотностью \(\sigma\), вообще говоря, различной в разных точках его поверхности. Заметим, что избыточный поверхностный заряд находится в очень тонком поверхностном слое (его толщина около одного-двух межатомных расстояний).

Отсутствие поля внутри проводника означает согласно (1.31), что потенциал φ в проводнике одинаков во всех его точках, т. е. любой проводник в электростатическом поле представляет собой эквипотенциальную область и его поверхность является эквипотенциальной.

Из того факта, что поверхность проводника эквипотенциальна, следует, что непосредственно у этой поверхности поле \(\mathbf{E}\) направлено по нормали к ней в каждой точке. Если бы это было не так, то под действием касательной составляющей \(\mathbf{E}\) заряды пришли бы в движение по поверхности проводника, т. е. равновесие зарядов было бы невозможным.

Пример. Найдем потенциал незаряженного проводящего шара, на расстоянии \(r\) от центра которого расположен точечный заряд \(q\) (рис. 2.1).

Потенциал φ всех точек шара одинаков. Раз так, вычислим его в центре шара \(O\), ибо только для этой точки расчет оказывается наиболее про-

\begin{figure}[h]
\begin{center}
  \includegraphics[alt={},max width=\textwidth]{3e870b25-6320-483d-9133-cdd5212c2707-048_234_342_1169_808}
\captionsetup{labelformat=empty}
\caption{Рис. 2.1}
\end{center}
\end{figure}

стым:


\begin{equation*}
\varphi=\frac{1}{4 \pi \varepsilon_{0}} \frac{q}{r}+\varphi^{\prime}, \tag{1}
\end{equation*}


где первое слагаемое - это потенциал от заряда \(q\), а второе потенциал от зарядов, индуцированных на поверхности шара. Но так как все индуцированные заряды находятся на одном и том же расстоянии \(a\) от точки \(O\) и суммарный инду-

цированный заряд равен нулю, то \(\varphi^{\prime}=0\). Таким образом, в данном случае потенциал шара будет определяться только первым слагаемым в (1).

На рис. 2.2 изображено поле и распределение зарядов для системы из двух проводящих шаров, один из которых (левый) заряжен. Вследствие электрической

\begin{figure}[h]
\begin{center}
  \includegraphics[alt={},max width=\textwidth]{3e870b25-6320-483d-9133-cdd5212c2707-049_392_415_411_72}
\captionsetup{labelformat=empty}
\caption{Рис. 2.2}
\end{center}
\end{figure}

индукции на поверхности правого незаряженного шара появились заряды противоположного знака. Поле этих зарядов в свою очередь вызовет некоторое перераспределение зарядов на поверхности левого шара их распределение по поверхности станет неравномерным. Сплошными линиями на рисунке показаны линии вектора E, пунктирными - пересечения эквипотенциальных поверхностей с плоскостью рисунка. По мере удаления от этой системы эквипотенциальные поверхности становятся все более близкими к сферическим, а линии вектора \(\mathbf{E}\) приближаются к радиальным, и само поле становится все более близким к полю точечного заряда \(q\) - полному заряду данной системы.

Поле у поверхности проводника. Напряженность электрического поля непосредственно у поверхности проводника связана, как мы сейчас увидим, простым соотношением с локальной плотностью заряда на поверхности проводника. Эту связь можно легко установить с помощью теоремы Гаусса.

Пусть интересующий нас участок поверхности проводника граничит с вакуумом. Линии вектора \(\mathbf{E}\) перпендикулярны поверхности проводника, поэтому в качестве замкнутой поверхности возьмем небольшой цилиндр, распо-

\begin{figure}[h]
\begin{center}
  \includegraphics[alt={},max width=\textwidth]{3e870b25-6320-483d-9133-cdd5212c2707-049_225_318_1496_72}
\captionsetup{labelformat=empty}
\caption{Рис. 2.3}
\end{center}
\end{figure}

ложив его так, как показано на рис. 2.3. Тогда поток вектора \(\mathbf{E}\) через эту поверхность будет равен только потоку через «наружный» торец цилиндра (потоки через боковую поверхность и внутренний торец равны нулю), и мы имеем \(E_{n} \Delta S=\sigma \Delta S / \varepsilon_{0}\), где \(E_{n}\) - проекция вектора \(\mathbf{E}\) на внешнюю\\
(по отношению к проводнику) нормаль \(\mathbf{n}, \Delta S\) - площадь сечения цилиндра, \(\sigma\) - локальная поверхностная плотность заряда на проводнике. Сократив обе части этого равенства на \(\Delta S\), получим


\begin{equation*}
E_{n}=\sigma / \varepsilon_{0} . \tag{2.2}
\end{equation*}


Если \(\sigma>0\), то и \(E_{n}>0\), т. е. вектор \(\mathbf{E}\) направлен от поверхности проводника - совпадает по направлению с нормалью \(\mathbf{n}\); если же \(\sigma<0\), то \(E_{n}<0\) - вектор \(\mathbf{E}\) направлен к поверхности проводника.

В связи с соотношением (2.2) может возникнуть ошибочное заключение, что \(\mathbf{E}\) вблизи проводника зависит только от локальной плотности σ заряда. Это не так. Напряженность \(\mathbf{E}\) определяется всеми зарядами рассматриваемой системы, как и само значение \(\sigma\).

\section*{§ 2.3. Силы, действующие на поверхность проводника}
Рассмотрим случай, когда заряженный участок поверхности проводника граничит с вакуумом. На малый элемент \(\Delta S\) поверхности проводника действует сила


\begin{equation*}
\Delta \mathbf{F}=\sigma \Delta S \cdot \mathbf{E}_{0}, \tag{2.3}
\end{equation*}


где \(\sigma \Delta S\) - заряд этого элемента, \(\mathrm{E}_{0}\) - напряженность поля, создаваемого всеми остальньлми зарядами системы в месте нахождения заряда \(\sigma \Delta S\). Сразу же заметим, что \(\mathbf{E}_{0}\) не равно напряженности \(\mathbf{E}\) поля вблизи данного элемента поверхности проводника, однако между ними имеется простая связь. Найдем ее, т.е. выразим \(\mathbf{E}_{0}\) через \(\mathbf{E}\).

Пусть \(\mathbf{E}_{\sigma}\) - напряженность поля, создаваемого зарядом на площадке \(\Delta S\) в точках, очень близких к этой площадке - здесь она ведет себя как бесконечная равномерно заряженная плоскость. Тогда согласно (1.10) \(E_{\sigma}=\sigma / 2 \varepsilon_{0}\).

Результирующее поле как внутри, так и вне проводника (вблизи площадки \(\Delta S\) ) является суперпозицией полей \(\mathbf{E}_{0}\) и \(\mathbf{E}_{\sigma}\). По разные стороны площадки \(\Delta S\) поле \(\mathbf{E}_{0}\) практически одинако-

\begin{figure}[h]
\begin{center}
  \includegraphics[alt={},max width=\textwidth]{3e870b25-6320-483d-9133-cdd5212c2707-051_360_370_151_88}
\captionsetup{labelformat=empty}
\caption{Рис. 2.4}
\end{center}
\end{figure}

во, поле же \(\mathbf{E}_{\sigma}\) имеет противоположные направления (рис. 2.4, где для определенности взято \(\sigma>0\) ). Из условия \(\mathbf{E}=\mathbf{0}\) в проводнике следует, что \(E_{\sigma}=E_{0}\), тогда снаружи проводника у его поверхности \(\quad E=E_{0}+E_{\sigma}=2 E_{0}\). Итак,


\begin{equation*}
\mathbf{E}_{0}=\mathbf{E} / 2, \tag{2.4}
\end{equation*}


и уравнение (2.3) примет вид


\begin{equation*}
\Delta \mathbf{F}=1 / 2 \sigma \Delta S \cdot \mathbf{E} . \tag{2.5}
\end{equation*}


Разделив обе части этого уравнения на \(\Delta S\), получим выражение для силы, действующей на единицу поверхности проводника:


\begin{equation*}
\mathbf{F}_{\mathrm{e} д}=1 / 2 \sigma \mathbf{E} . \tag{2.6}
\end{equation*}


Это выражение можно переписать и в другой форме, ибо входящие в него величины \(\sigma\) и \(\mathbf{E}\) являются взаимно связанными. Действительно, согласно (2.2) \(E_{n}=\sigma / \varepsilon_{0}\) или \(\mathbf{E}=\left(\sigma / \varepsilon_{0}\right) \mathbf{n}\), где \(\mathbf{n}\) - внешняя нормаль к элементу поверхности в данной точке проводника. Поэтому


\begin{equation*}
\mathbf{F}_{\text {ед }}=\frac{\sigma^{2}}{2 \varepsilon_{0}} \mathbf{n}=\frac{\varepsilon_{0} E^{2}}{2} \mathbf{n}, \tag{2.7}
\end{equation*}


где учтено, что \(\sigma=\varepsilon_{0} E_{n}\) и \(E_{n}^{2}=E^{2}\). Величину \(\mathbf{F}_{\text {ед }}\) называют поверхностной плотностью сил. Независимо от знака \(\sigma\), а значит, и направления \(\mathbf{E}\), сила \(\mathbf{F}_{\text {ед }}\) всегда направлена, как видно из (2.7), наружу проводника, стремясь его растянуть.

Пример. Найдем выражение для электрической силы, действующей в вакууме на проводник в целом, полагая, что известна напряженность \(\mathbf{E}\) поля во всех точках у поверхности проводника.\\
Умножив (2.7) на \(\mathrm{d} S\), получим выражение для силы dF , действующей на элемент поверхности \(\mathrm{d} S\) :

\[
\mathrm{d} \mathbf{F}=1 / 2 \varepsilon_{0} E^{2} \mathrm{~d} \mathbf{S},
\]

где \(\mathrm{dS}=\mathrm{nd} S\). Результирующая сила, действующая на весь проводник, определяется интегрированием этого уравнения по всей поверхности проводника:

\[
F=\frac{\varepsilon_{0}}{2} \oint E^{2} \mathrm{~d} S .
\]

\section*{§ 2.4. Свойства замкнутой проводящей оболочки}
Мы выяснили, что в состоянии равновесия избыточных зарядов внутри проводника нет - вещество внутри проводника электрически нейтрально. Поэтому удаление вещества из некоторого объема внутри проводника (создание замкнутой полости) поля нигде не изменит, т. е. никак не отразится на равновесном расположении зарядов. Это значит, что избыточный заряд распределяется на проводнике с полостью так же, как и на сплошном - по его наружной поверхности.

Таким образом, если в полости нет электрических зарядов, электрическое поле в ней равно нулю. Внешние заряды, в частности заряды на наружной поверхности проводника, не создают в полости внутри проводника никакого электрического поля. Именно на этом основана электростатическая защи\(\boldsymbol{m} \boldsymbol{a}\) - экранирование тел, например измерительных приборов, от влияния внешних электростатических полей. Практически сплошной проводник-оболочка может быть заменен достаточно густой металлической сеткой.

Доказать отсутствие электрического поля в пустой полости можно и иначе. Возьмем замкнутую поверхность \(S\), которая охватывает полость и целиком находится в веществе проводника. Так как поле \(\mathbf{E}\) всюду в проводнике равно нулю, то и поток вектора \(\mathbf{E}\) через \(S\) тоже равен нулю. Отсюда согласно теореме Гаусса равен нулю и суммарный заряд внутри \(S\). Это, правда, не исключает ситуации, показанной на рис. 2.5 , когда на поверхности самой полости имеются равные количества положительного и отрицательного зарядов. Такое предположение, однако, запрещает другая теорема - теорема о циркуляции вектора Е. В самом деле, пусть кон-

\begin{figure}[h]
\begin{center}
  \includegraphics[alt={},max width=\textwidth]{3e870b25-6320-483d-9133-cdd5212c2707-052_334_342_1403_806}
\captionsetup{labelformat=empty}
\caption{Рис. 2.5}
\end{center}
\end{figure}

тур Г пересекает полость по одной из линий вектора \(\mathbf{E}\) и замыкается в веществе проводника. Ясно, что линейный интеграл вектора \(\mathbf{E}\) вдоль этого контура не равен нулю, чего согласно теореме о циркуляции быть не может.

Теперь обратимся к случаю, когда полость не пустая, а в ней есть какой-то электрический заряд \(q\) (может быть и не один). Представим себе также, что все внешнее пространство заполнено проводящей средой. Поле в ней при равновесии равно нулю, значит, среда электрически нейтральна и не содержит нигде избыточных зарядов.

Так как всюду в проводнике \(\mathbf{E}=\mathbf{0}\), то равным нулю будет и поток вектора \(\mathbf{E}\) сквозь замкнутую поверхность, окружающую полость. По теореме Гаусса это означает, что алгебраическая сумма зарядов внутри этой замкнутой поверхности также будет равна нулю. Таким образом, алгебраическая сумма индуцированных зарядов на поверхности полости равна по модулю и противоположна по знаку алгебраической сумме зарядов внутри этой полости.

При равновесии заряды, индуцированные на поверхности полости, располагаются так, чтобы полностью скомпенсировать снаружи полости поле зарядов, находящихся внутри полости.

Поскольку проводящая среда внутри всюду электрически нейтральна, то она не оказывает никакого влияния на электрическое поле. Поэтому, если ее удалить, оставив только проводящую оболочку вокруг полости, от этого поле нигде не изменится и вне оболочки оно останется равным нулю.

Таким образом, поле зарядов, окруженных проводящей оболочкой, и зарядов, индуцированных на поверхности полости (на внутренней поверхности оболочки), равно нулю во всем внешнем пространстве.

Мы приходим к следующему важному выводу: замкнутая проводящая оболочка разделяет все пространство на внутреннюю и внешнюю части, в электрическом отношении совершенно не зависящие друг от друга. Это надо понимать так: после любого перемещения зарядов внутри оболочки никаких изменений поля во внешнем пространстве не произойдет, а значит, распределение зарядов на внешней поверхности оболочки останется прежним. То же относится и к полю внутри полости (если там есть заряды) и к распределению индуцированных на

стенках полости зарядов - они также останутся неизменными в результате перемещения зарядов вне оболочки. Все сказанное справедливо, разумеется, только в рамках электростатики.\\
Пример. Точечный заряд \(q\) находится внутри электрически нейтральной оболочки, наружной поверхностью которой является сфера (рис. 2.6). Найдем потенциал φ в точке \(P\) вне оболочки на расстоянии \(r\) от центра \(O\) наружной поверхности.\\
Поле в точке \(P\) определяется только зарядами, индуцированными на наружной поверхности

\begin{figure}[h]
\begin{center}
  \includegraphics[alt={},max width=\textwidth]{3e870b25-6320-483d-9133-cdd5212c2707-054_278_406_350_746}
\captionsetup{labelformat=empty}
\caption{Рис. 2.6}
\end{center}
\end{figure}

оболочки - сфере, ибо, как было показано, поле точечного заряда \(q\) и зарядов, индуцированных на внутренней поверхности оболочки, равно всюду нулю вне полости. Далее, заряд на наружной оболочке вследствие ее симметрии распределяется равномерно, поэтому

\[
\varphi=\frac{1}{4 \pi \varepsilon_{0}} \frac{q}{r} .
\]

Частным случаем замкнутой проводящей оболочки является безграничная проводящая плоскость. Все пространство с одной стороны такой плоскости в электрическом отношении независимо от пространства с другой стороны ее.

Указанным свойством замкнутой проводящей оболочки мы будем пользоваться в дальнейшем неоднократно.

\section*{§ 2.5. Общая задача электростатики. Метод изображений}
Очень часто приходится встречаться с задачами, в которых распределение зарядов неизвестно, но заданы потенциалы проводников, их форма и относительное расположение. И требуется определить потенциал \(\varphi(\mathbf{r})\) в любой точке поля между проводниками. Напомним, что, зная \(\varphi(\mathbf{r})\), можно легко восстановить само поле \(\mathbf{E}(\mathbf{r})\) и по значению \(\mathbf{E}\) непосредственно у поверхности проводников найти распределение поверхностных зарядов на них.

Уравнения Пуассона и Лапласа. Найдем дифференциальное уравнение, которому должна удовлетворять функция \(\varphi\) - потенциал. Для этого подставим в левую часть (1.20) вместо \(\mathbf{E}\) его выражение через \(\varphi\), т. е. \(\mathbf{E}=-\boldsymbol{\nabla} \varphi\). В результате получим общее дифференциальное уравнение для потенциала - уравнение Пуассона:


\begin{equation*}
\nabla^{2} \varphi=-\rho / \varepsilon_{0}, \tag{2.8}
\end{equation*}


где \(\nabla^{2}\) - оператор Лапласа (лапласиан). В декартовых координатах он имеет вид

\[
\nabla^{2}=\frac{\partial^{2}}{\partial x^{2}}+\frac{\partial^{2}}{\partial y^{2}}+\frac{\partial^{2}}{\partial z^{2}}
\]

т. е. представляет собой скалярное произведение \(\nabla \cdot \nabla\) [см. (1.19)]. Если между проводниками нет зарядов ( \(\rho=0\) ), то уравнение (2.8) переходит в более простое - уравнение Лапласа:


\begin{equation*}
\nabla^{2} \varphi=0 . \tag{2.9}
\end{equation*}


Определение потенциала сводится к нахождению такой функции φ, которая во всем пространстве между проводниками удовлетворяет уравнениям (2.8) или (2.9), а на поверхностях самих проводников принимает заданные значения \(\varphi_{1}, \varphi_{2}\) и т. д.

В теории доказывается, что эта задача имеет единственное решение. Это утверждение называют теоремой единственноcmu. С физической точки зрения этот вывод довольно очевиден: если решение не одно, то будет не один потенциальный «рельеф», следовательно, в каждой точке поле \(\mathbf{E}\), вообще говоря, неоднозначно - мы пришли к физическому абсурду.

По теореме единственности можно также утверждать, что заряд на поверхности проводника в статическом случае распределяется тоже единственным образом. Действительно, между зарядами на проводнике и электрическим полем вблизи его поверхности имеется однозначная связь (2.2): \(\sigma=\varepsilon_{0} E_{n}\). Отсюда сразу и следует, что единственность поля \(\mathbf{E}\) определяет и единственность распределения заряда на поверхности проводника.

Решение уравнений (2.8) и (2.9) в общем случае - задача сложная и кропотливая. Аналитические решения этих уравнений получены лишь для немногих частных случаев. Использование же теоремы единственности весьма облегчает решение ряда электростатических задач. Если решение задачи удовлетворяет уравнению Лапласа (или Пуассона) и граничным условиям, то можно утверждать, что оно является правильным и единственным, каким бы способом (хотя бы путем догадки) мы ни нашли его.

Пример. Покажем, что поле в пустой полости проводника отсутствует.\\
Потенциал φ в полости должен удовлетворять уравнению Лапласа (2.9) и на стенках полости принимать какое-то значение \(\varphi_{0}\). Решение уравнения Лапласа, удовлетворяющее этому условию, можно угадать сразу: \(\varphi=\varphi_{0}\). Согласно теореме единственности других решений быть не может. Поэтому \(\mathbf{E}=-\boldsymbol{\nabla} \varphi=0\).\\
Метод изображений. Это искусственный метод, позволяющий в ряде случаев (к сожалению, немногих) рассчитать электрическое поле достаточно просто. Рассмотрим идею этого метода на самом простом примере, когда точечный заряд \(q\) находится около безграничной проводящей плоскости (рис. 2.7, \(a\) ).

\begin{figure}[h]
\begin{center}
  \includegraphics[alt={},max width=\textwidth]{3e870b25-6320-483d-9133-cdd5212c2707-056_318_842_1034_197}
\captionsetup{labelformat=empty}
\caption{Рис. 2.7}
\end{center}
\end{figure}

Идея метода заключается в том, что мы должны найти другую задачу, которая решается просто и решение которой или часть его может быть использовано. В нашем случае такой простой задачей является задача с двумя зарядами \(q\) и \(-q\). Поле этой системы известно (его эквипотенциали и линии вектора \(\mathbf{E}\) показаны на рис. 2.7, б).

Совместим со средней эквипотенциальной поверхностью (ее потенциал \(\varphi=0\) ) проводящую плоскость и уберем заряд \(-q\). Согласно теореме единственности поле в верхнем полупространст-

ве останется прежним. Действительно, на проводящей плоскости и всюду в бесконечности \(\varphi=0\), точечный же заряд \(q\) можно рассматривать как предельный случай малого сферического проводника, радиус которого стремится к нулю, а потенциал к бесконечности. Таким образом, в верхнем полупространстве граничные условия для потенциала остались теми же, стало быть, тем же осталось и поле в этой области (рис. 2.7, в).

Заметим, что к этому выводу можно прийти, исходя и из свойств замкнутой проводящей оболочки (см. § 2.4), поскольку оба полупространства, разделенные проводящей плоскостью, в электрическом отношении независимы друг от друга, удаление заряда \(-q\) никак не скажется на поле в верхнем полупространстве, оно останется прежним.

Итак, в рассматриваемом случае поле отлично от нуля только в верхнем полупространстве, и для вычисления этого поля достаточно ввести фиктивный заряд-изображение \(q^{\prime}=-q\), противоположный по знаку заряду \(q\), поместив его по другую сторону проводящей плоскости на таком же расстоянии от нее, что и заряд \(q\). Фиктивный заряд \(q^{\prime}\) создает в верхнем полупространстве точно такое же поле, как и индуцированные заряды на плоскости. Именно это подразумевают, когда говорят, что фиктивный заряд заменяет собой «действие» всех индуцированных зарядов. Надо только иметь в виду, что «действие» фиктивного заряда распространяется лишь на то полупространство, в котором находится действительный заряд \(q\). В другом полупространстве поле отсутствует.

Резюмируя, можно сказать, что метод изображений по существу основан на подгонке потенциала под граничные условия: мы стараемся найти другую задачу (конфигурацию зарядов), у которой конфигурация поля в интересующей нас части пространства была бы той же. Если это удается сделать с помощью достаточно простых конфигураций, то метод изображений оказывается весьма эффективным. Рассмотрим еще один пример.

Пример. Точечный заряд \(q\) находится между двумя проводящими взаимно перпендикулярными полуплоскостями (рис. 2.8, \(a\) ). Найдем расположение точечных фиктивных зарядов, действие которых на заряд \(q\) будет эквивалентно действию всех индуцированных зарядов на данных полуплоскостях.

Нужно найти систему из точечных зарядов, у которой эквипотенциальные поверхности с \(\varphi=0\) совпадали бы с проводящими полуплоскостями. Одним и двумя фиктивными зарядами здесь не обойтись, таких зарядов должно быть три

\begin{figure}[h]
\begin{center}
  \includegraphics[alt={},max width=\textwidth]{3e870b25-6320-483d-9133-cdd5212c2707-058_297_417_149_695}
\captionsetup{labelformat=empty}
\caption{Рис. 2.8}
\end{center}
\end{figure}

(рис. 2.8, б). Только при такой конфигурации системы из четырех зарядов можно осуществить необходимую «подгонку» - обеспечить, чтобы на проводящих полуплоскостях потенциал был равен нулю. Именно эти три фиктивных заряда и создают то же поле внутри «прямого угла», что и заряды, индуцированные на проводящих полуплоскостях.

Найдя эту конфигурацию точечных зарядов (другую задачу), можно затем просто решить ряд других вопросов, например найти потенциал и напряженность поля в любой точке внутри «прямого угла», силу, действующую на заряд \(q\), и др.

\section*{§ 2.6. Электроемкость. Конденсаторы}
Электроемкость уединенного проводника. Рассмотрим ка-кой-либо уединенный проводник, т. е. проводник, удаленный от других проводников, тел и зарядов. Опыт показывает, что между зарядом \(q\) такого проводника и его потенциалом \(\varphi\) (потенциал на бесконечности мы условились считать равным нулю) существует прямая пропорциональность: \(\varphi \sim q\). Следовательно, \(q / \varphi\) не зависит от заряда \(q\), для каждого уединенного проводника это отношение имеет свое значение. Величину


\begin{equation*}
C=q / \varphi \tag{2.10}
\end{equation*}


называют электроемкостью уединенного проводника (сокращенно емкостью). Она численно равна заряду, сообщение которого проводнику повышает его потенциал на единицу. Емкость зависит от размеров и формы проводника.

Пример. Найдем емкость уединенного проводника, имеющего форму шара радиусом \(R\).

Для этого, как видно из формулы (2.10), надо мысленно зарядить данный проводник зарядом \(q\) и вычислить его потенциал φ. Согласно (1.23) потенциал шара

\[
\varphi=\int_{R}^{\infty} E_{r} \mathrm{~d} r=\frac{1}{4 \pi \varepsilon_{0}} \int_{R}^{\infty} \frac{q}{r^{2}} \mathrm{~d} r=\frac{1}{4 \pi \varepsilon_{0}} \frac{q}{R} .
\]

После подстановки полученного результата в (2.10) найдем


\begin{equation*}
C=4 \pi \varepsilon_{0} R . \tag{2.11}
\end{equation*}


За единицу емкости принимают емкость такого проводника, потенциал которого изменяется на 1 B при сообщении ему заряда 1 Кл. Эту единицу емкости называют фарадом (Ф).

Фарад - очень большая величина: емкостью 1 Ф обладал бы уединенный шар радиусом 9 млн. км, что в 1500 раз больше радиуса Земли (емкость Земли \(C=0,7\) мФ). На практике чаще всего приходится встречаться с емкостями в интервале от 1 мк \(\Phi\) до 1 п \(\Phi\).

Конденсаторы. Если проводник не уединен, то его емкость будет существенно увеличиваться при приближении к нему других тел. Это обусловлено тем, что поле данного проводника вызывает перераспределение зарядов на окружающих телах появление индуцированных зарядов. Пусть заряд проводника \(q>0\). Тогда отрицательные индуцированные заряды оказываются ближе к проводнику, нежели положительные. Поэтому потенциал проводника, являющийся алгебраической суммой потенциала собственных зарядов и зарядов, индуцированных на других телах, уменьшится при приближении к нему других незаряженных тел. А значит, его емкость увеличится.

Это позволило создать систему проводников, которая обладает емкостью, значительно большей, чем уединенный проводник, и притом не зависящей от окружающих тел. Такую систему называют конденсатором. Простейший конденсатор состоит из двух проводников (обкладок), расположенных на малом расстоянии друг от друга.

Чтобы внешние тела не оказывали влияния на емкость конденсатора, его обкладки располагают так относительно друг друга, чтобы поле, создаваемое накапливающимися на них зарядами, было сосредоточено практически полностью внутри

конденсатора. Это означает, что линии вектора \(\mathbf{E}\), начинающиеся на одной обкладке, должны заканчиваться на другой, т. е. заряды на обкладках должны быть одинаковы по модулю и противоположны по знаку ( \(q\) и \(-q\) ).

Основной характеристикой конденсатора является его емкость. В отличие от емкости уединенного проводника под емкостью конденсатора понимают отношение заряда конденсатора к разности потенциалов между обкладками (эту разность называют напряжением):


\begin{equation*}
C=q / U . \tag{2.12}
\end{equation*}


Под зарядом \(q\) конденсатора имеют в виду заряд, расположенный на положительно заряженной обкладке.

Естественно, емкость конденсатора измеряют также в фарадах.

Емкость конденсатора зависит от его геометрии (размеров и формы обкладок), от зазора между ними и от заполняющей конденсатор среды. Найдем выражения для емкости некоторых конденсаторов, считая пока, что между обкладками находится вакуум.

Емкость плоского конденсатора. Этот конденсатор состоит из двух параллельных пластин, разделенных зазором шириной \(h\). Если заряд конденсатора \(q\), то согласно (1.11) напряженность поля между его обкладками \(E=\sigma / \varepsilon_{0}\), где \(\sigma=q / S, S\) - площадь каждой пластины. Следовательно, напряжение между обкладками

\[
U=E h=q h / \varepsilon_{0} S .
\]

После подстановки этого выражения в (2.12) получим


\begin{equation*}
C=\varepsilon_{0} S / h . \tag{2.13}
\end{equation*}


Этот расчет был проведен без учета искажения поля у краев пластин (без учета краевых эффектов). Емкость реального плоского конденсатора определяется полученным выражением тем точнее, чем меньше зазор \(h\) по сравнению с линейными размерами пластин.

Емкость сферического конденсатора. Пусть радиусы внутренней и внешней обкладок конденсатора равны соответственно \(a\) и \(b\). Если заряд конденсатора \(q\), то напряженность поля между обкладками определяется по теореме Гаусса:

\[
E_{r}=\frac{1}{4 \pi \varepsilon_{0}} \frac{q}{r^{2}} .
\]

Напряжение на конденсаторе

\[
U=\int_{a}^{b} E_{r} \mathrm{~d} r=\frac{q}{4 \pi \varepsilon_{0}}\left(\frac{1}{a}-\frac{1}{b}\right) .
\]

Отсюда легко видеть, что емкость сферического конденсатора


\begin{equation*}
C=4 \pi \varepsilon_{0} \frac{a b}{a-b} . \tag{2.14}
\end{equation*}


Полезно убедиться, что в случае малого зазора между обкладками, т. е. при условии \((b-a) \ll a\) (или \(b\) ), полученное выражение переходит в (2.13) - в выражение для емкости плоского конденсатора.

Емкость цилиндрического конденсатора. Рассуждая так же, как и в случае со сферическим конденсатором, получим


\begin{equation*}
C=\frac{2 \pi \varepsilon_{0} l}{\ln (b / a)}, \tag{2.15}
\end{equation*}


где \(l\) - длина конденсатора; \(a\) и \(b\) - радиусы внутренней и наружной цилиндрических обкладок. Здесь так же, как и в предыдущем случае, при малом зазоре между обкладками полученное выражение переходит в (2.13).

О влиянии среды на емкость конденсатора см. § 3.6.

\section*{Задачи}
2.1. О нахождении потенциала. Точечный заряд \(q\) находится на расстоянии \(r\) от центра \(O\) незаряженного сферического проводящего слоя, внутренний и наружный радиусы которого равны соответственно \(a\) и \(b\). Найти потенциал в точке \(O\), если \(r<a\).

Решение. В результате электростатической индукции на внутренней поверхности слоя выступят, допустим, отрицательные заряды, а на наружной - положительные (рис. 2.9). Согласно принципу суперпозиции искомый потенциал в точке \(O\) можно представить как

\[
\varphi=\frac{1}{4 \pi \varepsilon_{0}}\left(\frac{q}{r}+\oint \frac{\sigma_{-} \mathrm{d} S}{a}+\oint \frac{\sigma_{+} \mathrm{d} S}{b}\right),
\]

где первый интеграл берется по всем индуцированным зарядам на внутренней поверхности слоя, а второй интеграл - по всем зарядам на внешней поверхности слоя. Из этого выражения следует:

\begin{figure}[h]
\begin{center}
  \includegraphics[alt={},max width=\textwidth]{3e870b25-6320-483d-9133-cdd5212c2707-062_314_315_279_836}
\captionsetup{labelformat=empty}
\caption{Рис. 2.9}
\end{center}
\end{figure}

\[
\varphi=\frac{q}{4 \pi \varepsilon_{0}}\left(\frac{1}{r}-\frac{1}{a}+\frac{1}{b}\right)
\]

Заметим, что так просто потенциал в полости можно найти только в точке \(O\), поскольку только от этой точки все индуцированные заряды одного знака находятся на одинаковом расстоянии и их распределение (нам не известное) не играет роли.\\
2.2. Система состоит из двух концентрических проводящих сфер, причем на внутренней сфере радиусом \(R_{1}\) находится заряд \(q_{1}\). Какой заряд \(q_{2}\) следует поместить на внешнюю сферу радиусом \(R_{2}\), чтобы потенциал внутренней сферы стал равным нулю? Как будет зависеть при этом потенциал φ от расстояния \(r\) до центра системы? Изобразить примерный график этой зависимости, если \(q_{1}<0\).

Решение. Запишем выражения для потенциала вне системы ( \(\varphi_{\text {II }}\) ) и в области между сферами ( \(\varphi_{\mathrm{I}}\) ):

\[
\varphi_{\mathrm{II}}=\frac{1}{4 \pi \varepsilon_{0}} \frac{q_{1}+q_{2}}{r}, \quad \varphi_{\mathrm{I}}=\frac{1}{4 \pi \varepsilon_{0}} \frac{q_{1}}{r}+\varphi_{0},
\]

где \(\varphi_{0}\) - некоторая постоянная. Ее значение легко найти из граничного условия: при \(r=R_{2}\) потенциал \(\varphi_{\text {II }}=\varphi_{\text {I }}\). Отсюда

\[
\varphi_{0}=q_{2} / 4 \pi \varepsilon_{0} R_{2} .
\]

Из условия \(\varphi_{1}\left(R_{1}\right)=0\) находим \(q_{2}= =-q_{1} R_{2} / R_{1}\). Зависимость \(\varphi(r)\) будет иметь вид (рис. 2.10):

\begin{figure}[h]
\begin{center}
  \includegraphics[alt={},max width=\textwidth]{3e870b25-6320-483d-9133-cdd5212c2707-062_248_406_1492_753}
\captionsetup{labelformat=empty}
\caption{Рис. 2.10}
\end{center}
\end{figure}

\[
\varphi_{\mathrm{II}}=\frac{q_{1}}{4 \pi \varepsilon_{0}} \frac{1-R_{2} / R_{1}}{r}, \quad \varphi_{\mathrm{I}}=\frac{q_{1}}{4 \pi \varepsilon_{0}}\left(\frac{1}{r}-\frac{1}{R_{1}}\right) .
\]

2.3. Сила, действующая на поверхностный заряд. Незаряженный металлический шар радиусом \(R\) поместили во внешнее однородное электрическое поле, в результате чего на поверхности шара появился индуцированный заряд с поверхностной плотностью \(\sigma=\sigma_{0} \cos \vartheta\), где \(\sigma_{0}\) - положительная постоянная, \(\vartheta\) - полярный угол. Найти модуль результирующей электрической силы, которая действует на заряд одного знака.

Решение. Согласно (2.5) на элементарную площадку \(d S\) действует электрическая сила


\begin{equation*}
\mathrm{d} \mathbf{F}=1 / 2 \sigma \mathbf{E} \mathrm{~d} S . \tag{1}
\end{equation*}


Из соображений симметрии ясно, что искомая результирующая сила \(\mathbf{F}\) направлена по оси \(Z\) (рис. 2.11) и поэтому ее можно представить как сумму (интеграл) проекций элементарных сил (1) на ось \(Z\) :

\begin{figure}[h]
\begin{center}
  \includegraphics[alt={},max width=\textwidth]{3e870b25-6320-483d-9133-cdd5212c2707-063_363_290_859_163}
\captionsetup{labelformat=empty}
\caption{Рис. 2.11}
\end{center}
\end{figure}


\begin{equation*}
\mathrm{d} F_{z}=\mathrm{d} F \cos \vartheta . \tag{2}
\end{equation*}


В качестве площадки \(d S\) целесообразно сразу же взять элементарный пояс, для которого \(\mathrm{d} S=2 \pi R \sin \vartheta \cdot R \mathrm{~d} \vartheta\). Учитывая, кроме того, что \(E=\sigma / \varepsilon_{0}\), преобразуем (2) к виду

\[
\begin{aligned}
\mathrm{d} F_{z} & =\frac{\pi \sigma^{2} R^{2}}{\varepsilon_{0}} \sin \vartheta \cos \vartheta \mathrm{~d} \vartheta= \\
& =-\frac{\pi \sigma_{0}^{2} R^{2}}{\varepsilon_{0}} \cos ^{3} \vartheta \mathrm{~d}(\cos \vartheta)
\end{aligned}
\]

Проинтегрировав это выражение по полусфере (т. е. по \(\cos \vartheta\) от 1 до 0 ), получим

\[
F=\pi \sigma_{0}^{2} R^{2} / 4 \varepsilon_{0} .
\]

2.4. Метод изображений. Точечный заряд \(q\) находится на расстоянии \(l\) от проводящей плоскости. Определить поверхностную плотность зарядов, индуцированных на плоскости, как функцию расстояния \(r\) от основания перпендикуляра, опущенного из заряда \(q\) на плоскость.

Решение. Согласно (2.2) поверхностная плотность зарядов на проводнике связана с электрическим полем вблизи проводника (в вакууме), как \(\sigma=\varepsilon_{0} E_{n}\). Следовательно, задача сводится к нахождению поля \(E\) вблизи проводящей плоскости. Методом изображений получаем, что в точке \(P\) (рис. 2.12), находящейся на расстоянии \(r\) от точки \(O\), поле вблизи плоскости:

\[
E=2 E_{q} \cos \alpha=2 \frac{q l}{4 \pi \varepsilon_{0} x^{2}} \frac{1}{x} .
\]

Значит,

\[
\sigma=-\frac{q l}{2 \pi\left(l^{2}+r^{2}\right)^{3 / 2}},
\]

где знак минус показывает, что инду-

\begin{figure}[h]
\begin{center}
  \includegraphics[alt={},max width=\textwidth]{3e870b25-6320-483d-9133-cdd5212c2707-064_363_382_319_766}
\captionsetup{labelformat=empty}
\caption{Рис. 2.12}
\end{center}
\end{figure}

цированный заряд противоположен по знаку точечному заряду \(q\).\\
2.5. Точечный заряд \(q\) находится на расстоянии \(l\) от проводящей плоскости. Найти работу, которую совершит электрическая сила, действующая на заряд \(q\) при его медленном удалении на очень большое расстояние от плоскости.

Решение. По определению работа этой силы при элементарном перемещении \(\mathrm{d} x\) (рис. 2.13)

\[
\delta A=F_{x} \mathrm{~d} x=-\frac{q^{2}}{4 \pi \varepsilon_{0}(2 x)^{2}} \mathrm{~d} x
\]

где выражение для силы получено с помощью метода изображений. Проинтегрировав это уравнение по \(x\) от \(l\) до \(\infty\), найдем

\begin{figure}[h]
\begin{center}
  \includegraphics[alt={},max width=\textwidth]{3e870b25-6320-483d-9133-cdd5212c2707-064_245_387_1041_765}
\captionsetup{labelformat=empty}
\caption{Рис. 2.13}
\end{center}
\end{figure}

\[
A=-\frac{q^{2}}{16 \pi \varepsilon_{0}} \int_{l}^{\infty} \frac{\mathrm{d} x}{x^{2}}=-\frac{q^{2}}{16 \pi \varepsilon_{0} l} .
\]

Замечание. Попытка решить эту задачу другим способом - через потенциал - приводит к неверному результату (он вдвое отличается от полученного нами). Это связано с тем, что соотношение \(A=q\left(\varphi_{1}-\varphi_{2}\right)\) справедливо только для потенциального поля. В системе же отсчета, связанной с проводящей плоскостью, электрическое поле индуцированных зарядов не потенциально: перемещение заряда \(q\) приводит к изменению распределения индуцированных зарядов, и их поле оказывается зависящим от времени.\\
2.6. Тонкое проводящее кольцо радиусом \(R\), имеющее заряд \(q\), расположено параллельно проводящей плоскости на расстоянии \(l\) от нее. Найти: 1) поверхностную плотность заряда в точке плоскости, расположенной симметрично относительно кольца; 2) потенциал электрического поля в центре кольца.

Решение. Легко догадаться, что в соответствии с методом изобра-

\begin{figure}[h]
\begin{center}
  \includegraphics[alt={},max width=\textwidth]{3e870b25-6320-483d-9133-cdd5212c2707-065_317_376_399_161}
\captionsetup{labelformat=empty}
\caption{Рис. 2.14}
\end{center}
\end{figure}

жений фиктивный заряд \(-q\) должен быть расположен на таком же кольце, но по другую сторону проводящей плоскости (рис. 2.14). Действительно, только в этом случае потенциал на средней плоскости между этими кольцами равен нулю, т. е. совпадает с потенциалом проводящей плоскости. Теперь воспользуемся известными нам формулами.

\begin{enumerate}
  \item Для нахождения \(\sigma\) в точке \(O\) необходимо согласно (2.2) найти напряженность \(E\) поля в этой точке (рис. 2.14). Выражение для \(E\) от одного кольца на оси было получено в примере 1 (см. с. 12). В нашем случае это выражение надо удвоить. В результате
\end{enumerate}

\[
\sigma=\frac{q l}{2 \pi\left(R^{2}+l^{2}\right)^{3 / 2}} .
\]

\begin{enumerate}
  \setcounter{enumi}{1}
  \item Потенциал в центре кольца равен алгебраической сумме потенциалов в этой точке, создаваемых зарядами \(q\) и \(-q\) :
\end{enumerate}

\[
\varphi=\frac{1}{4 \pi \varepsilon_{0}}\left(\frac{q}{R}-\frac{q}{\sqrt{R^{2}+4 l^{2}}}\right) .
\]

2.7. Три разноименных точечных заряда расположены так, как показано на рис. 2.15, \(a\), где \(A O B\) - прямой угол, образованный двумя проводящими полуплоскостями. Модуль каждого заряда равен \(|q|\), расстояния между ними указаны на рисунке. Найти: 1) суммарный заряд, индуцированный на

\begin{figure}[h]
\begin{center}
  \includegraphics[alt={},max width=\textwidth]{3e870b25-6320-483d-9133-cdd5212c2707-065_265_528_1501_166}
\captionsetup{labelformat=empty}
\caption{Рис. 2.15}
\end{center}
\end{figure}

проводящих полуплоскостях; 2) силу, действующую на заряд \(-q\).

Решение. Полуплоскости, образующие угол \(A O B\), уходят в бесконечность,

поэтому их потенциал \(\varphi=0\). Нетрудно сообразить, что системой, у которой эквипотенциальные поверхности с \(\varphi=0\) совпадают с проводящими полуплоскостями, является та, которая показана на рис. 2.15 , б. Поэтому действие зарядов, индуцированных на проводящих полуплоскостях, эквивалентно действию фиктивного точечного заряда \(-q\), помещенного в нижний левый угол пунктирного квадрата.

\begin{enumerate}
  \item Ответ на этот вопрос мы уже получили: \(-q\).
  \item Сведя систему к четырем точечным зарядам, легко найти и искомую силу как (см. рис. 2.15, б)
\end{enumerate}

\[
F=F_{2}-F_{1}=\frac{2 \sqrt{2}-1}{4 \pi \varepsilon_{0}} \frac{q^{2}}{2 a^{2}} .
\]

2.8. Емкость параллельных проводов. Два длинных прямых провода с одинаковым радиусом сечения расположены в воздухе параллельно друг другу. Расстояние между осями проводов в η раз больше радиуса сечения каждого провода. Найти емкость проводов на единицу их длины, при условии, что \(\eta \gg 1\).\\
Решение. Зарядим мысленно оба провода одинаковыми по модулю, но противоположными по знаку зарядами так, чтобы на единицу длины приходился заряд \(\lambda\). Тогда, по определению, искомая емкость


\begin{equation*}
C_{\text {ед }}=\lambda / U, \tag{1}
\end{equation*}


и все дальнейшее сводится к нахождению разности потенциалов между проводами.

Из рис. 2.16, на котором показаны графики зависимостей потенциалов \(\varphi_{+}\)и \(\varphi_{-}\)от положительно и отрицательно заряженных проводов, нетрудно понять, что искомая разность потенциалов


\begin{equation*}
U=\left|\Delta \varphi_{+}\right|+\left|\Delta \varphi_{-}\right|=2\left|\Delta \varphi_{+}\right| . \tag{2}
\end{equation*}


Напряженность электрического поля, создаваемого одним из проводов на расстоя-

\begin{figure}[h]
\begin{center}
  \includegraphics[alt={},max width=\textwidth]{3e870b25-6320-483d-9133-cdd5212c2707-066_317_330_1194_814}
\captionsetup{labelformat=empty}
\caption{Рис. 2.16}
\end{center}
\end{figure}

нии \(x\) от его оси, можно легко найти с помощью теоремы Гаусса: \(E=\lambda / 2 \pi \varepsilon_{0} x\). Тогда


\begin{equation*}
\left|\Delta \varphi_{+}\right|=\int_{a}^{b-a} E \mathrm{~d} x=\frac{\lambda}{2 \pi \varepsilon_{0}} \ln \frac{b-a}{a}, \tag{3}
\end{equation*}


где \(a\) - радиус сечения провода; \(b\) - расстояние между осями проводов. Из (1), (2) и (3) следует, что

\[
C_{\text {ед }}=\pi \varepsilon_{0} / \ln \eta,
\]

здесь учтено, что \(b \gg a\).\\
2.9. Четыре одинаковые металлические пластины расположены в воздухе на одинаковом расстоянии \(h\) друг от друга, при-

\begin{figure}[h]
\begin{center}
  \includegraphics[alt={},max width=\textwidth]{3e870b25-6320-483d-9133-cdd5212c2707-067_279_151_418_156}
\captionsetup{labelformat=empty}
\caption{Рис. 2.17}
\end{center}
\end{figure}

чем наружные пластины соединены между собой проводником. Площадь каждой пластины \(S\). Найти емкость этой системы (между точками 1 и 2 , рис. 2.17).

Решение. Сообщим пластинам 1 и 2 соответственно заряды \(q_{0}\) и \(-q_{0}\). Под действием возникшего между этими пластинами поля рассеяния (краевой эффект) произойдет перемещение заряда в замыкающем проводнике, после чего пластина \(A\) зарядится отрицательно, а пластина \(B\) - положительно. В пространстве между всеми пластинами возникает электрическое поле и соответствующее распределение потенциала φ (рис. 2.18). Заметим, что из симметрии системы следует, что потенциалы в ее середине, а также на наружных пластинах равны нулю. По определению емкость системы в данном случае


\begin{equation*}
C=q_{0} / U, \tag{1}
\end{equation*}


\begin{figure}[h]
\begin{center}
  \includegraphics[alt={},max width=\textwidth]{3e870b25-6320-483d-9133-cdd5212c2707-067_304_294_1101_166}
\captionsetup{labelformat=empty}
\caption{Рис. 2.18}
\end{center}
\end{figure}

где \(U\) - разность потенциалов между точками 1 и 2 , ее и надо найти. Из рис. 2.18 видно, что разность потенциалов между средними пластинами, т.е. \(U\), вдвое больше разности потенциалов между крайней парой пластин (как слева, так и справа). Это же относится и к напряженности поля:


\begin{equation*}
E=2 E^{\prime} . \tag{2}
\end{equation*}


А так как \(E\) c \(\sigma\), то мы можем утверждать, что в соответствии с (2) заряд \(q_{0}\) на пластине 1 делится на две части: \(q_{0} / 3\) - на левой стороне пластины 1 и \(2 q_{0} / 3\) - на правой стороне. Поэтому

\[
U=E h=\sigma h / \varepsilon_{0}=2 q_{0} h / 3 \varepsilon_{0} S,
\]

и емкость системы (между точками 1 и 2) равна

\[
C=3 \varepsilon_{0} S / 2 h .
\]

2.10. Распределение индуцированного заряда. Точечный заряд \(q\) находится между двумя большими параллельными проводящими пластинами 1 и 2 , отстоящими друг от друга на расстояние \(l\). Найти полные заряды \(q_{1}\) и \(q_{2}\), наведенные на каждой из пластин, если пластины соединены проводом и заряд \(q\) расположен на расстоянии \(l_{1}\) от левой пластины 1 (рис. 2.19. a).\\
Решение. Воспользуемся принципом суперпозиции. Поместим мысленно на плоскости \(P\) где-то еще такой же заряд \(q\). Ясно, что это удвоит поверхностный заряд на каждой пластине. Если же на поверхности \(P\) равномерно распределить некоторый заряд с поверхностной плотностью \(\sigma\), то электрическое поле станет простым для расчета (рис. 2.19, б).

\begin{figure}[h]
\begin{center}
  \includegraphics[alt={},max width=\textwidth]{3e870b25-6320-483d-9133-cdd5212c2707-068_365_435_419_714}
\captionsetup{labelformat=empty}
\caption{Рис. 2.19}
\end{center}
\end{figure}

Пластины соединены проводом, поэтому разность потенциалов между ними равна нулю. Отсюда

\[
E_{1 x} l_{1}+E_{2 x}\left(l-l_{1}\right)=0
\]

где \(E_{1 x}\) и \(E_{2 x}\) - проекции вектора \(\mathbf{E}\) на ось \(X\) слева и справа от плоскости \(P\) (рис. 2.19, б).\\
С другой стороны, очевидно, что

\[
\sigma=-\left(\sigma_{1}+\sigma_{2}\right),
\]

где согласно (2.2) \(\sigma_{1}=\varepsilon_{0} E_{1 n_{1}}=\varepsilon_{0} E_{1 x}\) и \(\sigma_{2}=\varepsilon_{0} E_{2 n_{2}}=\varepsilon_{0} E_{2 x}\) (знак минус, так как нормаль \(\mathbf{n}_{2}\) противоположна орту оси \(X\) ).\\
Исключив \(E_{1 x}\) и \(E_{2 x}\) из этих уравнений, получим

\[
\sigma_{1}=-\sigma\left(l-l_{1}\right) / l, \quad \sigma_{2}=-\sigma l_{1} / l .
\]

Аналогичный вид имеют и формулы для искомых зарядов \(q_{1}\) и \(q_{2}\) через заряд \(q\).\\
Решение же этой задачи с помощью метода изображений весьма затруднительно: необходим бесконечный ряд фиктивных зарядов, располагающихся по обе стороны от нашего заряда \(q\), и нахождение поля такой системы оказывается сложной задачей.

\section*{Глава 3}
\section*{Электрическое поле в диэлектрике}
\section*{§ 3.1. Поляризация диэлектрика}
Диэлектрики. Диэлектриками (или изоляторами) называют вещества, практически не проводящие электрического тока. Это значит, что в диэлектриках в отличие, например, от проводников нет зарядов, способных перемещаться на значительные расстояния, создавая ток.

При внесении даже нейтрального диэлектрика во внешнее электрическое поле обнаруживаются существенные изменения как в поле, так и в самом диэлектрике; последнее следует хотя бы из того, что на диэлектрик начинает действовать сила, увеличивается емкость конденсатора при заполнении его диэлектриком и др.

Чтобы понять, почему это происходит, надо прежде всего учесть, что диэлектрики состоят либо из нейтральных молекул, либо из заряженных ионов, находящихся в узлах кристаллической решетки (ионные кристаллы, например, типа NaCl ). Сами же молекулы могут быть полярными и неполярными. У полярных молекул центр «тяжести» отрицательного заряда сдвинут относительно центра тяжести положительных зарядов, в результате чего они обладают собственным дипольным моментом p. Неполярные же молекулы собственным дипольным моментом не обладают: у них центры тяжести положительного и отрицательного зарядов совпадают.

Поляризация. Под действием внешнего электрического поля происходит поляризация диэлектрика. Это явление заключается в следующем. Если диэлектрик состоит из неполярных молекул, то в пределах каждой молекулы происходит смещение зарядов - положительных по полю, отрицательных против поля. Если же диэлектрик состоит из полярных молекул, то при отсутствии внешнего поля их дипольные моменты

ориентированы совершенно хаотически (из-за теплового движения). Под действием же внешнего поля дипольные моменты ориентируются преимущественно в направлении внешнего поля. Наконец, в диэлектрических кристаллах типа NaCl при включении внешнего поля все положительные ионы смещаются по полю, отрицательные - против поля*.

Таким образом, механизм поляризации связан с конкретным строением диэлектрика. Однако для дальнейшего существенно лишь то, что независимо от механизма поляризации в этом процессе все положительные заряды смещаются по полю, а отрицательные - против поля. Заметим, что смещения зарядов в обычных условиях весьма малы даже по сравнению с размерами молекул, это связано с тем, что напряженность внешнего поля, действующего на диэлектрик, значительно меньше напряженности внутренних электрических полей в молекулах.

Объемные и поверхностные связанные заряды. В результате поляризации на поверхности диэлектрика, а также, вообще говоря, и в его объеме появляются нескомпенсированные заряды. Чтобы понять, каким образом возникают эти заряды (особенно объемные), обратимся к следующей модели. Пусть имеется пластина из нейтрального неоднородного диэлектрика (рис. 3.1, \(a\) ), у которого, например, плотность как-то увеличивается с ростом координаты \(x\). Обозначим \(\rho_{+}^{\prime}\) и \(\rho_{-}^{\prime}\) - модули объемной плотности положительного и отрицательного зарядов в веществе (эти заряды связаны с ядрами и электронами).

\begin{figure}[h]
\begin{center}
  \includegraphics[alt={},max width=\textwidth]{3e870b25-6320-483d-9133-cdd5212c2707-070_348_636_1244_294}
\captionsetup{labelformat=empty}
\caption{Рис. 3.1}
\end{center}
\end{figure}

\footnotetext{\begin{itemize}
  \item Существуют ионные кристаллы, поляризованные даже при отсутствии внешнего поля. Этим же свойством обладают диэлектрики, называемые электретами (они подобны постоянным магнитам).
\end{itemize}
}При отсутствии внешнего поля в каждой точке диэлектрика \(\rho_{+}^{\prime}=\rho_{-}^{\prime}\), ибо диэлектрик электрически нейтрален, но в силу неоднородности диэлектрика как \(\rho_{+}^{\prime}\), так и \(\rho_{-}^{\prime}\) увеличиваются с ростом \(x\) (рис. 3.1, б). Из этого рисунка видно, что если внешнего поля нет, то оба распределения в точности накладываются друг на друга (распределение \(\rho_{+}^{\prime}(x)\) показано сплошной линией, а распределение \(\rho_{-}^{\prime}(x)\) - пунктирной).

Включение внешнего поля \(\mathbf{E}\) приведет к смещению положительных зарядов по полю, отрицательных - против поля, и оба распределения сдвинутся относительно друг друга (рис. 3.1, в). В итоге появятся нескомпенсированные заряды на поверхности диэлектрика и в его объеме (на нашем рисунке в объеме появился отрицательный нескомпенсированный заряд). Заметим, что изменение направления поля на обратное приведет к изменению знака всех этих зарядов. Нетрудно также видеть, что в случае пластины из однородного диэлектрика каждое распределение \(\rho_{+}^{\prime}(x)\) и \(\rho_{-}^{\prime}(x)\) имело бы П-образную форму, и при их относительном смещении в поле Е возникли бы только поверхностные нескомпенсированные заряды.

Нескомпенсированные заряды, появляющиеся в результате поляризации диэлектрика, называют поляризационньлми или связанными. Последним термином хотят подчеркнуть, что свобода перемещения таких зарядов ограничена. Они могут смещаться лишь внутри электрически нейтральных молекул. Связанные заряды мы будем отмечать штрихом ( \(q^{\prime}, \rho^{\prime}, \sigma^{\prime}\) ).

Итак, при поляризации диэлектрика в нем могут возникать в общем случае и объемные и поверхностные связанные заряды.

Заряды, которые не входят в состав молекул диэлектрика, называют сторонними*. Эти заряды могут находиться как внутри, так и вне диэлектрика.

Поле в диэлектрике. Полем \(\mathbf{E}\) в диэлектрике мы будем называть величину, являющуюся суперпозицией поля \(\mathbf{E}_{0}\) сторонних зарядов и поля \(\mathbf{E}^{\prime}\) связанных зарядов:


\begin{equation*}
\mathbf{E}=\mathbf{E}_{0}+\mathbf{E}^{\prime}, \tag{3.1}
\end{equation*}


\footnotetext{\begin{itemize}
  \item Сторонние заряды часто называют свободньми, но последнее название для ряда случаев является неудачным: сторонние заряды бывают и не свободными.
\end{itemize}
}где \(\mathbf{E}_{0}\) и \(\mathbf{E}^{\prime}\) представляют собой макрополя, т. е. усредненные по физически бесконечно малому объему микрополя соответственно сторонних и связанных зарядов. Ясно, что определенное таким образом поле \(\mathbf{E}\) в диэлектрике является также макрополем.

\section*{§ 3.2. Поляризованность P}
Определение. Для количественного описания поляризации диэлектрика естественно взять дипольный момент единицы объема. Если внешнее поле или диэлектрик (или то и другое) неоднородны, степень поляризации оказывается различной в разных точках диэлектрика. Чтобы охарактеризовать поляризацию в данной точке, мысленно выделяют физически бесконечно малый объем \(\Delta V\), содержащий эту точку, затем находят векторную сумму дипольных моментов молекул в этом объеме и составляют отношение


\begin{equation*}
\mathbf{P}=\frac{1}{\Delta V} \sum \mathbf{p}_{i} . \tag{3.2}
\end{equation*}


Определенный таким образом вектор \(\mathbf{P}\) называют поляризованностъю диэлектрика. Этот вектор равен дипольному моменту единицы объема вещества.

Есть еще два полезных представления вектора \(\mathbf{P}\). Пусть в объеме \(\Delta V\) содержится \(\Delta N\) диполей. Умножим и разделим правую часть выражения (3.2) на \(\Delta N\). Тогда можно записать


\begin{equation*}
\mathbf{P}=n\langle\mathbf{p}\rangle, \tag{3.3}
\end{equation*}


где \(n=\Delta N / \Delta V\) - концентрация молекул (их число в единице объема); \(\langle\mathbf{p}\rangle=\left(\sum \mathbf{p}_{i}\right) / \Delta N\) - средний дипольный момент одной молекулы.

Другое выражение для \(\mathbf{P}\) соответствует модели диэлектрика как совокупности положительной и отрицательной «жидкостей». Выделим очень малый объем \(\Delta V\) внутри диэлектрика. При возникновении поляризации входящий в этот объем положительный заряд \(\rho_{+}^{\prime} \Delta V\) сместится относительно отрицательного заряда на величину 1 , и эти заряды приобретут дипольный момент \(\Delta \mathbf{p}=\rho_{+}^{\prime} \Delta V \cdot \mathbf{1}\). Разделив обе части этого равенства на \(\Delta V\), получим выражение для дипольного момента единицы объема,\\
т. е. вектор P:


\begin{equation*}
\mathbf{P}=\rho_{+}^{\prime} \mathbf{l} . \tag{3.4}
\end{equation*}


Единицей поляризованности \(\mathbf{P}\) является кулон на квадратный метр (Кл/м²).

Связь между \(\mathbf{P}\) и \(\mathbf{E}\). Как показывает опыт, для обширного класса диэлектриков и широкого круга явлений поляризованность \(\mathbf{P}\) зависит линейно от напряженности \(\mathbf{E}\) поля в диэлектрике. Если диэлектрик изотропный и \(\mathbf{E}\) не слишком велико, то


\begin{equation*}
\mathbf{P}=\varkappa \varepsilon_{0} \mathbf{E}, \tag{3.5}
\end{equation*}


где \(x\) - безразмерная величина, называемая диэлектрической восприимчивостью вещества. Эта величина не зависит от \(\mathbf{E}\), она характеризует свойства самого диэлектрика. Всегда \(x>0\).

В дальнейшем, если специально не оговорено, мы будем иметь в виду только изотропные диэлектрики, для которых справедливо соотношение (3.5).

Существуют, однако, и диэлектрики, для которых (3.5) не применимо. Это некоторые ионные кристаллы и электреты (см. сноску на с. 69), а также сегнетоэлектрики. У сегнетоэлектриков связь между \(\mathbf{P}\) и \(\mathbf{E}\) нелинейная и зависит, кроме того, от предыстории диэлектрика, т. е. от предшествующих значений \(\mathbf{E}\) (это явление называют гистерезисом).

\section*{§ 3.3. Свойства поля вектора P}
Теорема Гаусса для поля вектора Р. Как мы сейчас покажем, поле вектора \(\mathbf{P}\) обладает следующим замечательным и важным свойством. Оказывается, поток вектора \(\mathbf{P}\) сквозь произвольную замкнутую поверхность \(S\) равен взятому с обратным знаком избыточному связанному заряду диэлектрика в объеме, охватываемом поверхностью \(S\), т. е.


\begin{equation*}
\oint \mathbf{P} \mathrm{d} \mathbf{S}=-q_{\text {внутр }}^{\prime} \tag{3.6}
\end{equation*}


Это уравнение и выражает теорему Гаусса для вектора \(\mathbf{P}\).\\
Доказательство теоремы. Пусть произвольная замкнутая поверхность \(S\) охватывает часть диэлектрика (рис. 3.2, \(a\), где диэлектрик заштрихован). При включении внешнего электри-

ческого поля диэлектрик поляризуется - положительные заряды сместятся относительно отрицательных. Найдем заряд, который проходит через элемент \(\mathrm{d} S\) замкнутой поверхности \(S\) нару-

\begin{figure}[h]
\begin{center}
  \includegraphics[alt={},max width=\textwidth]{3e870b25-6320-483d-9133-cdd5212c2707-074_271_580_145_588}
\captionsetup{labelformat=empty}
\caption{Рис. 3.2}
\end{center}
\end{figure}

жу (рис. 3.2, б).

Пусть \(\mathbf{l}_{+}\)и \(\mathbf{l}_{-}\)- векторы, характеризующие смещения положительного и отрицательного связанных зарядов в результате поляризации. Тогда ясно, что через элемент поверхности \(d S\) наружу поверхности \(S\) выйдет положительный заряд \(\rho_{+}^{\prime} l_{+} \mathrm{d} S \cos \alpha\), заключенный во «внутренней» части косого цилиндра (рис. 3.2, б). Кроме того, через элемент \(d S\) войдет внутрь поверхности \(S\) отрицательный заряд \(\rho_{-}^{\prime} l_{-} \mathrm{d} S \cos \alpha\), заключенный во «внешней» части косого цилиндра. Но мы знаем, что перенос отрицательного заряда в некотором направлении эквивалентен переносу положительного заряда в противоположном направлении. Учитывая это, можно записать суммарный связанный заряд, выходящий наружу поверхности \(S\) через элемент \(\mathrm{d} S\), как

\[
\mathrm{d} q^{\prime}=\rho_{+} l_{+} \mathrm{d} S \cos \alpha+\left|\rho_{-}^{\prime}\right| l_{-} \mathrm{d} S \cos \alpha .
\]

Поскольку \(\left|\rho_{-}^{\prime}\right|=\rho_{+}^{\prime}\),


\begin{equation*}
\mathrm{d} q^{\prime}=\rho_{+}^{\prime}\left(l_{+}+l_{-}\right) \mathrm{d} S \cos \alpha=\rho_{+}^{\prime} l \mathrm{~d} S \cos \alpha, \tag{3.7}
\end{equation*}


где \(l=l_{+}+l_{-}\)- расстояние, на которое сместились относительно друг друга положительные и отрицательные связанные заряды диэлектрика при поляризации.

Далее, согласно (3.4) \(\rho_{+}^{\prime} l=P\) и \(\mathrm{d} q^{\prime}=P \mathrm{~d} S \cos \alpha\), или


\begin{equation*}
\mathrm{d} q^{\prime}=P_{n} \mathrm{~d} S=\mathbf{P} \mathrm{d} \mathbf{S} . \tag{3.8}
\end{equation*}


Проинтегрировав это выражение по всей замкнутой поверхности \(S\), мы найдем весь заряд, который вышел при поляризации из объема, охватываемого поверхностью \(S\), он равен \(\oint \mathbf{P d S}\). В результате внутри поверхности \(S\) останется некоторый избыточный связанный заряд \(q^{\prime}\). Ясно, что вышедший заряд должен быть равен с обратным знаком оставшемуся внутри поверхности \(S\) избыточному связанному заряду, и мы приходим к (3.6).

Дифференциальная форма уравнения (3.6). В дифференциальной форме уравнение (3.6) - теорема Гаусса для поля вектора \(\mathbf{P}\) - имеет следующий вид:


\begin{equation*}
\nabla \cdot \mathbf{P}=-\rho^{\prime}, \tag{3.9}
\end{equation*}


т. е. дивергенция поля вектора \(\mathbf{P}\) равна с обратным знаком объемной плотности избыточного связанного заряда в той же точке. Это уравнение можно получить из (3.6) точно таким же путем, как и аналогичное уравнение для вектора \(\mathbf{E}\) (см. с. 23). Достаточно в проводимых там рассуждениях заменить \(\mathbf{E}\) на \(\mathbf{P}\) и \(\rho\) на \(\rho^{\prime}\).

Когда в диэлектрике \(\rho^{\prime}=\mathbf{0}\) ? Как мы сейчас покажем, объемная плотность избыточных связанных зарядов внутри диэлектрика будет равна нулю при одновременном выполнении двух условий:

\begin{enumerate}
  \item диэлектрик должен быть однородным;
  \item внутри него не должно быть сторонних зарядов ( \(\rho=0\) ).
\end{enumerate}

Действительно, из основного свойства поля вектора \(\mathbf{P}\) (3.6) следует, что в случае однородного диэлектрика можно, заменив \(\mathbf{P}\) на \(\varkappa \varepsilon_{0} \mathbf{E}\) согласно (3.5), вынести \(\varkappa\) из-под знака интеграла и записать

\[
\varkappa \oint \varepsilon_{0} E \mathrm{~d} S=-q^{\prime} .
\]

Оставшийся интеграл есть не что иное, как алгебраическая сумма всех зарядов - сторонних и связанных - внутри рассматриваемой замкнутой поверхности \(S\), т. е. \(q+q^{\prime}\). Поэтому \(\varkappa\left(q+q^{\prime}\right)=-q^{\prime}\), откуда


\begin{equation*}
q^{\prime}=\frac{\varkappa}{1+\varkappa} q . \tag{3.10}
\end{equation*}


Это соотношение между избыточным связанным зарядом \(q^{\prime}\) и сторонним зарядом \(q\) справедливо для любого объема внутри диэлектрика, в частности и для физически бесконечно малого, когда \(q^{\prime}-\mathrm{d} q^{\prime}=\rho^{\prime} \mathrm{d} V\) и \(q-\mathrm{d} q=\rho \mathrm{d} V\). Тогда (3.10) после сокращения на \(\mathrm{d} V\) примет вид


\begin{equation*}
\rho^{\prime}=\frac{x}{1+x} \rho . \tag{3.11}
\end{equation*}


Значит, в однородном диэлектрике \(\rho^{\prime}=0\), если \(\rho=0\).

Таким образом, если в произвольное электрическое поле поместить однородный изотропный диэлектрик какой угодно формы, можно быть уверенным, что при его поляризации появятся только поверхностные связанные заряды, объемные же избыточные связанные заряды во всех точках такого диэлектрика будут равны нулю.

Граничные условия для вектора \(\mathbf{P}\). Рассмотрим поведение вектора \(\mathbf{P}\) на границе раздела двух однородных изотропных диэлектриков. Мы только что установили, что у таких диэлектриков объемного избыточного связанного заряда нет и в результате поляризации появляется только поверхностный связанный заряд.

Найдем связь между поляризованностью \(\mathbf{P}\) и поверхностной плотностью \(\sigma^{\prime}\) связанных зарядов на границе раздела диэлектриков. Для этого воспользуемся свойством (3.6) поля вектора P. Возьмем в качестве замкнутой поверхности небольшой плоский цилиндр, торцы которого расположим по разные стороны границы раздела (рис. 3.3). Высоту цилиндра будем предполагать ничтожно малой, а площадь \(\Delta S\) каждого торца настолько малой, что во всех точках каждого торца цилиндра вектор \(\mathbf{P}\) был бы одинаков (это же касается и поверхностной плотности \(\sigma^{\prime}\) связанного заряда). Пусть \(\mathbf{n}\) - общая нормаль к границе раздела в данном месте. Условимся всегда проводить вектор n от диэлектрика 1 к диэлектрику 2.

Пренебрегая потоком вектора \(\mathbf{P}\) сквозь боковую поверхность выбранного нами цилиндра, запишем согласно (3.6):

\[
P_{2 n} \Delta S+P_{1 n^{\prime}} \Delta S=-\sigma^{\prime} \mathrm{DS},
\]

где \(P_{2 n}\) и \(P_{1 n^{\prime}}\) - проекции вектора \(\mathbf{P}\) в диэлектрике 2 на нормаль \(\mathbf{n}\) и в диэлектрике 1 на нормаль \(\mathbf{n}^{\prime}\) (рис. 3.3).

Учитывая, что проекция вектора \(\mathbf{P}\) на нормаль \(\mathbf{n}^{\prime}\) равна с обратным знаком проекции этого вектора на противоположную (общую) нормаль \(\mathbf{n}\), т. е. \(P_{1 n^{\prime}}=-P_{1 n}\), пере-

\begin{figure}[h]
\begin{center}
  \includegraphics[alt={},max width=\textwidth]{3e870b25-6320-483d-9133-cdd5212c2707-076_245_324_1367_826}
\captionsetup{labelformat=empty}
\caption{Рис. 3.3}
\end{center}
\end{figure}

пишем предыдущее уравнение после сокращения на \(\Delta S\) в следующем виде:


\begin{equation*}
P_{2 n}-P_{1 n}=-\sigma^{\prime} . \tag{3.12}
\end{equation*}


Это значит, что на границе раздела диэлектриков нормальная составляющая вектора \(\mathbf{P}\) испытывает разрыв, величина которого зависит от \(\sigma^{\prime}\). В частности, если среда 2 вакуум, то \(P_{2 n}=0\), и условие (3.12) приобретает более простой вид:


\begin{equation*}
\sigma^{\prime}=P_{n}, \tag{3.13}
\end{equation*}


где \(P_{n}\) - проекция вектора \(\mathbf{P}\) на внешнюю нормаль к поверхности данного диэлектрика. Знак проекции \(P_{n}\) определяет и знак поверхностного связанного заряда \(\sigma^{\prime}\) в данном месте. Последнюю формулу можно представить в другом виде, а именно в соответствии с формулой (3.5) можно записать:


\begin{equation*}
\sigma^{\prime}=\varkappa \varepsilon_{0} E_{n}, \tag{3.14}
\end{equation*}


где \(E_{n}\) - проекция вектора \(\mathbf{E}\) (внутри диэлектрика вблизи от его поверхности) на внешнюю нормаль. Здесь также знак \(E_{n}\) определяет знак \(\sigma^{\prime}\).

Замечание о поле вектора Р. Соотношения (3.6) и (3.13) нередко дают основание ошибочно думать, что поле вектора \(\mathbf{P}\) зависит только от связанных зарядов. На самом деле это не так. Поле вектора \(\mathbf{P}\), как и \(\mathbf{E}\), зависит от всех зарядов, как связанных, так и сторонних, об этом говорит хотя бы уже тот факт, что векторы \(\mathbf{P}\) и \(\mathbf{E}\) связаны друг с другом соотношением \(\mathbf{P}=\varkappa \varepsilon_{0} \mathbf{E}\). Связанные заряды определяют не поле вектора \(\mathbf{P}\), а лишь поток этого вектора сквозь замкнутую поверхность \(S\). Более того, этот поток определяется не всеми связанными зарядами, а только теми, которые охватывает поверхность \(S\).

\section*{§ 3.4. Вектор D}
Tеорема Гаусса для поля вектора D. Поскольку источниками поля \(\mathbf{E}\) являются все электрические заряды - сторонние и связанные, теорему Гаусса для поля \(\mathbf{E}\) можно записать так:


\begin{equation*}
\oint \varepsilon_{0} \mathbf{E} \mathrm{~d} \mathbf{S}=\left(q+q^{\prime}\right)_{\text {внутр }}, \tag{3.15}
\end{equation*}


где \(q\) и \(q^{\prime}\) - сторонние и связанные заряды, охватываемые поверхностью \(S\). Появление связанных зарядов \(q^{\prime}\) усложняет дело, и формула (3.15) оказывается малополезной для нахож-

дения поля E в диэлектрике даже при «достаточно хорошей» симметрии. Действительно, эта формула выражает свойства неизвестного поля \(\mathbf{E}\) через связанные заряды \(q^{\prime}\), которые в свою очередь определяются неизвестным полем Е.

Это затруднение, однако, можно обойти, если выразить заряд \(q^{\prime}\) через поток вектора \(\mathbf{P}\) по формуле (3.6). Тогда выражение (3.15) можно преобразовать к такому виду:


\begin{equation*}
\oint\left(\varepsilon_{0} \mathbf{E}+\mathbf{P}\right) \mathrm{d} \mathbf{S}=q_{\text {внутр }} . \tag{3.16}
\end{equation*}


Величину, стоящую под интегралом в скобках, обозначают буквой \(\mathbf{D}\). Итак, мы нашли вспомогательный вектор \(\mathbf{D}\) :


\begin{equation*}
\mathbf{D}=\varepsilon_{0} \mathbf{E}+\mathbf{P}, \tag{3.17}
\end{equation*}


поток которого сквозь произвольную замкнутую поверхность равен алгебраической сумме сторонних зарядов, охватываемых этой поверхностью:


\begin{equation*}
\oint \mathbf{D} \mathrm{d} \mathbf{S}=q_{\text {внутр }} . \tag{3.18}
\end{equation*}


Это и есть теорема Гаусса для поля вектора D.\\
Заметим, что вектор D представляет собой сумму двух совершенно различных величин: \(\varepsilon_{0} \mathbf{E}\) и \(\mathbf{P}\). Поэтому он действительно вспомогательный вектор, не имеющий какого-либо глубокого физического смысла. Однако свойство поля вектора \(\mathbf{D}\), выражаемое уравнением (3.18), оправдывает введение этого вектора: во многих случаях он значительно упрощает изучение поля в диэлектриках*.

Соотношения (3.17) и (3.18) справедливы для любого диэлектрика, как изотропного, так и анизотропного.

Как видно из выражения (3.17), размерность вектора \(\mathbf{D}\) та же, что и вектора P. Единицей величины \(\mathbf{D}\) служит кулон на квадратный метр (Кл/м²).

\footnotetext{\begin{itemize}
  \item Величину \(\mathbf{D}\) часто называют электрическим смещением или электрической индукцией, однако мы не будем пользоваться этими терминами, чтобы лишний раз подчеркнуть вспомогательный характер вектора D.
\end{itemize}
}Дифференциальная форма уравнения (3.18):


\begin{equation*}
\nabla \cdot \mathbf{D}=\rho, \tag{3.19}
\end{equation*}


т. е. дивергенция поля вектора \(\mathbf{D}\) равна объемной плотности стороннего заряда в той же точке.

Это уравнение можно получить из (3.18) тем же способом, как это было проделано в случае поля E (см. с. 23). Достаточно в проводимых там рассуждениях заменить \(\mathbf{E}\) на \(\mathbf{D}\) и учесть лишь сторонние заряды.

В тех точках, где дивергенция \(\mathbf{D}\) положительна, мы имеем источники поля \(\mathbf{D}(\rho>0)\), а в тех точках, где она отрицательна, - стоки поля \(\mathbf{D}\) ( \(\rho<0\) ).

Связь между векторами D и E. B случае изотропных диэлектриков поляризованность \(\mathbf{P}=\varkappa \varepsilon_{0} \mathbf{E}\). Подставив это соотношение в (3.17), получим \(\mathbf{D}=\varepsilon_{0}(1+\varkappa) \mathbf{E}\), или


\begin{equation*}
\mathbf{D}=\varepsilon \varepsilon_{0} \mathbf{E}, \tag{3.20}
\end{equation*}


где \(\varepsilon\) - диэлектрическая проницаемость вещества:


\begin{equation*}
\varepsilon=1+\varkappa . \tag{3.21}
\end{equation*}


Диэлектрическая проницаемость \(\varepsilon\) (как и \(\varkappa\) ) является основной электрической характеристикой диэлектрика. Для всех веществ \(\varepsilon>1\), для вакуума \(\varepsilon=1\). Значения \(\varepsilon\) зависят от природы диэлектрика и колеблются от величин, весьма мало отличающихся от единицы (газы) до многих тысяч (у некоторых керамик). Большое значение \(\varepsilon\) имеет вода \((\varepsilon=81)\).

Из формулы (3.20) видно, что в изотропных диэлектриках вектор D коллинеарен вектору E. B анизотропных же диэлектриках эти векторы, вообще говоря, не коллинеарны.

Поле вектора \(\mathbf{D}\) наглядно можно изобразить с помощью линий вектора \(\mathbf{D}\), направление и густота которых определяются точно так же, как и для линий вектора Е. Линии вектора E могут начинаться и заканчиваться как на сторонних, так и на связанных зарядах; мы говорим, что источниками и стоками поля Е являются любые заряды. Источниками же и стоками поля вектора \(\mathbf{D}\) являются только сторонние заряды: только на них могут начинаться и заканчиваться линии вектора D. Через области поля, где находятся связанные заряды, линии вектора D проходят не прерываясь.

Замечание о поле вектора D. Поле вектора D зависит, вообще говоря, как от сторонних, так и от связанных зарядов (как и поле вектора \(\mathbf{E}\) ). Об этом говорит уже соотношение \(\mathbf{D}=\varepsilon \varepsilon_{0} \mathbf{E}\). Однако в некоторых симметричных случаях поле вектора \(\mathbf{D}\) можно определить, используя только сторонние заряды. Именно для таких случаев вектор D является особенно полезным, резко упрощая расчет. Вместе с тем это дает повод довольно часто ошибочно думать, что поле \(\mathbf{D}\) якобы зависит всегда только от сторонних зарядов, и неверно трактовать законы (3.18) и (3.19). Эти законы выражают только определенное свойство поля вектора \(\mathbf{D}\), само же поле этого вектора они не определяют.

Проиллюстрируем вышесказанное на нескольких примерах.\\
Пример 1. Точечный сторонний заряд \(q\) находится в центре шара радиусом \(a\) из однородного изотропного диэлектрика проницаемости ɛ. Найти напряженность \(E\) поля как функцию расстояния \(r\) от центра данного шара.\\
Симметрия системы позволяет для решения интересующего нас вопроса использовать теорему Гаусса для вектора D (воспользоваться аналогичной теоремой для поля \(\mathbf{E}\) здесь не представляется возможным, поскольку нам не известен связанный заряд). Для сферы радиусом \(r\) с центром в точке нахождения заряда \(q\) можно записать: \(4 \pi r^{2} D_{r}=q\). Отсюда находим \(D_{r}\) и по формуле (3.20) иско-

\begin{figure}[h]
\begin{center}
  \includegraphics[alt={},max width=\textwidth]{3e870b25-6320-483d-9133-cdd5212c2707-080_310_342_903_806}
\captionsetup{labelformat=empty}
\caption{Рис. 3.4}
\end{center}
\end{figure}

мую напряженность \(E\) :

\[
E_{r}(r<a)=\frac{1}{4 \pi \varepsilon_{0}} \frac{q}{\varepsilon r^{2}}, \quad E_{r}(r>a)=\frac{1}{4 \pi \varepsilon_{0}} \frac{q}{r^{2}} .
\]

Графики зависимостей \(D(r)\) и \(E(r)\) показаны на рис. 3.4.\\
Пример 2. Пусть система состоит из точечного заряда \(q>0\) и произвольного куска однородного изотропного диэлектрика (рис. 3.5), где \(S\) - некоторая замкнутая поверхность. Выясним, что произойдет с полем векторов \(\mathbf{E}\) и \(\mathbf{D}\), а также с их потоками сквозь поверхность \(S\), если диэлектрик

\begin{figure}[h]
\begin{center}
  \includegraphics[alt={},max width=\textwidth]{3e870b25-6320-483d-9133-cdd5212c2707-080_185_330_1520_824}
\captionsetup{labelformat=empty}
\caption{Рис. 3.5}
\end{center}
\end{figure}

удалить.

В любой точке пространства поле \(\mathbf{E}\) обусловлено как зарядом \(q\), так и связанными зарядами поляризованного диэлектрика. Так как в нашем случае \(\mathbf{D}=\varepsilon \varepsilon_{0} \mathbf{E}\), то это относится и к полю вектора \(\mathbf{D}\) : оно также определяется как сторонним зарядом \(q\), так и связанными зарядами диэлектрика.

Удаление диэлектрика приведет к изменению поля \(\mathbf{E}\), а значит, и поля \(\mathbf{D}\). Изменится и поток вектора \(\mathbf{E}\) сквозь поверхность \(S\), так как внутри этой поверхности исчезнут отрицательные связанные заряды. Поток же вектора D сквозь поверхность \(S\) остается прежним, несмотря на изменение самого поля \(\mathbf{D}\).\\
Пример 3. Рассмотрим систему, в которой нет сторонних зарядов, но имеются только связанные заряды. Такой системой может быть, например, шар из электрета (см. сноску на с. 69). На рис. 3.6, \(a\) показано поле \(\mathbf{E}\) такой системы. Что можно сказать о соответствующем поле вектора \(\mathbf{D}\) ?

\begin{figure}[h]
\begin{center}
  \includegraphics[alt={},max width=\textwidth]{3e870b25-6320-483d-9133-cdd5212c2707-081_294_487_806_229}
\captionsetup{labelformat=empty}
\caption{Рис. 3.6}
\end{center}
\end{figure}

Прежде всего отсутствие сторонних зарядов означает, что нет источников поля D: линии вектора D нигде не начинаются и нигде не кончаются. Но поле D есть, оно показано на рис. 3.6, б. Вне шара направления линий векторов E и D совпадают, внутри же шара их направления противоположны: здесь соотношение \(\mathbf{D}=\varepsilon \varepsilon_{0} \mathbf{E}\) уже несправедливо, и \(\mathbf{D}=\varepsilon_{0} \mathbf{E}+\mathbf{P}\).

\section*{§ 3.5. Условия на границе}
Рассмотрим поведение векторов \(\mathbf{E}\) и \(\mathbf{D}\) сначала на границе раздела двух однородных изотропных диэлектриков. Пусть для большей общности на границе раздела этих диэлектриков находится поверхностный сторонний заряд. Искомые условия нетрудно получить с помощью двух теорем: теоремы о циркуляции вектора \(\mathbf{E}\) и теоремы Гаусса для вектора \(\mathbf{D}\) :

\[
\oint \mathbf{E} \mathrm{d} \mathbf{l}=0, \quad \oint \mathbf{D} \mathrm{~d} \mathbf{S}=q_{\text {внутр }} .
\]

Условие для вектора Е. Пусть поле вблизи границы раздела в диэлектрике 1 равно \(\mathbf{E}_{1}\), а в диэлектрике \(2-\mathbf{E}_{2}\). Возьмем небольшой вытянутый прямоугольный контур, ориентировав его так, как

\begin{figure}[h]
\begin{center}
  \includegraphics[alt={},max width=\textwidth]{3e870b25-6320-483d-9133-cdd5212c2707-082_146_354_149_806}
\captionsetup{labelformat=empty}
\caption{Рис. 3.7}
\end{center}
\end{figure}

показано на рис. 3.7.

Стороны контура, параллельные границе раздела, должны иметь такую длину, чтобы в ее пределах поле \(\mathbf{E}\) в каждом диэлектрике можно было считать одинаковым, а «высота» контура должна быть пренебрежимо малой. Тогда согласно теореме о циркуляции вектора \(\mathbf{E}\)

\[
E_{2 \tau} l+E_{1 \tau^{\prime}} l=0,
\]

где проекции вектора \(\mathbf{E}\) взяты на направление обхода контура, указанное на рисунке стрелками. Если на нижнем участке контура проекцию вектора \(\mathbf{E}\) взять не на орт \(\boldsymbol{\tau}\) ', а на общий орт \(\boldsymbol{\tau}\), то \(E_{1 \tau^{\prime}}=-E_{1 \tau}\) и из предыдущего уравнения следует, что


\begin{equation*}
E_{1 \tau}=E_{2 \tau}, \tag{3.22}
\end{equation*}


т. е. тангенциальная составляющая \(\mathbf{E}\) оказывается одинаковой по обе стороны границы раздела (не претерпевает скачка).

Условие для вектора D. Возьмем очень малой высоты цилиндр, расположив его на границе раздела двух диэлектриков (рис. 3.8). Сечение цилиндра должно быть таким, чтобы в пределах каждого его торца вектор D был одинаков. Тогда согласно теореме Гаусса для

\begin{figure}[h]
\begin{center}
  \includegraphics[alt={},max width=\textwidth]{3e870b25-6320-483d-9133-cdd5212c2707-082_225_384_1129_776}
\captionsetup{labelformat=empty}
\caption{Рис. 3.8}
\end{center}
\end{figure}

вектора D

\[
D_{2 n} \Delta S+D_{1 n^{\prime}} \Delta S=\sigma \Delta S,
\]

где \(\sigma\) - поверхностная плотность стороннего заряда на границе раздела. Взяв обе проекции вектора \(\mathbf{D}\) на общую нормаль \(\mathbf{n}\) (она направлена от диэлектрика 1 к диэлектрику 2), получим \(D_{1 n^{\prime}}=-D_{1 n}\), и предыдущее уравнение можно привести к виду


\begin{equation*}
D_{2 n}-D_{1 n}=\sigma . \tag{3.23}
\end{equation*}


Из этого соотношения видно, что нормальная составляющая вектора D, вообще говоря, претерпевает скачок при переходе границы раздела. Однако если сторонние заряды на границе раздела отсутствуют ( \(\sigma=0\) ), то


\begin{equation*}
D_{1 n}=D_{2 n} \tag{3.24}
\end{equation*}


В этом случае нормальные составляющие вектора \(\mathbf{D}\) скачка не испытывают, они оказываются одинаковыми по разные стороны границы раздела.

Таким образом, если на границе раздела двух однородных изотропных диэлектриков сторонних зарядов нет, то при переходе этой границы составляющие \(E_{\tau}\) и \(D_{n}\) не изменяются. Cоставляющие же \(E_{n}\) и \(D_{\tau}\) претерпевают скачок.

Преломление линий Е и D. Полученные нами условия для со-

\begin{figure}[h]
\begin{center}
  \includegraphics[alt={},max width=\textwidth]{3e870b25-6320-483d-9133-cdd5212c2707-083_362_241_790_70}
\captionsetup{labelformat=empty}
\caption{Рис. 3.9}
\end{center}
\end{figure}

ставляющих векторов \(\mathbf{E}\) и \(\mathbf{D}\) на границе раздела двух диэлектриков означают, как мы сейчас увидим, что линии этих векторов испытывают на этой границе излом, преломляются (рис. 3.9). Найдем соотношение между углами \(\alpha_{1}\) и \(\alpha_{2}\).

Если сторонних зарядов на границе раздела нет, то согласно (3.22) и (3.24) \(E_{2 \tau}=E_{1 \tau}, \quad \varepsilon_{2} E_{2 n}= =\varepsilon_{1} E_{1 n}\). Из рис 3.9 следует, что

\[
\frac{\operatorname{tg} \alpha_{2}}{\operatorname{tg} \alpha_{1}}=\frac{E_{2 \tau} / E_{2 n}}{E_{1 \tau} / E_{1 n}} .
\]

Отсюда с учетом предыдущих условий получаем закон преломления линий \(\mathbf{E}\), а значит, и линий \(\mathbf{D}\) :


\begin{equation*}
\frac{\operatorname{tg} \alpha_{2}}{\operatorname{tg} \alpha_{1}}=\frac{\varepsilon_{2}}{\varepsilon_{1}} . \tag{3.25}
\end{equation*}


Это означает, что в диэлектрике с большим значением \(\varepsilon\) линии \(\mathbf{E}\) и \(\mathbf{D}\) будут составлять больший угол с нормалью к границе раздела (на рис. \(3.9 \varepsilon_{2}>\varepsilon_{1}\) ).

Пример. Изобразим графически поля \(\mathbf{E}\) и \(\mathbf{D}\) у границы раздела двух однородных диэлектриков 1 и 2 , считая, что \(\varepsilon_{2}>\varepsilon_{1}\) и что сто-

роннего заряда на этой поверхности нет.

Так как \(\varepsilon_{2}>\varepsilon_{1}\), то согласно (3.25) \(\alpha_{2}>\alpha_{1}\) (рис. 3.10).

Далее, из равенства тангенциальной составляющей вектора \(\mathbf{E}\) нетрудно сообразить с помощью рис. 3.9, что по мо-

\begin{figure}[h]
\begin{center}
  \includegraphics[alt={},max width=\textwidth]{3e870b25-6320-483d-9133-cdd5212c2707-084_273_483_145_677}
\captionsetup{labelformat=empty}
\caption{Рис. 3.10}
\end{center}
\end{figure}

Мы видим, что в нашем случае линии вектора \(\mathbf{E}\) испытывают преломление и, кроме того, терпят разрыв (из-за наличия связанных зарядов), линии же вектора \(\mathbf{D}\) испытывают только преломление, без разрыва (так как сторонних зарядов на границе нет).

Условие на границе проводник - диэлектрик. Если среда 1 - проводник, а среда 2 - диэлектрик (см. рис. 3.8), то из формулы (3.23) следует, что


\begin{equation*}
D_{n}=\sigma, \tag{3.26}
\end{equation*}


где \(\mathbf{n}\) - внешняя по отношению к проводнику нормаль (двойка в индексе здесь опущена, поскольку она не существенна в данном случае). Убедимся в справедливости формулы (3.26). В состоянии равновесия электрическое поле внутри проводника \(\mathbf{E}=0\), значит, и поляризованность \(\mathbf{P}=0\). А это, в свою очередь, означает согласно (3.17), что и вектор \(\mathbf{D}=0\) внутри проводника, т. е. в обозначениях формулы (3.23) \(\mathbf{D}_{1}=0\) и \(D_{1 n}=0\). Остается \(D_{2 n}=\sigma\).

Связанный заряд у поверхности проводника. Если к заряженному участку поверхности проводника прилегает однородный диэлектрик, то на границе этого диэлектрика с проводником выступают связанные заряды некоторой плотности \(\sigma^{\prime}\) (напомним, что для однородного диэлектрика объемная плотность связанных зарядов \(\rho^{\prime}=0\) ). Применим теперь теорему Гаусса к вектору \(\mathbf{E}\) - аналогично тому, как это было сделано при выводе

формулы (2.2). Имея в виду, что на границе раздела проводника с диэлектриком есть как сторонние, так и связанные заряды (б и \(\left.\sigma^{\prime}\right)\), придем к следующему выражению: \(E_{n}=\left(\sigma+\sigma^{\prime}\right) / \varepsilon_{0}\). С другой стороны, согласно (3.26) \(E_{n}=D_{n} / \varepsilon \varepsilon_{0}=\sigma / \varepsilon \varepsilon_{0}\). Из этих двух уравнений находим: \(\sigma / \varepsilon=\sigma+\sigma^{\prime}\), откуда


\begin{equation*}
\sigma^{\prime}=-\frac{\varepsilon-1}{\varepsilon} \sigma \tag{3.27}
\end{equation*}


Видно, что поверхностная плотность \(\sigma^{\prime}\) связанного заряда в диэлектрике однозначно связана с поверхностной плотностью σ стороннего заряда на проводнике, причем знаки этих зарядов противоположны.

\section*{§ 3.6. Поле в однородном диэлектрике}
Еще в § 2.1 было отмечено, что определение результирующего поля E в веществе сопряжено с большими трудностями, поскольку мы не знаем заранее, как распределяются индуцированные заряды в веществе. Ясно только, что распределение этих зарядов зависит от природы и формы вещества, а также от конфигурации внешнего поля \(\mathbf{E}_{0}\).

Поэтому в общем случае решение вопроса о результирующем поле \(\mathbf{E}\) в диэлектрике наталкивается на серьезные трудности: определение макрополя \(\mathbf{E}^{\prime}\) связанных зарядов в каждом конкретном случае представляет собой, вообще говоря, сложную самостоятельную задачу - универсальной формулы для нахождения \(\mathbf{E}^{\prime}\), к сожалению, нет.

Исключение составляет случай, когда все пространство, где имеется поле \(\mathbf{E}_{0}\), заполнено однородным изотропным диэлектриком. Рассмотрим этот случай более подробно. Представим себе заряженный проводник (или проводники) в вакууме обычно сторонние заряды располагаются на проводниках. Как мы уже знаем, в состоянии равновесия поле внутри проводника \(\mathbf{E}=0\), это при определенном и единственном распределении поверхностного заряда \(\sigma\). Пусть в окружающем проводник пространстве создано при этом поле \(\mathbf{E}_{0}\).

Теперь заполним все пространство, где есть поле, однородным диэлектриком. В таком диэлектрике вследствие его поля-

ризации появятся только поверхностные связанные заряды \(\sigma^{\prime}\) - на границе с проводником, причем заряды \(\sigma^{\prime}\) однозначно связаны со сторонними зарядами \(\sigma\) на поверхности проводника согласно (3.27).

Внутри же проводника поле по-прежнему будет отсутствовать \((\mathbf{E}=0)\). Это значит, что распределение поверхностных зарядов (сторонних \(\sigma+\) связанных \(\sigma^{\prime}\) ) на границе раздела проводника и диэлектрика будет подобно прежнему распределению сторонних зарядов \((\sigma)\), и конфигурация результирующего поля \(\mathbf{E}\) в диэлектрике останется той же, что и при отсутствии диэлектрика. Изменится только значение поля в каждой точке.

Согласно теореме Гаусса \(\sigma+\sigma^{\prime}=\varepsilon_{0} E_{n}\), где \(E_{n}=D_{n} / \varepsilon \varepsilon_{0}=\sigma / \varepsilon \varepsilon_{0}\), поэтому


\begin{equation*}
\sigma+\sigma^{\prime}=\sigma / \varepsilon . \tag{3.28}
\end{equation*}


Но если заряды, создающие электрическое поле, всюду на границе раздела уменьшились в \(\varepsilon\) раз, значит, и само поле \(\mathbf{E}\) тоже стало всюду меньше поля \(\mathbf{E}_{0}\) во столько же раз:


\begin{equation*}
\mathbf{E}=\mathbf{E}_{0} / \varepsilon . \tag{3.29}
\end{equation*}


Умножив обе части этого равенства на \(\varepsilon \varepsilon_{0}\), получим


\begin{equation*}
\mathbf{D}=\mathbf{D}_{0}, \tag{3.30}
\end{equation*}


поле вектора \(\mathbf{D}\) в рассматриваемом случае не меняется.\\
Формулы (3.29) и (3.30), оказывается, справедливы и в более общем случае, когда однородный диэлектрик целиком заполняет объем между эквипотенциальными поверхностями поля \(\mathbf{E}_{0}\) сторонних зарядов (или внешнего поля). И здесь внутри диэлектрика \(\mathbf{E}=\mathbf{E}_{0} / \varepsilon\) и \(\mathbf{D}=\mathbf{D}_{0}\). В указанных случаях напряженность \(\mathbf{E}^{\prime}\) поля связанных зарядов находится в простой связи с поляризованностью \(\mathbf{P}\) диэлектрика, а именно


\begin{equation*}
\mathbf{E}^{\prime}=-\mathbf{P} / \varepsilon_{0} . \tag{3.31}
\end{equation*}


Это соотношение легко получить из формулы \(\mathbf{E}=\mathbf{E}_{0}+\mathbf{E}^{\prime}\), если учесть, что \(\mathbf{E}_{0}=\varepsilon \mathbf{E}\) и \(\mathbf{P}=\varkappa \varepsilon_{0} \mathbf{E}\).

В других случаях, как уже было отмечено, дело обстоит значительно сложнее, и формулы (3.29)-(3.31) становятся не справедливыми.

Следствия. Итак, если однородный диэлектрик заполняет все пространство, занимаемое полем, то напряженность \(\mathbf{E}\) поля будет в \(\varepsilon\) раз меньше напряженности \(\mathbf{E}_{0}\) поля тех же сторонних зарядов, но при отсутствии диэлектрика. Отсюда следует, что потенциал \(\varphi\) во всех точках также уменьшается в \(\varepsilon\) раз:


\begin{equation*}
\varphi=\varphi_{0} / \varepsilon, \tag{3.32}
\end{equation*}


где \(\varphi_{0}\) - потенциал поля в отсутствие диэлектрика. Это же относится и к разности потенциалов


\begin{equation*}
U=U_{0} / \varepsilon, \tag{3.33}
\end{equation*}


где \(U_{0}\) - разность потенциалов в вакууме, без диэлектрика.\\
В простейшем случае, когда однородный диэлектрик заполняет все пространство между обкладками конденсатора, разность потенциалов \(U\) между его обкладками будет в \(\varepsilon\) раз меньше, чем при отсутствии диэлектрика (разумеется, при том же значении заряда \(q\) на обкладках). А раз так, то емкость конденсатора ( \(C=q / U\) ) при заполнении его диэлектриком увеличится в \(\varepsilon\) раз:


\begin{equation*}
C^{\prime}=\varepsilon C, \tag{3.34}
\end{equation*}


где \(C\) - емкость конденсатора без диэлектрика. Следует обратить внимание на то, что эта формула справедлива при заполнении всего пространства между обкладками конденсатора и без учета краевых эффектов.

С помощью некоторых диэлектриков (керамик) удалось создать конденсаторы емкостью в несколько фарад (!).

\section*{Задачи}
3.1. Поляризованность диэлектрика и связанный заряд. Точечный сторонний заряд \(q\) находится в центре сферического слоя неоднородного изотропного диэлектрика, проницаемость которого изменяется только в радиальном направлении по закону \(\varepsilon=\alpha / r\), где \(\alpha\) - постоянная, \(r\) - расстояние от центра системы. Найти объемную плотность \(\rho^{\prime}\) связанных зарядов как функцию \(r\) внутри слоя.\\
Решение. Воспользуемся уравнением (3.6), взяв в качестве замкнутой поверхности сферу радиусом \(r\), центр которой совпадает с

центром системы. Тогда

\[
4 \pi r^{2} \cdot P_{r}=-q^{\prime}(r),
\]

где \(q^{\prime}(r)\) - связанный заряд внутри этой сферы. Запишем дифференциал этого выражения:


\begin{equation*}
4 \pi \mathrm{~d}\left(r^{2} P_{r}\right)=-\mathrm{d} q^{\prime} . \tag{1}
\end{equation*}


Здесь \(\mathrm{d} q^{\prime}\) - связанный заряд в тонком слое между сферами радиусов \(r\) и \(r+\mathrm{d} r\). Имея в виду, что \(\mathrm{d} q^{\prime}=r^{\prime} 4 \pi r^{2} \mathrm{~d} r\), преобразуем (1) к виду

\[
r^{2} \mathrm{~d} P_{r}+2 r P_{r} \mathrm{~d} r=-\rho^{\prime} r^{2} \mathrm{~d} r
\]

откуда


\begin{equation*}
\rho^{\prime}=-\left(\frac{\mathrm{d} P_{r}}{\mathrm{~d} r}+\frac{2}{r} P_{r}\right) . \tag{2}
\end{equation*}


В нашем случае

\[
P_{r}=\varkappa \varepsilon_{0} E_{r}=\frac{\varepsilon-1}{\varepsilon} D_{r}=\frac{\varepsilon-1}{\varepsilon} \frac{q}{4 \pi r^{2}},
\]

и выражение (2) после соответствующих преобразований будет иметь вид

\[
\rho^{\prime}=\frac{1}{4 \pi \alpha} \frac{q}{r^{2}} .
\]

Это и есть искомый результат.\\
3.2. Теорема Гаусса для вектора D. Бесконечно большая пластина из однородного диэлектрика с проницаемостью \(\varepsilon\) заряжена равномерно сторонним зарядом с объемной плотностью \(\rho>0\). Толщина пластины \(2 a\). Найти:

\begin{enumerate}
  \item модуль вектора \(\mathbf{E}\) и потенциал φ как функции расстояния \(l\) от середины пластины (потенциал в середине пластины положить равным нулю); взяв координатную ось \(X\) перпендикулярно пластине, изобразить примерные графики зависимостей проекции \(E_{x}(x)\) вектора \(\mathbf{E}\) и потенциала \(\varphi(x)\);
  \item поверхностную и объемную плотности связанного заряда.
\end{enumerate}

Решение. 1. Из соображений симметрии ясно, что в середине пластины \(E=0\), а во всех остальных точках вектор \(\mathbf{E}\) перпендикулярен поверхности пластины. Для определения \(E\) воспользуемся те-

оремой Гаусса для вектора D (ибо нам известно распределение только сторонних зарядов). Возьмем в качестве замкнутой поверхности прямой цилиндр высотой \(l\), один торец которого совпадает со средней плоскостью пластины. Пусть площадь сечения этого цилиндра равна \(S\), тогда

\[
D S=\rho S l, \quad D=\rho l, \quad E=\rho l / \varepsilon \varepsilon_{0}, \quad(l \leqslant a),
\]

\begin{figure}[h]
\begin{center}
  \includegraphics[alt={},max width=\textwidth]{3e870b25-6320-483d-9133-cdd5212c2707-089_360_370_439_83}
\captionsetup{labelformat=empty}
\caption{Рис. 3.11}
\end{center}
\end{figure}

\[
D S=\rho S a, \quad D=\rho a, \quad E=\rho a / \varepsilon_{0}, \quad(l \geqslant a) .
\]

Графики функций \(E_{x}(x)\) и \(\varphi(x)\) показаны на рис. 3.11. Полезно убедиться, что график \(E_{x}(x)\) соответствует производной \(-\partial \varphi / \partial x\).\\
2. Согласно (3.13) поверхностная плотность связанного заряда

\[
\sigma^{\prime}=P_{n}=\varkappa \varepsilon_{0} E_{n}=\frac{\varepsilon-1}{\varepsilon} \rho a>0 .
\]

Этот результат справедлив для обеих поверхностей пластины. Таким образом, если сторонний заряд \(\rho>0\), то на обеих поверхностях пластины выступит также положительный связанный заряд.\\
Для определения объемной плотности связанного заряда воспользуемся уравнением (3.9), которое в нашем случае будет иметь наиболее простой вид:

\[
\rho^{\prime}=-\frac{\partial P_{x}}{\partial x}=-\frac{\partial}{\partial x}\left(\frac{\varepsilon-1}{\varepsilon} \rho x\right)=-\frac{\varepsilon-1}{\varepsilon} \rho .
\]

Отсюда видно, что связанный заряд распределен по объему равномерно и имеет знак, противоположный стороннему заряду.\\
3.3. Однородный диэлектрик имеет вид сферического слоя, внутренний и внешний радиусы которого равны \(a\) и \(b\). Изобразить примерные графики напряженности \(E\) и потенциала φ электрического поля как функции расстояния \(r\) от центра системы, если данному диэлектрику сообщили положительный сторонний заряд, распределенный равномерно: 1) по внутренней поверхности слоя; 2) по объему слоя.

Решение. 1. Воспользуемся теоремой Гаусса для вектора D, взяв в качестве замкнутой поверхности сферу радиусом \(r\) :

\[
4 \pi r^{2} D=q,
\]

где \(q\) - сторонний заряд внутри этой сферы. Отсюда следует, что

\[
D(r<a)=0, \quad D(r>a)=q / 4 \pi r^{2} .
\]

Искомая напряженность

\[
E(r<a)=0, \quad E(r>a)=D / \varepsilon \varepsilon_{0} .
\]

График зависимости \(E(r)\) показан на рис. 3.12, \(a\). На этом же рисунке изображен и график зависимости φ от \(r\). График \(\varphi(r)\) должен иметь такой вид, чтобы производная \(\partial \varphi / \partial r\), взятая с обратным знаком, соответствовала графику функции \(E(r)\).\\
При этом должно быть учте-

\begin{figure}[h]
\begin{center}
  \includegraphics[alt={},max width=\textwidth]{3e870b25-6320-483d-9133-cdd5212c2707-090_338_567_398_595}
\captionsetup{labelformat=empty}
\caption{Рис. 3.12}
\end{center}
\end{figure}

но и условие нормировки: \(\varphi \rightarrow 0\) при \(r \rightarrow \infty\).\\
Следует обратить внимание на то, что график \(\varphi(r)\) является непрерывным. В местах конечных разрывов функции \(E(r)\) график \(\varphi(r)\) испытывает лишь изломы.\\
2. В данном случае согласно теореме Гаусса

\[
4 \pi r^{2} D=4 / 3 \pi\left(r^{3}-a^{3}\right) \rho,
\]

где \(\rho\) - объемная плотность стороннего заряда. Отсюда

\[
E=\frac{D}{\varepsilon \varepsilon_{0}}=\frac{\rho}{3 \varepsilon \varepsilon_{0}} \frac{r^{3}-a^{3}}{r^{2}} .
\]

Графики зависимостей \(E(r)\) и \(\varphi(r)\) показаны на рис. 3.12, б.\\
3.4. Сторонние заряды равномерно распределены с объемной плотностью \(\rho>0\) по шару радиусом \(a\) из однородного диэлектрика с проницаемостью ε. Найти: 1) модуль вектора \(\mathbf{E}\) как функцию расстояния \(r\) от центра шара, изобразить примерные графики функции \(E(r)\) и потенциала \(\varphi(r) ; 2)\) поверхностную и объемную плотности связанных зарядов.\\
Решение. 1. Для определения \(E\) воспользуемся теоремой Гаусса для вектора \(\mathbf{D}\), поскольку задано распределение лишь сторонних зарядов:

\[
r \leqslant a, \quad 4 \pi r^{2} D=\frac{4}{3} \pi r^{3} \rho, \quad D=\frac{\rho}{3} r, \quad E=\frac{D}{\varepsilon \varepsilon_{0}}=\frac{\rho}{3 \varepsilon \varepsilon_{0}} r
\]

\[
r \geqslant a, \quad 4 \pi r^{2} D=\frac{4}{3} \pi a^{3} \rho, \quad D=\frac{\rho a^{3}}{3 r^{2}}, \quad E=\frac{D}{\varepsilon_{0}}=\frac{\rho}{3 \varepsilon_{0}} \frac{1}{r^{2}} .
\]

\begin{figure}[h]
\begin{center}
  \includegraphics[alt={},max width=\textwidth]{3e870b25-6320-483d-9133-cdd5212c2707-091_293_353_267_141}
\captionsetup{labelformat=empty}
\caption{Рис. 3.13}
\end{center}
\end{figure}

Графики функций \(E(r)\) и \(\varphi(r)\) показаны на рис. 3.13.\\
2. Поверхностная плотность связанного заряда

\[
\sigma^{\prime}=P_{n}=\frac{\varepsilon-1}{\varepsilon} \frac{\rho a}{3} .
\]

Для нахождения объемной плотности связанных зарядов достаточно повторить рассуждения, которые привели нас к формуле (3.11), и мы получим


\begin{equation*}
\rho^{\prime}=-\frac{\varepsilon-1}{\varepsilon} \rho . \tag{1}
\end{equation*}


Этот результат можно получить и иначе - с помощью уравнения (3.9). А именно, так как \(\mathbf{P}=\varkappa \varepsilon_{0} \mathbf{E}\) и \(\varkappa\) не зависит от координат (внутри шара), то

\[
\rho^{\prime}=-\nabla \cdot \mathbf{P}=-\varkappa \varepsilon_{0} \nabla \cdot \mathbf{E} \text {, }
\]

где \(\varepsilon_{0} \nabla \cdot \mathbf{E}=\rho+\rho^{\prime}\). Поэтому \(\rho^{\prime}=-\varkappa\left(\rho+\rho^{\prime}\right)\), откуда и следует (1).\\
3.5. Емкость проводника. Найти емкость шарового проводника радиусом \(a\), окруженного примыкающим к нему слоем однородного диэлектрика с наружным радиусом \(b\) и проницаемостью \(\varepsilon\). Изобразить примерные графики зависимостей поля \(E(r)\) и потенциала \(\varphi(r)\), где \(r\) - расстояние от центра шара, если проводник заряжен положительно.

Решение. По определению, емкость \(C=q / \varphi\). Найдем потенциал \(\varphi\) проводника, мысленно сообщив ему заряд \(q\) :

\[
\varphi=\int_{a}^{\infty} E_{r} \mathrm{~d} r=\frac{1}{4 \pi \varepsilon_{0}} \int_{a}^{b} \frac{q}{\varepsilon r^{2}} \mathrm{~d} r+\frac{1}{4 \pi \varepsilon_{0}} \int_{b}^{\infty} \frac{q}{r^{2}} \mathrm{~d} r .
\]

После интегрирования этого выражения получим:

\[
\varphi=\frac{q}{4 \pi \varepsilon_{0} \varepsilon}\left(\frac{1}{a}+\frac{\varepsilon-1}{b}\right), \quad C=\frac{4 \pi \varepsilon_{0} \varepsilon a}{1+(\varepsilon-1) a / b} .
\]

Графики зависимостей \(E(r)\) и \(\varphi(r)\) показаны на рис. 3.14.\\
3.6. Емкость конденсатора. Сферический конденсатор с радиусами обкладок \(a\) и \(b\), где \(a<b\), заполнен изотропным, но неоднородным диэлектриком, проницаемость которого зависит от расстояния \(r\) до центра системы как \(\varepsilon=\alpha / r, \alpha\) - постоянная. Найти емкость такого конденсатора.

\begin{figure}[h]
\begin{center}
  \includegraphics[alt={},max width=\textwidth]{3e870b25-6320-483d-9133-cdd5212c2707-092_279_350_155_806}
\captionsetup{labelformat=empty}
\caption{Рис. 3.14}
\end{center}
\end{figure}

Решение. Согласно определению емкости конденсатора ( \(C=q / U\) ) задача сводится к нахождению разности потенциалов \(U\) при заданном заряде \(q\) :


\begin{equation*}
U=\int_{a}^{b} E \mathrm{~d} r \tag{1}
\end{equation*}


где предполагается, что внутренняя обкладка имеет заряд \(q>0\). Определим \(E\) с помощью теоремы Гаусса для вектора D:

\[
4 \pi r^{2} D=q, \quad E=\frac{D}{\varepsilon \varepsilon_{0}}=\frac{1}{4 \pi \varepsilon_{0}} \frac{q}{\varepsilon r^{2}}=\frac{1}{4 \pi \varepsilon_{0}} \frac{q}{\alpha r} .
\]

После подстановки последнего выражения в (1) и соответствующего интегрирования найдем:

\[
U=\frac{q}{4 \pi \varepsilon_{0} \alpha} \ln \frac{b}{a}, \quad C=\frac{4 \pi \varepsilon_{0} \alpha}{\ln (b / a)}
\]

3.7. Теорема Гаусса и принцип суперпозиции. Имеется диэлектрический шар, который сохраняет состояние поляризации после выключения внешнего электрического поля. Если шар поляризован однородно, то напряженность поля внутри него \(\mathbf{E}^{\prime}=-\mathbf{P} / 3 \varepsilon_{0}\), где \(\mathbf{P}\) - поляризованность.

\begin{enumerate}
  \item Получить эту формулу, считая что так поляризованный шар есть результат малого сдвига всех положительных зарядов диэлектрика относительно всех отрицательных зарядов.
  \item Воспользовавшись этой формулой, найти напряженность \(\mathbf{E}_{0}\) поля в сферической полости внутри безграничного статически поляризованного (P) диэлектрика, если вдали от полости напряженность в диэлектрике равна Е.
\end{enumerate}

Решение. 1. Представим такой шар в виде двух шаров одинакового радиуса, имеющих равномерно распределенные заряды с плот-

\begin{figure}[h]
\begin{center}
  \includegraphics[alt={},max width=\textwidth]{3e870b25-6320-483d-9133-cdd5212c2707-093_305_281_155_159}
\captionsetup{labelformat=empty}
\caption{Рис. 3.15}
\end{center}
\end{figure}

ностями \(\rho\) и \(-\rho\). Пусть в результате малого сдвига центры шаров сместились относительно друг друга на расстояние 1 (рис. 3.15). Тогда в произвольной точке \(A\) внутри шара

\[
\mathbf{E}^{\prime}=\mathbf{E}_{+}^{\prime}+\mathbf{E}_{-}^{\prime}=\frac{\rho}{3 \varepsilon_{0}}\left(\mathbf{r}_{+}-\mathbf{r}_{-}\right)=-\frac{\rho \mathbf{l}}{3 \varepsilon_{0}},
\]

где использовано, что напряженность поля внутри равномерно заряженного шара \(\mathbf{E}= =\rho \mathbf{r} / 3 \varepsilon_{0}\), это непосредственно следует из теоремы Гаусса. Остается учесть, что, согласно (3.4), \(\rho \mathbf{l}=\mathbf{P}\).\\
2. Создание сферической полости в диэлектрике эквивалентно удалению шарика из поляризованного вещества. Поэтому по принципу суперпозиции поле \(\mathbf{E}\) внутри диэлектрика может быть представлено как сумма \(\mathbf{E}=\mathbf{E}^{\prime}+\mathbf{E}_{0}\). Отсюда

\[
\mathbf{E}_{0}=\mathbf{E}-\mathbf{E}^{\prime}=\mathbf{E}+\mathbf{P} / 3 \varepsilon_{0} .
\]

Т. е. поле в сферической полости больше поля \(E\) в диэлектрике на величину \(P / 3 \varepsilon_{0}\).\\
3.8. Граничные условия. Вблизи точки \(A\) (рис. 3.16) границы раздела диэлектрик - вакуум напряженность электрического поля в вакууме равна \(E_{0}\), причем вектор \(\mathbf{E}_{0}\) составляет

\begin{figure}[h]
\begin{center}
  \includegraphics[alt={},max width=\textwidth]{3e870b25-6320-483d-9133-cdd5212c2707-093_238_324_1039_129}
\captionsetup{labelformat=empty}
\caption{Рис. 3.16}
\end{center}
\end{figure}

угол \(\alpha_{0}\) с нормалью к поверхности раздела в данной точке. Проницаемость диэлектрика \(\varepsilon\). Найти отношение \(E / E_{0}\), где \(E\) - напряженность поля внутри диэлектрика вблизи точки \(A\).

Решение. Напряженность поля внутри диэлектрика


\begin{equation*}
E=\sqrt{E_{\tau}^{2}+E_{n}^{2}} . \tag{1}
\end{equation*}


Воспользовавшись условиями (3.22) и (3.24) на границе раздела диэлектриков, найдем:

\[
E_{\tau}=E_{0} \sin \alpha_{0}, \quad E_{n}=D_{n} / \varepsilon \varepsilon_{0}=E_{0 n} / \varepsilon=\left(E_{0} / \varepsilon\right) \cos \alpha_{0},
\]

где \(E_{0 n}\) - нормальная составляющая вектора \(\mathbf{E}_{0}\) в вакууме. Подставив эти выражения в (1), получим

\[
\frac{E}{E_{0}}=\sqrt{\sin ^{2} \alpha_{0}+\frac{\cos ^{2} \alpha_{0}}{\varepsilon^{2}}}<1
\]

т. е. \(E<E_{0}\).\\
3.9. Точечный заряд \(q\) находится в вакууме на расстоянии \(l\) от плоской поверхности однородного диэлектрика, заполняющего все полупространство. Проницаемость диэлектрика \(\varepsilon\). Найти: 1) поверхностную плотность связанных зарядов как функцию расстояния \(r\) от точечного заряда \(q\), исследовать полученный результат; 2) суммарный

\begin{figure}[h]
\begin{center}
  \includegraphics[alt={},max width=\textwidth]{3e870b25-6320-483d-9133-cdd5212c2707-094_309_408_141_752}
\captionsetup{labelformat=empty}
\caption{Рис. 3.17}
\end{center}
\end{figure}

связанный заряд на поверхности диэлектрика.

Решение. 1. Воспользуемся непрерывностью нормальной составляющей вектора \(\mathbf{D}\) на границе диэлектрика (рис. 3.17):

\[
D_{2 n}=D_{1 n}, \quad E_{2 n}=\varepsilon E_{1 n}
\]

или

\[
-\frac{1}{4 \pi \varepsilon_{0}} \frac{q}{r^{2}} \cos \vartheta+\frac{\sigma^{\prime}}{2 \varepsilon_{0}}=\varepsilon\left(-\frac{1}{4 \pi \varepsilon_{0}} \frac{q}{r^{2}} \cos \vartheta-\frac{\sigma^{\prime}}{2 \varepsilon_{0}}\right),
\]

где слагаемое \(\sigma^{\prime} / 2 \varepsilon_{0}\) - это составляющая напряженности поля, создаваемая вблизи рассматриваемого участка плоскости, на котором поверхностная плотность заряда равна \(\sigma^{\prime}\). Из последнего уравнения следует, что


\begin{equation*}
\sigma^{\prime}=-\frac{\varepsilon-1}{\varepsilon+1} \frac{q l}{2 \pi r^{3}} . \tag{1}
\end{equation*}


Здесь учтено, что \(\cos \vartheta=l / r\). При \(l \rightarrow 0\) величина \(\sigma^{\prime} \rightarrow 0\), т. е. если заряд \(q\) находится на самой границе раздела, то поверхностный заряд на плоскости отсутствует.\\
2. Рассмотрим тонкое кольцо на границе раздела с центром в точке \(O\) (рис. 3.17). Пусть внутренний и внешний радиусы этого кольца \(r^{\prime}\) и \(r^{\prime}+\mathrm{d} r^{\prime}\). Поверхностный связанный заряд в пределах данного кольца \(\mathrm{d} q^{\prime}=\sigma^{\prime} \cdot 2 \pi r^{\prime} \mathrm{d} r^{\prime}\). Из рисунка видно, что \(r^{2}=l^{2}+r^{\prime 2}\), откуда \(r \mathrm{~d} r=r^{\prime} \mathrm{d} r^{\prime}\), и выражение для \(\mathrm{d} q^{\prime}\) с учетом (1) приобретает вид

\[
\mathrm{d} q^{\prime}=-\frac{\varepsilon-1}{\varepsilon+1} q l \frac{\mathrm{~d} r}{r^{2}} .
\]

Проинтегрировав это уравнение по \(r\) от \(l\) до \(\infty\), получим

\[
q^{\prime}=-\frac{\varepsilon-1}{\varepsilon+1} q
\]

3.10. Точечный заряд \(q\) находится на плоскости, отделяющей вакуум от безграничного однородного диэлектрика с проницаемостью \(\varepsilon\). Найти модуль векторов \(\mathbf{D}\) и \(\mathbf{E}\) во всем пространстве.

Решение. В данном случае из условия непрерывности нормальной составляющей вектора \(\mathbf{D}\) следует, что \(E_{2 n}=\varepsilon E_{1 n}\). Вклад в нормальную составляющую вектора \(\mathbf{E}\) будет давать только поверхностный заряд \(\sigma^{\prime}\) вблизи интересующей нас точки, поэтому предыдущее равенство можно переписать так:

\[
\sigma^{\prime} / 2 \varepsilon_{0}=\varepsilon\left(-\sigma^{\prime} / 2 \varepsilon_{0}\right) .
\]

Отсюда сразу видно, что \(\sigma^{\prime}=0\).\\
Итак, в данном случае поверхностный связанный заряд отсутствует (кроме точек, непосредственно прилегающих к точечному стороннему заряду \(q\) ). Значит, электрическое поле в окружающем пространстве - это поле точечного заряда \(q+q^{\prime}\), и \(E\) зависит только от расстояния \(r\) до этого заряда. Но заряд \(q^{\prime}\) нам не известен, поэтому воспользуемся теоремой Гаусса для вектора \(\mathbf{D}\). Взяв в качестве замкнутой поверхности сферу радиусом \(r\) с центром в точке нахождения заряда \(q\), запишем

\[
2 \pi r^{2} D_{0}+2 \pi r^{2} D=q,
\]

где \(D_{0}\) и \(D\) - модули вектора \(\mathbf{D}\) соответственно в вакууме и диэлектрике на расстоянии \(r\) от заряда \(q\).

Кроме того, из условия непрерывности тангенциальной составляющей вектора \(\mathbf{E}\) следует, что

\[
D=\varepsilon D_{0} .
\]

Из последних двух условий находим, что

\[
D_{0}=\frac{q}{2 \pi(1+\varepsilon) r^{2}}, \quad D=\frac{\varepsilon q}{2 \pi(1+\varepsilon) r^{2}},
\]

и напряженность электрического поля во всем пространстве

\[
E=\frac{D_{0}}{\varepsilon_{0}}=\frac{q}{2 \pi(1+\varepsilon) \varepsilon_{0} r^{2}}
\]

Видно, что при \(\varepsilon=1\) эти формулы переходят в известные нам выражения для \(D\) и \(E\) точечного заряда в вакууме.

Полученные результаты графически представлены на рис. 3.18. Следует обратить внимание на то, что поле \(\mathbf{D}\) в данном случае определяется не только сторонними зарядами (иначе оно имело бы вид поля точечного заряда).

\begin{figure}[h]
\begin{center}
  \includegraphics[alt={},max width=\textwidth]{3e870b25-6320-483d-9133-cdd5212c2707-096_312_536_350_383}
\captionsetup{labelformat=empty}
\caption{Рис. 3.18}
\end{center}
\end{figure}

\section*{Глава 4}
\section*{Энергия электрического поля}
\section*{§ 4.1. Электрическая энергия системы зарядов}
Энергетический подход к взаимодействию. Такой подход к взаимодействию электрических зарядов является, как мы увидим, весьма плодотворным по своим практическим применениям, а кроме того, открывает возможность по-иному взглянуть и на само электрическое поле как физическую реальность.

Прежде всего мы выясним, как можно прийти к понятию о энергии взаимодействия системы зарядов.

\begin{enumerate}
  \item Сначала рассмотрим систему, состоящую из двух точечных зарядов 1 и 2. Найдем алгебраическую сумму элементарных работ сил \(\mathbf{F}_{1}\) и \(\mathbf{F}_{2}\), с которыми эти заряды взаимодействуют. Пусть в некоторой \(K\)-системе отсчета за время \(\mathrm{d} t\) заряды совершили перемещения \(\mathrm{dl}_{1}\) и \(\mathrm{dl}_{2}\). Тогда работа этих сил
\end{enumerate}

\[
\delta A_{1,2}=\mathbf{F}_{1} \mathrm{dl}_{1}+\mathrm{F}_{2} \mathrm{dl}_{2} .
\]

Учитывая, что \(\mathbf{F}_{2}=-\mathbf{F}_{1}\) (по третьему закону Ньютона), перепишем предыдущее выражение:

\[
\delta A_{1,2}=\mathbf{F}_{1}\left(\mathrm{dl}_{1}-\mathrm{dl}_{2}\right) .
\]

Величина в скобках - это перемещение заряда 1 относительно заряда 2 . Точнее, это есть перемещение заряда 1 в \(K^{\prime}\)-системе отсчета, жестко связанной с зарядом 2 и перемещающейся вместе с ним поступательно по отношению к исходной \(K\)-системе. Действительно, перемещение \(\mathrm{dl}_{1}\) заряда 1 в \(K\)-системе может быть представлено как перемещение \(\mathrm{dl}_{2} K^{\prime}\)-системы плюс перемещение \(\mathrm{dl}_{1}^{\prime}\) заряда 1 относительно этой \(K^{\prime}\)-системы: \(\mathrm{dl}_{1}=\mathrm{dl}_{2}+\mathrm{dl}_{1}^{\prime}\). Отсюда \(\mathrm{dl}_{1}-\mathrm{dl}_{2}=\mathrm{dl}_{1}^{\prime}\) и

\[
\delta A_{1,2}=\mathbf{F}_{1} \mathrm{dl}_{1}^{\prime} .
\]

Итак, оказывается, что сумма элементарных работ в произвольной \(K\)-системе отсчета всегда равна элементарной работе, которую совершает сила, действующая на один заряд, в системе отсчета, где другой заряд покоится. Иначе говоря, работа \(\delta A_{1,2}\) не зависит от выбора исходной \(K\)-системы отсчета.

Сила \(\mathbf{F}_{1}\), действующая на заряд 1 со стороны заряда 2 , консервативная (как сила центральная). Поэтому работа данной силы на перемещении \(\mathrm{dl}_{1}^{\prime}\) может быть представлена как убыль потенциальной энергии заряда 1 в поле заряда 2 или как убыль потенциальной энергии взаимодействия этой пары зарядов:

\[
\delta A_{1,2}=-\mathrm{d} W_{12},
\]

где \(W_{12}\) - величина, зависящая только от расстояния между данными зарядами.\\
2. Теперь перейдем к системе из mpex точечных зарядов (полученный для этого случая результат легко будет обобщить на систему из произвольного числа зарядов). Работа, которую совершают все силы взаимодействия при элементарных перемещениях всех зарядов, может быть представлена как сумма работ всех трех пар взаимодействий, т. е. \(\delta A=\delta A_{1,2}+\delta A_{1,3}+\delta A_{2,3}\). Но для каждой пары взаимодействий, как только что было показано, \(\delta A_{i, k}=-\mathrm{d} W_{i k}\), поэтому

\[
\delta A=-\mathrm{d}\left(W_{12}+W_{13}+W_{23}\right)=-\mathrm{d} W,
\]

где \(W\) - энергия взаимодействия данной системы зарядов,

\[
W=W_{12}+W_{13}+W_{23} .
\]

Каждое слагаемое этой суммы зависит от расстояния между соответствующими зарядами, поэтому энергия \(W\) данной системы зарядов есть функция ее конфигурации.

Подобные рассуждения, очевидно, справедливы и для системы из любого числа зарядов. Значит, можно утверждать, что каждой конфигурации произвольной системы зарядов присуще свое значение энергии \(W\), и работа всех сил взаимодействия при изменении этой конфигурации равна убыли \(W\) :


\begin{equation*}
\delta A=-\mathrm{d} W . \tag{4.1}
\end{equation*}


Энергия взаимодействия. Найдем выражение для энергии \(W\). Сначала рассмотрим опять систему из трех точечных зарядов, для которой мы показали, что \(W=W_{12}+W_{13}+W_{23}\). Преобразуем эту сумму следующим образом. Представим каждое слагаемое \(W_{i k}\) в симметричном виде: \(W_{i k}=\left(W_{i k}+W_{k i}\right) / 2\), поскольку \(W_{i k}=W_{k i}\). Тогда

\[
W=\left(W_{12}+W_{21}+W_{13}+W_{31}+W_{23}+W_{32}\right) / 2 .
\]

Сгруппируем члены с одинаковыми первыми индексами:

\[
W=\left[\left(W_{12}+W_{13}\right)+\left(W_{21}+W_{23}\right)+\left(W_{31}+W_{32}\right)\right] / 2 .
\]

Каждая сумма в круглых скобках - это энергия \(W_{i}\) взаимодействия \(i\)-го заряда с остальными зарядами. Поэтому последнее выражение можно переписать так:

\[
W=1 / 2\left(W_{1}+W_{2}+W_{3}\right)=1 / 2 \sum_{i=1}^{3} W_{i}
\]

Обобщение полученного выражения на систему из произвольного числа зарядов очевидно, ибо ясно, что проведенные рассуждения совершенно не зависят от числа зарядов, составляющих систему. Итак, энергия взаимодействия системы точечных зарядов


\begin{equation*}
W=1 / 2 \sum W_{i} . \tag{4.2}
\end{equation*}


Имея в виду, что \(W_{i}=q_{i} \varphi_{i}\), где \(q_{i}-i\)-й заряд системы; \(\varphi_{i}-\) потенциал, создаваемый в месте нахождения \(i\)-го заряда всеми остальньлми зарядами системы, получим окончательное выражение для энергии взаимодействия системы точечных зарядов:


\begin{equation*}
W=1 / 2 \sum q_{i} \varphi_{i} \tag{4.3}
\end{equation*}


Пример. Четыре одинаковых точечных заряда \(q\) находятся в вершинах тетраэдра с ребром \(a\) (рис. 4.1). Найдем энергию взаимодействия зарядов этой системы.\\
Энергия взаимодействия каждой пары зарядов здесь одинакова и равна \(W_{1}=q^{2} / 4 \pi \varepsilon_{0} a\). Всего таких взаимодействующих пар, как видно из рисунка, шесть, поэтому энергия взаимо-

действия всех точечных зарядов данной системы

\[
W=6 W_{1}=6 q^{2} / 4 \pi \varepsilon_{0} a .
\]

Иной подход к решению этого вопроса основан на использовании формулы (4.3). Потенциал φ в месте нахождения одного из

\begin{figure}[h]
\begin{center}
  \includegraphics[alt={},max width=\textwidth]{3e870b25-6320-483d-9133-cdd5212c2707-100_237_229_141_919}
\captionsetup{labelformat=empty}
\caption{Рис. 4.1}
\end{center}
\end{figure}

зарядов, обусловленный полем всех остальных зарядов, равен \(\varphi=3 q / 4 \pi \varepsilon_{0} a\). Поэтому

\[
W=1 / 2 \sum_{i=1}^{4} q_{i} \varphi_{i}=1 / 24 q \varphi=\frac{1}{4 \pi \varepsilon_{0}} \frac{6 q^{2}}{a} .
\]

Полная энергия взаимодействия. Если заряды распределены непрерывно, то, разлагая систему зарядов на совокупность элементарных зарядов \(\mathrm{d} q=\rho \mathrm{d} V\) и переходя от суммирования в \((4.3)\) к интегрированию, получаем


\begin{equation*}
W=1 / 2 \int \rho \varphi \mathrm{~d} V, \tag{4.4}
\end{equation*}


где φ - потенциал, создаваемый всеми зарядами системы в элементе объемом \(\mathrm{d} V\). Аналогичное выражение можно записать для распределения зарядов, например, по поверхности; для этого достаточно в формуле (4.4) заменить \(\rho\) на \(\sigma\) и \(\mathrm{d} V\) на \(\mathrm{d} S\).

Можно ошибочно подумать (и это часто приводит к недоразумениям), что выражение (4.4) - это только видоизмененное выражение (4.3), соответствующее замене представления о точечных зарядах представлением о непрерывно распределенном заряде. В действительности это не так - оба выражения отличаются по своему содержанию. Происхождение этого различия - в разном смысле потенциала \(\varphi\), входящего в оба выражения, что лучше всего пояснить на следующем примере.

Пусть система состоит из двух шариков, имеющих заряды \(q_{1}\) и \(q_{2}\). Расстояние между шариками значительно больше их размеров, поэтому заряды \(q_{1}\) и \(q_{2}\) можно считать точечными. Найдем энергию \(W\) данной системы с помощью обеих формул.

Согласно формуле (4.3)

\[
W=\left(q_{1} \varphi_{1}+q_{2} \varphi_{2}\right) / 2=q_{1} \varphi_{1}=q_{2} \varphi_{2},
\]

где \(\varphi_{1}\) - потенциал, создаваемый зарядом \(q_{2}\) в месте нахождения заряда \(q_{1}\); аналогичный смысл имеет и потенциал \(\varphi_{2}\).

Согласно же формуле (4.4) мы должны разбить заряд каждого шарика на бесконечно малые элементы \(\rho \mathrm{d} V\) и каждый из них умножить на потенциал \(\varphi\), создаваемый не только зарядами другого шарика, но и элементами заряда этого шарика. Ясно, что результат будет совершенно другим, а именно:


\begin{equation*}
W=W_{1}+W_{2}+W_{12}, \tag{4.5}
\end{equation*}


где \(W_{1}\) - энергия взаимодействия друг с другом элементов заряда первого шарика; \(W_{2}\) - то же, но для второго шарика; \(W_{12}\) - энергия взаимодействия элементов заряда первого шарика с элементами заряда второго шарика. Энергии \(W_{1}\) и \(W_{2}\) называют собственньми энергиями зарядов \(q_{1}\) и \(q_{2}\), а \(W_{12}\) энергией взаимодействия заряда \(q_{1}\) с зарядом \(q_{2}\).

Таким образом, мы видим, что расчет энергии \(W\) по формуле (4.3) дает только \(W_{12}\), а расчет по формуле (4.4) - полную энергию взаимодействия: кроме \(W_{12}\) еще и собственные энергии \(W_{1}\) и \(W_{2}\). Игнорирование этого обстоятельства зачастую является источником грубых ошибок.

К данному вопросу мы еще вернемся в § 4.4, а сейчас получим с помощью формулы (4.4) несколько важных результатов.

\section*{§ 4.2. Энергия заряженных проводника и конденсатора}
Энергия уединенного проводника. Пусть проводник имеет заряд \(q\) и потенциал \(\varphi\). Поскольку значение \(\varphi\) во всех точках, где имеется заряд, одинаково, \(\varphi\) можно вынести из-под знака интеграла в формуле (4.4). Тогда оставшийся интеграл есть не что иное, как заряд \(q\) на проводнике, и


\begin{equation*}
W=\frac{q \varphi}{2}=\frac{C \varphi^{2}}{2}=\frac{q^{2}}{2 C} . \tag{4.6}
\end{equation*}


Эти три выражения написаны с учетом того, что \(C=q / \varphi\).\\
Энергия конденсатора. Пусть \(q\) и \(\varphi_{+}\)- заряд и потенциал положительно заряженной обкладки конденсатора. Согласно формуле (4.4) интеграл можно разбить на две части - для од-

ной и другой обкладок. Тогда

\[
W=\left(q_{+} \varphi_{+}+q_{-} \varphi_{-}\right) / 2 .
\]

Так как \(q_{-}=-q_{+}\), то

\[
W=q_{+}\left(\varphi_{+}-\varphi_{-}\right) / 2=q U / 2 \text {, }
\]

где \(q=q_{+}\)- заряд конденсатора, \(U\) - разность потенциалов на его обкладках. Приняв во внимание, что \(C=q / U\), получим следующие выражения для энергии конденсатора:


\begin{equation*}
W=\frac{q U}{2}=\frac{C U^{2}}{2}=\frac{q^{2}}{2 C} . \tag{4.7}
\end{equation*}


Здесь надо заметить, что эти формулы определяют полную энергию взаимодействия: не только энергию взаимодействия зарядов одной обкладки с зарядами другой, но и энергию взаимодействия зарядов внутри каждой обкладки.

А если есть диэлектрик? Мы сейчас убедимся, что формуль (4.6) и (4.7) справедливы и при наличии диэлектрика. С этой целью рассмотрим процесс зарядки конденсатора как перенос заряда малыми порциями \(\mathrm{d} q^{\prime}\) с одной обкладки на другую.

Элементарная работа, совершенная нами при этом против сил поля, запишется как

\[
\delta A=U^{\prime} \mathrm{d} q^{\prime}=\left(q^{\prime} / C\right) \mathrm{d} q^{\prime},
\]

где \(U^{\prime}\) - разность потенциалов между обкладками в момент, когда переносится очередная порция заряда \(\mathrm{d} q^{\prime}\).

Проинтегрировав это выражение по \(q^{\prime}\) от 0 до \(q\), получим

\[
A=q^{2} / 2 C,
\]

что совпадает с выражением для полной энергии конденсатора. Значит, совершаемая нами работа против сил электрического поля целиком идет на создание энергии \(W\) заряженного конденсатора. Кроме того, полученное выражение для работы \(A\) справедливо и в том случае, когда между обкладками конденсатора имеется произвольный диэлектрик. Этим самым мы доказали справедливость формул (4.7) и при наличии диэлектрика.

Все сказанное относится, очевидно, и к формулам (4.6).

\section*{§ 4.3. Энергия электрического поля}
О локализации энергии. Формула (4.4) определяет электрическую энергию \(W\) любой системы через заряды и потенциалы. Но, оказывается, энергию \(W\) можно выразить также и через величину, характеризующую само электрическое поле, - через напряженность Е. Убедимся в этом сначала на простейшем примере плоского конденсатора, пренебрегая искажением поля у краев пластин (краевым эффектом). Подстановка в формулу \(W=C U^{2} / 2\) выражения \(C=\varepsilon \varepsilon_{0} S / h\) дает

\[
W=\frac{C U^{2}}{2}=\frac{\varepsilon \varepsilon_{0} S U^{2}}{2 h}=\frac{\varepsilon \varepsilon_{0}}{2}\left(\frac{U}{h}\right)^{2} S h .
\]

А поскольку \(U / h=E\) и \(S h=V\) (объем между обкладками конденсатора), то


\begin{equation*}
W=\left(\varepsilon \varepsilon_{0} E^{2} / 2\right) V=(E D / 2) V . \tag{4.8}
\end{equation*}


Полученная формула справедлива для однородного поля, заполняющего объем \(V\).

В общей теории доказывается, что в случае неоднородного поля энергия \(W\) для изотропных диэлектриков определяется формулой


\begin{equation*}
W=\int \frac{\varepsilon \varepsilon_{0} E^{2}}{2} \mathrm{~d} V=\int \frac{\mathbf{E D}}{2} \mathrm{~d} V . \tag{4.9}
\end{equation*}


Подынтегральное выражение в этом уравнении имеет смысл энергии, заключенной в объеме \(\mathrm{d} V\). Это подводит нас к весьма важной и плодотворной физической идее о локализации энергии в самом поле. Данное предположение нашло опытное подтверждение в области переменных во времени полей. Только там встречаются явления, которые можно истолковать на основе идеи о локализации энергии в поле. Именно переменные поля могут существовать независимо от возбудивших их электрических зарядов и распространяться в пространстве в виде электромагнитных волн. И опыт показывает, что электромагнитные волны переносят энергию - уже это заставляет нас признать, что носителем энергии является само поле.

Из последних двух формул следует, что электрическая энергия распределена в пространстве с объемной плотностью


\begin{equation*}
w=\frac{\varepsilon \varepsilon_{0} E^{2}}{2}=\frac{\mathbf{E D}}{2} . \tag{4.10}
\end{equation*}


Заметим, что эта формула справедлива только в случае изотропного диэлектрика, для которого выполняется соотношение \(\mathbf{D}=\varepsilon \varepsilon_{0} \mathbf{E}\). Для анизотропных диэлектриков дело обстоит сложнее.

Еще об обосновании формулы (4.9). Энергия уединенного заряженного проводника, как известно, есть \(W=q \varphi / 2\). Покажем, что это так, исходя из идеи о локализации энергии в поле.

Рассмотрим произвольный положительно заряженный проводник. Выделим мысленно бесконечно малого сечения трубку, ограниченную линиями вектора \(\mathbf{E}\) (рис. 4.2), и в ней возьмем элементарный объем \(\mathrm{d} V=\mathrm{d} S \mathrm{~d} l\). В этом объеме заключена энергия

\begin{figure}[h]
\begin{center}
  \includegraphics[alt={},max width=\textwidth]{3e870b25-6320-483d-9133-cdd5212c2707-104_306_380_685_778}
\captionsetup{labelformat=empty}
\caption{Рис. 4.2}
\end{center}
\end{figure}

\[
\frac{E D}{2} \mathrm{~d} S \mathrm{~d} l=\frac{D \mathrm{~d} S}{2} E \mathrm{~d} l .
\]

Теперь найдем энергию, локализованную во всей выделенной нами трубке. Для этого проинтегрируем последнее выражение, учитывая, что произведение \(D \mathrm{~d} S\) одинаково во всех сечениях трубки, и поэтому его можно вынести за знак интеграла:

\[
\mathrm{d} W=\frac{D \mathrm{~d} S}{2} \int_{A}^{\infty} E \mathrm{~d} l=\frac{D \mathrm{~d} S}{2} \varphi,
\]

где \(A\) - начало трубки.\\
Остается сделать последний шаг - проинтегрировать полученное выражение по всем трубкам, и мы найдем энергию, локализованную во всем поле. Принимая во внимание, что потенциал φ одинаков у торцов всех трубок (они ведь начинаются на поверхности проводника), запишем

\[
W=\frac{\varphi}{2} \oint D \mathrm{~d} S
\]

где интегрирование проводится по замкнутой поверхности, совпадающей с одной из эквипотенциальных поверхностей. По теореме Гаусса этот интеграл равен заряду \(q\) на проводнике, и мы получим окончательно

\[
W=q \varphi / 2
\]

что и требовалось доказать.\\
Рассмотрим два примера, иллюстрирующих возможности и преимущества, которые дает использование идеи о локализации энергии в поле.

Пример 1. Точечный заряд \(q\) находится в центре шарового слоя из однородного диэлектрика с проницаемостью \(\varepsilon\). Внутренний и наружный радиусы слоя равны соответственно \(a\) и \(b\). Найдем электрическую энергию, заключенную в данном диэлектрическом слое.\\
Мысленно выделим в диэлектрике очень тонкий концентрический сферический слой радиусом от \(r\) до \(r+\mathrm{d} r\). Энергия, локализованная в этом слое:

\[
\mathrm{d} W=\frac{\varepsilon \varepsilon_{0} E^{2}}{2} 4 \pi r^{2} \mathrm{~d} r
\]

где \(E=q^{2} / 4 \pi \varepsilon_{0} \varepsilon r^{2}\). Проинтегрировав предыдущее выражение по \(r\) от \(a\) до \(b\), получим

\[
W=\frac{q^{2}}{8 \pi \varepsilon_{0} \varepsilon}\left(\frac{1}{a}-\frac{1}{b}\right)
\]

Пример 2. Найдем работу, которую надо совершить против электрических сил, чтобы удалить диэлектрическую пластинку из плоского заряженного конденсатора. Предполагается, что заряд \(q\) конденсатора остается неизменным и диэлектрик заполняет все пространство между обкладками. Емкость конденсатора без диэлектрика равна \(C\).\\
Работа против электрических сил в этой системе пойдет на приращение ее электрической энергии:

\[
A=\Delta W=W_{2}-W_{1},
\]

где \(W_{1}\) - энергия поля между обкладками конденсатора при наличии диэлектрика; \(W_{2}\) - то же, но при отсутствии диэлектрика. Имея в виду, что модуль вектора \(\mathbf{D}\) не изме-

нится в результате извлечения пластины, т. е. \(D_{2}=D_{1}=\sigma\), запишем

\[
A=W_{2}-W_{1}=\left(\frac{D^{2}}{2 \varepsilon_{0}}-\frac{D^{2}}{2 \varepsilon \varepsilon_{0}}\right) V=\frac{q^{2}}{2 C}\left(1-\frac{1}{\varepsilon}\right),
\]

где \(V=S h\) и \(C=\varepsilon_{0} S / h ; S\) и \(h\) - площадь каждой обкладки и расстояние между ними.

Работа поля при поляризации диэлектрика. Анализируя формулу (4.10) для объемной плотности энергии, мы замечаем, что при одном и том же значении \(E\) величина \(w\) при наличии диэлектрика оказывается в \(\varepsilon\) раз больше, чем при отсутствии диэлектрика. На первый взгляд это кажется странным: ведь напряженность поля в обоих случаях мы поддерживаем одной и тойже. Как мы сейчас увидим, все дело в том, что при создании поля в диэлектрике оно совершает дополнительную работу, связанную с поляризацией. И под энергией поля в диэлектрике следует понимать всю энергию, которую нужно затратить на возбуждение электрического поля, а она складывается из собственной электрической энергии и той дополнительной работы, которая совершается при поляризации диэлектрика.

Чтобы в этом убедиться, подставим в (4.10) вместо \(\mathbf{D}\) величину \(\varepsilon_{0} \mathbf{E}+\mathbf{P}\), тогда


\begin{equation*}
w=\frac{\varepsilon_{0} E^{2}}{2}+\frac{\mathbf{E P}}{2} . \tag{4.11}
\end{equation*}


Первое слагаемое здесь совпадает с плотностью энергии поля E в вакууме. Остается проверить, что «дополнительная» энергия EP/2 связана с поляризацией диэлектрика.

Подсчитаем работу, которую совершает электрическое поле на поляризацию единицы объема диэлектрика, т. е. на смещение зарядов \(\rho_{+}^{\prime}\) и \(\rho_{-}^{\prime}\) соответственно по и против поля - при возрастании напряженности от \(\mathbf{E}\) до \(\mathbf{E}+\mathrm{dE}\). Пренебрегая членами второго порядка малости, запишем

\[
\delta A=\rho_{+}^{\prime} \mathbf{E} \mathrm{d} \mathbf{l}_{+}+\rho_{-}^{\prime} \mathbf{E} \mathrm{d} \mathbf{l}_{-},
\]

где \(\mathrm{dl}_{+}\)и \(\mathrm{dl}_{-}\)- дополнительные смещения при увеличении поля на dE (рис. 4.3). Учитывая, что \(\rho_{-}^{\prime}=-\rho_{+}^{\prime}\), получаем

\[
\delta A=\rho_{+}^{\prime}\left(\mathrm{d} \mathbf{l}_{+}-\mathrm{d} \mathbf{l}_{-}\right) \mathbf{E}=\rho_{+}^{\prime} \mathrm{d} \mathbf{l} \cdot \mathbf{E},
\]

\begin{figure}[h]
\begin{center}
  \includegraphics[alt={},max width=\textwidth]{3e870b25-6320-483d-9133-cdd5212c2707-107_47_430_177_399}
\captionsetup{labelformat=empty}
\caption{Рис. 4.3}
\end{center}
\end{figure}

где \(\mathrm{dl}=\mathrm{dl} \mathrm{l}_{+}-\mathrm{dl} \_\)- дополнительное смещение положительных зарядов относительно отрицательных. Согласно (3.4) \(\rho_{+}^{\prime} \mathrm{dl}=\mathrm{EdP}\), и


\begin{equation*}
\delta A=\mathbf{E} \mathrm{d} \mathbf{P} . \tag{4.12}
\end{equation*}


Так как \(\mathbf{P}=\varkappa \varepsilon_{0} \mathbf{E}\), то

\[
\delta A=\mathbf{E} \cdot \varkappa \varepsilon_{0} \mathrm{~d} \mathbf{E}=\mathrm{d}\left(\frac{\varkappa \varepsilon_{0} E^{2}}{2}\right)=\mathrm{d}\left(\frac{\mathbf{E} \mathbf{P}}{2}\right) .
\]

Отсюда вся работа на поляризацию единицы объема диэлектрика


\begin{equation*}
A=\mathbf{E P} / 2, \tag{4.13}
\end{equation*}


что совпадает со вторым слагаемым формулы (4.11).\\
Таким образом, объемная плотность энергии \(w=\mathbf{E D} / 2\) включает в себя собственную энергию поля \(\varepsilon_{0} E^{2} / 2\) и энергию \(\mathbf{E P} / 2\), связанную с поляризацией вещества.

\section*{§ 4.4. Система двух заряженных тел}
Представим себе систему из двух заряженных тел в вакууме. Пусть одно тело создает в окружающем пространстве поле \(\mathbf{E}_{1}\), а другое - поле \(\mathbf{E}_{2}\). Результирующее поле \(\mathbf{E}=\mathbf{E}_{1}+\mathbf{E}_{2}\) и квадрат этой величины

\[
E^{2}=E_{1}^{2}+E_{2}^{2}+2 \mathbf{E}_{1} \mathbf{E}_{2} .
\]

Поэтому полная энергия \(W\) данной системы согласно (4.9) равна сумме трех интегралов:


\begin{equation*}
W=\int \frac{\varepsilon_{0} E_{1}^{2}}{2} \mathrm{~d} V+\int \frac{\mathrm{e}_{0} \mathrm{E}_{2}^{2}}{2} \mathrm{~d} V+\int \varepsilon_{0} \mathbf{E}_{1} \mathbf{E}_{2} \mathrm{~d} V, \tag{4.14}
\end{equation*}


что совпадает с формулой (4.5) и раскрывает полевой смысл входящих в нее слагаемых. Первые два интеграла в (4.14) представляют собой собственную энергию первого и второго за-

ряженных тел ( \(W_{1}\) и \(W_{2}\) ), последний интеграл - энергию их взаимодействия ( \(W_{12}\) ).

Отметим следующие важные обстоятельства в связи с формулой (4.14).

\begin{enumerate}
  \item Собственная энергия каждого заряженного тела - величина существенно положительная. Положительной является всегда и полная энергия (4.9) - это сразу видно из того, что под интегралом находятся существенно положительные величины. Энергия же взаимодействия может быть как положительной, так и отрицательной.
  \item При всех возможных перемещениях заряженных тел, не изменяющих конфигурации зарядов на каждом теле, собственная энергия тел остается постоянной, и поэтому ее можно считать аддитивной постоянной в выражении для полной энергии \(W\). В этих случаях изменения \(W\) определяются всецело только изменениями взаимной энергии \(W_{12}\). В частности, именно так ведет себя энергия системы двух точечных зарядов при изменении расстояния между ними.
  \item В отличие от вектора \(\mathbf{E}\) энергия электрического поля величина не аддитивная, т. е. энергия поля \(\mathbf{E}\), являющегося суммой полей \(\mathbf{E}_{1}\) и \(\mathbf{E}_{2}\), вообще говоря, не равна сумме энергий обоих полей из-за взаимной энергии \(W_{12}\). В частности, при возрастании всюду \(E\) в \(n\) раз энергия поля увеличивается в \(n^{2}\) раз.
\end{enumerate}

\section*{§ 4.5. Силы при наличии диэлектрика}
Электрострикция. Опыт показывает, что на диэлектрик в электрическом поле действуют силы (их иногда называют пондеромоторньми). Эти силы возникают и в тех случаях, когда диэлектрик в целом не заряжен. Причиной их возникновения является в конечном счете действие неоднородного электрического поля на дипольные молекулы поляризованного диэлектрика (как известно, на диполи в неоднородном электрическом поле действует сила, направленная в сторону возрастания данного поля). Причем эти силы обусловлены неоднородностью не только макрополя, но и микрополя, создаваемого в основном ближайшими молекулами поляризованного диэлектрика.

Под действием указанных электрических сил поляризованный диэлектрик деформируется. Это явление называют элект-

рострикцией. Вследствие электрострикции в диэлектрике возникают механические напряжения.

Все это приводит к тому, что на проводник, находящийся в поляризованном диэлектрике, действует не только электрическая сила, зависящая от зарядов на проводнике, но и дополнительная механическая сила со стороны диэлектрика. В общем случае влияние диэлектрика на результирующую силу, испытываемую проводником, не может быть учтено никакими простыми соотношениями, и задача вычисления сил с одновременным исследованием механизма их возникновения, как правило, оказывается весьма сложной. Однако во многих случаях эти силы можно вычислить достаточно просто без детального анализа их происхождения - с помощью закона сохранения энергии.

Энергетический метод определения сил. Этот метод является наиболее общим. Он позволяет, отвлекаясь от причин возникновения сил, автоматически учитывать все силовые взаимодействия (электрические и механические) и поэтому приводит к правильному результату.

Покажем, в чем суть энергетического метода расчета сил. Наиболее просто обстоит дело в случае, когда заряженные проводники отключены от источников напряжения. В этом случае заряды на проводниках остаются постоянными, и мы можем утверждать, что работа \(A\) всех внутренних сил системы при медленных перемещениях проводников и диэлектриков совершается целиком за счет убыли электрической энергии \(W\) системы (или ее поля). Здесь предполагается, что при указанных перемещениях не происходит преобразования электрической энергии в другие формы, или, точнее, считается, что такие преобразования пренебрежимо малы. Таким образом, для бесконечно малых перемещений можно записать


\begin{equation*}
\delta A=-\left.\mathrm{d} W\right|_{q}, \tag{4.15}
\end{equation*}


где символ \(q\) подчеркивает, что убыль энергии системы должна быть вычислена при постоянных зарядах на проводниках.

Уравнение (4.15) является исходным для определения сил, действующих на проводники и диэлектрики в электрическом поле. Делается это, например, так. Пусть нас интересует сила,

действующая на данное тело (проводник или диэлектрик). Совершим бесконечно малое поступательное перемещение \(\mathrm{d} x\) этого тела в интересующем нас направлении \(X\). Тогда работа искомой силы \(\mathbf{F}\) на перемещении \(\mathrm{d} x\) есть \(\mathrm{d} A=F_{x} \mathrm{~d} x\), где \(F_{x}\) - проекция силы \(\mathbf{F}\) на положительное направление оси \(X\). После подстановки последнего выражения для \(\mathrm{d} A\) в (4.15) и деления на \(\mathrm{d} x\) получим


\begin{equation*}
F_{x}=-\partial W /\left.\partial x\right|_{q} . \tag{4.16}
\end{equation*}


Следует обратить внимание вот на что. Сила, как известно, зависит только от положения тел и распределения зарядов в данный момент. Она не может зависеть от того, как будет развиваться энергетический процесс в том случае, если система придет в движение под действием сил. А это значит, что для вычисления \(F_{x}\) по формуле (4.16) нет надобности подбирать такой режим, при котором обязательно все заряды проводников оставались бы постоянными ( \(q=\operatorname{const}\) ). Надо просто найти приращение \(\mathrm{d} W\) при условии, что \(q=\) const, а это - чисто математическая операция.

Заметим, что если перемещения проводить при постоянном потенциале на проводниках, то соответствующий расчет приводит к другой формуле для силы: \(F_{x}=+\partial W /\left.\partial x\right|_{\varphi}\). Однако - и это важно - результат расчета \(F_{x}\) по этой формуле или по (4.16) оказывается одинаковым, как и должно быть. Поэтому мы ограничимся в дальнейшем использованием только формулы (4.16) и будем применять ее для любых условий, включая и такие, где при малых перемещениях \(q \neq\) const. Нас это не должно смущать: производную \(\partial W / \partial x\) мы в подобных случаях будем вычислять при \(q=\) const.\\
Пример. Найдем силу, действующую на одну из обкладок плоского конденсатора в жидком диэлектрике, если расстояние между обкладками \(h\), емкость конденсатора в этих условиях \(C\) и на нем поддерживается напряжение \(U\).\\
В данном случае при мысленном раздвижении обкладок постоянным оказывается напряжение \(U\), а заряд \(q\) конденсатора меняется (это видно из соотношения \(C=q / U\) ). Несмотря на это, расчет силы мы будем проводить в предположении, что \(q=\) const, т. е. по формуле (4.16). Здесь наиболее подходящим

выражением для энергии конденсатора является следующее:

\[
W=\frac{q^{2}}{2 C}=\frac{q^{2}}{2 \varepsilon \varepsilon_{0} S} x,
\]

где \(\varepsilon\) - проницаемость диэлектрика, \(S\) - площадь каждой обкладки, \(x\) - расстояние между ними ( \(x=h\) ).

Выберем далее положительное направ-

\begin{figure}[h]
\begin{center}
  \includegraphics[alt={},max width=\textwidth]{3e870b25-6320-483d-9133-cdd5212c2707-111_203_331_439_196}
\captionsetup{labelformat=empty}
\caption{Рис. 4.4}
\end{center}
\end{figure}

ление оси \(X\), как показано на рис. 4.4. Согласно (4.16) сила, действующая на верхнюю обкладку конденсатора:


\begin{equation*}
F_{x}=-\left.\frac{\partial W}{\partial x}\right|_{q}=-\frac{q^{2}}{2 \varepsilon \varepsilon_{0} S} . \tag{1}
\end{equation*}


Здесь знак минус указывает на то, что вектор \(\mathbf{F}\) направлен в отрицательную сторону оси \(X\), т. е. сила имеет характер притяжения. Учитывая, что \(q=\sigma S=D S= =\varepsilon \varepsilon_{0} E S\) и \(E=U / h\), преобразуем (1) к виду

\[
F_{x}=-C U^{2} / 2 h .
\]

Силы в жидком диэлектрике. Из формулы (1) предыдущего примера видно, что сила взаимодействия обкладок плоского конденсатора в жидком диэлектрике в \(\varepsilon\) раз меньше, чем в вакууме (где \(\varepsilon=1\) ). Этот результат, как показывает опыт, можно обобщить: при заполнении всего пространства, где есть электрическое поле, жидким или газообразным диэлектриком силы взаимодействия между заряженными проводниками (при неизменных зарядах на них) уменьшаются в \(\varepsilon\) раз:


\begin{equation*}
F=F_{0} / \varepsilon . \tag{4.17}
\end{equation*}


Отсюда следует, что два точечных заряда \(q_{1}\) и \(q_{2}\), находящиеся на расстоянии \(r\) друг от друга внутри безграничного жидкого или газообразного диэлектрика, взаимодействуют с силой


\begin{equation*}
F=\frac{1}{4 \pi \varepsilon_{0}} \frac{\left|q_{1} q_{2}\right|}{\varepsilon r^{2}}, \tag{4.18}
\end{equation*}


т. е. тоже в \(\varepsilon\) раз меньшей, чем в вакууме. Эта формула выражает закон Кулона для точечных зарядов в безграничном диэлектрике.

Следует обратить особое внимание на то, что в последнем законе под точечными подразумеваются сторонние заряды, сосредоточенные на макроскопических телах, размеры которых малы по сравнению с расстоянием между ними. Таким образом, закон (4.18) в отличие от закона Кулона в вакууме имеет весьма ограниченную область применения: диэлектрик должен быть однородным, безграничным, обязательно жидким или газообразным, а взаимодействующие тела - точечными в макроскопическом смысле.

Интересно, что в однородном жидком или газообразном диэлектрике, заполняющем все пространство, где есть поле, как напряженность \(\mathbf{E}\), так и сила \(\mathbf{F}\), действующая на точечный заряд \(q\), в \(\varepsilon\) раз меньше \(\mathbf{E}_{0}\) и \(\mathbf{F}_{0}\) при отсутствии диэлектрика. А это значит, что сила \(\mathbf{F}\), действующая на точечный заряд \(q\), определяется в этом случае такой же формулой, как и в вакууме:


\begin{equation*}
\mathbf{F}=q \mathbf{E}, \tag{4.19}
\end{equation*}


где \(\mathbf{E}\) - напряженность поля в диэлектрике в том месте, куда помещают сторонний заряд \(q\). Только в этом случае по силе \(\mathbf{F}\) формула (4.19) позволяет определить поле \(\mathbf{E}\) в диэлектрике. Следует обратить внимание, что на сам сторонний заряд - он сосредоточен на каком-то небольшом теле - будет действовать әpyгое поле - не то, что в самом диэлектрике. И тем не менее, формула (4.19) дает, как это ни удивительно, верный результат.

Поверхностная плотность силы. Речь пойдет о силе, действующей на единицу поверхности заряженного проводника в жидком или газообразном диэлектрике. Рассмотрим с этой целью плоский конденсатор в жидком диэлектрике. Пусть конденсатор заряжен и отключен от источника напряжения чтобы заряд конденсатора и поле Е внутри него не менялись при раздвигании обкладок.

Вернемся к рис. 4.4. Энергия конденсатора - это энергия поля внутри него. В соответствии с (4.9) она равна \(W=(E D / 2) S x\), где \(S\) - площадь каждой обкладки, \(x\) - расстояние между ними ( \(S x\) - объем, занимаемый полем). Согласно (4.16) сила, действующая на верхнюю обкладку,

\[
F_{x}=-\partial W /\left.\partial x\right|_{q}=-E D S / 2,
\]

откуда поверхностная плотность силы


\begin{equation*}
F_{\text {ед }}=\frac{\mathbf{E D}}{2} . \tag{4.20}
\end{equation*}


Мы получили интересный и важный результат, имеющий общий характер (в жидком или газообразном диэлектрике). Оказывается, поверхностная плотность силы, действующей на проводник, равна объемной плотности электрической энергии вблизи поверхности. Направлена эта сила всегда по нормали к поверхности, причем наружу проводника (стремясь его растянуть) независимо от знака поверхностного заряда.

\section*{Задачи}
4.1. Энергия взаимодействия. Точечный заряд \(q\) находится на расстоянии \(l\) от безграничной проводящей плоскости. Найти энергию взаимодействия \(W\) этого заряда с зарядами, индуцированными на плоскости.

Решение. Мысленно «заморозим» распределенный по плоскости заряд, и в этих условиях переместим точечный заряд \(q\) в бесконечность. Заряд \(q\) при этом будет перемещаться в потенциальном поле, которое эквивалентно полю неподвижного точечного фиктивного заряда \(-q\), расположенного на неизменном расстоянии \(l\) по другую сторону от плоскости. И мы сразу можем написать

\[
W=-\frac{1}{4 \pi \varepsilon_{0}} \frac{q^{2}}{2 l} .
\]

4.2. Собственная, взаимная и полная энергии. Система состоит из двух концентрических металлических оболочек радиусами \(R_{1}\) и \(R_{2}\) с соответствующими зарядами \(q_{1}\) и \(q_{2}\). Найти собственную энергию \(W_{1}\) и \(W_{2}\) каждой оболочки, энергию \(W_{\text {вз }}\) взаимодействия оболочек и полную электрическую энергию \(W\) данной системы, если \(R_{2}>R_{1}\).\\
Решение. Собственная энергия каждой оболочки согласно (4.6) равна \(q \varphi / 2\), где \(\varphi\) - потенциал оболочки, обусловленный только зарядом \(q\) на ней, т. е. \(\varphi=q / 4 \pi \varepsilon_{0} R\), где \(R\) - радиус оболочки. Таким образом, собственная энергия каждой оболочки

\[
W_{1,2}=-\frac{1}{4 \pi \varepsilon_{0}} \frac{q_{1,2}}{2 R_{1,2}}
\]

Энергия же взаимодействия заряженных оболочек равна заряду \(q\) одной из них, умноженному на потенциал \(\varphi\), который создает заряд другой оболочки в месте нахождения заряда \(q: W_{\text {вз }}=q \varphi\). В нашем случае ( \(R_{2}>R_{1}\) )

\[
W_{\text {вз }}=q_{1} \frac{1}{4 \pi \varepsilon_{0}} \frac{q_{2}}{R_{2}}=\frac{1}{4 \pi \varepsilon_{0}} \frac{q_{1} q_{2}}{R_{2}} .
\]

Полная электрическая энергия системы

\[
W=W_{1}+W_{2}+W_{\text {вз }}=\frac{1}{4 \pi \varepsilon_{0}}\left(\frac{q_{1}^{2}}{2 R_{1}}+\frac{q_{2}^{2}}{2 R_{2}}+\frac{q_{1} q_{2}}{R_{2}}\right) .
\]

4.3. Два небольших металлических шарика радиусами \(R_{1}\) и \(R_{2}\) находятся в вакууме на расстоянии, значительно превышающем их размеры, и имеют некоторый определенный суммарный заряд. При каком отношении \(q_{1} / q_{2}\) зарядов на шариках электрическая энергия системы будет минимальной? Какова при этом разность потенциалов между шариками?

Решение. Электрическая энергия данной системы

\[
W=W_{1}+W_{2}+W_{12}=\frac{1}{4 \pi \varepsilon_{0}}\left(\frac{q_{1}^{2}}{2 R_{1}}+\frac{q_{2}^{2}}{2 R_{2}}+\frac{q_{1} q_{2}}{l}\right)
\]

где \(W_{1}\) и \(W_{2}\) - собственные электрические энергии шариков \((q \varphi / 2) ; W_{12}\) - энергия их взаимодействия ( \(q_{1} \varphi_{2}\) или \(q_{2} \varphi_{1}\) ); \(l\) - расстояние между шариками. Так как \(q_{2}=q-q_{1}\), где \(q\) - суммарный заряд системы, то

\[
W=\frac{1}{4 \pi \varepsilon_{0}}\left[\frac{q_{1}^{2}}{2 R_{1}}+\frac{\left(q-q_{1}\right)^{2}}{2 R_{2}}+\frac{q_{1}\left(q-q_{1}\right)}{l}\right]
\]

Энергия \(W\) будет минимальной при \(\partial W / \partial q_{1}=0\). Отсюда

\[
q_{1} \approx q \frac{R_{1}}{R_{1}+R_{2}} \quad \text { и } \quad q_{2} \approx q \frac{R_{2}}{R_{1}+R_{2}},
\]

где учтено, что \(R_{1}\) и \(R_{2}\) значительно меньше \(l\). Таким образом

\[
q_{1} / q_{2}=R_{1} / R_{2}
\]

Потенциал каждого шарика (их можно рассматривать как изолированные) \(\varphi \sim q / R\), поэтому из предыдущего равенства следует, что \(\varphi_{1}=\varphi_{2}\), т. е. разность потенциалов при таком распределении равна нулю.\\
4.4. Локализация энергии в поле. Заряд \(q\) распределен равномерно по объему шара радиусом \(R\). Полагая диэлектрическую проницаемость всюду равной единице, найти собственную электрическую энергию шара и отношение энергии \(W_{1}\), локализованной внутри шара, к энергии \(W_{2}\) в окружающем пространстве.\\
Решение. Прежде всего найдем с помощью теоремы Гаусса поле внутри и вне шара:

\[
E_{1}=\frac{q}{4 \pi \varepsilon_{0} R^{3}} r(r \leqslant R) ; \quad E_{2}=\frac{q}{4 \pi \varepsilon_{0}} \frac{1}{r^{2}}(r \geqslant R) .
\]

Теперь вычислим собственную электрическую энергию шара:\\
\(W=W_{1}+W_{2}=\int_{0}^{R} \frac{\varepsilon_{0} E_{1}^{2}}{2} 4 \pi r^{2} \mathrm{~d} r+\int_{R}^{\infty} \frac{\varepsilon_{0} E_{2}^{2}}{2} 4 \pi r^{2} \mathrm{~d} r=\frac{q^{2}}{8 \pi \varepsilon_{0} R}\left(\frac{1}{5}+1\right)\).\\
Отсюда следует, что

\[
W=\frac{1}{4 \pi \varepsilon_{0}} \frac{3 q^{2}}{5 R}, \quad \frac{W_{1}}{W_{2}}=\frac{1}{5}
\]

Интересно, что отношение \(W_{1} / W_{2}\) не зависит от радиуса шара.\\
4.5. Имеется сферическая оболочка, заряженная равномерно зарядом \(q\). В центре ее расположен точечный заряд \(q_{0}\). Найти работу электрических сил этой системы при расширении оболочки - увеличении ее радиуса от \(R_{1}\) до \(R_{2}\).\\
Решение. Работа электрических сил равна убыли электрической энергии системы:

\begin{figure}[h]
\begin{center}
  \includegraphics[alt={},max width=\textwidth]{3e870b25-6320-483d-9133-cdd5212c2707-115_315_309_1212_125}
\captionsetup{labelformat=empty}
\caption{Рис. 4.5}
\end{center}
\end{figure}

\[
A=W_{1}-W_{2} .
\]

Чтобы найти разность \(W_{1}-W_{2}\), заметим, что при расширении оболочки (рис. 4.5) электрическое поле, а следовательно, и локализованная в нем энергия изменились только в заштрихованном сферическом слое. Значит,

\[
W_{1}-W_{2}=\int_{R_{1}}^{R_{2}} \frac{\varepsilon_{0}}{2}\left(E_{1}^{2}-E_{2}^{2}\right) 4 \pi r^{2} \mathrm{~d} r
\]

где \(E_{1}\) и \(E_{2}\) - напряженность поля (в заштрихованном слое на расстоянии \(r\) от центра системы) до и после расширения оболочки. С помощью теоремы Гаусса находим

\[
E_{1}=\frac{1}{4 \pi \varepsilon_{0}} \frac{q+q_{0}}{r^{2}}, \quad E_{2}=\frac{1}{4 \pi \varepsilon_{0}} \frac{q_{0}}{r^{2}} .
\]

В результате интегрирования получим

\[
A=\frac{q\left(q_{0}+q / 2\right)}{4 \pi \varepsilon_{0}}\left(\frac{1}{R_{1}}-\frac{1}{R_{2}}\right) .
\]

Замечание. Если эту работу искать через потенциал как \(A=q\left(\varphi_{1}-\right. -\varphi_{2}\) ), где \(\varphi\) - потенциал, создаваемый зарядом \(q_{0}\) в месте нахождения заряда \(q\), ответ будет другим - неверным. Связано это с тем, что при таком подходе не учитывается та дополнительная работа, которую совершают электрические силы при изменении конфигурации заряда \(q\) на расширяющейся оболочке.\\
4.6. Точечный заряд \(q\) находится в центре сферического незаряженного проводящего слоя, внутренний и наружный радиусы которого равны соответственно \(a\) и \(b\). Какую работу произведут электрические силы в данной системе, если заряд \(q\) переместить из его первоначального положения через малое отверстие (рис. 4.6) на очень большое расстояние от сферического слоя?

\begin{figure}[h]
\begin{center}
  \includegraphics[alt={},max width=\textwidth]{3e870b25-6320-483d-9133-cdd5212c2707-116_310_392_588_762}
\captionsetup{labelformat=empty}
\caption{Рис. 4.6}
\end{center}
\end{figure}

Решение. Будем исходить из того, что работа электрических сил равна убыли электрической энергии системы. Последняя же, как известно, локализована в самом поле. Поэтому вопрос сводится, по существу, к выяснению, как изменится само поле в результате этого процесса.

Нетрудно сообразить, что поле вокруг заряда \(q\) изменится только в сферическом слое с внутренним и наружным радиусами \(a\) и \(b\). В самом деле, в начальном положении заряда поля здесь не было, а в конечном положении поле в этом слое есть (ведь сам сферический проводящий слой будет находиться далеко от заряда \(q\) ). Следовательно, искомая работа

\[
A=0-W_{\text {сл }}=-\int_{a}^{b} \frac{\varepsilon_{0} E^{2}}{2} \mathrm{~d} V
\]

где \(W_{\text {сл }}\) - энергия, локализованная в сферическом слое. Имея в виду, что \(E=q / 4 \pi \varepsilon_{0} r^{2}\) и \(\mathrm{d} V=4 \pi r^{2} \mathrm{~d} r\), получим после интегрирования

\[
A=\frac{q^{2}}{8 \pi \varepsilon_{0}} \frac{a-b}{a b}<0 .
\]

4.7. Работа при раздвижении пластин конденсатора. Имеется плоский воздушный конденсатор, площадь каждой обкладки которого равна \(S\). Какую работу \(A^{\prime}\) против электрических сил надо совершить, чтобы увеличить расстояние между обкладками от \(x_{1}\) до \(x_{2}\), если при этом поддерживать неизменным: 1) заряд конденсатора, равный \(q ; 2\) ) напряжение на конденсаторе, равное \(U\) ? Чему равно приращение электрической энергии конденсатора в обоих случаях?\\
Решение. 1. Искомая работа

\[
A^{\prime}=q E_{1}\left(x_{2}-x_{1}\right)=\frac{q^{2}}{2 \varepsilon_{0} S}\left(x_{2}-x_{1}\right)
\]

где \(E_{1}\) - напряженность поля, создаваемого одной обкладкой ( \(E=\sigma / 2 \varepsilon_{0}\) ). Именно в этом поле перемещается заряд, находящийся на другой обкладке. Данная работа целиком идет на приращение электрической энергии: \(\Delta W=A^{\prime}\).\\
2. В этом случае сила, действующая на каждую обкладку конденсатора, будет зависеть от расстояния между ними. Запишем элементарную работу силы, действующей на обкладку при ее перемещении на \(\mathrm{d} x\) относительно другой обкладки:

\[
\delta A^{\prime}=q E_{1} \mathrm{~d} x=\frac{\varepsilon_{0} S U^{2}}{2} \frac{\mathrm{~d} x}{x^{2}},
\]

где учтено, что \(q=C U, E_{1}=U / 2 x\) и \(C=\varepsilon_{0} S / x\). После интегрирования получим

\[
A^{\prime}=\frac{\varepsilon_{0} S U^{2}}{2}\left(\frac{1}{x_{1}}-\frac{1}{x_{2}}\right)>0 .
\]

Приращение электрической энергии конденсатора

\[
\Delta W=\frac{\left(C_{2}-C_{1}\right) U^{2}}{2}=\frac{\varepsilon_{0} S U^{2}}{2}\left(\frac{1}{x_{2}}-\frac{1}{x_{1}}\right)<0 .
\]

Заметим, что \(\Delta W=-A^{\prime}\).\\
Таким образом, раздвигая обкладки, мы совершим положительную работу (против электрических сил), энергия же конденсатора при этом уменьшается. Чтобы понять, в чем тут дело, надо обратиться к источнику, поддерживающему неизменной разность потенциалов на конденсаторе. Этот источник тоже совершает работу \(A_{\text {ист }}\), причем согласно закону сохранения энергии \(A_{\text {ист }}+A^{\prime}=\Delta W\), откуда видно, что \(A_{\text {ист }}=\Delta W-A^{\prime}=-2 A^{\prime}<0\).\\
4.8. Силы, действующие между проводниками в диэлектрике. Плоский конденсатор опустили в горизонтальном положении в жидкий диэлектрик с проницаемостью \(\varepsilon\), который заполнил зазор между пластинами. Ширина зазора \(h\). Затем конденсатор подключили к постоянному напряжению \(U\). Найти силу \(f^{\prime}\), действующую на единицу поверхности пластины со стороны диэлектрика.

Решение. Результирующая сила \(f\), которая действует на единицу площади каждой из пластин, может быть представлена как


\begin{equation*}
f=f_{0}-f^{\prime}, \tag{1}
\end{equation*}


где \(f_{0}\) - электрическая сила, действующая на единицу площади со стороны другой пластины (она представляет собой не что иное, как силу на единицу площади при отсутствии диэлектрика). В нашем случае


\begin{equation*}
f=f_{0} / \varepsilon, \quad f_{0}=\sigma E=\sigma^{2} / 2 \varepsilon_{0}, \tag{2}
\end{equation*}


где \(E\) - напряженность поля в месте нахождения одной из пластин, создаваемая зарядами Әругой пластины. Имея в виду, что \(\sigma=D=\varepsilon \varepsilon_{0} U / h\), получим после подстановки (2) в (1):

\[
f^{\prime}=f_{0}(1-1 / \varepsilon)=\varepsilon(\varepsilon-1) \varepsilon_{0} U^{2} / 2 h^{2} .
\]

Например, при \(U=500 \mathrm{~B}, h=1,0\) мм и \(\varepsilon=81\) (вода) \(f^{\prime}=7\) кПа (0,07 атм).\\
4.9. Сила, действующая на диэлектрик. В цилиндрический конденсатор вводят цилиндрический слой однородного диэлектрика с проницаемостью \(\varepsilon\), который заполняет практически все пространство между обкладками. Средний радиус обкладок \(R\), зазор между ними \(d\), причем \(d \ll R\). Конденсатор подключен к источнику постоянного напряжения \(U\). Найти силу, втягивающую диэлектрик в конденсатор.\\
Решение. Воспользовавшись формулой \(W=q^{2} / 2 C\) для энергии конденсатора, найдем согласно (4.16), что искомая сила


\begin{equation*}
F_{x}=-\left.\frac{\partial W}{\partial x}\right|_{q}=\frac{q^{2}}{2} \frac{\partial C / \partial x}{C^{2}}=\frac{U^{2}}{2} \frac{\partial C}{\partial x} . \tag{1}
\end{equation*}


Емкость данного конденсатора при условии \(d \ll R\) определяется формулой для плоского конденсатора, поэтому если диэлектрик вдвинут на глубину \(x\), а длина конденсатора \(l\), то


\begin{equation*}
C=\frac{\varepsilon \varepsilon_{0} x \cdot 2 \pi R}{d}+\frac{\varepsilon_{0}(l-x) \cdot 2 \pi R}{d}=\frac{\varepsilon_{0} \cdot 2 \pi R}{d}(\varepsilon x+l-x) . \tag{2}
\end{equation*}


После подстановки (2) в (1) получим

\[
F_{x}=\varepsilon_{0}(\varepsilon-1) \pi R U^{2} / d .
\]

4.10. Конденсатор состоит из двух неподвижных пластин, имеющих

\begin{figure}[h]
\begin{center}
  \includegraphics[alt={},max width=\textwidth]{3e870b25-6320-483d-9133-cdd5212c2707-119_250_304_312_166}
\captionsetup{labelformat=empty}
\caption{Рис. 4.7}
\end{center}
\end{figure}

форму полукруга радиусом \(R\), и расположенной между ними подвижной пластины из диэлектрика с проницаемостью \(\varepsilon\). Пластина может свободно поворачиваться вокруг оси \(O\) (рис. 4.7), ее толщина \(h\), что практически равно расстоянию между неподвижными пластинами. Между пластинами конденсатора поддерживается постоянное напряжение \(U\). Найти момент сил \(N\) относительно оси \(O\), действующий на подвижную пластину в положении, показанном на рисунке.

Решение. Работа, которую совершает момент сил \(N\) при повороте пластины на элементарный угол \(d \alpha\), равна убыли электрической энергии системы при \(q=\) const [cm. (4.16)]:

\[
N_{z} \mathrm{~d} \alpha=-\left.\mathrm{d} W\right|_{q},
\]

где \(W=q^{2} / 2 C\). Поэтому


\begin{equation*}
N_{z}=-\left.\frac{\partial W}{\partial \alpha}\right|_{q}=\frac{q^{2}}{2} \frac{\partial C / \partial \alpha}{C^{2}} . \tag{1}
\end{equation*}


В данном случае \(C=C_{1}+C_{\varepsilon}\), где \(C_{1}\) и \(C_{\varepsilon}\) - емкости частей конденсатора без диэлектрика и с диэлектриком. Площадь сектора с углом \(\alpha\) определяется как \(S=\alpha R^{2} / 2\), поэтому

\[
C=\varepsilon_{0} \alpha R^{2} / 2 h+\varepsilon \varepsilon_{0}(\pi-\alpha) R^{2} / 2 h .
\]

Отсюда \(\frac{\partial C}{\partial \alpha}=\frac{\varepsilon_{0} R^{2}}{2 h}(1-\varepsilon)\). Подставим это выражение в формулу (1) и учтем, что \(C=q / U\), тогда

\[
N_{z}=\frac{U^{2}}{2} \frac{\varepsilon_{0} R^{2}}{2 h}(1-\varepsilon)=-(\varepsilon-1) \frac{\varepsilon_{0} R^{2} U^{2}}{4 h}<0 .
\]

Отрицательное значение \(N_{z}\) показывает, что момент этих сил действует по часовой стрелке (против положительного направления отсчета угла \(\alpha\); см. рис. 4.7). Этот момент стремится втянуть диэлектрик внутрь конденсатора.\\
Заметим, что \(N_{z}\) не зависит от угла \(\alpha\). Однако в положении равновесия, когда \(\alpha=0\), момент \(N_{z}=0\). Это расхождение связано с тем, что при малых углах \(\alpha\) нельзя пренебрегать краевыми эффектами, как мы делали при решении этой задачи.

\section*{Постоянный электрический ток}
\section*{§ 5.1. Плотность тока. Уравнение непрерывности}
Электрический ток. В этой главе мы ограничимся рассмотрением тока проводимости в проводящей среде, главным образом в металлах. Электрический ток, как известно, представляет собой перенос заряда через ту или иную поверхность \(S\) (например, через сечение проводника).

Носителями тока в проводящей среде могут быть электроны (в металлах), либо ионы (в электролитах), либо другие частицы. При отсутствии электрического поля носители тока совершают хаотическое движение и через любую воображаемую поверхность \(S\) проходит в обе стороны в среднем одинаковое число носителей того и другого знака, так что ток через поверхность \(S\) равен нулю. При включении же электрического поля на хаотическое движение носителей накладывается упорядоченное движение с некоторой средней скоростью \(\mathbf{u}\) и через поверхность \(S\) появится ток. Таким образом, электрический ток - это, по существу, упорядоченный перенос электрических зарядов.

Количественной мерой электрического тока служит сила тока \(I\), т. е. заряд, переносимый сквозь рассматриваемую поверхность \(S\) в единицу времени:

\[
I=\mathrm{d} q / \mathrm{d} t
\]

Единицей силы тока является ампер (А).\\
Плотность тока. Электрический ток может быть распределен по поверхности, через которую он протекает, неравномерно. Поэтому для более детальной характеристики тока вводят вектор плотности тока \(\mathbf{j}\). Модуль этого вектора численно равен отношению силы тока \(\mathrm{d} I\) через элементарную площадку, распо-

ложенную в данной точке перпендикулярно направлению движения носителей, к ее площади \(\mathrm{d} S_{\perp}: j=\mathrm{d} I / \mathrm{d} S_{\perp}\). За направление вектора \(\mathbf{j}\) принимают направление вектора скорости \(\mathbf{u}\) упорядоченного движения положительных носителей (или направление, противоположное направлению вектора скорости упорядоченного движения отрицательных носителей). Если носителями являются как положительные, так и отрицательные заряды, то плотность тока определяется формулой


\begin{equation*}
\mathbf{j}=\rho_{+} \mathbf{u}_{+}+\rho_{-} \mathbf{u}_{-}, \tag{5.1}
\end{equation*}


где \(\rho_{+}\)и \(\rho_{-}\)- объемные плотности положительного и отрицательного зарядов-носителей; \(\mathbf{u}_{+}\)и \(\mathbf{u}_{-}\)- скорости их упорядоченного движения. В проводниках же, где носителями являются только электроны ( \(\rho_{-}<0\) и \(\mathbf{u}_{+}=0\) ), плотность тока


\begin{equation*}
\mathbf{j}=\rho_{-} \mathbf{u}_{-} . \tag{5.2}
\end{equation*}


Поле вектора \(\mathbf{j}\) можно изобразить графически с помощью линий тока (линий вектора j), которые проводят так же, как и линии вектора Е.

Зная вектор плотности тока в каждой точке интересующей нас поверхности \(S\), можно найти и силу тока через эту поверхность как поток вектора j:


\begin{equation*}
I=\int \mathbf{j} \mathrm{d} \mathbf{S} \tag{5.3}
\end{equation*}


Сила тока \(I\) является величиной скалярной и алгебраической. Ее знак, как видно из формулы (5.3), определяется, кроме всего прочего, выбором направления нормали в каждой точке поверхности \(S\), т. е. выбором направления векторов dS. При изменении направления всех векторов \(d S\) на противоположное величина \(I\) меняет знак.

Уравнение непрерывности. Представим себе в некоторой проводящей среде, где течет ток, замкнутую поверхность \(S\). Для замкнутых поверхностей векторы нормалей, а следовательно, и векторы dS принято брать наружу, поэтому интеграл \(\oint \mathbf{j d} \mathbf{S}\) дает заряд, выходящий в единицу времени наружу из объема \(V\), охватываемого поверхностью \(S\). В силу закона сохранения заряда этот интеграл равен убыли заряда в единицу време-

ни внутри объема \(V\) :


\begin{equation*}
\oint \mathbf{j} \mathrm{d} \mathbf{S}=-\frac{\mathrm{d} q}{d t} \tag{5.4}
\end{equation*}


Это соотношение называют уравнением непрерывности. Оно является, по существу, выражением закона сохранения электрического заряда.

В случае стационарного (постоянного) тока распределение зарядов в пространстве должно оставаться неизменным, т. е. в правой части (5.4) \(\mathrm{d} q / \mathrm{d} t=0\). Следовательно, для постоянного тока


\begin{equation*}
\oint \mathbf{j} \mathrm{d} \mathbf{S}=0, \tag{5.5}
\end{equation*}


иначе говоря, линии вектора \(\mathbf{j}\) в этом случае нигде не начинаются и нигде не заканчиваются. Мы говорим, что в случае постоянного тока поле вектора \(\mathbf{j}\) не имеет источников.

Дифференциальная форма уравнения непрерывности. Преобразуем последние два уравнения к дифференциальной форме. Для этого представим заряд \(q\) как \(\int \rho d V\) и правую часть (5.4) как

\[
-\frac{\mathrm{d}}{\mathrm{~d} t} \int \rho \mathrm{~d} V=-\int \frac{\partial \rho}{\partial t} \mathrm{~d} V .
\]

Здесь взят знак частной производной \(\rho\) по времени, поскольку \(\rho\) может зависеть не только от времени, но и от координат. Итак,

\[
\oint \mathbf{j} \mathrm{d} \mathbf{S}=-\int \frac{\partial \rho}{\partial t} \mathrm{~d} V .
\]

Дальнейшее следует проделать так же, как это было сделано для потока вектора \(\mathbf{E}\) в § 1.4. В результате получим, что дивергенция вектора \(\mathbf{j}\) в некоторой точке равна убыли плотности заряда в единицу времени в той же точке:


\begin{equation*}
\nabla \cdot \mathbf{j}=-\partial \rho / \partial t \tag{5.6}
\end{equation*}


Отсюда вытекает условие стационарности (когда \(\partial \rho / \partial t=0\) ):


\begin{equation*}
\nabla \cdot \mathbf{j}=0 . \tag{5.7}
\end{equation*}


Оно означает, что в случае постоянного тока поле вектора \(\mathbf{j}\) не имеет источников.

\section*{§ 5.2. Закон Ома для однородного проводника}
Закон Ома, открытый экспериментально, гласит: сила тока, протекающего по однородному проводнику, пропорциональна разности потенциалов на его концах (напряжению \(U\) ):


\begin{equation*}
I=U / R, \tag{5.8}
\end{equation*}


где \(R\) - электрическое сопротивление проводника.\\
Единицей сопротивления служит ом (Ом).\\
Сопротивление \(R\) зависит от формы и размеров проводника, от его материала и температуры, а также - это следует помнить - от конфигурации (распределения) тока по проводнику. В случае провода смысл сопротивления не вызывает сомнений. В более общем случае объемного распределения тока уже нельзя говорить о сопротивлении, пока не указаны или расположение подводящих к интересующему нас проводнику проводов, или конфигурация тока.

В простейшем случае однородного цилиндрического проводника сопротивление


\begin{equation*}
R=\rho \frac{l}{S}, \tag{5.9}
\end{equation*}


где \(l\) - длина проводника; \(S\) - площадь его поперечного сечения; \(\rho\) - удельное электрическое сопротивление. Последнее зависит от материала проводника и его температуры. Выражают ρ в ом-метрах (Ом‧м).

Значения удельного электрического сопротивления для наиболее хороших проводников (медь, алюминий) составляют при комнатной температуре несколько единиц на \(10^{-8}\) Ом ⋅ м.

Закон Ома в локальной форме. Найдем связь между плотностью тока \(\mathbf{j}\) и полем \(\mathbf{E}\) в одной и той же точке проводящей среды. Ограничимся случаем изотропного проводника, в котором направления векторов \(\mathbf{j}\) и \(\mathbf{E}\) совпадают.

Выделим мысленно в окрестности некоторой точки проводящей среды элементарный цилиндрический объем с образующими, параллельными вектору \(\mathbf{j}\), а значит, и вектору \(\mathbf{E}\). Если поперечное сечение цилиндра \(\mathrm{d} S\), а его длина \(\mathrm{d} l\), то на основании (5.8) и (5.9) можно записать для такого элементарного ци-

линдра

\[
j \mathrm{~d} S=\frac{E \mathrm{~d} l}{\rho \mathrm{~d} l / \mathrm{d} S},
\]

и после соответствующих сокращений получим, уже в векторном виде,


\begin{equation*}
\mathbf{j}=\frac{1}{\rho} \mathbf{E}=\sigma \mathbf{E}, \tag{5.10}
\end{equation*}


где \(\sigma=1 / \rho\) - удельная электропроводимость среды. Единицу, обратную ому, называют сименсом (См), поэтому единицей \(\sigma\) является сименс на метр (См/м).

Таким образом, соотношение (5.10) устанавливает связь между величинами, относящимися к одной и той же точке проводящей среды. Именно поэтому и говорят, что соотношение (5.10) выражает локальный закон Ома.

Способы вычисления сопротивления \(\boldsymbol{R}\). Существует несколько таких способов, и все они, в конечном счете, основаны на использовании соотношений (5.8) - (5.10). Целесообразность применения того или иного способа в каждом случае зависит от конкретной постановки задачи и от характера ее симметрии. Как это практически делается, показано на примерах задач 5.1-5.3 и 5.6.

О заряде внутри проводника с током. Если ток постоянный, то избыточный заряд внутри однородного проводника всюду равен нулю. В самом деле, для постоянного тока справедливо уравнение (5.5). Перепишем его с учетом закона (5.10) в виде

\[
\oint \sigma \mathbf{E} \mathrm{d} S=0,
\]

где интеграл взят по произвольной замкнутой поверхности \(S\) внутри проводника. Для однородного проводника величину \(\sigma\) можно вынести из-под интеграла:

\[
\sigma \oint \mathbf{E} \mathrm{d} \mathbf{S}=0 .
\]

Оставшийся интеграл согласно теореме Гаусса пропорционален алгебраической сумме зарядов внутри замкнутой поверхности \(S\), т. е. пропорционален избыточному заряду внутри этой

поверхности. Но из последнего равенства сразу видно, что этот интеграл равен нулю (ибо \(\sigma \neq 0\) ), а значит, равен нулю и избыточный заряд. В силу произвольности поверхности \(S\) мы заключаем, что избыточный заряд в этих условиях всюду внутри проводника равен нулю.

Избыточный заряд может появиться только на поверхности однородного проводника, в местах соприкосновения с другими проводниками, а также там, где проводник имеет неоднородности.

Электрическое поле проводника с током. Итак, при протекании тока на поверхности проводника (область неоднородности) выступает избыточный заряд, а это означает согласно (2.2), что

\begin{figure}[h]
\begin{center}
  \includegraphics[alt={},max width=\textwidth]{3e870b25-6320-483d-9133-cdd5212c2707-125_312_322_701_78}
\captionsetup{labelformat=empty}
\caption{Рис. 5.1}
\end{center}
\end{figure}

снаружи проводника имеется нормальная составляющая вектора Е. Далее, из непрерывности тангенциальной составляющей вектора \(\mathbf{E}\) мы приходим к выводу о наличии и тангенциальной составляющей этого вектора вблизи поверхности проводника. Таким образом, вектор Е вблизи поверхности проводника составляет (при наличии тока) с нормалью к ней некоторый не равный нулю угол \(\alpha\) (рис. 5.1).\\
И еще. Если токи стационарны, то распределение электрических зарядов в проводящей среде (вообще говоря, неоднородной) не меняется во времени, хотя и происходит движение зарядов: в каждой точке на место уходящих зарядов непрерывно поступают новые. Эти движущиеся заряды создают такое же кулоновское поле, что и неподвижные заряды той же конфигурации. Стало быть, электрическое поле стационарных токов поле потенциальное.

Вместе с тем электрическое поле в случае стационарных токов существенно отличается от электростатического - кулоновского поля неподвижных зарядов. Последнее внутри проводников при равновесии зарядов равно нулю. Электрическое поле у стационарных токов есть также кулоновское поле, однако заряды, его возбуждающие, находятся в движении. Поэтому поле \(\mathbf{E}\) у стационарных токов существует и внутри проводников с током.

\section*{§ 5.3. Обобщенный закон Ома}
Сторонние силы. Если бы все действующие на носители тока силы сводились к силам электростатического поля, то под действием этих сил положительные носители перемещались бы из мест с большим потенциалом к местам с меньшим потенциалом, а отрицательные носители двигались бы в обратном направлении. Это вело бы к выравниванию потенциалов, и в результате все соединенные между собой проводники приобрели бы одинаковый потенциал - ток прекратится. Иными словами, при наличии лишь кулоновских сил стационарное поле должно быть полем статическим.

Чтобы этого не произошло, в цепи постоянного тока наряду с участками, где положительные носители тока движутся в сторону уменьшения потенциала \(\varphi\), должны иметься участки, на которых перенос положительных носителей происходит в сторону возрастания \(\varphi\), т. е против сил электрического поля. Перенос носителей на этих участках возможен лишь с помощью сил не электростатического происхождения. Это так называемые сторонние силь.

Таким образом, для поддержания постоянного тока необходимы сторонние силы, действующие либо на отдельных участках цепи, либо во всей цепи. Физическая природа сторонних сил может быть весьма различной. Они могут быть обусловлены, например, химической и физической неоднородностью проводника - таковы силы, возникающие при соприкосновении разнородных проводников (гальванические элементы, аккумуляторы) или проводников различной температуры (термоэлементы) и др.

Обобщенный закон Ома. Для количественной характеристики сторонних сил вводят понятия поля сторонних сил и его напряженности \(\mathbf{E}^{*}\). Этот вектор численно равен сторонней силе, действующей на единичный положительный заряд.

Теперь обратимся к плотности тока. Если под действием электрического поля \(\mathbf{E}\) в проводнике возникает ток плотности \(\mathbf{j}=\sigma \mathbf{E}\), то очевидно, что под совместным действием поля \(\mathbf{E}\) и поля сторонних сил \(\mathbf{E}^{*}\) плотность тока


\begin{equation*}
\mathbf{j}=\sigma\left(\mathbf{E}+\mathbf{E}^{*}\right) . \tag{5.11}
\end{equation*}


Это уравнение обобщает закон (5.10) на случай неоднородных участков проводящей среды. Оно выражает обобщенный закон Ома в локальной форме.

Закон Ома для неоднородного участка цепи. Неоднородным называют участок цепи, на котором действуют сторонние силы.

Рассмотрим частный, но практически важный случай, когда электрический ток течет вдоль тонких проводов. В этом случае направление тока будет совпадать с направлением оси провода и плотность тока \(j\) может считаться одинаковой во всех точках сечения провода. Пусть площадь сечения провода равна \(S\), причем \(S\) может быть и не одинаковой по длине провода.

Разделим уравнение (5.11) на \(\sigma\), полученное выражение умножим скалярно на элемент оси провода dl, взятый по направлению от сечения 1 к сечению 2 (его мы примем за положительное), и затем проинтегрируем по длине провода от сечения 1 до сечения 2:


\begin{equation*}
\int_{1}^{2} \frac{\mathbf{j} \mathrm{~d} \mathbf{l}}{\sigma}=\int_{1}^{2} \mathbf{E} \mathrm{~d} \mathbf{l}+\int_{1}^{2} \mathbf{E}^{*} \mathrm{~d} \mathbf{l} \tag{5.12}
\end{equation*}


Преобразуем подынтегральное выражение у первого интеграла: заменим \(\sigma\) на \(1 / \rho\) и \(\mathbf{j} \mathrm{d} \mathbf{l}\) на \(j_{l} \mathrm{~d} l\), где \(j_{l}\) — проекция вектора \(\mathbf{j}\) на направление вектора dl. Далее учтем, что \(j_{l}\) - величина алгебраическая; она зависит от того, как направлен вектор ј по отношению к dl: если \(\mathbf{j} \uparrow \uparrow \mathrm{dl}\), то \(j_{l}>0\), если же \(\mathbf{j} \downarrow \uparrow \mathrm{dl}\), то \(j_{l}<0\). И последнее, заменим \(j_{l}\) на \(I / S\), где \(I\) - сила тока, величина тоже алгебраическая (как и \(j_{l}\) ). Поскольку для постоянного тока \(I\) одинаково во всех сечениях цепи, эту величину можно вынести за знак интеграла. В результате получим


\begin{equation*}
\int_{1}^{2} \frac{\mathrm{j} \mathrm{~d} \mathbf{l}}{\sigma}=I \int_{1}^{2} \rho \frac{\mathrm{~d} l}{S} \tag{5.13}
\end{equation*}


Выражение \(\rho \mathrm{d} l / S\) определяет не что иное, как сопротивление участка цепи длиной \(\mathrm{d} l\), а интеграл от этого выражения полное сопротивление \(R\) участка цепи между сечениями 1 и 2 .

Теперь обратимся к правой части (5.12). Первый интеграл здесь - это разность потенциалов \(\varphi_{1}-\varphi_{2}\), а второй интеграл представляет собой электродвижущую силу (э.д.с.) \(\mathcal{E}\), действу-

ющую на данном участке цепи:


\begin{equation*}
\mathcal{E}_{12}=\int_{1}^{2} \mathbf{E}^{*} \mathrm{~d} l . \tag{5.14}
\end{equation*}


Эта величина, как и сила тока \(I\), является алгебраической: если э.д.с. способствует движению положительных носителей тока в выбранном направлении, то \(\mathcal{E}_{12}>0\), если же препятствует, то \(\mathcal{E}_{12}<0\).

После всех указанных преобразований уравнение (5.12) будет иметь следующий вид:


\begin{equation*}
R I=\varphi_{1}-\varphi_{2}+\mathcal{E}_{12}, \tag{5.15}
\end{equation*}


где, напомним, положительным считается направление от точки 1 к точке 2.

Это уравнение выражает закон Ома для неоднородного участка цепи, в отличие от уравнения (5.11), представляющего тот же закон в локальной форме.

Пример. Рассмотрим участок цепи, показанный на рис. 5.2. Сопротивление отлично от нуля только на отрезке \(R\). На нижней части рисунка представлен ход потенциала φ вдоль данного участка. Выясним, что здесь происходит.

Из того факта, что потенциал на отрезке \(R\) уменьшается слева направо, следует, что \(I>0\), т.е. ток течет в положительном на-

\begin{figure}[h]
\begin{center}
  \includegraphics[alt={},max width=\textwidth]{3e870b25-6320-483d-9133-cdd5212c2707-128_322_366_1016_766}
\captionsetup{labelformat=empty}
\caption{Рис. 5.2}
\end{center}
\end{figure}

правлении (от \(1 к 2\) ). В данном случае \(\varphi_{1}<\varphi_{2}\), но ток течет от точки 1 к точке 2 - в сторону большего значения потенциала. Это возможно лишь потому, что на данном участке имеется э.д.с. \(\mathcal{E}\), действующая в положительном направлении (от 1 к 2).

Вернемся к (5.15). Из этого уравнения следует, что для замкнутой цепи точки 1 и 2 совпадают, \(\varphi_{1}=\varphi_{2}\) и оно приобретает более простой вид:


\begin{equation*}
R I=\mathcal{E}, \tag{5.16}
\end{equation*}


где \(R\) представляет собой уже полное сопротивление замкнутой цепи, а \(\mathcal{E}\) - алгебраическую сумму отдельных э.д.с. в данной цепи.

Далее представим себе участок цепи, содержащий сам источник э.д.с., - между его клеммами 1 и 2. Тогда в уравнении (5.15) для выбранного нами участка \(R\) - это внутреннее сопротивление источника, а \(\varphi_{1}-\varphi_{2}\) - разность потенциалов на его клеммах. Если источник разомкнут, то \(I=0\) и \(\mathcal{E}=\varphi_{2}-\varphi_{1}\), т. е. э.д.с. источника можно определить как разность потенциалов на его клеммах в разомкнутом состоянии.

Разность потенциалов на клеммах данного источника э.д.с., замкнутого на внешнее сопротивление, всегда меньше его э.д.с. Она зависит от внешней нагрузки.\\
Пример. Внешнее сопротивление цепи в \(\eta\) раз больше внутреннего сопротивления источника. Найдем отношение разности потенциалов на клеммах источника к его э.д.с.\\
Пусть \(R_{i}\) - внутреннее сопротивление источника, а \(R_{a}\) внешнее сопротивление цепи. Согласно уравнению (5.15) \(\varphi_{2}-\varphi_{1}=\mathcal{E}-R_{i} I\), согласно же (5.16) \(\left(R_{i}+R_{a}\right) I=\mathcal{E}\). Из этих двух уравнений получим

\[
\frac{\varphi_{2}-\varphi_{1}}{\mathcal{E}}=1-\frac{R_{i} I}{\mathcal{E}}=1-\frac{R_{i}}{R_{i}+R_{a}}=\frac{R_{a}}{R_{i}+R_{a}}=\frac{\eta}{1+\eta} .
\]

Отсюда видно, что чем больше \(\eta\), тем больше приближается разность потенциалов на клеммах источника к его э.д.с., и наоборот.\\
В заключение полезно привести наглядную картину, позволяющую лучше уяснить, что происходит в замкнутой цепи по-

\begin{figure}[h]
\begin{center}
  \includegraphics[alt={},max width=\textwidth]{3e870b25-6320-483d-9133-cdd5212c2707-129_253_292_1371_74}
\captionsetup{labelformat=empty}
\caption{Рис. 5.3}
\end{center}
\end{figure}

стоянного тока. На рис. 5.3 показано распределение потенциала φ вдоль замкнутой цепи, содержащей источник э.д.с. на участке \(A B\). Потенциал φ для наглядности отложен вдоль образующих цилиндрической поверхности, которая опирается на контур с током. Точки \(A\) и \(B\) соответствуют положительной и отрицательной клеммам источника Из рисунка видно, что процесс протекания тока можно представить себе так: положительные заря-ды-носители «соскальзывают» по наклонному «желобу» от точ-

ки \(\varphi_{A}\) к точке \(\varphi_{B}\) - по внешнему участку цепи, внутри же источника «подняться» от точки \(\varphi_{B}\) к точке \(\varphi_{A}\) им помогают сторонние силы, обозначенные стрелкой.

\section*{§ 5.4. Разветвленные цепи. Правила Кирхгофа}
Расчет разветвленных цепей, например нахождение токов в отдельных ее ветвях, значительно упрощается, если пользоваться двумя правилами Кирхгофа.

Первое правило Кирхгофа - оно относится к узлам цепи, т. е. к точкам ее разветвления: алгебраическая сумма токов, сходящихся в узле, равна нулю:


\begin{equation*}
\sum I_{k}=0 . \tag{5.17}
\end{equation*}


При этом токи, идущие к узлу, и токи, исходящие из узла, следует считать величинами разных знаков, например: первые - положительными, вторые - отрицательными (или наоборот - это не существенно). Применительно к рис. 5.4 уравнение (5.17) запишется так:

\begin{figure}[h]
\begin{center}
  \includegraphics[alt={},max width=\textwidth]{3e870b25-6320-483d-9133-cdd5212c2707-130_239_253_768_875}
\captionsetup{labelformat=empty}
\caption{Рис. 5.4}
\end{center}
\end{figure}

\[
I_{1}-I_{2}+I_{3}=0 .
\]

Уравнение (5.17) является следствием условия стационарности (5.5); если бы это было не так, в узле изменялся бы заряд и токи не были бы стационарными.

Второе правило Кирхгофа - оно относится к любому выделенному в разветвленной цепи замкнутому контуру: алгебраическая сумма произведений сил токов в отдельных участках произвольного замкнутого контура на их сопротивления равна алгебраической сумме э.д.с., действующих в этом контуре:


\begin{equation*}
\sum I_{k} R_{k}=\sum \mathcal{E}_{k} . \tag{5.18}
\end{equation*}


Для доказательства этого правила достаточно рассмотреть случай, когда выделенный контур состоит из трех участков (рис. 5.5). Зададим направление обхода, например, по часовой стрелке, как показано на рисунке. Затем применим к каждому

\begin{figure}[h]
\begin{center}
  \includegraphics[alt={},max width=\textwidth]{3e870b25-6320-483d-9133-cdd5212c2707-131_281_322_145_100}
\captionsetup{labelformat=empty}
\caption{Рис. 5.5}
\end{center}
\end{figure}

из трех участков закон Ома (5.15):

\[
\begin{aligned}
& I_{1} R_{1}=\varphi_{2}-\varphi_{3}+\mathcal{E}_{1}, \\
& I_{2} R_{2}=\varphi_{3}-\varphi_{1}+\mathcal{E}_{2}, \\
& I_{3} R_{3}=\varphi_{1}-\varphi_{2}+\mathcal{E}_{3} .
\end{aligned}
\]

Сложив эти равенства, приходим после сокращения всех потенциалов к формуле (5.18), т. е. ко второму правилу Кирхгофа.\\
Таким образом, уравнение (5.18) является следствием закона Ома для неоднородных участков цепи.

Составление системы уравнений. Правила Кирхгофа в каждом конкретном случае позволяют написать полную систему алгебраических уравнений, из которой могут быть найдены, например, все неизвестные токи.

Уравнений (5.17) и (5.18) надо составлять столько, чтобы их число было равно числу искомых величин. При этом надо следить, чтобы одни уравнения не являлись следствием других:

\begin{enumerate}
  \item если в разветвленной цепи имеется \(N\) узлов, то независимые уравнения типа (5.17) можно составить лишь для \(N-1\) узлов; уравнение для последнего узла будет следствием предыдущих;
  \item если в разветвленной цепи можно выделить несколько замкнутых контуров, то независимые уравнения типа (5.18) можно составить только для тех контуров, которые не получаются в результате наложения уже рассмотренных. Например, для цепи (рис. 5.6) такие уравнения для контуров 124 и 234 будут независимыми. Уравнение же для контура 1234 является
\end{enumerate}

\begin{figure}[h]
\begin{center}
  \includegraphics[alt={},max width=\textwidth]{3e870b25-6320-483d-9133-cdd5212c2707-131_253_360_1371_88}
\captionsetup{labelformat=empty}
\caption{Рис. 5.6}
\end{center}
\end{figure}

следствием двух предыдущих. Можно составить независимые уравнения для двух других контуров, например для контуров 124 и 1234 , но тогда уравнение для контура 234 будет следствием двух первых. Число независимых уравнений типа (5.18) оказывается равным наименьшему числу разрывов, которые следует сделать в цепи, чтобы нарушить все контуры. Это число, кстати, равно числу областей, ограниченных про-

водниками, если схему удастся изобразить на плоскости без пересечений.

Например, для цепи (рис. 5.7), содержащей четыре узла, надо составить три уравнения типа (5.17) и три уравнения типа (5.18), ибо минимальное число разрывов (они помечены крестиками), нарушающее все контуры, равно трем (трем равно и число областей). Если неизвестными являются токи, то их число равно шести - по числу отдельных участков между узлами, что соответствует числу независимых уравнений.

При составлении уравнений типа (5.17) и

\begin{figure}[h]
\begin{center}
  \includegraphics[alt={},max width=\textwidth]{3e870b25-6320-483d-9133-cdd5212c2707-132_217_317_356_843}
\captionsetup{labelformat=empty}
\caption{Рис. 5.7}
\end{center}
\end{figure}

(5.18) необходимо поступать так.

\begin{enumerate}
  \item Обозначить стрелками предположительные направления токов, не задумываясь над тем, куда эти стрелки направить. Если в результате вычисления окажется, что такой-то ток положителен, то это значит, что его направление выбрано правильно. Если же ток окажется отрицательным, то его истинное направление противоположно направлению стрелки.
  \item Выбрав произвольно замкнутый контур, все его участки следует обойти в одном направлении, например по часовой стрелке. Если предположительное направление некоторого тока совпадает с выбранным направлением обхода, то соответствующее слагаемое \(I R\) в уравнении (5.18) надо брать со знаком плюс, если же эти направления противоположны, то со знаком минус. Аналогично следует поступать и с Е: если какая-то э.д.с. E повышает потенциал в направлении обхода, ее надо брать со знаком плюс, в противном случае - со знаком минус.
\end{enumerate}

Пример. Найдем силу тока и его направление через сопротивление \(R\) в схеме (рис. 5.8). Все сопротивления и э.д.с. предполагаются известными.

Здесь три участка, следовательно, три неизвестных тока \(I, I_{1}\) и \(I_{2}\). Обозначим стрелками (не задумываясь) их предположительные направления (у правого узла).

\begin{figure}[h]
\begin{center}
  \includegraphics[alt={},max width=\textwidth]{3e870b25-6320-483d-9133-cdd5212c2707-132_390_279_1375_857}
\captionsetup{labelformat=empty}
\caption{Рис. 5.8}
\end{center}
\end{figure}

Цепь содержит \(N=2\) узла. Значит, независимых уравнений типа (5.17) только одно:

\[
I+I_{1}+I_{2}=0 .
\]

Теперь составим уравнения типа (5.18) - их должно быть два (по числу областей). Возьмем контур, содержащий \(R\) и \(R_{1}\), и контур с \(R\) и \(R_{2}\). Выбрав направление обхода каждого контура по часовой стрелке, запишем

\[
-I R+I_{1} R_{1}=-\mathcal{E}_{1}, \quad-I R+I_{2} R_{2}=-\mathcal{E}_{2} .
\]

Полезно убедиться, что соответствующее уравнение для контура, содержащего \(R_{1}\) и \(R_{2}\), является следствием этих двух. Решив систему написанных трех уравнений, получим

\[
I=\frac{-R_{1} \mathcal{E}_{2}+R_{2} \mathcal{E}_{1}}{R_{1} R_{2}+R R_{1}+R R_{2}}
\]

Если после подстановки числовых значений окажется, что \(I>0\), то это значит, что в действительности ток течет так, как мы предположили на рис. 5.8, если же \(I<0\), то в противоположном направлении.

\section*{§ 5.5. Закон Джоуля-Ленца}
С прохождением тока через проводник, обладающий сопротивлением, неразрывно связано выделение теплоты (нагревание проводников). Наша задача - найти количество теплоты, выделяющееся за единицу времени на определенном участке цепи. Здесь возможны два случая, которые мы и рассмотрим последовательно, - однородный и неоднородный участки цепи. В основу решения этого вопроса мы возьмем закон сохранения энергии и закон Ома.

Однородный участок цепи. Пусть интересующий нас участок заключен между сечениями 1 и 2 проводника (рис. 5.9). Найдем работу, которую совершают силы поля над носителями тока на участке 12 за время \(\mathrm{d} t\).

\begin{figure}[h]
\begin{center}
  \includegraphics[alt={},max width=\textwidth]{3e870b25-6320-483d-9133-cdd5212c2707-133_180_400_1585_74}
\captionsetup{labelformat=empty}
\caption{Рис. 5.9}
\end{center}
\end{figure}

Если сила тока в проводнике равна \(I\), то за время \(\mathrm{d} t\) через каждое сечение проводника пройдет заряд \(\mathrm{d} q=I \mathrm{~d} t\). В частности, такой заряд \(\mathrm{d} q\) войдет внутрь участка через сечение 1 и такой же заряд выйдет из этого участка

через сечение 2. Так как распределение зарядов в проводнике остается при этом неизменным (ток постоянный), то весь процесс эквивалентен непосредственному переносу заряда \(\mathrm{d} q\) от сечения \(1 к\) сечению 2 , имеющих потенциалы \(\varphi_{1}\) и \(\varphi_{2}\).

Поэтому совершаемая при таком переносе работа сил поля

\[
\delta A=\mathrm{d} q\left(\varphi_{1}-\varphi_{2}\right)=I\left(\varphi_{1}-\varphi_{2}\right) \mathrm{d} t .
\]

Согласно закону сохранения энергии эквивалентная этой работе энергия должна выделяться в иной форме. Если проводник неподвижен и в нем не происходят химические превращения, то эта энергия должна выделяться в форме внутренней (тепловой) энергии, в результате чего проводник нагревается. Механизм этого превращения достаточно прост: носители тока (например, электроны в металлах) в результате работы сил поля приобретают дополнительную кинетическую энергию и затем расходуют ее на возбуждение колебаний решетки при столкновении с ее узлами-атомами.

Итак, согласно закону сохранения энергии элементарная работа \(\delta A=\dot{Q} \mathrm{~d} t\), где \(\dot{Q}\) — теплота, выделяемая в единицу времени (тепловая мощность). Из сравнения последнего равенства с предыдущим получаем

\[
\dot{Q}=I\left(\varphi_{1}-\varphi_{2}\right) .
\]

А так как по закону Ома \(\varphi_{1}-\varphi_{2}=R I\), то


\begin{equation*}
\dot{Q}=R I^{2} . \tag{5.19}
\end{equation*}


Эта формула выражает известный закон Джоуля-Ленца.\\
Получим выражение этого закона в локальной форме, характеризующей выделение теплоты в различных местах проводящей среды. Для этой цели выделим в данной среде элементарный объем в виде цилиндрика с образующими, параллельными вектору j - плотности тока в данном месте. Пусть поперечное сечение цилиндрика \(\mathrm{d} S\), а его длина \(\mathrm{d} l\). Тогда на основании закона Джоуля-Ленца в этом объеме за время \(\mathrm{d} t\) выделяется количество теплоты

\[
\delta Q=R I^{2} \mathrm{~d} t=\frac{\rho \mathrm{d} l}{\mathrm{~d} S}(j \mathrm{~d} S)^{2} \mathrm{~d} t=\rho j^{2} d V \mathrm{~d} t,
\]

где \(\mathrm{d} V=\mathrm{d} S \mathrm{~d} l\) - объем цилиндрика. Разделив последнее уравнение на \(\mathrm{d} V \mathrm{~d} t\), получим формулу, которая определяет количество теплоты, выделяющееся за единицу времени в единице объема проводящей среды, - удельную тепловую мощность тока:


\begin{equation*}
\dot{Q}_{\text {уд }}=\rho j^{2} . \tag{5.20}
\end{equation*}


Эта формула выражает закон Джоуля-Ленца в локальной форме: удельная тепловая мощность тока пропорциональна квадрату плотности электрического тока и удельному сопротивлению средь в данной точке.

Уравнение (5.20) представляет собой наиболее общую форму закона Джоуля-Ленца, применимую к любым проводникам вне зависимости от их формы, однородности и от природы сил, возбуждающих электрический ток. Если на носители тока действуют только электрические силы, то на основании закона Ома (5.10)


\begin{equation*}
\dot{Q}_{\text {уд }}=\mathbf{j} \mathbf{E}=\sigma E^{2} . \tag{5.21}
\end{equation*}


Таким образом, последнее уравнение имеет менее общий характер, нежели (5.20).

Неоднородный участок цепи. Если участок цепи содержит источник э.д.с., то на носители тока будут действовать не только электрические силы, но и сторонние. В этом случае выделяемое в неподвижном проводнике тепло будет равно по закону сохранения энергии алгебраической сумме работ электрических и сторонних сил. Это же относится и к соответствующим мощностям: тепловая мощность должна быть равна алгебраической сумме мощностей электрических и сторонних сил. Проще всего в этом можно убедиться, умножив выражение (5.15) на \(I\) :


\begin{equation*}
R I^{2}=\left(\varphi_{1}-\varphi_{2}\right) I+\mathcal{E} I . \tag{5.22}
\end{equation*}


Здесь слева стоит выделяющаяся на участке тепловая мощность \(\dot{Q}\); при наличии сторонних сил величина \(\dot{Q}\) определяется той же формулой (5.19), что и для однородного участка цепи. Последнее же слагаемое справа представляет собой мощность, развиваемую сторонними силами на данном участке. Заметим еще,

что последняя величина ( \(\mathcal{E} I\) ) является алгебраической: в отличие от \(R I^{2}\) она изменяет знак при изменении направления тока \(I\).

Таким образом, уравнение (5.22) означает, что тепловая мощность, выделяемая на участке цепи между точками 1 и 2 , равна алгебраической сумме мощностей электрических и сторонних сил. Сумму этих мощностей, т. е. правую часть (5.22), называют мощностью тока на рассматриваемом участке цепи. Тогда можно сказать, что в случае неподвижного участка цепи мощность выделяемой на этом участке теплоты равна мощности тока.

Применив (5.22) ко всей неразветвленной цепи (тогда \(\varphi_{1}=\varphi_{2}\) ), получим


\begin{equation*}
\dot{Q}=\mathcal{E} I, \tag{5.23}
\end{equation*}


т. е. общее количество выделяемой за единицу времени во всей цепи джоулевой теплоты равно мощности только сторонних сил. Значит, теплота производится только сторонними силами. Роль же электрического поля сводится к тому, что оно перераспределяет эту теплоту по различным участкам цепи.

Получим теперь уравнение (5.22) в локальной форме. Для этого умножим обе части уравнения (5.11) на \(\mathbf{j}\), а также учтем, что \(\sigma=1 / \rho\) и \(\rho j^{2}=\dot{Q}_{\text {уд }}\) [см. (5.20)]. Тогда удельная тепловая мощность тока в неоднородной проводящей среде


\begin{equation*}
\dot{Q}_{\text {уд }}=\rho j^{2}=\mathbf{j}\left(\mathbf{E}+\mathbf{E}^{*}\right) \tag{5.24}
\end{equation*}


\section*{§ 5.6. Переходные процессы в цепи с конденсатором}
О переходных процессах. Так называют процессы при переходе от одного установившегося в цепи режима к другому. Примером таких процессов является зарядка и разрядка конденсатора, на них мы и остановимся более подробно в этом параграфе.

До сих пор мы рассматривали только постоянные токи. Оказывается, однако, что полученные законы во многих случаях можно применять и к изменяющимся токам. Это касается всех тех случаев, когда изменение тока происходит не слишком быстро. В этих случаях мгновенное значение тока будет практиче-

ски одно и то же во всех поперечных сечениях цепи. Такие токи и соответствующие им поля называют квазистационарньмми (более точный критерий квазистационарности дан в § 11.1).

Именно квазистационарные токи можно описывать законами постоянного тока, если только их применять к мгновенным значениям величин.

А теперь обратимся к процессам разрядки и зарядки конденсатора, предполагая токи в этих процессах квазистационарными.

Разрядка конденсатора. Если обкладки заряженного конденсатора емкости \(C\) замкнуть через сопротивление \(R\), то через него потечет ток. Пусть \(I, q, U\) - мгновенные значения тока,

\begin{figure}[h]
\begin{center}
  \includegraphics[alt={},max width=\textwidth]{3e870b25-6320-483d-9133-cdd5212c2707-137_253_322_689_100}
\captionsetup{labelformat=empty}
\caption{Рис. 5.10}
\end{center}
\end{figure}

заряда положительной обкладки и разности потенциалов между обкладками (напряжения). Считая ток \(I\) положительным, когда он течет от положительной обкладки к отрицательной (рис. 5.10), запишем \(I=-\mathrm{d} q / \mathrm{d} t\). Согласно закону Ома для внешнего участка цепи, содержащего сопротивление \(R\) :

\[
R I=U .
\]

Учитывая, что \(I=-\mathrm{d} q / \mathrm{d} t\) и \(U=q / C\), преобразуем предыдущее уравнение к виду


\begin{equation*}
\frac{\mathrm{d} q}{\mathrm{~d} t}+\frac{q}{R C}=0 \tag{5.25}
\end{equation*}


В этом дифференциальном уравнении переменные разделяются, и после интегрирования мы получим


\begin{equation*}
q=q_{0} \mathrm{e}^{-t / \tau}, \tag{5.26}
\end{equation*}


где \(q_{0}\) - начальный заряд конденсатора, а \(\tau\) - постоянная, имеющая размерность времени:


\begin{equation*}
\tau=R C \tag{5.27}
\end{equation*}


Эту постоянную называют временем релаксации. Из (5.26) видно, что \(\tau\) есть время, за которое заряд конденсатора уменьшается в е раз.

Продифференцировав (5.26) по времени, найдем закон изменения тока:


\begin{equation*}
I=-\frac{\mathrm{d} q}{\mathrm{~d} t}=I_{0} \mathrm{e}^{-t / \tau}, \tag{5.28}
\end{equation*}


где \(I_{0}=q_{0} / \tau\) - сила тока в момент \(t=0\).\\
На рис. 5.11 показан график зависимости \(q(t)\) - заряда на конденсаторе от времени. График зависимости \(I(t)\) имеет такой же вид.

Зарядка конденсатора. Рассмотрим цепь, содержащую последовательно соединенные конденсатор \(C\), сопротивление \(R\) и источник э.д.с. \(\mathcal{E}\) (рис. 5.12). Первоначально конденсатор не заряжен (ключ \(K\) разомкнут). В момент \(t=0\) ключ замкнули, и в цепи пошел ток, заряжающий конденсатор. Увеличивающиеся заряды на обкладках конденсатора будут все в большей степени препятствовать прохождению тока, постепенно уменьшая его.

\begin{figure}[h]
\begin{center}
  \includegraphics[alt={},max width=\textwidth]{3e870b25-6320-483d-9133-cdd5212c2707-138_237_275_314_873}
\captionsetup{labelformat=empty}
\caption{Рис. 5.11}
\end{center}
\end{figure}

\begin{figure}[h]
\begin{center}
  \includegraphics[alt={},max width=\textwidth]{3e870b25-6320-483d-9133-cdd5212c2707-138_227_293_635_867}
\captionsetup{labelformat=empty}
\caption{Рис. 5.12}
\end{center}
\end{figure}

Теперь ток в цепи будем считать положительным, когда он течет в направлении к положительно заряженной обкладке конденсатора: \(I=\mathrm{d} q / \mathrm{d} t\). Применим закон Ома для неоднородного участка цепи к участку \(1 \mathrm{E} R 2\) :

\[
R I=\varphi_{1}-\varphi_{2}+\mathcal{E},
\]

где под \(R\) понимается полное сопротивление этого участка, включая внутреннее сопротивление источника э.д.с. Учитывая, что \(I=\mathrm{d} q / \mathrm{d} t\) и \(\varphi_{2}-\varphi_{1}=U=q / C\), перепишем предыдущее уравнение в виде

\[
\frac{\mathrm{d} q}{\mathrm{~d} t}=\frac{\mathcal{E}-q / C}{R} .
\]

Разделение переменных дает

\[
\frac{R \mathrm{~d} q}{\mathcal{E}-q / C}=\mathrm{d} t .
\]

Проинтегрировав это уравнение с учетом начального условия ( \(q=0\) при \(t=0\) ), получим

\[
R C \ln \left(1-\frac{q}{\mathcal{E} C}\right)=-t,
\]

откуда


\begin{equation*}
q=q_{m}\left(1-\mathrm{e}^{-t / \tau}\right) . \tag{5.29}
\end{equation*}


Здесь \(q_{m}=\mathcal{E} C\) - предельное значение заряда на конденсаторе (при \(t \rightarrow \infty\) ), \(\tau=R C\). Закон изменения тока со временем


\begin{equation*}
I=\frac{\mathrm{d} q}{\mathrm{~d} t}=I_{0} \mathrm{e}^{-t / \tau}, \tag{5.30}
\end{equation*}


\begin{figure}[h]
\begin{center}
  \includegraphics[alt={},max width=\textwidth]{3e870b25-6320-483d-9133-cdd5212c2707-139_253_316_512_98}
\captionsetup{labelformat=empty}
\caption{Рис. 5.13}
\end{center}
\end{figure}

\[
\text { где } I_{0}=\mathcal{E} / R \text {. }
\]

Графики зависимостей \(q(t)\) и \(I(t)\) показаны на рис. 5.13.

\section*{Задачи}
5.1. Сопротивление проводящей среды. Металлический шар радиусом \(a\) окружен концентрической тонкой металлической оболочкой радиусом \(b\). Пространство между этими электродами заполнено однородной слабо проводящей средой с удельным сопротивлением \(\rho\). Найти сопротивление межэлектродного промежутка.

Решение. Выделим мысленно тонкий сферический слой между радиусами \(r\) и \(r+\mathrm{d} r\). Линии тока во всех точках этого слоя идут перпендикулярно ему, поэтому такой слой можно рассматривать как цилиндрический проводник длиной \(\mathrm{d} r\) с площадью поперечного сечения \(4 \pi r^{2}\). Воспользовавшись формулой (5.9), запишем

\[
\mathrm{d} R=\rho \frac{\mathrm{d} r}{4 \pi r^{2}} .
\]

Проинтегрировав это выражение по \(r\) от \(a\) до \(b\), получим

\[
R=\frac{\rho}{4 \pi}\left(\frac{1}{a}-\frac{1}{b}\right) .
\]

5.2. Два металлических шарика одинакового радиуса \(a\) находятся в однородной слабо проводящей среде с удельным сопротивлением \(\rho\). Найти сопротивление среды между шариками при условии, что расстояние между шариками значительно больше их размеров.

Решение. Мысленно зарядим шарики \(+q\) и \(-q\). Поскольку шарики находятся далеко друг от друга, электрическое поле вблизи поверхности каждого из них определяется практически только зарядом самого шарика, причем его заряд можно считать распределенным равномерно по поверхности. Окружив шарик с положительным зарядом концентрической сферой, непосредственно прилегающей к его поверхности, запишем выражение для тока, протекающего через эту сферу:

\[
I=4 \pi a^{2} j,
\]

где \(j\) - плотность тока. Воспользовавшись законом Ома ( \(j=E / \rho\) ) и формулой \(E=q / 4 \pi \varepsilon_{0} a^{2}\), получим

\[
I=q / \varepsilon_{0} \rho .
\]

Теперь найдем разность потенциалов между шариками:

\[
U=\varphi_{+}-\varphi_{-} \approx 2 q / 4 \pi \varepsilon_{0} a .
\]

Искомое сопротивление

\[
R=U / I=\rho / 2 \pi a .
\]

Этот результат справедлив независимо от значения диэлектрической проницаемости среды.\\
5.3. Два проводника произвольной формы находятся в безграничной однородной слабо проводящей среде с удельным сопротивлением \(\rho\) и диэлектрической проницаемостью \(\varepsilon\). Найти значение произведения \(R C\) для данной системы, где \(R\) - сопротивление среды между проводниками, \(C\) - взаимная емкость проводников при наличии среды.\\
Решение. Зарядим мысленно проводники зарядами \(+q\) и \(-q\). Так как среда между ними слабо проводящая, то поверхности проводников являются эквипотенциальными и конфигурация поля такова же, как и при отсутствии среды.

Окружим, например, положительно заряженный проводник замкнутой поверхностью \(S\), непосредственно прилегающей к поверхности проводника, и вычислим отдельно \(R\) и \(C\) :

\[
\begin{gathered}
R=\frac{U}{I}=\frac{U}{\oint j_{n} \mathrm{~d} S}=\frac{U}{\sigma \oint E_{n} \mathrm{~d} S} \\
C=\frac{q}{U}=\frac{\oint D_{n} \mathrm{~d} S}{U}=\frac{\varepsilon \varepsilon_{0} \oint E_{n} \mathrm{~d} S}{U}
\end{gathered}
\]

где интегралы взяты по данной поверхности \(S\). При вычислении \(R\) был использован закон Ома \(\mathbf{j}=\sigma \mathbf{E}\), а при вычислении \(C\) - теорема Гаусса.

Произведение полученных выражений

\[
R C=\varepsilon_{0} \varepsilon / \sigma=\varepsilon_{0} \varepsilon \rho
\]

5.4. Условия на границе проводника. Проводник с удельным сопротивлением \(\rho\) граничит с диэлектриком, проницаемость которого \(\varepsilon\). В некоторой точке \(A\) у поверхности проводника электрическая индукция равна \(D\), причем вектор \(\mathbf{D}\) направлен от проводника и составляет угол \(\alpha\) с нормалью к поверхности. Найти поверхностную плотность зарядов на проводнике и плотность тока вблизи точки \(A\).

Решение. Поверхностная плотность зарядов на проводнике

\[
\sigma=D_{n}=D \cos \alpha .
\]

Плотность тока найдем по закону Ома: \(\mathbf{j}=\mathbf{E} / \rho\). Из уравнения непрерывности (5.5) следует, что нормальные составляющие вектора \(\mathbf{j}\) равны, а так как в диэлектрике \(j_{n}=0\) (тока нет), то и в проводнике \(j_{n}=0\). Стало быть, вектор \(\mathbf{j}\) в проводнике касателен его поверхности. Это же относится и к вектору \(\mathbf{E}\) внутри проводника.\\
С другой стороны, из теоремы о циркуляции вектора \(\mathbf{E}\) следует, что тангенциальные составляющие его по разные стороны границы раздела одинаковы, а значит, \(E=E_{\tau}=\left(D / \varepsilon \varepsilon_{0}\right) \sin \alpha\), где \(E_{\tau}\) тангенциальная составляющая поля \(\mathbf{E}\) в диэлектрике.\\
Учитывая все это, получим

\[
j=\frac{E}{\rho}=\frac{D \sin \alpha}{\varepsilon_{0} \varepsilon \rho} .
\]

5.5. Зазор между обкладками плоского конденсатора заполнен последовательно двумя диэлектрическими слоями 1 и 2 толщиной \(l_{1}\) и \(l_{2}\) с проницаемостями \(\varepsilon_{1}\) и \(\varepsilon_{2}\) и удельными сопротивлениями \(\rho_{1}\) и \(\rho_{2}\). Конденсатор находится под постоянным напряжением \(U\), причем электрическое поле направлено от слоя 1 к слою 2 . Найти поверхностную плотность сторонних зарядов на границе раздела диэлектрических слоев.\\
Решение. Искомая поверхностная плотность зарядов


\begin{equation*}
\sigma=D_{2 n}-D_{1 n}=\varepsilon_{0} \varepsilon_{2} E_{2}-\varepsilon_{0} \varepsilon_{1} E_{1} . \tag{1}
\end{equation*}


Для определения \(E_{1}\) и \(E_{2}\) воспользуемся двумя условиями: из того факта, что \(j_{1}=j_{2}\), следует \(E_{1} / \rho_{1}=E_{2} / \rho_{2}\) и, кроме того, \(E_{1} l_{1}+E_{2} l_{2}=U\). Решив два последних уравнения, найдем \(E_{1}\) и \(E_{2}\). Их подстановка в (1) приводит к следующему результату:

\[
\sigma=\frac{\varepsilon_{2} \rho_{2}-\varepsilon_{1} \rho_{1}}{\rho_{1} l_{1}+\rho_{2} l_{2}} \varepsilon_{0} U
\]

Отсюда видно, что \(\sigma=0\) при \(\varepsilon_{1} \rho_{1}=\varepsilon_{2} \rho_{2}\).\\
5.6. Неоднородный проводник. Длинный проводник круглого сечения площадью \(S\) сделан из материала, удельное сопротивление которого зависит только от расстояния \(r\) до оси проводника как \(\rho=\alpha / r^{2}\), где \(\alpha\) - постоянная. По проводнику течет ток \(I\). Найти:

\begin{enumerate}
  \item напряженность \(E\) поля в проводнике; 2) сопротивление единицы длины проводника.
\end{enumerate}

Решение. 1. Напряженность \(E\) поля по закону Ома связана с плотностью тока \(j\), а \(j\) - с током \(I\), поэтому можно записать

\[
I=\int j 2 \pi r \mathrm{~d} r=\int(E / \rho) 2 \pi r \mathrm{~d} r
\]

Напряженность \(E\) одинакова во всех точках сечения данного проводника, т. е. не зависит от \(r\). В этом легко убедиться, взяв прямоугольный контур внутри проводника так, чтобы одна сторона контура совпадала, например, с осью проводника, и затем применив к этому контуру теорему о циркуляции вектора Е.

Таким образом, \(E\) можно вынести из-под интеграла и мы получим в результате интегрирования

\[
E=2 \pi \alpha I / S^{2} .
\]

\begin{enumerate}
  \setcounter{enumi}{1}
  \item Сопротивление единицы длины проводника можно определить с помощью формулы \(R=U / I\). Поделив обе части этого равенства на длину \(l\) участка проводника, к которому относятся \(R\) и \(U\), найдем
\end{enumerate}

\[
R_{\mathrm{ed}}=E / I=2 \pi \alpha / S^{2} .
\]

5.7. Закон Ома для неоднородного участка цепи. В схеме (рис. 5.14) известны э.д.с. \(\mathcal{E}\) и \(\mathcal{E}_{0}\) источников, сопротивления \(R\) и \(R_{0}\), а также емкость \(C\) конденсатора. Внутренние сопротивления источников пренебрежимо малы. Найти заряд на обкладке 1 конденсатора.

\begin{figure}[h]
\begin{center}
  \includegraphics[alt={},max width=\textwidth]{3e870b25-6320-483d-9133-cdd5212c2707-142_257_342_1516_774}
\captionsetup{labelformat=empty}
\caption{Рис. 5.14}
\end{center}
\end{figure}

Решение. В соответствии с законом Ома для замкнутой цепи, содержащей сопротивления \(R\) и \(R_{0}\), запишем

\[
\left(R+R_{0}\right) I=\mathcal{E}-\mathcal{E}_{0}
\]

где положительное направление выбрано по часовой стрелке. С другой стороны, для неоднородного участка \(a R b\) цепи

\[
R I=\varphi_{a}-\varphi_{b}+\mathcal{E},
\]

а для участка \(a C b\)

\[
\mathcal{E}+\varphi_{2}-\varphi_{1}=\varphi_{b}-\varphi_{a}
\]

Решив совместно эти три уравнения, получим

\[
\varphi_{1}-\varphi_{2}=\frac{R}{R+R_{0}}\left(\mathcal{E}-\mathcal{E}_{0}\right)
\]

Заряд на обкладке 1 определяется формулой \(q_{1}=C\left(\varphi_{1}-\varphi_{2}\right)\). Поэтому окончательный результат

\[
q_{1}=\frac{R C}{R+R_{0}}\left(\mathcal{E}-\mathcal{E}_{0}\right) .
\]

Видно, что при \(\mathcal{E}>\mathcal{E}_{0}\) заряд \(q_{1}>0\), и наоборот.\\
5.8. Работа источника э.д.с. Стеклянная пластина целиком заполняет зазор между обкладками плоского конденсатора, емкость которого при отсутствии пластины равна \(C_{0}\). Конденсатор подключен к источнику постоянного напряжения \(U\). Найти механическую работу, которую необходимо совершить против электрических сил, чтобы извлечь пластину из конденсатора.\\
Решение. Согласно закону сохранения энергии


\begin{equation*}
A_{\text {мех }}+A_{\text {ист }}=\Delta W, \tag{1}
\end{equation*}


где \(A_{\text {мех }}\) - совершенная внешними силами механическая работа против электрических сил; \(A_{\text {ист }}\) - работа источника в этом процессе; \(\Delta W\) - соответствующее приращение энергии конденсатора (мы считаем, что участие других видов энергии в изменении энергии системы пренебрежимо мало).

Найдем \(\Delta W\) и \(A_{\text {ист }}\). Из формулы для энергии конденсатора ( \(W= =C U^{2} / 2=q U / 2\) ) следует, что при \(U=\mathrm{const}\)


\begin{equation*}
\Delta W=\Delta C U^{2} / 2=\Delta q U / 2 . \tag{2}
\end{equation*}


Так как емкость конденсатора при извлечении пластины уменьшается ( \(\Delta C<0\) ), то уменьшается и заряд конденсатора ( \(\Delta q<0\) ). Последнее означает, что заряд прошел через источник против направления действия сторонних сил и источник совершил отрицательную работу:


\begin{equation*}
A_{\text {ист }}=\Delta q U . \tag{3}
\end{equation*}


Из сравнения формул (3) и (2) следует

\[
A_{\text {ист }}=2 \Delta W .
\]

После подстановки последнего выражения в (1) получим

\[
A_{\mathrm{mex}}=-\Delta W, \text { или } A_{\mathrm{mex}}=(\varepsilon-1) C_{0} U^{2} / 2 .
\]

Таким образом, извлекая пластину из конденсатора, мы (внешние силы) совершаем положительную работу (против электрических сил), при этом источник э.д.с. совершает отрицательную работу и энергия конденсатора уменьшается:

\[
A_{\text {мех }}>0, \quad A_{\text {ист }}<0, \quad \Delta W<0 .
\]

5.9. Переходные процессы. Цепь состоит из источника постоянной э.д.с. Е и последовательно подключенных к нему сопротивления \(R\) и конденсатора \(C\). Внутреннее сопротивление источника пренебрежимо мало. В момент \(t=0\) емкость конденсатора быстро (скачком) уменьшили в η раз. Найти ток в цепи как функцию времени.

Решение. Запишем закон Ома для неоднородного участка цепи \(1 \mathcal{E} R 2\) (рис. 5.15):

\begin{figure}[h]
\begin{center}
  \includegraphics[alt={},max width=\textwidth]{3e870b25-6320-483d-9133-cdd5212c2707-144_222_283_986_836}
\captionsetup{labelformat=empty}
\caption{Рис. 5.15}
\end{center}
\end{figure}

\[
R I=\varphi_{1}-\varphi_{2}-\mathcal{E}=U-\mathcal{E} .
\]

Учтем, что \(U=q / C^{\prime}\), где \(C^{\prime}=C / \eta\), тогда


\begin{equation*}
R I=\eta q / C-\mathcal{E} . \tag{1}
\end{equation*}


Продифференцируем это равенство по времени, принимая во внимание, что в нашем случае ( \(q\) уменьшается) \(\mathrm{d} q / \mathrm{d} t=-I\) :

\[
R \frac{\mathrm{~d} I}{\mathrm{~d} t}=-\frac{\eta}{C} I, \quad \frac{\mathrm{~d} I}{I}=-\frac{\eta}{R C} \mathrm{~d} t .
\]

Интегрирование последнего уравнения дает

\[
\ln \frac{I}{I_{0}}=-\frac{\eta t}{R C}, \quad I=I_{0} \mathrm{e}^{-\eta t / R C}
\]

где \(I_{0}\) определяется условием (1). Действительно,

\[
R I_{0}=\eta q_{0} / C-\mathcal{E},
\]

причем \(q_{0}=\mathcal{E} C\) - заряд конденсатора до изменения его емкости. Поэтому

\[
I_{0}=(\eta-1) \mathcal{E} / R .
\]

5.10. Конденсатору емкостью \(C\) сообщили заряд \(q_{0}\) и затем в момент \(t=0\) его замкнули на сопротивление \(R\). Найти зависимость от времени \(t\) количества теплоты, выделившегося на сопротивлении.

Решение. Искомое количество теплоты


\begin{equation*}
Q=\int_{0}^{t} R I^{2} \mathrm{~d} t \tag{1}
\end{equation*}


\begin{figure}[h]
\begin{center}
  \includegraphics[alt={},max width=\textwidth]{3e870b25-6320-483d-9133-cdd5212c2707-145_210_354_764_143}
\captionsetup{labelformat=empty}
\caption{Рис. 5.16}
\end{center}
\end{figure}

откуда видно, что прежде всего надо найти зависимость \(I(t)\). Воспользуемся с этой целью законом Ома для участка цепи \(1 R 2\) (рис. 5.16):

\[
R I=\varphi_{1}-\varphi_{2}=U,
\]

или


\begin{equation*}
R I=q / C . \tag{2}
\end{equation*}


Продифференцируем (2) по времени:

\[
R \frac{\mathrm{~d} I}{\mathrm{~d} t}=\frac{1}{C} I, \quad \frac{\mathrm{~d} I}{I}=\frac{\mathrm{d} t}{R C} .
\]

Проинтегрировав последнее уравнение, получим


\begin{equation*}
\ln \frac{I}{I_{0}}=\frac{t}{R C}, \quad I=I_{0} \mathrm{e}^{-t / R C} \tag{3}
\end{equation*}


где \(I_{0}\) определяется условием (2) при \(q=q_{0}\), т. е. \(I_{0}=q_{0} / R C\). После подстановки (3) в (1) и соответствующего интегрирования получим

\[
Q=\frac{q_{0}^{2}}{2 C}\left(1-\mathrm{e}^{-2 t / R C}\right)
\]

\section*{Магнитное поле в вакууме}
\section*{§ 6.1. Сила Лоренца. Поле В}
Сила Лоренца. Опыт показывает, что сила \(\mathbf{F}\), действующая на точечный заряд \(q\), зависит в общем случае не только от положения этого заряда, но и от его скорости \(\mathbf{v}\). Соответственно этому силу \(\mathbf{F}\) разделяют на две составляющие - электрическую \(\mathbf{F}_{\text {э }}\) (она не зависит от движения заряда) и магнитную \(\mathbf{F}_{\mathrm{m}}\) (она зависит от скорости заряда). В любой точке пространства направление и модуль магнитной силы зависят от скорости \(\mathbf{v}\) заряда, причем эта сила всегда перпендикулярна вектору \(\mathbf{v}\); кроме того, в любом месте магнитная сила перпендикулярна определенному в данном месте направлению и, наконец, ее модуль пропорционален той составляющей скорости, которая перпендикулярна этому выделенному направлению.

Все эти свойства магнитной силы можно описать, если ввести понятие магнитного поля. Характеризуя это поле вектором В, определяющим выделенное в каждой точке пространства направление, запишем выражение для магнитной силы в виде


\begin{equation*}
\mathbf{F}_{\mathrm{M}}=q[\mathbf{v B}] . \tag{6.1}
\end{equation*}


Тогда полная электромагнитная сила, действующая на заряд \(q\) :


\begin{equation*}
\mathbf{F}=q \mathbf{E}+q[\mathbf{v B}] . \tag{6.2}
\end{equation*}


Ее называют силой Лоренца. Последнее выражение является универсальньмм: оно справедливо как для постоянных, так и для переменных электрических и магнитных полей, причем при любых значениях скорости \(\mathbf{v}\) заряда. Заметим, что \(\mathbf{v}\) - это скорость заряда относительно интересующей нас системы отсчета.

По действию силы Лоренца на заряд можно в принципе определить модули и направления векторов \(\mathbf{E}\) и В. Поэтому выражение для силы Лоренца можно рассматривать как опреде-

ление электрического и магнитного полей (в случае электрического поля мы так и поступили)*.

Следует подчеркнуть, что на покоящийся электрический заряд магнитное поле не действует. В этом существенное отличие магнитного поля от электрического. Магнитное поле действует только на движущийся заряд.

Вектор В характеризует силовое действие магнитного поля на движущийся заряд и, следовательно, является в этом отношении аналогом вектора \(\mathbf{E}\), характеризующего силовое действие электрического поля.

Важной особенностью магнитной силы является то, что она всегда перпендикулярна вектору скорости заряда, поэтому paботь над зарядом не совершает. Это значит, что в постоянном магнитном поле энергия движущейся заряженной частицы всегда остается неизменной, как бы частица ни двигалась.

В нерелятивистском приближении сила Лоренца (6.2), как и любая другая сила, не зависит от выбора системы отсчета (инерциальной). Вместе с тем магнитная составляющая силы Лоренца меняется при переходе от одной системы отсчета к другой (из-за v). Поэтому должна меняться и электрическая составляющая \(q\) Е. Отсюда следует, что разделение полной силы \(\mathbf{F}\) - силы Лоренца - на электрическую и магнитную зависит от выбора системы отсчета. Без указания системы отсчета такое разделение не имеет смысла.

Магнитное поле равномерно движущегося заряда. Опыт показывает, что само магнитное поле порождается движущимися зарядами (токами). В результате обобщения экспериментальных данных был получен элементарный закон, определяющий поле В точечного заряда \(q\), движущегося с постоянной нерелятивистской скоростью \(\mathbf{v}\). Этот закон записывается в виде**


\begin{equation*}
\mathbf{B}=\frac{\mu_{0}}{4 \pi} \frac{q[\mathbf{v r}]}{r^{3}}, \tag{6.3}
\end{equation*}


\footnotetext{\begin{itemize}
  \item Разработан ряд способов измерения поля B, но все они, в конечном счете, базируются на явлениях, в основе которых лежит уравнение (6.2).\\
** Формула (6.3) справедлива и в случае, когда заряд движется с ускорением, однако только на достаточно малых расстояниях \(r\) от заряда (малых настолько, что за время \(r / c\) скорость \(\mathbf{v}\) заряда заметно не меняется).
\end{itemize}
}где \(\mu_{0}\) - магнитная постоянная; коэффициент

\[
\mu_{0} / 4 \pi=10^{-7} \Gamma \mathrm{H} / \mathrm{m} ;
\]

\(\mathbf{r}\) - радиус-вектор, проведенный от заряда \(q\) к точке наблюдения. Конец ради-уса-вектора \(\mathbf{r}\) неподвижен в данной системе отсчета, а его начало движется со скоростью \(\mathbf{v}\) (рис. 6.1), поэтому вектор B в данной системе отсчета зависит не только от положения точки наблюдения, но и от времени.

\begin{figure}[h]
\begin{center}
  \includegraphics[alt={},max width=\textwidth]{3e870b25-6320-483d-9133-cdd5212c2707-148_233_366_318_794}
\captionsetup{labelformat=empty}
\caption{Рис. 6.1}
\end{center}
\end{figure}

В соответствии с формулой (6.3) вектор В направлен перпендикулярно плоскости, в которой расположены векторы \(\mathbf{v}\) и \(\mathbf{r}\), причем вращение вокруг вектора \(\mathbf{v}\) в направлении вектора \(\mathbf{B}\) образует с направлением \(\mathbf{v}\) правовинтовую систему (рис. 6.1). Отметим, что вектор В является аксиальным (псевдовектором). Величину В называют магнитной индукцией.

Единицей магнитной индукции служит тесла (Тл).\\
Электрическое поле точечного заряда \(q\), движущегося с нерелятивистской скоростью, описывается тем же законом (1.2). Поэтому выражение (6.3) можно представить как


\begin{equation*}
\mathbf{B}=\varepsilon_{0} \mu_{0}[\mathbf{v E}]=[\mathbf{v E}] / c^{2}, \tag{6.4}
\end{equation*}


где \(c\) - электродинамическая постоянная ( \(c=1 / \sqrt{\varepsilon_{0} \mu_{0}}\) ), она равна скорости света в вакууме (совпадение, как потом выяснилось, не случайное).

Пример. Сравнение сил магнитного и электрического взаимодействий движущихся зарядов. Пусть два достаточно массивных точечных заряда \(q\) движутся параллельно друг другу с одинаковой нерелятивистской скоростью \(\mathbf{v}\), как показано на рис. 6.2. Найдем отношение магнитной \(F_{\text {м }}\) и электрической \(F_{\text {э }}\) сил, действующих, например, со стороны заряда 1 на заряд 2.

Согласно (6.2) \(F_{\mathrm{m}}=q v B\) и \(F_{9}=q E\), где \(v\) - скорость заряда 2 , а \(B\) и \(E\) - индукция магнит-

\begin{figure}[h]
\begin{center}
  \includegraphics[alt={},max width=\textwidth]{3e870b25-6320-483d-9133-cdd5212c2707-148_164_192_1468_956}
\captionsetup{labelformat=empty}
\caption{Рис. 6.2}
\end{center}
\end{figure}

ного и напряженность электрического полей, создаваемых зарядом 1 в месте нахождения заряда 2.

Отношение \(F_{\mathrm{M}} / F_{\text {э }}=v B / E\). В нашем случае согласно (6.4) \(B=v E / c^{2}\), поэтому


\begin{equation*}
F_{\mathrm{M}} / F_{Э}=(v / c)^{2} . \tag{6.5}
\end{equation*}


Даже для достаточно больших скоростей, например \(v=300 \mathrm{~km} / \mathrm{c}\), это отношение равно \(10^{-6}\), т. е. магнитная часть силы в миллион раз меньше электрической и составляет ничтожную поправку к электрической силе.

Рассмотренный пример может вызвать естественный вопрос - стоит ли такие силы изучать? Оказывается, стоит, и на это есть две веские причины.

Во-первых, нам приходится встречаться с пучками частиц, движущихся почти со световыми скоростями, и там эта «поправка» к электрической силе становится сравнимой с последней (заметим, что отношение (6.5) справедливо и при релятивистских скоростях).

Во-вторых, при движении, например, электронов вдоль проводов их направленная скорость при обычных плотностях составляет несколько десятых миллиметра в секунду, и отношение \((v / c)^{2} \approx 10^{-24}\). Ничтожная поправка к электрической силе! Но дело в том, что в данном случае магнитная сила - это практически вся действующая сила, ибо электрические силы исчезли в результате почти идеального баланса отрицательных и положительных зарядов в проводах. Этот баланс намного точнее чем \(10^{-24}\), и «ничтожная» магнитная сила оказывается, по существу, единственной. А участие громадного числа зарядов в создании тока компенсирует малость этого члена.

Другими словами, избыточные заряды на проводах ничтожно малы по сравнению с суммарным зарядом носителей тока. Поэтому магнитные силы в данном случае намного превосходят электрические силы, действующие на избыточные заряды проводов.

\section*{§ 6.2. Закон Био-Савара}
Принцип суперпозиции. Опыт дает, что для магнитного поля, как и для электрического, справедлив принцип суперпозиции: магнитное поле, создаваемое несколькими движущимися зарядами или токами, равно векторной сумме магнитных

полей, создаваемых каждым зарядом или током в отдельности:


\begin{equation*}
B=\sum B_{i} \tag{6.6}
\end{equation*}


Закон Био-Савара. Рассмотрим вопрос о нахождении магнитного поля, создаваемого постоянными электрическими токами. Этот вопрос будем решать, исходя из закона (6.3), определяющего индукцию поля В равномерно движущегося точечного заряда. Подставим в (6.3) вместо \(q\) заряд \(\rho \mathrm{d} V\), где \(\mathrm{d} V\) элементарный объем, \(\rho\) - объемная плотность заряда, являющегося носителем тока, и учтем, что \(\rho \mathbf{v}=\mathbf{j}\) согласно (5.2). Тогда формула (6.3) приобретет следующий вид:


\begin{equation*}
\mathrm{d} \mathbf{B}=\frac{\mu_{0}}{4 \pi} \frac{[\mathbf{j r}] \mathrm{d} V}{r^{3}} . \tag{6.7}
\end{equation*}


Если же ток \(I\) течет по тонкому проводу с площадью поперечного сечения \(\Delta S\), то

\[
j \mathrm{~d} V=j \Delta S \mathrm{~d} l=I \mathrm{~d} l,
\]

где \(\mathrm{d} l\) - элемент длины провода. Введя вектор dl в направлении тока \(I\), перепишем предыдущее равенство так:


\begin{equation*}
\mathbf{j} \mathrm{d} V=I \mathrm{~d} \mathbf{l} . \tag{6.8}
\end{equation*}


Векторы \(\mathbf{j} d V\) и \(I \mathrm{~d} l\) называют соответственно объемным и линейным элементами тока. Произведя в формуле (6.7) замену объемного элемента тока на линейный, получим


\begin{equation*}
\mathrm{d} \mathbf{B}=\frac{\mu_{0}}{4 \pi} \frac{I[\mathrm{~d} \mathbf{l}, \mathbf{r}]}{r^{3}} . \tag{6.9}
\end{equation*}


Формулы (6.7) и (6.9) выражают закон Био-Caвapa.\\
Полное поле В в соответствии с принципом суперпозиции определяется в результате интегрирования выражений (6.7) или (6.9) по всем элементам тока:


\begin{equation*}
\mathbf{B}=\frac{\mu_{0}}{4 \pi} \int \frac{[\mathbf{j r}]}{r^{3}} \mathrm{~d} V, \quad \mathbf{B}=\frac{\mu_{0}}{4 \pi} \int \frac{I[\mathrm{~d} \mathbf{l}, \mathbf{r}]}{r^{3}} . \tag{6.10}
\end{equation*}


Расчет по этим формулам индукции магнитного поля тока произвольной конфигурации, вообще говоря, сложен. Однако расчет значительно упрощается, если распределение тока имеет определенную симметрию. Приведем несколько простейших примеров на нахождение индукции магнитного поля тока.

Пример 1. Магнитное поле прямого тока, т. е. тока, текущего по тонкому прямому проводу бесконечной длины (рис. 6.3).

\begin{figure}[h]
\begin{center}
  \includegraphics[alt={},max width=\textwidth]{3e870b25-6320-483d-9133-cdd5212c2707-151_398_355_459_209}
\captionsetup{labelformat=empty}
\caption{Рис. 6.3}
\end{center}
\end{figure}

Согласно (6.9) в произвольной точке \(A\) векторы dB от всех элементов тока имеют одинаковое направление - за плоскость рисунка. Поэтому сложение векторов dB можно заменить сложением их модулей \(\mathrm{d} B\), причем

\[
\mathrm{d} B=\frac{\mu_{0}}{4 \pi} \frac{I \mathrm{~d} l \cos \alpha}{r^{2}} .
\]

Из рисунка видно, что \(\mathrm{d} l \cos _{\alpha}=r \mathrm{~d} \alpha\) и \(r=b / \cos _{\alpha}\). Значит

\[
\mathrm{d} B=\frac{\mu_{0}}{4 \pi} \frac{I \cos \alpha \mathrm{~d} \alpha}{b} .
\]

Интегрируя последнее выражение по всем элементам тока, что эквивалентно интегрированию по \(\alpha\) от \(-\pi / 2\) до \(\pi / 2\), находим


\begin{equation*}
B=\frac{\mu_{0}}{4 \pi} \frac{2 I}{b} . \tag{6.11}
\end{equation*}


Пример 2. Магнитное поле на оси кругового тока. На рис. 6.4 показан вектор dB от элемента тока \(I d l\), находящегося справа. От всех элементов тока будет образовываться конус векторов

\begin{figure}[h]
\begin{center}
  \includegraphics[alt={},max width=\textwidth]{3e870b25-6320-483d-9133-cdd5212c2707-151_395_322_1359_206}
\captionsetup{labelformat=empty}
\caption{Рис. 6.4}
\end{center}
\end{figure}

dB, и легко сообразить, что результирующий вектор В в точке \(A\) будет направлен вверх по оси \(Z\). Это значит, что для нахождения модуля вектора В достаточно сложить проекции векторов dB на ось \(Z\). Каждая такая проекция имеет вид

\[
\mathrm{d} B_{z}=\mathrm{d} B \cos \beta=\frac{\mu_{0}}{4 \pi} \frac{I \mathrm{~d} l \cos \beta}{r^{2}},
\]

где учтено, что угол между элементом dl и радиусом-вектором \(\mathbf{r}\) равен \(\pi / 2\), поэтому синус равен единице. Интегрируя это выражение по всем \(\mathrm{d} l\) (это дает \(2 \pi R\) ) и учитывая, что \(\cos \beta=R / r\) и \(r^{2}=z^{2}+R^{2}\), получаем


\begin{equation*}
B=\frac{\mu_{0}}{4 \pi} \frac{2 \pi R^{2} I}{\left(z^{2}+R^{2}\right)^{3 / 2}} . \tag{6.12}
\end{equation*}


Отсюда следует, что в центре витка с током ( \(z=0\) ) и на расстоянии \(z \gg R\) модуль вектора \(\mathbf{B}\) равен соответственно


\begin{equation*}
B_{z=0}=\frac{\mu_{0}}{4 \pi} \frac{2 \pi I}{R}, \quad B_{z J R} \approx \frac{\mu_{0}}{4 \pi} \frac{2 \pi R^{2} I}{z^{3}} . \tag{6.13}
\end{equation*}


\section*{§ 6.3. Основные законы магнитного поля}
Магнитное поле обладает, как и электрическое поле, двумя важнейшими свойствами. Эти свойства, связанные также с потоком и циркуляцией векторного поля, и выражают основные законы магнитного поля.

Прежде чем перейти к их изучению, несколько слов о графическом представлении поля В. Как и любое другое векторное поле, поле В может быть представлено наглядно с помощью линий вектора В. Их проводят обычным способом - так, чтобы касательная к этим линиям в каждой точке совпадала с направлением вектора В, а густота линий была бы пропорциональна модулю вектора В в данном месте.

Полученная таким образом геометрическая картина позволяет легко судить о конфигурации данного магнитного поля и сильно облегчает анализ некоторых ситуаций.

А теперь обратимся к основным законам магнитного поля теореме Гаусса и теореме о циркуляции.

Теорема Гаусса для поля В. Поток вектора В сквозь любую замкнутую поверхность равен нулю:


\begin{equation*}
\oint \mathbf{B} \mathrm{d} \mathbf{S}=0 . \tag{6.14}
\end{equation*}


Эта теорема является, по существу, обобщением опыта. Она выражает собой в постулативной форме тот экспериментальный факт, что линии вектора В не имеют ни начала, ни конца.

Поэтому число линий вектора \(\mathbf{B}\), выходящих из любого объема, ограниченного замкнутой поверхностью \(S\), всегда равно числу линий, входящих в этот объем.

Отсюда вытекает важное следствие, которым мы будем пользоваться в дальнейшем неоднократно. А именно: поток вектора В сквозь поверхность S, ограниченную некоторым замкнутым контуром, не зависит от формь поверхности \(S\). Это легко понять с помощью представления о линиях вектора В: так как они нигде не прерываются, их число сквозь поверхность \(S\), ограниченную данным контуром (т. е. поток вектора В), действительно не должно зависеть от формы поверхности \(S\).

Закон (6.14) выражает также и тот факт, что в природе нет магнитных зарядов, на которых начинались бы или заканчивались линии вектора В. Иначе говоря, магнитное поле не имеem источников в противоположность полю электрическому.

Теорема о циркуляции вектора В (для магнитного поля постоянных токов в вакууме). Циркуляция вектора В по произвольному контуру Г равна произведению \(\mu_{0}\) на алгебраическую сумму токов, охватываемых контуром Г:


\begin{equation*}
\oint \mathbf{B} \mathrm{d} \mathbf{l}=\mu_{0} I, \tag{6.15}
\end{equation*}


где \(I=\sum I_{k}\), причем \(I_{k}\) - величины алгебраические. Ток считается положительным, если его направление связано с направле-

\begin{figure}[h]
\begin{center}
  \includegraphics[alt={},max width=\textwidth]{3e870b25-6320-483d-9133-cdd5212c2707-153_257_338_1254_96}
\captionsetup{labelformat=empty}
\caption{Рис. 6.5}
\end{center}
\end{figure}

нием обхода по контуру правилом правого винта. Ток противоположного направления считается отрицательным. Это правило иллюстрирует рис. 6.5: здесь токи \(I_{1}\) и \(I_{3}\) положительные, ибо их направления связаны с направлением обхода по контуру правилом правого винта, а ток \(I_{2}\) - отрицательный.\\
Теорема о циркуляции (6.15) может быть доказана исходя из закона Био-Савара. В общем случае произвольных токов это доказательство достаточно кропотливо, и мы не будем приводить его здесь. Мы будем рассматривать утверждение (6.15) как постулат, подтвержденный экспериментально.

Еще одно замечание. Если ток \(I\) в \((6.15)\) распределен по объему, где расположен контур \(\Gamma\), то его можно представить как


\begin{equation*}
I=\int \mathbf{j} \mathrm{d} \mathbf{S} \tag{6.16}
\end{equation*}


Интеграл здесь берется по произвольной поверхности \(S\), натянутой на контур Г. Плотность тока \(\mathbf{j}\) под интегралом соответствует точке, где расположена площадка dS, причем вектор dS образует с направлением обхода по контуру правовинтовую систему.

Итак, в общем случае уравнение (6.15) можно записать так:


\begin{equation*}
\oint \mathbf{B} \mathrm{d} \mathbf{l}=\mu_{0} \int \mathbf{j} \mathrm{~d} \mathbf{S}=\mu_{0} \int j_{n} \mathrm{~d} S \tag{6.17}
\end{equation*}


Тот факт, что циркуляция вектора В, вообще говоря, не равна нулю, означает, что поле В не потенциально (в отличие от электростатического поля). Такое поле называют вихревьм или соленоидальньмм.

Так как циркуляция вектора В пропорциональна току \(I\), охватываемому контуром, то магнитному полю, в общем случае, нельзя приписать скалярный потенциал, который был бы связан с вектором В соотношением, аналогичным \(\mathbf{E}=-\nabla \varphi\). Этот потенциал был бы неоднозначным: при каждом обходе по контуру и возвращении в исходную точку он получал бы приращение, равное \(\mu_{0} I\). Впрочем, в той области пространства, где токов нет, магнитный потенциал \(\varphi_{m}\) вводят и достаточно эффективно используют.

Роль теоремы о циркуляции вектора В. Эта теорема играет примерно ту же роль, что и теорема Гаусса для векторов \(\mathbf{E}\) и \(\mathbf{D}\). Мы знаем, что поле В определяется всеми токами, циркуляция же вектора В только теми токами, которые охватывает данный контур. Несмотря на это, в некоторых случаях - при наличии специальной симметрии - теорема о циркуляции оказывается весьма эффективной, позволяя очень просто находить В.

Это бывает в тех случаях, когда вычисление циркуляции вектора В можно свести, выбрав разумно контур, к произведению \(B\) (или \(B_{l}\) ) на длину контура или его часть. Если этого нет, расчет поля В приходится проводить иными способами, например с помощью закона Био-Савара или путем решения соответствующих дифференциальных уравнений, и расчет становится значительно сложнее.

\section*{§ 6.4. Применения теоремы о циркуляции вектора В}
Рассмотрим несколько практически важных примеров, иллюстрирующих эффективность использования теоремы о циркуляции при расчете поля B, а затем обсудим, насколько универсален этот способ расчета.\\
Пример 1. Магнитное поле прямого тока. Пусть постоянный ток \(I\) течет вдоль бесконечно длинного прямого провода, имеющего круглое сечение радиусом \(a\). Найдем индукцию \(\mathbf{B}\) поля снаружи и внутри провода.

\begin{figure}[h]
\begin{center}
  \includegraphics[alt={},max width=\textwidth]{3e870b25-6320-483d-9133-cdd5212c2707-155_289_314_617_233}
\captionsetup{labelformat=empty}
\caption{Рис. 6.6}
\end{center}
\end{figure}

Из симметрии задачи следует, что линии вектора В в данном случае должны иметь вид окружностей с центром на оси провода. Причем модуль вектора В должен быть одинаков во всех точках на расстоянии \(r\) от оси провода. Поэтому по теореме о циркуляции вектора В для круглого контура \(\Gamma_{1}\) (рис. 6.6) \(B \cdot 2 \pi r=\mu_{0} I\), откуда следует, что вне провода


\begin{equation*}
B=\left(\mu_{0} / 2 \pi\right) I / r(r \geqslant a) . \tag{6.18}
\end{equation*}


Заметим, что решение этого вопроса непосредственно (с помощью закона Био-Савара) оказывается гораздо более сложным.

Внутри провода из тех же соображений симметрии следует, что линии вектора \(\mathbf{B}\) являются тоже окружностями. По теореме о циркуляции вектора \(\mathbf{B}\) для круглого контура \(\Gamma_{2}\) (см. рис. 6.6) \(B \cdot 2 \pi r=\mu_{0} I_{r}\), где \(I_{r}=I(r / a)^{2}\) - ток, охватываемый данным контуром. Отсюда мы находим, что внутри провода


\begin{equation*}
B=\left(\mu_{0} / 2 \pi\right) \operatorname{Ir} / a^{2}(r \leqslant a) . \tag{6.19}
\end{equation*}


Зависимость \(B(r)\) показана графически на рис. 6.7.

\begin{figure}[h]
\begin{center}
  \includegraphics[alt={},max width=\textwidth]{3e870b25-6320-483d-9133-cdd5212c2707-155_253_380_1516_187}
\captionsetup{labelformat=empty}
\caption{Рис. 6.7}
\end{center}
\end{figure}

Если провод имеет вид трубки круглого сечения, то снаружи индукция \(B\) определяется формулой (6.18), а внутри - магнитное поле отсутствует. Это также легко показать с помощью теоремы о циркуляции вектора В.

Пример 2. Магнитное поле соленоида. Пусть ток \(I\) течет по проводнику, намотанному по винтовой линии на поверхность цилиндра. Такой обтекаемый током цилиндр называют соленоидом. Пусть на единицу длины соленоида приходится \(n\) витков проводника. Если шаг винтовой линии достаточно мал, то каждый виток соленоида можно приближенно заменить замкнутым витком. Будем также предполагать, что сечение проводника настолько мало, что ток в соленоиде можно считать текущим по его поверхности.

Опыт и расчет показывают, что чем длиннее соленоид, тем меньше индукция магнитного поля снаружи него. Для бесконечно длинного соленоида магнитное поле снаружи отсутствует вообще.\\
Из соображений симметрии ясно, что линии вектора В внутри соленоида направлены вдоль его оси, причем вектор В составляет с направлением тока в соленоиде правовинтовую систему.\\
Уже то, что мы выяснили относительно конфигурации магнитного поля соленоида, подсказывает выбрать прямоугольный контур так, как показано на рис. 6.8. Циркуляция вектора В по данному кон-

\begin{figure}[h]
\begin{center}
  \includegraphics[alt={},max width=\textwidth]{3e870b25-6320-483d-9133-cdd5212c2707-156_208_351_829_802}
\captionsetup{labelformat=empty}
\caption{Рис. 6.8}
\end{center}
\end{figure}

туру равна \(B l\), и контур охватывает ток \(n l I\). Согласно теореме о циркуляции \(B l=\mu_{0} n l I\), откуда следует, что внутри длинного соленоида


\begin{equation*}
B=\mu_{0} n I, \tag{6.20}
\end{equation*}


т. е. поле внутри длинного соленоида однородно (за исключением областей, прилегающих к торцам соленоида, но этим при расчетах зачастую пренебрегают). Произведение \(n I\) называют числом ампервитков. При \(n=2000\) витков/м и \(I=2 \mathrm{~A}\) магнитное поле внутри соленоида \(B=5\) мТл.\\
Пример 3. Магнитное поле тороида. Тороид представляет собой провод, навитый на каркас, имеющий форму тора (рис. 6.9).

Из соображений симметрии нетрудно понять, что линии вектора В должны быть окружностями, центры которых расположе-

\begin{figure}[h]
\begin{center}
  \includegraphics[alt={},max width=\textwidth]{3e870b25-6320-483d-9133-cdd5212c2707-156_295_353_1479_792}
\captionsetup{labelformat=empty}
\caption{Рис. 6.9}
\end{center}
\end{figure}

ны на оси \(O O^{\prime}\) тороида. Поэтому ясно, что в качестве контура следует взять одну из таких окружностей.\\
Если контур расположен внутри тороида, он охватывает ток \(N I\), где \(N\) - число витков в тороидальной катушке; \(I\) - ток в проводе. Пусть радиус контура \(r\), тогда по теореме о циркуляции \(B \cdot 2 \pi r=\mu_{0} N I\), откуда следует, что внутри тороида


\begin{equation*}
B=\left(\mu_{0} / 2 \pi\right) N I / r . \tag{6.21}
\end{equation*}


Из сравнения (6.21) с (6.18) видно, что внутри тороида магнитное поле совпадает с полем прямого тока \(N I\), текущего вдоль оси \(O O^{\prime}\). Устремив \(N\) и радиус тороида \(R\) к бесконечности (при неизменном сечении тороида), в пределе получим выражение (6.20) для магнитного поля бесконечно длинного соленоида.\\
Если выбранный нами круглый контур проходит вне тороида, то токов он не охватывает, поэтому для такого контура \(B \cdot 2 \pi r=0\). Это значит, что вне тороида магнитное поле отсутствует.\\
В предыдущих рассуждениях предполагалось, что линии тока лежат в меридиональных плоскостях, т. е. в плоскостях, проходящих через ось \(O O^{\prime}\) тороида. У реального тороида линии тока (витки) не лежат строго в этих плоскостях, поэтому имеется составляющая тока вокруг оси \(O O^{\prime}\). Эта составляющая создает дополнительное поле, аналогичное полю кругового тока.

\section*{Пример 4. Магнитное поле плоскости с током. Рассмотрим безгранич-}
\begin{figure}[h]
\begin{center}
  \includegraphics[alt={},max width=\textwidth]{3e870b25-6320-483d-9133-cdd5212c2707-157_313_207_1233_246}
\captionsetup{labelformat=empty}
\caption{Рис. 6.10}
\end{center}
\end{figure}

ную проводящую плоскость, по которой течет равномерно распределенный ток одного направления. На рис. 6.10 показан след такой плоскости с током, текущим за плоскость рисунка (что отмечено крестиками). Введем понятие линейной плотности тока как вектор і, направленный вдоль линий тока. Модуль этого вектора представляет собой ток, приходящийся на единицу длины, которая играет роль «поперечного сечения».

Разбив мысленно плоскость с током на тонкие нити с током, нетрудно сообразить, что результирующее поле В будет направлено параллельно плоскости, причем справа от плоскости - вниз, слева - вверх (см. рис. 6.10). Эти направления легко установить по правилу правого винта.

Для определения индукции поля В воспользуемся теоремой о циркуляции вектора В. Зная, как расположены в этом случае линии вектора \(\mathbf{B}\), выберем контур в виде прямоугольника 1234. Тогда по теореме о циркуляции \(2 B l=\mu_{0} i l\), где \(l\) - длина стороны контура, параллельной плоскости с током. Из последнего равенства находим:


\begin{equation*}
B=\mu_{0} i / 2 . \tag{6.22}
\end{equation*}


Из полученной формулы видно, что магнитное поле как с одной стороны плоскости, так и с другой является однородным. Этот результат справедлив и для ограниченной пластины с током, но лишь для точек вблизи пластины и удаленных от ее краев.

Общие соображения. Результаты, полученные в рассмотренных примерах, можно было бы найти и непосредственно с помощью закона Био-Савара. Однако теорема о циркуляции позволяет получить эти результаты значительно проще и быстрее.

Вместе с тем легкость, с которой был проведен расчет поля в этих примерах, не должна создать у читателя ошибочного впечатления о силе метода, основанного на применении теоремы о циркуляции. Как и в случае теоремы Гаусса для электрического поля, число задач, легко решаемых с помощью теоремы о циркуляции вектора В, к сожалению, тоже весьма ограничено. Достаточно сказать, что уже для такой симметричной конфигурации тока, как круговой виток, теорема о циркуляции оказывается беспомощной. Здесь конфигурация магнитного поля, несмотря на вроде бы хорошую симметрию, не позволяет, однако, найти необходимый для расчета простой контур, и задачу приходится решать другими способами, гораздо более громоздкими.

\section*{§ 6.5. Дифференциальная форма основных законов магнитного поля}
Дивергенция поля В. Теорема Гаусса (6.14) для поля В в дифференциальной форме имеет вид


\begin{equation*}
\nabla \cdot \mathbf{B}=0, \tag{6.23}
\end{equation*}


т. е. дивергенция поля В всюду равна нулю. Это означает, повторяем, что магнитное поле не имеет источников (магнитных зарядов). Маг-

нитное поле порождают не магнитные заряды (которых в природе нет), а электрические токи.

Закон (6.23) является фундаментальньмм: он справедлив не только для постоянных, но и для переменных магнитных полей.

Ротор поля В. Важное свойство магнитного поля, которое выражает теорема о циркуляции вектора В, побуждает представить и эту теорему в дифференциальной форме, расширяющей ее возможности как инструмента исследования и расчета.

С этой целью рассмотрим отношение циркуляции вектора \(\mathbf{B}\) к площади \(S\), ограниченной контуром. Оказывается, это отношение стремится к некоторому пределу при \(S \rightarrow 0\), причем этот предел зависит от ориентации контура в данной точке пространства. Ориентация контура задается вектором \(\mathbf{n}\) нормали к плоскости контура, причем направление \(\mathbf{n}\) связано с направлением обхода по контуру правилом правого винта.

Предел, получаемый при указанной операции, представляет собой скалярную величину, которая ведет себя как проекция некоторого вектора на направление нормали \(\mathbf{n}\) к плоскости контура, по которому берется циркуляция. Этот вектор называют ротором поля В и обозначают символом rot В. Таким образом,


\begin{equation*}
\lim _{S \rightarrow 0} \frac{\oint \mathbf{B} \mathrm{~d} \mathbf{l}}{S}=(\operatorname{rot} \mathbf{B})_{n}, \tag{6.24}
\end{equation*}


где справа стоит проекция вектора \(\operatorname{rot} \mathbf{B}\) на нормаль \(\mathbf{n}\).\\
Итак, в каждой точке векторного поля \(\mathbf{B}\) имеется вектор \(\operatorname{rot} \mathbf{B}\), направление и модуль которого связаны со свойствами самого поля в данной точке. Направление векторa \(\operatorname{rot} \mathbf{B}\) определяется тем направлением нормали \(\mathbf{n}\) к площадке \(S\), при котором достигается максимальное значение величины (6.24), являющееся одновременно модулем вектоpa rot B.

В математике получают выражение для \(\operatorname{rot} \mathbf{B}\) в координатном представлении. Для наших целей важно другое: оказывается, формально rot \(\mathbf{B}\) можно рассматривать как векторное произведение оператора \(\nabla\) на вектор В, т. е. как \(\nabla \times\) В. Мы будем пользоваться последним, более удобным обозначением: оно сразу же позволяет записать векторное произведение \(\nabla \times \mathbf{B}\) с помощью определителя:

\[
\nabla \times \mathbf{B}=\left|\begin{array}{ccc}
\mathrm{e}_{x} & \mathrm{e}_{y} & \mathrm{e}_{z}  \tag{6.25}\\
\partial / \partial x & \partial / \partial y & \partial / \partial z \\
B_{x} & B_{y} & B_{z}
\end{array}\right|,
\]

где \(\mathbf{e}_{x}, \mathbf{e}_{y}, \mathbf{e}_{z}\) - орты осей декартовых координат. Данное выражение справедливо для ротора не только поля \(\mathbf{B}\), но и для любого другого векторного поля, в частности и для поля Е.

Обратимся теперь к теореме о циркуляции вектора В. Согласно (6.24) уравнение (6.17) можно представить в виде

\[
\lim _{S \rightarrow 0} \frac{\oint \mathbf{B} \mathrm{~d} \mathbf{l}}{S}=\mu_{0} j_{n},
\]

или \((\boldsymbol{\nabla} \times \mathbf{B})_{n}=\mu_{0} j_{n}\). Отсюда


\begin{equation*}
\nabla \times \mathbf{B}=\mu_{0} \mathbf{j} . \tag{6.26}
\end{equation*}


Это и есть дифференциальная форма теоремы о циркуляции вектора \(\mathbf{B}\). Видно, что ротор поля В совпадает по направлению с вектором \(\mathbf{j}\) плотностью тока в данной точке, а модуль \(\boldsymbol{\nabla} \times \mathbf{B}\) равен \(\mu_{0} j\).

В электростатическом поле циркуляция вектора \(\mathbf{E}\) равна нулю, поэтому


\begin{equation*}
\nabla \times \mathbf{E}=0 \text {. } \tag{6.27}
\end{equation*}


Векторное поле, ротор которого всюду равен нулю, является потенциальным, в противном случае поле является соленоидальньм. Значит, электростатическое поле есть поле потенциальное, магнитное же поле - соленоидальное.

\section*{§ 6.6. Сила Ампера}
Закон Ампера. Каждый носитель тока испытывает действие магнитной силы. Действие этой силы передается проводнику, по которому заряды движутся. В результате магнитное поле действует с определенной силой на сам проводник с током. Найдем эту силу.

Пусть объемная плотность заряда, являющегося носителем тока (электроны в металле, например), равна р. Выделим мысленно элемент объема \(d V\) проводника. В нем находится заряд - носитель тока, равный \(\rho \mathrm{d} V\). Тогда сила, действующая на элемент \(\mathrm{d} V\) проводника, может быть записана по формуле (6.1) в виде

\[
\mathrm{d} \mathbf{F}=\rho[\mathbf{u B}] \mathrm{d} V,
\]

где \(\mathbf{u}\) - скорость упорядоченного движения зарядов.

Так как \(\mathbf{j}=\rho \mathbf{u}\), то


\begin{equation*}
\mathrm{d} \mathbf{F}=[\mathbf{j} \mathbf{B}] \mathrm{d} V . \tag{6.28}
\end{equation*}


Если ток течет по тонкому проводнику, то согласно (6.8) \(\mathbf{j} \mathbf{~ d} V=I \mathrm{~d} \mathbf{l}\) и


\begin{equation*}
\mathrm{d} \mathbf{F}=I[\mathrm{~d} \mathbf{l}, \mathbf{B}], \tag{6.29}
\end{equation*}


где dl - вектор, совпадающий по направлению с током и характеризующий элемент длины тонкого проводника.

Формулы (6.28) и (6.29) выражают закон \(\boldsymbol{A}\) мпера. Интегрируя эти выражения по элементам тока (объемным или линейным), можно найти магнитную силу, действующую на тот или иной объем проводника или его линейный участок.

Силы, действующие на токи в магнитном поле, называют амперовыми или силами Ампера.\\
Пример. Сила взаимодействия параллельных токов. Найдем амперову силу, с которой взаимодействуют в вакууме два параллельных бесконечно длинных провода с токами \(I_{1}\) и \(I_{2}\), если расстояние между проводами равно \(b\). Расчет силы произведем на единицу длины этой системы.

Каждый элемент тока \(I_{2}\) находится в магнитном поле тока \(I_{1}\), а именно в поле \(B_{1}=\left(\mu_{0} / 4 \pi\right) 2 I_{1} / b\) согласно (6.18). Угол между элементом тока \(I_{2}\) и вектором \(\mathbf{B}_{1}\) прямой, поэтому, как следует из формулы (6.29), на единицу длины проводника с током \(I_{2}\) действует сила \(F_{\text {ед }}=I_{2} B_{1}\), или


\begin{equation*}
F_{\mathrm{en}}=\frac{\mu_{0}}{4 \pi} \frac{2 I_{1} I_{2}}{b} . \tag{6.30}
\end{equation*}


Для силы, действующей на единицу длины проводника с током \(I_{1}\), получается, разумеется, то же выражение.

И последнее. Нетрудно убедиться в том, что токи, одинаково направленные, притягиваются, а противоположно направленные - отталкиваются. Здесь речь идет только о магнитной силе. Не следует, однако, забывать, что кроме магнитной имеется еще и электрическая сила - сила, обусловленная избыточными зарядами на поверхности проводников. Поэтому, если говорить о полной силе взаимодействия между проводами, то

она может быть как силой притяжения, так и силой отталкивания - все зависит от соотношения магнитной и электрической составляющих полной силы (см. задачу 6.7).

Сила, действующая на контур с током. Результирующая амперова сила, которая действует на контур с током в магнитном поле, определяется в соответствии с (6.29) как


\begin{equation*}
\mathbf{F}=I \oint[\mathrm{~d} \mathbf{l}, \mathbf{B}], \tag{6.31}
\end{equation*}


где интегрирование проводится по данному контуру с током \(I\).\\
Если магнитное поле однородно, то вектор В можно вынести из-под интеграла и задача сводится к вычислению векторного интеграла \(\oint \mathrm{dl}\). Этот интеграл представляет собой замкнутую цепочку элементарных векторов dl, поэтому он равен нулю. Значит, и \(\mathbf{F}=0\), т. е. результирующая амперова сила равна нулю в однородном магнитном поле.

Если же магнитное поле неоднородно, то результирующая сила (6.31), вообще говоря, отлична от нуля и в каждом конкретном случае она определяется с помощью выражения (6.31). Для дальнейшего особый интерес представляет случай, когда контур с током плоский и его размеры достаточно малы. Такой контур с током называют элементарньлм.

Поведение элементарного контура с током удобно описывать с помощью магнитного момента \(\mathbf{p}_{\mathrm{m}}\). По определению


\begin{equation*}
\mathbf{p}_{\mathrm{m}}=I S \mathbf{n}, \tag{6.32}
\end{equation*}


где \(I\) - ток; \(S\) - площадь, ограниченная контуром; n - нормаль к контуру, направление которой связано с направлением тока в контуре правилом правого винта (рис. 6.11). В магнитном отношении элементарный контур с током вполне характеризуется его магнитным моментом \(\mathbf{p}_{\mathrm{m}}\).

\begin{figure}[h]
\begin{center}
  \includegraphics[alt={},max width=\textwidth]{3e870b25-6320-483d-9133-cdd5212c2707-162_239_253_1238_891}
\captionsetup{labelformat=empty}
\caption{Рис. 6.11}
\end{center}
\end{figure}

Довольно кропотливый расчет по формуле (6.31) с учетом малости контура приводит к следующему выражению для силы, действующей на элементарный контур с током в неоднородном магнитном поле:


\begin{equation*}
\mathrm{F}=p_{\mathrm{m}} \frac{\partial \mathrm{~B}}{\partial n}, \tag{6.33}
\end{equation*}


где \(p_{\mathrm{m}}\) - модуль магнитного момента контура; \(\partial \mathbf{B} / \partial n\) - производная вектора \(\mathbf{B}\) по направлению нормали \(\mathbf{n}\) или по направлению вектора \(\mathbf{p}_{\mathrm{m}}\). Последнее выражение аналогично (1.39) для силы, действующей на электрический диполь в электрическом поле.

Из формулы (6.33) видно, что, как и в случае электрического диполя:

\begin{enumerate}
  \item в однородном магнитном поле \(\mathbf{F}=0\), ибо \(\partial \mathbf{B} / \partial n=0\);
  \item направление вектора \(\mathbf{F}\), вообще говоря, не совпадает ни с
\end{enumerate}

\begin{figure}[h]
\begin{center}
  \includegraphics[alt={},max width=\textwidth]{3e870b25-6320-483d-9133-cdd5212c2707-163_364_324_568_68}
\captionsetup{labelformat=empty}
\caption{Рис. 6.12}
\end{center}
\end{figure}

вектором \(\mathbf{B}\), ни с вектором \(\mathbf{p}_{\mathrm{m}}\); вектор \(\mathbf{F}\) совпадает лишь с направлением элементарного приращения вектора \(\mathbf{B}\), взятого в направлении вектора \(\mathrm{p}_{\mathrm{m}}\) в месте расположения контура. Сказанное иллюстрирует рис. 6.12, где показаны три расположения контура в магнитном поле прямого тока \(I_{0}\). Здесь же показан и вектор результирующей силы \(\mathbf{F}\), которая действует на контур в каждом случае (полезно самостоятельно убедиться, что это действительно так).\\
Если нас интересует проекция силы \(\mathbf{F}\) на некоторое направление \(X\), то достаточно записать выражение (6.33) в проекциях на это направление, и мы получим


\begin{equation*}
F_{x}=p_{\mathrm{m}} \frac{\partial B_{x}}{\partial n}, \tag{6.34}
\end{equation*}


где \(\partial B_{x} / \partial n\) - производная соответствующей проекции вектора \(\mathbf{B}\) опять же по направлению нормали \(\mathbf{n}\) к контуру (или по \(\mathbf{p}_{\mathrm{m}}\) ).\\
Пример. Пусть элементарный контур с током, имеющий магнитный момент \(\mathbf{p}_{\mathrm{m}}\), расположен перпендикулярно оси симметрии неоднородного магнитного поля, причем вектор \(\mathbf{p}_{\mathrm{m}}\) - в направ-

\begin{figure}[h]
\begin{center}
  \includegraphics[alt={},max width=\textwidth]{3e870b25-6320-483d-9133-cdd5212c2707-163_259_310_1510_229}
\captionsetup{labelformat=empty}
\caption{Рис. 6.13}
\end{center}
\end{figure}

лении вектора В. Выберем положительное направление оси \(X\), как показано на рис. 6.13. Так как в направлении вектора \(\mathbf{p}_{\mathrm{m}}\) приращение проекции \(B_{x}\) будет отрицательным, то \(F_{x}<0\). Значит, вектор \(\mathbf{F}\) направлен влево - в сторону, где \(B\) больше.

Если же контур с током (и вектор \(\mathbf{p}_{\mathrm{m}}\) ) повернуть на \(90^{\circ}\) так, чтобы центр контура совпал с осью симметрии поля В, то в этом положении \(F_{x}=0\), а вектор \(\mathbf{F}\) будет направлен перпендикулярно оси \(X\), причем в ту же сторону, что и \(\mathbf{p}_{\mathrm{m}}\).

\section*{§ 6.7. Момент сил, действующих на контур С ТОКОМ}
Рассмотрим плоский контур с током \(I\) в однородном магнитном поле В. Выше (см. с. 161) мы выяснили, что результирующая сила (6.31), которая действует на контур с током в однородном магнитном поле, равна нулю. А из механики известно, что если результирующая сил, действующих на любую систему, равна нулю, то суммарный момент этих сил не зависит от точки \(O\), относительно которой определяют моменты этих сил. Раз так, можно просто говорить о результирующем моменте амперовых сил в нашем случае.

По определению, результирующий момент амперовых сил


\begin{equation*}
\mathbf{N}=\oint[\mathbf{r}, \mathrm{d} \mathbf{F}] \tag{6.35}
\end{equation*}


где dF дается формулой (6.29). Если провести расчет по формуле (6.35) - он довольно громоздок и мало интересен, поэтому мы не будем его приводить, - то оказывается, что для произвольной формы контура с током этот момент сил можно представить как


\begin{equation*}
\mathbf{N}=\left[\mathbf{p}_{\mathrm{m}} \mathbf{B}\right], \tag{6.36}
\end{equation*}


где \(\mathbf{p}_{\mathrm{m}}\) - магнитный момент контура с током (для плоского контура \(\left.\mathbf{p}_{\mathrm{m}}=I S \mathbf{n}\right)^{*}\).

Из (6.36) видно, что момент \(\mathbf{N}\) амперовых сил, действующих на контур с током в однородном магнитном поле, перпендикулярен как вектору \(\mathbf{p}_{\mathrm{m}}\), так и вектору \(\mathbf{B}\). Модуль вектора \(\mathbf{N}\) равен \(N=p_{\mathrm{m}} B \sin _{\alpha}\), где \(\alpha\) - угол между векторами \(\mathbf{p}_{\mathrm{m}}\) и В. В тех слу-

\footnotetext{\begin{itemize}
  \item Если виток не плоский, то его магнитный момент \(\mathbf{p}_{\mathrm{m}}=I \int \mathrm{~d} \mathbf{S}\), где интеграл берется по поверхности \(S\), натянутой на контур с током. Этот интеграл не зависит от выбора поверхности \(S\), а зависит только от контура, на который она натянута.
\end{itemize}
}чаях, когда \(\mathbf{p}_{\mathrm{m}} \uparrow \uparrow \mathbf{B}\), момент сил \(\mathbf{N}=0\), и нетрудно убедиться в том, что положение контура будет устойчивым. Если \(\mathbf{p}_{\mathrm{m}} \downarrow \uparrow \mathbf{B}\), то тоже \(\mathbf{N}=0\), но такое положение контура является неустойчивым: малейшее отклонение от этого положения приведет к появлению момента сил, стремящегося отклонить контур еще больше от начального положения.

Пример. Убедимся в справедливости формулы (6.36) на простейшем случае прямоугольного контура с током (рис. 6.14).

\begin{figure}[h]
\begin{center}
  \includegraphics[alt={},max width=\textwidth]{3e870b25-6320-483d-9133-cdd5212c2707-165_273_265_508_254}
\captionsetup{labelformat=empty}
\caption{Рис. 6.14}
\end{center}
\end{figure}

Как видно из данного рисунка, силы, действующие на стороны \(a\), перпендикулярны им и вектору В, поэтому эти силы направлены горизонтально (на рисунке они не показаны) и стремятся только растянуть (или сжать) контур. Стороны \(b\) перпендикулярны В, поэтому на каждую из них действует сила

\[
F=I b B .
\]

Эти силы стремятся повернуть контур так, чтобы его вектор \(\mathbf{p}_{\mathrm{m}}\) оказался сонаправленным с вектором В. Стало быть, на контур действует пара сил, момент которой равен произведению плеча пары \(a \sin \alpha\) на \(F\), т.е.

\[
N=I b B a \sin \alpha .
\]

Учитывая, что \(a b\) - это площадь, ограниченная контуром, и \(I b a=p_{\mathrm{m}}\), получим

\[
N=p_{m} B \sin \alpha,
\]

что в векторной форме записывается как (6.36).\\
В заключение необходимо отметить, что выражение (6.36) справедливо и для неоднородных магнитных полей. Надо только, чтобы размеры контура с током были достаточно малы. Тогда влиянием неоднородности на вращающий момент \(\mathbf{N}\) можно пренебречь. Именно это относится к элементарному контуру с током.

Во внешнем неоднородном магнитном поле элементарный контур с током ведет себя аналогично тому, как и электрический диполь во внешнем неоднородном электрическом поле: он будет поворачиваться к положению устойчивого равновесия\\
(при котором \(\mathbf{p}_{\mathrm{m}} \uparrow \uparrow \mathbf{B}\) ) и, кроме того, под действием результирующей силы \(\mathbf{F}\) втягиваться туда, где индукция \(B\) больше.

\section*{§ 6.8. Работа при перемещении контура с током}
Когда контур с током находится во внешнем магнитном поле - мы будем предполагать, что оно постоянное, - на отдельные элементы контура действуют амперовы силы, а поэтому при перемещении контура эти силы будут совершать работу. В этом параграфе мы покажем, что работа, которую совершают амперовы силы при элементарном перемещении контура с током \(I\), определяется как


\begin{equation*}
\delta A=I \mathrm{~d} \Phi, \tag{6.37}
\end{equation*}


где dФ - приращение магнитного потока сквозь контур при данном перемещении.

Доказательство этой формулы проведем в три этапа.

\begin{enumerate}
  \item Сначала рассмотрим частный случай: контур (рис. 6.15) с подвижной перемычкой длины \(l\) находится в однородном магнитном поле, перпендикулярном плоскости контура и направленном за плоскость рисунка. На перемычку согласно (6.29) действует амперова сила \(F=I l B\). При перемещении перемычки вправо на \(\mathrm{d} x\) эта сила совершает положительную работу
\end{enumerate}


\begin{equation*}
\delta A=F \mathrm{~d} x=I B l \mathrm{~d} x=I B \mathrm{~d} S, \tag{6.38}
\end{equation*}


где \(\mathrm{d} S\) - приращение площади, ограниченной контуром. Для определения знака магнитного потока Ф условимся всегда брать нормаль \(\mathbf{n}\) к поверхности, ограниченной контуром, так, чтобы она образовывала с направлением тока в контуре правовинтовую систему (см. рис. 6.15). При этом ток \(I\) будет всегда величиной положительной. Поток же Ф может быть как положительным, так и отрицательным. Но в нашем случае как \(\Phi\), так и \(\mathrm{d} \Phi=B \mathrm{~d} S\) являются ве-

\begin{figure}[h]
\begin{center}
  \includegraphics[alt={},max width=\textwidth]{3e870b25-6320-483d-9133-cdd5212c2707-166_217_374_1359_786}
\captionsetup{labelformat=empty}
\caption{Рис. 6.15}
\end{center}
\end{figure}

личинами положительными (если бы поле В было направлено на нас или перемычка перемещалась бы влево, то в обоих случаях \(\mathrm{d} \Phi<0\) ). Как бы то ни было, в любом из этих случаев выражение (6.38) можно представить в виде (6.37).\\
2. Полученный результат справедлив и для произвольного направления поля В. Чтобы убедиться в этом, разложим вектор \(\mathbf{B}\) на три составляющие: \(\mathbf{B}=\mathbf{B}_{n}+\mathbf{B}_{l}+\mathbf{B}_{x}\). Составляющая \(\mathbf{B}_{l}-\) вдоль перемычки - параллельна току в ней и поэтому не оказывает на перемычку силового действия. Составляющая \(\mathbf{B}_{x}\) вдоль перемещения - дает силу, перпендикулярную перемещению, работы она не совершает. Остается лишь составляющая \(\mathbf{B}_{n}\) - перпендикулярная плоскости, в которой перемещается перемычка. Поэтому в формуле (6.38) вместо \(B\) надо брать только \(B_{n}\). Но \(B_{n} \mathrm{~d} S=\mathrm{d} \Phi\), и мы опять приходим к формуле (6.37).\\
3. Теперь перейдем к рассмотрению любого контура при произвольном перемещении его в постоянном неоднородном магнитном поле (контур может при этом и произвольным образом деформироваться). Разобъем мысленно данный контур на бесконечно малые элементы тока и рассмотрим бесконечно малые перемещения их. В этих условиях магнитное поле, в котором перемещается каждый элемент тока, можно считать однородным. Для такого перемещения к каждому элементу тока применимо выражение \(\mathrm{d} A=I \mathrm{~d}^{\prime} \Phi\) для элементарной работы, где под d'Ф надо понимать вклад в приращение потока сквозь контур от данного элемента контура. Сложив такие элементарные работы для всех элементов контура, снова получим выражение (6.37), где dФ есть приращение магнитного потока сквозь весь контур.

Чтобы найти работу амперовых сил при полном перемещении контура с током от начального положения 1 до конечного 2 , достаточно проинтегрировать выражение (6.37):


\begin{equation*}
A=\int_{1}^{2} I \mathrm{~d} \Phi . \tag{6.39}
\end{equation*}


Если при этом перемещении поддерживать ток \(I\) постоянным, то


\begin{equation*}
A=I\left(\Phi_{2}-\Phi_{1}\right), \tag{6.40}
\end{equation*}


где \(\Phi_{1}\) и \(\Phi_{2}\) - магнитные потоки сквозь контур в начальном и конечном положениях. Таким образом, работа амперовых сил в этом случае равна произведению силы тока на приращение магнитного потока сквозь контур. Выражение (6.40) дает не только величину, но и знак совершаемой работы.

Пример. Плоский контур с током \(I\) поворачивают в магнитном поле В из положения, при котором нормаль к контуру \(\mathbf{n} \downarrow \uparrow \mathbf{B}\), в положение, при котором \(\mathbf{n} \uparrow \uparrow \mathbf{B}\) (напоминаем, что направление нормали n связано с направлением тока правилом правого винта). Площадь, ограниченная контуром, равна \(S\). Найдем работу амперовых сил при указанном перемещении, считая, что ток \(I\) поддерживается постоянным.\\
Согласно (6.40)

\[
A=I[B S-(-B S)]=2 I B S
\]

В данном случае работа \(A>0\), при обратном же повороте \(A<0\). Следует отметить, что работа (6.40) совершается не за счет энергии внешнего магнитного поля (оно не меняется), а за счет источника э.д.с., поддерживающего ток в контуре. (Но об этом более подробно будет рассказано в гл. 9.)

\section*{Задачи}
6.1. Непосредственный расчет индукции В. Ток \(I\) течет по тонкому проводнику, изогнутому, как показано на рис. 6.16. Найти магнитную индукцию \(B\) в точке \(O\). Необходимые данные указаны на рисунке.

Решение. Искомая величина \(B=B_{-}+B_{\cup}\), где \(B_{-}\)- магнитное поле от прямолинейного участка контура; \(B_{\cup}\) - от его криволи-

\begin{figure}[h]
\begin{center}
  \includegraphics[alt={},max width=\textwidth]{3e870b25-6320-483d-9133-cdd5212c2707-168_273_294_907_849}
\captionsetup{labelformat=empty}
\caption{Рис. 6.16}
\end{center}
\end{figure}

нейной части. Согласно закону Био-Савара (см. пример 1 на с. 150)

\[
\begin{gathered}
B_{-}=2 \int_{0}^{\alpha_{0}} \frac{\mu_{0} I \cos \alpha \mathrm{~d} \alpha}{4 \pi a \cos \alpha_{0}}=\frac{\mu_{0} I}{2 \pi a} \operatorname{tg} \alpha_{0} \\
B_{\cup}=\frac{\mu_{0}}{4 \pi} \frac{I\left(2 \pi-2 \alpha_{0}\right) a}{a^{2}}=\frac{\mu_{0} I}{2 \pi a}\left(\pi-\alpha_{0}\right)
\end{gathered}
\]

В результате

\[
B=\left(\pi-\alpha_{0}+\operatorname{tg} \alpha_{0}\right) \mu_{0} I / 2 \pi a .
\]

Полезно убедиться, что при \(\alpha_{0} \rightarrow 0\) мы приходим к известному выражению (6.13).\\
6.2. Тонкий провод с изоляцией образует плоскую спираль из большого числа \(N\) плотно расположенных витков, по

\begin{figure}[h]
\begin{center}
  \includegraphics[alt={},max width=\textwidth]{3e870b25-6320-483d-9133-cdd5212c2707-169_296_275_224_141}
\captionsetup{labelformat=empty}
\caption{Рис. 6.17}
\end{center}
\end{figure}

которым течет постоянный ток \(I\). Радиусы внутреннего и внешнего витков равны \(a\) и \(b\) (рис.6.17). Найти: 1) магнитную индукцию \(B\) в центре спирали - точке \(O ; 2)\) магнитный момент спирали при данном токе.

Решение. 1. Вклад в магнитную индукцию от одного витка радиуса \(r\) равен согласно (6.13)


\begin{equation*}
B_{1}=\mu_{0} I / 2 r, \tag{1}
\end{equation*}


а от всех витков


\begin{equation*}
B=\int B_{1} \mathrm{~d} N, \tag{2}
\end{equation*}


где \(\mathrm{d} N\) - число витков в интервале ( \(r, r+\mathrm{d} r\) ),


\begin{equation*}
\mathrm{d} N=\frac{N}{b-a} \mathrm{~d} r . \tag{3}
\end{equation*}


После подстановки (1) и (3) в (2) и последующего интегрирования по \(r\) от \(a\) до \(b\) получим

\[
B=\frac{\mu_{0} I N}{2(b-a)} \ln \frac{b}{a} .
\]

\begin{enumerate}
  \setcounter{enumi}{1}
  \item Магнитный момент одного витка радиуса \(r\) есть \(p_{\mathrm{m} 1}=I \pi r^{2}\), а всех витков \(p_{\mathrm{m}}=\int p_{\mathrm{m} 1} \mathrm{~d} N\), где \(\mathrm{d} N\) определяется формулой (3). Интегрирование дает
\end{enumerate}

\[
B=\frac{\pi I N}{b-a} \frac{b^{3}-a^{3}}{3}=\frac{\pi I N}{3}\left(a^{2}+a b+b^{2}\right) .
\]

6.3. Ток \(I\) течет по длинному прямому проводнику, имеющему форму желоба с поперечным сечением в виде тонкого полукольца радиусом \(R\) (рис. 6.18). Найти магнитную индукцию \(B\) на оси \(O\).

Решение. Прежде всего выяс-

\begin{figure}[h]
\begin{center}
  \includegraphics[alt={},max width=\textwidth]{3e870b25-6320-483d-9133-cdd5212c2707-169_201_370_1533_206}
\captionsetup{labelformat=empty}
\caption{Рис. 6.18}
\end{center}
\end{figure}

ним, куда направлен вектор В в точке \(O\). Для этого мысленно разобьем весь проводник на элементарные нити с током \(\mathrm{d} I\). Тогда ясно, что любые две симметричные нити дадут в сумме

вектор dB, направленный вправо (рис. 6.19). Значит, туда же будет направлен и вектор В. Поэтому для нахождения поля В в точке \(O\) достаточно найти сумму проекций элементарных

\begin{figure}[h]
\begin{center}
  \includegraphics[alt={},max width=\textwidth]{3e870b25-6320-483d-9133-cdd5212c2707-170_207_336_144_738}
\captionsetup{labelformat=empty}
\caption{Рис. 6.19}
\end{center}
\end{figure}

векторов dВ от каждой нити тока на направление вектора В:


\begin{equation*}
B=\int \mathrm{d} B \sin \varphi \tag{1}
\end{equation*}


Согласно (6.11)


\begin{equation*}
\mathrm{d} B=\mu_{0} \mathrm{~d} I / 2 \pi R \tag{2}
\end{equation*}


где \(\mathrm{d} I=(I / \pi) \mathrm{d} \varphi\) (см. рис. 6.19). После подстановки (2) в (1) получим

\[
B=\frac{\mu_{0} I}{2 \pi^{2} R} \int_{0}^{\pi} \sin \varphi \mathrm{d} \varphi=\frac{\mu_{0} I}{\pi^{2} R} .
\]

6.4. Теорема о циркуляции В и принцип суперпозиции. Внутри однородного длинного прямого провода круглого сечения имеется круглая цилиндрическая полость, ось которой параллельна оси провода и смещена относительно последней на расстояние 1 . По проводу течет постоянный ток плотности \(\mathbf{j}\). Найти магнитную индукцию В внутри полости.\\
Решение. Искомую величину можно представить согласно принципу суперпозиции как


\begin{equation*}
\mathbf{B}=\mathbf{B}_{0}-\mathbf{B}^{\prime}, \tag{1}
\end{equation*}


где \(\mathbf{B}_{0}\) - магнитная индукция в том случае, если бы проводник был сплошным (без полости), а \(\mathbf{B}^{\prime}\) - магнитная индукция поля в той же точке от тока, текущего по части провода, которую мы удалили, образовав полость круглого сечения.

Таким образом, задача предусматривает прежде всего вычисление магнитной индукции В внутри сплошного провода на расстоянии \(\mathbf{r}\) от его оси. Воспользовавшись теоремой о циркуляции, запишем \(2 \pi r B=\mu_{0} \pi r^{2} j\), откуда \(B=\mu_{0} r j / 2\). Последнее равенство можно представить с помощью рис. 6.20 в векторной форме:

\[
\mathbf{B}=\mu_{0}[\mathbf{j r}] / 2
\]

\begin{figure}[h]
\begin{center}
  \includegraphics[alt={},max width=\textwidth]{3e870b25-6320-483d-9133-cdd5212c2707-171_312_293_147_252}
\captionsetup{labelformat=empty}
\caption{Рис. 6.20}
\end{center}
\end{figure}

\begin{figure}[h]
\begin{center}
  \includegraphics[alt={},max width=\textwidth]{3e870b25-6320-483d-9133-cdd5212c2707-171_312_318_147_744}
\captionsetup{labelformat=empty}
\caption{Рис. 6.21}
\end{center}
\end{figure}

Представив теперь по этой формуле \(\mathbf{B}_{0}\) и \(\mathbf{B}^{\prime}\), из выражения (1) найдем их разность:

\[
B=\frac{\mu_{0}}{2}[j r]-\frac{\mu_{0}}{2}\left[j r^{\prime}\right]=\frac{\mu_{0}}{2}\left[j, r-r^{\prime}\right] .
\]

Из рис. 6.21 видно, что \(\mathbf{r}=\mathbf{l}+\mathbf{r}^{\prime}\), откуда \(\mathbf{r}-\mathbf{r}^{\prime}=\mathbf{l}\) и

\[
\mathbf{B}=\mu_{0}[\mathbf{j} \mathbf{l}] / 2
\]

Таким образом, в нашем случае магнитное поле В в полости является однородным, и если ток течет к нам (рис. 6.21), то поле В направлено в плоскости этого рисунка вверх.\\
6.5. Принцип суперпозиции. Имеется длинный соленоид с током \(I\). Площадь поперечного сечения соленоида \(S\), число витков на единицу длины \(n\). Найти магнитный поток сквозь торец этого соленоида.

Решение. Пусть поток вектора В сквозь торец соленоида равен Ф. Если приставить к данному соленоиду еще такой же, то поток через соприкасающиеся торцы будет \(\Phi+\Phi=\Phi_{0}\), где \(\Phi_{0}\) - поток сквозь поперечное сечение соленоида вдали от его торца. Тогда

\[
\Phi=\Phi_{0} / 2=\mu_{0} n I S / 2
\]

Попутно следует обратить внимание на следующие особенности поля В у торца длинного соленоида.

\begin{enumerate}
  \item Линии вектора В расположены так, как показано на рис. 6.22.
\end{enumerate}

\begin{figure}[h]
\begin{center}
  \includegraphics[alt={},max width=\textwidth]{3e870b25-6320-483d-9133-cdd5212c2707-171_238_364_1521_138}
\captionsetup{labelformat=empty}
\caption{Рис. 6.22}
\end{center}
\end{figure}

Это нетрудно понять с помощью принципа суперпозиции: если приставить справа еще такой же соленоид, поле В вне образованного таким образом составного соленоида должно обратиться в нуль, а это возможно только при указанной на рисунке конфигурации поля.\\
2. Из того же принципа суперпозиции следует, что нормальная составляющая \(B_{n}\) будет одинакова по площади торца, ибо при образовании составного соленоида \(B_{n}+B_{n}=B_{0}\), где \(B_{0}\) - поле внутри соленоида вдали от его торцов. В центре торца \(B=B_{n}\), и мы получаем, что \(B=B_{0} / 2\).\\
6.6. Поле соленоида. Намоткой длинного соленоида с радиусом сечения \(a\) служит тонкая лента-проводник шириной \(h\), намотанная в один слой практически вплотную. Вдоль ленты течет постоянный ток \(I\). Найти магнитное поле \(B\) внутри и вне соленоида как функцию расстояния \(r\) от его оси.

\begin{figure}[h]
\begin{center}
  \includegraphics[alt={},max width=\textwidth]{3e870b25-6320-483d-9133-cdd5212c2707-172_304_422_338_738}
\captionsetup{labelformat=empty}
\caption{Рис. 6.23}
\end{center}
\end{figure}

Решение. Вектор линейной плотности тока i можно представить в виде суммы двух составляющих:

\[
\mathbf{i}=\mathbf{i}_{\perp}+\mathbf{i}_{\|},
\]

смысл векторов \(\mathbf{i}_{\perp}\) и \(\mathbf{i}_{\|}\)ясен из рис. 6.23, б. В нашем случае модули этих векторов можно найти с помощью рис. 6.23, \(a\) по формулам:

\[
\begin{gathered}
i_{\perp}=i \cos \alpha=i \sqrt{1-\sin ^{2} \alpha}=(I / h) \sqrt{1-(h / 2 \pi a)^{2}}, \\
i_{\|}=i \sin _{\alpha}=I / 2 \pi a .
\end{gathered}
\]

Магнитная индукция В внутри соленоида определяется согласно (6.20) величиной \(i_{\perp}\), а вне соленоида - величиной \(i_{\|}\):

\[
\begin{gathered}
B_{i}=\mu_{0} i_{\perp}=\left(\mu_{0} I / h\right) \sqrt{1-(h / 2 \pi a)^{2}} \quad(r<a), \\
B_{a}=\mu_{0} i_{\|} a / r=\mu_{0} I / 2 \pi r \quad(r>a),
\end{gathered}
\]

где при вычислении \(B_{a}\) вне соленоида была использована теорема о циркуляции: \(2 \pi r B_{a}=\mu_{0} 2 \pi a i_{\|}\).\\
Таким образом, представив ток в соленоиде в виде суперпозиции «поперечной» и «продольной» составляющих, мы пришли к выводу, что внутри такого соленоида существует только продольная составляющая поля В, а вне соленоида - только поперечная (как от прямого тока).

Кроме того, если уменьшить ширину ленты, оставляя неизменной плотность тока, то при \(h \rightarrow 0\) сила тока \(I \rightarrow 0\), но \(I / h=\) const. В этом случае остается только поле внутри соленоида - соленоид становится «идеальным».\\
6.7. Взаимодействие параллельных проводников. Два длинных провода с пренебрежимо малым сопротивлением замкнуты с одного конца на сопротивление \(R\), а с другого конца подключены к источнику постоянного напряжения. Радиус сечения каждого провода в \(\eta=20\) раз меньше расстояния между осями проводов. При каком значении сопротивления \(R\) результирующая сила взаимодействия проводов обратится в нуль?\\
Решение. На каждом из проводов (протекает по ним ток или нет) имеются избыточные поверхност-

\begin{figure}[h]
\begin{center}
  \includegraphics[alt={},max width=\textwidth]{3e870b25-6320-483d-9133-cdd5212c2707-173_187_453_512_135}
\captionsetup{labelformat=empty}
\caption{Рис. 6.24}
\end{center}
\end{figure}

ные заряды (рис. 6.24). Поэтому кроме магнитной силы \(F_{\text {м }}\) необходимо учесть и электрическую \(F_{\text {э }}\). Пусть на единицу длины провода приходится избыточный заряд \(\lambda\). Тогда электрическая сила, действующая на единицу длины провода со стороны другого провода, может быть найдена с помощью теоремы Гаусса:

\[
F_{\Im}=\lambda E=\lambda \frac{1}{4 \pi \varepsilon_{0}} \frac{2 \lambda}{l}=\frac{2 \lambda^{2}}{4 \pi \varepsilon_{0} l},
\]

где \(l\) - расстояние между осями проводов. Магнитную же силу, действующую также на единицу длины провода, можно найти с помощью теоремы о циркуляции вектора B:

\[
F_{\mathrm{M}}=\left(\mu_{0} / 4 \pi\right) 2 l^{2} / l,
\]

где \(I\) - сила тока в проводе.\\
Заметим, что обе силы - электрическая и магнитная - направлены в противоположные стороны. Электрическая сила обусловливает притяжение проводов, магнитная - их отталкивание. Найдем отношение этих сил:


\begin{equation*}
F_{\mathrm{M}} / F_{Э}=\varepsilon_{0} \mu_{0} I^{2} / \lambda^{2} \tag{1}
\end{equation*}


Между величинами \(I\) и \(\lambda\) существует определенная связь (см. задачу 2.8):


\begin{equation*}
\lambda=C_{1} U=\frac{\pi \varepsilon_{0}}{\ln \eta} U, \tag{2}
\end{equation*}


где \(U=R I\). Поэтому из соотношения (2) следует, что


\begin{equation*}
I / \lambda=\ln \eta / \pi \varepsilon_{0} R . \tag{3}
\end{equation*}


После подстановки (3) в (1) получим


\begin{equation*}
\frac{F_{\mathrm{M}}}{F_{\ni}}=\frac{\mu_{0}}{\varepsilon_{0}} \frac{\ln ^{2} \eta}{\pi^{2} R^{2}} . \tag{4}
\end{equation*}


Результирующая сила взаимодействия обращается в нуль, когда последнее отношение равно единице. Это будет при \(R=R_{0}\), где

\[
R_{0}=\sqrt{\frac{\mu_{0}}{\varepsilon_{0}}} \frac{\ln \eta}{\pi}=360 \text { Ом. }
\]

Если \(R<R_{0}\), то \(F_{\mathrm{m}}>F_{\text {э }}\) - провода отталкиваются, если же \(R>R_{0}\), то \(F_{\mathrm{m}}<F_{\ni}\) - провода притягиваются. Это можно наблюдать на опыте.

Таким образом, утверждение, что провода, по которым текут токи противоположного направления, отталкиваются, справедливо тогда, когда электрической частью взаимодействия можно пренебречь, т. е. при достаточно малом сопротивлении \(R\) в схеме (6.24).

Кроме того, измерив силу взаимодействия между проводами с током (а сила всегда измеряется как результирующая), мы не можем, вообще говоря, определить силу тока \(I\). Это необходимо иметь в виду во избежание недоразумений.\\
6.8. Момент сил Ампера. В поле длинного прямого провода с током \(I_{0}\) находится контур с током \(I\) (рис. 6.25). Плоскость контура перпендикулярна прямому проводу. Найти момент сил Ампера, действующий на этот контур. Необходимые размеры системы указаны на рисунке.

Решение. Силы Ампера, действующие на

\begin{figure}[h]
\begin{center}
  \includegraphics[alt={},max width=\textwidth]{3e870b25-6320-483d-9133-cdd5212c2707-174_249_343_1021_813}
\captionsetup{labelformat=empty}
\caption{Рис. 6.25}
\end{center}
\end{figure}

криволинейные участки контура, равны нулю. Силы же, действующие на прямолинейные участки, создают пару сил. Момент этой пары сил нам и надо вычислить.\\
Выделим два малых элемента контура (рис. 6.26). Из рисунка видно, что момент соответствующей им пары сил

\begin{figure}[h]
\begin{center}
  \includegraphics[alt={},max width=\textwidth]{3e870b25-6320-483d-9133-cdd5212c2707-174_257_361_1396_755}
\captionsetup{labelformat=empty}
\caption{Рис. 6.26}
\end{center}
\end{figure}


\begin{equation*}
\mathrm{d} N=2 x \operatorname{tg}_{\varphi} \mathrm{d} F, \tag{1}
\end{equation*}


где элементарная сила Ампера


\begin{equation*}
\mathrm{d} F=I \mathrm{~d} l B . \tag{2}
\end{equation*}


Зависимость магнитной индукции \(B\) от расстояния \(r\) до прямого провода находим с помощью теоремы о циркуляции:


\begin{equation*}
B=\mu_{0} I / 2 \pi r . \tag{3}
\end{equation*}


Теперь подставим (3) в (2), затем (2) в (1) и, учитывая, что \(\mathrm{d} l=\mathrm{d} r\) и \(x=r \cos \varphi\), проинтегрируем полученное выражение по \(r\) от \(a\) до \(b\). В результате найдем

\[
N=\left(\mu_{0} / \pi\right) I I_{0}(b-a) \sin \varphi,
\]

причем вектор N направлен влево (см. рис. 6.26).\\
6.9. Небольшая катушка с током, имеющая магнитный момент \(\mathbf{p}_{\mathrm{m}}\), находится на оси кругового витка радиу-

\begin{figure}[h]
\begin{center}
  \includegraphics[alt={},max width=\textwidth]{3e870b25-6320-483d-9133-cdd5212c2707-175_211_314_769_186}
\captionsetup{labelformat=empty}
\caption{Рис. 6.27}
\end{center}
\end{figure}

сом \(R\), по которому течет ток \(I\). Найти силу \(\mathbf{F}\), действующую на катушку, если ее расстояние от центра витка равно \(l\), а вектор \(\mathbf{p}_{\mathrm{m}}\) ориентирован, как показано на рис. 6.27.

Решение. Искомая сила согласно (6.33) определяется так:


\begin{equation*}
\mathbf{F}=p_{\mathrm{m}} \partial \mathbf{B} / \partial n, \tag{1}
\end{equation*}


где В - магнитная индукция поля, создаваемого витком в месте нахождения катушки. Выберем ось \(Z\) в направлении вектора \(\mathbf{p}_{\mathrm{m}}\), тогда проекция (1) на эту ось будет иметь вид

\[
F_{z}=p_{\mathrm{m}} \partial B_{z} / \partial z=p_{\mathrm{m}} \partial B / \partial z
\]

где учтено, что при заданном направлении тока в витке \(B_{z}=B\). Магнитная индукция \(B\) определяется формулой (6.12), откуда

\[
\frac{\partial B}{\partial z}=-\frac{3}{2} \frac{\mu_{0} R^{2} I l}{2\left(l^{2}+R^{2}\right)^{5 / 2}}
\]

Вследствие того что \(\partial B / \partial z<0\), проекция силы \(F_{z}<0\), т. е. вектор \(\mathbf{F}\) направлен в сторону витка с током \(I\). В векторном виде полученный результат можно представить так:

\[
\mathbf{F}=-\frac{3}{2} \frac{\mu_{0} R^{2} I l}{2\left(l^{2}+R^{2}\right)^{5 / 2}} \mathbf{p}_{\mathrm{m}}
\]

Заметим, что если бы вектор \(\mathbf{p}_{\mathrm{m}}\) (а значит, и ось \(Z\) ) был направлен в противоположную сторону, то \(B_{z}=-B\) и \(\partial B_{z} / \partial z>0\), а следовательно, \(F_{z}>0\) и вектор \(\mathbf{F}\) был бы направлен вправо, т. е. опять против вектора \(\mathbf{p}_{\mathrm{m}}\). Таким образом, полученное выражение для \(\mathbf{F}\) справедливо для обеих ориентаций вектора \(\mathbf{p}_{\mathrm{m}}\).\\
6.10. Вдоль длинного тонкостенного круглого цилиндра радиусом \(R\) течет ток \(I\). Какое давление испытывают стенки цилиндра?

Решение. Рассмотрим поверхностный элемент тока \(\mathbf{i ~ d} S\), где \(\mathbf{i}\) линейная плотность тока, \(\mathrm{d} S\) - элемент поверхности. Найдем связь между поверхностным и объемным элементами тока:

\[
j \mathrm{~d} V=j_{\delta} h \cdot \delta b \mathrm{~d} l=i \mathrm{~d} S .
\]

Смысл входящих сюда величин пояснен на рис. 6.28 . В векторном виде


\begin{equation*}
\mathbf{j} \mathrm{d} V=\mathbf{i} \mathrm{d} S . \tag{1}
\end{equation*}


Сила Ампера, действующая на поверхностный элемент тока, в этом случае определяется формулой, полученной из (6.28) путем замены (1):

\begin{figure}[h]
\begin{center}
  \includegraphics[alt={},max width=\textwidth]{3e870b25-6320-483d-9133-cdd5212c2707-176_272_187_590_945}
\captionsetup{labelformat=empty}
\caption{Рис. 6.28}
\end{center}
\end{figure}


\begin{equation*}
\mathrm{d} \mathbf{F}=\left[\mathbf{i} \mathbf{B}^{\prime}\right] \mathrm{d} S, \tag{2}
\end{equation*}


где \(\mathbf{B}^{\prime}\) - магнитная индукция поля в месте нахождения данного элемента тока от всех других элементов тока, исключая данный. Чтобы найти \(\mathbf{B}^{\prime}\), поступим аналогично тому, как это было сделано для электрической силы (§ 2.3). Пусть \(B_{i}\) - магнитная индукция поля, создаваемого самим поверхностным элементом тока в точ-

\begin{figure}[h]
\begin{center}
  \includegraphics[alt={},max width=\textwidth]{3e870b25-6320-483d-9133-cdd5212c2707-176_144_330_1105_818}
\captionsetup{labelformat=empty}
\caption{Рис. 6.29}
\end{center}
\end{figure}

ках, очень близких к его поверхности (рис. 6.29 , где предполагается, что ток течет от нас). Согласно (6.22) \(B_{i}=\mu_{0} i / 2\).

Далее, воспользовавшись теоремой о циркуляции вектора В и соображениями симметрии, легко установить, что магнитная индукция поля снаружи цилиндра у его поверхности


\begin{equation*}
B=\mu_{0} I / 2 \pi R, \tag{3}
\end{equation*}


а внутри цилиндра поле отсутствует.\\
Последнее означает, что поле \(\mathbf{B}^{\prime}\) от всех остальных элементов тока в двух очень близких к поверхности цилиндра точках 1 и 2 (см. рис. 6.29) должно быть одинаково и удовлетворять следую-

щим условиям внутри и вне поверхности цилиндра:

\[
B^{\prime}=B_{i} \text { и } B=B^{\prime}+B_{i}=2 B^{\prime} \text {. }
\]

Отсюда следует, что


\begin{equation*}
B^{\prime}=B / 2 \text {. } \tag{4}
\end{equation*}


Подставив этот результат в (2), получим следующее выражение для искомого давления:

\[
p=\frac{\mathrm{d} F}{\mathrm{~d} S}=i B^{\prime}=\frac{2 B^{\prime}}{\mu_{0}} B^{\prime}=\frac{B^{2}}{2 \mu_{0}} .
\]

Учитывая (3), найдем окончательно

\[
p=\mu_{0} I / 8 \pi^{2} R^{2} .
\]

Из формулы (2) видно, что цилиндр испытывает боковое сжатие.

\section*{Магнитное поле в веществе}
\section*{§ 7.1. Намагничение вещества. Намагниченность J}
Поле в магнетике. Если в магнитное поле, образованное токами в проводах, ввести то или иное вещество, поле изменится. Это объясняется тем, что всякое вещество является магнетиком, т. е. способно под действием магнитного поля намагничиваться - приобретать магнитный момент. Намагниченное вещество создает свое магнитное поле \(\mathbf{B}^{\prime}\), которое вместе с первичным полем \(\mathbf{B}_{0}\), обусловленным токами проводимости, образует результирующее поле


\begin{equation*}
\mathbf{B}=\mathbf{B}_{0}+\mathbf{B}^{\prime} . \tag{7.1}
\end{equation*}


Здесь под \(\mathbf{B}^{\prime}\) и \(\mathbf{B}_{0}\) имеются в виду поля, усредненные по физически бесконечно малому объему.

Поле \(\mathbf{B}^{\prime}\), как и поле \(\mathbf{B}_{0}\) токов проводимости, не имеет источников (магнитных зарядов), поэтому для результирующего поля В при наличии магнетика справедлива теорема Гаусса:


\begin{equation*}
\oint \mathbf{B} \mathrm{d} \mathbf{S}=0 . \tag{7.2}
\end{equation*}


Это означает, что линии вектора В и при наличии вещества остаются всюду непрерывными.

Механизм намагничения. В настоящее время установлено, что молекулы многих веществ обладают собственным магнитным моментом, обусловленным внутренним движением зарядов. Каждому магнитному моменту соответствует элементарный круговой ток, создающий в окружающем пространстве магнитное поле. При отсутствии внешнего магнитного поля магнитные моменты молекул ориентированы беспорядочно, поэтому обусловленное ими результирующее магнитное поле рав-

но нулю. Равен нулю и суммарный магнитный момент вещества. Последнее относится и к тем веществам, молекулы которых при отсутствии внешнего поля не имеют магнитных моментов.

Если же вещество поместить во внешнее магнитное поле, то под действием этого поля магнитные моменты молекул приобретают преимущественную ориентацию в одном направлении, и вещество намагничивается - его суммарный магнитный момент становится отличным от нуля. При этом магнитные поля отдельных молекул уже не компенсируют друг друга, в результате возникает поле \(\mathbf{B}^{\prime}\).

Иначе происходит намагничивание веществ, молекулы которых при отсутствии внешнего поля не имеют магнитного момента. Внесение таких веществ во внешнее поле индуцирует элементарные круговые токи в молекулах, и молекулы, а вместе с ними и все вещество приобретают магнитный момент, что также приводит к возникновению поля \(\mathbf{B}^{\prime}\). Большинство веществ при внесении в магнитное поле намагничиваются слабо. Сильными магнитными свойствами обладают только ферромагнитные вещества: железо, никель, кобальт, многие их сплавы и др.

Намагниченность. Степень намагничения магнетика характеризуют магнитным моментом единицы объема. Эту величину называют намагниченностью и обозначают J. По определению


\begin{equation*}
\mathrm{J}=\frac{1}{\Delta V} \sum \mathrm{p}_{\mathrm{m}}, \tag{7.3}
\end{equation*}


где \(\Delta V\) - физически бесконечно малый объем в окрестности данной точки, \(\mathbf{p}_{\mathrm{m}}\) - магнитный момент отдельной молекулы. Суммирование проводится по всем молекулам в объеме \(\Delta V\).

Аналогично тому, как это было сделано для поляризованности \(\mathbf{P}\) [см. (3.3)], намагниченность можно представить как


\begin{equation*}
\mathbf{J}=n\left\langle\mathbf{p}_{\mathrm{m}}\right\rangle, \tag{7.4}
\end{equation*}


где \(n\) - концентрация молекул; \(\left\langle\mathbf{p}_{\mathrm{m}}\right\rangle\) - средний магнитный момент одной молекулы. Из последней формулы видно, что вектор \(\mathbf{J}\) сонаправлен именно со средним вектором \(\left\langle\mathbf{p}_{\mathrm{m}}\right\rangle\), поэтому в дальнейшем достаточно знать поведение вектора \(\left\langle\mathbf{p}_{\mathrm{m}}\right\rangle\) и представлять себе все молекулы в пределах объема \(\Delta V\) имеющими одинаковый магнитный момент \(\left\langle\mathbf{p}_{\mathrm{m}}\right\rangle\). Это будет значительно об-

легчать понимание вопросов, связанных с явлением намагничивания. Например, увеличение намагниченности \(\mathbf{J}\) вещества означает соответствующее увеличение вектора \(\left\langle\mathbf{p}_{\mathrm{m}}\right\rangle\) : если \(\mathbf{J}=\mathbf{0}\), то и \(\left\langle\mathbf{p}_{\mathrm{m}}\right\rangle=0\).

Если во всех точках вещества вектор \(\mathbf{J}\) одинаков, говорят, что вещество намагничено однородно.

Токи намагничивания \(I^{\prime}\). Намагничивание вещества, как уже было сказано, обусловлено преимущественной ориентацией или индуцированием магнитных моментов отдельных молекул в одном направлении. Это же можно сказать и об элементарных круговых токах, связанных с каждой молекулой, их называют молекулярньми токами. Такое поведение молекулярных токов приводит, как мы сейчас увидим, к появлению макроскопических токов \(I^{\prime}\), называемых токами намагничивания. Обычные токи, текущие по проводникам, связаны с перемещением в веществе носителей тока, их называют токами проводимости I.

Чтобы понять, как возникают токи намагничивания, представим себе сначала цилиндр из однородного магнетика, намагниченность \(\mathbf{J}\) которого однородна и направлена вдоль оси. Молекулярные токи в намагниченном магнетике ориентированы, как показано на рис. 7.1. У соседних молекул молекулярные токи в местах их соприкосновения текут в противоположных направлениях и макроскопически взаимно компенсируют друг друга. Некомпенсированными остаются

\begin{figure}[h]
\begin{center}
  \includegraphics[alt={},max width=\textwidth]{3e870b25-6320-483d-9133-cdd5212c2707-180_400_301_889_859}
\captionsetup{labelformat=empty}
\caption{Рис. 7.1}
\end{center}
\end{figure}

только те молекулярные токи, которые выходят на боковую поверхность цилиндра. Эти токи и образуют макроскопический поверхностный ток намагничивания \(I^{\prime}\), циркулирующий по боковой поверхности цилиндра. Ток \(I^{\prime}\) возбуждает такое же макроскопическое магнитное поле, что и молекулярные токи вместе взятые.

Теперь представим себе другой случай: намагниченный магнетик является неоднородным. Пусть, например, молекулярные токи расположены так, как на рис. 7.2 , где толщина линий соответствует силе молекулярных токов. Эта картина означает,

\begin{figure}[h]
\begin{center}
  \includegraphics[alt={},max width=\textwidth]{3e870b25-6320-483d-9133-cdd5212c2707-181_321_243_141_74}
\captionsetup{labelformat=empty}
\caption{Рис. 7.2}
\end{center}
\end{figure}

что вектор \(\mathbf{J}\) направлен за плоскость рисунка и растет по модулю при увеличении координаты \(x\). Здесь видно, что компенсации молекулярных токов внутри неоднородного магнетика уже нет, и в результате возникает макроскопический объемный ток намагничивания \(I^{\prime}\), текущий в положительном направлении оси \(Y\). Соответственно говорят о линейной \(\mathbf{i}^{\prime}\) и объемной \(\mathbf{j}^{\prime}\) плотностях тока, \(i^{\prime}(\mathrm{A} / \mathrm{m})\) и \(j^{\prime}\left(\mathrm{A} / \mathrm{m}^{2}\right)\).\\
О расчете поля В в магнетике. Можно утверждать, что вклад от намагниченного магнетика в поле В равен вкладу, который был создан тем же распределением токов \(I^{\prime}\) в вакууме. Иначе говоря, установив распределение токов намагничивания \(I^{\prime}\), можно с помощью закона Био-Савара найти соответствующее им поле \(\mathbf{B}^{\prime}\) и по формуле (7.1) - результирующее поле В.

Однако неприятность состоит в том, что распределение токов \(I^{\prime}\) зависит не только от конфигурации и свойств магнетика, но и от самого искомого поля В. Поэтому задача о нахождении поля В в магнетике в общем случае непосредственно решена быть не может. Остается попытаться найти иной путь подхода к решению этого вопроса. И первым шагом на этом пути является установление важной связи между током намагничивания \(I^{\prime}\) и определенным свойством поля вектора \(\mathbf{J}\), а именно его циркуляцией.

\section*{§ 7.2. Циркуляция вектора J}
Оказывается - в этом мы сейчас убедимся, - для стационарного случая циркуляция намагниченности \(\mathbf{J}\) по произвольному контуру \(\Gamma\) равна алгебраической сумме токов намагничивания \(I^{\prime}\), охватываемых контуром \(\Gamma\) :


\begin{equation*}
\oint \mathbf{J} \mathrm{d} \mathbf{l}=I^{\prime} \tag{7.5}
\end{equation*}


где \(I^{\prime}=\int \mathbf{j}^{\prime} \mathrm{dS}\), причем интегрирование проводится по произвольной поверхности, натянутой на контур Г.

Для доказательства этой теоремы вычислим алгебраическую сумму молекулярных токов, охватываемых контуром Г. Натянем на контур \(\Gamma\) произвольную поверхность \(S\) (рис. 7.3). Из это-

го рисунка видно, что одни молекулярные токи пересекают поверхность \(S\) дважды - раз в одном направлении, второй раз в другом. Поэтому такие токи не вносят никакого вклада в результирующий ток намагничивания

\begin{figure}[h]
\begin{center}
  \includegraphics[alt={},max width=\textwidth]{3e870b25-6320-483d-9133-cdd5212c2707-182_192_380_139_770}
\captionsetup{labelformat=empty}
\caption{Рис. 7.3}
\end{center}
\end{figure}

через поверхность \(S\).

Но те молекулярные токи, которые обвиваются вокруг контура \(\Gamma\), пересекают поверхность \(S\) только один раз. Такие молекулярные токи и создают макроскопический ток намагничивания \(I^{\prime}\), пронизывающий поверхность \(S\). Пусть каждый молекулярный ток равен \(I_{\mathrm{M}}\) и площадь, охватываемая им, \(S_{\mathrm{m}}\). Тогда, как видно из рис. 7.4 , элемент \(\mathrm{d} l\) контура \(\Gamma\) обвивают те молекулярные токи, центры которых попадают внутрь косого цилиндрика с объемом \(\mathrm{d} V=S_{\mathrm{M}} \cos \alpha \mathrm{d} l\), где \(\alpha\) - угол между элементом dl контура и направлением век-

\begin{figure}[h]
\begin{center}
  \includegraphics[alt={},max width=\textwidth]{3e870b25-6320-483d-9133-cdd5212c2707-182_255_493_673_663}
\captionsetup{labelformat=empty}
\caption{Рис. 7.4}
\end{center}
\end{figure}

тора \(\mathbf{J}\) в данном месте. Все эти молекулярные токи пересекают поверхность \(S\) один раз, и их вклад в ток намагничивания \(\mathrm{d} I^{\prime}=I_{\mathrm{m}} n \mathrm{~d} V\), где \(n\) - концентрация молекул. Подставив сюда выражение для \(\mathrm{d} V\), получим

\[
\mathrm{d} I^{\prime}=I_{\mathrm{M}} S_{\mathrm{M}} n \cos _{\alpha} \mathrm{d} l=J \cos \alpha \mathrm{~d} l=\mathbf{J} \mathrm{d} \mathbf{l} ;
\]

здесь учтено, что \(I_{\mathrm{M}} S_{\mathrm{M}}=p_{\mathrm{M}}\) - магнитный момент отдельного молекулярного тока, а произведение \(I_{\mathrm{M}} S_{\mathrm{m}} n\) - магнитный момент единицы объема вещества.

Проинтегрировав полученное выражение по всему контуру Г, получим (7.5). Теорема доказана.

Остается заметить, что если магнетик неоднородный, то ток намагничивания \(I^{\prime}\), вообще говоря, пронизывает всю поверхность (см. рис. 7.3), а не только у ее границы, прилегающей к контуру Г. Именно поэтому его и можно представить как \(I^{\prime}=\int \mathbf{j}^{\prime} \mathrm{dS}\), где интегрирование распространяется по всей поверхности \(S\), ограниченной контуром Г. В приведенном же доказательстве нам удалось весь ток \(I^{\prime}\) как бы «согнать» к границе поверхности \(S\) - прием, единственной целью которого является упростить вычисление этого тока.

Дифференциальная форма уравнения (7.5):


\begin{equation*}
\nabla \times \mathbf{J}=\mathbf{j}^{\prime}, \tag{7.6}
\end{equation*}


т. е. ротор намагниченности \(\mathbf{J}\) равен плотности тока намагничивания в той же точке пространства.

Замечание о поле вектора J. Свойства поля вектора J, выраженные уравнениями (7.5) и (7.6), разумеется, не означают, что само поле \(\mathbf{J}\) определяется только токами \(I^{\prime}\). Поле вектора \(\mathbf{J}\) (оно ограничено только той областью пространства, которое заполнено магнетиком) зависит от всех токов - как от тока намагничивания \(I^{\prime}\), так и от тока проводимости \(I\). Однако в некоторых случаях с определенной симметрией дело обстоит так,\\
\includegraphics[max width=\textwidth, alt={}, center]{3e870b25-6320-483d-9133-cdd5212c2707-183_44_1040_677_68}\\
Пример. Найдем поверхностный ток намагничивания, приходящийся на единицу длины цилиндра из однородного магнетика, если его намагниченность \(\mathbf{J}\), причем вектор \(\mathbf{J}\) направлен всюду вдоль оси цилиндра.

\begin{figure}[h]
\begin{center}
  \includegraphics[alt={},max width=\textwidth]{3e870b25-6320-483d-9133-cdd5212c2707-183_368_237_905_201}
\captionsetup{labelformat=empty}
\caption{Рис. 7.5}
\end{center}
\end{figure}

Применим уравнение (7.5) к контуру, выбранному так, как показано на рис. 7.5. Циркуляция вектора \(\mathbf{J}\) по этому контуру равна, как нетрудно сообразить, произведению Jl. Ток намагничивания здесь поверхностный. Если обозначить его линейную плотность буквой \(i^{\prime}\), то рассматриваемый контур охватывает ток намагничивания \(i^{\prime} l\). Из равенства \(J l=i^{\prime} l\) получаем


\begin{equation*}
i^{\prime}=J . \tag{7.7}
\end{equation*}


Отметим попутно, что векторы \(\mathbf{i}^{\prime}\) и \(\mathbf{J}\) взаимно перпендикулярны: \(\mathbf{i}^{\prime} \perp \mathbf{J}\).

\section*{§ 7.3. Вектор H}
Теорема о циркуляции вектора \(\mathbf{H}\) (для магнитного поля постоянных токов). В магнетиках, помещенных во внешнее магнитное поле, возникают токи намагничивания, поэтому циркуляция вектора В теперь будет определяться не только токами проводимости, но и токами намагничивания, а именно:


\begin{equation*}
\oint \mathbf{B} \mathrm{d} \mathbf{l}=\mu_{0}\left(I+I^{\prime}\right), \tag{7.8}
\end{equation*}


где \(I\) и \(I^{\prime}\) - токи проводимости и намагничивания, охватываемые заданным контуром Г.

Ввиду того что определение токов \(I^{\prime}\) в общем случае задача сложная, формула (7.8) становится малопригодной в практическом отношении. Оказывается, однако, можно найти некоторый вспомогательный вектор, циркуляция которого будет определяться только токами проводимости, охватываемыми контуром Г. Действительно, мы уже знаем, что с током \(I^{\prime}\) связана циркуляция намагниченности:


\begin{equation*}
\oint \mathbf{J} \mathbf{d} \mathbf{l}=I^{\prime} . \tag{7.9}
\end{equation*}


Предполагая, что циркуляция векторов В и \(\mathbf{J}\) берется по одному и тому же контуру \(\Gamma\), выразим \(I^{\prime}\) в уравнении (7.8) по формуле (7.9), тогда:


\begin{equation*}
\oint\left(\frac{\mathbf{B}}{\mu_{0}}-\mathbf{J}\right) \mathrm{d} \mathbf{l}=I . \tag{7.10}
\end{equation*}


Величину, стоящую под интегралом в скобках, обозначают буквой Н.

Итак, мы нашли некоторый вспомогательный вектор H:


\begin{equation*}
\mathbf{H}=\frac{\mathbf{B}}{\mu_{0}}-\mathbf{J}, \tag{7.11}
\end{equation*}


циркуляция которого по произвольному контуру Г равна алгебраической сумме токов проводимости \(I\), охватываемых этим контуром:


\begin{equation*}
\oint \mathbf{H} \mathrm{d} \mathbf{l}=I . \tag{7.12}
\end{equation*}


Эта формула выражает теорему о циркуляции вектора H: циркуляция вектора H по произвольному замкнутому контуру равна алгебраической сумме токов проводимости, охватываемых этим контуром.

Правило знаков для токов то же, что и в случае циркуляции вектора В (см. с. 152).

Заметим, что вектор H представляет собой комбинацию двух совершенно различных величин \(\mathbf{B} / \mu_{0}\) и \(\mathbf{J}\). Поэтому вектор \(\mathbf{H}\) это действительно вспомогательный вектор, не имеющий ско-лько-нибудь глубокого физического смысла*. Однако важное свойство вектора Н, выраженное в теореме о его циркуляции, оправдывает введение этого вектора: во многих случаях он значительно упрощает изучение поля в магнетиках.

И еще, соотношения (7.11) и (7.12) справедливы для любых магнетиков, в том числе и анизотропных.

Из формулы (7.12) видно, что модуль вектора \(\mathbf{H}\) имеет размерность силы тока, деленной на длину. В связи с этим единицей величины \(H\) является ампер на метр (A/м).

Дифференциальная форма теоремы о циркуляции вектора H :


\begin{equation*}
\nabla \times \mathbf{H}=\mathbf{j}, \tag{7.13}
\end{equation*}


т. е. ротор вектора \(\mathbf{H}\) равен плотности тока проводимости в той же точке вещества.

Связь между векторами J и Н. Мы уже знаем, что намагниченность \(\mathbf{J}\) зависит от магнитной индукции B в данной точке вещества. Однако \(\mathbf{J}\) принято связывать не с \(\mathbf{B}\), а с вектором \(\mathbf{H}\). Мы ограничимся пока рассмотрением только таких магнетиков, для которых зависимость между \(\mathbf{J}\) и \(\mathbf{H}\) имеет линейный характер, а именно:


\begin{equation*}
\mathbf{J}=\chi \mathbf{H} \tag{7.14}
\end{equation*}


где \(\chi\) - магнитная восприимчивость, безразмерная величина, характерная для каждого данного магнетика (безразмерность \(\chi\) следует из того, что согласно (7.11) размерности \(H\) и \(J\) одинаковы).

В отличие от диэлектрической восприимчивости \(\varkappa\), которая всегда положительна, магнитная восприимчивость \(\chi\) бывает как положительной, так и отрицательной. Соответственно магнетики, подчиняющиеся зависимости (7.14), подразделяют на парамагнетики ( \(\chi>0\) ) и диамагнетики ( \(\chi<0\) ). У парамагнетиков \(\mathbf{J} \uparrow \uparrow \mathbf{H}\), у диамагнетиков \(\mathbf{J} \uparrow \downarrow \mathbf{H}\).

\footnotetext{\begin{itemize}
  \item Величину \(\mathbf{H}\) часто называют напряженностью магнитного поля, однако мы не будем пользоваться этим термином, чтобы лишний раз подчеркнуть вспомогательный характер вектора H.
\end{itemize}
}Заметим, что кроме этих магнетиков существуют ферромагнетики, у которых зависимость \(\mathbf{J} \boldsymbol{(} \mathbf{H} \boldsymbol{)}\) имеет весьма сложный характер: она не линейная и, помимо того, наблюдается гистерезис, т. е. зависимость J от предыстории магнетика. (Более подробно о ферромагнетиках будет рассказано в § 7.6.)

Связь между В и Н. Для магнетиков, которые подчиняются зависимости (7.14), выражение (7.11) принимает вид ( \(1+\chi\) ) \(\mathrm{H}= =\mathbf{B} / \mu_{0}\). Отсюда


\begin{equation*}
\mathbf{B}=\mu \mu_{0} \mathbf{H}, \tag{7.15}
\end{equation*}


где \(\mu\) - магнитная проницаемость среды,


\begin{equation*}
\mu=1+\chi . \tag{7.16}
\end{equation*}


У парамагнетиков \(\mu>1\), у диамагнетиков \(\mu<1\), причем как у тех, так и у других \(\mu\) отличается от единицы весьма мало, т. е. магнитные свойства этих магнетиков выражены очень слабо.

Замечание о поле вектора Н. Обратимся к вопросу, с которым связано довольно часто встречающееся заблуждение: от каких токов зависит поле вектора \(\mathbf{H}\) ? Поле \(\mathbf{H}\) зависит, вообще говоря, от всех токов - и от токов проводимости, и от токов намагничивания (как и поле вектора В). Об этом говорит уже формула (7.15). Однако в некоторых случаях поле Н определяется только токами проводимости - именно для таких случаев вектор H является весьма полезным. Вместе с тем это дает повод ошибочно думать, что поле вектора \(\mathbf{H}\) якобы зависит всегда только от токов проводимости, и неверно трактовать теорему о циркуляции вектора \(\mathbf{H}\) и уравнение (7.13). Указанная теорема выражает только определенное свойство поля вектора \(\mathbf{H}\), само же поле этого вектора она не определяет.

Пример. Система состоит из длинного прямого провода с током \(I\) и произвольного куска парамагнетика (рис. 7.6). Выясним, что произойдет с полями векторов В и Н, а также с циркуляцией вектора \(\mathbf{H}\) по некоторому фиксированному контуру \(\Gamma\), если магнетик удалить.

\begin{figure}[h]
\begin{center}
  \includegraphics[alt={},max width=\textwidth]{3e870b25-6320-483d-9133-cdd5212c2707-186_241_352_1464_808}
\captionsetup{labelformat=empty}
\caption{Рис. 7.6}
\end{center}
\end{figure}

В каждой точке пространства поле В обусловлено как током проводимости \(I\), так и токами намагничивания в парамагнетике. А так как в нашем случае согласно (7.15) \(\mathbf{H}=\mathbf{B} / \mu \mu_{0}\), то сказанное относится и к полю вектора \(\mathbf{H}\) - оно тоже зависит и от тока проводимости \(I\), и от токов намагничивания.

Удаление куска парамагнетика приведет к изменению поля В, а значит, и поля Н. Изменится и циркуляция вектора В по контуру \(\Gamma\), так как поверхность, натянутую на контур \(\Gamma\), уже не будут пронизывать токи намагничивания, остается только ток проводимости. Циркуляция же вектора \(\mathbf{H}\) по контуру \(\Gamma\) остается прежней, несмотря на изменение самого поля \(\mathbf{H}\).\\
Когда внутри магнетика \(\mathbf{j}^{\prime}=\mathbf{0}\) ? Мы сейчас покажем, что токи намагничивания внутри магнетика будут отсутствовать, если: 1) магнетик однородный и 2) внутри него нет токов проводимости \((\mathrm{j}=0)\). В этом случае при любой форме магнетика и при любой конфигурации магнитного поля можно быть уверенным, что объемные токи намагничивания равны нулю и остаются только поверхностные токи намагничивания.

Для доказательства этого воспользуемся теоремой о циркуляции вектора \(\mathbf{J}\) по произвольному контуру \(\Gamma\), взятому целиком внутри магнетика. В случае однородного магнетика можно, заменив \(\mathbf{J}\) на \(\chi \mathbf{H}\), вынести в уравнении (7.5) \(\chi\) из-под интеграла и записать

\[
I^{\prime}=\chi \oint \mathbf{H} \mathrm{d} \mathbf{l} .
\]

Оставшийся интеграл равен согласно (7.12) алгебраической сумме токов проводимости \(I\), охватываемых контуром \(\Gamma\), поэтому для однородного магнетика


\begin{equation*}
I^{\prime}=\chi I . \tag{7.17}
\end{equation*}


Это соотношение между токами \(I^{\prime}\) и \(I\) справедливо для любого контура внутри магнетика, в частности и для очень малого контура, когда \(I^{\prime} \rightarrow \mathrm{d} I^{\prime}=j_{n}^{\prime} \mathrm{d} S\) и \(I \rightarrow \mathrm{~d} I=j_{n} \mathrm{~d} S\). Тогда \(j_{n}^{\prime} \mathrm{d} S=\chi j_{n} \mathrm{~d} S\), и после сокращения на \(\mathrm{d} S\) мы получим \(j_{n}^{\prime}=\chi j_{n}\). Последнее равенство выполняется при любой ориентации малого контура, т. е. при любом направлении нормали \(\mathbf{n}\) к нему. А это значит, что таким же равенством связаны и сами векторы \(\mathbf{j}^{\prime}\) и \(\mathbf{j}\) :


\begin{equation*}
\mathbf{j}^{\prime}=\chi \mathbf{j} . \tag{7.18}
\end{equation*}


Отсюда следует, что в однородном магнетике \(\mathbf{j}^{\prime}=0\), если \(\mathbf{j}=0\). Это и требовалось доказать.

\section*{§ 7.4. Граничные условия для В и Н}
Речь идет об условиях для векторов \(\mathbf{B}\) и \(\mathbf{H}\) на границе раздела двух однородных магнетиков. Эти условия, как и в случае диэлектрика, мы получим с помощью теоремы Гаусса и теоремы о циркуляции. Для векторов В и Н эти теоремы, напомним, имеют вид


\begin{equation*}
\oint \mathbf{B} \mathrm{d} \mathbf{S}=0, \quad \oint \mathbf{H} \mathrm{~d} \mathbf{l}=I . \tag{7.19}
\end{equation*}


Условие для вектора В. Представим себе очень малой высоты цилиндрик, расположенный на границе раздела магнетиков, как показано на рис. 7.7. Тогда поток вектора \(\mathbf{B}\) наружу из этого цилиндрика (потоком через боковую поверхность пренебрегаем) можно записать так:

\begin{figure}[h]
\begin{center}
  \includegraphics[alt={},max width=\textwidth]{3e870b25-6320-483d-9133-cdd5212c2707-188_233_408_734_757}
\captionsetup{labelformat=empty}
\caption{Рис. 7.7}
\end{center}
\end{figure}

\[
B_{2 n} \Delta S+B_{1 n^{\prime}} \Delta S=0
\]

Взяв обе проекции вектора \(\mathbf{B}\) на общую нормаль \(\mathbf{n}\), получим \(B_{1 n^{\prime}}=-B_{1 n}\), и предыдущее уравнение после сокращения на \(\Delta S\) примет следующий вид:


\begin{equation*}
B_{2 n}=B_{1 n}, \tag{7.20}
\end{equation*}


т. е. нормальная составляющая вектора В оказывается одинаковой по обе стороны границы раздела. Эта величина скачка не испытывает.

Условия для вектора Н. Для большей общности будем предполагать, что вдоль поверхности раздела магнетиков течет поверхностный ток проводимости с линейной плотностью \(\mathbf{i}\). Применим теорему о циркуляции вектора \(\mathbf{H}\) к очень малому прямоугольному контуру, высота которого пренебрежимо мала по сравнению с его длиной \(l\), расположив этот контур так, как по-

\begin{figure}[h]
\begin{center}
  \includegraphics[alt={},max width=\textwidth]{3e870b25-6320-483d-9133-cdd5212c2707-189_212_386_145_80}
\captionsetup{labelformat=empty}
\caption{Рис. 7.8}
\end{center}
\end{figure}

казано на рис. 7.8. Пренебрегая вкладом в циркуляцию на боковых сторонах контура, запишем для всего контура:

\[
H_{2 \tau} l+H_{1 \tau^{\prime}} l=i_{N} l,
\]

где \(i_{N}\) - проекция вектора \(\mathbf{i}\) на нормаль \(\mathbf{N}\) к контуру (вектор \(\mathbf{N}\) образует с направлением обхода по контуру правовинтовую систему). Взяв обе проекции вектора \(\mathbf{H}\) на общий орт касательной \(\tau\) (в среде 2), получим \(H_{1 \tau^{\prime}}=-H_{1 \tau}\), и после сокращения на \(l\) предыдущее уравнение примет вид


\begin{equation*}
H_{2 \tau}-H_{1 \tau}=i_{N}, \tag{7.21}
\end{equation*}


т. е. тангенциальная составляющая вектора \(\mathbf{H}\), вообще говоря, при переходе границы раздела магнетиков претерпевает скачок, связанный с наличием поверхностных токов проводимости.

Однако если на границе раздела магнетиков токов проводимости нет ( \(\mathbf{i}=\mathbf{0}\) ), то тангенциальная составляющая вектора \(\mathbf{H}\) оказывается одинаковой по обе стороны границы раздела:


\begin{equation*}
H_{2 \tau}=H_{1 \tau} . \tag{7.22}
\end{equation*}


Итак, если на границе раздела двух однородных магнетиков тока проводимости нет, то при переходе этой границы составляющие \(B_{n}\) и \(H_{\tau}\) изменяются непрерывно, без скачка. Составляющие же \(B_{\tau}\) и \(H_{n}\) при этом претерпевают скачок.

Заметим, что на границе раздела вектор В ведет себя аналогично вектору \(\mathbf{D}\), а вектор \(\mathbf{H}\) - аналогично вектору \(\mathbf{E}\).

Преломление линий вектора В. На границе раздела двух магнетиков линии вектора В испытывают преломление (рис. 7.9). Как и в случае диэлектриков, найдем отношение тангенсов углов \(\alpha_{1}\) и \(\alpha_{2}\) :

\[
\frac{\operatorname{tg} \alpha_{2}}{\operatorname{tg} \alpha_{1}}=\frac{B_{2 \tau} / B_{2 n}}{B_{1 \tau} / B_{1 n}} .
\]

\begin{figure}[h]
\begin{center}
  \includegraphics[alt={},max width=\textwidth]{3e870b25-6320-483d-9133-cdd5212c2707-190_358_294_141_197}
\captionsetup{labelformat=empty}
\caption{Рис. 7.9}
\end{center}
\end{figure}

\begin{figure}[h]
\begin{center}
  \includegraphics[alt={},max width=\textwidth]{3e870b25-6320-483d-9133-cdd5212c2707-190_261_515_201_625}
\captionsetup{labelformat=empty}
\caption{Поле В Поле Н}
\end{center}
\end{figure}

Рис. 7.10

Ограничимся случаем, когда на границе раздела тока проводимости нет. В этом случае согласно (7.22) и (7.20):

\[
B_{2 \tau} / \mu_{2}=B_{1 \tau} / \mu_{1}, \quad B_{2 n}=B_{1 n} .
\]

С учетом последних соотношений получим аналогичный (3.25) закон преломления линий В (а значит, и линий Н):


\begin{equation*}
\frac{\operatorname{tg} \alpha_{2}}{\operatorname{tg} \alpha_{1}}=\frac{\mu_{2}}{\mu_{1}} . \tag{7.23}
\end{equation*}


На рис. 7.10 изображено поле векторов В и \(\mathbf{H}\) вблизи границы раздела двух магнетиков (при отсутствии токов проводимости). Здесь \(\mu_{2}>\mu_{1}\); из сравнения густоты линий видно, что \(B_{2}>B_{1}\), а \(H_{2}<H_{1}\). Линии \(\mathbf{B}\) не терпят разрыва при переходе границы, линии же \(\mathbf{H}\) терпят разрыв (из-за поверхностных токов намагничивания).

На преломлении магнитных линий основана магнитная защита. При внесении, например, замкнутой железной оболочки (слоя) во внешнее магнитное поле линии этого поля будут концентрироваться (сгущаться) преимущественно в самой оболочке. Внутри же этой оболочки - в полости - магнитное поле оказывается сильно ослабленным по сравнению с внешним полем. Другими словами, железная оболочка обладает экранирующим действием. Это используют для предохранения чувствительных приборов от внешних магнитных полей.

\section*{§ 7.5. Поле в однородном магнетике}
Как уже было отмечено в § 7.1, нахождение результирующего магнитного поля В при наличии произвольных магнетиков представляет собой, вообще говоря, весьма сложную задачу. Действительно, для этого необходимо согласно (7.1) к полю \(\mathbf{B}_{0}\) токов проводимости добавить макрополе \(\mathbf{B}^{\prime}\), создаваемое токами намагничивания. Неприятность состоит в том, что нам заранее не известна конфигурация токов намагничивания. Мы можем лишь утверждать, что распределение этих токов зависит от природы и конфигурации магнетика, а также от конфигурации внешнего поля \(\mathbf{B}_{0}\) - поля токов проводимости. А поскольку мы не знаем распределения токов намагничивания, мы не можем рассчитать и поле \(\mathbf{B}^{\prime}\).

Исключение составляет случай, когда все пространство, где имеется поле В, заполнено однородным изотропным магнетиком. Рассмотрим этот случай более подробно. Но прежде всего обратимся к явлениям, возникающим при протекании тока проводимости по однородному проводнику в вакууме. Так как каждый проводник является магнетиком, то в нем будут протекать и токи намагничивания - объемные согласно (7.18) и поверхностные. Возьмем контур, охватывающий наш проводник с током. По теореме о циркуляции вектора \(\mathbf{J}(7.5)\), поскольку во всех точках контура \(\mathbf{J}=0\), алгебраическая сумма токов намагничивания (объемных и поверхностных) равна нулю: \(I^{\prime}=I_{\text {об }}^{\prime}+I_{\text {пов }}^{\prime}=0\). Отсюда \(I_{\text {об }}^{\prime}=-I_{\text {пов }}^{\prime}\), т. е. объемные и поверхностные токи намагничивания равны и противоположны по направлению.

Таким образом, можно утверждать, что в обычных случаях, когда токи текут по достаточно тонким проводам, магнитное поле в окружающем пространстве (в вакууме) зависит только от токов проводимости, ибо поля от токов намагничивания компенсируют друг друга.

Теперь заполним окружающее проводник пространство однородным непроводящим магнетиком (пусть для конкретности это будет парамагнетик, \(\chi>0\) ). На границе этого магнетика с проводом появится поверхностный ток намагничивания \(I^{\prime}\), имеющий, как нетрудно сообразить, то же направление, что и ток проводимости \(I\) (это при \(\chi>0\) ).

В результате мы будем иметь ток проводимости \(I\), объемный и поверхностный токи намагничивания в проводнике (магнитные поля этих токов компенсируют друг друга, поэтому их можно не учитывать в дальнейшем) и поверхностный ток намагничивания \(I^{\prime}\) на непроводящем магнетике. При достаточно тонких проводах магнитное поле В в магнетике будет определяться как поле тока \(I+I^{\prime}\).

Таким образом, задача сводится к нахождению тока \(I^{\prime}\). С этой целью окружим проводник контуром, расположенном в поверхностном слое непроводящего магнетика. Пусть плоскость контура перпендикулярна оси провода, т. е. токам намагничивания. Тогда, принимая во внимание (7.7) и (7.14), можно записать:

\[
I^{\prime}=\oint i^{\prime} \mathrm{d} l=\oint J \mathrm{~d} l=\chi \oint H \mathrm{~d} l .
\]

Отсюда согласно (7.12) следует, что \(I^{\prime}=\chi I\).\\
Конфигурации тока намагничивания \(I^{\prime}\) и тока проводимости \(I\) практически совпадают (провода тонкие), поэтому индукция \(\mathbf{B}^{\prime}\) поля токов намагничивания отличается от индукции \(\mathbf{B}_{0}\) поля токов проводимости во всех точках только по модулю и эти векторы связаны друг с другом так же, как и соответствующие токи, а именно:


\begin{equation*}
\mathbf{B}^{\prime}=\chi \mathbf{B}_{0} . \tag{7.24}
\end{equation*}


Тогда результирующее поле \(\mathbf{B}=\mathbf{B}_{0}+\mathbf{B}^{\prime}=(1+\chi) \mathbf{B}_{0}\), или


\begin{equation*}
\mathbf{B}=\mu \mathbf{B}_{0} . \tag{7.25}
\end{equation*}


Это значит, что В при заполнении пространства однородным магнетиком возрастает в \(\mu\) раз. Иначе говоря, величина \(\mu\) показывает, во сколько раз увеличивается магнитная индукция В при заполнении магнетиком всего пространства, занимаемого полем.

Если разделить обе части равенства (7.25) на \(\mu \mu_{0}\), то получим


\begin{equation*}
\mathbf{H}=\mathbf{H}_{0} \tag{7.26}
\end{equation*}


(в рассматриваемом случае поле \(\mathbf{H}\) оказывается таким же, как и в вакууме).

Формулы (7.24)-(7.26) справедливы и в тех случаях, когда однородный магнетик заполняет весь объем, ограниченный поверхностями, которые образованы линиями вектора \(\mathbf{B}_{0}\) (поля тока проводимости). И в этих случаях магнитная индукция В внутри магнетика будет в \(\mu\) раз больше \(\mathbf{B}_{0}\).

В указанных случаях магнитная индукция \(\mathbf{B}^{\prime}\) поля токов намагничивания связана простым соотношением с намагниченностью J магнетика:


\begin{equation*}
\mathbf{B}^{\prime}=\mu_{0} \mathbf{J} . \tag{7.27}
\end{equation*}


Это выражение можно легко получить из формулы \(\mathbf{B}=\mathbf{B}_{0}+\mathbf{B}^{\prime}\), если учесть, что \(\mathbf{B}^{\prime}=\chi \mathbf{B}_{0}\) и \(\mathbf{B}=\mu \mu_{0} \mathbf{H}\), где \(\mathbf{H}=\mathbf{J} / \chi\).

В других случаях, как уже было сказано, дело обстоит значительно сложнее, и формулы (7.24)-(7.27) оказываются не справедливыми. В заключение рассмотрим два простых примера.\\
Пример 1. Поле в в соленоиде. Пусть соленоид, имеющий \(n I\) ам-пер-витков на единицу длины, заполнен однородным магнетиком с магнитной проницаемостью \(\mu>1\). Найдем магнитную индукцию \(B\) поля в магнетике.\\
При отсутствии магнетика согласно (6.20) внутри соленоида магнитная индукция \(B_{0}=\mu_{0} n I\). Так как магнетик заполняет все пространство, где поле отлично от нуля (краевыми эффектами мы пренебрегаем), то магнитная индукция \(B\) должна быть в \(\mu\) раз больше:


\begin{equation*}
B=\mu \mu_{0} n I . \tag{7.28}
\end{equation*}


В этом случае поле вектора \(\mathbf{H}\) остается тем же, что и при отсутствии магнетика, т. е. \(\mathbf{H}=\mathbf{H}_{0}\).\\
Изменение поля В вызвано появлением токов намагничивания, обтекающих поверхность магнетика в том же направлении, что и ток проводимости в обмотке соленоида, это при \(\mu>1\). Если же \(\mu<1\), то направления указанных токов будут противоположными.

Полученные результаты справедливы и в случае, когда магнетик имеет вид очень длинного стержня, расположенного внутри соленоида параллельно его оси.

Пример 2. Поле прямого тока при наличии магнетика. Предположим, что магнетик заполняет длинный цилиндр радиусом \(a\), вдоль оси которого течет заданный ток \(I\). Проницаемость

магнетика \(\mu>1\). Найдем магнитную индукцию \(B\) в зависимости от расстояния \(r\) до оси цилиндра.

Непосредственно воспользоваться теоремой о циркуляции вектора В нельзя, так как не известны токи намагничивания. Положение спасает вектор H: его циркуляция определяется только токами проводимости. Для окружности радиусом \(r\) имеем \(2 \pi r H=I\), откуда

\[
B=\mu \mu_{0} H=\mu \mu_{0} I / 2 \pi r .
\]

При переходе границы раздела магнетик - вакуум магнитная индукция \(B\) претерпевает скачок в отличие от \(H\) (рис. 7.11).\\
Усиление \(B\) внутри магнетика вызвано появлением поверхностных токов намагничивания: у провода на оси системы эти токи совпадают по направлению с током \(I\), а значит, «усиливают» ток \(I\), снаружи

\begin{figure}[h]
\begin{center}
  \includegraphics[alt={},max width=\textwidth]{3e870b25-6320-483d-9133-cdd5212c2707-194_338_350_516_806}
\captionsetup{labelformat=empty}
\caption{Рис. 7.11}
\end{center}
\end{figure}

же цилиндра поверхностный ток намагничивания направлен в противоположную сторону, но он не оказывает влияния на поле \(B\) в магнетике. Вне магнетика магнитные поля обоих токов намагничивания компенсируют друг друга.

\section*{§ 7.6. Ферромагнетизм}
Ферромагнетики. B магнитном отношении все вещества можно разделить на слабомагнитные (парамагнетики и диамагнетики) и сильномагнитные (ферромагнетики). Пара- и диамагнетики при отсутствии магнитного поля, как мы знаем, не намагничены и характеризуются однозначной зависимостью (7.14) намагниченности \(\mathbf{J}\) от \(\mathbf{H}\).

Ферромагнетиками называют вещества (твердые), которые могут обладать спонтанной намагниченностью, т. е. намагничены уже при отсутствии внешнего магнитного поля. Типичные представители ферромагнетиков - это железо, кобальт и многие их сплавы.

Основная кривая намагничения. Характерной особенностью ферромагнетиков является сложная нелинейная зависимость \(\mathbf{J} \boldsymbol{(} \mathbf{H} \boldsymbol{)}\) или \(\mathbf{B} \boldsymbol{(} \mathbf{H} \boldsymbol{)}\). На рис. 7.12 дана кривая намагничения ферро-

\begin{figure}[h]
\begin{center}
  \includegraphics[alt={},max width=\textwidth]{3e870b25-6320-483d-9133-cdd5212c2707-195_304_384_141_153}
\captionsetup{labelformat=empty}
\caption{Рис. 7.12}
\end{center}
\end{figure}

\begin{figure}[h]
\begin{center}
  \includegraphics[alt={},max width=\textwidth]{3e870b25-6320-483d-9133-cdd5212c2707-195_298_347_144_713}
\captionsetup{labelformat=empty}
\caption{Рис. 7.13}
\end{center}
\end{figure}

магнетика, намагниченность которого при \(H=0\) тоже равна нулю, ее называют основной кривой намагничения. Уже при сравнительно небольших значениях \(H\) намагниченность \(J\) достигает насыщения \(J_{\text {нас }}\). Магнитная индукция \(B=\mu_{0}(H+J)\) также растет с увеличением \(H\), а после достижения состояния насыщения \(B\) продолжает расти с увеличением \(H\) по линейному закону: \(B=\mu_{0} H+\) const, где const \(=\mu_{0} J_{\text {нас }}\). На рис. 7.13 приведена основная кривая намагничения на диаграмме \(B-H\). Ввиду нелинейной зависимости \(B(H)\) для ферромагнетиков нельзя ввести магнитную проницаемость \(\mu\) как определенную постоянную величину, характеризующую маг-

\begin{figure}[h]
\begin{center}
  \includegraphics[alt={},max width=\textwidth]{3e870b25-6320-483d-9133-cdd5212c2707-195_290_393_1026_67}
\captionsetup{labelformat=empty}
\caption{Рис. 7.14}
\end{center}
\end{figure}

Заметим, что понятие магнитной проницаемости применяют только к основной кривой намагничения, ибо, как мы сейчас увидим, зависимость \(B(H)\) неоднозначна.

Магнитный гистерезис. Кроме нелинейной зависимости \(B(H)\) или \(J(H)\) для ферромагнетиков характерно также явление магнитного гистерезиса: связь между \(B\) и \(H\) или \(J\) и \(H\) оказывается неоднозначной, а определяется предшествующей историей намагничивания ферромагнетика. Если первоначально ненамагниченный ферромагнетик намагничивать, увеличивая\\
\(H_{\tau}\) от нуля до значения, при котором наступает насыщение (точка 1 на рис. 7.15), а затем уменьшать \(H_{\tau}\) от \(H_{1}\) до \(-H_{1}\), то кривая намагничения \(B_{\tau}\left(H_{\tau}\right)\) пойдет не по первоначальному пути 10, а выше - по пути 1234. Если дальше изменять \(H_{\tau}\) в обратном направлении от \(-H_{1}\) до \(+H_{1}\), то кривая намагничения пройдет ниже - по пути 4561.

Здесь индекс \(\tau\) означает проекцию

\begin{figure}[h]
\begin{center}
  \includegraphics[alt={},max width=\textwidth]{3e870b25-6320-483d-9133-cdd5212c2707-196_372_402_155_758}
\captionsetup{labelformat=empty}
\caption{Рис. 7.15}
\end{center}
\end{figure}

Получившуюся замкнутую кривую называют петлей гистеpeзuca. В том случае, когда в точках 1 и 4 достигается насыщение, получается максимальная петля гистерезиса. Когда же в крайних точках насыщения нет, получаются аналогичные петли гистерезиса, но меньшего размера, как бы вписанные в максимальную петлю гистерезиса.

Из рис. 7.15 видно, что при \(H_{\tau}=0\) намагничение не исчезает (точка 2) и характеризуется величиной \(B_{r}\), называемой остаточной индукцией. Ей соответствует остаточная намагниченность \(J_{r}\). С наличием такого остаточного намагничения связано существование постоянных магнитов. Величина \(B_{\tau}\) обращается в нуль (точка 3 ) лишь под действием поля \(H_{c}\), имеющего направление, противоположное полю, вызвавшему намагничение. Величину \(H_{c}\) называют коэрцитивной силой.

Значения \(B_{r}\) и \(H_{c}\) для разных ферромагнетиков меняются в широких пределах. Для трансформаторного железа петля гистерезиса узкая ( \(H_{c}\) мало), для ферромагнетиков, используемых для изготовления постоянных магнитов, - широкая ( \(H_{c}\) велико, например, для сплава алнико \(H_{c}=50000 \mathrm{~A} / \mathrm{m}\), \(\mathrm{B}_{r}=0,9\) Тл).

На этих особенностях кривых намагничения основан удобный практический прием для размагничивания ферромагнетика. Намагниченный образец помещают в катушку, по которой пропускают переменный ток и амплитуду его постепенно уме-

ньшают до нуля. При этом ферромагнетик подвергается многократным циклическим перемагничиваниям, в которых петли гистерезиса постепенно уменьшаются, стягиваясь к точке \(O\), где намагниченность равна нулю.

Опыт показывает, что при перемагничивании ферромагнетик нагревается. Можно показать, что в единице объема ферромагнетика выделяется при этом теплота \(Q_{\text {ед }}\), численно равная «площади» \(S_{\text {п }}\) петли гистерезиса:


\begin{equation*}
Q_{\text {ед }}=\oint H_{\tau} \mathrm{d} B_{\tau}=S_{\Pi} . \tag{7.29}
\end{equation*}


Температура Кюри. При повышении температуры способность ферромагнетиков намагничиваться уменьшается, в частности, уменьшается намагниченность насыщения. При некоторой температуре, называемой температурой или точкой Кюри, ферромагнитные свойства исчезают.

При температурах, более высоких, чем температура Кюри, ферромагнетик превращается в парамагнетик.

О теории ферромагнетизма. Физическую природу ферромагнетизма удалось понять только с помощью квантовой физики. При определенных условиях в кристаллах могут возникать так называемые обленные силь, которые заставляют магнитные моменты электронов устанавливаться параллельно друг другу. В результате возникают области (размером \(1-10\) мкм) спонтанного, т. е. самопроизвольного, намагничения - эти области называют доменами. В пределах каждого домена ферромагнетик намагничен до насыщения и имеет определенный магнитный момент. Направления этих моментов для разных доменов различны, поэтому при отсутствии внешнего поля суммарный момент образца равен нулю и образец в целом представляется макроскопически ненамагниченным.

При включении внешнего магнитного поля домены, ориентированные по полю, растут за счет доменов, ориентированных против поля. Такой рост в слабых полях имеет обратимый характер. В более сильных полях происходит одновременная переориентация магнитных моментов в пределах всего домена. Этот процесс является необратимым, что и служит причиной гистерезиса и остаточного намагничения.

\section*{Задачи}
7.1. Условия на границе раздела. Вблизи точки \(A\) (рис. 7.16) границы раздела магнетик - вакуум магнитная индукция в вакууме равна \(B_{0}\), причем вектор \(\mathbf{B}_{0}\) составляет угол \(\alpha_{0}\) с нормалью к границе раздела в данной точке. Магнитная проницаемость магнетика равна \(\mu\). Найти магнитную индукцию \(B\) в магнетике вблизи той же точки \(A\).

\begin{figure}[h]
\begin{center}
  \includegraphics[alt={},max width=\textwidth]{3e870b25-6320-483d-9133-cdd5212c2707-198_246_286_230_876}
\captionsetup{labelformat=empty}
\caption{Рис. 7.16}
\end{center}
\end{figure}

Решение. Искомая величина


\begin{equation*}
B=\sqrt{B_{n}^{2}+B_{\tau}^{2}} \tag{1}
\end{equation*}


Имея в виду условия (7.20) и (7.22) на границе раздела, найдем

\[
\begin{gathered}
B_{n}=B_{0} \cos \alpha_{0}, \\
\mathrm{~B}_{\tau}=\mu \mu_{0} H_{\tau}=\mu \mu_{0} H_{0 \tau}=\mu B_{0 \tau}=\mu B_{0} \sin \alpha_{0}
\end{gathered}
\]

где \(H_{0 \tau}\) - тангенциальная составляющая вектора \(\mathbf{H}_{0}\) в вакууме. Подставив эти выражения в (1), получим

\[
B=B_{0} \sqrt{\cos ^{2} \alpha_{0}+\mu^{2} \sin ^{2} \alpha_{0}} .
\]

7.2. Поверхностный ток намагничивания. Длинный тонкий проводник с током \(I\) расположен перпендикулярно плоской границе раздела вакуум магнетик (рис. 7.17). Проницаемость магнетика \(\mu\). Найти линейную плотность поверхностного тока намагничивания \(i^{\prime}\) на этой границе раздела в зависимости от расстояния \(r\) до проводника. Решение. Прежде всего о конфигурации поверх-

\begin{figure}[h]
\begin{center}
  \includegraphics[alt={},max width=\textwidth]{3e870b25-6320-483d-9133-cdd5212c2707-198_243_238_1070_922}
\captionsetup{labelformat=empty}
\caption{Рис. 7.17}
\end{center}
\end{figure}

ностного тока намагничивания. Из рис. 7.17 нетрудно сообразить, что этот ток направлен радиально. Воспользуемся теоремой о циркуляции намагниченности \(\mathbf{J}\), взяв в качестве контура небольшой прямоугольник, плоскость которого перпендикулярна току намагничивания в данном месте. Расположение этого контура показано на рис. 7.18, где крестиками отмечено направление поверхностного тока намагничивания. Из равенства \(J l=i^{\prime} l\) получим \(i^{\prime}=J\).

\begin{figure}[h]
\begin{center}
  \includegraphics[alt={},max width=\textwidth]{3e870b25-6320-483d-9133-cdd5212c2707-198_161_312_1521_842}
\captionsetup{labelformat=empty}
\caption{Рис. 7.18}
\end{center}
\end{figure}

Далее, \(J=\chi H\), где \(H\) находим из циркуляции вектора \(\mathbf{H}\) по окружности радиусом \(r\) с центром на оси проводника: \(2 \pi r H=I\) (из соображений симметрии ясно, что линии вектора \(\mathbf{H}\) должны иметь вид окружностей, лежащих в плоскостях, перпендикулярных проводнику с током \(I\) ). В результате находим

\[
i^{\prime}=(\mu-1) I / 2 \pi r .
\]

7.3. Циркуляция вектора Н. Прямой длинный тонкий проводник с током \(I\) лежит в плоскости, отделяющей пространство, которое заполнено непроводящим магнетиком с проницаемостью \(\mu\), от вакуума. Найти магнитную индукцию \(B\) во всем пространстве как функцию расстояния \(r\) до проводника. Иметь в виду, что линии вектора В являются окружностями с центром на оси проводника.\\
Решение. Ясно, что линии вектора \(\mathbf{H}\) являются тоже окружностями, причем на границе раздела вакуум - магнетик вектор H будет испытывать скачок (в отличие от вектора В). Обозначим Н и \(\mathbf{H}_{0}\) магнитное поле соответственно в магнетике и вакууме. Тогда по теореме о циркуляции вектора \(\mathbf{H}\) по контуру, имеющему вид окружности радиусом \(r\) с центром на оси проводника, имеем


\begin{equation*}
\pi r H+\pi r H_{0}=I . \tag{1}
\end{equation*}


Кроме того, на границе раздела \(B=B_{0}\) или


\begin{equation*}
\mu H=H_{0} . \tag{2}
\end{equation*}


Решив совместно уравнения (1) и (2), получим

\[
H=\frac{I}{(1+\mu) \pi r}, \quad B=\mu \mu_{0} H=\frac{\mu \mu_{0} I}{(1+\mu) \pi r} .
\]

Конфигурация полей В и \(\mathbf{H}\) в данном случае показана на рис. 7.19. Полезно убедиться в том, что при \(\mu=1\) мы приходим к известным нам формулам для \(B\) и \(H\) в вакууме.

\begin{figure}[h]
\begin{center}
  \includegraphics[alt={},max width=\textwidth]{3e870b25-6320-483d-9133-cdd5212c2707-199_307_582_1454_324}
\captionsetup{labelformat=empty}
\caption{Рис. 7.19}
\end{center}
\end{figure}

7.4. Циркуляция векторов Н и Ј. Постоянный ток \(I\) течет вдоль длинного однородного цилиндрического провода круглого сечения радиусом \(R\). Материалом провода является парамагнетик с восприимчивостью \(\chi\). Найти: 1) зависимость индукции \(B\) от расстояния \(r\) до оси провода; 2) плотность тока намагничивания \(j^{\prime}\) внутри провода.

Решение. 1. Из циркуляции вектора \(\mathbf{H}\) по окружности радиусом \(r\) с центром на оси провода следует, что

\[
\begin{array}{ll}
r<R, & 2 \pi r H=I(r / R)^{2}, \quad B \sim(1+\chi) r / R^{2}, \\
r>R, & 2 \pi r H=I, \quad B \sim 1 / r .
\end{array}
\]

На рис. 7.20 показаны графики зависимостей \(H(r)\) и \(B(r)\).\\
2. Воспользуемся теоремой о циркуляции намагниченности \(\mathbf{J}\) по окружности радиусом \(r\) (см. рис. 7.20 ): \(2 \pi r J=I^{\prime}\), где \(I^{\prime}\) - ток намагничивания, охватываемый этим контуром. Найдем дифференциал этого выражения (при перехо-

\begin{figure}[h]
\begin{center}
  \includegraphics[alt={},max width=\textwidth]{3e870b25-6320-483d-9133-cdd5212c2707-200_285_378_625_774}
\captionsetup{labelformat=empty}
\caption{Рис. 7.20}
\end{center}
\end{figure}

де от \(r к r+\mathrm{d} r\) ):

\[
2 \pi \mathrm{~d}(r J)=\mathrm{d} I^{\prime} .
\]

Так как \(\mathrm{d} I^{\prime}=j^{\prime} 2 \pi r \mathrm{~d} r\), то предыдущее уравнение можно преобразовать к виду

\[
j^{\prime}=\frac{J}{r}+\frac{\mathrm{d} J}{\mathrm{~d} r}
\]

Теперь учтем, что \(J=\chi H=\left(\chi I / 2 \pi R^{2}\right) r\). Тогда получим

\[
j^{\prime}=\chi I / \pi R^{2} .
\]

Нетрудно сообразить, что этот ток течет в ту же сторону, что и ток проводимости (в отличие от поверхностного тока намагничивания, текущего в противоположную сторону).\\
7.5. Длинный соленоид заполнен неоднородным изотропным парамагнетиком, восприимчивость которого зависит только от расстояния \(r\) до оси соленоида как \(\chi=a r^{2}\), где \(a\) - постоянная. На оси соленоида магнитная индукция равна \(B_{0}\). Найти зависимость от расстояния \(r: 1\) ) намагниченности, \(J(r) ; 2\) ) плотности тока намагничивания, \(j^{\prime}(r)\).

Решение. 1. Намагниченность \(J=\chi H\). В нашем случае \(H\) не зависит от \(r\) (это непосредственно следует из циркуляции вектора \(\mathbf{H}\) по контуру, показанному на рис. 7.21 слева). Поэтому \(H=H_{0}\) - на оси соленоида, и мы получаем

\[
J=a r^{2} H_{0}=a r^{2} B_{0} / \mu_{0}
\]

\begin{enumerate}
  \setcounter{enumi}{1}
  \item Из теоремы о циркуляции намагниченности \(\mathbf{J}\) по бесконечно уз-
\end{enumerate}

\begin{figure}[h]
\begin{center}
  \includegraphics[alt={},max width=\textwidth]{3e870b25-6320-483d-9133-cdd5212c2707-201_344_221_452_146}
\captionsetup{labelformat=empty}
\caption{Рис. 7.21}
\end{center}
\end{figure}

кому контуру, показанному на рис. 7.21 справа, следует

\[
J l-(J+\mathrm{d} J) l=j_{n}^{\prime} l \mathrm{~d} r,
\]

где \(l\) - высота контура, \(\mathrm{d} r\) - его ширина. Отсюда

\[
j_{n}^{\prime}=-\frac{\mathrm{d} J}{\mathrm{~d} r}=-\frac{2 a B_{0}}{\mu_{0}} r .
\]

Знак минус показывает, что вектор j' направлен против вектора нормали \(\mathbf{n}\), образующего с направлением обхода контура правовинтовую систему. Другими словами, вектор j' направлен в месте расположения правого (на рисунке) контура на нас, т. е. объемные токи намагничивания образуют c вектором \(\mathbf{B}_{0}\) левовинтовую систему.\\
7.6. Постоянный магнит имеет вид кольца с узким зазором между полюсами. Средний диаметр кольца равен \(d\).

\begin{figure}[h]
\begin{center}
  \includegraphics[alt={},max width=\textwidth]{3e870b25-6320-483d-9133-cdd5212c2707-201_304_350_1147_129}
\captionsetup{labelformat=empty}
\caption{Рис. 7.22}
\end{center}
\end{figure}

Ширина зазора \(b\), магнитная индукция поля в зазоре В. Пренебрегая рассеянием поля на краях зазора, найти модули векторов Н и \(\mathbf{J}\) внутри вещества.

Решение. Воспользовавшись теоремой о циркуляции вектора \(\mathbf{H}\) по пунктирной окружности диаметром \(d\) (рис. 7.22) и учитывая, что токов проводимости нет, запишем

\[
(\pi d-b) H_{\tau}+b B / \mu_{0}=0
\]

где \(H_{\tau}\) - проекция вектора \(\mathbf{H}\) на направление обхода контура (оно взято совпадающим с направлением вектора \(\mathbf{B}\) в зазоре). Отсюда


\begin{equation*}
H_{\tau}=-\frac{b B}{\mu_{0}(\pi d-b)} \approx-\frac{b B}{\mu_{0} \pi d} . \tag{1}
\end{equation*}


Знак минус показывает, что направление вектора \(\mathbf{H}\) внутри вещества магнита противоположно вектору В в той же точке. Заметим, что при \(b \rightarrow 0\) и \(H \rightarrow 0\).

Модуль намагниченности \(\mathbf{J}\) найдем по формуле (7.11), используя результат (1):

\[
J=\frac{B / \mu_{0}}{1-b / \pi d} \approx \frac{B}{\mu_{0}} .
\]

Соотношение между векторами \(\mathbf{B} / \mu_{0}\), H и \(\mathbf{J}\) в любой точке вещества магнита показано на рис. 7.23.

\begin{figure}[h]
\begin{center}
  \includegraphics[alt={},max width=\textwidth]{3e870b25-6320-483d-9133-cdd5212c2707-202_96_319_423_841}
\captionsetup{labelformat=empty}
\caption{Рис. 7.23}
\end{center}
\end{figure}

7.7. На железном сердечнике в виде тора со средним диаметром \(d\) имеется обмотка с общим числом витков \(N\). В сердечнике сделана узкая поперечная прорезь шириной \(b\) (см. рис. 7.22). При токе \(I\) через обмотку магнитная индукция в прорези равна \(B\). Пренебрегая рассеянием поля на краях прорези, найти магнитную проницаемость железа в этих условиях.

Решение. Согласно теореме о циркуляции вектора \(\mathbf{H}\) по окружности диаметром \(d\) (см. рис. 7.22) имеем

\[
(\pi d-b) H+b H_{0}=N I,
\]

где \(H\) и \(H_{0}\) - модули вектора \(\mathbf{H}\) соответственно в железе и прорези. Кроме того, отсутствие рассеяния поля на краях прорези означает, что

\[
B=B_{0} .
\]

Из этих двух уравнений с учетом того, что \(B=\mu \mu_{0} H\) и \(b \ll d\), получим

\[
\mu=\frac{\pi d B}{\mu_{0} N I-b B} .
\]

7.8. Сила, действующая на магнетик. В установке (рис. 7.24) с помощью весов измеряют силу, с которой небольшой парамагнитный шарик объемом \(V\) притягивается к полюсу магнита \(M\). Магнитная индукция на оси полюсного наконечника зависит от высоты \(x\) как \(B=B_{0} \exp \left(-a x^{2}\right)\), где \(B_{0}\) и \(a-\) постоянные. Найти: 1) на какой высоте \(x_{m}\) надо поместить шарик, чтобы сила притяжения была максимальной; 2) магнитную

\begin{figure}[h]
\begin{center}
  \includegraphics[alt={},max width=\textwidth]{3e870b25-6320-483d-9133-cdd5212c2707-202_382_319_1367_841}
\captionsetup{labelformat=empty}
\caption{Рис. 7.24}
\end{center}
\end{figure}

восприимчивость парамагнетика, если максимальная сила притяжения равна \(F_{m}\).

Решение. 1. Пусть для определенности вектор В на оси направлен вверх, туда же направим и ось \(X\). Тогда согласно (6.34) \(F_{x}= =p_{\mathrm{m}} \partial B / \partial x\), где учтено, что магнитный момент \(\mathbf{p}_{\mathrm{m}}\) направлен туда же, куда и вектор В (для парамагнетика), поэтому \(\partial n\) заменено на \(\partial x\).

Далее, так как \(p_{\mathrm{m}}=J V=\chi H V\) и \(\partial B / \partial x=-2 a B_{0} x \exp \left(-a x^{2}\right)\), то


\begin{equation*}
F_{x}=-A x \exp \left(-2 a x^{2}\right), \tag{1}
\end{equation*}


где \(A=2 a B_{0}^{2} \chi V / \mu \mu_{0}\).\\
Вычислив производную \(\mathrm{d} F_{x} / \mathrm{d} x\) и приравняв ее к нулю, получим следующее уравнение для определения \(x_{m}: 1-4 a x^{2}=0\), откуда


\begin{equation*}
x_{m}=1 / \sqrt{4 a} . \tag{2}
\end{equation*}


\begin{enumerate}
  \setcounter{enumi}{1}
  \item После подстановки (2) в (1) найдем
\end{enumerate}

\[
\chi=\frac{\mu_{0} F_{m}}{B_{0}^{2} V} \sqrt{\frac{\mathrm{e}}{a}},
\]

где учтено, что для парамагнетика \(\mu \approx 1\).\\
7.9. Длинный тонкий цилиндрический стержень из парамагнетика с магнитной восприимчивостью \(\chi\) и

\begin{figure}[h]
\begin{center}
  \includegraphics[alt={},max width=\textwidth]{3e870b25-6320-483d-9133-cdd5212c2707-203_221_426_1129_130}
\captionsetup{labelformat=empty}
\caption{Рис. 7.25}
\end{center}
\end{figure}

площадью поперечного сечения \(S\) расположен вдоль оси катушки с током. Один конец стержня находится в центре катушки, где магнитное поле равно \(B\), а другой конец - в области, где магнитное поле практически отсутствует. С какой силой катушка действует на стержень?

Решение. Выделим мысленно элемент стержня длиной \(\mathrm{d} x\) (рис. 7.25). На него действует сила

\[
\mathrm{d} F_{x}=\mathrm{d} p_{\mathrm{m}} \frac{\partial B_{x}}{\partial n}
\]

Пусть вектор В на оси катушки направлен вправо (на рисунке), в сторону положительных \(x\). Тогда \(B_{x}=B, \partial n=\partial x\), и так как \(\mathrm{d} p_{\mathrm{m}}=J S \mathrm{~d} x=\chi H S \mathrm{~d} x\), то

\[
\mathrm{d} F_{x}=\chi H S \mathrm{~d} x \frac{\partial B}{\partial x}=\frac{\chi S}{\mu \mu_{0}} B \mathrm{~d} B
\]

Проинтегрировав это выражение, получим

\[
F_{x}=\frac{\chi S}{\mu \mu_{0}} \int_{B}^{0} B \mathrm{~d} B=-\frac{\chi S B^{2}}{2 \mu \mu_{0}} .
\]

Знак минус показывает, что вектор \(\mathbf{F}\) направлен влево, т. е. стержень притягивается к катушке с током.\\
7.10. Небольшой шарик объемом \(V\) из парамагнетика с магнитной восприимчивостью \(\chi\) переместили вдоль оси катушки с током из точки, где магнитная индукция равна \(B\), в область, где поле практически отсутствует. Какую при этом совершили работу против магнитных сил?

Решение. Направим ось \(X\) вдоль оси катушки. Тогда элементарная работа против магнитных сил при перемещении шарика на \(\mathrm{d} x\) будет иметь вид


\begin{equation*}
\delta A=-F_{x} \mathrm{~d} x=-p_{\mathrm{m}} \frac{\partial B_{x}}{\partial n} \mathrm{~d} x \tag{1}
\end{equation*}


где \(F_{x}\) - проекция на ось \(X\) магнитной силы (6.34), а знак минус означает, что работа производится против этой силы.\\
Пусть вектор В на оси направлен в сторону положительных \(x\), тогда \(B_{x}=B\) и \(\partial n=\partial x\) (в противном случае \(B_{x}=-B, \partial n=-\partial x\), т. е. производная \(\partial B_{x} / \partial n\) не зависит от того, куда направлен вектор В). Учитывая, что \(p_{\mathrm{m}}=J V=\chi H V\), перепишем уравнение (1) в виде

\[
\delta A=-\chi N V \frac{\partial B}{\partial x} \mathrm{~d} x=-\frac{\chi V}{\mu \mu_{0}} B \mathrm{~d} B .
\]

Проинтегрировав это выражение от \(B\) до 0 , получим

\[
A=-\frac{\chi V}{\mu \mu_{0}} \int_{B}^{0} B \mathrm{~d} B=\frac{\chi B^{2} V}{2 \mu \mu_{0}} .
\]

Следует обратить внимание на тот факт, что полученный результат - работа \(A\) - не зависит от характера зависимости \(B(x)\).

\section*{Относительность электрического и магнитного полей}
\section*{§ 8.1. Электромагнитное поле. Инвариантность заряда}
До сих пор мы рассматривали электрическое и магнитное поля раздельно, не обнаруживая никакой видимой связи между ними. Это возможно было сделать лишь потому, что оба поля являлись статическими, в других же случаях так поступать нельзя.

Мы увидим, что электрическое и магнитное поля всегда должны рассматриваться вместе как одно полное электромагнитное поле. Другими словами, оказывается, что электрическое и магнитное поля являются в некотором смысле различными компонентами единого физического объекта, который мы называем электромагнитным полем.

Деление же электромагнитного поля на электрическое и магнитное имеет относительный характер: такое деление в решающей степени зависит от системы отсчета, в которой рассматриваются явления. При этом поле, постоянное в одной системе отсчета, в общем случае оказывается переменным в другой системе. Приведем некоторые примеры.

Заряд движется в инерциальной \(K\)-системе отсчета с постоянной скоростью \(v\). В этой системе отсчета мы будем наблюдать как электрическое, так и магнитное поля данного заряда, причем оба поля переменные во времени. Если же перейти в инерциальную \(K^{\prime}\)-систему, перемещающуюся вместе с зарядом, то в ней заряд покоится и мы будем наблюдать только электрическое поле.

Два одинаковых заряда движутся в \(K\)-системе отсчета навстречу друг другу с одинаковой скоростью \(u\). В этой системе отсчета мы будем наблюдать и электрическое, и магнитное поля, оба переменные. Найти такую \(K^{\prime}\)-систему, где наблюдалось бы только одно из полей, в данном случае нельзя.

В \(K\)-системе отсчета существует постоянное неоднородное магнитное поле (например, поле неподвижного постоянного магнита). Тогда в \(K^{\prime}\)-системе, движущейся относительно \(K\)-системы, мы будем наблюдать переменное магнитное поле, и как увидим далее, электрическое поле.

Таким образом, становится ясным, что соотношения между электрическим и магнитным полями оказываются разными в различных системах отсчета.

Прежде чем обратиться к основному содержанию этой главы - законам преобразования полей при переходе от одной системы отсчета к другой, выясним следующий важный для дальнейшего вопрос: как ведут себя при таких переходах сам электрический заряд \(q\) и теорема Гаусса для вектора Е.

Инвариантность заряда. В настоящее время имеются исчерпывающие доказательства того, что полный заряд изолированной системы не меняется при изменении движения носителей заряда.

В качестве доказательства можно сослаться на нейтральность газа, состоящего из молекул водорода. В этих молекулах электроны движутся со значительно большими скоростями, нежели протоны. Поэтому если бы заряд зависел от скорости, то заряды электронов и протонов не были бы скомпенсированы газ оказался бы заряженным. Наблюдения же никакого заряда не обнаружили (с точностью до \(10^{-20}\) !).

Или, например, нагрев куска вещества. Поскольку масса электрона значительно меньше массы ядер, скорость электронов при нагреве должна увеличиваться больше, чем у ядер. И если бы заряд зависел от скорости, то при нагреве вещество становилось бы заряженным. Ничего подобного никогда не наблюдалось.

Далее, если бы заряд электрона зависел от скорости, то в ходе химических реакций суммарный заряд вещества изменялся бы, поскольку средние скорости электронов в веществе зависят от его химического состава. Расчет показывает, что даже небольшая зависимость заряда от скорости приводила бы даже в простейших химических реакциях к огромным электрическим полям. Но и здесь ничего похожего не наблюдалось.

И наконец, расчет и работа всех современных ускорителей заряженных частиц основаны на предположении, что заряд ча-

стиц не меняется при изменении их скорости. Итак, мы приходим к выводу, что заряд любой частицы - релятивистски инвариантная величина, не зависящая от скорости частицы, от выбора системы отсчета.

Инвариантность теоремы Гаусса для поля Е. Оказывается это следует как обобщение экспериментальных фактов, - что теорема Гаусса \(\oint \mathbf{E d S}=q / \varepsilon_{0}\) справедлива не только для покоящихся зарядов, но и для движущихся. При этом поверхностный интеграл должен быть вычислен для одного и того же момента времени в данной системе отсчета.

Кроме того, поскольку различные инерциальные системы отсчета физически эквивалентны друг другу (согласно принципу относительности), мы можем утверждать, что теорема Гаусса справедлива во всех инерциальных системах отсчета.

\section*{§ 8.2. Законы преобразования полей Е и В}
При переходе от одной системы отсчета к другой поля \(\mathbf{E}\) и \(\mathbf{B}\) определенным образом преобразуются. Законы этого преобразования устанавливаются в специальной теории относительности, причем довольно сложным образом. По этой причине мы не будем воспроизводить здесь соответствующие выводы, а сосредоточим внимание на содержании этих законов, на вытекающих из них следствиях, а также на том, как следует пользоваться этими законами при решении некоторых конкретных вопросов.

Постановка вопроса. Пусть имеются две инерциальные системы отсчета: \(K\)-система и движущаяся относительно нее со скоростью \(\mathbf{v}_{0}\) система \(K^{\prime}\). В некоторой пространственно-временной точке \(K\)-системы отсчета известны значения полей \(\mathbf{E}\) и \(\mathbf{B}\). Какими будут значения полей \(\mathbf{E}^{\prime}\) и \(\mathbf{B}^{\prime}\) в той же самой простран-ственно-временной точке в \(K^{\prime}\)-системе отсчета? Напомним, что одной и той же пространственно-временной точкой называют такую, координаты и время которой в обеих системах отсчета связаны между собой преобразованиями Лоренца*.

\begin{itemize}
  \item T. e. \(x^{\prime}=\frac{x-v_{0} t}{\sqrt{1-\left(v_{0} / c\right)^{2}}}, y^{\prime}=y, z^{\prime}=z, \quad t^{\prime}=\frac{t-x v_{0} / c^{2}}{\sqrt{1-\left(v_{0} / c\right)^{2}}}\).
\end{itemize}

Ответ на этот вопрос, как уже было сказано, дает теория относительности, которая показывает, что законы преобразования полей выражаются следующими формулами:

\[
\begin{array}{ll}
\mathbf{E}_{\|}^{\prime}=\mathbf{E}_{\|}, & \mathbf{B}_{\|}^{\prime}=\mathbf{B}_{\|}, \\
\mathbf{E}_{\perp}^{\prime}=\frac{\mathbf{E}_{\perp}+\left[\mathbf{v}_{0} \mathbf{B}\right]}{\sqrt{1-\beta^{2}}}, & \mathbf{B}_{\perp}^{\prime}=\frac{\mathbf{B}_{\perp}-\left[\mathbf{v}_{0} \mathbf{E}\right]}{\sqrt{1-\beta^{2}}} . \tag{8.1}
\end{array}
\]

Здесь символами \(\|\) и \(\perp\) отмечены продольные и поперечные (по отношению к вектору \(\mathbf{v}_{0}\) ) составляющие электрического и магнитного полей, \(\beta=v_{0} / c, c-\) скорость света в вакууме ( \(c^{2}=1 / \varepsilon_{0} \mu_{0}\) ).

Эти же формулы, записанные в проекциях, имеют вид:

\[
\begin{array}{ll}
E_{x}^{\prime}=E_{x}, & B_{x}^{\prime}=B_{x}, \\
E_{y}^{\prime}=\frac{E_{y}-v_{0} B_{z}}{\sqrt{1-\beta^{2}}}, & B_{y}^{\prime}=\frac{B_{y}+v_{0} E_{z} / c^{2}}{\sqrt{1-\beta^{2}}},  \tag{8.2}\\
E_{z}^{\prime}=\frac{E_{z}+v_{0} B_{y}}{\sqrt{1-\beta^{2}}}, & B_{z}^{\prime}=\frac{B_{z}-v_{0} E_{y} / c^{2}}{\sqrt{1-\beta^{2}}},
\end{array}
\]

где предполагается, что оси координат \(X\) и \(X^{\prime}\) направлены вдоль вектора \(\mathbf{v}_{0}\), ось \(Y^{\prime}\) параллельна оси \(Y\), ось \(Z^{\prime}\) - оси \(Z\).

Из уравнений (8.1) и (8.2) видно, что каждый из векторов \(\mathbf{E}^{\prime}\) и \(\mathbf{B}^{\prime}\) выражается как через Е, так и через В. Это свидетельствует о единой природе электрического и магнитного полей. Каждое из них в отдельности не имеет абсолютного смысла: об электрическом и магнитном полях можно говорить лишь с обязательным указанием системы отсчета, в которой эти поля рассматриваются.

Подчеркнем, что свойства электромагнитного поля, выраженные в законах его преобразования, являются локальньмии: значения \(\mathbf{E}^{\prime}\) и \(\mathbf{B}^{\prime}\) в некоторой пространственно-временной точке \(K^{\prime}\)-системы отсчета однозначно определяются только через значения \(\mathbf{E}\) и В в той же пространственно-временной точке \(K\)-системы отсчета.

Необходимо обратить внимание еще на следующие особенности законов преобразования полей:

\begin{enumerate}
  \item В отличие от поперечных составляющих \(\mathbf{E}\) и \(\mathbf{B}\), которые изменяются при переходе к другой системе отсчета, продольные составляющие не изменяются - во всех системах отсчета они оказываются одинаковыми.
  \item Векторы Е и В связаны друг с другом в разных системах отсчета в высшей степени симметричным образом. Это особенно полно обнаруживается в форме записи законов преобразования через проекции полей [см. (8.2)].
  \item Если надо получить формулы обратного преобразования (от \(K^{\prime}\) к \(K\) ), то достаточно в формулах (8.1) и (8.2) заменить все штрихованные величины на нештрихованные (и наоборот), а также - знак перед \(v_{0}\).
\end{enumerate}

Частный случай преобразования полей ( \(v_{0} \ll c\) ). Если \(K^{\prime}\)-система движется относительно \(K\)-системы со скоростью \(v_{0} \ll c\), то корень в знаменателе формул (8.1) можно заменить на единицу, и мы будем иметь

\[
\begin{array}{ll}
\mathbf{E}_{\|}^{\prime}=\mathbf{E}_{\|}, & \mathbf{B}_{\|}^{\prime}=\mathbf{B}_{\|}, \\
\mathbf{E}_{\perp}^{\prime}=\mathbf{E}_{\perp}+\left[\mathbf{v}_{0} \mathbf{B}\right], & \mathbf{B}_{\perp}^{\prime}=\mathbf{B}_{\perp}-\left[\mathbf{v}_{0} \mathbf{E}\right] / c^{2} . \tag{8.3}
\end{array}
\]

Отсюда следует, что


\begin{equation*}
\mathbf{E}^{\prime}=\mathbf{E}+\left[\mathbf{v}_{0} \mathbf{B}\right], \quad \mathbf{B}^{\prime}=\mathbf{B}-\left[\mathbf{v}_{0} \mathbf{E}\right] / c^{2} \tag{8.4}
\end{equation*}


Заметим, что первую из формул (8.4) можно получить непосредственно и очень просто. Пусть в \(K\)-системе в некоторый момент \(t\) заряд \(q\) имеет скорость \(\mathbf{v}_{0}\). Действующая на него сила Лоренца \(\mathbf{F}=q \mathbf{E}+q\left[\mathbf{v}_{0} \mathbf{B}\right]\). Перейдем в инерциальную \(K^{\prime}\)-систему, движущуюся относительно \(K\)-системы с той же скоростью, что и заряд \(q\) в момент \(t\), т. е. со скоростью \(\mathbf{v}_{0}\). В этот момент заряд \(q\) неподвижен в \(K^{\prime}\)-системе, и сила, действующая на покоящийся заряд, является чисто электрической: \(\mathbf{F}^{\prime}=q \mathbf{E}^{\prime}\). При \(v_{0} \ll c\), как в нашем случае, сила инвариантна ( \(\mathbf{F}^{\prime}=\mathbf{F}\) ), откуда и следует первая из формул (8.4).

Формулу же для преобразования магнитного поля можно получить только с помощью теории относительности в результате довольно громоздких выкладок.

Рассмотрим простой пример на применение формул (8.4).

Пример. Большая металлическая пластинка движется с постоянной нерелятивистской скоростью \(v\) в однородном магнитном поле \(B\) (рис. 8.1). Найдем поверхностную плотность зарядов, возникающих на плоскостях пластинки из-за ее движения.

\begin{figure}[h]
\begin{center}
  \includegraphics[alt={},max width=\textwidth]{3e870b25-6320-483d-9133-cdd5212c2707-210_172_259_145_891}
\captionsetup{labelformat=empty}
\caption{Рис. 8.1}
\end{center}
\end{figure}

Перейдем в систему отсчета, связанную с пластинкой. Cогласно первой из формул (8.4) в этой системе отсчета будет наблюдаться постоянное однородное электрическое поле

\[
\mathbf{E}^{\prime}=[\mathbf{v B}] .
\]

Оно будет направлено к нам. Под действием этого внешнего поля произойдет смещение зарядов так, что на обращенной к нам поверхности пластинки выступят положительные заряды, а на противоположной поверхности - отрицательные.

Поверхностная плотность σ этих зарядов будет такой, чтобы создаваемое ими поле внутри пластинки полностью компенсировало внешнее поле \(\mathbf{E}^{\prime}\), ибо при равновесии результирующее электрическое поле внутри пластинки должно быть равно нулю. Имея в виду соотношение (1.11), получим

\[
\sigma=\varepsilon_{0} E^{\prime}=\varepsilon_{0} v B .
\]

Заметим, что при решении этого вопроса можно было рассуждать и иначе - с точки зрения системы отсчета, где пластинка движется со скоростью \(\mathbf{v}\). В этой системе отсчета внутри пластинки будет электрическое поле. Оно возникает вследствие действия магнитной части силы Лоренца, вызывающей смещение всех электронов в пластинке за плоскость рис. 8.1. В результате передняя поверхность пластинки оказывается заряженной положительно, задняя - отрицательно, и внутри пластинки появляется электрическое поле, причем такое, что электрическая сила \(q \mathbf{E}\) компенсирует магнитную часть силы Лоренца \(q[\mathbf{v B}]\), откуда \(\mathbf{E}=-[\mathbf{v B}]\). Это поле связано с \(\sigma\) той же формулой \(\sigma=\varepsilon_{0} v B\).

Оба подхода к решению данного вопроса одинаково законны.\\
Релятивистская природа магнетизма. Из формул преобразования полей (8.1) и (8.2) вытекает весьма замечательный вывод: возникновение магнитного поля является чисто релятивистским эффектом, следствием наличия в природе предельной скорости \(c\), равной скорости свете в вакууме.

Если бы эта скорость была бесконечной (соответственно и скорость распространения взаимодействий), никакого магнетизма вообще не существовало бы. В самом деле, рассмотрим свободный электрический заряд. В системе отсчета \(K\), где он покоится, существует только электрическое поле. А это значит согласно (8.1), что в любой другой \(K^{\prime}\)-системе отсчета, если бы \(c \rightarrow \infty\), никакого магнитного поля \(B^{\prime}\) не возникало бы. Оно возникает только из-за конечности \(c\), т. е. в конечном счете вследствие релятивистского эффекта.

Релятивистская природа магнетизма является универсальным физическим фактом, и его происхождение обусловлено отсутствием магнитных зарядов.

В отличие от большинства релятивистских явлений магнетизм во многих случаях обнаруживается сравнительно легко, например магнитное поле проводника с током. Причина подобных благоприятных обстоятельств обусловлена тем, что магнитное поле может создаваться очень большим числом движущихся зарядов при условии почти полного исчезновения электрического поля из-за практически идеального баланса числа электронов и протонов в проводниках. В этих случаях магнитное взаимодействие оказывается преобладающим.

Почти полная компенсация электрических зарядов и позволила физикам изучить релятивистские эффекты (т. е. магнетизм) и открыть правильные законы. По этой причине после создания теории относительности законы электромагнетизма в отличие от законов Ньютона не пришлось уточнять.

Поле не движется, а изменяется. Поскольку электрическое и магнитное поля появляются в разных соотношениях при изменении системы отсчета, следует проявлять определенную осторожность в обращении с полями \(\mathbf{E}\) и В. Скажем, уже вопрос о силе, действующей на заряд со стороны движущегося магнитного поля, не имеет сколько-нибудь точного содержания. Сила определяется значениями величин \(\mathbf{E}\) и \(\mathbf{B}\) в точке нахождения заряда. Если в результате движения источников полей Е и В их значения в этой точке будут меняться, изменится и сила, в противном случае движение источников на значении силы не отразится.

Таким образом, при решении вопроса о силе, действующей на заряд, необходимо знать \(\mathbf{E}\) и \(\mathbf{B}\) в точке нахождения заряда и его скорость \(\mathbf{v}\), причем все эти величины должны быть взяты относительно интересующей нас инерциальной системы отсчета.

Если же когда и говорят о «движущемся» поле, то это нужно понимать просто как краткий и удобный способ словесного описания изменяющегося поля в определенных условиях и ничего более.

Насколько надо проявлять осторожность в обращении с полем при переходе из одной системы отсчета к другой, станет ясно хотя бы уже из такого простого примера.

Пример. Заряженная частица покоится между полюсами магнита, неподвижного в \(K\)-системе отсчета. Перейдем в \(K^{\prime}\)-систему, которая движется вправо (рис. 8.2) с нерелятивистской скоростью \(\mathbf{v}_{0}\) - относительно \(K\)-системы. 1. Можем ли мы утверждать, что в \(K^{\prime}\)-системе заряженная частица движет-

\begin{figure}[h]
\begin{center}
  \includegraphics[alt={},max width=\textwidth]{3e870b25-6320-483d-9133-cdd5212c2707-212_229_465_750_695}
\captionsetup{labelformat=empty}
\caption{Рис. 8.2}
\end{center}
\end{figure}

ся в магнитном поле? 2. Найдем силу, действующую на эту частицу в \(K^{\prime}\)-системе.

\begin{enumerate}
  \item Да, частица движется в магнитном поле. Но, заметим, в магнитном поле, а не относительно магнитного поля. Имеет смысл говорить о движении частицы относительно системы отсчета, магнита и других тел, но только не относительно магнитного поля. Последнее просто не имеет физического смысла. Все это касается не только магнитного, но и электрического поля.
  \item Чтобы найти силу, надо учесть, что в \(K^{\prime}\)-системе появится и электрическое поле \(\mathbf{E}^{\prime}=\left[\mathbf{v}_{0} \mathbf{B}\right]\), оно направлено на нас (рис. 8.2). В \(K^{\prime}\)-системе заряд будет двигаться влево со скоростью \(-\mathbf{v}_{0}\), причем это движение будет происходить в скрещенных электрическом и магнитном полях. Пусть для определенности заряд частицы \(q>0\), тогда сила Лоренца в \(K^{\prime}\)-системе
\end{enumerate}

\[
\mathbf{F}^{\prime}=q \mathbf{E}^{\prime}+q\left[-\mathbf{v}_{0} \mathbf{B}\right]=q\left(\left[\mathbf{v}_{0} \mathbf{B}\right]-\left[\mathbf{v}_{0} \mathbf{B}\right]\right)=0,
\]

что, впрочем, можно было и сразу сказать исходя из факта инвариантности силы при нерелятивистских преобразованиях из одной системы отсчета в другую.

\section*{§ 8.3. Следствия из законов преобразования полей}
Некоторые простые следствия. Из формул преобразования (8.1) вытекают в ряде случаев простые и вместе с тем полезные соотношения.

\begin{enumerate}
  \item Если в \(K\)-системе имеется только электрическое поле \(\mathbf{E}\) (а магнитное \(\mathbf{B}=0\) ), то между полями \(\mathbf{E}^{\prime}\) и \(\mathbf{B}^{\prime}\) в \(K^{\prime}\)-системе существует такая связь
\end{enumerate}


\begin{equation*}
\mathbf{B}^{\prime}=-\left[\mathbf{v}_{0} \mathbf{E}^{\prime}\right] / c^{2} . \tag{8.5}
\end{equation*}


Действительно, если \(\mathbf{B}=0\), то \(\mathbf{E}_{\perp}^{\prime}=\mathbf{E}_{\perp} / \sqrt{1-\beta^{2}}\) и \(\mathbf{B}_{\|}^{\prime}=0, \mathbf{B}_{\perp}^{\prime}= =-\left[\mathbf{v}_{0} \mathbf{E}\right] / c^{2} \sqrt{1-\beta^{2}}=-\left[\mathbf{v}_{0} \mathbf{E}^{\prime}\right] / c^{2}\), где учтено, что в векторном произведении можно писать как \(\mathbf{E}\), так и \(\mathbf{E}_{\perp}\) (это же относится и к штрихованным величинам). Приняв во внимание, что \(\mathbf{B}^{\prime}=\mathbf{B}_{\|}^{\prime}+\mathbf{B}_{\perp}^{\prime}=\mathbf{B}_{\perp}^{\prime}\), приходим к формуле (8.5).\\
2. Если в \(K\)-системе имеется только магнитное поле В (а электрическое \(\mathbf{E}=0\) ), то в \(K^{\prime}\)-системе


\begin{equation*}
\mathbf{E}^{\prime}=\left[\mathbf{v}_{0} \mathbf{B}^{\prime}\right] . \tag{8.6}
\end{equation*}


В самом деле, если \(\mathbf{E}=0\), то \(\mathbf{B}_{\perp}^{\prime}=\mathbf{B}_{\perp} / \sqrt{1-\beta^{2}}\) и \(\mathbf{E}_{\|}^{\prime}=0, \mathbf{E}_{\perp}^{\prime}= =\left[\mathbf{v}_{0} \mathbf{B}\right] / \sqrt{1-\beta^{2}}\). Заменив в последнем векторном произведении \(\mathbf{B}\) на \(\mathbf{B}_{\perp}\) и затем \(\mathbf{B}_{\perp}^{\prime}\) на \(\mathbf{B}^{\prime}\), приходим к формуле (8.6).

Из формул (8.5) и (8.6) вытекает следующий важный вывод: если в К-системе имеется лишь одно из полей (Е или В), то в \(K^{\prime}\)-системе электрическое и магнитное поля взаимно перпендикулярньс ( \(\mathbf{E}^{\prime} \perp \mathbf{B}^{\prime}\) ). Заметим, что обратное утверждение справедливо не всегда, а лишь при определенных дополнительных ограничениях, накладываемых на модули векторов \(\mathbf{E}\) и \(\mathbf{B}\).

И последнее замечание. Ввиду того что в уравнения (8.5) и (8.6) входят только величины, относящиеся к одной и той же системе отсчета, эти уравнения легко применять к полям, изменяющимся в пространстве и времени. Хорошим примером может служить поле равномерно движущегося точечного заряда.

Поле свободно движущегося релятивистского заряда. Формулы преобразования полей представляют большой интерес прежде всего в том отношении, что выражают собой удивитель-

ные свойства электромагнитного поля. Но, кроме того, они важны и в чисто практическом отношении, позволяя иногда проще решать некоторые вопросы.

Например, задача о нахождении поля равномерно движущегося точечного заряда может быть решена путем преобразования чисто кулоновского поля, которое наблюдается в системе отсчета, связанной с самим зарядом. Расчет показывает (см. задачу 8.10), что линии \(\mathbf{E}\) поля свободно движущегося точечного заряда \(q\) имеют вид, показанный на рис. 8.3, где \(\mathbf{v}\) - скорость заряда. Изображенная здесь картина соответст-

\begin{figure}[h]
\begin{center}
  \includegraphics[alt={},max width=\textwidth]{3e870b25-6320-483d-9133-cdd5212c2707-214_342_241_274_917}
\captionsetup{labelformat=empty}
\caption{Рис. 8.3}
\end{center}
\end{figure}

вует мгновенной «фотографии» конфигурации электрического поля. Вектор \(\mathbf{E}\) в произвольной точке \(P\) системы отсчета направлен вдоль радиуса-вектора \(\mathbf{r}\), проведенного из точки, где находится заряд в данный момент, в точку \(P\).

Модуль вектора \(\mathbf{E}\) определяется формулой


\begin{equation*}
E=\frac{1}{4 \pi \varepsilon_{0}} \frac{q}{r^{2}} \frac{1-\beta^{2}}{\left(1-\beta^{2} \sin ^{2} \vartheta\right)^{3 / 2}}, \tag{8.7}
\end{equation*}


где \(\beta=v / c\); Э - угол между радиусом-вектором \(\mathbf{r}\) и вектором \(\mathbf{v}\) - скоростью заряда.

Электрическое поле «сплющивается» в направлении движения заряда (см. рис. 8.3), причем в тем большей степени, чем ближе скорость заряда \(v\) к скорости \(c\). Следует также иметь в виду, что поле, показанное на этом рисунке, «перемещается» вместе с зарядом, вследствие чего поле \(\mathbf{E}\) в системе отсчета, относительно которой заряд движется, изменяется со временем.

Зная поле E, можно найти и поле B в этой же системе отсчета:


\begin{equation*}
\mathbf{B}=\frac{1}{c^{2}}[\mathbf{v E}]=\frac{\mu_{0}}{4 \pi} \frac{q[\mathbf{v r}]}{r^{3}} \frac{1-\beta^{2}}{\left(1-\beta^{2} \sin ^{2} \vartheta\right)^{3 / 2}} . \tag{8.8}
\end{equation*}


Эта формула является следствием соотношения (8.5), в котором произведена замена штрихованных величин на нештрихованные и одновременно \(\mathbf{v}\) на \(-\mathbf{v}\).

При \(v \ll c\) ( \(\beta \ll 1\) ) выражения (8.7) и (8.8) переходят соответственно в (1.2) и (6.3).

\section*{§ 8.4. Инварианты электромагнитного поля}
Поскольку векторы E и B, характеризующие электромагнитное поле, зависят от системы отсчета (в той же самой про-странственно-временной точке), возникает естественный вопрос об инвариантах, т. е. не зависящих от системы отсчета количественных характеристиках электромагнитного поля.

Можно показать, что существуют два таких инварианта, представляющие собой комбинации векторов \(\mathbf{E}\) и \(\mathbf{B}\), это


\begin{equation*}
\mathbf{E B}=\operatorname{inv}, \tag{8.9}
\end{equation*}


Инвариантность этих величин (относительно преобразований Лоренца) является следствием формул преобразования полей (8.1) или (8.2). Более подробно этот вопрос рассмотрен в задаче 8.9.

Использование данных инвариантов позволяет в ряде случаев быстро и просто находить решение и делать соответствующие выводы и предсказания. Приведем наиболее важные из них.

\begin{enumerate}
  \item Из инвариантности ЕВ сразу следует, что в случае, когда в какой-либо системе отсчета \(\mathbf{E} \perp \mathbf{B}\), т. е. \(\mathbf{E B}=0\), то и во всех других инерциальных системах отсчета \(\mathbf{E}^{\prime} \perp \mathbf{B}^{\prime}\).
  \item Из инвариантности \(E^{2}-c^{2} B^{2}\) следует, что в случае, когда \(E=c B\) (т. е. \(E^{2}-c^{2} B^{2}=0\) ), то и в любой другой инерциальной системе отсчета \(E^{\prime}=c B^{\prime}\).
  \item Если в какой-либо системе отсчета угол между векторами \(\mathbf{E}\) и \(\mathbf{B}\) острый (или тупой), - это значит, что EB больше (либо меньше) нуля, - то угол между векторами \(\mathbf{E}^{\prime}\) и \(\mathbf{B}^{\prime}\) также будет острым (или тупым) во всякой другой системе отсчета.
  \item Если в какой-либо системе отсчета \(E>c B\) (или \(E<c B\) ) это значит, что \(E^{2}-c^{2} B^{2}\) больше (либо меньше) нуля, - то в любой другой системе отсчета будет также \(E^{\prime}>c B^{\prime}\) (или \(E^{\prime}<c B^{\prime}\) ).
  \item Если оба инварианта равны нулю, то во всех системах отсчета \(\mathbf{E} \perp \mathbf{B}\) и \(E=c B\). Именно это и наблюдается, как мы увидим, в электромагнитной волне.
  \item Если равен нулю только инвариант EB, то можно найти такую систему отсчета, в которой или \(\mathbf{E}^{\prime}=0\), или \(\mathbf{B}^{\prime}=0\); какое именно, определяется знаком другого инварианта. Справедли-
\end{enumerate}

во и обратное утверждение: если в какой-либо системе отсчета \(\mathbf{E}=0\) или \(\mathbf{B}=0\), то во всякой другой системе отсчета \(\mathbf{E}^{\prime} \perp \mathbf{B}^{\prime}\). (Этот вывод был уже в § 8.3.)

И последнее. Нужно помнить, что поля \(\mathbf{E}\) и \(\mathbf{B}\), вообще говоря, зависят и от координат, и от времени. Поэтому каждый из инвариантов (8.9) относится к одной и той же пространствен-но-временной точке поля, координаты и время которой в разных системах отсчета связаны преобразованиями Лоренца.

\section*{Задачи}
8.1. Частный случай преобразования полей. Нерелятивистский точечный заряд \(q\) движется с постоянной скоростью \(\mathbf{v}\). Найти с помощью формул преобразования полей магнитное поле В этого заряда в точке, положение которой относительно заряда определяется ра-диусом-вектором r.

Решение. Перейдем в \(K^{\prime}\)-систему отсчета, связанную с зарядом. В этой системе имеется только кулоновское поле напряженностью

\[
\mathbf{E}^{\prime}=\frac{1}{4 \pi \varepsilon_{0}} \frac{q}{r^{3}} \mathbf{r},
\]

где учтено, что в \(K^{\prime}\)-системе радиус-вектор \(\mathbf{r}^{\prime}=\mathbf{r}\) (нерелятивистский случай). Теперь перейдем обратно, из \(K^{\prime}\)-системы в \(K\)-систему, которая движется относительно \(K^{\prime}\)-системы со скоростью -v. Для этого воспользуемся формулой для поля В из (8.4), в которой роль штрихованных величин будут играть нештрихованные (и на-

\begin{figure}[h]
\begin{center}
  \includegraphics[alt={},max width=\textwidth]{3e870b25-6320-483d-9133-cdd5212c2707-216_226_587_1236_321}
\captionsetup{labelformat=empty}
\caption{Рис. 8.4}
\end{center}
\end{figure}

оборот), а скорость \(\mathbf{v}_{0}\) надо заменить на \(-\mathbf{v}_{0}\) (рис. 8.4). В нашем случае \(\mathbf{v}_{0}=\mathbf{v}\), поэтому \(\mathbf{B}=\mathbf{B}^{\prime}+\left[\mathbf{v E} \mathbf{E}^{\prime}\right] / c^{2}\). Учитывая, что в \(K^{\prime}\)-системе \(\mathbf{B}^{\prime}=0\) и что \(c^{2}=1 / \varepsilon_{0} \mu_{0}\), находим

\[
\mathbf{B}=\frac{\mu_{0}}{4 \pi} \frac{q[\mathbf{v r}]}{r^{3}} .
\]

Мы получили формулу (6.3), которая ранее была постулирована как результат обобщения опытных фактов.\\
8.2. Большая пластинка из однородного диэлектрика с проницаемо-

\begin{figure}[h]
\begin{center}
  \includegraphics[alt={},max width=\textwidth]{3e870b25-6320-483d-9133-cdd5212c2707-217_206_285_284_135}
\captionsetup{labelformat=empty}
\caption{Рис. 8.5}
\end{center}
\end{figure}

стью \(\varepsilon\) движется с постоянной нерелятивистской скоростью \(\mathbf{v}\) в однородном магнитном поле В, как показано на рис. 8.5. Найти поляризованность \(\mathbf{P}\) диэлектрика и поверхностную плотность \(\sigma^{\prime}\) связанных зарядов.

Решение. В системе отсчета, связанной с пластинкой, будет наблюдаться кроме магнитного поля и электрическое, обозначим его \(\mathbf{E}_{0}^{\prime}\), согласно формулам преобразования полей (8.4)

\[
\mathbf{E}_{0}^{\prime}=[\mathbf{v B}] .
\]

Поляризованность диэлектрика

\[
\mathbf{P}=\varkappa \varepsilon_{0} \mathbf{E}^{\prime}=\varepsilon_{0} \frac{\varepsilon-1}{\varepsilon}[\mathbf{v B}],
\]

где учтено, что внутри диэлектрика согласно (3.29) \(\mathbf{E}^{\prime}=\mathbf{E}_{0}^{\prime} / \varepsilon\). Поверхностная плотность связанных зарядов

\[
\left|\sigma^{\prime}\right|=P=\varepsilon_{0} \frac{\varepsilon-1}{\varepsilon} \nu B,
\]

причем на той поверхности пластинки, которая обращена к нам (см. рис. 8.5), \(\sigma^{\prime}>0\), на противоположной \(\sigma^{\prime}<0\).\\
8.3. Имеется незаряженный длинный прямой провод с током \(I\). Найти заряд на единицу длины этого провода в системе отсчета, движущейся поступательно с нерелятивистской скоростью \(v_{0}\) вдоль проводника в направлении тока \(I\).

Решение. В движущейся системе отсчета согласно формулам преобразования (8.4) появится электрическое поле \(\mathbf{E}^{c}=\left[\mathbf{v}_{0} \mathbf{B}\right]\), или


\begin{equation*}
E_{r}^{\prime}=-v_{0} \mu_{0} I / 2 \pi r . \tag{1}
\end{equation*}


Здесь выражение для \(B\) получено с помощью теоремы о циркуляции.\\
С другой стороны, по теореме Гаусса (в движущейся системе отсчета)


\begin{equation*}
E_{r}^{\prime}=\lambda^{\prime} / 2 \pi \varepsilon_{0} r, \tag{2}
\end{equation*}


где \(\lambda^{\prime}\) - заряд на единицу длины провода. Из сравнения (1) и (2) находим

\[
\lambda^{\prime}=-v_{0} I / c^{2},
\]

где \(c^{2}=1 / \varepsilon_{0} \mu_{0}\). Происхождение этого заряда связано с различным лоренцевым сокращением, которое испытывают «цепочки» положительных и отрицательных зарядов (ведь их скорости разные!).\\
8.4. В \(K\)-системе отсчета имеется узкий пучок протонов, движущихся с релятивистской скоростью \(v\). На некотором расстоянии от пучка напряженность электрического поля равна \(E\). Найти индукцию \(B^{\prime}\) магнитного поля на том же расстоянии от пучка в \(K^{\prime}\)-системе отсчета, перемещающейся со скоростью \(v_{0}\) относительно \(K\)-системы в направлении движения протонов.

Решение. Этот вопрос проще всего решить с помощью формул (8.1). Но предварительно надо найти индукцию \(B\) в \(K\)-системе на том же расстоянии от пучка, где задана напряженность \(E\).

Воспользовавшись теоремой о циркуляции вектора В и теоремой Гаусса для вектора E, найдем:

\[
B=\mu_{0} I / 2 \pi r, \quad E=\lambda / 2 \pi \varepsilon_{0} r,
\]

где \(r\) - расстояние от пучка, \(I=\lambda v\) - сила тока, \(\lambda\) - заряд на единицу длины пучка. Из этих формул следует, что

\[
B / E=\varepsilon_{0} \mu_{0} I / \lambda=v / c^{2},
\]

здесь \(c^{2}=1 / \varepsilon_{0} \mu_{0}\). Подставив выражение для \(B\) из этого уравнения в последнюю из формул преобразования (8.1), получим:

\[
B^{\prime}=\frac{E\left|v-v_{0}\right|}{c^{2} \sqrt{1-\left(v_{0} / c\right)^{2}}} .
\]

При этом, если \(v_{0}<v\), то линии вектора \(\mathbf{B}^{\prime}\) имеют правовинтовое направление с вектором \(\mathbf{v}_{0}\), если же \(v_{0}>v\), то - левовинтовое (ибо ток \(I^{\prime}\) в \(K^{\prime}\)-системе в этом случае будет течь в обратную сторону).\\
8.5. Движение заряда в скрещенных Е и В полях. Релятивистская заряженная частица движется в пространстве, где имеются однородные взаимно перпендикулярные электрическое и магнитное поля \(\mathbf{E}\) и \(\mathbf{B}\). Частица движется прямолинейно по направлению, перпендикулярному векторам \(\mathbf{E}\) и \(\mathbf{B}\). Найти \(\mathbf{E}^{\prime}\) и \(\mathbf{B}^{\prime}\) в системе отсчета, перемещающейся поступательно вместе с частицей.

Решение. Из характера движения частицы следует, что ее скорость должна удовлетворять условию


\begin{equation*}
v B=E . \tag{1}
\end{equation*}


Согласно формулам преобразования (8.1)

\[
\mathbf{E}^{\prime}=\frac{\mathbf{E}+[\mathbf{v B}]}{\sqrt{1-\beta^{2}}}=0
\]

ибо в нашем случае сила Лоренца, а значит, и величина \(\mathbf{E}+[\mathbf{v B}]\) равны нулю.

Для магнитного поля согласно тем же формулам преобразования

\[
\mathbf{B}^{\prime}=\frac{\mathbf{B}-[\mathbf{v E}] / c^{2}}{\sqrt{1-\beta^{2}}}
\]

\begin{figure}[h]
\begin{center}
  \includegraphics[alt={},max width=\textwidth]{3e870b25-6320-483d-9133-cdd5212c2707-219_252_284_556_130}
\captionsetup{labelformat=empty}
\caption{Рис. 8.6}
\end{center}
\end{figure}

Расположение векторов показано на рис. 8.6, откуда видно, что [ \(\mathbf{v E}\) ] ↑ \(\uparrow\) В. Поэтому с учетом того, что согласно (1) \(v=E / B\), можно записать

\[
B^{\prime}=\frac{B-E^{2} / B c^{2}}{\sqrt{1-\beta^{2}}}=\frac{B\left(1-\beta^{2}\right)}{\sqrt{1-\beta^{2}}}
\]

или в векторном виде

\[
\mathbf{B}^{\prime}=\mathbf{B} \sqrt{1-(E / c B)^{2}}
\]

Полезно убедиться, что полученные выражения удовлетворяют обоим инвариантам поля.\\
8.6. Движение заряда в скрещенных Е и В полях. Нерелятивистская частица с удельным зарядом \(q / m\) движется в области, где созданы однородные взаимно перпендикулярные \(\mathbf{E}\) и \(\mathbf{B}\)

\begin{figure}[h]
\begin{center}
  \includegraphics[alt={},max width=\textwidth]{3e870b25-6320-483d-9133-cdd5212c2707-219_247_228_1161_163}
\captionsetup{labelformat=empty}
\caption{Рис. 8.7}
\end{center}
\end{figure}

поля (рис. 8.7). В момент \(t=0\) частица находилась в точке \(O\) и имела нулевую скорость. Найти закон движения частицы, \(x(t)\) и \(y(t)\).

Решение. Движение частицы происходит под действием силы Лоренца, причем, как нетрудно сообразить, все время в плоскости XY. Проще всего ее движение будет выглядеть в такой \(K^{\prime}\)-системе отсчета, где будет наблюдаться только магнитное поле. Найдем эту систему отсчета.

Из преобразований (8.4) следует, что \(E^{\prime}=0\) в такой системе отсчета, которая движется со скоростью \(\mathbf{v}_{0}\), удовлетворяющей соотношению \(\mathbf{E}=-\left[\mathbf{v}_{0} \mathbf{B}\right]\). Лучше всего взять ту \(K^{\prime}\)-систему, скорость \(\mathbf{v}_{0}\) которой направлена в положительную сторону оси \(X\) (рис. 8.7), ибо в такой системе отсчета частица будет двигаться перпендикулярно вектору \(\mathbf{B}^{\prime}\) и ее движение будет наиболее простым.

Итак, в \(K^{\prime}\)-системе отсчета, которая движется вправо со скоростью \(v_{0}=E / B\), поле \(\mathbf{E}^{\prime}=0\) и будет наблюдаться только поле В. Согласно (8.4) и рис. 8.7

\[
B^{\prime}=\mathbf{B}-\left[\mathbf{v}_{0} \mathbf{E}\right] / c^{2}=\mathbf{B}\left(1-v_{0}^{2} / c^{2}\right) .
\]

Для нерелятивистской частицы \(v_{0} \ll c\), и можно считать, что \(\mathbf{B}^{\prime}=\mathbf{B}\).

В данной \(K^{\prime}\)-системе отсчета частица будет двигаться только в магнитном поле, причем перпендикулярно его направлению. Уравнение движения частицы в этой системе отсчета будет иметь вид


\begin{equation*}
m v_{0}^{2} / R=q v_{0} B . \tag{1}
\end{equation*}


Это уравнение записано для момента \(t=0\), когда в \(K^{\prime}\)-системе частица двигалась, как показано на рис. 8.8. Так как сила Лоренца F направлена всегда перпендикулярно скорости частицы, то \(v_{0}=\) const и из (1) следует, что частица в \(K^{\prime}\)-системе будет двигаться по окружности радиусом

\[
R=m v_{0} / q B .
\]

\begin{figure}[h]
\begin{center}
  \includegraphics[alt={},max width=\textwidth]{3e870b25-6320-483d-9133-cdd5212c2707-220_236_236_732_901}
\captionsetup{labelformat=empty}
\caption{Рис. 8.8}
\end{center}
\end{figure}

Таким образом, частица движется равномерно со скоростью \(v_{0}\) по окружности в \(K^{\prime}\)-системе, которая, в свою очередь, перемещается равномерно вправо также со скоростью \(v_{0}=E / B\). Так ведет себя точка \(q\) на ободе колеса (рис. 8.9), катящегося с угловой скоростью \(\omega=v_{0} / R=q B / m\).\\
Из рис. 8.9 сразу видно, что координаты частицы \(q\) в момент \(t\) есть

\[
\begin{aligned}
& x=v_{0} t-R \sin \omega t=a(\omega t-\sin \omega t), \\
& y=R-R \cos \omega t=a(1-\cos \omega t),
\end{aligned}
\]

где \(\quad a=m E / q B^{2}, \quad \omega=q B / m\).

\begin{figure}[h]
\begin{center}
  \includegraphics[alt={},max width=\textwidth]{3e870b25-6320-483d-9133-cdd5212c2707-220_295_384_1198_778}
\captionsetup{labelformat=empty}
\caption{Рис. 8.9}
\end{center}
\end{figure}

8.7. В инерциальной \(K\)-системе отсчета имеется только однородное электрическое поле \(\mathbf{E}\). Найти модули и направления векторов \(\mathbf{E}^{\prime}\) и \(\mathbf{B}^{\prime}\) в \(K^{\prime}\)-системе отсчета, движущейся по отношению к \(K\)-системе с постоянной релятивистской скоростью \(\mathbf{v}_{0}\) под углом \(\alpha\) к вектоpy E.

Решение. Согласно формулам преобразования (8.1) с учетом того, что в \(K\)-системе \(\mathbf{B}=0\), получим

\[
E_{\|}^{\prime}=E \cos \alpha, \quad E_{\perp}^{\prime}=E \sin \alpha / \sqrt{1-\beta^{2}}, \quad \beta=v_{0} / c .
\]

Отсюда найдем модуль вектора \(\mathbf{E}^{\prime}\) :

\[
E^{\prime}=\sqrt{E_{\|}^{\prime 2}+E_{\perp}^{\prime 2}}=E \sqrt{\left(1-\beta^{2} \cos ^{2} \alpha\right) /\left(1-\beta^{2}\right)},
\]

а угол \(\alpha^{\prime}\) между векторами \(\mathbf{E}^{\prime}\) и \(\mathbf{v}_{0}\) - по формуле

\[
\operatorname{tg} \alpha^{\prime}=E_{\perp}^{\prime} / E_{\|}^{\prime}=\operatorname{tg} \alpha / \sqrt{1-\beta^{2}}
\]

Аналогичным образом найдем модуль и направление вектора \(\mathbf{B}^{\prime}\) :

\[
\mathbf{B}_{\|}^{\prime}=0, \quad \mathbf{B}_{\perp}^{\prime}=-\frac{\left[\mathbf{v}_{0} \mathbf{E}\right]}{c^{2} \sqrt{1-\beta^{2}}}, \quad \mathbf{B}^{\prime}=\mathbf{B}_{\perp}^{\prime} .
\]

Это значит, что вектор \(\mathbf{B}^{\prime} \perp \mathbf{v}_{0}\) и его модуль

\[
B^{\prime}=\frac{v_{0} E \sin \alpha}{c^{2} \sqrt{1-\beta^{2}}} .
\]

8.8. В \(K\)-системе отсчета имеются однородные электрическое \(\mathbf{E}\) и маг-

\begin{figure}[h]
\begin{center}
  \includegraphics[alt={},max width=\textwidth]{3e870b25-6320-483d-9133-cdd5212c2707-221_341_453_1022_135}
\captionsetup{labelformat=empty}
\caption{Рис. 8.10}
\end{center}
\end{figure}

нитное В поля одного направления. Найти модули векторов \(\mathbf{E}^{\prime}\) и \(\mathbf{B}^{\prime}\) и угол между ними в \(K^{\prime}\)-системе отсчета, движущейся с постоянной релятивистской скоростью \(\mathbf{v}_{0}\) в направлении, перпендикулярном векторам Е и В.

Решение. Согласно формулам (8.1) в \(K^{\prime}\)-системе отсчета оба вектора \(\mathbf{E}^{\prime}\) и \(\mathbf{B}^{\prime}\) будут также расположены перпендикулярно вектору \(\mathbf{v}_{0}\) (рис. 8.10). Модули векторов \(\mathbf{E}^{\prime}\) и \(\mathbf{B}^{\prime}\) находим по формулам:

\[
E^{\prime}=\sqrt{\frac{E^{2}+\left(v_{0} B\right)^{2}}{1-\left(v_{0} / c\right)^{2}}}, \quad B^{\prime}=\sqrt{\frac{B^{2}+\left(v_{0} E / c^{2}\right)^{2}}{1-\left(v_{0} / c\right)^{2}}} .
\]

Угол между векторами \(\mathbf{E}^{\prime}\) и \(\mathbf{B}^{\prime}\) определим через тангенс по формуле

\[
\operatorname{tg} \alpha^{\prime}=\operatorname{tg}\left(\alpha_{E}^{\prime}+\alpha_{B}^{\prime}\right)=\left(\operatorname{tg} \alpha_{E}^{\prime}+\operatorname{tg} \alpha_{B}^{\prime}\right) /\left(1-\operatorname{tg} \alpha_{E}^{\prime} \operatorname{tg} \alpha_{B}^{\prime}\right) .
\]

Поскольку \(\operatorname{tg} \alpha_{E}^{\prime}=v_{0} B / E\) и \(\operatorname{tg} \alpha_{B}^{\prime}=v_{0} E / c^{2} B\) (рис. 8.10), то

\[
\operatorname{tg} \alpha^{\prime}=\frac{v_{0}\left(B^{2}+E^{2} / c^{2}\right)}{\left(1-\beta^{2}\right) E B} .
\]

Отсюда видно, что при \(v_{0} \rightarrow c(\beta \rightarrow 1)\) угол \(\alpha^{\prime} \rightarrow \pi / 2\). Можно сделать и обратное заключение: если в одной системе отсчета известны \(\mathbf{E}\) и \(\mathbf{B}\), причем угол между этими векторами меньше \(90^{\circ}\), то существуют системы отсчета, где оба вектора \(\mathbf{E}^{\prime}\) и \(\mathbf{B}^{\prime}\) взаимно параллельны.\\
8.9. Инвариант ЕВ. Показать с помощью формул преобразования (8.1), что величина ЕВ является инвариантом.

Решение. В \(K^{\prime}\)-системе отсчета это произведение


\begin{equation*}
\mathbf{E}^{\prime} \mathbf{B}^{\prime}=\left(\mathbf{E}_{\|}^{\prime}+\mathbf{E}_{\perp}^{\prime}\right)\left(\mathbf{B}_{\|}^{\prime}+\mathbf{B}_{\perp}^{\prime}\right)=\mathbf{E}_{\|}^{\prime} \mathbf{B}_{\|}^{\prime}+\mathbf{E}_{\perp}^{\prime} \mathbf{B}_{\perp}^{\prime} . \tag{1}
\end{equation*}


Перепишем последнее слагаемое с помощью формул (8.1):


\begin{equation*}
\mathbf{E}_{\perp}^{\prime} \mathbf{B}_{\perp}^{\prime}=\frac{\left(\mathbf{E}_{\perp}+\left[\mathbf{v}_{0} \mathbf{B}_{\perp}\right]\right)\left(\mathbf{B}_{\perp}-\left[\mathbf{v}_{0} \mathbf{E}_{\perp}\right] / c^{2}\right)}{1-\beta^{2}} \tag{2}
\end{equation*}


Учитывая, что векторы \(\mathbf{E}_{\perp}\) и \(\mathbf{B}_{\perp}\) перпендикулярны вектору \(\mathbf{v}_{0}\), преобразуем числитель выражения (2) к виду


\begin{equation*}
\mathbf{E}_{\perp} \mathbf{B}_{\perp}-\left(v_{0} / c\right)^{2} \mathbf{E}_{\perp} \mathbf{B}_{\perp}=\mathbf{E}_{\perp} \mathbf{B}_{\perp}\left(1-\beta^{2}\right), \tag{3}
\end{equation*}


где использован тот факт, что \(\left[\mathbf{v}_{0} \mathbf{B}_{\perp}\right] \cdot\left[\mathbf{v}_{0} \mathbf{E}_{\perp}\right]=v_{0}^{2} B_{\perp} E_{\perp} \cos \alpha=v_{0}^{2} \mathbf{B}_{\perp} \mathbf{E}_{\perp}\) (рис. 8.11). Остальные два скалярных произведения в (2) равны нулю, поскольку векторы взаимно перпендикулярны.

Таким образом, правая часть равенства (1) приобретает следующий вид:

\begin{figure}[h]
\begin{center}
  \includegraphics[alt={},max width=\textwidth]{3e870b25-6320-483d-9133-cdd5212c2707-222_270_361_1156_778}
\captionsetup{labelformat=empty}
\caption{Рис. 8.11}
\end{center}
\end{figure}

\[
\mathbf{E}_{| |}^{\prime} \mathbf{B}_{| |}^{\prime}+\mathbf{E}_{\perp}^{\prime} \mathbf{B}_{\perp}^{\prime}=\mathbf{E}_{| |} \mathbf{B}_{| |}+\mathbf{E}_{\perp} \mathbf{B}_{\perp}=\mathbf{E} \mathbf{B},
\]

что и требовалось показать.\\
8.10. Поле Е равномерно движущегося заряда. Точечный заряд \(q\) движется равномерно и прямолинейно с релятивистской скоростью \(\mathbf{v}\). Найти напряженность \(\mathbf{E}\) поля этого заряда в точке, ради-ус-вектор которой относительно заряда равен \(\mathbf{r}\) и составляет угол \(\vartheta\) с вектором \(\mathbf{v}\).

Решение. Пусть заряд движется в положительном направлении оси \(X K\)-системы отсчета. Перейдем в \(K^{\prime}\)-систему, в начале координат которой этот заряд покоится (оси \(X^{\prime}\) и \(X\) обеих систем совпадают, оси \(Y^{\prime}\) и \(Y\) - параллельны). В \(K^{\prime}\)-системе поле \(\mathbf{E}^{\prime}\) заряда имеет наиболее простой вид

\[
\mathrm{E}^{\prime}=\frac{1}{4 \pi \varepsilon_{0}} \frac{q}{r^{\prime 3}} \mathrm{r}^{\prime}
\]

и в плоскости \(X^{\prime} Y^{\prime}\)


\begin{equation*}
E_{x}^{\prime}=\frac{1}{4 \pi \varepsilon_{0}} \frac{q}{r^{\prime 3}} x^{\prime}, \quad E_{y}^{\prime}=\frac{1}{4 \pi \varepsilon_{0}} \frac{q}{r^{\prime 3}} y^{\prime} . \tag{1}
\end{equation*}


Теперь совершим обратный переход в исходную \(K\)-систему, которая движется относительно \(K^{\prime}\)-системы со скоростью \(-\mathbf{v}\). В момент, когда заряд проходит через начало координат \(K\)-системы, проекции \(x\) и \(y\) вектора \(\mathbf{r}\) связаны с проекциями \(x^{\prime}\) и \(y^{\prime}\) вектора \(\mathbf{r}^{\prime}\) следующими соотношениями:


\begin{equation*}
x=r \cos \vartheta=x^{\prime} \sqrt{1-\beta^{2}}, \quad y=r \sin \vartheta=y^{\prime}, \tag{2}
\end{equation*}


где \(\beta=v / c\). Здесь учтено, что продольные размеры испытывают лоренцево сокращение, поперечные же не меняются. Кроме того, согласно преобразованиям, обратным (8.2),

\[
E_{x}=E_{x}^{\prime}, \quad E_{y}=E_{y}^{\prime} / \sqrt{1-\beta^{2}} .
\]

Подставив сюда выражения (1), а в них вместо \(x^{\prime}\) и \(y^{\prime}\) соответствующие выражения из формул (2), получим

\[
E_{x}=\frac{1}{4 \pi \varepsilon_{0}} \frac{q}{r^{\prime 3}} \frac{x}{\sqrt{1-\beta^{2}}}, \quad E_{y}=\frac{1}{4 \pi \varepsilon_{0}} \frac{q}{r^{\prime 3}} \frac{y}{\sqrt{1-\beta^{2}}} .
\]

Заметим, что \(E_{x} / E_{y}=x / y\), т. е. вектор \(\mathbf{E}\) направлен радиально, вдоль вектора \(\mathbf{r}\). Дело обстоит так, как если бы эффект запаздывания вообще отсутствовал. Но это имеет место только в случае \(\mathbf{v}=\) const, если же заряд движется с ускорением, поле Е оказывается не радиальным. Остается найти модуль вектора \(\mathbf{E}\) :

\[
E=\sqrt{E_{x}^{2}+E_{y}^{2}}=\frac{1}{4 \pi \varepsilon_{0}} \frac{q}{r^{\prime 3}} \sqrt{\frac{x^{2}+y^{2}}{1-\beta^{2}}} .
\]

Так как \(x^{2}+y^{2}=r^{2}\) и согласно (2)

\[
r^{\prime 3}=\left(x^{\prime 2}+y^{\prime 2}\right)^{3 / 2}=r^{3}\left(\frac{1-\beta^{2} \sin ^{2} \vartheta}{1-\beta^{2}}\right)^{3 / 2},
\]

то напряженность

\[
E=\frac{1}{4 \pi \varepsilon_{0}} \frac{q}{r^{2}} \frac{1-\beta^{2}}{\left(1-\beta^{2} \sin ^{2} \vartheta\right)^{3 / 2}}
\]

8.11. Взаимодействие двух движущихся зарядов. Две релятивистские частицы с одинаковым зарядом \(q\) движутся параллельно друг другу с одинаковой скоростью \(v\), как показано на рис. 8.12. Расстояние между частицами \(l\). Воспользовавшись выражением (8.7), найти силу взаимодействия между частицами.\\
Решение. В данном случае угол между век-

\begin{figure}[h]
\begin{center}
  \includegraphics[alt={},max width=\textwidth]{3e870b25-6320-483d-9133-cdd5212c2707-224_212_305_378_855}
\captionsetup{labelformat=empty}
\caption{Рис. 8.12}
\end{center}
\end{figure}

тором \(\mathbf{v}\) одной из частиц и направлением на другую частицу \(\vartheta=90^{\circ}\), поэтому электрическая часть силы Лоренца в соответствии с формулой (8.7)


\begin{equation*}
F_{Э}=q E=\frac{1}{4 \pi \varepsilon_{0}} \frac{q^{2}}{l^{2} \sqrt{1-\beta^{2}}} \tag{1}
\end{equation*}


и магнитная часть силы Лоренца


\begin{equation*}
F_{\mathrm{M}}=q v B=\frac{\mu_{0}}{4 \pi} \frac{q^{2} v^{2}}{l^{2} \sqrt{1-\beta^{2}}} \tag{2}
\end{equation*}


где принято во внимание, что в нашем случае \(\mathbf{B}\) связано с \(\mathbf{E}\) формулой (8.5), из которой \(B=v E / c^{2}, c^{2}=1 / \varepsilon_{0} \mu_{0}\). Заметим, что отношение

\[
F_{\mathrm{M}} / F_{Э}=\varepsilon_{0} \mu_{0} v^{2}=(v / c)^{2}
\]

как и в нерелятивистском случае (6.5). Видно, что при \(v \rightarrow c\) магнитная часть силы \(F_{\text {м }} \rightarrow F_{\text {э }}\).\\
Результирующая сила взаимодействия (отталкивания)

\[
F=F_{\vartheta}-F_{\mathrm{M}}=\frac{1}{4 \pi \varepsilon_{0}} \frac{q^{2}}{l^{2}} \sqrt{1-\beta^{2}} .
\]

\section*{Электромагнитная индукция}
\section*{§ 9.1. Закон электромагнитной индукции. Правило Ленца}
В предыдущей главе мы установили, что существует электромагнитное поле, соотношение между «компонентами» которого - электрическим и магнитным полями - в решающей степени зависит от системы отсчета. Другими словами, обе компоненты электромагнитного поля связаны друг с другом. В этой главе мы увидим, что существует еще более глубокая связь между \(\mathbf{E}\) - и \(\mathbf{B}\)-полями и обнаруживается она в явлениях электромагнитной индукции.

Открытие Фарадея. В 1831 г. Фарадеем было сделано одно из наиболее фундаментальных открытий в электродинамике явление электромагнитной индукции. Оно заключается в том, что в замкнутом проводящем контуре при изменении магнитного потока (т. е. потока вектора В), охватываемого этим контуром, возникает электрический ток - его назвали индукционным.

Появление индукционного тока означает, что при изменении магнитного потока в контуре возникает э.д.с. индукции \(\mathrm{E}_{i}\). При этом весьма замечателен тот факт, что \(\mathrm{E}_{i}\) совершенно не зависит от того, каким образом осуществляется изменение магнитного потока \(Φ\), и определяется лишь скоростью его изменения, т. е. величиной \(\mathrm{d} \Phi / \mathrm{d} t\).

И еще, изменение знака производной \(\mathrm{d} \Phi / \mathrm{d} t\) приводит к изменению знака или «направления» \(\mathrm{E}_{i}\).

Фарадей обнаружил, что индукционный ток можно вызвать двумя различными способами. Дальнейшее поясняет рис. 9.1, где изображены катушка \(K\) с током \(I\) (она создает магнитное поле) и рамка \(P\) с гальванометром \(\Gamma\) - индикатором индукционного тока.

1-й способ - перемещение рамки \(P\) (или отдельных ее частей) в поле неподвижной катушки \(K\).

2-й способ - рамка \(P\) неподвижна, но изменяется магнитное поле - или за счет движения катушки \(K\), или вследствие изменения силы тока \(I\) в ней, или в результате того и другого вместе.

Во всех этих случаях гальванометр \(\Gamma\) будет показывать наличие индукцион-

\begin{figure}[h]
\begin{center}
  \includegraphics[alt={},max width=\textwidth]{3e870b25-6320-483d-9133-cdd5212c2707-226_344_378_149_782}
\captionsetup{labelformat=empty}
\caption{Рис. 9.1}
\end{center}
\end{figure}

ного тока в рамке \(P\).

Правило Ленца. Направление индукционного тока (а значит, и знак э.д.с. индукции) определяется правилом Ленца: индукционный ток всегда направлен так, чтобы противодействовать причине, его вызывающей. Иначе говоря, индукционный ток создает магнитный поток, препятствующий изменению магнитного потока, вызывающего э.д.с. индукции.

Если, например, рамку \(P\) (рис. 9.1) приближать к катушке \(K\), то магнитный поток сквозь рамку возрастает. При этом в рамке возникает индукционный ток, направленный по часовой стрелке (если смотреть справа на рамку). Этот ток создает магнитный поток, «направленный» влево, он и препятствует возрастанию магнитного потока, вызывающего этот ток.

То же произойдет, если увеличивать силу тока в катушке \(K\), оставляя катушку и рамку \(P\) неподвижными. При уменьшении же силы тока в катушке \(K\) индукционный ток в рамке \(P\) изменит свое направление на противоположное (против часовой стрелки, если смотреть справа).

Индукционные токи возбуждаются и в массивных сплошных проводниках. Они имеют вихревой характер, и их называют токами Фуко. Эти токи могут достигать очень большой силы, что и используют (например, в некоторых тормозных системах и др.).

Правило Ленца выражает существенный физический факт - стремление системы противодействовать изменению ее состояния (электромагнитная инериия).

Закон электромагнитной индукции. Согласно этому закону, какова бы ни была причина изменения магнитного потока, охватываемого замкнутым проводящим контуром, возникающая

в контуре э.д.с. индукции определяется формулой


\begin{equation*}
\mathcal{E}_{i}=-\frac{\mathrm{d} \Phi}{\mathrm{~d} t} . \tag{9.1}
\end{equation*}


Знак минус в этом уравнении связан с определенным правилом знаков. Знак магнитного потока Ф связан с выбором нормали к поверхности \(S\), ограниченной рассматриваемым контуром, а знак э.д.с. индукции \(\mathcal{E}_{i}\) - с выбором положительного направления обхода по контуру.

Здесь предполагается (как и ранее), что направление норма-

\begin{figure}[h]
\begin{center}
  \includegraphics[alt={},max width=\textwidth]{3e870b25-6320-483d-9133-cdd5212c2707-227_189_225_588_68}
\captionsetup{labelformat=empty}
\caption{Рис. 9.2}
\end{center}
\end{figure}

ли n к поверхности \(S\) и положительное направление обхода контура связаны друг с другом правилом правого винта* (рис. 9.2). Поэтому, выбирая (произвольно) направление нормали, мы определяем как знак потока Ф, так и знак (а значит, и «направление») э.д.с. индукции \(\mathcal{E}_{i}\).\\
При сделанном нами выборе положительных направлений - в соответствии с правилом правого винта - величины \(\mathcal{E}_{i}\) и \(\mathrm{d} \Phi / \mathrm{d} t\) имеют противоположные знаки.

Единицей магнитного потока является вебер (Вб). При скорости изменения магнитного потока 1 Вб/с в контуре индуцируется э.д.с., равная 1 В [см. (9.1)].

Полный магнитный поток (потокосцепление). Если замкнутый контур, в котором индуцируется э.д.с., состоит не из одного витка, а из \(N\) витков (например, катушка), то \(\mathcal{E}_{i}\) будет равна сумме э.д.с., индуцируемых в каждом из витков. И если магнитный поток, охватываемый каждым витком, одинаков и равен \(\Phi_{1}\), то суммарный поток \(\Phi\) сквозь поверхность, натянутую на такой сложный контур, можно представить как


\begin{equation*}
\Phi=N \Phi_{1} . \tag{9.2}
\end{equation*}


Эту величину называют полным магнитным потоком или потокосцеплением. В этом случае соответствующая э.д.с. индукции в контуре определяется согласно (9.1) формулой


\begin{equation*}
\mathcal{E}_{i}=-N \frac{\mathrm{~d} \Phi_{1}}{\mathrm{~d} t} . \tag{9.3}
\end{equation*}


\footnotetext{\begin{itemize}
  \item Если бы оба эти направления были связаны правилом левого винта, знака минус в уравнении (9.1) просто не было бы.
\end{itemize}
}\section*{§ 9.2. Природа электромагнитной индукции}
Теперь мы должны разобраться в тех физических причинах, которые приводят к возникновению э.д.с. индукции, и попытаться вывести закон индукции (9.1) из того, что нам уже известно. Рассмотрим последовательно два случая.

Контур движется в постоянном магнитном поле. Прежде всего обратимся к контуру с подвижной перемычкой длиной \(l\) (рис. 9.3). Пусть он находится в однородном магнитном поле, перпендикулярном плоскости контура и направленном за плоскость рисунка. Начнем двигать перемычку вправо со скоростью \(\mathbf{v}\). С такой же скоростью нач-

\begin{figure}[h]
\begin{center}
  \includegraphics[alt={},max width=\textwidth]{3e870b25-6320-483d-9133-cdd5212c2707-228_220_314_484_826}
\captionsetup{labelformat=empty}
\caption{Рис. 9.3}
\end{center}
\end{figure}

нут двигаться и носители тока в перемычке - электроны. В результате на каждый электрон начнет действовать вдоль перемычки магнитная сила \(\mathbf{F}=-e[\mathbf{v B}]\), и электроны начнут перемещаться по перемычке вниз - потечет ток, направленный вверх. Это и есть индукционный ток. Перераспределившиеся заряды (на поверхности проводников) создадут электрическое поле, которое возбудит ток и в остальных участках контура.

Магнитная сила \(\mathbf{F}\) играет роль сторонней силы. Ей соответствует поле \(\mathbf{E}^{*}=\mathbf{F} /(-e)=[\mathbf{v B}]\). Заметим, что это выражение можно получить и с помощью формул преобразования полей (8.4).

Циркуляция вектора \(\mathbf{E}^{*}\) по контуру дает по определению величину э.д.с. индукции. В нашем случае


\begin{equation*}
\mathcal{E}_{i}=-v B l, \tag{9.4}
\end{equation*}


где знак минус поставлен в связи с принятым правилом знаков: нормаль \(\mathbf{n}\) к поверхности, натянутой на наш контур, мы выбрали за плоскость рис. 9.3 (в сторону поля В), и поэтому по правилу правого винта положительное направление обхода контура - по часовой стрелке, как показано на рисунке. При этом стороннее поле \(\mathbf{E}^{*}\) направлено против положительного направления обхода контура и \(\mathcal{E}_{i}\) - величина отрицательная.

Произведение \(v l\) в (9.4) есть приращение площади, ограниченной контуром, в единицу времени ( \(\mathrm{d} S / \mathrm{d} t\) ), поэтому \(v B l= =B \mathrm{~d} S / \mathrm{d} t=\mathrm{d} \Phi / \mathrm{d} t\), где \(\mathrm{d} \Phi\) - приращение магнитного потока сквозь площадь контура (в нашем случае \(\mathrm{d} \Phi>0\) ). Таким образом,


\begin{equation*}
\mathcal{E}_{i}=-\mathrm{d} \Phi / \mathrm{d} t . \tag{9.5}
\end{equation*}


Можно в общем виде доказать, что закон (9.1) справедлив для любого контура, движущегося произвольным образом в постоянном неоднородном магнитном поле (см. задачу 9.2).

Итак, возбуждение э.д.с. индукции при движении контура в постоянном магнитном поле объясняется действием магнитной силы c [vB], которая возникает при движении проводника.

Заметим попутно, что идея схемы (рис. 9.3) лежит в основе действия всех индукционных генераторов тока, в которых ротор с обмоткой вращается во внешнем магнитном поле.

Контур покоится в переменном магнитном поле. Возникновение индукционного тока и в этом случае свидетельствует о том, что изменяющееся во времени магнитное поле вызывает в контуре появление сторонних сил. Но что это за силы? Какова их природа? Ясно, что это не магнитные силы ० [vB]: привести в движение покоившиеся ( \(\mathbf{v}=0\) ) заряды эти силы не могут. Но других сил, кроме \(q \mathbf{E}\) и \(q[\mathbf{v B}]\), нет! Остается заключить, что индукционный ток обусловлен возникающим в проводе электрическим полем Е. Именно это поле и ответственно за появление э.д.с. индукции в неподвижном контуре при изменении во времени магнитного поля.

Максвелл предположил, что изменяющееся во времени магнитное поле приводит к появлению в пространстве электрического поля независимо от наличия проводящего контура. Последний лишь позволяет обнаружить по возникновению в нем индукционного тока существование этого электрического поля.

Таким образом, согласно Максвеллу изменяющееся со временем магнитное поле порождает электрическое поле. Циркуляция вектора \(\mathbf{E}\) этого поля по любому неподвижному контуру определяется как


\begin{equation*}
\oint \mathbf{E} \mathrm{d} \mathbf{l}=-\frac{\partial \Phi}{\partial t} . \tag{9.6}
\end{equation*}


Здесь символ частной производной по времени ( \(\partial / \partial t\) ) подчеркивает тот факт, что контур и натянутая на него поверхность неподвижны. Так как поток \(\Phi=\int \mathbf{B} d \mathbf{S}\) (интегрирование проводится по произвольной поверхности, натянутой на интересующий нас контур), то

\[
\frac{\partial}{\partial t} \int \mathbf{B} \mathrm{~d} \mathbf{S}=\int \frac{\partial \mathbf{B}}{\partial t} \mathrm{~d} \mathbf{S} .
\]

В этом равенстве мы поменяли местами операции дифференцирования по времени и интегрирования по поверхности, поскольку контур и поверхность неподвижны. Тогда уравнение (9.6) можно представить в виде


\begin{equation*}
\oint \mathbf{E} \mathrm{d} \mathbf{l}=-\int \frac{\partial \mathbf{B}}{\partial t} \mathrm{~d} \mathbf{S} . \tag{9.7}
\end{equation*}


Данное уравнение имеет ту же структуру, что и уравнение (6.17), причем роль вектора \(\mathbf{j}\) играет вектор \(-\partial \mathbf{B} / \partial t\). Стало быть, оно может быть преобразовано в дифференциальную форму так же, как и уравнение (6.26). И в результате мы получим


\begin{equation*}
\nabla \times \mathbf{E}=-\partial \mathbf{B} / \partial t . \tag{9.8}
\end{equation*}


Это уравнение выражает локальную связь между электрическим и магнитным полями: изменение поля В во времени в данной точке определяет ротор поля \(\mathbf{E}\) в этой же точке. Отличие же \(\nabla \times \mathbf{E}\) от нуля свидетельствует о наличии самого электрического поля.

Тот факт, что циркуляция электрического поля, возбуждаемого изменяющимся со временем магнитным полем, отлична от нуля, означает, что это электрическое поле не потенциально. Оно, как и магнитное поле, является вихревьлм. Таким образом, электрическое поле может быть как потенциальным (в электростатике), так и вихревым.

В общем случае электрическое поле \(\mathbf{E}\) может слагаться из электростатического поля и поля, обусловленного изменяющимся во времени магнитным полем. Поскольку циркуляция электростатического поля равна нулю, уравнения (9.6)-(9.8) оказываются справедливыми и для общего случая, когда поле E представляет собой векторную сумму этих двух полей.

Бетатрон. Вихревое электрическое поле нашло замечательное применение в индукционном ускорителе электронов - бетатроне. Этот ускоритель состоит из тороидальной откачанной камеры, расположен-

\begin{figure}[h]
\begin{center}
  \includegraphics[alt={},max width=\textwidth]{3e870b25-6320-483d-9133-cdd5212c2707-231_263_296_268_70}
\captionsetup{labelformat=empty}
\caption{Рис. 9.4}
\end{center}
\end{figure}

ной между полюсами электромагнита (рис. 9.4). Изменение тока в обмотке электромагнита создает переменное магнитное поле, которое вызывает вихревое электрическое поле, ускоряющее электроны, и одновременно удерживает электроны на равновесной круговой орбите определенного радиуса (см. задачу 9.5). Так как электрическое поле вихревое, направление силы, действующей на электроны, все время совпадает с направлением движения и электроны непрерывно увеличивают свою энергию. За время нарастания магнитного поля ( \(\sim\) мс) электроны успевают сделать до миллиона оборотов и приобретают энергию, которая может достигать 400 МэВ (скорость электронов при таких энергиях становится почти равной \(c\) - скорости света в вакууме).

Индукционный ускоритель (бетатрон) подобен трансформатору, у которого роль вторичной обмотки из одного витка играет пучок электронов.

Заключение. Итак, закон электромагнитной индукции (9.1) справедлив, когда магнитный поток сквозь контур меняется за счет движения контура или за счет изменения магнитного поля со временем (или когда происходит и то и другое). Вместе с тем для объяснения закона в этих двух случаях пришлось использовать два совершенно разньιx явления: для движущегося контура - действие магнитной силы с [vB], а для меняющегося во времени поля \(\partial \mathbf{B} / \partial t\) - представление о возникающем вихревом электрическом поле Е.

Ввиду того что никакого единого глубокого принципа, объединяющего оба явления, не видно, мы должны воспринимать закон электромагнитной индукции как совместный эффект двух совершенно различных явлений. Оба эти явления, вообще говоря, независимы друг от друга, и тем не менее - что удивительно - э.д.с. индукции в контуре всегда равна скорости изменения магнитного потока сквозь контур.

Иначе говоря, в тех случаях, когда меняется и поле В во времени, и конфигурация или расположение контура в поле, э.д.с. индукции надо рассчитывать по формуле (9.1), где справа стоит полная производная \(\mathrm{d} \Phi / \mathrm{d} t\) по времени, автоматически

учитывающая оба фактора. В связи с этим закон (9.1) можно представить в таком виде:


\begin{equation*}
\oint \mathbf{E} \mathrm{d} \mathbf{l}=-\frac{\partial \Phi}{\partial t}+\oint[\mathbf{v B}] \mathrm{d} \mathbf{l} . \tag{9.9}
\end{equation*}


Выражение, стоящее в правой части этого равенства, представляет собой полную производную \(-\mathrm{d} \Phi / \mathrm{d} t\). Здесь первое слагаемое связано с изменением магнитного поля во времени, второе - с движением контура. Происхождение второго слагаемого более подробно объяснено в задаче 9.2.

Возможные затруднения. Иногда приходится сталкиваться с ситуациями, где закон электромагнитной индукции в форме (9.1) оказывается неприменимым (в основном из-за трудностей, связанных с выбором самого контура). В этих случаях необходимо обращаться к основным законам - силе Лоренца \(q \mathbf{E}+q[\mathbf{v B}]\) и закону \(\boldsymbol{\nabla} \times \mathbf{E}=-\partial \mathbf{B} / \partial t\). Именно они во всех случаях выражают физическое содержание закона электромагнитной индукции.

Вот два поучительных примера.\\
Пример 1. Проводящую ленту перемещают со скоростью \(v\) через область, в которой имеется магнитное поле В (рис. 9.5). Этой областью на рисунке является отмеченный точками кру-

\begin{figure}[h]
\begin{center}
  \includegraphics[alt={},max width=\textwidth]{3e870b25-6320-483d-9133-cdd5212c2707-232_168_434_1008_726}
\captionsetup{labelformat=empty}
\caption{Рис. 9.5}
\end{center}
\end{figure}

жок, где поле В направлено на нас. Гальванометр \(\Gamma\) подключен к неподвижным контактам (стрелкам), с которыми соприкасается движущаяся лента. Будет ли гальванометр показывать наличие тока?\\
Вопрос на первый взгляд кажется не простым, поскольку здесь затруднительно выбрать сам контур: не ясно, где его «замкнуть» в ленте и как этот участок контура должен вести себя при движении ленты. Однако если мы обратимся к силе Лоренца, то станет сразу же понятно, что электроны в движущейся ленте будут смещаться вверх, и это даст ток в цепи, направленный по часовой стрелке.

Заметим, что идея этого опыта легла в основу создания магнитогидродинамического генератора (МГДГ), в котором используется метод прямого превращения внутренней\\
(тепловой) энергии в электрическую. Вместо проводящей ленты там с большой скоростью продувают плазму (она состоит из электронов и положительных ионов). В остальном все обстоит так же, как и для проводящей ленты.

Пример 2. На рис. 9.6 отмеченный точками кружок показывает область, в которой локализовано постоянное магнитное поле

\begin{figure}[h]
\begin{center}
  \includegraphics[alt={},max width=\textwidth]{3e870b25-6320-483d-9133-cdd5212c2707-233_342_335_403_242}
\captionsetup{labelformat=empty}
\caption{Рис. 9.6}
\end{center}
\end{figure}

В (оно направлено перпендикулярно плоскости рисунка). Эта область охватывается неподвижным металлическим кольцом \(K\). Переместив скользящие контакты на другую сторону кольца, мы введем магнитный поток Ф в замкнутый контур, содержащий гальванометр \(\Gamma\) (1 исходное положение, 2 - конечное). Покажет ли при этом гальванометр импульс тока?

Применяя формально закон (9.1), мы должны заключить, что индукционный ток будет. Однако это не так! Тока нет, ибо здесь и \(\partial \mathbf{B} / \partial t\), и сила Лоренца равны нулю: поле \(\mathbf{B}\) постоянное и замкнутый контур перемещается в области, где нет магнитного поля. Таким образом, здесь нет ни одной из двух физических причин, лежащих в основе закона электромагнитной индукции.

О кажущемся парадоксе. Мы знаем, что сила, испытываемая электрическим зарядом в магнитном поле, перпендикулярна его скорости и потому никакой работы не совершает. Между тем при движении проводника с током (движущиеся заряды!) силы Ампера, несомненно, совершают работу (электромотор!). В чем здесь дело?

Это кажущееся противоречие исчезает, если учесть, что движение проводника в магнитном поле неизбежно сопровождается явлением электромагнитной индукции. И именно потому, что в проводнике индуцируется э.д.с., совершающая работу над зарядами, полная работа сил магнитного поля (работа силы Ампера и работа э.д.с. индукции) равна нулю. В самом деле, при элементарном перемещении контура с током в магнитном поле силы Ампера совершают (см. § 6.8) работу


\begin{equation*}
\delta A_{A}=I \mathrm{~d} \Phi, \tag{9.10}
\end{equation*}


а э.д.с. индукции за это же время выполняет работу


\begin{equation*}
\delta A_{i}=\mathcal{E}_{i} I \mathrm{~d} t=-I \mathrm{~d} \Phi, \tag{9.11}
\end{equation*}


где учтено, что \(\mathcal{E}_{i}=-\mathrm{d} \Phi / \mathrm{d} t\). Из последних двух формул видно, что полная работа


\begin{equation*}
\delta A_{A}+\delta A_{i}=0 . \tag{9.12}
\end{equation*}


Итак, в работу сил магнитного поля входит не только механическая работа (обусловленная силами Ампера), но и работа э.д.с., индуцируемой при движении контура. Обе работы равны по модулю и противоположны по знаку, поэтому их сумма и равна нулю.

Работа сил Ампера совершается не за счет энергии внешнего магнитного поля, а за счет источника, поддерживающего ток в контуре. При этом источник совершает дополнительную работу против э.д.с. индукции \(\mathrm{d} A^{\text {доп }}=-\mathcal{E}_{i} I \mathrm{~d} t=I \mathrm{~d} \Phi\), которая оказывается одинаковой с работой \(\delta A_{A}\) сил Ампера.

Работа \(\delta A\), которая совершается при перемещении контура против тормозящих амперовых сил (они возникают благодаря появлению индукционного тока в соответствии с правилом Ленца), преобразуется в работу э.д.с. индукции:


\begin{equation*}
\delta A=-\delta A_{A}=\delta A_{i} . \tag{9.13}
\end{equation*}


С энергетической точки зрения в этом заключается сущность действия всех индукционных генераторов тока.

\section*{§ 9.3. Явление самоиндукции}
Электромагнитная индукция возникает во всех случаях, когда изменяется магнитный поток сквозь контур. При этом совершенно не важно, чем вызывается это изменение потока. Если в некотором контуре течет изменяющийся во времени ток, то магнитное поле этого тока также будет изменяться. Это влечет за собой изменение магнитного потока через контур, а следовательно, и появление э.д.с. индукции.

Таким образом, изменение тока в контуре ведет к возникновению э.д.с. индукции в этом же самом контуре. Данное явление называют самоиндукцией.

Индуктивность. Если в пространстве, где находится контур с током \(I\), нет ферромагнетиков, поле В, а значит, и полный магнитный поток Ф через контур будут пропорциональны силе тока \(I\), и можно написать


\begin{equation*}
\Phi=L I, \tag{9.14}
\end{equation*}


где \(L\) - коэффициент, называемый индуктивностью контура. В соответствии с принятым правилом знаков для величин Ф и \(I\) оказывается, что Ф и \(I\) всегда имеют одинаковые знаки. Это означает, что индуктивность \(L\) - величина существенно положительная.

Индуктивность \(L\) зависит от формы и размеров контура, а также от магнитных свойств окружающей среды. Если контур жесткий и поблизости от него нет ферромагнетиков, индуктивность является величиной постоянной, не зависящей от силы тока \(I\).

Единицей индуктивности является генри (Гн). Согласно (9.14) индуктивностью 1 Гн обладает контур, магнитный поток сквозь который при токе 1 А равен 1 Вб, значит 1 Гн \(=1\) Вб/А.\\
Пример. Найдем индуктивность соленоида, пренебрегая краевыми эффектами. Пусть \(V\) - объем соленоида, \(n\) - число витков на единицу его длины, \(\mu\) - магнитная проницаемость вещества внутри соленоида.\\
Согласно (9.14) \(L=Φ / I\). Следовательно, задача сводится к тому, чтобы, задавшись током \(I\), определить полный магнитный поток Ф. При токе \(I\) магнитное поле в соленоиде \(B= =\mu \mu_{0} n I\). Магнитный поток через один виток соленоида \(\Phi_{1}=B S=\mu \mu_{0} n I S\), а полный магнитный поток, пронизывающий \(N\) витков:

\[
\Phi=N \Phi_{1}=n l \cdot B S=\mu \mu_{0} n^{2} V I,
\]

где \(V=S l\). Отсюда индуктивность соленоида


\begin{equation*}
L=\mu \mu_{0} n^{2} V \tag{9.15}
\end{equation*}


О некоторых трудностях. Отметим, что определение индуктивности по формуле \(L=\Phi / I\) связано с определенными трудностями. Как бы ни был тонок провод, его сечение конечно, и мы просто не знаем, как надо провести в теле проводника геометрический контур, необходимый для вычисления Ф. Результат

оказывается неоднозначным. Для достаточно тонкого провода эта неоднозначность мало существенна, чего совершенно нельзя сказать о толстых проводах: здесь из-за неопределенности выбора геометрического контура результат вычисления \(L\) может содержать большую ошибку. Об этом не следует забывать.

Дальше (см. § 9.5) будет показано, что существует другой способ определения \(L\), полностью свободный от указанной трудности.\\
Э.д.с. самоиндукции. При изменении силы тока в контуре согласно (9.1) возникает э.д.с. самоиндукции \(\mathcal{E}_{s}\) :


\begin{equation*}
\mathcal{E}_{s}=-\frac{\partial \Phi}{\partial t}=-\frac{\mathrm{d}}{\mathrm{~d} t}(L I) \tag{9.16}
\end{equation*}


Если при изменении тока индуктивность \(L\) остается постоянной (не меняется конфигурация контура и нет ферромагнетиков), то


\begin{equation*}
\mathcal{E}_{s}=-L \frac{\mathrm{~d} I}{\mathrm{~d} t}(L=\text { const }) \tag{9.17}
\end{equation*}


Здесь знак минус показывает, что \(\mathcal{E}_{s}\) всегда направлена так, чтобы препятствовать изменению силы тока - в соответствии с правилом Ленца. Эта э.д.с. стремится сохранить ток неизменным: она противодействует току, когда он увеличивается, и поддерживает ток, когда он уменьшается. В явлениях самоиндукции ток обладает «инерцией», потому что эффекты индукции стремятся сохранить магнитный поток постоянным, точно так же, как механическая инерция стремится сохранить скорость тела неизменной.

Примеры проявления самоиндукции. Характерные проявления самоиндукции наблюдаются при замыкании и размыкании тока в цепи. Установление тока при замыкании цепи и убывание тока при размыкании цепи происходят не мгновенно, а постепенно. Причем эти эффекты замедления тем значительнее, чем больше индуктивность цепи.

Любой большой электромагнит обладает большой индуктивностью. Если его обмотку отсоединить от источника, ток быстро уменьшается до нуля и в процессе уменьшения создает огромную э.д.с. самоиндукции. Это часто приводит к образова-

нию вольтовой дуги между контактами выключателя и является весьма опасньмм, причем не только для обмотки электромагнита, но и для человека, размыкающего цепь. По этим причинам параллельно обмотке электромагнита обычно включают лампочку с сопротивлением того же порядка, что и сопротивление обмотки. В этом случае ток в обмотке спадает медленно и опасности не представляет.

Теперь рассмотрим более подробно характер исчезновения и установления тока в цепи.

\section*{Пример 1. Исчезновение тока при размыкании цепи.}
Пусть цепь состоит из постоянной индуктивности \(L\), сопротивления \(R\), амперметра \(A\), источника э.д.с. Е и специаль-

\begin{figure}[h]
\begin{center}
  \includegraphics[alt={},max width=\textwidth]{3e870b25-6320-483d-9133-cdd5212c2707-237_344_430_692_203}
\captionsetup{labelformat=empty}
\caption{Рис. 9.7}
\end{center}
\end{figure}

ного ключа \(K\) (рис. 9.7, a). Первоначально ключ \(K\) находится в нижнем положении (рис. 9.7, б), и в цепи течет ток \(I_{0}=\mathbf{E} / R\) (coпротивление источника э.д.с. Е считаем пренебрежимо малым).

В момент \(t=0\) быстро повернем ключ \(K\) по часовой стрелке из нижнего положения в верхнее (рис. 9.7, \(a\) ). При этом произойдет следующее: на очень короткое время ключ закоротил источник \(\mathbf{E}\) и тут же выключил его из цепи, не нарушая ее замкнутости.

Ток через индуктивность \(L\) начнет убывать, а это значит, что возникнет э.д.с. самоиндукции \(\mathcal{E}_{s}=-L \mathrm{~d} I / \mathrm{d} t\), противодействующая, по Ленцу, убыванию тока. В каждый момент ток в цепи будет определяться законом Ома \(I=\mathcal{E}_{s} / R\), или


\begin{equation*}
R I=-L \frac{\mathrm{~d} I}{\mathrm{~d} t} \tag{9.18}
\end{equation*}


Разделив переменные, получим

\[
\frac{\mathrm{d} I}{I}=-\frac{R}{L} \mathrm{~d} t .
\]

Интегрирование этого уравнения по \(I\) (от \(I_{0}\) до \(I\) ) и по \(t\) (от 0 до \(t\) ) дает \(\ln \left(I / I_{0}\right)=-R t / L\), или


\begin{equation*}
I=I_{0} \mathrm{e}^{-t / \tau}, \tag{9.19}
\end{equation*}


где \(\tau\) - постоянная, имеющая размерность времени,


\begin{equation*}
\tau=L / R . \tag{9.20}
\end{equation*}


Ее называют постоянной времени (временем релаксации). Эта величина характеризует скорость убывания тока: из (9.19) следует, что \(\tau\) есть время, в течение которого сила тока уменьшается в е раз. Чем больше значение \(\tau\), тем медленнее спадает ток. На рис. 9.8 показан график зависимости \(I(t)\) - убывания силы тока со временем (кри-

\begin{figure}[h]
\begin{center}
  \includegraphics[alt={},max width=\textwidth]{3e870b25-6320-483d-9133-cdd5212c2707-238_264_384_347_778}
\captionsetup{labelformat=empty}
\caption{Рис. 9.8}
\end{center}
\end{figure}

\section*{Пример 2. Установление тока при замыкании цепи.}
В момент \(t=0\) быстро повернем ключ \(K\) против часовой стрелки из верхнего положения в нижнее (рис. 9.7, б). Этим самым мы подключили к индуктивности \(L\) источник Е. Ток в цепи начнет нарастать и опять возникает э.д.с. самоиндукции, противодействующая этому нарастанию. Cогласно закону Ома \(R I=\mathcal{E}+\mathcal{E}_{s}\), или


\begin{equation*}
R I=\mathcal{E}-L \frac{\mathrm{~d} I}{\mathrm{~d} t} \tag{9.21}
\end{equation*}


Перенесем \(\mathcal{E}\) в левую часть уравнения и введем новую переменную \(u=R I-\mathcal{E}, \mathrm{d} u=R \mathrm{~d} I\). После этого полученное уравнение преобразуем к виду

\[
\mathrm{d} u / u=-\mathrm{d} t / \tau,
\]

где \(\tau=L / R\) - постоянная времени.\\
Интегрирование по \(u\) (от \(-\mathcal{E}\) до \(R I-\mathcal{E}\) ) и по \(t\) (от 0 до \(t\) ) дает \(\ln [(R I-\mathcal{E}) /(-\mathcal{E})]=-t / \tau\) или


\begin{equation*}
I=I_{0}\left(1-\mathrm{e}^{-t / \tau}\right), \tag{9.22}
\end{equation*}


где \(I_{0}=\mathcal{E} / R\) представляет собой установившийся ток (при \(t \rightarrow \infty\) ). Из уравнения (9.22) видно, что быстрота установления тока определяется той же постоянной \(\tau\). График зависимости \(I(t)\) - возрастания силы тока со временем показан на рис. 9.8 (кривая 2).

О сохранении магнитного потока. Пусть в произвольном внешнем магнитном поле - постоянном или переменном движется и деформируется контур с током. При этом в контуре индуцируется ток

\[
I=\frac{\mathcal{E}_{i}+\mathcal{E}_{s}}{R}=-\frac{1}{R} \frac{\mathrm{~d} \Phi}{\mathrm{~d} t} .
\]

Если сопротивление контура \(R=0\), то должно быть и \(\mathrm{d} \Phi / \mathrm{d} t=0\), поскольку сила тока \(I\) не может быть бесконечно большой. Отсюда следует, что \(\Phi=\) const.

Таким образом, при движении сверхпроводящего контура в магнитном поле пронизывающий его магнитный поток остается постоянным. Такое сохранение потока обеспечивают индукционные токи, которые согласно правилу Ленца препятствуют всякому изменению магнитного потока сквозь контур.

Тенденция к сохранению магнитного потока сквозь контур имеется в любом случае, но наиболее полно она проявляется в контурах из сверхпроводников.

Пример. Сверхпроводящее круглое кольцо радиусом \(a\) с индуктивностью \(L\) находится в однородном магнитном поле В. В начальном положении плоскость кольца параллельна вектору В и ток в кольце равен нулю. Кольцо повернули в положение, перпендикулярное вектору В. Найдем силу тока в кольце после поворота и магнитную индукцию в его центре.

При повороте кольца магнитный поток сквозь него не меняется и остается равным нулю. Это значит, что магнитные потоки через кольцо поля индукционного тока и внешнего поля одинаковы по модулю, но противоположны по знаку. Поэтому \(L I=\pi a^{2} B\), откуда

\[
I=\pi a^{2} B / L
\]

Этот ток создает в центре кольца согласно (6.13) поле \(B_{I}= =\pi \mu_{0} a B / 2 L\). Тогда результирующая магнитная индукция в этой точке

\[
B_{\mathrm{pe} 3}=B-B_{I}=B\left(1-\pi \mu_{0} a / 2 L\right) .
\]

\section*{§ 9.4. Взаимная индукция}
Взаимная индуктивность. Рассмотрим два неподвижных контура 1 и 2 (рис. 9.9), расположенные достаточно близко друг к другу. Если в контуре 1 течет ток \(I_{1}\), он создает через контур 2 полный магнитный поток \(\Phi_{2}\), пропорциональный (при отсутствии ферромагнетиков) току \(I_{1}\) :


\begin{equation*}
\Phi_{2}=L_{21} I_{1} . \tag{9.23}
\end{equation*}


\begin{figure}[h]
\begin{center}
  \includegraphics[alt={},max width=\textwidth]{3e870b25-6320-483d-9133-cdd5212c2707-240_340_380_219_778}
\captionsetup{labelformat=empty}
\caption{Рис. 9.9}
\end{center}
\end{figure}

Совершенно так же, если в контуре 2 течет ток \(I_{2}\), он создает через контур 1 полный магнитный поток


\begin{equation*}
\Phi_{1}=L_{12} I_{2} . \tag{9.24}
\end{equation*}


Коэффициенты пропорциональности \(L_{12}\) и \(L_{21}\) называют взаимной индуктивностью контуров. Очевидно, взаимная индуктивность численно равна магнитному потоку сквозь один из контуров, создаваемому единичным током в другом контуре. Коэффициенты \(L_{12}\) и \(L_{21}\) зависят от формы, размеров и взаимного расположения контуров, а также от магнитной проницаемости окружающей контуры среды. Выражаются эти коэффициенты в тех же единицах, что и индуктивность \(L\).

Теорема взаимности. Соответствующий расчет дает (и опыт его подтверждает), что при отсутствии ферромагнетиков коэффициенты \(L_{12}\) и \(L_{21}\) одинаковы:


\begin{equation*}
L_{12}=L_{21} . \tag{9.25}
\end{equation*}


Это замечательное свойство взаимной индуктивности принято называть теоремой взаимности. Благодаря этой теореме можно не делать различия между \(L_{12}\) и \(L_{21}\) и просто говорить о взаимной индуктивности двух контуров.

Смысл равенства (9.25) в том, что в любом случае магнитный поток \(\Phi_{1}\) сквозь контур 1 , созданный током \(I\) в контуре 2 , равен магнитному потоку \(\Phi_{2}\) сквозь контур 2 , созданному \(m a\) \(к и м\) же током \(I\) в контуре 1 . Это обстоятельство нередко позво-

ляет сильно упрощать решение вопроса о нахождении, например, магнитных потоков. Вот два примера.

Пример 1. В некоторой плоскости лежат два круговых витка 1 и 2 , центры которых совпадают (рис. 9.10). Радиусы витков \(a_{1}\) и \(a_{2}\). В витке 1 течет ток \(I\). Найдем магнитный поток \(\Phi_{2}\), охватываемый витком 2 , если \(a_{1} \ll a_{2}\).

\begin{figure}[h]
\begin{center}
  \includegraphics[alt={},max width=\textwidth]{3e870b25-6320-483d-9133-cdd5212c2707-241_108_374_435_242}
\captionsetup{labelformat=empty}
\caption{Рис. 9.10}
\end{center}
\end{figure}

Ясно, что непосредственно вычислить поток \(\Phi_{2}\) - задача весьма сложная, ибо сложной является конфигурация самого поля. Использование же теоремы взаимности чрезвычайно упрощает решение поставленного. вопроса. Действительно, пустим тот же ток \(I\) по витку 2 . Тогда магнитный поток \(\Phi_{1}\), создаваемый этим током через виток 1 , при условии \(a_{1} \ll a_{2}\) может быть найден очень просто: достаточно умножить магнитную индукцию \(B\) в центре витка ( \(B=\mu_{0} I / 2 a_{2}\) ) на площадь круга \(\pi a_{1}^{2}\) и учесть, что согласно теореме взаимности \(\Phi_{2}=\Phi_{1}\).\\
Пример 2. Пусть контур с током \(I\) имеет форму прямоугольника. Как

\begin{figure}[h]
\begin{center}
  \includegraphics[alt={},max width=\textwidth]{3e870b25-6320-483d-9133-cdd5212c2707-241_341_305_964_242}
\captionsetup{labelformat=empty}
\caption{Рис. 9.11}
\end{center}
\end{figure}

найти магнитный поток Ф через заштрихованную полуплоскость (рис. 9.11), граница которой находится на заданном расстоянии от контура? Предполагается, что эта полуплоскость и контур лежат в одной плоскости.

Магнитное поле тока \(I\) здесь также имеет сложную конфигурацию, поэтому непосредственно вычислить интересующий нас поток Ф очень трудно. Однако решение и здесь можно весьма резко упростить, если воспользоваться теоремой взаимности.

Представим себе, что ток \(I\) течет не по прямоугольному контуру, а вдоль границы полуплоскости, огибая ее на бесконечности. Магнитное поле, создаваемое этим током в области прямоугольного контура, имеет простую конфигурацию - это поле прямого тока. Поэтому найти магнитный поток \(\Phi^{\prime}\) сквозь прямоугольный контур достаточно легко (путем несложного интегрирования). А по теореме взаимности искомый поток \(\Phi=\Phi^{\prime}\), и задача решена.

Однако наличие ферромагнетиков меняет дело, и теорема взаимности перестает выполняться. Убедимся в этом на следующем конкретном примере.

Пример. Длинный ферромагнитный цилиндр объемом \(V\) имеет две обмотки (одна на другой). Одна обмотка содержит \(n_{1}\) витков на единицу длины, другая \(-n_{2}\). Найдем их взаимную индуктивность, пренебрегая краевыми эффектами.\\
Согласно (9.23) \(L_{21}=\Phi_{2} / I_{1}\). Это значит, что мы должны создать ток \(I_{1}\) в обмотке 1 и вычислить полный магнитный поток через все витки обмотки 2 . Если в обмотке 2 содержится \(N_{2}\) витков, то

\[
\Phi_{2}=N_{2} B_{1} S,
\]

где \(S\) - площадь сечения цилиндра. Имея в виду, что \(N_{2}=n_{2} l, l\) - длина цилиндра, \(B_{1}=\mu_{1} \mu_{0} n_{1} I_{1}, \mu_{1}\) - магнитная проницаемость при токе \(I_{1}\), запишем: \(\Phi_{2}=\mu_{1} \mu_{0} n_{1} n_{2} V I_{1}\), \(V=l S\). Отсюда

\[
L_{21}=\mu_{1} \mu_{0} n_{1} n_{2} V .
\]

Аналогично находим и \(L_{12}\) :

\[
L_{12}=\mu_{2} \mu_{0} n_{1} n_{2} V .
\]

Ввиду того что значения \(\mu_{1}\) и \(\mu_{2}\) в последних двух выражениях, вообще говоря, разные (в ферромагнетиках они зависят от токов \(I_{1}\) и \(I_{2}\) ), значения \(L_{21}\) и \(L_{12}\) не совпадают.

Взаимная индукция. Наличие магнитной связи между контурами проявляется в том, что при всяком изменении тока в одном из контуров в другом контуре возникает э.д.с. индукции. Это явление и называют взаимной индукцией.

Согласно закону электромагнитной индукции э.д.с., возникающие в контурах 1 и 2 , равны соответственно:


\begin{equation*}
\mathcal{E}_{1}=-\frac{\mathrm{d} \Phi_{1}}{\mathrm{~d} t}=-L_{12} \frac{\mathrm{~d} I_{2}}{\mathrm{~d} t}, \quad \mathcal{E}_{2}=-\frac{\mathrm{d} \Phi_{2}}{\mathrm{~d} t}=-L_{21} \frac{\mathrm{~d} I_{1}}{\mathrm{~d} t} . \tag{9.26}
\end{equation*}


Здесь предполагается, что контуры неподвижны и ферромагнетиков поблизости нет.

С учетом явления самоиндукции ток, например, в контуре 1 при изменении токов в обоих контурах определяется по закону

Ома как

\[
R_{1} I_{1}=\mathcal{E}_{1}-L_{1} \frac{\mathrm{~d} I_{1}}{\mathrm{~d} t}-L_{12} \frac{\mathrm{~d} I_{2}}{\mathrm{~d} t},
\]

где \(\mathbf{E}_{1}\) - сторонняя э.д.с. в контуре 1 (помимо индукционных э.д.с.), \(L_{1}\) - индуктивность контура 1. Аналогичное уравнение можно записать и для определения силы тока \(I_{2}\) в контуре 2.

Отметим, что на явлении взаимной индукции основано действие трансформаторов - устройств, служащих для преобразования токов и напряжений.

Замечание о знаке \(L_{12}\). В отличие от индуктивности \(L\), которая, как было сказано, является существенно положительной величиной, взаимная индуктивность \(L_{12}\) - величина алгебраическая (в частности, равная нулю). Это связано с тем обстоятельством, что, например в (9.23), величины \(\Phi_{2}\) и \(I_{1}\) относятся к разньлм контурам. Из рис. 9.9 сразу видно, что знак магнитного потока \(\Phi_{2}\) при данном направлении тока \(I_{1}\) будет зависеть от выбора нормали к поверхности, ограниченной контуром 2 (или от выбора положительного направления обхода этого контура).

Положительные направления для токов (и э.д.с.) в обоих контурах всегда можно выбрать произвольно (а с положительным направлением обхода контура однозначно - правилом правого винта - связано направление нормали \(\mathbf{n}\) к поверхности, ограниченной контуром, т. е. в конечном счете знак магнитного потока). Раз эти направления выбраны, величину \(L_{12}\) мы должны считать положительной, когда при положительных токах магнитные потоки взаимной индукции через контуры оказываются также положительными, т. е. совпадают по знаку с потоками самоиндукции.

\begin{figure}[h]
\begin{center}
  \includegraphics[alt={},max width=\textwidth]{3e870b25-6320-483d-9133-cdd5212c2707-243_245_287_1444_66}
\captionsetup{labelformat=empty}
\caption{Рис. 9.12}
\end{center}
\end{figure}

Другими словами, \(L_{12}>0\), если при положительных токах в обоих контурах они «подмагничивают» друг друга, в противном случае \(L_{12}<0\). В частных случаях можно заранее так установить положительные направления обхода контуров, чтобы получить желательный нам знак величины \(L_{12}\) (рис. 9.12).

\section*{§ 9.5. Энергия магнитного поля}
Магнитная энергия тока. Замкнем неподвижную цепь, содержащую индуктивность \(L\) и сопротивление \(R\), на источник тока с э.д.с. \(\mathcal{E}_{0}\). В контуре, как мы уже знаем, начнет возрастать ток. Это приводит к появлению э.д.с. самоиндукции \(\mathcal{E}_{s}\). Согласно закону Ома \(R I=\mathcal{E}_{0}+\mathcal{E}_{s}\), откуда

\[
\mathcal{E}_{0}=R I-\mathcal{E}_{s} .
\]

Найдем элементарную работу, которую совершают сторонние силы (т. е. источник \(\mathcal{E}_{0}\) ) за время \(\mathrm{d} t\). Для этого умножим предыдущее равенство на \(I \mathrm{~d} t\) :

\[
\mathcal{E}_{0} I \mathrm{~d} t=R I^{2} \mathrm{~d} t-\mathcal{E}_{s} I \mathrm{~d} t .
\]

Учитывая смысл каждого слагаемого и соотношение \(\mathcal{E}_{s}=-\mathrm{d} \Phi / \mathrm{d} t\), запишем

\[
\delta A_{\text {стор }}=\delta Q+I \mathrm{~d} \Phi .
\]

Мы видим, что в процессе установления тока, когда поток Ф меняется и \(\mathrm{d} \Phi>0\) (если \(I>0\) ), работа, которую совершает источник \(\mathcal{E}_{0}\), оказывается больше выделяемой в цепи джоулевой теплоты. Часть этой работы (дополнительная работа) совершается против э.д.с. самоиндукции. Заметим, что после того как ток установится, \(\mathrm{d} \Phi=0\) и вся работа источника \(\mathcal{E}_{0}\) будет идти только на выделение джоулевой теплоты.

Итак, дополнительная работа, совершаемая сторонними силами против э.д.с. самоиндукции в процессе установления тока:


\begin{equation*}
\delta A^{\text {доп }}=I \mathrm{~d} \Phi . \tag{9.27}
\end{equation*}


Это соотношение имеет общий характер. Оно справедливо и при наличии ферромагнетиков, так как при его выводе не вводилось никаких предположений относительно магнитных свойств окружающей среды.

Теперь (и далее) будем считать, что ферромагнетики отсутствуют. Тогда \(\mathrm{d} \Phi=L \mathrm{~d} I\) и


\begin{equation*}
\delta A^{\text {доп }}=L I \mathrm{~d} I . \tag{9.28}
\end{equation*}


Проинтегрировав это уравнение, получим \(A^{\text {доп }}=L I^{2} / 2\). По закону сохранения энергии любая работа идет на приращение ка-кого-то вида энергии. Мы видим, что часть работы сторонних сил ( \(\mathrm{E}_{0}\) ) идет на увеличение внутренней энергии проводников (с ней связано выделение джоулевой теплоты) и другая часть - в процессе установления тока - на что-то еще. Это «что-то» есть не что иное, как магнитное поле, именно его появление и связано с появлением тока.

Таким образом, мы приходим к выводу, что при отсутствии ферромагнетиков контур с индуктивностью \(L\), по которому течет ток \(I\), обладает энергией


\begin{equation*}
W=\frac{1}{2} L I^{2}=\frac{1}{2} I \Phi=\frac{\Phi^{2}}{2 L} \tag{9.29}
\end{equation*}


Эту энергию называют магнитной энергией тока или собственной энергией тока. Она может быть целиком превращена во внутреннюю энергию проводников, если отключить источник \(\mathbf{E}_{0}\) так, как показано на рис. 9.7: быстро повернуть ключ \(K\) из положения б в положение \(a\).

Энергия магнитного поля. Формула (9.29) выражает магнитную энергию тока через индуктивность и ток (при отсутствии ферромагнетиков). Однако и здесь, как и в случае электрической энергии заряженных тел, энергию можно выразить непосредственно через магнитную индукцию В. Убедимся, что это так сначала на простейшем примере длинного соленоида, пренебрегая искажением поля на его торцах (краевыми эффектами). Подстановка в формулу (9.29) выражения \(L=\mu \mu_{0} n^{2} V\) дает

\[
W=L I^{2} / 2=\mu \mu_{0} n^{2} I^{2} V / 2
\]

А так как \(n I=H=B / \mu \mu_{0}\), то


\begin{equation*}
W=\frac{B^{2}}{2 \mu \mu_{0}} V=\frac{\mathbf{B H}}{2} V \tag{9.30}
\end{equation*}


Эта формула справедлива для однородного поля, заполняющего объем \(V\) (как в нашем случае с соленоидом).

В общей теории показывается, что энергию \(W\) можно выразить через векторы В и Н в любом случае (но при отсутствии

ферромагнетиков) по формуле


\begin{equation*}
W=\int \frac{\mathbf{B H}}{2} \mathrm{~d} V . \tag{9.31}
\end{equation*}


Подынтегральное выражение в этом уравнении имеет смысл энергии, заключенной в элементе объемом \(\mathrm{d} V\).

Отсюда, как и в случае электрического поля, мы приходим к выводу, что магнитная энергия также локализована в пространстве, занимаемом магнитным полем.

Из формул (9.30) и (9.31) следует, что магнитная энергия распределена в пространстве с объемной плотностью


\begin{equation*}
w=\frac{\mathbf{B H}}{2}=\frac{B^{2}}{2 \mu \mu_{0}} . \tag{9.32}
\end{equation*}


Отметим, что полученное выражение относится лишь к тем средам, для которых зависимость В от Н линейная, т. е. \(\mu\) в соотношении \(\mathbf{B}=\mu \mu_{0} \mathbf{H}\) не зависит от \(\mathbf{H}\). Другими словами, выражения (9.31) и (9.32) относятся только к пара- и диамагнетикам. К ферромагнетикам они не применимы*.

Отметим также, что магнитная энергия - величина существенно положительная. Это легко усмотреть из последних двух формул.

Еще об обосновании формулы (9.32). Убедимся в справедливости этой формулы, рассуждая в «обратном» порядке, а именно покажем. что если формула (9.32) справедлива, то магнитная энергия контура с током \(W=L I^{2} / 2\).

С этой целью рассмотрим магнитное поле произвольного контура с током \(I\) (рис. 9.13). Представим все поле разделенным на элементарные трубки, образующие которых являются линиями вектора В. Выделим в одной из таких трубок элементарный объем \(\mathrm{d} V=\mathrm{d} l \mathrm{~d} S\). В соответствии с формулой (9.32) в этом объеме локализована энергия \((B H / 2) \mathrm{d} l \mathrm{~d} S\).

\begin{figure}[h]
\begin{center}
  \includegraphics[alt={},max width=\textwidth]{3e870b25-6320-483d-9133-cdd5212c2707-246_255_384_1296_776}
\captionsetup{labelformat=empty}
\caption{Рис. 9.13}
\end{center}
\end{figure}

\footnotetext{\begin{itemize}
  \item Это обусловлено тем, что в конечном счете выражения (9.31) и (9.32) являются следствиями формулы \(\delta A^{\text {доп }}=I \mathrm{~d} \Phi\) и того факта, что при отсутствии гистерезиса работа \(\delta A^{\text {доп }}\) идет только на приращение магнитной энергии \(\mathrm{d} W\). Для ферромагнитной среды дело обстоит иначе: работа \(\delta A^{\text {доп }}\) идет еще и на приращение внутренней энергии среды, т. е. на ее нагревание.
\end{itemize}
}Теперь найдем энергию \(d W\) в объеме всей элементарной трубки. Для этого проинтегрируем последнее выражение вдоль оси трубки. Поток \(\mathrm{d} \Phi=B \mathrm{~d} S\) сквозь сечение трубки постоянен вдоль всей трубки, поэтому dФ можно вынести за знак интеграла:

\[
\mathrm{d} W=\frac{\mathrm{d} \Phi}{2} \oint H \mathrm{~d} l=I \frac{\mathrm{~d} \Phi}{2},
\]

где использована теорема о циркуляции вектора \(\mathbf{H}\) (в нашем случае проекция \(H_{l}=H\) ).

И наконец, просуммируем энергию всех элементарных трубок:

\[
W=1 / 2 I \int \mathrm{~d} \Phi=I \Phi / 2=L I^{2} / 2,
\]

где Ф - полный магнитный поток, охватываемый контуром с током, \(\Phi=L I\). Это и требовалось показать.

Определение индуктивности из выражения энергии. Мы ввели индуктивность \(L\) как коэффициент пропорциональности между полным магнитным потоком \(\Phi\) и током \(I\). Существует, однако, и другая возможность расчета \(L\) - из выражения энергии. В самом деле, из сопоставления формул (9.31) и (9.29) следует, что при отсутствии ферромагнетика


\begin{equation*}
L=\frac{1}{I^{2}} \int \frac{B^{2}}{\mu \mu_{0}} \mathrm{~d} V \tag{9.33}
\end{equation*}


Нахождение \(L\) таким путем свободно от неопределенности, связанной с вычислением магнитного потока Ф в формуле (9.14) - см. с. 235. К каким расхождениям иногда приводит определение \(L\) по формуле (9.33) и из выражения потока (9.14), показано в задаче 9.9 на примере коаксиального кабеля.

\section*{§ 9.6. Магнитная энергия двух контуров с токами}
Собственная и взаимная энергии. Возьмем два неподвижных контура 1 и 2, расположив их достаточно близко друг к другу (чтобы была магнитная связь между ними). Предполагается, что в каждом контуре есть свой источник постоянной э.д.с. Замкнем в момент \(t=0\) каждый из контуров. Как в том, так и в другом контуре начнет устанавливаться свой ток и, следовательно, появятся э.д.с. самоиндукции \(\mathcal{E}_{s}\) и э.д.с. взаимной индукции \(\mathcal{E}_{i}\). Дополнительная работа, совершаемая при этом

источниками постоянной э.д.с. против \(\mathcal{E}_{s}\) и \(\mathcal{E}_{i}\), идет, как мы уже знаем, на создание магнитной энергии.

Найдем эту работу за время \(\mathrm{d} t\) :

\[
\delta A^{\text {доп }}=-\left(\mathcal{E}_{s 1}+\mathcal{E}_{i 1}\right) I_{1} \mathrm{~d} t-\left(\mathcal{E}_{s 2}+\mathcal{E}_{i 2}\right) I_{2} \mathrm{~d} t=\mathrm{d} W .
\]

Преобразуем эту формулу, учитывая, что \(\mathcal{E}_{s 1}=-L_{1} \mathrm{~d} I_{1} / \mathrm{d} t\), \(\mathcal{E}_{i 1}=-L_{12} \mathrm{~d} I_{2} / \mathrm{d} t\) и т. д.:

\[
\mathrm{d} W=L_{1} I_{1} \mathrm{~d} I_{1}+L_{12} I_{1} \mathrm{~d} I_{2}+L_{2} I_{2} \mathrm{~d} I_{2}+L_{21} I_{2} \mathrm{~d} I_{1} .
\]

Имея в виду, что \(L_{12}=L_{21}\), представим последнее уравнение в виде

\[
\mathrm{d} W=\mathrm{d}\left(L_{1} I_{1}^{2} / 2\right)+\mathrm{d}\left(L_{2} I_{2}^{2} / 2\right)+\mathrm{d}\left(L_{12} I_{1} I_{2}\right),
\]

откуда


\begin{equation*}
W=\frac{L_{1} I_{1}^{2}}{2}+\frac{L_{2} I_{2}^{2}}{2}+L_{12} I_{1} I_{2} \tag{9.34}
\end{equation*}


Здесь первые два слагаемых называют собственной энергией тока \(I_{1}\) и тока \(I_{2}\), последнее слагаемое - взаимной энергией обоих токов. Взаимная энергия токов - величина алгебраическая в отличие от собственных энергий токов. Изменение направления одного из токов приводит к изменению знака взаимной энергии - последнего слагаемого в (9.34).

Пример. Имеются два концентрических контура с токами \(I_{1}\) и \(I_{2}\), направления которых показаны на рис. 9.14. Взаимная энергия этих токов ( \(W_{12}=L_{12} I_{1} I_{2}\) ) зависит от трех алгебраических величин, знаки которых

\begin{figure}[h]
\begin{center}
  \includegraphics[alt={},max width=\textwidth]{3e870b25-6320-483d-9133-cdd5212c2707-248_128_281_1286_879}
\captionsetup{labelformat=empty}
\caption{Рис. 9.14}
\end{center}
\end{figure}

определяются выбором положительных направлений обхода обоих контуров. Полезно, однако, убедиться в том, что знак величины \(W_{12}\) (в данном случае \(W_{12}>0\) ) определяется только взаимным направлением самих токов и совершенно не зависит от выбора положительных направлений обхода контуров. Напомним, что о знаке величины \(L_{12}\) говорилось в § 9.4.

Полевая трактовка энергии (9.34). Есть несколько важных вопросов, которые мы сможем решить, вычислив магнитную энергию двух контуров еще и иначе - с точки зрения локализации энергии в поле.

Пусть \(\mathbf{B}_{1}\) - магнитное поле тока \(I_{1}\), а \(\mathbf{B}_{2}\) - поле тока \(I_{2}\). Тогда по принципу суперпозиции поле в каждой точке \(\mathbf{B}=\mathbf{B}_{1}+\mathbf{B}_{2}\) и согласно (9.31) энергия магнитного поля этой системы токов \(W=\int\left(B^{2} / 2 \mu \mu_{0}\right) \mathrm{d} V\). Подставив сюда \(B^{2}=B_{1}^{2}+B_{2}^{2}+2 \mathbf{B}_{1} \mathbf{B}_{2}\), получим


\begin{equation*}
W=\int \frac{B_{1}^{2}}{2 \mu \mu_{0}} \mathrm{~d} V+\int \frac{B_{2}^{2}}{2 \mu \mu_{0}} \mathrm{~d} V+\int \frac{\mathbf{B}_{1} \mathbf{B}_{2}}{\mu \mu_{0}} \mathrm{~d} V \tag{9.35}
\end{equation*}


Соответствие друг другу отдельных слагаемых в формулах (9.35) и (9.34) не вызывает сомнения.

Формулы (9.34) и (9.35) приводят к таким важным следствиям.

\begin{enumerate}
  \item Магнитная энергия системы двух (и более) токов - величина всегда положительная, \(W>0\). Это вытекает из того факта, что \(W\) c \(\int B^{2} \mathrm{~d} V\), где под интегралом стоят положительные величины.
  \item Энергия токов - величина не аддитивная (из-за наличия взаимной энергии).
  \item Последний интеграл в (9.35) пропорционален произведению токов \(I_{1} I_{2}\), так как \(B_{1}\) つ \(I_{1}\) и \(B_{2}\) つ \(I_{2}\). Коэффициент же пропорциональности (т. е. оставшийся интеграл) оказывается симметричным относительно индексов 1 и 2 , а поэтому его можно обозначить \(L_{12}\) или \(L_{21}\) [в соответствии с формулой (9.34)]. Таким образом, действительно, \(L_{12}=L_{21}\).
  \item Из выражения (9.35) вытекает другое определение взаимной индуктивности \(L_{12}\). В самом деле, сопоставление выражений (9.35) и (9.34) показывает, что
\end{enumerate}


\begin{equation*}
L_{12}=\frac{1}{I_{1} I_{2}} \int \frac{\mathbf{B}_{1} \mathbf{B}_{2}}{\mu \mu_{0}} \mathrm{~d} V \tag{9.36}
\end{equation*}


\section*{§ 9.7. Энергия и силы в магнитном поле}
Наиболее общим методом определения сил в магнитном поле является энергетический. В этом методе используют выражение для энергии магнитного поля.

Ограничимся случаем, когда система состоит из двух контуров с токами \(I_{1}\) и \(I_{2}\). Магнитная энергия такой системы может быть представлена в виде


\begin{equation*}
W=\left(I_{1} \Phi_{1}+I_{2} \Phi_{2}\right) / 2 \tag{9.37}
\end{equation*}


где \(\Phi_{1}\) и \(\Phi_{2}\) - полные магнитные потоки, пронизывающие контуры 1 и 2 соответственно. Это выражение нетрудно получить из формулы (9.34), если представить последнее слагаемое как сумму \(L_{12} I_{1} I_{2} / 2+L_{21} I_{2} I_{1} / 2\), а затем учесть, что


\begin{equation*}
\Phi_{1}=L_{1} I_{1}+L_{12} I_{2}, \quad \Phi_{2}=L_{2} I_{2}+L_{21} I_{1} . \tag{9.38}
\end{equation*}


Согласно закону сохранения энергии работа \(\delta A^{*}\), которую совершают источники тока, включенные в контуры 1 и 2 , идет на теплоту \(\delta Q\), на приращение магнитной энергии системы \(\mathrm{d} W\) (из-за движения контуров или изменения токов в них) и на механическую работу \(\delta A_{\text {мех }}\) (вследствие перемещения или деформации контуров):


\begin{equation*}
\delta A^{*}=\delta Q+\mathrm{d} W+\delta A_{\mathrm{mex}} . \tag{9.39}
\end{equation*}


Мы предположили, что емкость контуров пренебрежимо мала, и поэтому электрическую энергию учитывать не будем.

В дальнейшем нас будет интересовать не вся работа источника тока \(\delta A^{*}\), а только та ее часть, которая совершается против э.д.с. индукции и самоиндукции (в каждом контуре). Эта работа (мы назвали ее дополнительной) \(\delta A^{\text {доп }}=-\left(\mathcal{E}_{i 1}+\mathcal{E}_{s 1}\right) I_{1} \mathrm{~d} t- -\left(\mathcal{E}_{i 2}+\mathcal{E}_{s 2}\right) I_{2} \mathrm{~d} t\). Учитывая, что для каждого контура \(\mathcal{E}_{i}+\mathcal{E}_{s}= =-\mathrm{d} \Phi / \mathrm{d} t\), перепишем выражение для дополнительной работы в виде


\begin{equation*}
\delta A^{\text {доп }}=I_{1} \mathrm{~d} \Phi_{1}+I_{2} \mathrm{~d} \Phi_{2} . \tag{9.40}
\end{equation*}


Именно эта часть работы источников тока (работа против э.д.с. индукции и самоиндукции), связанная с изменением по-

токов \(\Phi_{1}\) и \(\Phi_{2}\), и идет на приращение магнитной энергии системы и на механическую работу:


\begin{equation*}
I_{1} \mathrm{~d} \Phi_{1}+I_{2} \mathrm{~d} \Phi_{2}=\mathrm{d} W+\mathrm{d} A_{\mathrm{mex}} \tag{9.41}
\end{equation*}


Эта формула является основной для расчета механической работы \(\mathrm{d} A_{\text {мех }}\), а из нее и сил в магнитном поле.

Из формулы (9.41) можно получить и более простые выражения для \(\mathrm{d} A_{\text {мех }}\), если считать, что в процессе перемещения остаются неизменными или все магнитные потоки сквозь контуры, или токи в них. Рассмотрим это более подробно.

\begin{enumerate}
  \item Если потоки постоянны, \(\Phi_{k}=\) const, где \(k=1\) и 2 , то из (9.41) сразу следует, что
\end{enumerate}


\begin{equation*}
\delta A_{\mathrm{mex}}=-\left.\mathrm{d} W\right|_{\Phi}, \tag{9.42}
\end{equation*}


где символ Ф подчеркивает, что приращение магнитной энергии системы должно быть вычислено при постоянных потоках через контуры. Полученная формула аналогична соответствующей ей (4.15) для работы в электрическом поле.\\
2. Если токи постоянны, \(I_{k}=\) const, то


\begin{equation*}
\delta A_{\mathrm{mex}}=\left.\mathrm{d} W\right|_{I} \tag{9.43}
\end{equation*}


Действительно, при \(I_{k}=\) const из формулы (9.37) следует, что

\[
\left.\mathrm{d} W\right|_{I}=\left(I_{1} \mathrm{~d} \Phi_{1}+I_{2} \mathrm{~d} \Phi_{2}\right) / 2
\]

т.е. в этом случае приращение магнитной энергии системы равно согласно (9.40) половине дополнительной работы источников э.д.с. Другая половина этой работы идет на совершение механической работы. Иначе говоря, при постоянстве токов \(\left.\mathrm{d} W\right|_{I}=\mathrm{d} A_{\text {мех }}\), что и требовалось показать.

Необходимо подчеркнуть, что оба полученные нами выражения (9.42) и (9.43) определяют механическую работу одной и той же силы, т. е. можно написать:


\begin{equation*}
\mathbf{F} \mathrm{d} \mathbf{l}=-\left.\mathrm{d} W\right|_{\Phi}=\left.\mathrm{d} W\right|_{I} . \tag{9.44}
\end{equation*}


Для вычисления силы с помощью этих формул, конечно, нет необходимости подбирать такой режим, при котором обязатель-

но оставались бы постоянными или магнитные потоки, или токи. Надо просто найти приращение \(\mathrm{d} W\) магнитной энергии системы при условии, что либо \(\Phi_{k}=\) const, либо \(I_{k}=\) const, а это является чисто математической операцией.

Ценность полученных выражений (9.42) и (9.43) в их общности: они пригодны для системы, состоящей из любого числа контуров - одного, двух и т. д.

Рассмотрим несколько примеров на применение этих формул.

Пример 1. Сила в случае одного контура с током. Имеется контур с током, у которого \(A B\) - подвижная перемычка (рис. 9.15). Индуктивность этого контура зависит определенным образом от координаты \(x\), т. е. известно \(L(x)\). Найдем силу Ампера, дей-

\begin{figure}[h]
\begin{center}
  \includegraphics[alt={},max width=\textwidth]{3e870b25-6320-483d-9133-cdd5212c2707-252_239_352_548_810}
\captionsetup{labelformat=empty}
\caption{Рис. 9.15}
\end{center}
\end{figure}

ствующую на перемычку, двумя способами: при \(I=\) const и при \(\Phi=\) const.

В нашем случае магнитную энергию системы можно представить согласно (9.29):

\[
W=L l^{2} / 2=\Phi^{2} / 2 L,
\]

где \(\Phi=L I\). Переместим перемычку, например, вправо на \(\mathrm{d} x\). Так как \(\mathrm{d} A_{\text {мех }}=F_{x} \mathrm{~d} x\), то

\[
F_{x}=\left.\frac{\partial W}{\partial x}\right|_{I}=\frac{I^{2}}{2} \frac{\partial L}{\partial x},
\]

или

\[
F_{x}=-\left.\frac{\partial W}{\partial x}\right|_{\Phi}=\frac{\Phi^{2}}{2 L} \frac{\partial L}{\partial x}=\frac{I^{2}}{2} \frac{\partial L}{\partial x},
\]

т. е. расчет по обеим формулам согласно (9.44) дает один и тот же результат.\\
Пример 2. Взаимодействие двух катушек с токами. На немагнитный сердечник (рис. 9.16) надеты катушки 1 и 2 с токами \(I_{1}\) и \(I_{2}\). Пусть взаимная индуктивность катушек зависит

\begin{figure}[h]
\begin{center}
  \includegraphics[alt={},max width=\textwidth]{3e870b25-6320-483d-9133-cdd5212c2707-252_170_432_1593_732}
\captionsetup{labelformat=empty}
\caption{Рис. 9.16}
\end{center}
\end{figure}

от расстояния \(x\) между ними по известному закону \(L_{12}(x)\). Найдем силу взаимодействия между катушками.

Магнитная энергия системы из двух катушек дается формулой (9.34). Для определения силы взаимодействия будем пользоваться выражением (9.43). Сместим катушку 2 на расстояние \(\mathrm{d} x\) при неизменных токах \(I_{1}\) и \(I_{2}\). Соответствующее приращение магнитной энергии системы

\[
\left.\mathrm{d} W\right|_{I}=I_{1} I_{2} \mathrm{~d} L_{12}(x) .
\]

Так как элементарная механическая работа \(\delta A_{\text {мех }}=F_{2 x} \mathrm{~d} x\), то согласно (9.43) получим

\[
F_{2 x}=I_{1} I_{2} \frac{\partial L_{12}(x)}{\partial x}
\]

Пусть токи \(I_{1}\) и \(I_{2}\) подмагничивают друг друга, тогда \(L_{12}>0\) и при \(\mathrm{d} x>0\) приращение \(\mathrm{d} L_{12}<0\), т. е. \(F_{2 x}<0\). Следовательно, сила, действующая на катушку 2 со стороны катушки 1 , является силой притяжения: вектор \(\mathbf{F}_{2}\) направлен влево на рисунке.

Пример 3. Магнитное давление на обмотку соленоида. Увеличим мысленно радиус сечения соленоида на \(\mathrm{d} r\), сохраняя при этом неизменным ток \(I\) через обмотку. Тогда силы Ампера совершат работу \(\delta A_{\text {мех }}=\left.\mathrm{d} W\right|_{I}\). В нашем случае

\[
\delta A_{\mathrm{mex}}=p S \mathrm{~d} r
\]

где \(p\) - искомое давление, \(S\) - боковая поверхность соленоида,

\[
\left.\mathrm{d} W\right|_{I}=\mathrm{d}\left(\frac{B^{2}}{2 \mu_{0}} V\right)=\frac{B^{2}}{2 \mu_{0}} S \mathrm{~d} r .
\]

Здесь учтено, что при \(I\) = const и \(B\) = const. Из равенства двух этих выражений находим

\[
p=B^{2} / 2 \mu_{0} .
\]

Магнитное давление. Полученное в последнем примере выражение для давления можно обобщить на случай, когда по разные стороны от поверхности с током (током проводимости или током намагничивания) магнитное поле разное \(-\mathbf{B}_{1}\) и \(\mathbf{B}_{2}\). В этом случае, оказывается, магнитное давление


\begin{equation*}
p=\left|\frac{\mathbf{B}_{1} \mathbf{H}_{1}}{2}-\frac{\mathbf{B}_{2} \mathbf{H}_{2}}{2}\right|, \tag{9.45}
\end{equation*}


причем дело обстоит так, как если бы область с большей плотностью магнитной энергии была бы областью большего давления.

Соотношение (9.45) является одним из основных в магнитогидродинамике, изучающей поведение электропроводящих жидкостей (в электротехнике и астрофизике).

\section*{Задачи}
9.1. Э.д.с. индукции. Провод, имеющий форму параболы \(y=k x^{2}\), находится в однородном магнитном поле В, перпендикулярном плоскости \(X Y\). Из вершины параболы перемещают поступательно и без начальной скорости перемычку с постоянным ускорением \(a\) (рис. 9.17). Найти э.д.с. индукции в образовавшемся контуре как функцию координаты \(y\).

Решение. По определению \(\mathcal{E}_{i}=-\mathrm{d} \Phi / \mathrm{d} t\). Вы-

\begin{figure}[h]
\begin{center}
  \includegraphics[alt={},max width=\textwidth]{3e870b25-6320-483d-9133-cdd5212c2707-254_311_277_506_883}
\captionsetup{labelformat=empty}
\caption{Рис. 9.17}
\end{center}
\end{figure}

\[
\mathcal{E}_{i}=-B \cdot 2 \sqrt{y / k} \mathrm{~d} y / \mathrm{d} t .
\]

При движении с постоянным ускорением скорость \(d y / d t=\sqrt{2 a y}\), поэтому

\[
\mathcal{E}_{i}=-B y \sqrt{8 a / k} .
\]

Из полученной формулы видно, что \(\mathcal{E}_{i} v y\). Знак минус показывает, что \(\mathcal{E}_{i}\) на рисунке действует против часовой стрелки.\\
9.2. Контур движется произвольным образом. Замкнутый проводящий контур перемещают произвольным образом (при этом даже деформируя) в постоянном неоднородном магнитном поле. Показать, что закон электромагнитной индукции (9.1) будет выполняться и в этом случае.\\
Решение. Рассмотрим элемент контура dl, который в данный момент движется со скоростью \(\mathbf{v}\) в магнитном поле В. Согласно формулам преобразования полей (8.4) в системе отсчета, связанной с данным элементом, будет наблюдаться электрическое поле \(\mathbf{E}=[\mathbf{v B}]\). Заметим, что это выражение можно получить и с помощью силы Лоренца, как было сделано в основном тексте перед формулой (9.4).

Циркуляция вектора \(\mathbf{E}\) по всему контуру по определению есть э.д.с. индукции:


\begin{equation*}
\mathcal{E}_{i}=\oint[\mathbf{v B}] \mathrm{d} \mathbf{l} . \tag{1}
\end{equation*}


Теперь найдем соответствующее приращение магнитного потока сквозь контур. С этой целью обратимся

\begin{figure}[h]
\begin{center}
  \includegraphics[alt={},max width=\textwidth]{3e870b25-6320-483d-9133-cdd5212c2707-255_225_394_375_145}
\captionsetup{labelformat=empty}
\caption{Рис. 9.18}
\end{center}
\end{figure}

к рис. 9.18. Пусть за время \(\mathrm{d} t\) наш контур переместился из положения \(\Gamma_{1}\) в положение \(\Gamma_{2}\). Если в первом положении магнитный поток через поверхность \(S_{1}\), натянутую на контур, был равен \(\Phi_{1}\), то соответствующий магнитный поток во втором положении контура может быть представлен как \(\Phi_{1}+\delta Ф\), т. е. как поток через поверхность \(S_{1}+\delta S\). Здесь \(\delta Ф\) - интересующее нас приращение магнитного потока сквозь узкую полоску \(\delta S\), ограниченную контурами \(\Gamma_{1}\) и \(\Gamma_{2}\).\\
С помощью рис. 9.18 запишем


\begin{equation*}
\mathrm{d} \Phi=\int \mathbf{B} \mathrm{d} \mathbf{S}=\int \mathbf{B}[\mathrm{d} \mathbf{r}, \mathrm{~d} \mathbf{l}]=-\oint[\mathrm{d} \mathbf{r}, \mathbf{B}] \mathrm{d} \mathbf{l} \tag{2}
\end{equation*}


Здесь: 1) направление нормали \(\mathbf{n}\) согласовано с направлением обхода контура - вектором dl (правовинтовая система); 2) направление вектора dS - элемента площади полоски - согласовано с выбором нормалей \(\mathbf{n} ; 3)\) использована циклическая перестановка в смешанном произведении:

\[
\mathbf{a}[\mathbf{b c}]=\mathbf{b}[\mathbf{c a}]=\mathbf{c}[\mathbf{a b}]=-[\mathbf{b a}] \mathbf{c} .
\]

Разделив выражение (2) на \(\mathrm{d} t\), найдем


\begin{equation*}
\mathrm{d} \Phi / \mathrm{d} t=-\oint[\mathbf{v B}] \mathrm{d} \mathbf{l}, \tag{3}
\end{equation*}


где \(\mathbf{v}=\mathrm{d} \mathbf{r} / \mathrm{d} t\). Остается сравнить (3) с (1), откуда и следует, что \(\mathrm{E}_{i}=-\mathrm{d} \Phi / \mathrm{d} t\).\\
9.3. Плоская спираль с большим числом \(N\) витков, плотно прилегающих друг к другу, находится в однородном магнитном поле, перпендикулярном плоскости спирали (рис. 9.19). Наружный радиус витков спирали равен \(a\). Магнитное поле изменяется во времени по закону \(B=B_{0} \sin \omega t\). Найти амплитудное значение э.д.с. индукции, наведенной в спирали.

Решение. Ввиду того что каждый виток спирали практически не отличается от окружности, в нем наводится э.д.с. индукции

\[
\varepsilon_{i}=-\mathrm{d} \Phi / \mathrm{d} t=-\pi r^{2} B_{0} \omega \cos \omega t,
\]

где \(r\) - радиус рассматриваемого витка. На интервал значений радиуса \(\mathrm{d} r\) приходится число витков \(\mathrm{d} N=(N / a) \mathrm{d} r\). Витки соединены последовательно, поэтому полная э.д.с. индукции в спирали

\begin{figure}[h]
\begin{center}
  \includegraphics[alt={},max width=\textwidth]{3e870b25-6320-483d-9133-cdd5212c2707-256_336_267_140_887}
\captionsetup{labelformat=empty}
\caption{Рис. 9.19}
\end{center}
\end{figure}

\[
\mathcal{E}_{i}=\int \varepsilon_{i}(r) \mathrm{d} N .
\]

Проинтегрировав, получим следующее выражение для амплитудного значения э.д.с. индукции:

\[
\mathcal{E}_{i m}=(\pi / 3) a^{2} N B_{0} \omega .
\]

9.4. Внутри длинного соленоида находится катушка из \(N\) витков с площадью поперечного сечения \(S\). Катушку поворачивают с постоянной угловой скоростью \(\omega\) вокруг оси, совпадающей с ее диаметром и перпендикулярной оси соленоида. При этом магнитное поле в соленоиде меняется во времени как \(B=B_{0} \sin \omega t\). Найти э.д.с. индукции в катушке, если в момент \(t=0\) ось катушки совпадала с осью соленоида.\\
Решение. В момент \(t\) полный магнитный поток сквозь катушку

\[
\Phi=N B S \cos _{\omega} t=N B_{0} S \sin _{\omega} t \cdot \cos _{\omega} t=1 / 2 N B_{0} S \sin 2 \omega t .
\]

Согласно закону электромагнитной индукции

\[
\mathbf{E}_{i}=-\mathrm{d} \Phi / \mathrm{dt}=-\frac{1}{2} N B_{0} S \cdot 2 \omega \cos 2_{\omega} \mathrm{t}=-N B_{0} S \omega \cos 2 \omega \mathrm{t} .
\]

9.5. Бетатронное условие. Показать, что электроны в бетатроне будут двигаться по орбите постоянного радиуса \(r_{0}\) при условии, что магнитное поле на орбите \(B_{0}\) равно половине среднего по площади внутри орбиты значения магнитного поля \(\langle B\rangle\), т. е. \(B_{0}=\langle B\rangle / 2\).

Решение. Представим релятивистское уравнение движения электрона


\begin{equation*}
\mathrm{d} \mathbf{p} / \mathrm{d} t=e \mathbf{E}+e\left[\mathbf{v B}_{0}\right], \tag{1}
\end{equation*}


где \(\mathbf{E}\) - вихревое электрическое поле, в проекциях на касательную \(\boldsymbol{\tau}\) и нормаль \(\mathbf{n} \boldsymbol{к}\) траектории. Для этого запишем импульс электрона как \(\mathbf{p}=p \boldsymbol{\tau}\) и найдем его производную по времени:


\begin{equation*}
\frac{\mathrm{d} \mathbf{p}}{\mathrm{~d} t}=\frac{\mathrm{d} p}{\mathrm{~d} t} \boldsymbol{\tau}+p \frac{\mathrm{~d} \boldsymbol{\tau}}{\mathrm{~d} t}=\frac{\mathrm{d} p}{\mathrm{~d} t} \boldsymbol{\tau}+m \frac{v^{2}}{r_{0}} \mathbf{n} \tag{2}
\end{equation*}


\begin{figure}[h]
\begin{center}
  \includegraphics[alt={},max width=\textwidth]{3e870b25-6320-483d-9133-cdd5212c2707-257_333_461_149_138}
\captionsetup{labelformat=empty}
\caption{Рис. 9.20}
\end{center}
\end{figure}

где учтено, что \(\mathbf{p}=m \mathbf{v}, m\) - релятивистская масса, и \(\mathrm{d} \boldsymbol{\tau} / \mathrm{d} t=\left(v / r_{0}\right) \mathbf{n}\), в чем нетрудно убедиться с помощью рис. 9.20. Действительно, \(\mathrm{d} \boldsymbol{\tau}=\mathrm{d} \varphi \cdot \mathbf{n}=\left(v \mathrm{~d} t / r_{0}\right) \mathbf{n}\), дальнейшее очевидно.

Кроме того, согласно закону электромагнитной индукции \(2 \pi r_{0} E= =|\mathrm{d} \Phi / \mathrm{d} t|\), где \(\Phi=\pi r_{0}^{2} p B q\). Отсюда


\begin{equation*}
E=\frac{r_{0}}{2} \frac{\mathrm{~d}}{\mathrm{~d} t}\langle B\rangle . \tag{3}
\end{equation*}


Теперь запишем уравнение (1) с учетом формул (2) и (3) в проекциях на касательную и нормаль к траектории:


\begin{gather*}
\frac{\mathrm{d} p}{\mathrm{~d} t}=e E=e \frac{r_{0}}{2} \frac{\mathrm{~d}}{\mathrm{~d} t}\langle B\rangle,  \tag{4}\\
m v^{2} / r_{0}=e v B_{0} .
\end{gather*}


Последнее уравнение можно переписать после сокращения на \(v\) в виде

\[
p=e r_{0} B_{0}
\]

Продифференцируем это уравнение по времени, приняв во внимание, что \(r_{0}=\) const:


\begin{equation*}
\frac{\mathrm{d} p}{\mathrm{~d} t}=e r_{0} \frac{\mathrm{~d} B_{0}}{\mathrm{~d} t} . \tag{5}
\end{equation*}


Из сравнения выражений (5) и (4) получаем

\[
\frac{\mathrm{d}}{\mathrm{~d} t} B_{0}=\frac{1}{2} \frac{\mathrm{~d}}{\mathrm{~d} t}\langle B\rangle .
\]

В частности, последнее условие будет выполнено, если

\[
B_{0}=\langle B\rangle / 2 .
\]

Практически это достигается путем изготовления полюсных наконечников специального вида (в форме усеченных конусов).\\
9.6. Индукционный ток. Квадратная проволочная рамка со стороной \(a\) и прямой длинный проводник с постоянным током \(I_{0}\) лежат в одной плоскости (рис. 9.21). Индуктивность рамки \(L\), ее сопротивле-

ние \(R\). Рамку повернули на \(180^{\circ}\) вокруг оси \(O O^{\prime}\) и остановили. Найти количество электричества, протекшее в рамке. Расстояние \(b\) между осью \(O O^{\prime}\) и прямым проводником предполагается известным.\\
Решение. Согласно закону Ома в процессе поворота рамки ток \(I\) в ней определяется по формуле

\[
R I=-\frac{\mathrm{d} \Phi}{\mathrm{~d} t}-L \frac{\mathrm{~d} I}{\mathrm{~d} t}
\]

\begin{figure}[h]
\begin{center}
  \includegraphics[alt={},max width=\textwidth]{3e870b25-6320-483d-9133-cdd5212c2707-258_321_324_141_838}
\captionsetup{labelformat=empty}
\caption{Рис. 9.21}
\end{center}
\end{figure}

Поэтому искомое количество электричества

\[
q=\int I \mathrm{~d} t=-\frac{1}{R} \int(\mathrm{~d} \Phi+L \mathrm{~d} I)=-\frac{1}{R}(\Delta \Phi+L \Delta I)
\]

Поскольку рамку после поворота остановили, ток в ней прекратился и, следовательно, \(\Delta I=0\). Остается выяснить, чему равно приращение потока \(\Delta \Phi\) сквозь рамку ( \(\Delta \Phi=\Phi_{2}-\Phi_{1}\) ).\\
Выберем нормаль \(\mathbf{n} к\) плоскости рамки, например, так, чтобы в конечном положении \(\mathbf{n}\) было направлено за плоскость рисунка (в сторону B). Тогда нетрудно видеть, что в конечном положении \(\Phi_{2}>0\), а в начальном \(\Phi_{1}<0\) (нормаль направлена против B), и \(\Delta \Phi\) оказывается равным просто потоку через площадь, ограниченную конечным и начальным положениями рамки:

\[
\Delta \Phi=\Phi_{2}+\left|\Phi_{1}\right|=\int_{b-a}^{b+a} B a \mathrm{~d} r
\]

где \(B\) является функцией \(r\), вид которой можно легко найти с помощью теоремы о циркуляции.

Окончательно получим, опуская знак минус:

\[
q=\frac{\Delta \Phi}{R}=\frac{\mu_{0} a I_{0}}{2 \pi R} \ln \frac{b+a}{b-a} .
\]

Найденная величина, как видим, от индуктивности контура не зависит (в случае если бы контур был сверхпроводящим, дело бы обстояло иначе).\\
9.7. Перемычка 12 массы \(m\) скользит без трения по двум длинным проводящим рельсам, расположенным на расстоянии \(l\) друг от друга (рис. 9.22). Система находится в однородном магнитном поле, перпендикулярном плоскости контура. Левые концы рель-

\begin{figure}[h]
\begin{center}
  \includegraphics[alt={},max width=\textwidth]{3e870b25-6320-483d-9133-cdd5212c2707-259_221_407_141_141}
\captionsetup{labelformat=empty}
\caption{Рис. 9.22}
\end{center}
\end{figure}

сов замкнуты через сопротивление \(R\). В момент \(t=0\) перемычке 12 сообщили вправо начальную скорость \(v_{0}\). Пренебрегая сопротивлением рельсов и перемычки, а также самоиндукцией контура, найти скорость перемычки в зависимости от времени \(t\).\\
Решение. Выберем положительное направление нормали к плоскости контура за рисунок (от нас). Это значит, что положительное направление обхода контура (для э.д.с. индукции и тока) мы взяли по часовой стрелке - в соответствии с правилом правого винта. Из закона Ома следует:


\begin{equation*}
R I=-\frac{\mathrm{d} \Phi}{\mathrm{~d} t}=-B \frac{\mathrm{~d} S}{\mathrm{~d} t}=-B l v, \tag{1}
\end{equation*}


где учтено, что при движении перемычки вправо \(\mathrm{d} \Phi>0\).\\
Индукционный ток \(I\) согласно правилу Ленца вызывает противодействующую движению силу Ампера - она будет направлена влево.

Выбрав ось \(X\) вправо, запишем уравнение движения перемычки


\begin{equation*}
m \mathrm{~d} v / \mathrm{d} t=I l B, \tag{2}
\end{equation*}


где справа записана проекция силы Ампера на ось \(X\) (эта величина является отрицательной, но знак минус мы не пишем, ибо, как видно из (1), ток \(I<0\) ). Исключив \(I\) из уравнений (1) и (2), получим

\[
\mathrm{d} v / v=-a \mathrm{~d} t, \quad a=B^{2} l^{2} / m R .
\]

Интегрирование этого выражения с учетом начальных условий дает

\[
\ln \left(v / v_{0}\right)=-a t, \quad v=v_{0} \mathrm{e}^{-a t} .
\]

9.8. Роль переходных процессов. В схеме (рис. 9.23) известны э.д.с. \(\mathcal{E}\)

\begin{figure}[h]
\begin{center}
  \includegraphics[alt={},max width=\textwidth]{3e870b25-6320-483d-9133-cdd5212c2707-259_277_265_1479_139}
\captionsetup{labelformat=empty}
\caption{Рис. 9.23}
\end{center}
\end{figure}

источника, его внутреннее сопротивление \(R\) и индуктивности сверхпроводящих катушек \(L_{1}\) и \(L_{2}\). Найти установившиеся токи в катушках после замыкания ключа \(K\).

Решение. Воспользуемся правилами Кирхгофа для контуров \(\mathcal{E} L_{1}\) и \(\mathcal{E} L_{2}\) :

\[
R I=\mathcal{E}-L_{1} \frac{\mathrm{~d} I_{1}}{\mathrm{~d} t}, \quad R I=\mathcal{E}-L_{2} \frac{\mathrm{~d} I_{2}}{\mathrm{~d} t}
\]

Из сравнения этих выражений видно, что \(L_{1} \mathrm{~d} I_{1}=L_{2} \mathrm{~d} I_{2}\), а для установившихся токов


\begin{equation*}
L_{1} I_{10}=L_{2} I_{20} \tag{1}
\end{equation*}


Кроме того,


\begin{equation*}
I_{10}+I_{20}=I_{0}=\mathcal{E} / R \tag{2}
\end{equation*}


Из уравнений (1) и (2) найдем:

\[
I_{10}=\frac{\mathcal{E}}{R} \frac{L_{2}}{L_{1}+L_{2}}, \quad I_{20}=\frac{\mathcal{E}}{R} \frac{L_{1}}{L_{1}+L_{2}} .
\]

9.9. Вычисление индуктивности. Коаксиальный кабель состоит из внутреннего сплошного проводника радиусом \(a\) и наружной проводящей тонкостенной трубки радиусом \(b\). Найти индуктивность единицы длины кабеля, считая распределение тока по сечению внутреннего проводника равномерным. Магнитная проницаемость всюду равна единице.

Решение. В данном случае внутренний проводник не является тонким, поэтому определять индуктивность надо не через магнитный поток, а энергетически. Согласно (9.33)


\begin{equation*}
L_{\text {ед }}=\frac{1}{I^{2}} \int_{0}^{b} \frac{B^{2}}{\mu_{0}} 2 \pi r \mathrm{~d} r \tag{1}
\end{equation*}


где \(r\) - расстояние от оси кабеля. Для вычисления этого интеграла надо найти зависимость \(B(r)\). С помощью теоремы о циркуляции имеем:


\begin{equation*}
B_{r<a}=\frac{\mu_{0} I}{2 \pi a^{2}} r, \quad B_{a<r<b}=\frac{\mu_{0} I}{2 \pi} \frac{1}{r}, \quad B_{r>b}=0 . \tag{2}
\end{equation*}


Графический вид этих зависимостей показан на рис 9.24. С учетом (2) интеграл (1) разбивается на две части, и в результате интегрирования мы получим

\[
L_{\mathrm{eд}}=\frac{\mu_{0}}{2 \pi}\left(\frac{1}{4}+\ln \frac{b}{a}\right)
\]

Заметим, что определение этой величины через магнитный поток по формуле \(L_{\text {ед }}= =\Phi_{\text {ед }} / I\) приводит к другому - неверному

\begin{figure}[h]
\begin{center}
  \includegraphics[alt={},max width=\textwidth]{3e870b25-6320-483d-9133-cdd5212c2707-260_332_333_1399_827}
\captionsetup{labelformat=empty}
\caption{Рис. 9,24}
\end{center}
\end{figure}

результату, а именно вместо \(1 / 4\) в круглых скобках получается \(1 / 2\). Чем тоньше центральный провод, т. е. больше отношение \(b / a\), тем меньше относительное различие результатов подсчета обоими способами: энергетически и из потока.\\
9.10. Взаимная индукция. Имеется тороидальная катушка и проходя-

\begin{figure}[h]
\begin{center}
  \includegraphics[alt={},max width=\textwidth]{3e870b25-6320-483d-9133-cdd5212c2707-261_328_369_363_159}
\captionsetup{labelformat=empty}
\caption{Рис. 9.25}
\end{center}
\end{figure}

щий по ее оси симметрии длинный прямой провод. Сечение катушки прямоугольное, его размеры указаны на рис. 9.25. Число витков катушки \(N\), магнитная проницаемость окружающей среды равна единице. Найти амплитуду э.д.с., индуцируемой в этой катушке, если по прямому проводу течет переменный ток \(I=I_{m} \cos _{\omega} t\).

Решение. Искомая э.д.с. \(\mathbf{E}_{i}=-\mathrm{d} \Phi / \mathrm{d} t\), где \(\Phi=N \Phi_{1}, \Phi_{1}\) - магнитный поток сквозь поперечное сечение катушки:

\[
\Phi_{1}=\int B_{n} \mathrm{~d} S=\int_{a}^{b} \frac{\mu_{0}}{2 \pi r} I h \mathrm{~d} r=\frac{\mu_{0} h I}{2 \pi} \ln \frac{b}{a},
\]

где \(B_{n}\) определяется с помощью теоремы о циркуляции вектора B. Взяв производную \(\Phi_{1}\) по времени и умножив полученный peзультат на \(N\), найдем следующее выражение для амплитуды э.д.с. индукции:

\[
\mathrm{E}_{i m}=\frac{\mu_{0} h \omega I_{m} N}{2 \pi} \ln \frac{b}{a} .
\]

9.11. Вычисление взаимной индуктивности. Два соленоида одинаковой длины и практически одинакового сечения вставлены полностью один в другой. Индуктивность соленоидов \(L_{1}\) и \(L_{2}\) Пренебрегая краевыми эффектами, найти их взаимную индуктивность (по модулю).\\
Решение. По определению взаимная индуктивность


\begin{equation*}
L_{12}=\Phi_{1} / I_{2}, \tag{1}
\end{equation*}


где \(\Phi_{1}\) - полный магнитный поток через все витки соленоида 1 , если в соленоиде 2 течет ток \(I_{2}\). Поток \(\Phi_{1}=N_{1} B_{2} S\), где \(N_{1}\) - число витков в соленоиде \(1, S\) - сечение соленоида, \(B_{2}=\mu \mu_{0} n_{2} I_{2}\). Поэтому формулу (1) можно переписать так (после сокращения на \(I_{2}\) ):


\begin{equation*}
\left|L_{12}\right|=\mu \mu_{0} n_{2} N_{1} S=\mu \mu_{0} n_{1} n_{2} V, \tag{2}
\end{equation*}


где учтено, что \(N_{1}=n_{1} l, l\) - длина соленоида и \(l S=V\) - его объем. Выражение (2) можно представить через \(L_{1}\) и \(L_{2}\) следующим образом:

\[
\left|L_{12}\right|=\sqrt{\mu \mu_{0} n_{1}^{2} V} \sqrt{\mu \mu_{0} n_{2}^{2} V}=\sqrt{L_{1} L_{2}} .
\]

Заметим, что это выражение определяет предельное (максимальное) значение \(\left|L_{12}\right|\), вообще же \(\left|L_{12}\right|<\sqrt{L_{1} L_{2}}\).\\
9.12. Теорема взаимности. В центре тонкой катушки радиусом \(a\), содержащей \(N\) витков, находится небольшой цилиндрический магнит \(M\) (рис. 9.26). Катушка подключена к баллистическому гальванометру \(\Gamma\).

\begin{figure}[h]
\begin{center}
  \includegraphics[alt={},max width=\textwidth]{3e870b25-6320-483d-9133-cdd5212c2707-262_112_491_512_661}
\captionsetup{labelformat=empty}
\caption{Рис. 9.26}
\end{center}
\end{figure}

Сопротивление цепи \(R\). После того как магнит быстро удалили из катушки, через гальванометр прошел заряд \(q\). Найти магнитный момент магнита.\\
Решение. В процессе удаления магнита полный магнитный поток через катушку изменялся, и в ней возник индукционный ток, определяемый уравнением

\[
R I=-\frac{\mathrm{d} \Phi}{\mathrm{~d} t}-L \frac{\mathrm{~d} I}{\mathrm{~d} t} .
\]

Умножим обе части этого уравнения на \(\mathrm{d} t\) и учтем, что \(I \mathrm{~d} t=\mathrm{d} q\), тогда

\[
R \mathrm{~d} q=-\mathrm{d} \Phi-L \mathrm{~d} I .
\]

Проинтегрировав последнее выражение, получим \(R q=-\Delta \Phi-L \Delta I\). Теперь примем во внимание, что \(\Delta I=0\) (ток был равен нулю как в начале, так и в конце процесса), поэтому


\begin{equation*}
q=\Delta \Phi / R=\Phi / R, \tag{1}
\end{equation*}


где Ф - магнитный поток через катушку в начале процесса (знак минус мы опустили - он не существен).

Итак, задача свелась к определению потока Ф через катушку. Непосредственно определить эту величину мы не можем. Однако данную трудность можно преодолеть, воспользовавшись теоремой взаимности. Заменим мысленно магнит на небольшой виток с током, создающий в окружающем пространстве то же магнитное поле, что и магнит. Если площадь витка \(S\) и ток в нем \(I\), то их произведение должно быть равно магнитному моменту \(p_{\mathrm{m}}\) магнита: \(p_{\mathrm{m}}=I S\). По теореме взаимности \(L_{12} I=L_{21} I\), и вопрос

сводится к нахождению магнитного потока через площадь \(S\) витка, который создает тот же ток \(I\), но текущий в катушке. Считая, что в пределах витка поле однородное, получим


\begin{equation*}
\Phi=B S=\mu_{0} N I S / 2 a \tag{2}
\end{equation*}


Остается подставить (2) в (1) и вспомнить, что \(I S=p_{\mathrm{m}}\). Тогда \(q= =\mu_{0} N p_{\mathrm{m}} / 2 a R\) и

\[
p_{\mathrm{m}}=2 a R q / \mu_{0} N
\]

\section*{Уравнения Максвелла. Энергия электромагнитного поля}
\section*{§ 10.1. Ток смещения}
Открытие Максвелла. Теория электромагнитного поля, начала которой заложил Фарадей, математически была завершена Максвеллом. При этом одной из важнейших новых идей, выдвинутых Максвеллом, была мысль о симметрии во взаимозависимости электрического и магнитного полей. А именно, поскольку меняющееся во времени магнитное поле ( \(\partial \mathbf{B} / \partial t\) ) создает электрическое поле, следует ожидать, что меняющееся во времени электрическое поле ( \(\partial \mathbf{E} / \partial t\) ) создает магнитное поле.

К этой идее о необходимости существования по сути нового явления индукции можно прийти путем, например, следующих рассуждений. Мы знаем, что согласно теореме о циркуляции вектора \(\mathbf{H}\)


\begin{equation*}
\oint \mathbf{H} \mathrm{d} \mathbf{l}=\int \mathbf{j} \mathrm{d} \mathbf{S} . \tag{10.1}
\end{equation*}


Применим эту теорему к случаю, когда предварительно заряженный плоский конденсатор разряжается через некоторое внешнее сопротивление (рис. 10.1, \(a\) ). В качестве контура \(\Gamma\) возьмем кривую, охватывающую провод. На контур Г можно натянуть разные поверхности, например \(S\) и \(S^{\prime}\). Обе поверхности имеют «равные права», однако че-

\begin{figure}[h]
\begin{center}
  \includegraphics[alt={},max width=\textwidth]{3e870b25-6320-483d-9133-cdd5212c2707-264_205_342_1246_566}
\captionsetup{labelformat=empty}
\caption{a)}
\end{center}
\end{figure}

\begin{figure}[h]
\begin{center}
  \includegraphics[alt={},max width=\textwidth]{3e870b25-6320-483d-9133-cdd5212c2707-264_203_215_1246_939}
\captionsetup{labelformat=empty}
\caption{б)}
\end{center}
\end{figure}

рез поверхность \(S\) течет ток \(I\), а через поверхность \(S^{\prime}\) не

Рис. 10.1\\
течет никакого тока!

Получается, что циркуляция вектора \(\mathbf{H}\) зависит от того, какую поверхность мы натягиваем на данный контур (?!), чего явно не может быть (в случае постоянных токов этого и не происходило).

А нельзя ли как-то изменить правую часть (10.1), чтобы избежать этой неприятности? Оказывается, можно, и вот как.

Первое, что мы замечаем, это то, что поверхность \(S^{\prime}\) «пронизывает» только электрическое поле. По теореме Гаусса поток вектора \(\mathbf{D}\) сквозь замкнутую поверхность \(\oint \mathbf{D} \mathrm{d} \mathbf{S}=q\), откуда


\begin{equation*}
\oint \frac{\partial \mathbf{D}}{\partial t} \mathrm{~d} \mathbf{S}=\frac{\partial q}{\partial t} \tag{10.2}
\end{equation*}


С другой стороны, согласно уравнению непрерывности (5.4)


\begin{equation*}
\oint \mathbf{j} \mathrm{d} \mathbf{S}=-\frac{\partial q}{\partial t} \tag{10.3}
\end{equation*}


Сложив отдельно левые и правые части уравнений (10.2) и (10.3), получим


\begin{equation*}
\oint\left(\mathbf{j}+\frac{\partial \mathbf{D}}{\partial t}\right) \mathrm{d} \mathbf{S}=0 \tag{10.4}
\end{equation*}


Это уравнение аналогично уравнению непрерывности для постоянного тока. Из него видно, что кроме плотности тока проводимости \(\mathbf{j}\) имеется еще одно слагаемое \(\partial \mathbf{D} / \partial t\), размерность которого равна размерности плотности тока. Максвелл назвал это слагаемое плотностью тока смещения:


\begin{equation*}
\mathbf{j}_{\mathrm{cM}}=\partial \mathbf{D} / \partial \mathrm{t} \tag{10.5}
\end{equation*}


Сумму же тока проводимости и тока смещения называют полньлм током. Его плотность


\begin{equation*}
\mathbf{j}_{\text {полн }}=\mathbf{j}+\frac{\partial \mathbf{D}}{\partial t} . \tag{10.6}
\end{equation*}


Согласно (10.4) линии полного тока являются непрерывными в отличие от линий тока проводимости. Токи проводимости, если они не замкнуты, замыкаются токами смещения.

Сейчас мы убедимся в том, что введение полного тока устраняет трудность, связанную с зависимостью циркуляции вектора Н от выбора поверхности, натягиваемой на контур Г. Оказывается, для этого достаточно в правой части уравнения (10.1)

вместо тока проводимости ввести полный ток, т. е. величину


\begin{equation*}
I_{\text {полн }}=\int\left(\mathbf{j}+\frac{\partial \mathbf{D}}{\partial t}\right) \mathrm{d} \mathbf{S} . \tag{10.7}
\end{equation*}


В самом деле, правая часть (10.7) представляет собой сумму тока проводимости \(I\) и тока смещения \(I_{\text {см }}: I_{\text {полн }}=I+I_{\text {см }}\). Покажем, что полный ток \(I_{\text {полн }}\) будет одинаков и для поверхности \(S\), и для поверхности \(S^{\prime}\), натянутых на один и тот же контур Г. Для этого применим (10.4) к замкнутой поверхности, составленной из поверхностей \(S\) и \(S^{\prime}\) (рис. 10.1, б). Учитывая, что для замкнутой поверхности нормаль \(\mathbf{n}\) направлена наружу, запишем

\[
I_{\text {полн }}\left(S^{\prime}\right)+I_{\text {полн }}(S)=0 .
\]

Теперь, если обернуть нормаль \(\mathbf{n}^{\prime}\) для поверхности \(S^{\prime}\) в ту же сторону, что и для \(S\), то первое слагаемое в последнем уравнении изменит знак, и мы получим

\[
I_{\text {полн }}\left(S^{\prime}\right)=I_{\text {полн }}(S),
\]

что и требовалось доказать. Итак, теорему о циркуляции вектора Н, которая была установлена для постоянных токов, можно обобщить для произвольного случая и записать


\begin{equation*}
\oint \mathbf{H} \mathrm{d} \mathbf{l}=\int\left(\mathbf{j}+\frac{\partial \mathbf{D}}{\partial t}\right) \mathrm{dS} \tag{10.8}
\end{equation*}


В таком виде теорема о циркуляции вектора \(\mathbf{H}\) справедлива всегда, свидетельством чему является согласие этого уравнения с результатами опыта во всех без исключения случаях.

Дифференциальная форма уравнения (10.8):


\begin{equation*}
\nabla \times \mathbf{H}=\mathbf{j}+\frac{\partial \mathbf{D}}{\partial t}, \tag{10.9}
\end{equation*}


т. е. ротор вектора \(\mathbf{H}\) определяется плотностью тока проводимости \(\mathbf{j}\) и тока смещения \(\partial \mathbf{D} / \partial t\) в той же точке.

Несколько замечаний о токе смещения. Следует иметь в виду, что ток смещения эквивалентен току проводимости только в отношении способности создавать магнитное поле.

Токи смещения существуют лишь там, где меняется со временем электрическое поле. В диэлектриках ток смещения состоит из двух существенно различных слагаемых. Так как вектор \(\mathbf{D}=\varepsilon_{0} \mathbf{E}+\mathbf{P}\), то отсюда видно, что плотность тока смещения \(\partial \mathbf{D} / \partial t\) складывается из «истинного» тока смещения \(\partial \mathbf{E} / \partial t\) и тока поляризации \(\partial \mathbf{P} / \partial t\) - величины, обусловленной движением связанных зарядов. В том, что токи поляризации возбуждают магнитное поле, нет ничего неожиданного, ибо эти токи по природе своей не отличаются от токов проводимости. Принципиально новое содержится в утверждении, что и другая часть тока смещения ( \(\varepsilon_{0} \partial \mathrm{E} / \partial t\) ), которая не связана ни с каким движением зарядов, а обусловлена только изменением электрического поля, также возбуждает магнитное поле. Даже в вакууме всякое изменение во времени электрического поля возбуждает в окружающем пространстве магнитное поле.

Открытие этого явления - наиболее существенный и решающий шаг, сделанный Максвеллом при построении теории электромагнитного поля. Это открытие вполне аналогично открытию электромагнитной индукции, согласно которому переменное магнитное поле возбуждает вихревое электрическое поле. Следует также отметить, что открытие Максвеллом тока смещения - чисто теоретическое открытие, причем первостепенной важности.

Рассмотрим пример, в котором проявляют себя токи смещения.

Пример. В неограниченной однородной проводящей среде находится

\begin{figure}[h]
\begin{center}
  \includegraphics[alt={},max width=\textwidth]{3e870b25-6320-483d-9133-cdd5212c2707-267_350_340_1290_205}
\captionsetup{labelformat=empty}
\caption{Рис. 10.2}
\end{center}
\end{figure}

металлический шар, которому сообщен положительный электрический заряд (рис. 10.2). Электрические токи, текущие в радиальных направлениях, должны возбуждать магнитное поле. Выясним, куда направлен вектор В в произвольной точке \(P\).

Прежде всего ясно, что вектор В не может иметь радиальной составляющей. Если бы это было не так, поток вектора В через поверхность сферы \(S\) (рис. 10.2) был бы отличен от нуля, что противоречит уравнению (7.2). Значит, вектор В должен быть перпендикулярен

радиальному направлению в точке \(P\). Но это также невозможно, так как все направления, перпендикулярные радиальному, совершенно равноправны, они ничем не выделены. Остается единственное - магнитное поле всюду равно нулю.

Отсутствие магнитного поля при наличии электрического тока плотностью \(\mathbf{j}\) означает, что кроме тока проводимости \(\mathbf{j}\) в системе имеется и ток смещения \(\mathbf{j}_{\mathrm{cm}}\), причем такой, что полный ток всюду равен нулю, т. е. в каждой точке \(\mathbf{j}_{\mathrm{cm}}=-\mathbf{j}\). Или

\[
j_{\mathrm{cm}}=j=\frac{\dot{q}}{4 \pi r^{2}} \frac{\partial D}{\partial t},
\]

где принято во внимание, что \(D=q / 4 \pi r^{2}\) согласно теореме Гаусса.

\section*{§ 10.2. Система уравнений Максвелла}
Уравнения Максвелла в интегральной форме. С введением тока смещения макроскопическая теория электромагнитного поля была блестяще завершена. Открытие тока смещения \((\partial \mathbf{D} / \partial t)\) позволило Максвеллу создать единую теорию электрических и магнитных явлений. Теория Максвелла не только объяснила все разрозненные явления электричества и магнетизма (причем с единой точки зрения), но и предсказала ряд новых явлений, существование которых подтвердилось впоследствии.

До сих пор мы рассматривали отдельные части этой теории. Теперь можно представить всю картину в виде системы фундаментальных уравнений электродинамики, называемых уравнениями Максвелла в неподвижных средах. Этих уравнений четыре (мы уже познакомились с каждым из них в отдельности в предшествующих разделах, а сейчас просто соберем их все вместе). В интегральной форме система уравнений Максвелла имеет следующий вид:

\[
\begin{array}{ll}
\oint \mathbf{E} \mathrm{d} \mathbf{l}=-\int \frac{\partial \mathbf{B}}{\partial t} \mathrm{~d} \mathbf{S}, & \oint \mathbf{B} \mathrm{~d} \mathbf{S}=0 \\
\oint \mathbf{H} \mathrm{~d} \mathbf{l}=\int\left(\mathbf{j}+\frac{\partial \mathbf{D}}{\partial t}\right) \mathrm{d} \mathrm{~S}, & \oint \mathbf{D} \mathrm{~d} \mathbf{S}=\int \rho \mathrm{d} V \tag{10.11}
\end{array}
\]

где \(\rho\) - объемная плотность сторонних зарядов, \(\mathbf{j}\) - плотность тока проводимости.

Эти уравнения в сжатой форме выражают всю совокупность наших сведений об электромагнитном поле. Содержание этих уравнений заключается в следующем:

\begin{enumerate}
  \item Циркуляция вектора \(\mathbf{E}\) по любому замкнутому контуру равна со знаком минус производной по времени от магнитного потока через любую поверхность, ограниченную данным контуром. При этом под \(\mathbf{E}\) понимается не только вихревое электрическое поле, но и электростатическое (циркуляция последнего, как известно, равна нулю).
  \item Поток вектора В сквозь произвольную замкнутую поверхность всегда равен нулю.
  \item Циркуляция вектора \(\mathbf{H}\) по любому замкнутому контуру равна полному току (току проводимости и току смещения) через произвольную поверхность, ограниченную данным контуром.
  \item Поток вектора D сквозь любую замкнутую поверхность равен алгебраической сумме сторонних зарядов, охватываемых этой поверхностью.
\end{enumerate}

Из уравнений Максвелла для циркуляции векторов \(\mathbf{E}\) и \(\mathbf{H}\) следует, что электрическое и магнитное поля нельзя рассматривать как независимые: изменение во времени одного из этих полей приводит к появлению другого. Поэтому имеет смысл лишь совокупность этих полей, описывающая единое электромагнитное поле.

Если же поля стационарны ( \(\mathbf{E}=\) const и \(\mathbf{B}=\) const), то уравнения Максвелла распадаются на две группы независимых уравнений:

\[
\begin{array}{ll}
\oint \mathbf{E} \mathrm{d} \mathbf{l}=0, & \oint \mathbf{B} \mathrm{~d} \mathbf{S}=0, \\
\oint \mathbf{H} \mathrm{~d} \mathbf{l}=0, & \oint \mathbf{D} \mathrm{~d} \mathbf{S}=0 . \tag{10.12}
\end{array}
\]

В этом случае электрическое и магнитное поля независимы друг от друга, что и позволило нам изучить сначала постоянное электрическое поле, а затем независимо от него и постоянное магнитное поле.

Необходимо подчеркнуть, что рассуждения, с помощью которых мы пришли к уравнениям Максвелла, ни в коей мере не могут претендовать на их доказательство. Эти уравнения нель-

зя «вывести», они являются основными аксиомами, постулатами электродинамики, полученными путем обобщения опытных фактов. Эти постулаты играют в электродинамике такую же роль, как законы Ньютона в классической механике или начала термодинамики.

Уравнения Максвелла в дифференциальной форме. Уравнения (10.10) и (10.11) можно представить в дифференциальной форме, т. е. в виде системы дифференциальных уравнений, а именно:


\begin{align*}
& \nabla \times \mathbf{E}=-\frac{\partial \mathbf{B}}{\partial t}, \quad \nabla \cdot \mathbf{B}=0  \tag{10.13}\\
& \nabla \times \mathbf{H}=\mathbf{j}+\frac{\partial \mathbf{D}}{\partial t}, \quad \nabla \cdot \mathbf{D}=\rho \tag{10.14}
\end{align*}


Эти уравнения говорят о том, что электрическое поле может возникнуть по двум причинам. Во-первых, его источником являются электрические заряды, как сторонние, так и связанные. Это следует из уравнения \(\boldsymbol{\nabla} \cdot \mathbf{D}=\rho\), если учесть, что \(\mathbf{D}=\varepsilon_{0} \mathbf{E} +\mathbf{P}\) и \(\boldsymbol{\nabla} \cdot \mathbf{P}=-\rho^{\prime}\), тогда \(\boldsymbol{\nabla} \cdot \mathbf{E}\) υ ( \(\rho+\rho^{\prime}\) ). Во-вторых, поле \(\mathbf{E}\) образуется всегда, когда меняется во времени магнитное поле (выражение закона электромагнитной индукции Фарадея).

Эти же уравнения говорят о том, что магнитное поле В может возбуждаться либо движущимися электрическими зарядами (электрическими токами), либо переменными электрическими полями, либо тем и другим одновременно. Это следует из уравнения \(\boldsymbol{\nabla} \times \mathbf{H}=\mathbf{j}+\partial \mathbf{D} / \partial t\), если учесть, что \(\mathbf{H}=\mathbf{B} / \mu_{0}-\mathbf{J}\) и \(\boldsymbol{\nabla} \times \mathbf{J}=\mathbf{j}^{\prime}\), тогда \(\boldsymbol{\nabla} \times \mathbf{B}\) つ ( \(\mathbf{j}+\mathbf{j}^{\prime}+\partial \mathbf{P} / \partial t+\varepsilon_{0} \partial \mathbf{E} / \partial t\) ), где \(\mathbf{j}^{\prime}\) - плотность тока намагничивания, \(\partial \mathbf{P} / \partial t\) - плотность тока поляризации. Первые три тока связаны с движением зарядов, последний ток - с изменяющимся во времени полем Е. Никаких источников магнитного поля, подобных электрическим зарядам (по аналогии их называют магнитными зарядами), в природе не существует, это следует из уравнения \(\boldsymbol{\nabla} \cdot \mathbf{B}=\mathbf{0}\).

Значение уравнений Максвелла в дифференциальной форме не только в том, что они выражают основные законы электромагнитного поля, но и в том, что путем их решения (интегрирования) могут быть найдены сами поля \(\mathbf{E}\) и \(\mathbf{B}\).

Уравнения Максвелла в дифференциальной форме совместно с уравнением движения заряженных частиц под действием силы Лоренца


\begin{equation*}
\mathrm{d} \mathbf{p} / \mathrm{d} t=q \mathbf{E}+q[\mathbf{v B}] \tag{10.15}
\end{equation*}


составляют фундаментальную систему уравнений. Эта система в принципе достаточна для описания всех электромагнитных явлений, в которых не проявляются квантовые эффекты.

Граничные условия. Уравнения Максвелла в интегральной форме обладают большей общностью, чем дифференциальные, ибо они справедливы и в тех случаях, когда существуют поверхности разрыва - поверхности, на которых свойства среды или полей меняются скачкообразно. Уравнения же Максвелла в дифференциальной форме предполагают, что все величины в пространстве и времени изменяются непрерывно.

Можно, однако, достигнуть такой же общности и для дифференциальной формы уравнений, если дополнить их граничными условиями, которым должно удовлетворять электромагнитное поле на границе раздела двух сред. Эти условия содержатся в интегральной форме уравнений Максвелла и имеют уже знакомый нам вид:


\begin{equation*}
D_{1 n}=D_{2 n}, \quad E_{1 \tau}=E_{2 \tau}, \quad B_{1 n}=B_{2 n}, \quad H_{1 \tau}=H_{2 \tau} \tag{10.16}
\end{equation*}


(здесь первое и последнее условия относятся к случаям, когда на границе раздела нет ни сторонних зарядов, ни токов проводимости). Заметим также, что приведенные граничные условия справедливы как для постоянных, так и для переменных полей.

Материальные уравнения. Фундаментальные уравнения Максвелла еще не составляют полной системы уравнений электромагнитного поля. Этих уравнений недостаточно для нахождения полей по заданным распределениям зарядов и токов.

Уравнения Максвелла необходимо дополнить соотношениями, в которые входили бы величины, характеризующие индивидуальные свойства среды. Эти соотношения называют материальными уравнениями. Вообще говоря, эти уравнения достаточно сложны и не обладают той общностью и фундаментальностью, которые свойственны уравнениям Максвелла.

Материальные уравнения наиболее просты в случае достаточно слабых электромагнитных полей, сравнительно медленно меняющихся в пространстве и во времени. В этом случае для изотропных сред, не содержащих сегнетоэлектриков и ферромагнетиков, материальные уравнения имеют следующий вид (он нам уже знаком):


\begin{equation*}
\mathbf{D}=\varepsilon \varepsilon_{0} \mathbf{E}, \quad \mathbf{B}=\mu \mu_{0} \mathbf{H}, \quad \mathbf{j}=\sigma\left(\mathbf{E}+\mathbf{E}^{*}\right), \tag{10.17}
\end{equation*}


где \(\varepsilon, \mu, \sigma\) - известные нам постоянные, характеризующие электрические и магнитные свойства среды (диэлектрическая и магнитная проницаемости и электропроводимость), \(\mathbf{E}^{*}\) - напряженность поля сторонних сил, обусловленная химическими или тепловыми процессами.

\section*{§ 10.3. Свойства уравнений Максвелла}
Уравнения Максвелла линейны. Они содержат только первые производные полей \(\mathbf{E}\) и \(\mathbf{B}\) по времени и пространственным координатам и первые степени плотности электрических зарядов \(\rho\) и токов \(\mathbf{j}\). Свойство линейности уравнений Максвелла непосредственно связано с принципом суперпозиции: если два ка-ких-нибудь поля удовлетворяют уравнениям Максвелла, то это относится и к сумме этих полей.

Уравнения Максвелла содержат уравнение непрерывности, выражающее закон сохранения электрического заряда. Чтобы убедиться в этом, возьмем бесконечно малый контур \(\Gamma\), натянем на него произвольную конечную поверхность \(S\) (рис. 10.3), а затем стянем этот контур в точку, оставляя поверхность \(S\) конечной. В пределе циркуляция \(\oint \mathbf{H d l}\) обращается в нуль, поверхность \(S\) стано-

\begin{figure}[h]
\begin{center}
  \includegraphics[alt={},max width=\textwidth]{3e870b25-6320-483d-9133-cdd5212c2707-272_235_255_1181_909}
\captionsetup{labelformat=empty}
\caption{Рис. 10.3}
\end{center}
\end{figure}

вится замкнутой и первое из уравнений (10.11) перейдет в

\[
\mathrm{K}\left(\mathbf{j}+\frac{\partial \mathbf{D}}{\partial t}\right) \mathrm{d} \mathbf{S}=0
\]

Отсюда следует, что

\[
\oint \mathbf{j} \mathbf{d} \mathbf{S}=-\frac{\partial}{\partial t} \oint \mathbf{D} \mathrm{~d} \mathbf{S}=-\frac{\partial q}{\partial t},
\]

а это и есть не что иное, как уравнение непрерывности (5.4), которое утверждает, что ток, вытекающий из объема \(V\) через замкнутую поверхность \(S\), равен убыли заряда в единицу времени внутри этого объема \(V\).

Тот же закон (уравнение непрерывности) можно получить и из дифференциальных уравнений Максвелла. Достаточно взять дивергенцию от обеих частей первого из уравнений (10.14) и воспользоваться вторым из уравнений (10.14), и мы получим \(\nabla \cdot \mathbf{j}=-\partial \rho / \partial t\).

Уравнения Максвелла выполняются во всех инерциальных системах отсчета. Они являются релятивистски инвариантными. Это есть следствие принципа относительности, согласно которому все инерциальные системы отсчета физически эквивалентны друг другу. Факт инвариантности уравнений Максвелла (относительно преобразований Лоренца) подтверждается многочисленными опытными данными. Вид уравнений Максвелла при переходе от одной инерциальной системы отсчета к другой не меняется, однако входящие в них величины преобразуются по определенным правилам. Как при этом преобразуются векторы Е и В, мы выяснили в гл. 8.

Итак, уравнения Максвелла являются правильными релятивистскими уравнениями в отличие, например, от уравнений механики Ньютона.

О симметрии уравнений Максвелла. Уравнения Максвелла не симметричны относительно электрического и магнитного полей. Это обусловлено опять же тем, что в природе существуют электрические заряды, но нет зарядов магнитных (насколько известно в настоящее время). Вместе с тем в нейтральной однородной непроводящей среде, где \(\rho=0\) и \(\mathbf{j}=0\), уравнения Максвелла приобретают симметричный вид, т. е. Е так связано с \(\partial \mathbf{B} / \partial t\), как \(\mathbf{B}\) с \(\partial \mathbf{E} / \partial t\) :

\[
\begin{array}{ll}
\boldsymbol{\nabla} \times \mathbf{E}=-\partial \mathbf{B} / \partial \mathrm{t}, & \boldsymbol{\nabla} \cdot \mathbf{B}=0, \\
\boldsymbol{\nabla} \times \mathbf{H}=\partial \mathbf{D} / \partial \mathrm{t}, & \boldsymbol{\nabla} \cdot \mathbf{D}=0 . \tag{10.18}
\end{array}
\]

Симметрия уравнений относительно электрического и магнитного полей не распространяется лишь на знак перед производными \(\partial \mathbf{B} / \partial t\) и \(\partial \mathbf{D} / \partial t\). Различие в знаках перед этими произ-

водными показывает, что силовые линии вихревого электрического поля, индуцированного изменением поля B, образуют с вектором \(\partial \mathbf{B} / \partial t\) левовинтовую систему, в то время как линии магнитного поля, индуцируемого изме-

\begin{figure}[h]
\begin{center}
  \includegraphics[alt={},max width=\textwidth]{3e870b25-6320-483d-9133-cdd5212c2707-274_194_386_141_774}
\captionsetup{labelformat=empty}
\caption{Рис. 10.4}
\end{center}
\end{figure}

нением \(\mathbf{D}\), образуют с вектором \(\partial \mathbf{D} / \partial t\) правовинтовую систему (рис. 10.4).

О электромагнитных волнах. Из уравнений Максвелла следует важный вывод о существовании принципиально нового физического явления: электромагнитное поле способно существовать самостоятельно - без электрических зарядов и токов. При этом изменение его состояния обязательно имеет волновой характер. Поля такого рода называют электромагнитными волнами. В вакууме они всегда распространяются со скоростью, равной скорости света \(c\).

Выяснилось также, что ток смещения ( \(\partial \mathbf{D} / \partial t\) ) играет в этом явлении первостепенную роль. Именно его присутствие наряду с величиной \(\partial \mathbf{B} / \partial t\) и означает возможность появления электромагнитных волн. Всякое изменение во времени магнитного поля возбуждает поле электрическое, изменение же поля электрического, в свою очередь, возбуждает магнитное поле. За счет непрерывного взаимопревращения или взаимодействия они и должны сохраняться - электромагнитное возмущение будет распространяться в пространстве.

Теория Максвелла не только предсказала возможность существования электромагнитных волн, но и позволила установить все их основные свойства, а именно: любая электромагнитная волна независимо от ее конкретной формы (это может быть гармоническая волна или электромагнитное возмущение произвольной формы) характеризуется следующими общими свойствами:

\begin{enumerate}
  \item ее скорость распространения в непроводящей нейтральной неферромагнитной среде
\end{enumerate}


\begin{equation*}
v=c / \sqrt{\varepsilon \mu}, \quad \text { где } c=1 / \sqrt{\varepsilon_{0} \mu_{0}} ; \tag{10.19}
\end{equation*}


\begin{enumerate}
  \setcounter{enumi}{1}
  \item векторы E, B и v (скорость волны) взаимно перпендикулярны и образуют правовинтовую систему (рис. 10.5). Такое
\end{enumerate}

\begin{figure}[h]
\begin{center}
  \includegraphics[alt={},max width=\textwidth]{3e870b25-6320-483d-9133-cdd5212c2707-275_233_185_141_70}
\captionsetup{labelformat=empty}
\caption{Рис. 10.5}
\end{center}
\end{figure}

правовинтовое соотношение является внутренним свойством электромагнитной волны, не зависящим ни от какой координатной системы;\\
3) в электромагнитной волне векторы \(\mathbf{E}\) и \(\mathbf{B}\) всегда колеблются в одинаковых фазах (рис. 10.6, где показана мгновенная «фотография» волны), причем между мгновенными значениями \(E\) и \(B\) в любой точке существует определенная связь, а именно \(E=v B\), или


\begin{equation*}
\sqrt{\varepsilon \varepsilon_{0}} E=\sqrt{\mu \mu_{0}} H . \tag{10.20}
\end{equation*}


Это значит, что \(E\) и \(H\) (или \(B\) ) одновременно достигают мак-

\begin{figure}[h]
\begin{center}
  \includegraphics[alt={},max width=\textwidth]{3e870b25-6320-483d-9133-cdd5212c2707-275_303_432_646_68}
\captionsetup{labelformat=empty}
\caption{Рис. 10.6}
\end{center}
\end{figure}

симума, одновременно обращаются в нуль и т. д.

Понимание того, что из дифференциальных уравнений (10.18) вытекала возможность существования электромагнитных волн, позволило Максвеллу с блестящим успехом развить электромагнитную теорию света.

\section*{§ 10.4. Энергия и поток энергии. Вектор Пойнтинга}
Теорема Пойнтинга. Исходя из представления о локализации энергии в самом поле и руководствуясь принципом сохранения энергии, мы должны заключить, что если в какой-то определенной области энергия уменьшается, то это может происходить только за счет ее «вытекания» через границы рассматриваемой области (среда предполагается неподвижной).

В этом отношении существует формальная аналогия с законом сохранения заряда - уравнением (5.4). Смысл этого закона в том, что убыль заряда в данном объеме за единицу времени равна потоку вектора \(\mathbf{j}\) сквозь поверхность, охватывающую этот объем.

Так и в случае закона сохранения энергии следует признать, что существует не только плотность энергии \(w\) в данной области, но и некоторый вектор П, характеризующий плотность потока энергии.

Если говорить только об энергии электромагнитного поля, то его полная энергия в данном объеме будет изменяться как за счет вытекания ее из объема, так и за счет того, что поле передает свою энергию веществу (заряженным частицам), т. е. производит работу над веществом. Макроскопически это утверждение можно записать так:


\begin{equation*}
-\frac{\mathrm{d} W}{\mathrm{~d} t}=\oint \boldsymbol{\Pi} \mathrm{d} \mathbf{S}+P \tag{10.21}
\end{equation*}


где dS - элемент поверхности.\\
Это уравнение выражает теорему Пойнтинга: убыль энергии за единицу времени в данном объеме равна потоку энергии сквозь поверхность, ограниченную этим объемом, плюс мощность \(P\), которую силь поля производят над зарядами вещества внутри данного объема.

В уравнении (10.21) \(W=\int w \mathrm{~d} V, w\) - плотность энергии поля, \(P=\int \mathbf{j} \mathbf{E} \mathrm{d} V, \mathbf{j}\) - плотность тока, \(\mathbf{E}\) - напряженность электрического поля. Приведенное выражение для \(P\) можно получить так. За время \(\mathrm{d} t\) поле \(\mathbf{E}\) совершит над точечным зарядом \(q\) работу \(\mathrm{d} A=\mathbf{E} \cdot \mathbf{u d} t\), где \(\mathbf{u}\) - скорость заряда. Отсюда мощность силы \(q \mathbf{E}\) равна \(P=q \mathbf{u E}\). Переходя к распределению зарядов, заменим \(q\) на \(\rho \mathrm{d} V, \rho\) - объемная плотность заряда. Тогда \(\mathrm{d} P=\rho \mathbf{u E} \mathrm{d} V=\mathbf{j E} \mathrm{d} V\). Остается проинтегрировать \(\mathrm{d} P\) по интересующему нас объему.

Следует отметить, что мощность \(P\) в (10.21) может быть как положительной, так и отрицательной. Последнее имеет место в тех случаях, когда положительные заряды в веществе движутся против направления поля \(\mathbf{E}\) или отрицательные - в противоположном направлении. Например, так обстоит дело в точках среды, где помимо электрического поля \(\mathbf{E}\) действует и поле \(\mathbf{E}^{*}\) сторонних сил. В этих точках \(\mathbf{j}=\sigma\left(\mathbf{E}+\mathbf{E}^{*}\right)\), и если \(\mathbf{E}^{*} \downarrow \uparrow \mathbf{E}\) и по модулю \(E^{*}>E\), то \(\mathbf{j E}\) в выражении для \(P\) оказывается отрицательным.

Пойнтинг получил выражения для плотности энергии \(w\) и вектора П, воспользовавшись уравнениями Максвелла (этот вывод мы приводить не будем). Если среда не содержит сегнетоэлектриков и ферромагнетиков (т. е. нет явления гистерезиса), то плотность энергии электромагнитного поля


\begin{equation*}
w=\frac{\mathbf{E D}}{2}+\frac{\mathbf{B H}}{2} . \tag{10.22}
\end{equation*}


Заметим, что отдельные слагаемые этого выражения мы получили ранее [см. (4.10) и (9.32)].

Плотность же потока энергии электромагнитного поля - вектор П, называемый вектором Пойнтинга, - определяется как


\begin{equation*}
\Pi=[\mathbf{E H}] . \tag{10.23}
\end{equation*}


Строго говоря, для обеих величин, \(w\) и \(П\), из уравнений Максвелла нельзя получить однозначных выражений; приведенные выражения являются простейшими из бесконечного числа возможных. Мы должны поэтому рассматривать эти выражения как постулаты, справедливость которых должна быть подтверждена согласием выводимых из них следствий с опытом.

На нескольких примерах мы увидим, что хотя результаты, получаемые с помощью последних двух формул, иногда выглядят странными, обнаружить в них чего-то невероятного, како-го-либо расхождения с опытом не удается. А это и является свидетельством тому, что оба выражения правильные.

Пример 1. Поток энергии в электромагнитной волне (в вакууме). Вычислим энергию \(\mathrm{d} W\), проходящую за время \(\mathrm{d} t\) через единичную площадку, перпендикулярную направлению распространения волны.

Если в месте нахождения этой площадки известны значения Е и В, то

\[
\mathrm{d} W=w c \mathrm{~d} t,
\]

где \(w\) - плотность энергии, \(w=\varepsilon_{0} E^{2} / 2+\mu_{0} H^{2} / 2\). Для электромагнитной волны в соответствии с (10.20)

\[
\varepsilon_{0} E^{2}=\mu_{0} H^{2} .
\]

Это значит, что в электромагнитной волне плотность электрической энергии в любой момент равна плотности магнитной энергии в той же точке, и можно записать для плотности энергии:

\[
w=\varepsilon_{0} E^{2} .
\]

А тогда

\[
\mathrm{d} W=\varepsilon_{0} E^{2} c \mathrm{~d} t=\sqrt{\varepsilon_{0} / \mu_{0}} E^{2} \mathrm{~d} t .
\]

Теперь выясним, что мы получим, если воспользуемся вектором Пойнтинга. Эту же величину \(d W\) можно представить

через модуль вектора П так:

\[
\mathrm{d} W=\Pi \mathrm{d} t=E H \mathrm{~d} t=\sqrt{\varepsilon_{0} / \mu_{0}} E^{2} \mathrm{~d} t .
\]

Таким образом, оба выражения - для \(w\) и \(П\) - приводят к одинаковому результату (последние две формулы).

Пример 2. Выделение теплоты в проводнике. Пусть по прямому проводу круглого сечения радиусом \(a\) течет ток \(I\) (рис. 10.7). Поскольку провод обладает сопротивлением, то вдоль него действует некоторое электрическое поле Е. Такое же значение Е будет и у поверхности провода в вакууме. Кроме того, наличие тока порождает и магнитное поле. По теореме о циркуляции вектора \(\mathbf{H}\) вблизи поверхности провода \(2 \pi a H=I, H=I / 2 \pi a\). Векторы \(\mathbf{E}\) и \(\mathbf{H}\) рас-

\begin{figure}[h]
\begin{center}
  \includegraphics[alt={},max width=\textwidth]{3e870b25-6320-483d-9133-cdd5212c2707-278_344_245_381_915}
\captionsetup{labelformat=empty}
\caption{Рис. 10.7}
\end{center}
\end{figure}

положены так, что вектор Пойнтинга направлен внутрь провода нормально к его боковой поверхности (рис. 10.7). Следовательно, электромагнитная энергия втекает внутрь провода из окружающего пространства! Но согласуется ли это с количеством теплоты, выделяемым в проводнике? Подсчитаем поток электромагнитной энергии сквозь боковую поверхность участка провода длины \(l\) :

\[
E H \cdot 2 \pi a l=2 \pi a H \cdot E l=I \cdot U=R I^{2},
\]

где учтено, что \(U\) - это разность потенциалов на концах данного участка, \(R\) - его сопротивление. Таким образом, мы приходим к тому, что поток электромагнитной энергии поступает в провод извне и целиком превращается в джоулеву теплоту. Согласимся, что вывод неожиданный.

Заметим, что в источнике тока вектор \(\mathbf{E}\) направлен против тока \(I\), поэтому в области источника вектор Пойнтинга направлен наружу: там электромагнитная энергия выходит в окружающее пространство, т. е. оказывается, что энергия от источника тока передается не вдоль проводов, а через окружающее проводник пространство в виде потока электромагнитной энергии - потока вектора П.

Пример 3. На рис. 10.8 показан участок двухпроводной линии. Известны направление тока в проводах и тот факт, что потенциалы проводов \(\varphi_{1}<\varphi_{2}\). Уста-

\begin{figure}[h]
\begin{center}
  \includegraphics[alt={},max width=\textwidth]{3e870b25-6320-483d-9133-cdd5212c2707-278_116_313_1649_851}
\captionsetup{labelformat=empty}
\caption{Рис. 10.8}
\end{center}
\end{figure}

новим с помощью вектора Пойнтинга П, где находится источник тока (генератор), слева или справа?

В нашем случае между проводами вектор \(\mathbf{E}\) направлен вниз, а вектор H - за плоскость рисунка, поэтому вектор \(\boldsymbol{\Pi}=[\mathbf{E H}]\) направлен вправо, т. е. источник тока находится слева, потребитель - справа.\\
Пример 4. Зарядка конденсатора. Возьмем плоский конденсатор с круглыми обкладками радиусом \(a\). Пренебрегая краевыми

\begin{figure}[h]
\begin{center}
  \includegraphics[alt={},max width=\textwidth]{3e870b25-6320-483d-9133-cdd5212c2707-279_237_297_488_242}
\captionsetup{labelformat=empty}
\caption{На этой поверхности имеется меняюще-}
\end{center}
\end{figure}

эффектами (рассеянием поля), найдем поток электромагнитной энергии сквозь боковую «поверхность» конденсатора, ибо только там вектор Пойнтинга П направлен внутрь конденсатора (рис. 10.9).\\
еся электрическое поле \(\mathbf{E}\) и вызванное его изменением магнитное поле \(\mathbf{H}\). По теореме о циркуляции вектора \(\mathbf{H}\) следует, что \(2 \pi a H=\pi a^{2} \partial D / \partial t\), где справа стоит ток смещения через контур, показанный на рис. 10.9 пунктиром. Отсюда \(H=(a / 2) \partial D / \partial t\). Если расстояние между обкладками \(h\), то поток вектора П сквозь боковую поверхность есть


\begin{equation*}
E H 2 \pi a h=E \frac{a}{2} \frac{\partial D}{\partial t} 2 \pi a h=E \frac{\partial D}{\partial t} V, \tag{1}
\end{equation*}


где \(V=\pi a^{2} h\) - объем конденсатора. Будем считать, что этот поток идет целиком на увеличение энергии конденсатора. Тогда, умножив (1) на \(\mathrm{d} t\), получим приращение энергии конденсатора за время \(\mathrm{d} t\) :

\[
\mathrm{d} W=E \mathrm{~d} D \cdot V=\mathrm{d}\left(\frac{\varepsilon \varepsilon_{0} E^{2}}{2} V\right)=\mathrm{d}\left(\frac{E D}{2} V\right) .
\]

Проинтегрировав это уравнение, найдем формулу для энергии \(W\) заряженного конденсатора. Таким образом, и здесь оказывается все в порядке.

\section*{§ 10.5. Импульс электромагнитного поля}
Давление электромагнитной волны. Максвелл теоретически показал, что электромагнитные волны, отражаясь или поглощаясь в телах, на которые они падают, оказывают на них Әавление. Это давление возникает в результате воздействия маг-

нитного поля волны на электрические токи, возбуждаемые электрическим полем той же волны.

Пусть электромагнитная волна распространяется в однородной среде, обладающей поглощением. Наличие поглощения означает, что в среде будет выделяться джоулева теплота с объемной плотностью \(\rho j^{2}=\sigma E^{2}\), а поэтому \(\sigma \neq 0\), т. е. поглощающая среда обладает проводимостью.

Электрическое поле волны в такой среде возбуждает электрический ток с плотностью \(\mathbf{j}=\sigma \mathbf{E}\). Вследствие этого на единицу объема среды действует амперова сила плотности \(\mathbf{F}_{\text {ед }}=[\mathbf{j B}]=\sigma[\mathbf{E B}]\), направленная в сторону распространения волны (рис. 10.10). Эта сила и вызывает давление электромагнитной волны.

\begin{figure}[h]
\begin{center}
  \includegraphics[alt={},max width=\textwidth]{3e870b25-6320-483d-9133-cdd5212c2707-280_249_307_463_857}
\captionsetup{labelformat=empty}
\caption{Рис. 10.10}
\end{center}
\end{figure}

При отсутствии поглощения проводимость \(\sigma=0\) и \(\mathbf{F}_{\text {ед }}=0\), т. е. в этом случае электромагнитная волна не оказывает никакого давления на среду.

Импульс электромагнитного поля. Поскольку электромагнитная волна оказывает давление на вещество, последнее приобретает определенный импульс. Но в замкнутой системе, состоящей из вещества и электромагнитной волны, возникло бы нарушение закона сохранения импульса, если бы импульсом обладало только вещество.

Импульс такой системы может сохраняться лишь при условии, что электромагнитное поле (волна) также обладает импульсом: вещество приобретает импульс за счет импульса, передаваемого ему электромагнитным полем.

Введем понятие плотности импульса \(\mathbf{G}\) электромагнитного поля как величину, численно равную импульсу поля в единице объема. Расчет, который мы не будем здесь приводить, показывает, что плотность импульса


\begin{equation*}
\mathbf{G}=\boldsymbol{\Pi} / c^{2} \tag{10.24}
\end{equation*}


где \(\boldsymbol{\Pi}=[\mathbf{E H}]\) - вектор Пойнтинга. Как и вектор П, плотность импульса G является, вообще говоря, функцией времени и координат.

Для электромагнитной волны в вакууме согласно (10.20) \(\sqrt{\varepsilon_{0}} E=\sqrt{\mu_{0}} H\), поэтому плотность энергии \(w\) и модуль \(\Pi\) вектора Пойнтинга равны соответственно:

\[
w=\varepsilon_{0} E^{2} / 2+\mu_{0} H^{2} / 2=\varepsilon_{0} E^{2}, \quad \Pi=E H=\sqrt{\varepsilon_{0} / \mu_{0}} E^{2} .
\]

Отсюда следует, что \(\Pi=w / \sqrt{\varepsilon_{0} \mu_{0}}\). А так как \(\sqrt{\varepsilon_{0} \mu_{0}}=1 / c\), \(c\) - скорость света в вакууме, то \(\Pi=w c\), и из формулы (10.24) вытекает, что для электромагнитной волны в вакууме


\begin{equation*}
G=w / c . \tag{10.25}
\end{equation*}


Такая же связь между энергией и импульсом присуща (как показывается в теории относительности) частицам с нулевой массой покоя. Это и естественно, поскольку согласно квантовым представлениям электромагнитная волна эквивалентна потоку фотонов - частиц с нулевой массой покоя.

Еще о давлении электромагнитных волн. Вычислим с помощью формулы (10.25) давление электромагнитной волны на тело, когда волна падает нормально на его поверхность и частично отражается в противоположном направлении. Согласно закону сохранения импульса \(\mathbf{p}_{0}=\mathbf{p}_{0}^{\prime}+\mathbf{p}\), где \(\mathbf{p}_{0}, \mathbf{p}_{0}^{\prime}\) - импульсы

\begin{figure}[h]
\begin{center}
  \includegraphics[alt={},max width=\textwidth]{3e870b25-6320-483d-9133-cdd5212c2707-281_204_364_1083_68}
\captionsetup{labelformat=empty}
\caption{Рис. 10.11}
\end{center}
\end{figure}

падающей и отраженной волн, \(\mathbf{p}\) - импульс, переданный телу (рис. 10.11). Спроектировав это равенство на направление падающей волны и отнеся все величины к единице времени и к единице площади поперечного сечения, получим

\[
p=p_{0}+p_{0}^{\prime}=\langle G\rangle_{c}+\left\langle G^{\prime}\right\rangle c,
\]

где \(\langle G\rangle\) и \(\left\langle G^{\prime}\right\rangle\) - средние значения плотности импульса в падающей и отраженной волнах. Остается учесть связь (10.25) между \(\langle G\rangle\) и \(\langle w\rangle\) и тот факт, что \(\left\langle w^{\prime}\right\rangle=\rho\langle w\rangle\), где \(\rho\) - коэффициент отражения. В результате предыдущее выражение примет вид


\begin{equation*}
p=(1+\rho)\langle w\rangle . \tag{10.26}
\end{equation*}


Здесь величина \(p\) по своему смыслу есть не что иное, как давление электромагнитной волны на тело. При полном отра-

жении \(\rho=1\) и давление \(p=2\langle w\rangle\), при полном поглощении \(\rho=0\) и \(p=\langle w\rangle\).

Остается добавить, что давление электромагнитного излучения обычно бывает очень малым (исключение составляет давление мощных пучков лазерного излучения, особенно после фокусировки пучка, а также давление излучения внутри горячих звезд). Например, давление солнечного излучения на Земле составляет несколько единиц на \(10^{-6}\) Па, что в \(10^{10}\) раз меньше атмосферного давления. Несмотря на ничтожные значения этих величин, экспериментальное доказательство существования электромагнитных волн - светового давления - было получено П. Н. Лебедевым. Результаты этих опытов оказались в согласии с электромагнитной теорией света.

\section*{Задачи}
10.1. Ток смещения. Точечный заряд \(q\) движется равномерно и прямолинейно с нерелятивистской скоростью \(\mathbf{v}\). Найти вектор плотности тока смещения в точке \(P\), находящейся на расстоянии \(r\) от заряда на прямой: 1) совпадающей с его траекторией; 2) перпендикулярной его траектории и проходящей через заряд.\\
Решение. Плотность тока смещения \(\mathbf{j}_{\mathrm{cm}}=\partial \mathbf{D} / \partial t\), поэтому решение задачи сводится к определению вектора \(\mathbf{D}\) в указанных точках и нахождению его производной по времени. В обоих случаях \(\mathbf{D}=q \mathbf{e}_{r} / 4 \pi r^{2}\), где \(\mathbf{e}_{r}\) - орт вектора \(\mathbf{r}\). Найдем производную \(\partial \mathbf{D} / \partial t\).

\begin{enumerate}
  \item В точке \(P_{1}\) (рис. 10.12, где предполагается, что \(q>0\) )
\end{enumerate}

\[
\frac{\partial \mathbf{D}}{\partial t}=-\frac{2 q}{4 \pi r^{3}} \frac{\partial r}{\partial t} \mathbf{e}_{r}=\frac{2 q \mathbf{v}}{4 \pi r^{3}},
\]

здесь учтено, что для точки \(P_{1}\) производная \(\partial r / \partial t=-v\). Если бы точка \(P_{1}\) находилась не перед зарядом \(q\) (по ходу его движения), а за ним, то вектор \(\mathbf{j}_{\mathrm{cM}}\) был бы направлен в ту же сторону и имел бы тот же модуль.

\begin{figure}[h]
\begin{center}
  \includegraphics[alt={},max width=\textwidth]{3e870b25-6320-483d-9133-cdd5212c2707-282_395_412_1167_750}
\captionsetup{labelformat=empty}
\caption{Рис. 10.12}
\end{center}
\end{figure}

Итак, если \(q>0\), вектор \(\mathbf{j}_{\mathrm{cm}} \uparrow \uparrow \mathbf{v}\), и наоборот.\\
2. В точке \(P_{2}\) (рис. 10.12) \(|\mathrm{d} \mathbf{D}| / D=v \mathrm{~d} t / r\), поэтому:

\[
\partial \mathbf{D} / \partial \mathrm{t}=-q \mathbf{v} / 4 \pi r^{3} .
\]

Если \(q>0\), то \(\mathbf{j}_{\mathrm{cm}} \downarrow \uparrow \mathbf{v}\), и наоборот.\\
10.2. Ток, текущий по длинному прямому соленоиду, радиус сечения которого \(R\), меняют так, что магнитное поле внутри соленоида возрастает со временем по закону \(B=\beta t^{2}\), где \(\beta\) - постоянная. Найти плотность тока смещения как функцию расстояния \(r\) от оси соленоида.

Решение. Чтобы определить плотность тока смещения, надо согласно (10.5) сначала найти напряженность электрического поля - здесь оно будет вихревым. Воспользовавшись уравнением Максвелла для циркуляции вектора \(\mathbf{E}\), запишем:

\begin{figure}[h]
\begin{center}
  \includegraphics[alt={},max width=\textwidth]{3e870b25-6320-483d-9133-cdd5212c2707-283_285_416_518_155}
\captionsetup{labelformat=empty}
\caption{Рис. 10.13}
\end{center}
\end{figure}

\[
\begin{gathered}
2 \pi r E=\pi r^{2} \partial B / \partial t, \quad E=r \beta t(r<R) ; \\
2 \pi r E=\pi r^{2} \partial B / \partial t, \quad E=R^{2} \beta t / r(r>R),
\end{gathered}
\]

Теперь по формуле \(j_{\mathrm{cm}}=\varepsilon_{0} \partial E / \partial t\) найдем плотность тока смещения:

\[
\begin{aligned}
& j_{\mathrm{cM}}=\varepsilon_{0} \beta r \quad(r<R) ; \\
& j_{\mathrm{cM}}=\varepsilon_{0} \beta R^{2} / r \quad(r>R) .
\end{aligned}
\]

График зависимости \(j_{\mathrm{cm}}(r)\) показан на рис. 10.13.\\
10.3. Плоский конденсатор образован двумя дисками, между которыми находится однородная слабо проводящая среда. Конденсатор зарядили и отключили от источника напряжения. Пренебрегая краевыми эффектами, показать, что магнитное поле внутри конденсатора отсутствует.

Решение. Магнитное поле будет отсутствовать, потому что полный ток (ток проводимости плюс ток смещения) равен нулю. Это и надо показать. Обратимся к плотности тока. Пусть в некото-

\begin{figure}[h]
\begin{center}
  \includegraphics[alt={},max width=\textwidth]{3e870b25-6320-483d-9133-cdd5212c2707-283_120_265_1264_157}
\captionsetup{labelformat=empty}
\caption{Рис. 10.14}
\end{center}
\end{figure}

рый момент плотность тока проводимости равна \(\mathbf{j}\). Ясно, что \(\mathbf{j}\) с \(\mathbf{D}\), причем \(\mathbf{D}=\sigma \mathbf{n}\), где \(\sigma\) поверхностная плотность заряда на положительно заряженной обкладке; \(\mathbf{n}\) - нормаль (рис. 10.14).

Наличие тока проводимости приводит к уменьшению поверхностной плотности заряда \(\sigma\), а следовательно, и \(\mathbf{D}\) - ток проводимости будет сопровождаться током смещения. Плотность последнего

\[
\mathbf{j}_{\mathrm{cM}}=\partial \mathbf{D} / \partial \mathrm{t}=(\partial \boldsymbol{\sigma} / \partial \mathrm{t}) \mathbf{n}=-j \mathbf{n}=-\mathbf{j} .
\]

Отсюда следует, что действительно

\[
\mathbf{j}_{\text {полн }}=\mathbf{j}+\mathbf{j}_{\mathrm{cM}}=0
\]

10.4. Пространство между обкладками плоского конденсатора, имеющими форму круглых дисков, заполнено однородной слабо проводящей средой с удельной проводимостью \(\sigma\) и диэлектрической проницаемостью \(\varepsilon\). Пренебрегая краевыми эффектами, найти модуль вектора \(\mathbf{H}\) между обкладками на расстоянии \(r\) от их оси, если напряженность электрического поля между обкладками меняется со временем по закону \(E=E_{m} \cos _{\omega} t\).\\
Решение. Из уравнения Максвелла для циркуляции вектора Н следует, что

\[
2 \pi r H=\left(j_{n}+\varepsilon \varepsilon_{0} \frac{\partial E_{n}}{\partial t}\right) \pi r^{2} .
\]

Принимая во внимание закон Ома \(j_{n}=\sigma E_{n}(t)\), получим

\[
H=\frac{r}{2}\left(\sigma E_{n}+\varepsilon \varepsilon_{0} \frac{\partial E_{n}}{\partial t}\right)=\frac{r E_{m}}{2}\left(\sigma \cos \omega t-\varepsilon \varepsilon_{0} \omega \sin \omega t\right) .
\]

Преобразуем выражение в скобках к косинусу. Для этого умножим и разделим это выражение на \(f=\sqrt{\sigma^{2}+\left(\varepsilon \varepsilon_{0} \omega\right)^{2}}\), а затем введем угол \(\delta\) по формулам \(\sigma / f=\cos \delta, \quad \varepsilon \varepsilon_{0} \omega / f=\sin \delta\). Тогда

\[
H=\frac{r}{2} E_{m} \sqrt{\sigma^{2}+\left(\varepsilon \varepsilon_{0} \omega\right)^{2}}|\cos (\omega t+\delta)| .
\]

10.5. Точечный заряд \(q\) движется в вакууме равномерно и прямолинейно с нерелятивистской скоростью \(\mathbf{v}\). Воспользовавшись уравнением Максвелла для циркуляции вектора \(\mathbf{H}\), получить выражение для \(\mathbf{H}\) в точке \(P\), положение которой относительно заряда характеризуется радиусом-вектором \(\mathbf{r}\) (рис. 10.15).

\begin{figure}[h]
\begin{center}
  \includegraphics[alt={},max width=\textwidth]{3e870b25-6320-483d-9133-cdd5212c2707-284_249_255_1018_907}
\captionsetup{labelformat=empty}
\caption{Рис. 10.15}
\end{center}
\end{figure}

Решение. Из соображений симметрии ясно, что в качестве контура, по которому надо брать циркуляцию вектора Н, следует взять окружность с центром \(O\) (ее след показан на рис. 10.16 штриховой линией). Тогда


\begin{equation*}
2 \pi R H=\frac{\partial}{\partial t} \int D_{n} \mathrm{~d} S, \tag{1}
\end{equation*}


где \(R\) - радиус окружности.\\
Найдем поток вектора \(\mathbf{D}\) сквозь поверхность, ограниченную этой окружно-

\begin{figure}[h]
\begin{center}
  \includegraphics[alt={},max width=\textwidth]{3e870b25-6320-483d-9133-cdd5212c2707-284_444_363_1330_799}
\captionsetup{labelformat=empty}
\caption{Рис. 10.16}
\end{center}
\end{figure}

стью. Проще всего, если эту поверхность \(S\) взять сферической с радиусом кривизны \(r\) (рис.10.16). Тогда поток вектора \(\mathbf{D}\) через элементарное кольцо, взятое на данной сферической поверхности, есть

\[
D \mathrm{~d} S=\frac{q}{4 \pi r^{2}} 2 \pi r \sin \alpha^{\prime} \cdot r \mathrm{~d} \alpha^{\prime}=\frac{q}{2} \sin \alpha^{\prime} \mathrm{d} \alpha^{\prime},
\]

а весь поток сквозь выбранную поверхность


\begin{equation*}
\int D \mathrm{~d} S=\frac{q}{2}(1-\cos \alpha) \tag{2}
\end{equation*}


Теперь согласно (1) продифференцируем (2) по времени:


\begin{equation*}
\frac{\partial}{\partial t} \int D \mathrm{~d} S=\frac{q}{2} \sin \alpha \frac{\mathrm{~d} \alpha}{\mathrm{~d} t} \tag{3}
\end{equation*}


\begin{figure}[h]
\begin{center}
  \includegraphics[alt={},max width=\textwidth]{3e870b25-6320-483d-9133-cdd5212c2707-285_314_340_728_147}
\captionsetup{labelformat=empty}
\caption{Рис. 10.17}
\end{center}
\end{figure}

При перемещении заряда из точки 1 в точку 2 (рис. 10.17) на расстояние \(v \mathrm{~d} t\) имеем \(v \mathrm{~d} t \cdot \sin _{\alpha}=r\) d \(\alpha\), откуда


\begin{equation*}
\frac{\mathrm{d} \alpha}{\mathrm{~d} t}=\frac{v \sin \alpha}{r} . \tag{4}
\end{equation*}



\begin{equation*}
H=q v r \sin \alpha / 4 \pi r^{3}, \tag{5}
\end{equation*}


где учтено, что \(R=r \sin _{\alpha}\). Соотношение (5) в векторной форме имеет вид

\[
\mathbf{H}=\frac{q}{4 \pi} \frac{[\mathbf{v r}]}{r^{3}} .
\]

Мы видим, таким образом, что постулированное нами ранее выражение (6.3) является следствием уравнений Максвелла.\\
10.6. Ротор Е. В некоторой области инерциальной системы отсчета имеется вращающееся с угловой скоростью \(\omega\) маг-\\
\includegraphics[max width=\textwidth, alt={}, center]{3e870b25-6320-483d-9133-cdd5212c2707-285_273_117_1470_187}\\
нитное поле, модуль которого \(B=\) const. Найти \(\nabla \times \mathbf{E}\) в этой области как функцию векторов \(\boldsymbol{\omega}\) и \(\mathbf{B}\).

Решение. Из уравнения \(\boldsymbol{\nabla} \times \mathbf{E}=-\partial \mathbf{B} / \partial t\) видно, что вектор \(\boldsymbol{\nabla} \times \mathbf{E}\) направлен противоположно вектору dB , а его модуль можно вычислить с помощью рис. 10.18:

\[
|\mathrm{d} \mathbf{B}|=B \cdot \omega \mathrm{~d} t, \quad|\mathrm{~d} \mathbf{B} / \mathrm{d} t|=B \omega .
\]

Рис. 10.18

Поэтому

\[
\nabla \times \mathbf{E}=-[\boldsymbol{\omega} \mathbf{B}] .
\]

10.7. Вектор Пойнтинга. Протоны, имеющие одинаковую скорость \(\mathbf{v}\), образуют пучок круглого сечения с током \(I\). Найти направление и модуль вектора Пойнтинга \(П\) вне пучка на расстоянии \(r\) от его оси.

Решение. Из рис. 10.19 видно, что \(\boldsymbol{\Pi} \uparrow \uparrow \mathbf{v}\). Найдем модуль вектора \(\boldsymbol{\Pi}\) : \(\Pi=E H\), где \(E\) и \(H\) зависят от \(r\). По теореме Гаусса

\[
2 \pi r E=\lambda / \varepsilon_{0},
\]

где \(\lambda\) - заряд на единицу длины пучка. Кроме того, по теореме о циркуляции вектора H

\begin{figure}[h]
\begin{center}
  \includegraphics[alt={},max width=\textwidth]{3e870b25-6320-483d-9133-cdd5212c2707-286_276_409_432_753}
\captionsetup{labelformat=empty}
\caption{Рис. 10.19}
\end{center}
\end{figure}

\[
2 \pi r H=I .
\]

Определяя \(E\) и \(H\) из последних двух уравнений и учитывая, что \(I=\lambda v\), получаем

\[
\Pi=E H=I^{2} / 4 \pi^{2} \varepsilon_{0} v r^{2} .
\]

10.8. Ток, протекающий по обмотке длинного прямого соленоида, увеличивают. Показать, что скорость возрастания энергии магнитного поля в соленоиде равна потоку вектора Пойнтинга через его боковую поверхность.\\
Решение. При возрастании тока увеличивается магнитное поле в соленоиде, а значит, появляется вихревое электрическое поле. Пусть радиус сечения соленоида равен \(a\). Тогда напряженность вихревого электрического поля у боковой поверхности соленоида можно определить с помощью уравнения Максвелла, выражающего закон электромагнитной индукции:

\[
2 \pi a E=\pi a^{2} \frac{\partial B}{\partial t}, \quad E=\frac{a}{2} \frac{\partial B}{\partial t} .
\]

Поток энергии через боковую поверхность соленоида можно представить в таком виде:

\[
\Phi=E H \cdot 2 \pi a l=\pi a^{2} l \frac{\partial}{\partial t}\left(\frac{B^{2}}{2 \mu_{0}}\right),
\]

где \(l\) - длина соленоида, \(\pi a^{2} l\) - его объем.

Таким образом, мы видим, что поток энергии через боковую поверхность соленоида (поток вектора Пойнтинга) равен скорости изменения магнитной энергии внутри соленоида:

\[
\Phi=\Pi \cdot 2 \pi a l=\partial W / \partial t .
\]

10.9. Энергия от источника постоянного напряжения \(U\) передается к потребителю по длинному коаксиальному кабелю с пренебрежимо малым сопротивлением. Ток в кабеле \(I\). Найти поток энергии через поперечное сечение кабеля. Внешняя проводящая оболочка кабеля тонкостенная.

Решение. Искомый поток энергии определяется формулой


\begin{equation*}
\Phi=\int_{a}^{b} \Pi \cdot 2 \pi r \mathrm{~d} r \tag{1}
\end{equation*}


\begin{figure}[h]
\begin{center}
  \includegraphics[alt={},max width=\textwidth]{3e870b25-6320-483d-9133-cdd5212c2707-287_190_474_741_153}
\captionsetup{labelformat=empty}
\caption{Рис. 10.20}
\end{center}
\end{figure}

где \(\Pi=E H\) - плотность потока, \(2 \pi r \mathrm{~d} r\) - площадь кольца шириной \(\mathrm{d} r\), в пределах которого \(\Pi\) одинаково, \(a\) и \(b\) - радиусы внутреннего провода и внешней оболочки кабеля (рис. 10.20). Для вычисления этого интеграла необходимо знать зависимость \(\Pi(r)\), или \(E(r)\) и \(H(r)\). С помощью теоремы Гаусса получим


\begin{equation*}
2 \pi r E=\lambda / \varepsilon_{0}, \tag{2}
\end{equation*}


где \(\lambda\) - заряд провода на единицу длины. Далее, по теореме о циркуляции


\begin{equation*}
2 \pi r H=I . \tag{3}
\end{equation*}


После подстановки \(E\) и \(H\) из формул (2) и (3) в выражение (1) и интегрирования получим


\begin{equation*}
\Phi=\frac{\lambda I}{2 \pi \varepsilon_{0}} \ln \frac{b}{a} . \tag{4}
\end{equation*}


В условии задачи \(\lambda, a\) и \(b\) не заданы, вместо них дано \(U\). Найдем связь между этими величинами:


\begin{equation*}
U=\int_{a}^{b} E \mathrm{~d} r=\frac{\lambda}{2 \pi \varepsilon_{0}} \ln \frac{b}{a} . \tag{5}
\end{equation*}


Из сопоставления (4) и (5) следует, что

\[
\Phi=U I .
\]

Это совпадает со значением мощности, выделяемой на нагрузке.\\
10.10. Плоский воздушный конденсатор, пластины которого имеют форму дисков радиусом \(a\), подключен к источнику переменного гармонического напряжения частоты \(\omega\). Найти отношение максимальных значений магнитной и электрической энергии внутри конденсатора.

Решение. Пусть напряжение на конденсаторе меняется по закону \(U=U_{m} \cos \omega t\) и расстояние между пластинами конденсатора равно \(h\). Тогда электрическая энергия конденсатора


\begin{equation*}
W_{\ni}=\frac{\varepsilon_{0} E^{2}}{2} \pi a^{2} h=\frac{\varepsilon_{0} \pi a^{2}}{2 h} U_{m}^{2} \cos ^{2} \omega t . \tag{1}
\end{equation*}


Магнитную энергию определим по формуле


\begin{equation*}
W_{\mathrm{M}}=\int \frac{B^{2}}{2 \mu_{0}} \mathrm{~d} V \tag{2}
\end{equation*}


Необходимую для вычисления этого интеграла величину \(B\) найдем из теоремы о циркуляции вектора \(\mathbf{H}: 2 \pi r H=\pi r^{2} \partial D / \partial t\). Отсюда, имея в виду, что \(H=B / \mu_{0}\) и \(\partial D / \partial t=-\varepsilon_{0}\left(U_{m} / h\right) \omega \sin \omega t\), получим


\begin{equation*}
B=\frac{1}{2} \varepsilon_{0} \mu_{0} \frac{r \omega U_{m}}{h}|\sin \omega t| . \tag{3}
\end{equation*}


Остается подставить (3) в (2), где в качестве \(\mathrm{d} V\) надо взять элементарный объем в виде кольца, для которого \(\mathrm{d} V=2 \pi r \mathrm{~d} r \cdot h\). В результате интегрирования найдем


\begin{equation*}
W_{\mathrm{M}}=\frac{\pi}{16} \frac{\mu_{0} \varepsilon_{0}^{2} \omega a^{4} U_{m}^{2}}{h} \sin ^{2} \omega t \tag{4}
\end{equation*}


Отношение максимальных значений магнитной энергии (4) и электрической энергии (1) таково:

\[
\frac{W_{\text {м. макс }}}{W_{\text {э.ммкс }}}=\frac{1}{8} \mu_{0} \varepsilon_{0} a^{2} \omega^{2} .
\]

Например, при \(a=6 с м\) и \(\omega=1000 \mathrm{c}^{-1}\) это отношение равно \(5 \cdot 10^{-15}\).

\section*{Электрические колебания}
\section*{§ 11.1. Уравнение колебательного контура}
Условие квазистационарности. Когда происходят электрические колебания, ток в цепи изменяется во времени и, вообще говоря, в каждый момент ток оказывается не одинаковым на разных участках цепи (из-за того что электромагнитные возмущения распространяются хотя и с очень большой, но конечной скоростью). Однако имеется много случаев, когда мгновенные значения тока оказываются практически одинаковыми на всех участках цепи (такой ток называют квазистационарньмм). Для этого все изменения во времени должны происходить настолько медленно, чтобы распространение электромагнитных возмущений можно было считать мгновенным. Если \(l\) - длина цепи, то на прохождение длины \(l\) электромагнитное возмущение затрачивает время порядка \(\tau=l / c\). Для периодически изменяющихся токов условие квазистационарности будет выполнено, если

\[
\tau=l / c \ll T,
\]

где \(T\) - период изменений.\\
Например, для цепи длиной \(l=3\) м время \(\tau=10^{-8}\) с, и токи можно считать квазистационарными вплоть до частот \(10^{6}\) Гц (это соответствует \(T=10^{-6}\) с).

В этой главе мы всюду будем предполагать, что в рассматриваемых нами случаях условие квазистационарности выполняется, и токи будем считать квазистационарньми. Это позволит нам использовать формулы, полученные в статических полях. В частности, мы будем использовать тот факт, что мгновенные значения квазистационарных токов подчиняются закону Ома.

Колебательный контур. В цепи, содержащей катушку индуктивности \(L\) и конденсатор емкости \(C\), могут возникнуть электрические колебания. Поэтому такую цепь называют колебательным контуром. Выясним, каким образом в колебательном контуре возникают и поддерживаются электрические колебания.

Пусть вначале верхняя обкладка конденсатора заряжена положительно, а нижняя отрицательно (рис. 11.1, a). При этом вся энергия колебательного контура сосредоточена в конденсаторе. Замкнем ключ \(K\). Конденсатор начнет разряжаться, и через катушку \(L\) потечет ток. Электрическая

\begin{figure}[h]
\begin{center}
  \includegraphics[alt={},max width=\textwidth]{3e870b25-6320-483d-9133-cdd5212c2707-290_255_543_500_617}
\captionsetup{labelformat=empty}
\caption{Рис. 11.1}
\end{center}
\end{figure}

энергия конденсатора начнет превращаться в магнитную энергию катушки. Этот процесс закончится, когда конденсатор полностью разрядится, а ток в цепи достигнет максимума (рис. 11.1, б). С этого момента ток, не меняя направления, начнет убывать. Однако он прекратится не сразу - его будет поддерживать э.д.с. самоиндукции. Ток будет перезаряжать конденсатор, возникнет электрическое поле, стремящееся ослабить ток. Наконец, ток прекратится, а заряд на конденсаторе достигнет максимума. С этого момента конденсатор начнет разряжаться опять, ток потечет в обратном направлении и т. д. - процесс будет повторяться.

В контуре при отсутствии сопротивления проводников будут совершаться строго периодические колебания. В ходе процесса периодически изменяются заряд на обкладках конденсатора, напряжение на нем и ток через катушку. Колебания сопровождаются взаимными превращениями энергии электрического и магнитного полей.

Если же сопротивление проводников \(R \neq 0\), то помимо описанного процесса будет происходить преобразование электромагнитной энергии в джоулеву теплоту.

Сопротивление проводников цепи \(R\) принято называть \(\boldsymbol{a} \boldsymbol{\kappa}\) тивным сопротивлением.

\begin{figure}[h]
\begin{center}
  \includegraphics[alt={},max width=\textwidth]{3e870b25-6320-483d-9133-cdd5212c2707-291_243_348_141_68}
\captionsetup{labelformat=empty}
\caption{Рис. 11.2}
\end{center}
\end{figure}

Уравнение колебательного контура. Найдем уравнение колебаний в контуре, содержащем последовательно соединенные конденсатор \(C\), катушку индуктивности \(L\), активное сопротивление \(R\) и внешнюю переменную э.д.с. \(\mathcal{E}\) (рис. 11.2).

Прежде всего выберем положительное направление обхода контура, например по часовой стрелке. Рассмотрим ситуацию в момент времени, когда нижняя обкладка 2 в процессе перезарядки конденсатора имеет некоторый заряд \(q>0\) и ток \(I\) течет в положительным направлении. Тогда за промежуток времени \(\mathrm{d} t\) заряд \(q\) получит приращение \(\mathrm{d} q>0\), и ток в контуре определяется как


\begin{equation*}
I=\mathrm{d} q / \mathrm{d} t . \tag{11.1}
\end{equation*}


Следовательно, если \(I>0\), то и \(\mathrm{d} q>0\), и наоборот (знак \(I\) совпадает со знаком dq).

Согласно закону Ома для участка цепи \(1 R L 2\)


\begin{equation*}
R I=\varphi_{1}-\varphi_{2}+\mathcal{E}_{s}+\mathcal{E}, \tag{11.2}
\end{equation*}


где \(\mathcal{E}_{s}\) - э.д.с. самоиндукции. В нашем случае

\[
\mathcal{E}_{s}=-L \mathrm{~d} I / \mathrm{d} t, \quad \varphi_{2}-\varphi_{1}=q / C
\]

(знак \(q\) должен совпадать со знаком разности \(\varphi_{2}-\varphi_{1}\), ибо \(C>0\) ). Поэтому уравнение (11.2) можно переписать в виде


\begin{equation*}
L \frac{\mathrm{~d} I}{\mathrm{~d} t}+R I+\frac{q}{C}=\mathcal{E}, \tag{11.3}
\end{equation*}


или с учетом (11.1) как


\begin{equation*}
L \frac{\mathrm{~d}^{2} q}{\mathrm{~d} t^{2}}+R \frac{\mathrm{~d} q}{\mathrm{~d} t}+\frac{1}{C} q=\mathcal{E} \tag{11.4}
\end{equation*}


Это и есть уравнение колебательного контура - линейное дифференциальное неоднородное уравнение второго порядка с постоянными коэффициентами. Найдя с помощью этого уравнения \(q(t)\), мы можем легко вычислить напряжение на конденсаторе как \(U_{C}=\varphi_{2}-\varphi_{1}=q / C\) и силу тока \(I\) - по формуле (11.1).

Уравнению колебательного контура можно придать иной вид:


\begin{equation*}
\ddot{q}+2 \beta \dot{q}+\omega_{0}^{2} q=\mathrm{E} / L, \tag{11.5}
\end{equation*}


где введены обозначения


\begin{equation*}
2 \beta=R / L, \quad \omega_{0}^{2}=1 / L C . \tag{11.6}
\end{equation*}


Величину \(\omega_{0}\) называют собственной частотой контура, \(\beta\) - коэффициентом затухания. Смысл этих названий мы выясним ниже.

Уравнение (11.5) можно получить и из энергетических соображений, имея в виду, что приращение энергии \(W\) контура в единицу времени, то есть \(\mathrm{d} W / \mathrm{d} t=\mathcal{E} I-R I^{2}\). Здесь \(\mathcal{E} I\) - мощность внешней э.д.с., а \(R I^{2}\) - тепловая мощность (ясно, что она должна быть со знаком минус). Учитывая, что \(W=L I^{2} / 2++q^{2} / 2 C\), получим после дифференцирования \(W\) по \(t\) :

\[
L I \dot{I}+q \dot{q} / C=\mathcal{E} I-R I^{2},
\]

где \(\dot{q}=I\). После сокращения всех слагаемых на \(I\) и переноса последнего слагаемого в левую часть равенства находим

\[
L \dot{I}+R I+q / C=\mathcal{E} .
\]

Остается учесть, что \(\dot{I}=\ddot{q}\) и \(I=\dot{q}\), и мы приходим после деления всех слагаемых на \(L\) к уравнению (11.5).

Если \(\mathcal{E}=0\), то колебания принято называть свободньми. При \(R=0\) они будут незатухающими, при \(R \neq 0\) - затухающими. Рассмотрим последовательно все эти случаи.

\section*{§ 11.2. Свободные электрические колебания}
Свободные незатухающие колебания. Если в контуре нет внешней э.д.с. \(\mathcal{E}\) и активное сопротивление \(R=0\), то колебания в таком контуре являются свободными незатухающими. Их уравнение - частный случай уравнения (11.5), когда \(\mathcal{E}=0\) и \(R=0\),


\begin{equation*}
\ddot{q}+\omega_{0}^{2} q=0 . \tag{11.7}
\end{equation*}


Решением этого уравнения является функция


\begin{equation*}
q=q_{m} \cos \left(\omega_{0} t+\alpha\right) \tag{11.8}
\end{equation*}


где \(q_{m}\) - амплитудное значение заряда на обкладке конденсатоpa, \(\omega_{0}\) - собственная частота контура, \(\alpha\) - начальная фаза. Значение \(\omega_{0}\) определяется только свойствами самого контура, значения же \(q_{m}\) и \(\alpha\) - начальными условиями. В качестве таковых можно взять, например, значения заряда \(q\) и тока \(I=\dot{q}\) в момент \(t=0\).

Согласно (11.6) \(\omega_{0}=1 / \sqrt{L C}\), поэтому период свободных незатухающих колебаний


\begin{equation*}
T_{0}=2 \pi / \sqrt{L C} \tag{11.9}
\end{equation*}


\section*{(формула Томсона).}
Найдя ток \(I\) (дифференцированием (11.8) по времени) и имея в виду, что напряжение на конденсаторе находится в фазе с зарядом \(q\), нетрудно убедиться, что при свободных незатухающих колебаниях ток \(I\) опережает по фазе напряжение на конденсаторе на \(\pi / 2\).

Свободные затухающие колебания. Каждый реальный контур обладает активным сопротивлением, и энергия, запасенная в контуре, постепенно расходуется на нагревание. Свободные колебания будут затухающими.

Уравнение данного колебательного контура мы получим, положив в (11.5) \(\mathcal{E}=0\). Тогда


\begin{equation*}
\ddot{q}+2 \beta \dot{q}+\omega_{0}^{2} q=0 . \tag{11.10}
\end{equation*}


Можно показать (но мы не будем этого делать, поскольку нас интересует другая сторона вопроса), что при \(\beta<\omega_{0}\) решение этого однородного дифференциального уравнения имеет вид


\begin{equation*}
q=q_{m} \mathrm{e}^{-\beta t} \cos (\omega t+\alpha), \tag{11.11}
\end{equation*}


где \(\omega\) - частота затухающих колебаний:


\begin{equation*}
\omega=\sqrt{\omega_{0}^{2}-\beta^{2}}=\sqrt{\frac{1}{L C}-\left(\frac{R}{2 L}\right)^{2}}, \tag{11.12}
\end{equation*}


\begin{figure}[h]
\begin{center}
  \includegraphics[alt={},max width=\textwidth]{3e870b25-6320-483d-9133-cdd5212c2707-293_299_398_1488_64}
\captionsetup{labelformat=empty}
\caption{Рис. 11.3}
\end{center}
\end{figure}

а \(q_{m}\) и \(\alpha\) - произвольные постоянные, определяемые из начальных условий. График функции (11.11) показан на рис. 11.3. Видно, что эта функция не периодическая, она определяет затухающие колебания.

Величину \(T=2 \pi / \omega\) называют тем не менее периодом затуxαющих κοлебаний:


\begin{equation*}
T=\frac{2 \pi}{\sqrt{\omega_{0}^{2}-\beta^{2}}}=\frac{T_{0}}{\sqrt{1-\left(\beta / \omega_{0}\right)^{2}}}, \tag{11.13}
\end{equation*}


где \(T_{0}\) - период свободных незатухающих колебаний.\\
Множитель \(q_{m} \mathrm{e}^{-\beta t}\) в (11.11) называют амплитудой затухающих колебаний. Зависимость ее от времени показана штриховой линией на рис. 11.3.

Напряжение на конденсаторе и ток в контуре. Зная \(q(t)\), можно найти напряжение на конденсаторе и ток в контуре. Напряжение на конденсаторе


\begin{equation*}
U_{C}=\frac{q}{C}=\frac{q_{m}}{C} \mathrm{e}^{-\beta t} \cos (\omega t+\alpha) . \tag{11.14}
\end{equation*}


Ток в контуре

\[
I=\frac{\mathrm{d} q}{\mathrm{~d} t}=q_{m} \mathrm{e}^{-\beta t}[-\beta \cos (\omega t+\alpha)-\omega \sin (\omega t+\alpha)] .
\]

Преобразуем выражение в квадратных скобках к косинусу. Для этого умножим и разделим это выражение на \(\sqrt{\omega^{2}+\beta^{2}}=\omega_{0}\), а затем введем угол \(\delta\) по формулам


\begin{equation*}
-\beta / \omega_{0}=\cos \delta, \quad \omega / \omega_{0}=\sin \delta . \tag{11.15}
\end{equation*}


После этого выражение для \(I\) примет вид


\begin{equation*}
I=\omega q_{m} \mathrm{e}^{-\beta t} \cos (\omega \mathrm{t}+\alpha+\delta) . \tag{11.16}
\end{equation*}


Из (11.15) следует, что угол \(\delta\) лежит во второй четверти ( \(\pi / 2<\delta<\pi\) ). Это означает, что при наличии активного сопротивления \(R\) ток в контуре опережает по фазе напряжение (11.14) на конденсаторе более чем на \(\pi / 2\). Заметим, что при \(R=0\) опережение \(\delta=\pi / 2\).

Графики зависимостей \(U_{C}(t)\) и \(I(t)\) имеют вид, аналогичный показанному на рис. 11.3 для \(q(t)\).

Пример. Колебательный контур содержит конденсатор емкости \(C\) и катушку с активным сопротивлением \(R\) и индуктивностью \(L\). Найдем отношение энергии магнитного поля к энергии электрического поля в контуре в момент максимума тока.

Согласно уравнению колебательного контура (11.3)

\[
L \frac{\mathrm{~d} I}{\mathrm{~d} t}+R I+\frac{q}{C}=0
\]

В момент максимума тока \(\mathrm{d} I / \mathrm{d} t=0\) и \(R I=-q / C\). Поэтому искомое отношение

\[
\frac{W_{\mathrm{M}}}{W_{\ni}}=\frac{L I^{2} / 2}{q^{2} / 2 C}=\frac{L}{C R^{2}}
\]

Величины, характеризующие затухание.

\begin{enumerate}
  \item Коэффициент затухания \(\beta\) и время релаксации \(\tau\) время, за которое амплитуда колебаний уменьшается в е раз. Из формулы (11.11) нетрудно видеть, что
\end{enumerate}


\begin{equation*}
\tau=1 / \beta . \tag{11.17}
\end{equation*}


\begin{enumerate}
  \setcounter{enumi}{1}
  \item Логарифмический декремент затухания \(\lambda\). Он определяется как натуральный логарифм отношения двух значений амплитуд, взятых через период колебания \(T\) :
\end{enumerate}


\begin{equation*}
\lambda=\ln \frac{a(t)}{a(t+T)}=\beta T, \tag{11.18}
\end{equation*}


где \(a\) - амплитуда соответствующей величины ( \(q, U, I\) ). Или иначе:


\begin{equation*}
\lambda=1 / N_{\mathrm{e}}, \tag{11.19}
\end{equation*}


где \(N_{\mathrm{e}}\) - число колебаний за время \(\tau\), т. е. за время, в течение которого амплитуда колебаний уменьшается в е раз. Это легко получить из формул (11.17) и (11.18).

Если затухание мало ( \(\beta \ll \omega_{0}\) ), то \(\omega \approx \omega_{0}=1 / \sqrt{L C}\) и согласно (11.18)


\begin{equation*}
\lambda \approx \beta \cdot 2 \pi / \omega_{0}=\pi R \sqrt{C / L} . \tag{11.20}
\end{equation*}


\begin{enumerate}
  \setcounter{enumi}{2}
  \item Добротность \(Q\) колебательного контура. По определению
\end{enumerate}


\begin{equation*}
\mathrm{Q}=\pi / \lambda=\pi N_{\mathrm{e}} \tag{11.21}
\end{equation*}


где \(\lambda\) - логарифмический декремент затухания. Чем меньше затухание, тем больше \(Q\). При слабом затухании ( \(\beta \ll \omega_{0}\) ) согласно (11.20) добротность


\begin{equation*}
Q \approx(1 / R) \sqrt{L / C} . \tag{11.22}
\end{equation*}


И еще одна полезная формула для \(Q\) в случае слабого затухания


\begin{equation*}
Q \approx 2 \pi \frac{W}{\delta W}, \tag{11.23}
\end{equation*}


где \(W\) - энергия, запасенная в контуре, \(\delta W\) - уменьшение этой энергии за период колебания \(T\). В самом деле, энергия \(W\) пропорциональна квадрату амплитуды заряда конденсатора, т. е. \(W\) ɔ \(\mathrm{e}^{-2 \beta t}\). Отсюда относительное уменьшение энергии за период \(\delta W / W=2 \beta T=2 \lambda\). Остается учесть согласно (11.21), что \(\lambda=\pi / Q\).

В заключение отметим, что при \(\beta \geqslant \omega_{0}\) вместо колебаний будет происходить апериодический разряд конденсатора. Активное сопротивление контура, при котором наступает апериодический процесс, называют критическим:


\begin{equation*}
R_{\text {кр }}=2 \sqrt{L / C} . \tag{11.24}
\end{equation*}


Рассмотрим два примера.\\
Пример 1. Колебательный контур имеет емкость \(C\), индуктивность \(L\) и активное сопротивление \(R\). Найдем, через сколько колебаний амплитуда тока в этом контуре уменьшится в е раз. Амплитуда тока ( \(I_{m} \sim \mathrm{e}^{-\beta t}\) ) уменьшится в е раз за время \(\tau=1 / \beta\). За это время совершится \(N_{\mathrm{e}}\) колебаний. Если \(T\) период затухающих колебаний, то

\[
N_{\mathrm{e}}=\frac{\tau}{T}=\frac{1 / \beta}{2 \pi / \sqrt{\omega_{0}^{2}-\beta^{2}}}=\frac{1}{2 \pi} \sqrt{\left(\frac{\omega_{0}}{\beta}\right)^{2}-1} .
\]

Имея в виду, что \(\omega_{0}^{2}=1 / L C\) и \(\beta=R / 2 L\), получим

\[
N_{\mathrm{e}}=\frac{1}{2 \pi} \sqrt{\frac{4 l}{C R^{2}}-1} .
\]

Пример 2. Найдем время, за которое амплитуда колебаний тока в контуре с добротностью \(Q\) уменьшится в \(\eta\) раз, если частота затухающих колебаний равна \(\omega\).\\
Так как амплитуда тока \(I_{m} \sim \mathrm{e}^{-\beta t}\), то время \(t_{0}\), за которое амплитуда уменьшится в \(\eta\) раз, определяется уравнением \(\eta=\exp \left(\beta t_{0}\right)\). Отсюда

\[
t_{0}=(\ln \eta) / \beta .
\]

С другой стороны, добротность \(Q\) также связана с \(\beta\) :

\[
Q=\pi / \lambda=\pi / \beta T=\omega / 2 \beta .
\]

Исключив \(\beta\) из последних двух уравнений, получим

\[
t_{0}=\frac{2 Q}{\omega} \ln \eta .
\]

\section*{§ 11.3. Вынужденные электрические колебания}
Установившиеся колебания. Вернемся к уравнениям колебательного контура (11.3) и (11.4) и рассмотрим случай, когда в контур включена внешняя переменная э.д.с. \(\mathcal{E}\), зависящая от времени по гармоническому закону:


\begin{equation*}
\mathcal{E}=\mathcal{E}_{m} \cos \omega t . \tag{11.25}
\end{equation*}


Этот закон занимает особое положение благодаря свойствам самого колебательного контура сохранять гармонический вид колебаний при действии внешней гармонической э.д.с.

В данном случае уравнение колебательного контура записывается как


\begin{equation*}
L \frac{\mathrm{~d} I}{\mathrm{~d} t}+R I+\frac{q}{C}=\mathcal{E}_{m} \cos \omega t, \tag{11.26}
\end{equation*}


или


\begin{equation*}
\ddot{q}+2 \beta \dot{q}+\omega_{0}^{2} q=\left(\mathcal{E}_{m} / L\right) \cos \omega t . \tag{11.27}
\end{equation*}


Решение этого уравнения, как известно из математики, представляет собой сумму общего решения однородного уравнения (без правой части) и частного решения неоднородного уравнения.

Нас будут интересовать только установившиеся колебания, т. е. частное решение этого уравнения (общее решение однородного уравнения экспоненциально затухает, и по прошествии некоторого времени оно практически исчезает, обращается в нуль). Нетрудно убедиться, что это решение имеет вид


\begin{equation*}
q=q_{m} \cos (\omega t-\psi), \tag{11.28}
\end{equation*}


где \(q_{m}\) - амплитуда заряда на конденсаторе; \(\psi\) - разность фаз между колебаниями заряда и внешней э.д.с. \(\mathcal{E}(11.25)\). Как мы

увидим, \(q_{m}\) и \(\psi\) определяются только свойствами самого контура и вынуждающей э.д.с. \(\mathcal{E}\), причем оказывается, что \(\psi>0\), поэтому \(q\) всегда отстает по фазе от \(\mathcal{E}\).

Чтобы определить постоянные \(q_{m}\) и \(\psi\), надо подставить (11.28) в исходное уравнение (11.27) и преобразовать полученное выражение. Мы же поступим несколько иначе (в целях достижения большей простоты): сначала найдем ток \(I\) и затем его выражение подставим в исходное уравнение (11.26). Попутно будет решен и вопрос с постоянными \(q_{m}\) и \(\psi\).

Продифференцировав (11.28) по \(t\), найдем

\[
I=-\omega q_{m} \sin (\omega t-\psi)=\omega q_{m} \cos (\omega t-\psi+\pi / 2) .
\]

Запишем это выражение так:


\begin{equation*}
I=I_{m} \cos (\omega t-\varphi), \tag{11.29}
\end{equation*}


где \(I_{m}\) - амплитуда тока, \(\varphi\) - сдвиг по фазе между током и внешней э.д.с. \(\mathcal{E}\),


\begin{equation*}
I_{m}=\omega q_{m}, \quad \varphi=\psi-\pi / 2 . \tag{11.30}
\end{equation*}


Наша задача найти \(I_{m}\) и \(\varphi\). С этой целью мы поступим следующим образом. Представим исходное уравнение (11.26) в виде


\begin{equation*}
U_{L}+U_{R}+U_{C}=\mathcal{E}_{m} \cos \omega t, \tag{11.31}
\end{equation*}


где слева записана сумма напряжений на индуктивности \(L\), сопротивлении \(R\) и емкости \(C\). Таким образом, мы видим, что сумма этих напряжений равна в каждый момент внешней э.д.с. \(\mathcal{E}\). Учитывая соотношения (11.30), запишем:


\begin{gather*}
U_{R}=R I=R I_{m} \cos (\omega t-\varphi),  \tag{11.32}\\
U_{C}=\frac{q}{C}=\frac{q_{m}}{C} \cos (\omega t-\psi)=\frac{I_{m}}{\omega C} \cos \left(\omega t-\varphi-\frac{\pi}{2}\right),  \tag{11.33}\\
U_{L}=L \frac{\mathrm{~d} I}{\mathrm{~d} t}=-\omega L I_{m} \sin (\omega t-\varphi)=\omega L I_{m} \cos \left(\omega t-\varphi+\frac{\pi}{2}\right) . \tag{11.34}
\end{gather*}


Векторная диаграмма. Из последних трех формул видно, что \(U_{R}\) находится в фазе с током \(I\), \(U_{C}\) omcmaem по фазе от \(I\) на \(\pi / 2\), а \(U_{L}\) опережает \(I\) на \(\pi / 2\). Все это можно наглядно представить с помощью векторной диаграммьι, изобразив амплитуды напряжений

\[
U_{R m}=R I_{m}, \quad U_{C m}=I_{m} / \omega C, \quad U_{L m}=\omega L I_{m}
\]

и их векторную сумму, равную согласно (11.31) вектору величины \(\mathcal{E}_{m}\) (рис. 11.4).

Из прямоугольного треугольника этой диаграммы легко получить следующие выражения для \(I_{m}\) и \(\varphi\) в (11.29):

\begin{figure}[h]
\begin{center}
  \includegraphics[alt={},max width=\textwidth]{3e870b25-6320-483d-9133-cdd5212c2707-299_360_429_651_68}
\captionsetup{labelformat=empty}
\caption{Рис. 11.4}
\end{center}
\end{figure}


\begin{gather*}
I_{m}=\frac{\mathcal{E}_{m}}{\sqrt{R^{2}+(\omega L-1 / \omega C)^{2}}}  \tag{11.35}\\
\operatorname{tg} \varphi=\frac{\omega L-1 / \omega C}{R} \tag{11.36}
\end{gather*}


Задача, таким образом, решена.\\
Заметим в заключение, что полученная нами векторная диаграмма оказывается весьма полезной при решении многих конкретных вопросов. Она позволяет наглядно, легко и быстро анализировать различные ситуации.

Резонансные кривые. Так называют графики зависимостей от частоты \(\omega\) внешней э.д.с. \(\mathcal{E}\) амплитуд следующих величин: тока \(I\), заряда \(q\) на конденсаторе и напряжений \(U_{R}, U_{C}\) и \(U_{L}\), определяемых формулами (11.32)-(11.34).

Резонансные кривые для силы тока \(I_{m}(\omega)\) показаны на рис. 11.5. Как видно из выражения (11.35), амплитуда силы тока имеет максимальное значение при \(\omega L-1 / \omega C=0\). Следовательно, резонансная частота для силы тока совпадает с собственной частотой контура:


\begin{equation*}
\omega_{I \text { рез }}=\omega_{0}=1 / \sqrt{L C} . \tag{11.37}
\end{equation*}


Максимум при резонансе оказывается тем выше и острее, чем меньше коэффициент затухания \(\beta=R / 2 L\).

\begin{figure}[h]
\begin{center}
  \includegraphics[alt={},max width=\textwidth]{3e870b25-6320-483d-9133-cdd5212c2707-300_355_384_144_138}
\captionsetup{labelformat=empty}
\caption{Рис. 11.5}
\end{center}
\end{figure}

\begin{figure}[h]
\begin{center}
  \includegraphics[alt={},max width=\textwidth]{3e870b25-6320-483d-9133-cdd5212c2707-300_355_421_144_664}
\captionsetup{labelformat=empty}
\caption{Рис. 11.6}
\end{center}
\end{figure}

Резонансные кривые для заряда на конденсаторе \(q_{m}(\omega)\) показаны на рис. 11.6 (резонансные кривые для напряжения \(U_{C m}\) на конденсаторе имеют такой же вид). Максимум амплитуды заряда достигается при резонансной частоте


\begin{equation*}
\omega_{q \text { рез }}=\sqrt{\omega_{0}^{2}-2 \beta^{2}}, \tag{11.38}
\end{equation*}


которая по мере уменьшения \(\beta\) все больше приближается к \(\omega_{0}\). Для получения выражения (11.38) надо представить \(q_{m}\) согласно (11.30) как \(q_{m}=I_{m} / \omega\), где \(I_{m}\) дается формулой (11.35). Тогда


\begin{equation*}
q_{m}=\frac{\mathcal{E}_{m} / L}{\sqrt{\left(\omega_{0}^{2}-\omega^{2}\right)+4 \beta^{2} \omega^{2}}} . \tag{11.39}
\end{equation*}


Максимум этой функции, или, что то же самое, минимум подкоренного выражения, найдем, приравняв производную по \(\omega\) от подкоренного выражения к нулю. Отсюда и получим резонансную частоту (11.38).

Теперь посмотрим, как перераспределяются амплитуды напряжений \(U_{R}, U_{C}\) и \(U_{L}\) в зависимости от частоты \(\omega\) внешней э.д.с. Эта картина изображена на рис. 11.7.

Резонансные частоты для \(U_{R}, U_{C}\) и \(U_{L}\) определяются следующими формулами:

\begin{figure}[h]
\begin{center}
  \includegraphics[alt={},max width=\textwidth]{3e870b25-6320-483d-9133-cdd5212c2707-300_349_524_1379_636}
\captionsetup{labelformat=empty}
\caption{Рис. 11.7}
\end{center}
\end{figure}


\begin{gather*}
\omega_{\text {Rрез }}=\omega_{0}, \\
\omega_{C \text { рез }}=\omega_{0} \sqrt{1-2\left(\beta / \omega_{0}\right)^{2}},  \tag{11.40}\\
\omega_{L \text { рез }}=\omega_{0} / \sqrt{1-2\left(\beta / \omega_{0}\right)^{2}} .
\end{gather*}


Чем меньше \(\beta\), тем ближе резонансные частоты всех величин к значению \(\omega_{0}\).

Резонансные кривые и добротность \(Q\). Форма резонансных кривых определенным образом связана с добротностью \(Q\) контура. Особенно простой эта связь оказывается для случая слабого затухания, т. е. при \(\beta \ll \omega_{0}\). В этом случае


\begin{equation*}
U_{\text {Cpe }_{3}} / \mathcal{E}_{m}=Q \tag{11.41}
\end{equation*}


(рис. 11.7). Действительно, при \(\beta \ll \omega_{0}\) величина \(\omega_{\text {рез }} \approx \omega_{0}\) и согласно (11.33) и (11.35) \(U_{\text {Срез }}=I_{m} / \omega_{0} C=\mathcal{E}_{m} / \omega_{0} C R\), или \(U_{C m \text { рез }} / \mathcal{E}_{m}= =\sqrt{L C} / C R=(1 / R) \sqrt{L / C}\), а это, как показывает сравнение с формулой (11.22), и есть \(Q\).

Таким образом, добротность контура (при \(\beta \ll \omega_{0}\) ) показывает, во сколько раз максимальное значение амплитуды напряжения на конденсаторе (и на индуктивности) превышает амплитуду внешней э.д.с.

Добротность контура связана и с другой важной характеристикой резонансной кривой - ее шириной. Оказывается, при \(\beta \ll \omega_{0}\)


\begin{equation*}
Q=\omega_{0} / \delta \omega \tag{11.42}
\end{equation*}


где \(\omega_{0}\) - резонансная частота; \(\delta \omega\) - ширина резонансной кривой на «высоте», равной 0,7 от максимальной, т. е. в резонансе.

Резонанс. Явление резонанса в нашем случае - это возбуждение сильных колебаний при частоте внешней э.д.с. или напряжения, равной или близкой к собственной частоте колебательного контура. Резонанс используют для выделения из сложного напряжения нужной составляющей. На этом основана вся техника радиоприема. Для того чтобы радиоприемник принимал интересующую нас радиостанцию, его необходимо настроить, т. е. изменением \(C\) и \(L\) колебательного контура добиться совпадения его собственной частоты с частотой электромагнитных волн, излучаемых радиостанцией.

С явлением резонанса связана и опасностъ: внешняя э.д.с. или напряжение могут быть малы, однако при этом напряжения на отдельных элементах контура (на емкости или индуктивности) могут достигать опасного для жизни значения. Об этом необходимо всегда помнить!

\section*{§ 11.4. Переменный ток}
Полное сопротивление (импеданс). Установившиеся вынужденные электрические колебания можно рассматривать как протекание в цепи, обладающей емкостью, индуктивностью и активным сопротивлением \(R\), переменного тока. Под действием внешнего напряжения (оно играет роль внешней э.д.с. \(\mathcal{E}\) )


\begin{equation*}
U=U_{m} \cos \omega t \tag{11.43}
\end{equation*}


ток в цепи изменяется по закону


\begin{equation*}
I=I_{m} \cos (\omega t-\varphi), \tag{11.44}
\end{equation*}


где


\begin{equation*}
I_{m}=\frac{U_{m}}{\sqrt{R^{2}+(\omega L-1 / \omega C)^{2}}}, \quad \operatorname{tg} \varphi=\frac{\omega L-1 / \omega C}{R} . \tag{11.45}
\end{equation*}


Задача сводится к определению амплитуды силы тока и сдвига тока по фазе относительно \(U\).

Полученное выражение для амплитуды силы тока \(I_{m}(\omega)\) можно формально толковать как закон Ома для амплитудных значений тока и напряжения. Стоящую в знаменателе этого выражения величину, имеющую размерность сопротивления, обозначают буквой \(Z\) и называют полньм сопротивлением или импедансом:


\begin{equation*}
Z=\sqrt{R^{2}+(\omega L-1 / \omega C)^{2}} . \tag{11.46}
\end{equation*}


Видно, что при \(\omega=\omega_{0}=1 / \sqrt{L C}\) это сопротивление минимально и равно активному сопротивлению \(R\). Величину, стоящую в круглых скобках формулы (11.46), обозначают \(X\) и называют реактивным сопротивлением:


\begin{equation*}
X=\omega L-1 / \omega C . \tag{11.47}
\end{equation*}


При этом величину \(\omega L\) называют индуктивньмм сопротивлением, а величину \(1 / \omega C\) - емкостным сопротивлением. Их обозначают соответственно \(X_{L}\) и \(X_{C}\). Итак,


\begin{equation*}
X_{L}=\omega L, \quad X_{C}=1 / \omega C, \quad X=X_{L}-X_{C}, \quad Z=\sqrt{R^{2}+X^{2}} . \tag{11.48}
\end{equation*}


Заметим, что индуктивное сопротивление растет с увеличением частоты \(\omega\), а емкостное - уменьшается. Когда говорят, что в цепи отсутствует емкость, то это надо понимать в смысле отсутствия емкостного сопротивления, которое равно \(1 / \omega C\) и, следовательно, обращается в нуль, если \(C \rightarrow \infty\) (при замене конденсатора закороченным участком).

И последнее. Хотя реактивное сопротивление измеряют в тех же единицах, что и активное, между ними существует принципиальное различие. Оно заключается в том, что только активное сопротивление определяет необратимые процессы в цепи, такие, например, как преобразование электромагнитной энергии в джоулеву теплоту.

Мощность, выделяемая в цепи переменного тока. Мгновенное значение мощности равно произведению мгновенных значений напряжения и тока:


\begin{equation*}
P(t)=U I=U_{m} I_{m} \cos \omega t \cos (\omega t-\varphi) . \tag{11.49}
\end{equation*}


Поскольку \(\cos (\omega t-\varphi)=\cos \omega t \cos \varphi+\sin \omega t \sin \varphi\), преобразуем (11.49) к виду

\[
P(t)=U_{m} I_{m}\left(\cos ^{2} \omega t \cos \varphi+\sin \omega t \cos \omega t \sin \varphi\right) .
\]

Практический интерес имеет среднее за период колебания значение мощности. Учитывая, что \(\left\langle\cos ^{2} \omega t\right\rangle=1 / 2\) и \(\langle\sin \omega t \cos \omega t\rangle= =0\), получим:


\begin{equation*}
\langle P\rangle=\frac{U_{m} I_{m}}{2} \cos \varphi . \tag{11.50}
\end{equation*}


Это выражение можно привести к иному виду, если принять во внимание, что из векторной диаграммы (см. рис. 11.4) следует \(U_{m} \cos \varphi=R I_{m}\). Поэтому


\begin{equation*}
\langle P\rangle=R I_{m}^{2} / 2 \tag{11.51}
\end{equation*}


Такую же мощность развивает постоянный ток \(I=I_{m} / \sqrt{2}\). Величины


\begin{equation*}
I=I_{m} / \sqrt{2}, \quad U=U_{m} / \sqrt{2} \tag{11.52}
\end{equation*}


называют действующими (или эффективньми) значениями тока и напряжения. Все амперметры и вольтметры градуированы по действующим значениям тока и напряжения.

Выражение средней мощности (11.50) через действующие значения напряжения и тока имеет вид


\begin{equation*}
\langle P\rangle=U I \cos \varphi, \tag{11.53}
\end{equation*}


где множитель \(\cos \varphi\) принято называть коэффициентом мощности. Таким образом, выделяемая в цепи мощность зависит не только от напряжения и силы тока, но еще и от сдвига фаз между током и напряжением.

При \(\varphi=\pi / 2\) значение \(\langle P\rangle=0\), каковы бы ни были величины \(U\) и \(I\). В этом случае энергия, передаваемая за четверть периода от генератора во внешнюю цепь, в точности равна энергии, передаваемой из внешней цепи в генератор в течение следующей четверти периода, и вся энергия бесполезно «колеблется» между генератором и внешней цепью.

Зависимость мощности от \(\cos \varphi\) необходимо учитывать при проектировании линий электропередачи на переменном токе. Если питаемые нагрузки имеют большое реактивное сопротивление \(X\), то \(\cos \varphi\) может быть заметно меньше единицы. В этих случаях для передачи потребителю нужной мощности (при данном напряжении генератора) необходимо увеличить ток \(I\), а это приводит к возрастанию бесполезных потерь энергии в подводящих проводах. Поэтому всегда нужно стремиться распределять нагрузки, индуктивности и емкости так, чтобы \(\cos \varphi\) был по возможности близок к единице. Для этого достаточно сделать реактивное сопротивление \(X\) как можно меньше, т. е. обеспечить равенство индуктивного и емкостного сопротивлений ( \(X_{L}=X_{C}\) ).

В заключение заметим, что понятие активного сопротивления шире, чем понятие электрического сопротивления проводников, образующих цепь. Последнее обусловливает переход

энергии тока только в джоулеву теплоту, но возможны и другие превращения этой энергии, например в механическую работу (электромоторы). Активное сопротивление тогда уже не сводится к электрическому сопротивлению, а обычно значительно превышает его.

\section*{Задачи}
11.1. Собственные незатухающие колебания. В контуре, состоящем из конденсатора емкости \(C\) и катушки с индуктивностью \(L\), происходят свободные незатухающие колебания с амплитудой напряжения на конденсаторе \(U_{m}\). Найти э.д.с. самоиндукции в катушке в моменты, когда ее магнитная энергия оказывается равной электрической энергии конденсатора.

Решение. Согласно закону Ома

\[
R I=U+\mathcal{E}_{s},
\]

где \(U\) - напряжение на конденсаторе ( \(U=\varphi_{1}-\varphi_{2}\) ). В нашем случае \(R=0\), поэтому \(\mathcal{E}_{s}=-U\).\\
Остается найти напряжение \(U\) в моменты, когда электрическая энергия конденсатора равна магнитной энергии катушки. При этом условии можно записать:

\[
\frac{C U_{m}^{2}}{2}=\frac{C U^{2}}{2}+\frac{L I^{2}}{2}=2 \frac{C U^{2}}{2},
\]

откуда \(|U|=U_{m} / \sqrt{2}\).\\
В результате имеем \(\left|\mathcal{E}_{s}\right|=U_{m} / \sqrt{2}\).\\
11.2. Колебательный контур состоит из катушки с индуктивностью \(L\) и незаряженного конденсатора емкости \(C\). Активное сопротивление контура \(R=0\). Катушка находится в постоянном магнитном поле так, что полный магнитный поток, пронизывающий все ее витки, равен Ф. В момент \(t=0\) магнитное поле резко выключили. Найти ток в контуре как функцию времени \(t\).\\
Решение. При резком выключении внешнего магнитного поля в момент \(t=0\) появится индукционный ток, но конденсатор будет еще не заряженным. Поэтому согласно закону Ома

\[
R I=-\frac{\mathrm{d} \Phi}{\mathrm{~d} t}-L \frac{\mathrm{~d} I}{\mathrm{~d} t} .
\]

В данном случае \(R=0\) и, значит, \(\dot{\Phi}+L \dot{I}=0\). Отсюда \(\Phi=L I_{0}\), где \(I_{0}\) - начальный ток (непосредственно после выключения поля).

После выключения внешнего поля процесс будет описываться уравнением


\begin{equation*}
0=-\frac{q}{C}-L \frac{\mathrm{~d} I}{\mathrm{~d} t} \tag{1}
\end{equation*}


Продифференцировав это уравнение по времени, получим

\[
\ddot{I}+\frac{1}{L C} I=0 .
\]

Это уравнение гармонических колебаний, его решение ищем в виде

\[
I=I_{m} \cos \left(\omega_{0}+\alpha\right) .
\]

Постоянные \(I_{m}\) и \(\alpha\) находим из начальных условий

\[
I(0)=I_{0}, \quad \frac{\mathrm{~d} I}{\mathrm{~d} t}(0)=0
\]

(второе условие следует из уравнения (1), ибо в начальный момент \(t=0\) конденсатор был не заряжен). Из этих условий найдем \(\alpha=0, I_{m}=I_{0}\). В результате

\[
I=I_{0} \cos \omega_{0} t=(\Phi / L) \cos \omega_{0} t
\]

где \(\omega_{0}=1 / \sqrt{L C}\).\\
11.3. В колебательном контуре происходят свободные незатухающие колебания с энергией \(W\). Пластины конденсатора медленно раздвинули так, что частота колебаний увеличилась в \(\eta\) раз. Какую работу совершили при этом против электрических сил?

Решение. Будем исходить из того, что приращение энергии \(W\) колебаний контура происходит за счет работы против электрических сил. Определим эту работу сначала за малый промежуток времени \(\delta t\), в течение которого расстояние между пластинами увеличилось на \(\mathrm{d} h\) :


\begin{equation*}
\mathrm{d} A=\langle F\rangle \mathrm{d} h=\langle q E\rangle \mathrm{d} h, \tag{1}
\end{equation*}


где \(\langle F\rangle\) - среднее за \(\delta t\) значение модуля силы электрического взаимодействия между пластинами. Так как \(E=q / 2 \varepsilon_{0} S\) и, согласно формуле \(C=\varepsilon_{0} S / h, \mathrm{~d} h=\varepsilon_{0} S \mathrm{~d}(1 / C)\), то


\begin{equation*}
\mathrm{d} A=\left\langle q^{2} / 2\right\rangle \mathrm{d}(1 / C) . \tag{2}
\end{equation*}


В данном случае (процесс медленный) возьмем промежуток времени \(\delta t\) таким, чтобы \(T \ll \delta t \ll t\), где \(T\) - период колебаний, \(t\) - вре-

мя всего процесса. При этом условии колебания за время \(\delta t\) можно считать практически гармоническими и \(\left\langle q^{2}\right\rangle=q_{m}^{2} / 2=W C\), поскольку \(W=q_{m}^{2} / 2 C\). С учетом этого перепишем (2) и приравняем полученное выражение к \(\mathrm{d} W\) :


\begin{equation*}
\frac{1}{2} W C \mathrm{~d}\left(\frac{1}{C}\right)=\mathrm{d} W, \quad \text { или } \quad \frac{\mathrm{d} W}{W}=-\frac{\mathrm{d} C}{2 C} . \tag{3}
\end{equation*}


Проинтегрировав последнее уравнение, найдем \(\ln (W \sqrt{C})=\) const. Остается учесть, что \(\omega\) ა \(1 / \sqrt{C}\), и мы получим


\begin{equation*}
W / \omega=\text { const } . \tag{4}
\end{equation*}


Это важное соотношение и справедливо оно только при медленном процессе.

Искомую работу можно представить как приращение колебательной энергии контура: \(A=W^{\prime}-W\). Воспользовавшись формулой (4), получим:

\[
A=W\left(\omega^{\prime} / \omega-1\right)=W(\eta-1) .
\]

11.4. Добротность контура. Колебательный контур с малым затуханием имеет емкость \(C\) и индуктивность \(L\). На поддержание в нем незатухающих гармонических колебаний с амплитудой напряжения на конденсаторе \(U_{m}\) необходимо подводить среднюю мощность \(\langle P\rangle\). Найти добротность контура.

Решение. Вследствие малости затухания воспользуемся формулой (11.23):


\begin{equation*}
Q=2 \pi W / \delta W, \tag{1}
\end{equation*}


где \(W \approx C U_{m}^{2} / 2\) и \(\mathrm{d} W=\langle P\rangle T, T\) - период затухающих колебаний. В нашем случае \(T \approx T_{0}=2 \pi \sqrt{L C}\). После подстановки этих выражений в (1) получим

\[
Q=\frac{U_{m}^{2}}{2\langle P\rangle} \sqrt{\frac{C}{L}} .
\]

11.5. Затухающие колебания. В колебательном контуре имеется конденсатор емкости \(C\), катушка с индуктивностью \(L\), активное сопротивление \(R\) и ключ. При разомкнутом ключе конденсатор зарядили, а затем ключ замкнули. Найти отношение напряжения на конденсаторе к его амплитудному значению в начальный момент (сразу после замыкания ключа).

Решение. Напряжение на конденсаторе будет зависеть от времени так же, как и заряд, поэтому запишем


\begin{equation*}
U=U_{m} \mathrm{e}^{-\beta t} \cos (\omega t+\alpha) . \tag{1}
\end{equation*}


В начальный момент \(t=0\) напряжение \(U(0)=U_{m} \cos \alpha\), где \(U_{m}\) амплитуда в этот момент. Нам надо найти \(U(0) / U_{m}\), т. е. \(\cos \alpha\).

Для этого воспользуемся другим начальным условием: в момент \(t=0\) ток \(I=\dot{q}=0\). Так как \(q=C U\), то достаточно продифференцировать (1) по времени и полученное выражение при \(t=0\) приравнять к нулю. Получим \(-\beta \cos \alpha-\omega \sin \alpha=0\), откуда \(\operatorname{tg} \alpha= -\beta / \omega\). Поэтому искомое отношение


\begin{equation*}
\frac{U(0)}{U_{m}}=\cos \alpha=\frac{1}{\sqrt{1+\operatorname{tg}^{2} \alpha}}=\frac{1}{\sqrt{1+(\beta / \omega)^{2}}} . \tag{2}
\end{equation*}


Величины \(U(0)\) и \(U_{m}\) показаны на рис. 11.8.

Принимая во внимание, что \(\omega^{2}=\omega_{0}^{2}-\beta^{2}\), преобразуем (2) к виду\\
\(U(0) / U_{m}=\sqrt{1-(\beta / \omega)^{2}}=\sqrt{1-R^{2} C / 4 L}\), где учтено, что \(\beta=R / 2 L\) и \(\omega_{0}^{2}=1 / L C\).\\
11.6. В колебательном контуре с емкостью \(C\) и

\begin{figure}[h]
\begin{center}
  \includegraphics[alt={},max width=\textwidth]{3e870b25-6320-483d-9133-cdd5212c2707-308_291_349_760_815}
\captionsetup{labelformat=empty}
\caption{Рис. 11.8}
\end{center}
\end{figure}

Решение. Выберем положительное направление обхода контура по часовой стрелке (рис. 11.9). Согласно закону Ома для участка контура \(1 R L 2\) имеем \(R I=\varphi_{1}-\varphi_{2}+\mathcal{E}_{s}\). В нашем случае \(\mathcal{E}_{s}=-L I\) и \(\quad \varphi_{2}-\varphi_{1}=q / C=U_{C}\), где \(q-\) заряд на обкладке 2 , поэтому первую формулу можно переписать так:

\[
U_{C}=-R I-L \dot{I}
\]

\begin{figure}[h]
\begin{center}
  \includegraphics[alt={},max width=\textwidth]{3e870b25-6320-483d-9133-cdd5212c2707-308_208_349_1332_813}
\captionsetup{labelformat=empty}
\caption{Рис. 11.9}
\end{center}
\end{figure}

После подстановки сюда выражения для \(I(t)\) и его производной получим

\[
U_{C}=\frac{R I_{m} \mathrm{e}^{-\beta t}}{2 \beta}(-\beta \sin \omega t-\omega \cos \omega t) .
\]

Преобразуем выражение в скобках к синусу. Для этого умножим и разделим его на \(\sqrt{\omega^{2}+\beta^{2}}=\omega_{0}\), а затем введем угол \(\delta\) по формулам


\begin{equation*}
-\beta / \omega_{0}=\cos \delta, \quad \omega / \omega_{0}=\sin \delta . \tag{1}
\end{equation*}


Тогда

\[
U_{C}=\frac{R I_{m} \omega_{0}}{2 \beta} \mathrm{e}^{-\beta t} \sin (\omega t-\delta)=I_{m} \sqrt{L / C} \mathrm{e}^{-\beta t} \sin (\omega t-\delta),
\]

где угол \(\delta\) согласно (1) находится во второй четверти, т. е. принимает значения \(\pi / 2<\delta<\pi\). Таким образом, напряжение на конденсаторе отстает по фазе от тока.\\
11.7. Установление колебаний. Катушку с индуктивностью \(L\) и активным сопротивлением \(R\) подключили в момент \(t=0\) к внешнему напряжению \(U=U_{m} \cos \omega t\). Найти ток в цепи как функцию времени \(t\).\\
Решение. В данном случае \(R I=U-L \dot{I}\), или

\[
\dot{I}+(R / L) I=\left(U_{m} / L\right) \cos \omega t .
\]

Решение этого уравнения есть общее решение однородного уравнения плюс частное решение неоднородного:

\[
I(t)=A \mathrm{e}^{-(R / L) t}+\frac{U_{m}}{\sqrt{R^{2}+\omega^{2} L^{2}}} \cos (\omega t-\varphi),
\]

где \(A\) - произвольная постоянная, а угол φ определяется условием (11.36): \(\operatorname{tg}_{\varphi}=\omega L / R\).\\
Постоянную \(A\) находим из начального условия \(I(0)=0\).\\
Отсюда \(A=-\left(U_{m} / \sqrt{R^{2}+\omega^{2} L^{2}}\right) \cos \varphi\). В результате

\[
I(t)=\frac{U_{m}}{\sqrt{R^{2}+\omega^{2} L^{2}}}\left[\cos (\omega t-\varphi)-\mathrm{e}^{-(R / L) t} \cos \varphi\right] .
\]

При достаточно большом \(t\) второе слагаемое в квадратных скобках становится пренебрежимо малым, и мы получаем установившееся решение \(I(t) \sim \cos (\omega t-\varphi)\).\\
11.8. Вынужденные колебания. Участок цепи, состоящий из последовательно соединенных конденсатора и активного сопротивления \(R\), подключили к внешнему переменному напряжению с амплитудой \(U_{m}\). При этом амплитуда установившегося тока оказалась равной \(I_{m}\). Найти разность фаз между током и внешним напряжением.

Решение. В данном случае

\[
U=U_{m} \cos \omega t, \quad I=I_{m} \cos (\omega t-\varphi),
\]

где φ определяется формулой (11.36): \(\operatorname{tg} \varphi=-1 / \omega C R\).\\
Неизвестное значение емкости \(C\) найдем из выражения для амплитуды тока: \(I_{m}=U_{m} / \sqrt{R^{2}+(1 / \omega C)^{2}}\), откуда

\[
C=\frac{1}{\omega \sqrt{\left(U_{m} / I_{m}\right)^{2}-R^{2}}}
\]

После подстановки в выражение для \(\operatorname{tg} \varphi\) получим

\[
\operatorname{tg} \varphi=-\sqrt{\left(U_{m} / R I_{m}\right)^{2}-1} .
\]

В нашем случае \(\varphi<0\), а это значит, что ток опережает по фазе внешнее напряжение (рис. 11.10).

\begin{figure}[h]
\begin{center}
  \includegraphics[alt={},max width=\textwidth]{3e870b25-6320-483d-9133-cdd5212c2707-310_196_270_575_889}
\captionsetup{labelformat=empty}
\caption{Рис. 11.10}
\end{center}
\end{figure}

11.9. Цепь переменного тока, содержащая последовательно соединенные конденсатор и катушку с активным сопротивлением, подключена к внешнему переменному напряжению, частоту которого можно менять, не меняя его амплитуды. При частотах \(\omega_{1}\) и \(\omega_{2}\) амплитуды силы тока в цепи оказались одинаковыми. Найти резонансную частоту тока.\\
Решение. Согласно (11.35) амплитуды будут одинаковыми при условии


\begin{equation*}
\left|\omega_{1} L-\frac{1}{\omega_{1} C}\right|=\left|\omega_{2} L-\frac{1}{\omega_{2} C}\right| . \tag{1}
\end{equation*}


Максимуму резонансной кривой тока соответствует частота, равная собственной частоте \(\omega_{0}=1 / \sqrt{L C}\). Пусть \(\omega_{1}<\omega_{0}<\omega_{2}\) (можно предположить и наоборот, от этого окончательный результат не изменится), тогда равенство (1) можно переписать, сняв модули, так: \(\omega_{0}^{2} / \omega_{1}-\omega_{1}=\omega_{2}-\omega_{0}^{2} / \omega_{2}\), или

\[
\omega_{2}+\omega_{1}=\omega_{0}^{2}\left(\frac{1}{\omega_{1}}+\frac{1}{\omega_{2}}\right) .
\]

После сокращения обеих частей этого равенства на \(\omega_{2}+\omega_{1}\) получим: \(1=\omega_{0}^{2} / \omega_{1} \omega_{2}\), откуда

\[
\omega_{0}=\sqrt{\omega_{1} \omega_{2}} .
\]

11.10. Векторная диаграмма. Цепь, состоящую из последовательно соединенных конденсатора емкости \(C\) и катушки с активным сопротивлением \(R\) и с индуктивностью \(L\), подключили к внешнему гармоническому напряжению с ампли-

\begin{figure}[h]
\begin{center}
  \includegraphics[alt={},max width=\textwidth]{3e870b25-6320-483d-9133-cdd5212c2707-311_313_316_304_166}
\captionsetup{labelformat=empty}
\caption{Рис. 11.11}
\end{center}
\end{figure}

тудой \(U_{m}\) и частотой \(\omega\). Считая, что ток в цепи опережает по фазе внешнее напряжение, построить соответствующую векторную диаграмму и с помощью нее найти амплитуду напряжения на катушке.

Решение. Векторная диаграмма для данного случая имеет вид, показанный на рис. 11.11. Из этой диаграммы сразу видно, что амплитуда напряжения на катушке

\[
U_{L R m}=I_{m} \sqrt{R^{2}+\omega^{2} L^{2}},
\]

где \(I_{m}=U_{m} / \sqrt{R^{2}+(\omega L-1 / \omega C)^{2}}\).\\
Напряжение на катушке при наличии активного сопротивления опережает ток по фазе менее чем на \(\pi / 2\).\\
11.11. Мощность в цепи переменного тока. Цепь, состоящую из последовательно соединенных безындукционного сопротивления \(R\) и катушки с некоторым активным сопротивлением, подключили к сети с действующим напряжением \(U\). Найти тепловую мощность, выделяемую в катушке, если действующие напряжения на сопротивлении \(R\) и катушке равны соответственно \(U_{1}\) и \(U_{2}\).\\
Решение. Воспользуемся векторной диаграммой, которая дана на рис. 11.12. Из этой диаграммы согласно

\begin{figure}[h]
\begin{center}
  \includegraphics[alt={},max width=\textwidth]{3e870b25-6320-483d-9133-cdd5212c2707-311_305_259_1272_175}
\captionsetup{labelformat=empty}
\caption{Рис. 11.12}
\end{center}
\end{figure}

теореме косинусов имеем


\begin{equation*}
U^{2}=U_{1}^{2}+U_{2}^{2}+2 U_{1} U_{2} \cos \varphi_{L} . \tag{1}
\end{equation*}


Мощность же, выделяемая на катушке:


\begin{equation*}
P_{2}=I U_{2} \cos \varphi_{L} \tag{2}
\end{equation*}


где \(I=U_{1} / R\). Из уравнений (1) и (2) получим

\[
P_{2}=\left(U^{2}-U_{1}^{2}-U_{2}^{2}\right) / 2 R .
\]

\section*{1. Единицы величин в СИ и системе Гаусса}
\begin{center}
\begin{tabular}{|l|l|l|l|l|}
\hline
\multirow{2}{*}{Величина} & \multirow{2}{*}{Обозначения} & \multicolumn{2}{|c|}{Единица величины} & \multirow{2}{*}{Отношение ед. СИ ед. СГС} \\
\hline
 &  & СИ & СГС &  \\
\hline
Сила & F & Н & дин & \(10^{5}\) \\
\hline
Работа, энергия & \(A, W\) & Дж & эрг & \(10^{7}\) \\
\hline
Заряд & \(q\) & Кл & ед. СГСЭ & \(3 \cdot 10^{9}\) \\
\hline
Напряженность электрического поля & E & B/м & ед. СГСЭ & \(1 /\left(3 \cdot 10^{4}\right)\) \\
\hline
Потенциал, напряжение & \(\varphi, U\) & B & ед. СГСЭ & 1/300 \\
\hline
Электрический момент & \(p\) & Кл • м & ед. СГСЭ & \(3 \cdot 10^{11}\) \\
\hline
Поляризованность & \(P\) & Кл \(/ \mathrm{M}^{2}\) & ед. СГСЭ & \(3 \cdot 10^{5}\) \\
\hline
Вектор D & D & Кл \(/ \mathrm{M}^{2}\) & ед. СГСЭ & \(12 \pi \cdot 10^{5}\) \\
\hline
Емкость & C & \(\Phi\) & CM & \(9 \cdot 10^{11}\) \\
\hline
Сила тока & I & A & ед. СГСЭ & \(3 \cdot 10^{9}\) \\
\hline
Плотность тока & \(j\) & \(\mathrm{A} / \mathrm{m}^{2}\) & ед. СГСЭ & \(3 \cdot 10^{5}\) \\
\hline
Сопротивление & R & Om & ед. СГСЭ & \(1 /\left(9 \cdot 10^{11}\right)\) \\
\hline
Удельное сопротивление & \(\rho\) & Om•m & ед. СГСЭ & \(1 /\left(9 \cdot 10^{9}\right)\) \\
\hline
Проводимость & Σ & См & ед. СГСЭ & \(9 \cdot 10^{11}\) \\
\hline
Удельная проводимость & \(\sigma\) & См/м & ед. СГСЭ & \(9 \cdot 10^{9}\) \\
\hline
Магнитная индукция & B & Тл & Гс & \(10^{4}\) \\
\hline
Магнитный поток, потокосцепление & \(\Phi, \Psi\) & Вб & Мкс & \(10^{8}\) \\
\hline
Магнитный момент & \(p_{m}\) & \(\mathrm{A} \cdot \mathrm{m}^{2}\) & ед. СГСМ & \(10^{3}\) \\
\hline
Намагниченность & J & A/m & ед. СГСМ & \(10^{-3}\) \\
\hline
Вектор H & H & A/m & Э & \(4 \pi \cdot 10^{-3}\) \\
\hline
Индуктивность & L & Гн & см & \(10^{9}\) \\
\hline
\end{tabular}
\end{center}

\section*{2. Основные формулы электромагнетизма в СИ и системе Гаусса}
\begin{center}
\begin{tabular}{|l|l|l|}
\hline
Наименование & СИ & Гауссова система \\
\hline
Поле \(E\) точечного заряда & \(E=\frac{1}{4 \pi \varepsilon_{0}} \frac{q}{r^{2}}\) & \(E=\frac{q}{r^{2}}\) \\
\hline
Поле Е в плоском конденсаторе и у поверхности проводника & \(E=\frac{\sigma}{\varepsilon_{0} \varepsilon}\) & \(E=\frac{4 \pi \sigma}{\varepsilon}\) \\
\hline
Потенциал поля точечного заряда & \(\varphi=\frac{1}{4 \pi \varepsilon_{0}} \frac{q}{r}\) & \(\varphi=\frac{q}{r}\) \\
\hline
Связь между \(\mathbf{E}\) и φ & \multicolumn{2}{|c|}{\(\mathbf{E}=-\nabla \varphi, \quad \varphi_{1}-\varphi_{2}=\int_{1}^{2} \mathrm{E} \mathrm{dl}\)} \\
\hline
Циркуляция вектора Е в электростатическом поле & \multicolumn{2}{|c|}{\(\oint \mathbf{E} \mathrm{d} \mathbf{l}=0\)} \\
\hline
Электрический момент диполя & \multicolumn{2}{|c|}{\(\mathbf{p}=q \mathbf{l}\)} \\
\hline
Электрический диполь p в поле Е & \multicolumn{2}{|c|}{\(\mathbf{F}=p \frac{\partial \mathbf{E}}{\partial l}, \quad \mathbf{M}=[\mathbf{p E}], \quad W=-\mathbf{p E}\)} \\
\hline
Связь между поляризованностью и напряженностью & \(\mathbf{P}=\varkappa \varepsilon_{0} \mathbf{E}\) &  \\
\hline
\begin{tabular}{l}
Связь между \(\sigma^{\prime}, \mathbf{P}\) и \(\mathbf{E}\) \\
Определение вектора \(\mathbf{D}\) \\
\end{tabular} &  &  \\
\hline
 &  & \(\mathbf{D}=\mathbf{E}+4 \pi \mathbf{P}\) \\
\hline
Связь между ɛ и \(\varkappa\) & \(\varepsilon=1+\varkappa\) & \(\varepsilon=1+4 \pi \varkappa\) \\
\hline
Связь между D и E & \(\mathbf{D}=\varepsilon \varepsilon_{0} \mathbf{E}\) &  \\
\hline
\begin{tabular}{l}
Теорема Гаусса для вектора D \\
Емкость конденсатора \\
\end{tabular} & \(\oint \mathbf{D} \mathrm{d} \mathbf{S}=q\) & \begin{tabular}{l}
\( \mathbf{D}=\varepsilon \mathbf{E} \) \\
\(\oint \mathbf{D} \mathrm{d} \mathbf{S}=4 \pi q\) \\
\end{tabular} \\
\hline
 & \multicolumn{2}{|c|}{\(C=q / U\)} \\
\hline
Емкость плоского конденсатора & \(C=\frac{\varepsilon \varepsilon_{0} S}{h}\) & \(C=\frac{\varepsilon \varepsilon_{0} S}{4 \pi h}\) \\
\hline
 & \multicolumn{2}{|c|}{\(W=\frac{1}{2} \sum q_{i} \varphi_{i}\)} \\
\hline
Полная энергия взаимодействия &  &  \\
\hline
Энергия конденсатора & \multicolumn{2}{|c|}{\(W=q U / 2=C U^{2} / 2=q^{2} / 2 C\)} \\
\hline
Плотность энергии электрического поля & \(w=\mathbf{E D} / 2\) & \(w=\mathbf{E D} / 8 \pi\) \\
\hline
Уравнение непрерывности & \multicolumn{2}{|c|}{\begin{tabular}{l}
\(\int \mathrm{j} \mathrm{dS}=-\frac{\mathrm{d} q}{\mathrm{~d} t}\), \\
\(D \cdot \mathrm{j}=-\frac{\partial \rho}{\partial t}\) \\
\end{tabular}} \\
\hline
\end{tabular}
\end{center}

\begin{table}[h]
\begin{center}
\captionsetup{labelformat=empty}
\caption{Продолжение табл. 2}
\begin{tabular}{|l|l|l|}
\hline
Наименование & СИ & Гауссова система \\
\hline
Закон Ома & \multicolumn{2}{|c|}{\(R I=\varphi_{1}-\varphi_{2}+\mathbf{E}_{12}, \quad \mathbf{j}=\sigma\left(\mathbf{E}+\mathbf{E}^{*}\right)\)} \\
\hline
Закон Джоуля-Ленца & \(\dot{Q}=R I^{2}\), & \(\dot{Q}_{\text {уд }}=\rho j^{2}\) \\
\hline
Сила Лоренца & \(\mathbf{F}=q \mathbf{E}+q[\mathbf{v B}]\) & \(\mathbf{F}=q \mathbf{E}+\frac{q}{c}[\mathbf{v B}]\) \\
\hline
Поле В движущегося заряда & \(\mathbf{B}=\frac{\mu_{0}}{4 \pi} \frac{q[\mathbf{v r}]}{r^{3}}\) & \(\mathbf{B}=\frac{1}{c} \frac{q[\mathbf{v r}]}{r^{3}}\) \\
\hline
Закон Био-Савара & \(\mathrm{d} \mathbf{B}=\frac{\mu_{0}}{4 \pi} \frac{[\mathbf{j r}] \mathrm{d} V}{r^{3}}\) & \(\mathrm{d} \mathbf{B}=\frac{1}{c} \frac{[\mathbf{j r}] \mathrm{d} V}{r^{3}}\) \\
\hline
 & \(\mathrm{d} \mathbf{B}=\frac{\mu_{0}}{4 \pi} \frac{I[\mathrm{~d} \mathbf{l}, \mathbf{r}]}{r^{3}}\) & \(\mathrm{d} \mathbf{B}=\frac{1}{c} \frac{I[\mathrm{~d} \mathbf{l}, \mathbf{r}]}{r^{3}}\) \\
\hline
Поле B: & \(B=\frac{\mu_{0}}{4 \pi} \frac{2 I}{b}\) & \(B=\frac{1}{c} \frac{2 I}{b}\) \\
\hline
б) в центре витка & \(B=\frac{\mu_{0}}{4 \pi} \frac{2 \pi I}{R}\) & \(B=\frac{1}{c} \frac{2 \pi I}{R}\) \\
\hline
в) в соленоиде & \(B=\mu_{0} n I\) & \(B=\frac{4 \pi}{c} n I\) \\
\hline
Закон Ампера & \(\mathbf{F}=I[\mathrm{~d} \mathbf{l}, \mathbf{B}]\) & \(\mathrm{d} \mathbf{F}=\frac{1}{c}[\mathrm{~d} \mathbf{l}, \mathbf{B}]\) \\
\hline
 & \(\mathrm{d} \mathbf{F}=[\mathbf{j B}] \mathrm{d} V\) & \(\mathrm{d} \mathbf{F}=\frac{1}{c}[\mathbf{j B}] \mathrm{d} V\) \\
\hline
Сила взаимодействия параллельных токов & \(F_{\text {ед }}=\frac{\mu_{0}}{4 \pi} \frac{2 I_{1} I_{2}}{b}\) & \(F_{\text {ед }}=\frac{2 I_{1} I_{2}}{b}\) \\
\hline
Магнитный момент контура с током & \(p_{\mathrm{m}}=I S\) & \(p_{\mathrm{m}}=\frac{1}{c} I S\) \\
\hline
Магнитный диполь \(p_{\mathrm{m}}\) в поле В & \(\mathbf{F}=p_{\mathrm{m}} \frac{\partial \mathbf{B}}{\partial n}\), & \(\mathbf{N}=\left[\mathbf{p}_{\mathrm{m}} \mathbf{B}\right]\) \\
\hline
Работа по перемещению контура с током & \(A=I\left(\Phi_{2}-\Phi_{1}\right)\) & \(A=\frac{1}{c} I\left(\Phi_{2}-\Phi_{1}\right)\) \\
\hline
Циркуляция намагниченности & \(\oint \mathbf{J} \mathrm{d} \mathbf{l}=I^{\prime}\) & \(\oint \mathbf{J} \mathrm{d} \mathbf{l}=\frac{1}{c} I^{\prime}\) \\
\hline
Определение вектора H & \(\mathbf{H}=\mathbf{B} / \mu_{0}-\mathbf{J}\) & \(\mathbf{H}=\mathbf{B}-4 \pi \mathbf{J}\) \\
\hline
Циркуляция вектора H в стационарном поле & \(\oint \mathbf{H} \mathrm{d} \mathbf{l}=I\) & \(K_{\mathrm{Hdl}}=\frac{4 \pi}{c} I\) \\
\hline
Связь между \(\mathbf{J}\) и \(\mathbf{H}\) & \multicolumn{2}{|c|}{\(\mathbf{J}=\chi \mathbf{H}\)} \\
\hline
μ и χ & \(\boldsymbol{\mu}=1+\boldsymbol{\chi}\) & \(\mu=1+4 \pi \chi\) \\
\hline
В и H & \(\mathbf{B}=\mu \mu_{0} \mathbf{H}\) & \(\mathbf{B}=\mu \mathbf{H}\) \\
\hline
\end{tabular}
\end{center}
\end{table}

\begin{table}[h]
\begin{center}
\captionsetup{labelformat=empty}
\caption{Продолжение табл. 2}
\begin{tabular}{|l|l|l|}
\hline
Наименование & СИ & Гауссова система \\
\hline
Закон преобразования полей \(\mathbf{E}\) и \(\mathbf{B}\) при \(v_{0} \ll c\) & \( \mathbf{E}^{\prime}=\mathbf{E}+\left[\mathbf{v}_{0} \mathbf{B}\right] \) \( \mathbf{B}^{\prime}=\mathbf{B}-\frac{1}{c^{2}}\left[\mathbf{v}_{0} \mathbf{E}\right] \) & \( \begin{aligned} \mathbf{E}^{\prime} & =\mathbf{E}+\frac{1}{c}\left[\mathbf{v}_{0} \mathbf{B}\right] \\ \mathbf{B}^{\prime} & =\mathbf{B}-\frac{1}{c}\left[\mathbf{v}_{0} \mathbf{E}\right] \end{aligned} \) \\
\hline
 & \multicolumn{2}{|c|}{\(\mathbf{E B}=\mathrm{inv}\)} \\
\hline
Инварианты электромагнитного поля & \(E^{2}-c^{2} B^{2}=\mathrm{inv}\) &  \\
\hline
Э.д.с. индукции & \(\mathcal{E}_{i}=-\frac{\mathrm{d} \Phi}{\mathrm{d} t}\) & \( \begin{gathered} E^{2}-B^{2}=\operatorname{inv} \\ \mathcal{E}_{i}=-\frac{1}{c} \frac{\mathrm{~d} \Phi}{\mathrm{~d} t} \end{gathered} \) \\
\hline
Индуктивность & \(L=\Phi / I\) & \(=c \Phi / I\) \\
\hline
Индуктивность соленоида & \(L=\mu \mu_{0} n^{2} V\) & \(L=c \Phi / I\) \(L=4 \pi \mu n^{2} V\) \\
\hline
 & \(\mathcal{E}_{s}=-L \frac{\mathrm{~d} I}{\mathrm{~d} t}\) & \(\mathcal{E}_{s}=-\frac{1}{c^{2}} L \frac{\mathrm{~d} I}{\mathrm{~d} t}\) \\
\hline
Энергия магнитного поля тока & \(W=\frac{L I^{2}}{2}\) & \(\begin{aligned} W & =\frac{1}{c} \frac{L I^{2}}{2} \\ w & =\frac{\mathbf{B H}}{8 \pi} \\ j_{\mathrm{cm}} & =\frac{1}{4 \pi} \frac{\partial \mathbf{D}}{\partial t}\end{aligned}\) \\
\hline
Плотность энергии магнитного поля & \(w=\frac{\mathbf{B H}}{2}\) &  \\
\hline
Плотность тока смещения & \(\mathbf{j}_{\mathrm{cM}}=\frac{\partial \mathbf{D}}{\partial t}\) & см \(-4 \pi \partial t\) \\
\hline
Уравнения Максвелла в интегральной форме &  & \(\oint \mathbf{E} \mathrm{d} \mathbf{l}=-\frac{1}{c} \int \dot{\mathbf{B}} \mathrm{~d} \mathbf{S}\) \\
\hline
 & \( \begin{gathered} \oint \mathbf{D} \mathrm{d} \mathbf{S}=\int \rho \mathrm{d} V \\ \oint \mathbf{H} \mathrm{~d} \mathbf{l}=\int(\mathbf{j}+\dot{\mathbf{D}}) \mathrm{dS} \end{gathered} \) & \(\oint \mathbf{D} \mathrm{d} \mathbf{S}=4 \pi \int \rho \mathrm{~d} V\) \\
\hline
 &  & \( \begin{gathered} \oint \mathbf{H} \mathrm{d} \mathbf{l}= \\ =\frac{4 \pi}{c} \int\left(\mathbf{j}+\frac{\dot{\mathbf{D}}}{4 \pi}\right) \mathrm{d} \mathbf{S} \end{gathered} \) \\
\hline
 & \(\oint \mathbf{B} \mathrm{d} \mathbf{S}=0\) & \(\oint \mathbf{B} \mathrm{d} \mathbf{S}=0\) \\
\hline
Уравнения Максвелла & \(\boldsymbol{\nabla} \times \mathbf{E}=-\dot{\mathbf{B}}\) & \(\nabla \times \mathbf{E}=-\frac{1}{c} \dot{\mathbf{B}}\) \\
\hline
 & \(\boldsymbol{\nabla} \cdot \mathbf{D}=\rho\) & \(\boldsymbol{\nabla} \cdot \mathbf{D}=4 \pi \rho\) \\
\hline
 & \(\boldsymbol{\nabla} \times \mathbf{H}=\mathbf{j}+\dot{\mathbf{D}}\) & \(\nabla \times \mathbf{H}=\frac{4 \pi}{c}\left(\mathbf{j}+\frac{\dot{\mathbf{D}}}{4 \pi}\right)\) \\
\hline
 & \(\boldsymbol{\nabla} \cdot \mathbf{B}=0\) & \(\nabla \cdot \mathbf{B}=0\) \\
\hline
Связь между \(E\) и \(H\) в электромагнитной волне & \(E \sqrt{\varepsilon_{0} \varepsilon}=H \sqrt{\mu_{0} \mu}\) & \(E \sqrt{\varepsilon}=H \sqrt{\mu}\) \\
\hline
Вектор Пойнтинга & \(\boldsymbol{\Pi}=[\mathbf{E H}]\) & \(\boldsymbol{\Pi}=\frac{c}{4 \pi}[\mathbf{E H}]\) \\
\hline
Плотность импульса электромагнитного поля & \(\mathbf{G}=\frac{1}{c^{2}}[\mathbf{E H}]\) & \(\mathbf{G}=\frac{1}{4 \pi c}[\mathbf{E H}]\) \\
\hline
\end{tabular}
\end{center}
\end{table}

\section*{3. Основные величины и единицы СИ}
Время \(t\) - величина, характеризующая последовательную смену явлений и состояний материи, характеризующая длительность их бытия; единица - секунда (c).

Секунда равна 9192631770 периодам излучения, соответствующего переходу между двумя сверхтонкими уровнями основного состояния атома цезия-133.\\
Длина \(l\) - величина, характеризующая протяженность, удаленность и перемещение тел или их частей вдоль заданной линии; единица метр (м).\\
\(М е т p\) есть длина пути, проходимого светом в вакууме за интервал времени \(1 / 299792458\) с.\\
Масса \(m\) - величина, определяющая инертные и гравитационные свойства материальных объектов; единица - килограмм (кг).

Килограмм равен массе платино-иридиевого эталона, хранящегося в Международном бюро мер и весов (в Севре, близ Парижа). Масса эталона близка к массе 1 дм \(^{3}\) чистой воды при \(4{ }^{\circ}\) С.

Сила электрического тока \(I\) - скалярная величина, численно равная электрическому заряду, переносимому сквозь рассматриваемую поверхность за единицу времени; единица - ампер (А).

Ампер равен силе постоянного тока, который при прохождении по двум параллельным проводникам бесконечной длины и ничтожно малой площади кругового поперечного сечения, расположенным в вакууме на расстоянии 1 м один от другого, вызвал бы на каждом участке проводника длиной \(1 м\) силу взаимодействия, равную \(2 \cdot 10^{-7} \mathrm{H}\).

Термодинамическая температура \(T\) - температура, отсчитываемая по термодинамической шкале температур от абсолютного нуля; единица - кельвин (К).

Кельвин равен \(1 / 273,16\) части термодинамической температуры тройной точки воды.\\
Количество вещества \(n\) - величина, равная числу структурных элементов, содержащихся в теле (системе тел); единица - моль.

Моль равен количеству вещества системы, содержащей столько же структурных элементов, сколько содержится атомов в углеро-де-12 массой 0,012 кг. При применении моля структурные элементы должны быть специфицированы и могут быть атомами, молекулами, электронами и другими частицами или специфицированными группами частиц.

\section*{4. Греческий алфавит}
\begin{center}
\begin{tabular}{|l|l|l|}
\hline
А, \(\alpha\) - альфа & I, ι - йота & P, \(\rho\) - po \\
\hline
B, \(\beta\) - бета & \(\mathrm{K}, x\) - каппа & Σ, \(\sigma\) - сигма \\
\hline
Г, \(\gamma\) - гамма & \(\Lambda, \lambda\) - ламбда & T, \(\tau\) - тау \\
\hline
\(\Delta, \delta\) - дельта & М, μ - мю & Y, υ - ипсилон \\
\hline
Ε, \(ε\) - эпсилон & \(\mathrm{N}, \boldsymbol{v}\) - ню & \(\Phi, φ\) - фи \\
\hline
\(\mathrm{Z}, \zeta\) - дзета & \(\Xi, \xi\) - кси & \(\mathrm{X}, \chi\) - хи \\
\hline
H, η - эта & О, о - омикрон & \(\Psi, ψ\) - пси \\
\hline
\(\Theta, \theta, \vartheta\) - тета & П, \(\pi\) - пи & \(\Omega, \omega\) - омега \\
\hline
\end{tabular}
\end{center}

\section*{5. Некоторые физические константы}
\begin{center}
\begin{tabular}{|l|l|}
\hline
Скорость света в вакууме & \( \begin{aligned} & c=2,998 \cdot 10^{8} \mathrm{M} / \mathrm{c} \\ & G=\left\{\begin{array}{l} 6,67 \cdot 10^{-11} \mathrm{M}^{3} /\left(\mathrm{K} \cdot \mathrm{c}^{2}\right) \\ 6,67 \cdot 10^{-8} \mathrm{~cm}^{3} /\left(\Gamma \cdot \mathrm{c}^{2}\right) \end{array}\right. \end{aligned} \) \\
\hline
Ускорение свободного падения & \(g=9,807 \mathrm{~m} / \mathrm{c}^{2}\) \\
\hline
Постоянная Авогадро & \(N_{A}=6,022 \cdot 10^{23}\) моль \(^{-1}\) \\
\hline
Элементарный заряд & \(e=\left\{\begin{array}{l}1,602 \cdot 10^{-19} \text { Кл } \\ 4,803 \cdot 10^{-10} \text { СГСЭ }\end{array}\right.\) \\
\hline
Масса электрона & \(m_{e}=0,911 \cdot 10^{-30}\) кг \\
\hline
Удельный заряд электрона & \(e / m_{e}=\left\{\begin{array}{l}1,76 \cdot 10^{11} \text { Кл } / \text { кг } \\ 5,27 \cdot 10^{17} \text { СГСЭ } / \text { г }\end{array}\right.\) \\
\hline
Масса протона & \(m_{p}=1,672 \cdot 10^{-27}\) кг \\
\hline
Электрическая постоянная & \( \begin{aligned} \varepsilon_{0}= & 0,885 \cdot 10^{-11} \Phi / \mathrm{M} \\ & 1 / 4 \pi \varepsilon_{0}=9 \cdot 10^{9} \mathrm{M} / \Phi \end{aligned} \) \\
\hline
Магнитная постоянная & \( \begin{aligned} & \mu_{0}=1,257 \cdot 10^{-6} \Gamma \mathrm{H} / \mathrm{M} \\ & \mu_{0} / 4 \pi=10^{-7} \Gamma \mathrm{H} / \mathrm{M} \end{aligned} \) \\
\hline
Связь между скоростью света и постоянными \(\varepsilon_{0}\) и \(\mu_{0}\) & \(c=1 / \sqrt{\varepsilon_{0} \mu_{0}}\) \\
\hline
\end{tabular}
\end{center}

\section*{Предметный указатель}
Ампер 119\\
Ампер на метр 184\\
Амплитуда колебаний затухающих 293

Бетатрон 230\\
Вебер 226\\
Вектор D 77\\
-H183

\begin{itemize}
  \item Пойнтинга 276
\end{itemize}

Взаимодействие зарядов движущихся 147, 223\\
Волны электромагнитные 273\\
Вольт 27\\
Вольт на метр 10\\
Восприимчивость диэлектрическая 72

\begin{itemize}
  \item магнитная 184
\end{itemize}

Время релаксации 137, 294\\
Генри 234\\
Гистерезис 72

\begin{itemize}
  \item магнитный 194
\end{itemize}

Градиент потенциала 30\\
Давление волны электромагнитной 278, 280

\begin{itemize}
  \item магнитное 252
\end{itemize}

Движение заряда в скрещенных ЕиВ полях 218\\
Декремент затухания логарифмический 294\\
Диаграмма векторная 298\\
Диамагнетики 184\\
Дивергенция поля В 157\\
--E 23, 24\\
Диполь электрический 33\\
Диэлектрики 68\\
Добротность контура 294\\
Домены 196\\
Емкость 57

\begin{itemize}
  \item конденсатора 59
  \item 
  \begin{itemize}
    \item плоского 60
  \end{itemize}
  \item 
  \begin{itemize}
    \item сферического 60
  \end{itemize}
  \item 
  \begin{itemize}
    \item цилиндрического 60
  \end{itemize}
  \item проводника уединенного 57
\end{itemize}

Закон Ампера 159, 160

\begin{itemize}
  \item Био- Савара 149
  \item Джоуля- Ленца 133
  \item индукции электромагнитной 226
  \item Кулона 10
  \item Ома 122, 123
  \item 
  \begin{itemize}
    \item для неоднородного участка цепи 127
  \end{itemize}
  \item 
  \begin{itemize}
    \item обобщенный 125, 127
  \end{itemize}
  \item сохранения электрического заряда 9, 121\\
Законы преобразования полей 207, 208\\
Зарядка конденсатора 137\\
Заряды индуцированные 468
  \item связанные 70
  \item свободные 70
  \item сторонние 70
\end{itemize}

Защита магнитная 187

\begin{itemize}
  \item электростатическая 51
\end{itemize}

Импеданс 302\\
Импульс поля 279\\
Инвариантность заряда 9, 205\\
Инварианты поля электромагнитного 214\\
Индуктивность 234

\begin{itemize}
  \item взаимная 239
\end{itemize}

Индукция взаимная 241

\begin{itemize}
  \item магнитная 147
  \item остаточная 195
  \item электростатическая 46
  \item электрическая 77
\end{itemize}

Инерция электромагнитная 225\\
Источники поля \(\mathbf{E}\) и \(\mathbf{D} 24,78\)

Колебания вынужденные 296

\begin{itemize}
  \item затухающие 292
  \item свободные 291
  \item установившиеся 297
\end{itemize}

Конденсатор 58\\
Контур колебательный 289\\
Кривая намагничения основная 194\\
Кривые резонансные 299\\
Коэффициент затухания 291, 294

\begin{itemize}
  \item мощности 303
  \item отражения 280
\end{itemize}

Кулон 10

\begin{itemize}
  \item на квадратный метр 72
\end{itemize}

Лапласиан 54\\
Магнетик 177\\
Макрополе 45\\
Метод изображений 55,56\\
Микрополе 45\\
Момент диполя электрический 34

\begin{itemize}
  \item магнитный 161
  \item сил, действующих на диполь 37
  \item 
  \begin{itemize}
    \item 
    \begin{itemize}
      \item 
      \begin{itemize}
        \item контур с током 163
      \end{itemize}
    \end{itemize}
  \end{itemize}
\end{itemize}

Мощность тока 135

\begin{itemize}
  \item 
  \begin{itemize}
    \item переменного 303
  \end{itemize}
  \item 
  \begin{itemize}
    \item постоянного 135
  \end{itemize}
  \item 
  \begin{itemize}
    \item удельная тепловая 134
  \end{itemize}
\end{itemize}

Намагниченность 178

\begin{itemize}
  \item остаточная 195
  \item спонтанная 196
\end{itemize}

Напряжение действующее 303\\
Напряженность поля магнитного 184

\begin{itemize}
  \item 
  \begin{itemize}
    \item сил сторонних 125
  \end{itemize}
  \item 
  \begin{itemize}
    \item электрического 10
  \end{itemize}
\end{itemize}

Объем физически бесконечно малый 45\\
Оператор Лапласа 54

\begin{itemize}
  \item набла 24
\end{itemize}

Парамагнетики 184\\
Период колебаний свободных незатухающих 292

\begin{itemize}
  \item 
  \begin{itemize}
    \item затухающих 293
  \end{itemize}
\end{itemize}

Петля гистерезиса 195\\
Плотность заряда 11

\begin{itemize}
  \item импульса поля электромагнитного 297
  \item потока энергии 274
  \item силы поверхностной (электрической) 50, 112
  \item тока 120
  \item энергии магнитной 245
  \item 
  \begin{itemize}
    \item поля электромагнитного 275
  \end{itemize}
  \item 
  \begin{itemize}
    \item электрической 103
  \end{itemize}
\end{itemize}

Поверхность эквипотенциальная 31\\
Поле вихревое 153, 229

\begin{itemize}
  \item диполя 33,34
  \item заряда релятивистского 213
  \item магнитное 145
  \item 
  \begin{itemize}
    \item на оси тока кругового 150
  \end{itemize}
  \item 
  \begin{itemize}
    \item плоскости с током 156
  \end{itemize}
  \item 
  \begin{itemize}
    \item соленоида 155
  \end{itemize}
\end{itemize}

Поле магнитное тока прямого 150, 154

\begin{itemize}
  \item 
  \begin{itemize}
    \item тороида 155
  \end{itemize}
  \item потенциальное 26,159
  \item соленоидальное 15,159
  \item электрическое 9 , 10
  \item электромагнитное 204
\end{itemize}

Поляризация диэлектрика 68\\
Поляризованность 71\\
Постоянная времени 237

\begin{itemize}
  \item магнитная 147
  \item электрическая 10
  \item электродинамическая 147
\end{itemize}

Потенциал 27\\
Поток вектора 14

\begin{itemize}
  \item магнитный полный 276
\end{itemize}

Потокосцепление 226\\
Правила Кирхгофа 129\\
Правило Ленца 225\\
Преломление линий В и Н 189

\begin{itemize}
  \item Е и D 82, 83
\end{itemize}

Принцип суперпозиции 11, 148\\
Природа магнетизма 209\\
Проницаемость диэлектрическая 78

\begin{itemize}
  \item магнитная 185
\end{itemize}

Процессы переходные 135

Работа поля при поляризации диэлектрика 105, 106\\
Разряд конденсатора апериодический 295\\
Разрядка конденсатора 136\\
Распределение зарядов 11\\
Резонанс 301\\
Ротор поля векторного 158,159

Самоиндукция 234\\
Связь между потенциалом и напряженностью поля 30,31\\
Сегнетоэлектрики 72\\
Сила Ампера 160

\begin{itemize}
  \item взаимодействия токов параллельных 160\\
Сила, действующая на диполь 35
  \item 
  \begin{itemize}
    \item 
    \begin{itemize}
      \item контур с током 161
    \end{itemize}
  \end{itemize}
  \item коэрцитивная 195
\end{itemize}

Сила Лоренца 145, 270

\begin{itemize}
  \item магнитная 145
  \item тока 119
  \item электродвижущая 126
\end{itemize}

Силы пондеромоторные 107

\begin{itemize}
  \item при наличии диэлектрика 107
  \item сторонние 125
\end{itemize}

Сименс 123

\begin{itemize}
  \item на метр 132
\end{itemize}

Смещение (индукция) D 77\\
Соленоид 155\\
Сопротивление активное 289 , 304

\begin{itemize}
  \item емкостное 302
  \item индуктивное 302
  \item критическое 295
  \item полное (импеданс) 302
  \item реактивное 302
  \item электрическое 122
  \item 
  \begin{itemize}
    \item удельное 122
  \end{itemize}
\end{itemize}

Стоки поля \(\mathbf{E}\) и \(\mathbf{D} 24,78\)\\
Температура Кюри 196\\
Теорема взаимности 239

\begin{itemize}
  \item Гаусса для поля В 177\\
----D 77\\
---- E 15, 24\\
----P 72
  \item единственности 54
  \item о циркуляции вектора E 25, 229, 231
  \item 
  \begin{itemize}
    \item 
    \begin{itemize}
      \item 
      \begin{itemize}
        \item H 183, 265
      \end{itemize}
    \end{itemize}
  \end{itemize}
  \item 
  \begin{itemize}
    \item 
    \begin{itemize}
      \item 
      \begin{itemize}
        \item J 180
      \end{itemize}
    \end{itemize}
  \end{itemize}
  \item Пойнтинга 275
\end{itemize}

Тесла 147\\
Ток действующий 303\\
Токи молекулярные 179

\begin{itemize}
  \item намагничивания 179
  \item Фуко 225
\end{itemize}

Ток индукционный 224

\begin{itemize}
  \item квазистационарный 136, 288
\end{itemize}

Ток переменный 301

\begin{itemize}
  \item полный 264
  \item поляризации 266
  \item проводимости 119,264
  \item смещения 264
  \item электрический 119
\end{itemize}

Точка Кюри 196\\
Уравнение контура колебательного 290

\begin{itemize}
  \item Лапласа 54
  \item непрерывности 121
  \item Пуассона 54
\end{itemize}

Уравнения Максвелла дифференциальные 269

\begin{itemize}
  \item 
  \begin{itemize}
    \item интегральные 267
  \end{itemize}
  \item материальные 270,271
\end{itemize}

Условие бетатронное 255

\begin{itemize}
  \item квазистационарности 288
  \item стационарности 121
\end{itemize}

Условия граничные 270

\begin{itemize}
  \item 
  \begin{itemize}
    \item для вектора В 187
  \end{itemize}
  \item 
  \begin{itemize}
    \item 
    \begin{itemize}
      \item 
      \begin{itemize}
        \item D 81, 82\\
----E 81
      \end{itemize}
    \end{itemize}
  \end{itemize}
  \item 
  \begin{itemize}
    \item 
    \begin{itemize}
      \item 
      \begin{itemize}
        \item H 188
      \end{itemize}
    \end{itemize}
  \end{itemize}
  \item 
  \begin{itemize}
    \item 
    \begin{itemize}
      \item 
      \begin{itemize}
        \item P 75, 76
      \end{itemize}
    \end{itemize}
  \end{itemize}
\end{itemize}

Фарад 58\\
Ферромагнетики 185, 193\\
Формула Томсона 292\\
Циркуляция вектора E 25, 229

\begin{itemize}
  \item 
  \begin{itemize}
    \item H 183, 265
  \end{itemize}
  \item 
  \begin{itemize}
    \item J 180
  \end{itemize}
\end{itemize}

Частота контура собственная 291\\
Э.д.с. индукции 224

\begin{itemize}
  \item самоиндукции 235
\end{itemize}

Электреты 69\\
Электроемкость 57\\
Электропроводимость удельная 123\\
Электрострикция 107, 108\\
Элемент тока 149\\
Энергия взаимная 247

\begin{itemize}
  \item взаимодействия зарядов 97,98
  \item 
  \begin{itemize}
    \item полная 99
  \end{itemize}
  \item диполя в поле 37,38
  \item конденсатора 101
  \item поля магнитного 244,245
  \item 
  \begin{itemize}
    \item электрического 102,103
  \end{itemize}
  \item проводника уединенного 100
  \item собственная 100, 244, 247
  \item тока (магнитная) 244
\end{itemize}

Учебное электронное издание



Минимальные системные требования определяются соответствующими требованиями программы Adobe Reader версии не ниже 10-й для операционных систем Windows, Android, iOS, Windows Phone и BlackBerry

\begin{itemize}
  \item 
\end{itemize}

\begin{itemize}
  \item 
\end{itemize}

\begin{itemize}
  \item 
\end{itemize}

\begin{itemize}
  \item 
\end{itemize}

\begin{itemize}
  \item 
\end{itemize}

\begin{itemize}
  \item 
\end{itemize}

\begin{itemize}
  \item 
\end{itemize}

\begin{itemize}
  \item 
\end{itemize}


\end{document}