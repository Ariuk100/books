% This LaTeX document needs to be compiled with XeLaTeX.
\documentclass[10pt]{article}
\usepackage[utf8]{inputenc}
\usepackage{ucharclasses}
\usepackage{graphicx}
\usepackage[export]{adjustbox}
\graphicspath{ {./images/} }
\usepackage{amsmath}
\usepackage{amsfonts}
\usepackage{amssymb}
\usepackage[version=4]{mhchem}
\usepackage{stmaryrd}
\usepackage{caption}
\usepackage{mathrsfs}
\usepackage{multirow}
\usepackage{hyperref}
\hypersetup{colorlinks=true, linkcolor=blue, filecolor=magenta, urlcolor=cyan,}
\urlstyle{same}
\usepackage[fallback]{xeCJK}
\usepackage{polyglossia}
\usepackage{fontspec}
\usepackage{newunicodechar}
\IfFontExistsTF{Noto Serif CJK TC}
{\setCJKmainfont{Noto Serif CJK TC}}
{\IfFontExistsTF{STSong}
  {\setCJKmainfont{STSong}}
  {\IfFontExistsTF{Droid Sans Fallback}
    {\setCJKmainfont{Droid Sans Fallback}}
    {\setCJKmainfont{SimSun}}
}}

\setmainlanguage{russian}
\setotherlanguages{belarusian, thai}
\IfFontExistsTF{Noto Serif Thai}
{\newfontfamily\thaifont{Noto Serif Thai}}
{\IfFontExistsTF{Thonburi}
  {\newfontfamily\thaifont{Thonburi}}
  {\IfFontExistsTF{FreeSerif}
    {\newfontfamily\thaifont{FreeSerif}}
    {\IfFontExistsTF{Tahoma}
      {\newfontfamily\thaifont{Tahoma}}
      {\newfontfamily\thaifont{Arial Unicode MS}}
}}}
\IfFontExistsTF{CMU Serif}
{\newfontfamily\lgcfont{CMU Serif}}
{\IfFontExistsTF{DejaVu Sans}
  {\newfontfamily\lgcfont{DejaVu Sans}}
  {\newfontfamily\lgcfont{Georgia}}
}
\setDefaultTransitions{\lgcfont}{}
\newfontfamily\cyrillicfont{CMU Serif}
\setDefaultTransitions{\cyrillicfont}{}
\setTransitionsFor{Thai}{\thaifont}{\lgcfont}

\title{ФИЗИКА МАКРОСИСТЕМ ОСНОВНЫе законы }

\author{И. Иродов}
\date{}


%New command to display footnote whose markers will always be hidden
\let\svthefootnote\thefootnote
\newcommand\blfootnotetext[1]{%
  \let\thefootnote\relax\footnote{#1}%
  \addtocounter{footnote}{-1}%
  \let\thefootnote\svthefootnote%
}

%Overriding the \footnotetext command to hide the marker if its value is `0`
\let\svfootnotetext\footnotetext
\renewcommand\footnotetext[2][?]{%
  \if\relax#1\relax%
    \ifnum\value{footnote}=0\blfootnotetext{#2}\else\svfootnotetext{#2}\fi%
  \else%
    \if?#1\ifnum\value{footnote}=0\blfootnotetext{#2}\else\svfootnotetext{#2}\fi%
    \else\svfootnotetext[#1]{#2}\fi%
  \fi
}

\newunicodechar{υ}{\ifmmode\upsilon\else{$\upsilon$}\fi}
\newunicodechar{𝜄}{\ifmmode\iota\else{$\iota$}\fi}
\newunicodechar{φ}{\ifmmode\varphi\else{$\varphi$}\fi}
\newunicodechar{𝜂}{\ifmmode\eta\else{$\eta$}\fi}
\newunicodechar{□}{\ifmmode\square\else{$\square$}\fi}
\newunicodechar{ε}{\ifmmode\varepsilon\else{$\varepsilon$}\fi}
\newunicodechar{σ}{\ifmmode\sigma\else{$\sigma$}\fi}
\newunicodechar{𝜆}{\ifmmode\lambda\else{$\lambda$}\fi}
\newunicodechar{Θ}{\ifmmode\Theta\else{$\Theta$}\fi}
\newunicodechar{⋅}{\ifmmode\cdot\else{$\cdot$}\fi}
\newunicodechar{ω}{\ifmmode\omega\else{$\omega$}\fi}
\newunicodechar{π}{\ifmmode\pi\else{$\pi$}\fi}
\newunicodechar{𝜇}{\ifmmode\mu\else{$\mu$}\fi}
\newunicodechar{ψ}{\ifmmode\psi\else{$\psi$}\fi}
\newunicodechar{α}{\ifmmode\alpha\else{$\alpha$}\fi}

\begin{document}
\maketitle
\captionsetup{singlelinecheck=false}
6-е издание (электронное)\\
\includegraphics[max width=\textwidth, alt={}, center]{bd023a69-cc46-4b6f-a48e-cc548fa9259d-001_144_101_1553_566}

Москва\\
БИНОМ. Лаборатория знаний\\
2015

И83 Физика макросистем. Основные законы [Электронный ресурс] : учебное пособие / И. Е. Иродов. - 6-е изд. (эл.). - Электрон. текстовые дан. (1 файл pdf : 210 с.). М. : БИНОМ. Лаборатория знаний, 2015. - (Технический университет. Общая физика). - Систем. требования: Adobe Reader XI ; экран \(10^{\prime \prime}\).

ISBN 978-5-9963-2589-4\\
Учебное пособие содержит теоретический материал, связанный с методами изучения свойств и поведения макросистем систем, состоящих из очень большого числа частиц. Это термодинамика, молекулярно-кинетическая теория и статистика (как классическая, так и квантовая). Помимо довольно большого числа примеров, в конце каждой главы приведены задачи на соответствующий материал. Показано, как, по мнению автора, следует подходить к их решению. Задачи тесно связаны с основным текстом, часто являясь его дополнением и развитием.

Для студентов физических специальностей вузов.\\
УДК 536.7+531.19\\
ББК 22.317

Деривативное электронное издание на основе печатного аналога: Физика макросистем. Основные законы : учебное пособие / И. Е. Иродов. - 4-е изд. - М. : БИНОМ. Лаборатория знаний, 2010. - 207 с. : ил. - (Технический университет. Общая физика). - ISBN 978-5-9963-0284-0.

В соответствии со ст. 1299 и 1301 ГК РФ при устранении ограничений, установленных техническими средствами защиты авторских прав, правообладатель вправе требовать от нарушителя возмещения убытков или выплаты компенсации

\section*{Содержание}
Предисловие ..... 5\\
Принятые обозначения ..... 6\\
Введение ..... 7\\
Глава 1. Первое начало термодинамики ..... 9\\
§ 1.1. Состояние системы. Процессы ..... 9\\
§ 1.2. Первое начало термодинамики ..... 11\\
§ 1.3. Теплоемкость идеального газа ..... 15\\
§ 1.4. Политропические процессы ..... 18\\
§ 1.5. Молекулярно-кинетическая теория ..... 21\\
§ 1.6. Гипотеза о равнораспределении энергии по сте- пеням свободы ..... 25\\
§ 1.7. Газ Ван-дер-Ваальса ..... 30\\
Задачи ..... 35\\
Глава 2. Статистическая физика. Распределения Максвелла и Больцмана ..... 43\\
§ 2.1. Вероятность. Средние значения ..... 43\\
§ 2.2. Распределение Максвелла ..... 49\\
§ 2.3. Опытная проверка распределения Максвелла ..... 57\\
§ 2.4. Распределение Больцмана ..... 58\\
Задачи ..... 68\\
Глава 3. Второе начало термодинамики. Энтропия ..... 77\\
§ 3.1. Второе начало термодинамики ..... 77\\
§ 3.2. Энтропия ..... 79\\
§ 3.3. О вычислении и применении энтропии ..... 82\\
§ 3.4. Статистический смысл второго начала термодина- мики ..... 88\\
§ 3.5. Энтропия и вероятность ..... 91\\
§ 3.6. Термодинамические соотношения ..... 96\\
Задачи ..... 99\\
Глава 4. Квантовые статистики и их применения ..... 106\\
§ 4.1. Квантовые статистики ..... 106\\
§ 4.2. Распределение Ферми-Дирака для электронов в металлах ..... 110\\
§ 4.3. О зонной теории. Электропроводность ..... 115\\
§ 4.4. Распределение Бозе-Эйнштейна для фотонного газа ..... 119\\
§ 4.5. Теплоемкость твердого тела ..... 124\\
Задачи ..... 131\\
Глава 5. Состояния вещества ..... 138\\
§ 5.1. Изотермы Ван-дер-Ваальса ..... 138\\
§ 5.2. Фазовые переходы ..... 141\\
§ 5.3. Жидкое состояние ..... 144\\
§ 5.4. Кристаллическое состояние ..... 152\\
§ 5.5. Плазма ..... 158\\
Задачи ..... 161\\
Глава 6. Неравновесные макросистемы ..... 169\\
§ 6.1. Инверсная среда. Лазеры ..... 169\\
§ 6.2. Явления переноса ..... 174\\
§ 6.3. Молекулярно-кинетическая интерпретация яв- лений переноса ..... 180\\
Задачи ..... 188\\
Приложения ..... 196

\begin{enumerate}
  \item Единицы физических величин в СИ и СГС ..... 196
  \item Десятичные приставки к названиям единиц ..... 197
  \item Греческий алфавит ..... 197
  \item Плотности веществ ..... 197
  \item Постоянные газов ..... 198
  \item Постоянные жидкостей ..... 198
  \item Постоянные твердых тел ..... 198
  \item Фундаментальные физические константы ..... 199\\
Предметный указатель ..... 200
\end{enumerate}

\section*{Предисловие}
Данное учебное пособие завершает новый в структурном и содержательном отношении курс общей физики, над которым автор работал несколько лет. Принятое в нем расположение материала (механика, электромагнетизм, волновые процессы, квантовая физика и физика макросистем) является по глубокому убеждению автора наиболее логически последовательным и оправданным в методическом отношении. Такая структура курса позволяет студентам легче и основательнее познакомиться с материалом и по достоинству оценить важность, глубину и универсальность основных идей курса физики, который по праву считается фундаментальным. Именно такая структура курса соответствует концепции современной программы по физике Министерства образования РФ.

Форма изложения материала та же, что и в предыдущих книгах: сначала даются основы теории, а затем разбор ряда задач, где показывается как следует подходить к их решению. Задачи тесно связаны с основным текстом, часто являются его дополнением и развитием, поэтому работа над ними должна проводиться параллельно с изучением основного материала. Кроме того, предлагаемый набор задач должен, по замыслу автора, дать возможность студенту дополнительно обдумать ряд важных вопросов и помочь представить (даже если многие задачи не решать, а просто прочитать их условия) большой диапазон приложения изучаемых идей.

Курсивом выделены важнейшие положения и термины. Петит используется для материала повышенной трудности (этот материал при первом чтении можно безболезненно опустить), а также для примеров и задач.

В настоящем издании сделаны некоторые дополнения и уточнения, устранены замеченные опечатки.

Книга как учебное пособие рассчитана на студентов вузов с расширенной программой по физике. Она может быть полезной и преподавателям вузов.



\section*{Принятые обозначения}
Векторы обозначены полужирным прямым шрифтом ( \(\mathbf{v}, \mathbf{F}\) ). Та же буква светлым шрифтом и курсивом ( \(v, F\) ) означает модуль соответствующего вектора.

Средние величины отмечены угловыми скобками <>, например, \(\langle v\rangle,\langle x\rangle\).

Символы перед величинами означают:\\
\(\Delta\) - конечное приращение величины, т.е. разность ее конечного и начального значений, например, \(\Delta U=U_{2}-U_{1}, \Delta S=S_{2}-S_{1}\),\\
\(-\Delta\) - убыль величины, например, \(-\Delta U=U_{1}-U_{2},-\Delta S=S_{1}-S_{2}\),\\
d - бесконечно малое приращение (точнее, дифференциал), \(\mathrm{d} U, \mathrm{~d} S\),\\
\(\mathrm{d}^{\prime}\) - элементарное количество (не приращение), например, \(\mathrm{d}^{\prime} A, \mathrm{~d}^{\prime} Q\),\\
ง - «пропорционально», например, \(p\) s \(V\),\\
\(\sim\) - «величина порядка...», например, \(T \sim 10^{3}\) К.\\
Ц-система (или система центра масс) - это система отсчета, связанная с центром масс и движущаяся поступательно по отношению к инерциальным системам. Все величины в Ц-системе отмечены сверху значком \(\sim\) (тильда), например, \(\tilde{K}, \tilde{\mathrm{v}}\).

А - ампер\\
Å - ангстрем\\
атм - атмосфера\\
б - барн\\
В - вольт\\
Вт - ватт\\
Гс - гаусс\\
Гц - герц

Дж - джоуль\\
дин - дина\\
К - кельвин\\
кг - килограмм\\
Кл - кулон\\
л - литр\\
м - метр\\
мин - минута

Ом - ом\\
Па - паскаль\\
H - ньютон\\
c - секунда\\
ср - стерадиан\\
Тл - тесла\\
ч - час\\
эВ - электроновольт

Объектом изучения в этом учебном пособии являются системы, состоящие из очень большого числа частиц (молекул, атомов, электронов, фотонов и др.). Примером может служить газ. Такие системы принято называть макросистемами.

Казалось бы, поведение таких систем можно рассматривать на основе законов классической механики. Однако для этого пришлось бы составить такое грандиозное число уравнений, что становится сразу ясно, что практически подобный путь несостоятелен (даже если бы частицы подчинялись классическим законам, имели бы точно известные начальные условия и двигались бы по траекториям).

Но именно колоссальное число частиц в системе привело к разработке двух радикально отличающихся друг от друга методов изучения макросистем - термодинамики и молекулярной (статистической) физики.

Термодинамика (или общая теория теплоты) является постулативной наукой. Её не интересуют конкретные представления о строении системы (вещества) и физическая природа самой теплоты. При таком подходе используют понятия и физические величины, относящиеся к системе в целом. Например, идеальный газ в состоянии равновесия характеризуют объемом, давлением и температурой.

Выводы термодинамики основаны на общих принципах или началах, которые представляют собой обобщение опытных фактов.

В отличие от термодинамики, молекулярная (статистическая) физика исходит из представлений о молекулярном строении вещества. То обстоятельство, что число частиц в макросистеме весьма велико, приводит к новому типу закономерностей, имеющих статистический, т.е. вероятностный характер. На основе этих представлений молекулярная физика раскрывает физическую природу величин, которыми оперирует термодинамика. А именно, молекулярная физика позволяет объяснить наблюдаемые свойства макросистем как суммарный эффект

фект действий отдельных молекул (на основе определенных моделей). При этом используется статистический метод, который интересует не действия отдельных молекул, а средние значения определенных величин. С помощью них, средних величин, открывается возможность интерпретации наблюдаемых величин.

Молекулярная физика должна основываться на законах, которым подчиняются атомы и молекулы. Это законы квантовой физики. Без знания этих законов достаточно корректное изложение современной молекулярной физики оказывается незаконченным и, значит, методически неоправданным. Поэтому мы будем предполагать, что читатель уже познакомился с основными представлениями квантовой теории. Но при необходимости нужные сведения будем все же напоминать.

Поскольку термодинамика и молекулярная физика изучают один и тот же объект - макросистемы, - только разными методами, то мы в целях большей простоты и ясности изложения и понимания будем привлекать оба метода как взаимно дополняющие друг друга.

Итак, термодинамический и статистический методы изучения макросистем дополняют друг друга. Термодинамический метод характеризуется своей общностью и позволяет изучать явления без знания их внутренних механизмов (и моделей). Статистический же метод позволяет понять суть явлений, установить связь поведения системы в целом с поведением и свойствами отдельных частиц. Их комбинированное применение способствует наиболее эффективному решению той или иной научной проблемы.

\section*{Первое начало термодинамики}
\section*{§ 1.1. Состояние системы. Процессы}
Температура. Это величина, характеризующая состояние термодинамического равновесия макросистемы. Если при установлении теплового контакта между телами одно из тел передает энергию другому посредством теплопередачи, то считают, что первое тело имеет большую температуру, чем второе.

Любой метод измерения температуры требует установления температурной шкалы. Для этого используют некоторые особые точки. По международному соглашению температурную шкалу строят по одной реперной точке, так называемой тройной точке водь ( \(T_{\text {тр }}\) ). В термодинамической шкале температур (шкале Кельвина) принимается по определению, что

\[
T_{\text {тр }}=273,16 \mathrm{~K} .
\]

При таком значении \(T_{\text {тр }}\) интервал между точками плавления льда и кипения воды практически равен 100 кельвин, и температуры этих точек равны приближенно 273,15 и 373,15 К. Напомним, \(1 \mathrm{~K}=1^{\circ} \mathrm{C}\). Температура \(t\) по шкале Цельсия связана с температурой по шкале Кельвина равенством

\[
t=T-273,15 .
\]

Температуру \(T=0\) называют абсолютным нулем, ему соответствует \(t=-273,15^{\circ} \mathrm{C}\).

В дальнейшем (§ 1.5) мы выясним физический смысл температуры \(T\).

Температура - это одна из макроскопических характеристик макросистемы. Она не имеет смысла для систем, состоящих из нескольких молекул (впрочем, при определенной договоренности условно говорят о температуре даже одной частицы).

Состояние системы. Состояние макросистемы характеризуют величинами, которые называют термодинамическими параметрами (давление \(p\), объем \(V\), температура \(T\) и др.). Если эти

параметры имеют определенные и постоянные значения для любой части макросистемы, то ее состояние называют равновесным. Будучи выведена из состояния равновесия (система становится неравновесной), система в дальнейшем возвращается в равновесное состояние. Это происходит за время, равное по порядку величины времени релаксации ( \(\tau\) ). Так называют время, за которое первоначальное отклонение какого-либо параметра состояния уменьшается в е раз. Для каждого параметра состояния время \(\tau\) имеет свое значение. Решающую роль играет то \(\tau\), которое является наибольшим. Его и считают за время релаксации системы.

Таким образом, состояние системы является равновесным, если все параметры ее имеют определенные и постоянные значения при неизменных внешних условиях.

Равновесное состояние можно представить точкой в пространстве, по координатным осям которого откладываются значения параметров состояния. Если независимыми являются всего две переменные (например, \(p\) и \(V\) ), то равновесное состояние изображают точкой на диаграмме \(p, V\).

Процесс. Любой процесс, т.е. переход системы из одного состояния в другое (происходящий, например, благодаря внешнему воздействию) проходит, вообще говоря, через последовательность неравновесных состояний. Но


правлении 2  1. По этой причине равновесные процессы называют обратимыми*. § l.2. Первое начало термодинамики Первое начало термодинамики относится к одному из фундаментальнейших законов физики макросистем, оно является обобщением очень большого экспериментального материала и представляет собой по сути обобщенный закон сохранения энергии на тепловые процессы в самом общем виде.

Этот закон содержи



ез последовательность неравновесных состояний. Но если такое воздействие осуществляется достаточно медленно, то можно сказать, что процесс проходит через последовательность равновесных состояний. Такой процесс называют равновесным или квазистатическим.

Он может быть изображен, например, на диаграмме р, V соответствующей кривой (рис. 1.1). Направление процесса показывают стрелкой.

Неравновесные процессы мы будем условно изображать пунктирными кривыми.

Равновесный процесс может быть проведен в обратном направлении через ту же совокупность равновесных состояний, т. е. по той же кривой 1  2 (см. рис. 1.1), но в обратном на10


\begin{figure}[h]
\begin{center}
  \includegraphics[alt={},max width=\textwidth]{bd023a69-cc46-4b6f-a48e-cc548fa9259d-010_316_396_1133_68}
\captionsetup{labelformat=empty}
\caption{Рис. 1.1}
\end{center}
\end{figure}

Неравновесные процессы мы будем условно изображать пунктирными кривыми.

Равновесный процесс может быть проведен в обратном направлении через ту же совокупность равновесных состояний, т. е. по той же кривой \(1 \rightarrow 2\) (см. рис. 1.1), но в обратном на-

правлении \(2 \rightarrow 1\). По этой причине равновесные процессы называют обратимыми*.

\section*{§ 1.2. Первое начало термодинамики}
Первое начало термодинамики относится к одному из фундаментальнейших законов физики макросистем, оно является обобщением очень большого экспериментального материала и представляет собой по сути обобщенный закон сохранения энергии на тепловые процессы в самом общем виде.

Этот закон содержит три величины: внутреннюю энергию \(U\), работу \(A\) и теплоту \(Q\). Прежде чем сформулировать сам закон, установим физический смысл этих величин.

Внутренняя энергия. Внутренней энергией \(U\) макросистемы называют величину, состоящую из:


\begin{enumerate}
  \item суммарной кинетической энергии хаотического движения молекул в Ц-системе, связанной с самой системой (в этой системе отсчета суммарный импульс всех молекул равен нулю, и система как целое покоится);
  \item собственной потенциальной энергии взаимодействия всех молекул, т.е. энергии взаимодействия только между молекулами, принадлежащими данной системе;
  \item внутренней энергии самих молекул, атомов, ядер,...
\end{enumerate}

Пока мы не будем включать во внутреннюю энергию \(U\) внутреннюю энергию молекул и атомов, считая во многих процессах вклад этой энергии в \(U\) постоянным, не зависящим от вида процесса. Т. е. будем считать, что \(U\) определена с точностью до некоторой постоянной. Но при необходимости мы снимем это ограничение и включим в рассмотрение и внутримолекулярную энергию.

Наиболее существенное отличие внутренней энергии \(U\) в том, что она является функиией состояния и не зависит от того, каким путем мы привели систему в данное состояние. При изменении состояния приращение внутренней энергии определяется только конечным и начальным состояниями и не зависит от процесса, который перевел систему из одного состояния в другое.

\footnotetext{\begin{itemize}
  \item Категорически утверждать, что каждый равновесный процесс является обратимым, нельзя. Он наверняка обратим только для изотропных макросистем. Если же система анизотропна, то это не всегда так. Примером может служить явление гистерезиса.
\end{itemize}
}Заметим, что в отличие, например, от температуры \(T\), которая характеризует равновесное состояние макросистемы, внутренняя энергия \(U\) присуща как равновесным, так и неравновесным состояниям. В дальнейшем, однако, мы будем рассматривать внутреннюю энергию лишь равновесных состояний.

Если макросистема состоит из нескольких частей, то энергией межмолекулярного взаимодействия на границах этих частей (в тонком слое) можно пренебречь и считать, что внутренняя энергия всей системы практически равна сумме внутренних энергий ее частей. Это значит, что внутренняя энергия является величиной аддитивной.

Работа и количество тепла. Внутреннюю энергию макросистемы можно изменить, совершив над системой работу \(A^{\prime}\) внешними макроскопическими силами, либо путем теплопередачи.

Совершение работы сопровождается перемещением внешних тел, действующих на систему (так, например, ведет себя поршень в цилиндре с газом).

Передача макросистеме тепла \(Q\) не связана с перемещением внешних тел. Она осуществляется путем непосредственной передачи внутренней энергии макросистеме от внешних тел при контакте с ними. Передача энергии при этом может происходить и через излучение.

Первое начало термодинамики утверждает, что приращение внутренней энергии макросистемы при ее переходе из начального состояния 1 в конечное 2 равно сумме совершенной над системой работы \(A^{\prime}\) всех внешних макроскопических сил и количества переданного системе тепла \(Q\) :


\begin{equation*}
U_{2}-U_{1}=Q+A^{\prime} . \tag{1.1}
\end{equation*}


Обычно рассматривают не работу \(A^{\prime}\) над системой, а работу \(A\), производимую самой системой над внешними телами. Имея в виду, что \(A^{\prime}=-A\), перепишем (1.1) в виде


\begin{equation*}
Q=\Delta U+A, \tag{1.2}
\end{equation*}


где \(\Delta U=U_{2}-U_{1}\). Это уравнение и выражает первое начало термодинамики: количество теплоть Q, сообщенное макроси-

стеме, идет на приращение \(\Delta U\) её внутренней энергии и на совершение системой работьи \(A\) над внешними телами.

Все входящие в (1.2) величины являются алгебраическими, т. е. могут иметь как положительные, так и отрицательные знаки. Если \(Q<0\), то это значит, что тепло отводится от системы, если \(A<0\), то работа производится над системой. Приращение \(\Delta U\) внутренней энергии может иметь любой знак, в частности быть равной нулю.

Еще раз подчеркнем, что можно говорить о приращении внутренней энергии \(U\), но нельзя говорить о приращении работы или тепла. Говорят только о количестве последних двух величин в том или ином процессе, т.е. \(A\) и \(Q\) являются функциями процесса.

Первое начало термодинамики в дифференциальной форме имеет вид


\begin{equation*}
\mathrm{d}^{\prime} Q=\mathrm{d} U+\mathrm{d}^{\prime} A, \tag{1.3}
\end{equation*}


где штрих означает, что мы имеем дело не с приращением ка-кой-либо функции, а с элементарными значениями теплоты \(\left(\mathrm{d}^{\prime} Q\right)\) и работы ( \(\mathrm{d}^{\prime} A\) ).

Работа, совершаемая макросистемой. Если объем макросистемы (например, газа) получает приращение \(\mathrm{d} V\), а давление, оказываемое ею на соседние тела (стенки), равно \(p\), то элементарная работа сил, действующих со стороны газа на стенки,


\begin{equation*}
\mathrm{d}^{\prime} A=p \mathrm{~d} V . \tag{1.4}
\end{equation*}


Это легко получить для случая, когда система (газ) находится в цилиндре с поршнем (рис. 1.2). Элементарная работа, совершаемая газом при перемещении поршня на \(\mathrm{d} h\), равна \(\mathrm{d}^{\prime} A=F \mathrm{~d} h\), где \(F\) - сила, с которой газ действует на поршень. Площадь сечения последнего \(S\), поэтому \(F=p S\) и \(\mathrm{d}^{\prime} A=p S \mathrm{~d} h\), где \(S \mathrm{~d} h=\mathrm{d} V\). Отсюда и следует (1.4).

При поднятии поршня давление газа, вообще говоря, может меняться. Поэтому работа, совершаемая газом при конечных изме-

\begin{figure}[h]
\begin{center}
  \includegraphics[alt={},max width=\textwidth]{bd023a69-cc46-4b6f-a48e-cc548fa9259d-013_374_297_1355_859}
\captionsetup{labelformat=empty}
\caption{Рис. 1.2}
\end{center}
\end{figure}

нениях объема, например от \(V_{1}\) до \(V_{2}\), должна быть представлена в виде интеграла:


\begin{equation*}
A=\int_{1}^{2} p \mathrm{~d} V \tag{1.5}
\end{equation*}


\begin{figure}[h]
\begin{center}
  \includegraphics[alt={},max width=\textwidth]{bd023a69-cc46-4b6f-a48e-cc548fa9259d-014_358_477_407_68}
\captionsetup{labelformat=empty}
\caption{Рис. 1.3}
\end{center}
\end{figure}

Еще раз отметим, что работа \(A\) существенным образом зависит от процесса (или «пути»), по которому система переводится из состояния 1 в 2 . Это наглядно видно из графика на рис. 1.3, где изображен процесс изменения объема на диаграмме \(p-V\). Мы знаем, что геометрическая интерпретация интеграла (1.5) - это «площадь» под кривой \(1-2\), а эта площадь зависит от вида кривой, т. е. от процесса.

Если в результате изменений макросистема возвращается в исходное состояние, то говорят, что она совершила круговой процесс или цикл. На диаграмме \(p-V\) такой процесс имеет вид замкнутой кривой (рис. 1.4). Работа, совершаемая системой за


нениях объема, например от V1 до V2, должна быть представлена в виде интеграла: A = p V d

. (1.5) Еще раз отметим, что работа А существенным образом зависит от процесса (или «пути»), по которому система переводится из состояния 1 в 2. Это наглядно видно из графика на рис. 1.3, где изображен процесс изменения объема на диаграмме р-V. Мы знаем, что геометрическая интерпретация интеграла (1.5)


\begin{figure}[h]
\begin{center}
  \includegraphics[alt={},max width=\textwidth]{bd023a69-cc46-4b6f-a48e-cc548fa9259d-014_396_427_1141_68}
\captionsetup{labelformat=empty}
\caption{Рис. 1.4}
\end{center}
\end{figure}

Знак работы зависит от знака \(\mathrm{d} V\) : на тех участках процесса, где \(\mathrm{d} V>0\), работа \(A>0\), на тех же участках, где \(\mathrm{d} V<0\), и \(A<0\).

Пример. Найдем графически работу, совершенную системой в процессе, показанном на рис. 1.5.

На участке \(1-O\) работа \(A_{10}>0\) и равна площади под кривой \(1-O\). На участке же \(O-2\) работа \(A_{02}<0\). Суммарная работа \(A\) будет равна площади, охватываемой кривой \(1-O-2-1\), причем в данном случае эта работа \(A<0\).

Возвращаясь к формулировке первого начала термодинамики, запишем (1.3) с учетом (1.4) в виде

\begin{figure}[h]
\begin{center}
  \includegraphics[alt={},max width=\textwidth]{bd023a69-cc46-4b6f-a48e-cc548fa9259d-015_404_336_139_822}
\captionsetup{labelformat=empty}
\caption{Рис. 1.5}
\end{center}
\end{figure}


\begin{equation*}
\mathrm{d}^{\prime} Q=\mathrm{d} U+p \mathrm{~d} V . \tag{1.6}
\end{equation*}


Как мы увидим далее, в такой форме этот закон использовать наиболее удобно для решения ряда вопросов.

В заключение отметим, что справедливость постулированного первого начала термодинамики доказывается не только прямыми опытами. Первое начало позволяет, не вдаваясь в детали механизма процессов, получить многочисленные следствия и количественные соотношения. В этом его громадное познавательное значение. Подтверждаемые опытом такие следствия и соотношения дают несравненно более точное и надежное доказательство самого первого начала, нежели прямые опыты.

Перейдем к рассмотрению важных следствий, вытекающих из первого начала. В качестве объекта исследования возьмем идеальный газ, уравнение состояния которого имеет наиболее простой вид.

\section*{§ 1.3. Теплоемкость идеального газа}
Уравнение состояния идеального газа. Идеальным называют газ, уравнение состояния которого имеет вид


\begin{equation*}
p V=v R T, \tag{1.7}
\end{equation*}


его называют уравнением Клапейрона. Здесь \(v\) - количество вещества, измеряемое числом молей, \(R\) - универсальная газовая постоянная:

\[
R=8,314 \text { Дж/(моль•К). }
\]

Моль - это количество вещества, содержащее число частиц, равное постоянной Авогадро:

\[
N_{A}=6,022 \cdot 10^{23} \text { моль }^{-1} .
\]

Молю соответствует масса - молярная масса, - разная для различных газов. Эти массы приведены в периодической системе элементов, где у каждого элемента первое число - порядковый номер, а второе - молярная масса в г/моль.

С молекулярной точки зрения идеальный газ состоит из молекул, взаимодействие между которыми пренебрежимо мало*. Это присуще всем газам при достаточно большом разряжении.

Простота модели идеального газа делает ее наиболее подходящей для ознакомления с методами изучения макросистем и с соответствующими понятиями.

Теплоемкость идеального газа. Прежде всего отметим тот важный экспериментальный факт, что внутренняя энергия \(U\) идеального газа зависит только от температуры \(T\), причем


\begin{equation*}
U \sim T \tag{1.8}
\end{equation*}


в довольно широком диапазоне температур. Коэффициент пропорциональности зависит от рода газа.

Теплоемкостыо \(C\) тела (газа) называют количество тепла, которое нужно сообщить телу, чтобы повысить его температуру на один кельвин:


\begin{equation*}
C=\mathrm{d}^{\prime} Q / \mathrm{d} T . \tag{1.9}
\end{equation*}


Эта величина, как и d'Q, зависит от процесса. Без указания процесса выражение (1.9) не имеет смысла. Еще раз: теплоемкость С является функцией процесса.

Мы будем пользоваться в основном молярной теплоемкостью С, Дж/(моль • К). В таблицах же обычно указывают удельную теплоемкость


\begin{equation*}
c=C / M, \tag{1.10}
\end{equation*}


где \(c\), Дж/(кг•К), \(M\) - молярная масса.

\footnotetext{\begin{itemize}
  \item Вместе с тем, взаимодействие между молекулами даже в случае идеальных газов должно быть, но весьма слабое. Оно необходимо, так как только благодаря нему в системе может установиться равновесие.
\end{itemize}
}Особое значение имеют теплоемкости для двух процессов: при постоянном объеме \(C_{V}\) и при постоянном давлении \(C_{p}\). При постоянном объеме \(\mathrm{d} V=0\), и согласно (1.6) имеем


\begin{equation*}
C_{V}=(\partial U / \partial T)_{V} . \tag{1.11}
\end{equation*}


Такая форма записи подчеркивает, что при дифференцировании \(U\) по \(T\) объем \(V\) следует считать постоянным (это так называемая частная производная).

Опыт показывает, что во многих случаях теплоемкость \(C\) в широком интервале температур почти не меняется. Если считать, что \(C\) совсем не зависит от \(T\), то из (1.11) следует: \(\mathrm{d} U=C_{V} \mathrm{~d} T\), и можно написать простую формулу


\begin{equation*}
U=C_{V} T . \tag{1.12}
\end{equation*}


Произвольную постоянную интегрирования мы опустили, поскольку она не существенна: во все соотношения входит не сама функция \(U\), а только разность ее значений (аналогично потенциальной энергии).

Представим выражение (1.9) для теплоемкости, учитывая (1.6) и (1.11), в виде


\begin{equation*}
C=C_{V}+p \frac{\mathrm{~d} V}{\mathrm{~d} T} . \tag{1.13}
\end{equation*}


Если процесс изобарический ( \(p=\) const), то из уравнения состояния (1.7) следует, что \(p(\mathrm{~d} V / \mathrm{d} T)=R\), и соответствующая молярная теплоемкость


\begin{equation*}
C_{p}=C_{V}+R . \tag{1.14}
\end{equation*}


Важной характеристикой газов является отношение \(C_{p} / C_{V}\), которое обозначают буквой \(\gamma\) и называют постоянной адиабаты. Имея в виду (1.14), запишем

\[
\gamma=\frac{C_{p}}{C_{V}}=1+\frac{R}{C_{V}},
\]

откуда молярная теплоемкость


\begin{equation*}
C_{V}=R /(\gamma-1) . \tag{1.15}
\end{equation*}


Из опыта следует, что значения \(\gamma\) для разных газов лежат в пределах \(1,3 \div 1,67\).

Подставив (1.15) в (1.12), получим другие формы выражения для внутренней энергии \(v\) молей идеального газа:


\begin{equation*}
U=\frac{v R T}{\gamma-1}=\frac{p V}{\gamma-1} . \tag{1.16}
\end{equation*}


Пример. Определим постоянную адиабаты \(\gamma\) для газовой смеси, состоящей из \(v_{1}\) молей газа с постоянной адиабаты \(\gamma_{1}\) и \(v_{2}\) молей газа с постоянной \(\gamma_{2}\).

Исходя из того, что \(\gamma=C_{p} / C_{V}\), запишем


\begin{equation*}
\gamma=\frac{v_{1} C_{p 1}+v_{2} C_{p 2}}{v_{1} C_{V 1}+v_{2} C_{V 2}} . \tag{*}
\end{equation*}


Учитывая, что молярные теплоемкости \(C_{V}=R /(\gamma-1)\) и \(C_{p}=\gamma R /(\gamma-1)\), преобразуем (*) к виду

\[
\gamma=\frac{v_{1} \gamma_{1}\left(\gamma_{2}-1\right)+v_{2} \gamma_{2}\left(\gamma_{1}-1\right)}{v_{1}\left(\gamma_{2}-1\right)+v_{2}\left(\gamma_{1}-1\right)} .
\]

\section*{§ 1.4. Политропические процессы}
Адиабатический процесс. Это процесс, который происходит без теплообмена с окружающей средой. Для идеального газа запишем (1.6) с учетом (1.16) в виде

\[
\mathrm{d}^{\prime} Q=\mathrm{d}\left(\frac{p V}{\gamma-1}\right)+p \mathrm{~d} V=0 .
\]

Выпишем числитель этого выражения:

\[
\mathrm{d}(p V)+(\gamma-1) p \mathrm{~d} V=\mathrm{d} p \cdot V+p \mathrm{~d} V+\gamma p \mathrm{~d} V-p \mathrm{~d} V=0 .
\]

После сокращения на \(p \mathrm{~d} V\) получим

\[
V \mathrm{~d} p+\gamma p \mathrm{~d} V=0 .
\]

Разделим оба слагаемых на \(p V\), тогда

\[
\frac{\mathrm{d} p}{p}+\gamma \frac{\mathrm{d} V}{V}=0 .
\]

Это выражение представляет собой сумму дифференциалов логарифмов \(p\) и \(V\) :

\[
\mathrm{d} \ln p+\gamma \mathrm{d} \ln V=\mathrm{d} \ln \left(p V^{\gamma}\right)=0 .
\]

Равенство нулю последнего дифференциала означает, что мы имеем дело с константой, т. е.


\begin{equation*}
p V^{\gamma}=\text { const. } \tag{1.17}
\end{equation*}


Это и есть уравнение адиабаты в переменных \(p, V\). Его называют уравнением Пуассона. Это уравнение можно представить и в других переменных. Например, в переменных \(T, V\) оно выглядит так:


\begin{equation*}
T V^{\gamma-1}=\text { const. } \tag{1.18}
\end{equation*}


Адиабата (1.17) идет круче изотермы ( \(p V=\) const): для этого достаточно сравнить производные \(\mathrm{d} p / \mathrm{d} V\) для обоих процессов. Для изотермического процесса \(p \mathrm{~d} V+\mathrm{d} p \cdot V=0\), откуда

\[
\mathrm{d} p / \mathrm{d} V=-(p / V) .
\]

А для адиабатического процесса (1.17) \(p \gamma V^{\gamma}-1 \mathrm{~d} V+\mathrm{d} p \cdot V=\) const, откуда

\[
\mathrm{d} p / \mathrm{d} V=-\gamma(p / V) .
\]

Поскольку \(\gamma>1\), то адиабаты, действительно, идут круче изотерм (рис. 1.6).

\begin{figure}[h]
\begin{center}
  \includegraphics[alt={},max width=\textwidth]{bd023a69-cc46-4b6f-a48e-cc548fa9259d-019_362_398_935_746}
\captionsetup{labelformat=empty}
\caption{Рис. 1.6}
\end{center}
\end{figure}

Политропические процессы. Так называют процессы, уравнение которых в переменных \(p, V\) имеет вид


\begin{equation*}
p V^{n}=\mathrm{const}, \tag{1.19}
\end{equation*}


где \(n\) - произвольное число, как положительное, так и отрицательное, а также равное нулю. Таким образом, любой процесс, уравнение которого можно свести к виду (1.19), является политропическим. Соответствующую кривую называют политропой.

Политропическими являются, в частности, процессы изохорический, изобарический, изотермический и адиабатический.

Отличительной особенностью всех политропических процессов является то, что в ходе этих процессов теплоемкость системы остается постоянной:


\begin{equation*}
C_{n}=\text { const. } \tag{1.20}
\end{equation*}


Убедимся в этом, воспользовавшись формулой (1.13). Входящую в нее производную \(\mathrm{d} V / \mathrm{d} t\) найдем с помощью уравнения \(T V^{n-1}=\) const - уравнения политропического процесса (1.18) в переменных \(T, V\). Продифференцируем последнее уравнение:

\[
\mathrm{d} T \cdot V^{n-1}+T(n-1) V^{n-2} \mathrm{~d} V=0,
\]

откуда

\[
\frac{\mathrm{d} V}{\mathrm{~d} T}=-\frac{1}{n-1} \frac{V}{T}=-\frac{R}{p(n-1)} .
\]

Остается подставить это выражение в (1.13), и мы получим


\begin{equation*}
C_{n}=C_{V}-\frac{R}{n-1}=\frac{R}{\gamma-1}-\frac{R}{n-1} . \tag{1.21}
\end{equation*}


Отсюда видно, что, действительно, в ходе политропических процессов \(C_{n}=\) const.

Если \(n=\gamma\), то \(C_{n}=0\) (это сразу следует и из определения адиабатического процесса). При \(n=1 C_{n} \rightarrow \infty\), как и должно быть при изотермическом процессе.

Интересный результат обнаруживается в случае \(1<n<\gamma\). Перепишем (1.21) в виде


\begin{equation*}
C_{n}=\frac{n-\gamma}{(\gamma-1)(n-1)} R . \tag{1.22}
\end{equation*}


\begin{figure}[h]
\begin{center}
  \includegraphics[alt={},max width=\textwidth]{bd023a69-cc46-4b6f-a48e-cc548fa9259d-020_394_421_1391_72}
\captionsetup{labelformat=empty}
\caption{Рис. 1.7}
\end{center}
\end{figure}

Видно, что в этом случае \(C_{n}<0\). Это значит, что мы сообщаем тепло системе, а она охлаждается, поскольку знаки \(\mathrm{d}^{\prime} Q\) и \(\mathrm{d} T\) должны быть при этом противоположными. Это относится ко всем политропическим процессам, «промежуточным» между изотермическим и адиабатическим (рис. 1.7).

Из (1.21) нетрудно получить выражение, определяющее \(n\) через теплоемкости:


\begin{equation*}
n=\frac{C_{n}-C_{p}}{C_{n}-C_{V}} . \tag{1.23}
\end{equation*}


Работа газа при политропических процессах. Работу, которую совершает газ при любом процессе, можно вычислять с помощью формулы (1.5). Но эту работу можно вычислять и иначе - с помощью первого начала, а именно:

\[
A=Q-\Delta U=C_{n} \Delta T-C_{V} \Delta T=\left(C_{n}-C_{V}\right) \Delta T,
\]

где \(\Delta T=T_{2}-T_{1}\). Для \(v\) молей согласно (1.21) \(C_{n}-C_{V}=-v R /(n-1)\), поэтому предыдущее выражение можно переписать так:


\begin{equation*}
A=\frac{v R\left(T_{1}-T_{2}\right)}{n-1}=\frac{p_{1} V_{1}-p_{2} V_{2}}{n-1} . \tag{1.24}
\end{equation*}


Отсюда следует, что при \(n=\gamma\) работа \(A=-\Delta U\), как и должно быть. Из (1.24) следует также, что при политропических процессах работа газа

\[
A \cup-\Delta U .
\]

Выражение (1.24) не удобно для изотермических процессов, поскольку при \(n=1, T_{1}=T_{2}\) и \(A=0 / 0\). Поэтому в данном случае работу \(A\) следует вычислять непосредственно с помощью (1.5):


\begin{equation*}
A=\int p \mathrm{~d} V=\int_{1}^{2} \frac{v R T}{V} \mathrm{~d} V=v R T \ln \frac{V_{2}}{V_{1}} . \tag{1.25}
\end{equation*}


\section*{§ 1.5. Молекулярно-кинетическая теория}
Теперь нам предстоит раскрыть физический смысл, природу и происхождение непосредственно наблюдаемых параметров \(p\), \(T\) и др., исходя из молекулярно-кинетических представлений. При этом мы будем использовать статистический метод, интересуясь движением не отдельных молекул, а лишь такими средними величинами, которые характеризуют движение колоссальной совокупности молекул.

Первый шаг на этом пути - выбор модели данной макросистемы. Модель, естественно, выбирают сначала простейшую и с помощью нее проводят оценочный расчет*. Полученные результаты сравнивают с экспериментом. При неудовлетворительном расхождении с опытом модель усложняют, выясняют, как это отразится на результатах и т. д. Это типичный путь познания природы.

Простейшей моделью обладает идеальный газ. Будем считать, что


\begin{enumerate}
  \item молекулы идеального газа не взаимодействуют (или практически не взаимодействуют) друг с другом;
  \item в равновесном состоянии движение молекул полностью хаотично. Это позволяет в грубом приближении считать, что все молекулы движутся только в направлениях \(X, Y\) и \(Z\), т. е. если в единице объема имеется \(n\) молекул, то в каждом из этих направлений движутся по \(n / 3\) молекул, или \(n / 6\) в одну сторону.
\end{enumerate}

Число ударов молекул о стенку. Разобьем молекулы в каждой единице объема на группы \(\Delta n_{i}\), в каждой из которых скорости молекул можно считать практически одинаковыми и равными \(v_{i}\), так что \(\sum \Delta n_{i}=n\) - полное число молекул в единице объема.

Число \(\Delta v_{i}\) молекул \(i\)-й группы, которые достигают за малый

\begin{figure}[h]
\begin{center}
  \includegraphics[alt={},max width=\textwidth]{bd023a69-cc46-4b6f-a48e-cc548fa9259d-022_362_331_1149_135}
\captionsetup{labelformat=empty}
\caption{Рис. 1.8}
\end{center}
\end{figure}

промежуток времени \(\Delta t\) элемента стенки площадью \(\Delta S\), двигаясь перпендикулярно к нему, равно, как нетрудно сообразить, числу таких молекул в цилиндре длиной \(v_{i} \Delta t\) и сечением \(\Delta S\) (рис. 1.8), т. е.

\[
\Delta v_{i}=\frac{1}{6} \Delta n_{i} v_{i} \Delta t \Delta S .
\]

Отсюда следует, что число ударов ежесекундно о единицу поверхности стенки \(v_{i}=\Delta v_{i} / \Delta t \Delta S\), или


\begin{equation*}
v_{i}=\frac{1}{6} \Delta n_{i} v_{i} . \tag{1.26}
\end{equation*}


\footnotetext{\begin{itemize}
  \item Заметим, что с оценочного подхода начинается построение практически любой серьезной теории. Его можно рассматривать как первое приближение.
\end{itemize}
}Суммируя по всем группам, находим

\[
v=\sum v_{i}=\frac{1}{6} \sum \Delta n_{i} v_{i} .
\]

Разделим и умножим последнюю сумму на \(n\). В результате приходим к тому, что полное число ударов молекул о единицу поверхности стенки за единицу времени определяется следующим выражением:


\begin{equation*}
v=\frac{1}{6} n\langle v\rangle, \tag{1.27}
\end{equation*}


где 〈υ〉 - среднее значение скорости молекул. Заметим, что точный расчет дает коэффициент \(1 / 4\) (а не \(1 / 6\) ). Т. е., несмотря на такие грубые предположения, результат получился вполне приличным.

Оценим число \(v\) для воздуха при нормальных условиях. Считая, что \(n \sim 10^{19} \mathrm{~cm}^{-3}\) (порядка постоянной Лошмидта) и \(v \sim \sim 1 \mathrm{~km} / \mathrm{c}\), получим

\[
v \sim 10^{19} \cdot 10^{5}=10^{24} \mathrm{c}^{-1} \mathrm{~cm}^{-2} .
\]

Давление газа на стенку. Природа давления, как мы догадываемся, это совокупное действие множества молекул. При оценочном подходе будем считать, что каждая молекула, налетая на стенку нормально, в результате столкновения с ней отлетает в противоположном направлении (хотя заведомо ясно, что это не так*). До столкновения со стенкой молекула имела импульс \(\mathbf{p}_{1}=m \mathbf{v}\) и после столкновения при сделанном предположении - импульс \(\mathbf{p}_{2}=-m \mathbf{v}\). Приращение импульса молекулы в результате столкновения

\[
\Delta \mathbf{p}=\mathbf{p}_{2}-\mathbf{p}_{1}=-m \mathbf{v}-m \mathbf{v}=-2 m \mathbf{v} .
\]

Такой же импульс, но в противоположном направлении, получила стенка - согласно закону сохранения импульса.

Импульс, передаваемый ежесекундно единице поверхности стенки молекулами \(i\)-й группы, найдем с помощью (1.26):


Суммируя по всем группам, находим     i i i n v

 .

Разделим и умножим последнюю сумму на n. В результате приходим к тому, что полное число ударов молекул о единицу поверхности стенки за единицу времени определяется следующим выражением:   1 6 npvq, (1.27) где pvq — среднее значение скорости молекул. Заметим, что точный расчет дает коэффициент 1/4 (а не 1/6). Т. е., несмотря на такие



\begin{equation*}
p_{i}=2 m v_{i} \cdot v_{i}=\frac{1}{3} m v_{i}^{2} \Delta n_{i} . \tag{1.28}
\end{equation*}


\footnotetext{\begin{itemize}
  \item Стенка не является идеальным зеркалом, она состоит из молекул, поэтому отражение от нее молекул имеет диффузный характер.
\end{itemize}
}Результирующее давление получим, просуммировав (1.28) по всем группам \(\Delta n_{i}\) молекул:

\[
p=\sum p_{i}=\frac{1}{3} m \sum v_{i}^{2} \Delta n_{i} .
\]

Разделив и умножив последнюю сумму в этой формуле на \(n\), приходим к следующему выражению:


\begin{equation*}
p=\frac{1}{3} n m\left\langle v^{2}\right\rangle . \tag{1.29}
\end{equation*}


Это выражение можно переписать иначе:


\begin{equation*}
p=\frac{2}{3} n\left\langle\varepsilon_{\text {пост }}\right\rangle, \tag{1.30}
\end{equation*}


где \(\left\langle\varepsilon_{\text {пост }}\right\rangle=\left\langle m v^{2} / 2\right\rangle\) - среднее значение поступательной кинетической энергии молекул.

Формулу (1.30) называют основным уравнением кинетической теории газов. Она раскрывает физический смысл макропараметра \(p\) : давление газа на стенку определяется средним значением поступательной кинетической энергии молекул (и только поступательной!).

Отметим, что это выражение является точным, несмотря на то, что расчет имеет довольно грубый оценочный характер. Дело в том, что в процессе расчета были допущены две неточности в числовых коэффициентах, которые случайным образом (так иногда бывает) компенсировали друг друга. А именно, для числа столкновений \(v\) и передаваемого стенке в среднем импульса \(\Delta p\) каждой молекулой мы использовали коэффициенты соответственно \(1 / 6\) и 2 . Точный же расчет дает \(1 / 4\) и \(4 / 3\). Как видим, их произведение в обоих случаях равно \(1 / 3\).

Физический смысл температуры \(T\). Сопоставив полученное выражение (1.30) с уравнением Клапейрона (1.7), находим


\begin{equation*}
\left\langle\varepsilon_{\text {пост }}\right\rangle=\frac{3}{2} k T, \tag{1.31}
\end{equation*}


где \(k\) - постоянная Больцмана


\begin{equation*}
k=R / N_{A}=1,38 \cdot 10^{-23} \text { Дж /К. } \tag{1.32}
\end{equation*}


Формула (1.31) замечательна тем, что вскрывает физический смысл температуры \(T\) : температура \(T\) выражает среднюю кинетическую энергию молекул.

Следует обратить внимание, что \(\left\langle\varepsilon_{\text {пост }}\right\rangle\) зависит только от \(T\), от массы же молекул не зависит.

Заменив в (1.30) \(\left\langle\varepsilon_{\text {пост }}\right\rangle\) его выражением (1.31), получим


\begin{equation*}
p=n k T, \tag{1.33}
\end{equation*}


где \(n\) - концентрация молекул. Это по существу просто иная форма уравнения состояния идеального газа. Формулу (1.33) можно, конечно, получить и сразу из уравнения \(p V_{M}=R T\), разделив обе части на \(V_{M}\) и представив \(R\) как \(k N_{A}\).

Давление фотонного газа. Предполагается, что излучение (фотонный газ) равновесное. Скорость фотонов равна \(c\) и их импульс \(p_{\phi}=\varepsilon / c\), где \(\varepsilon\) - энергия фотона ( \(\varepsilon=h v\) ). Представим формулу для давления (1.29) так:


\begin{equation*}
p=\frac{1}{3} n\langle m v \cdot v\rangle=\frac{1}{3} n\left\langle p^{\prime} v\right\rangle . \tag{1.34}
\end{equation*}


Подставив вместо импульса \(p^{\prime}\) импульс фотона \(\varepsilon / c\), а вместо \(v\) скорость \(c\), получим


\begin{equation*}
p=\frac{1}{3} n\langle\varepsilon\rangle=\frac{1}{3} u, \tag{1.35}
\end{equation*}


где \(u\) - плотность энергии фотонного газа (излучения).

\section*{§ 1.6. Гипотеза о равнораспределении энергии по степеням свободы}
Степени свободы. Так называют число независимых координат, определяющих положение системы, или в интересующем нас случае - молекулы. Для определения положения центра масс молекулы необходимо задать три координаты. Это означает, что молекула имеет три поступательных степени свободы.

Если молекула двухатомная и жесткая («гантель»), то, кроме трех поступательных степеней свободы, она имеет и две вращательные, связанные с углами поворота вокруг двух взаимно перпендикулярных осей \(1-1\) и \(2-2\), проходящих через центр масс \(C\), как показано на рис.1.9. Вращение вокруг оси молекулы для материальных точек лишено смысла.

\begin{figure}[h]
\begin{center}
  \includegraphics[alt={},max width=\textwidth]{bd023a69-cc46-4b6f-a48e-cc548fa9259d-025_289_251_1488_905}
\captionsetup{labelformat=empty}
\caption{Рис. 1.9}
\end{center}
\end{figure}

Таким образом, жесткая двухатомная молекула имеет пять степеней свободы: три поступательных и две вращательных.

Если молекула упругая, то возможны колебания атомов и необходима еще одна степень свободы (расстояние между атомами). Ее называют колебательной.

Гипотеза о равнораспределении. Тот факт, что средняя энергия поступательного движения молекулы согласно (1.31) равна \(\frac{3}{2} k T\), означает, что на каждую степень свободы в среднем приходится энергия \(k T / 2\). Больцман обобщил этот вывод в виде гипотезы о равном распределении средней энергии по степеням свободь. При этом на колебательную степень свободы должны приходиться в среднем по две половинки \(k T\) - одна в виде кинетической и одна в виде потенциальной (как мы знаем, их средние значения одинаковы).

Итак, средняя энергия молекулы


\begin{equation*}
\langle\varepsilon\rangle=\frac{i}{2} k T, \tag{1.36}
\end{equation*}


где \(i\) - сумма числа поступательных ( \(z_{\text {пост }}\) ), вращательных ( \(z_{\text {вр }}\) ) и удвоенного числа колебательных ( \(z_{\text {кол }}\) ) степеней свободы:


\begin{equation*}
i=z_{\text {пост }}+z_{\text {вр }}+2 z_{\text {кол }} . \tag{1.37}
\end{equation*}


Число \(i\) совпадает с числом степеней свободы только для жестких молекул.

Пример 1. Газ состоит из жестких двухатомных молекул, концентрация которых \(n\). При некоторой температуре средняя кинетическая энергия молекулы равна \(\langle\varepsilon\rangle\). Найдем давление такого газа.

Здесь важно обратить внимание на то, что давление определяется только поступательными степенями свободы. У нашей молекулы число степеней свободы \(z=5\), поэтому на поступательную энергию приходится \(\frac{3}{5}\langle\varepsilon\rangle\). В результате согласно (1.30) давление

\[
p=\frac{2}{3} n \cdot \frac{3}{5}\langle\varepsilon\rangle=\frac{2}{5} n\langle\varepsilon\rangle .
\]

Пример 2. Газ из жестких двухатомных молекул находится при температуре \(T=300\) К. Вычислим среднюю квадратичную угловую скорость \(\omega_{\text {кв }}\) вращения молекулы, если ее момент инерции \(I=2,1 \cdot 10^{-39}\) г.см \({ }^{2}\).

Из механики известно, что кинетическая энергия вращающегося твердого тела \(K=I \omega^{2} / 2\). В нашем случае две вращательные степени свободы, поэтому средняя вращательная кинетическая энергия

\[
\langle K\rangle=\left\langle I \omega^{2} / 2\right\rangle=\left\langle I \omega_{1}^{2} / 2\right\rangle+\left\langle I \omega_{2}^{2} / 2\right\rangle,
\]

поскольку \(\omega^{2}=\omega_{1}^{2}+\omega_{2}^{2}\), где \(\omega_{1}\) и \(\omega_{2}\) - угловые скорости вращения вокруг взаимно перпендикулярных осей, проходящих через центр масс молекулы. Ясно, что \(\left\langle\omega_{1}^{2}\right\rangle=\left\langle\omega_{2}^{2}\right\rangle\), поэтому

\[
\left\langle I \omega^{2} / 2\right\rangle=2 \cdot(k T / 2)=k T .
\]

Искомая угловая скорость \(\omega_{\text {кв }}=\sqrt{\left\langle\omega^{2}\right\rangle}\) или

\[
\omega_{\text {кв }}=\sqrt{\frac{2 k T}{I}}=\sqrt{\frac{2 \cdot 1,38 \cdot 10^{-16} \cdot 3 \cdot 10^{2}}{2,1 \cdot 10^{-39}}}=6,3 \cdot 10^{12} \text { рад } / \text { с. }
\]

Еще о степенях свободы. Приведем некоторые важные результаты для числа степеней свободы у линейны \(x\) и нелинейны \(x\) молекул, учитывая, что полное число степеней свободы у системы из \(N\) материальных точек равно \(3 N\). Эти результаты приведены в табл. 1.1.


Из механики известно, что кинетическая энергия вращающегося твердого тела K  I 2/2. В нашем случае две вращательные степени свободы, поэтому средняя вращательная кинетическая энергия pKq  pI 2/2q  pI 1 2/2q + pI 2 2/2q, поскольку  2   1 2 +  2 2, где  1 и  2 — угловые скорости вращения вокруг взаимно перпендикулярных осей, проходящих через центр масс молекулы. Ясно, что p 1 2q  p 2 2q, поэто


\begin{table}[h]
\begin{center}
\captionsetup{labelformat=empty}
\caption{Таблица 1.1}
\begin{tabular}{|l|l|l|}
\hline
Молекула & жесткая \( i=z_{\text {пост }}+z_{\text {вр }} \) & упругая \( i=z_{\text {пост }}+z_{\text {вр }}+2 z_{\text {кол }} \) \\
\hline
\begin{tabular}{l}
Нелинейная \\
\} \\
\end{tabular} & \( z=3_{\text {пост }}+2_{\text {вр }}=5 \) \( z=3_{\text {пост }}+3_{\text {вр }}=6 \) & \( \begin{aligned} & z_{\text {кол }}=3 N-5 \\ & z_{\text {кол }}=3 N-6 \end{aligned} \) \\
\hline
\end{tabular}
\end{center}
\end{table}

Пример. Выясним, какие колебательные степени свободы возможны у линейной молекулы \(\mathrm{CO}_{2}\).

\begin{figure}[h]
\begin{center}
  \includegraphics[alt={},max width=\textwidth]{bd023a69-cc46-4b6f-a48e-cc548fa9259d-028_269_332_141_197}
\captionsetup{labelformat=empty}
\caption{Рис. 1.10}
\end{center}
\end{figure}

Согласно приведенной таблице таких колебаний должно быть четыре ( \(3 \cdot 3\) -\(-5=4\) ). Эти колебания (в Ц-системе) показаны на рис. 1.10, и их действительно четыре, поскольку в случае \(a\) ) возможны колебания как в плоскости рисунка, так и в перпендикулярной к ней (две поляризации).

Внутренняя энергия. Эту энергию для моля идеального газа можно найти, умножив (1.36) на постоянную Авогадро:


\begin{equation*}
U_{M}=\frac{i}{2} R T, \tag{1.38}
\end{equation*}


где \(R=k N_{A}\).\\
Учитывая (1.11) и (1.14), получим выражения для молярных теплоемкостей:


\begin{equation*}
C_{V}=\frac{i}{2} R, \quad C_{p}=\frac{i+2}{2} R \tag{1.39}
\end{equation*}


и их отношение - постоянную адиабаты \(\gamma\), равную \(C_{p} / C_{V}\) :


\begin{equation*}
\gamma=\frac{i+2}{i} . \tag{1.40}
\end{equation*}


Согласно этой формуле для молекулы одноатомной \(\gamma=1,67\), жесткой двухатомной \(\gamma=1,40\) и упругой двухатомной \(\gamma=1,29\). Эти значения хорошо согласуются с опытными данными в области комнатных температур.

Отсюда, казалось бы, следует, что гипотеза о равнораспределении энергии по степеням свободы подтверждается, однако расширение температурного интервала исследований показало


Согласно приведенной таблице таких колебаний должно быть четыре (3·3 – – 5  4). Эти колебания (в Ц-системе) показаны на рис. 1.10, и их действительно четыре, поскольку в случае а) возможны колебания как в плоскости рисунка, так и в перпендикулярной к ней (две поляризации).

Внутренняя энергия. Эту энергию для моля идеального газа можно найти, умножив (1.36) на постоянную Авогадро: UM = i 2 RT, (1.


\begin{figure}[h]
\begin{center}
  \includegraphics[alt={},max width=\textwidth]{bd023a69-cc46-4b6f-a48e-cc548fa9259d-028_338_630_1427_68}
\captionsetup{labelformat=empty}
\caption{Рис. 1.11}
\end{center}
\end{figure}

Парадоксально, но успех гипотезы о равнораспределении энергии по степеням свободы связан с отказом от этой гипотезы. Успех был основан на том, что некоторые степени свободы искусственно исключались без каких-либо убедительных обоснований. А если заглянуть вглубь молекулы или атома, то сразу обнаруживается, что множество степеней свободы просто проигнорировано?!

Все эти вопросы были полностью разрешены только в рамках квантовой теории. Известно, что вращательная и колебательная энергии квантованы. Их уровни определяются соответственно формулами


\begin{equation*}
E_{r}=\frac{\hbar^{2}}{2 I} r(r+1), \quad E_{v}=\hbar \omega\left(v+\frac{1}{2}\right), \tag{1.41}
\end{equation*}


где \(r\) - вращательное квантовое число ( \(r=0,1,2, \ldots\) ), \(v\) - колебательное ( \(v=0,1,2, \ldots\) ), \(I\) - момент инерции молекулы относительно той или иной главной оси, \(\omega\) - собственная частота колебаний, \(\hbar\) - постоянная Планка ( \(\hbar=h / 2 \pi\) ).

Из этих формул следует, что минимальная вращательная энергия молекулы \(\mathrm{H}_{2}\) равна порядка одной сотой эВ. И при такой низкой температуре как 50 К (участок 1 на рис. 1.11) средняя энергия поступательного движения молекулы вдвое меньше минимальной вращательной энергии. Т. е. ее оказывается недостаточно, чтобы возбудить вращательные степени свободы. В этих условиях, как говорят, вращательные степени свободы «заморожены».

В области температур \(\sim 500\) К, соответствующих участку 2 , вращательные степени свободы полностью разморожены, и молекула \(\mathrm{H}_{2}\) ведет себя как жесткая двухатомная молекула с числом степеней свободы \(3+2=5\). Заметим, что включаются две, а не три вращательные степени свободы. Это связано с тем, что для включения степени свободы, соответствующей вращению молекулы вокруг ее оси, проходящей через оба ядра, требуется значительно большая энергия из-за малости момента инерции молекулы относительно этой оси, согласно первой из формул (1.41).

При температурах, превышающих 1000 К, энергии уже оказывается достаточно для постепенного возбуждения колебате-

льной степени свободы* (полностью - нет, поскольку раньше наступает диссоциация молекулы на два атома, участок 3).

Объяснение зависимости \(C_{V}(T)\), показанной на рис. 1.11, это серьезный успех квантовой теории.

\section*{§1.7. Газ Ван-дер-Ваальса}
Уравнение Ван-дер-Ваальса. До сих пор мы не выходили за рамки модели идеального газа ( \(p V_{M}=R T\) ). Однако опыт вынуждает уточнить эту модель, поскольку с ростом давления (при \(T\) = const) оказывается, что \(p V_{M} \neq R T\). При \(p=1000\) атм \(p V_{M}\) становится вдвое больше, чем предписывает модель идеального газа (газ не «сжимается»).

Причин этому две:

\begin{enumerate}
  \item собственный размер молекул; он и уменьшает объем, доступный для движения молекул, при нормальных условиях он составляет \(\sim 0,07 \%\) объема сосуда с газом, а при 100 атм уже 70\%!
  \item сложный характер взаимодействия между молекулами. Типичная кривая зависимости энергии взаимодействия \(U_{\text {вз }}\) от расстояния \(r\) между их центрами
\end{enumerate}

\begin{figure}[h]
\begin{center}
  \includegraphics[alt={},max width=\textwidth]{bd023a69-cc46-4b6f-a48e-cc548fa9259d-030_303_455_1083_68}
\captionsetup{labelformat=empty}
\caption{Рис. 1.12}
\end{center}
\end{figure}

Эти причины можно учесть путем введения поправок в уравнение состояния идеальных газов, что и сделал Ван-дер-Ваальс. В результате уравнение состояния одного моля реального газа приняло вид



\begin{equation*}
\left(p+\frac{a}{V_{M}^{2}}\right)\left(V_{M}-b\right)=R T . \tag{1.42}
\end{equation*}


\footnotetext{\begin{itemize}
  \item Заметим, что между нулевым и первым возбужденным колебательным уровнем расположены несколько десятков вращательных уровней.
\end{itemize}
}Это и есть уравнение Ван-дер-Ваальса. Здесь \(a\) и \(b\) - постоянные Ван-дер-Ваальса, для разных газов они имеют свои значения.

Если мы имеем дело не с одним, а с \(\nu\) молями газа объемом \(V\), то в уравнении (1.42) следует сделать замену: \(V_{M}=V / v\).

Поправка в первой скобке, \(a / V_{M}^{2}\), обусловлена силами притяжения между молекулами. Она имеет размерность давления, и ее иногда называют внутренним давлением. На стенку сосуда такой газ оказывает давление \(p\). Однако, если бы силы притяжения между молекулами мгновенно исчезли, то давление на стенку стало бы \(p+a / V_{M}^{2}\). Т. е. при переходе от идеального газа к реальному давление на стенку уменьшается - из-за сил притяжения между молекулами.

Поправка \(b\), как легко сообразить, связана с собственным объемом молекул, ее размерность \(\mathrm{m}^{3} /\) моль.

Газ, подчиняющийся уравнению (1.42), называют ван-дер-ваальсовским. Обращаем на это внимание в связи с тем, что к настоящему времени предложено много других уравнений состояния (свыше 100), более точных, но и более громоздких.

Для наших целей уравнение Ван-дер-Ваальса является предпочтительным: при большой простоте оно дает возможность объяснить, хотя бы качественно, широкий круг явлений в газах и даже в жидкостях. К этому мы позже вернемся (§ 5.1).\\
Пример. Найдем давление, при котором плотность углекислого газа с температурой \(T=300\) К окажется равной \(\rho=500\) г/л.\\
Считая газ ван-дер-ваальсовским, представим (1.42) в виде


\begin{equation*}
\left(p+\frac{v^{2} a}{V^{2}}\right)\left(\frac{V}{v}-b\right)=R T . \tag{1}
\end{equation*}


Отсюда, имея в виду, что \(v=m / M\) и \(m / V=\rho\), получим:


\begin{equation*}
p=\frac{v R T}{V-v b}-\frac{v^{2} a}{V^{2}}=\frac{\rho R T}{M-\rho b}-a \frac{\rho^{2}}{M} . \tag{2}
\end{equation*}


Для углекислого газа \(a=0,367\) Па \(\cdot \mathrm{M}^{6} /\) моль \(^{2}, b=4,3 \cdot 10^{-5} м^{3} /\) моль и \(M=44 г / м о л ь\). В результате подстановки найдем \(p \approx 80\) атм. Расчет же по формуле состояния идеального газа дает 280 атм. Различие весьма значительное.

Энергия ван-дер-ваальсовского газа. Внутренняя энергия такого газа \(U=\tilde{K}+U_{\text {вз }}\), где \(\tilde{K}\) - суммарная кинетическая энергия молекул в Ц-системе (связанной с сосудом), \(U_{\text {вз }}\) - суммарная энергия взаимодействий молекул (собственная потенциальная энергия). Сначала найдем \(U_{\text {вз }}\). Для этого воспользуемся тем, что работа сил притяжения равна убыли энергии \(U_{\text {вз }}\) : \(\mathrm{d} A=-\mathrm{d} U_{\text {вз }}\). Силы притяжения характеризуются внутренним давлением \(p_{i}=a / V_{M}^{2}\) в уравнении (1.42). Тогда элементарная работа этих сил \(\mathrm{d}^{\prime} A=-p_{i} \mathrm{~d} V_{M}\), где знак минус обусловлен тем, что при расширении газа ( \(\mathrm{d} V_{M}>0\) ) работа \(\mathrm{d}^{\prime} A\) должна быть отрицательной, т. е. \(\mathrm{d}^{\prime} A<0\). Итак,

\[
\mathrm{d}^{\prime} A=-\frac{a}{V_{M}^{2}} \mathrm{~d} V_{M}=-\mathrm{d}\left(-\frac{a}{V_{M}}\right) .
\]

Мы представили \(\mathrm{d}^{\prime} A\) как убыль некоторой величины - она и является энергией \(U_{\text {вз }}\) :


\begin{equation*}
U_{\text {вз }}=-a / V_{M} . \tag{1.43}
\end{equation*}


Константа, которую здесь следовало бы добавить, несущественна. Поэтому мы сразу же положили ее равной нулю. Получается естественный результат: при \(V_{M} \rightarrow \infty \quad U_{\text {в } 3} \rightarrow 0\).

Суммарная же кинетическая энергия \(\tilde{K}\) зависит от поступательного и внутреннего движений молекул, и определяется как \(C_{V} T\).

Таким образом, внутренняя энергия моля ван-дер-ваальсовского газа


\begin{equation*}
U_{M}=C_{V} T-\frac{a}{V_{M}}, \tag{1.44}
\end{equation*}


где \(C_{V}=(i / 2) R\).\\
Если газ расширяется в пустоту без теплообмена с окружающими телами, то \(A=0, Q=0\), и согласно первому началу в этом процессе \(U=\) const. Значит, как видно из \((1.44)\), с ростом объема температура газа уменьшается (в отличие от идеального газа).

Это можно представить и наглядно (рис. 1.13). Видно, что при расширении газа (увеличении объема \(V\) ) в случае \(U=\mathrm{const}\)

суммарная кинетическая энергия молекул газа, а значит и температура \(T\), уменьшается. Газ охлаждается.

Пример. Найдем приращение температуры \(v\) молей ван-дер-ваальсовского газа при расширении его в пустоту от объема \(V_{1}\) до \(V_{2}\) в теплоизолированном сосуде (газ был локализован в объеме \(V_{1}\) и в некоторый момент перегородку уб-

\begin{figure}[h]
\begin{center}
  \includegraphics[alt={},max width=\textwidth]{bd023a69-cc46-4b6f-a48e-cc548fa9259d-033_366_422_145_736}
\captionsetup{labelformat=empty}
\caption{Рис. 1.13}
\end{center}
\end{figure}

рали).

Как мы выяснили, в этих условиях внутренняя энергия газа \(U=\) const. Значит согласно (1.44)

\[
\Delta U=v C_{V} T-v^{2} a\left(\frac{1}{V_{2}}-\frac{1}{V_{1}}\right)=0,
\]

где \(C_{V}\) - молярная теплоемкость. Из этого выражения следует

\[
\Delta T=\frac{v a}{C_{V}}\left(\frac{1}{V_{2}}-\frac{1}{V_{1}}\right)<0,
\]

т. е. температура газа уменьшается.

Эффект Джоуля-Томсона. Исследования Джоуля и Томсона позволили не только экспериментально подтвердить зависимость внутренней энергии реального газа от его объема. Было открыто важное физическое явление, получившее название эффекта Джоуля-Томсона.

В их опытах использовалась теплоизолированная трубка, в середине которой помещалась пористая перегородка \(N\) (рис. 1.14). Исследуемый газ под действием перепада давлений ( \(p_{1}>p_{2}\) ) медленно протекал через перегородку. Благодаря медленности течения кинетическая энергия газа как целого практически не менялась. В этих условиях газ в каждый момент по обе стороны перегородки находился в термодинамически равновесных состояниях. Теплоизолированность

\begin{figure}[h]
\begin{center}
  \includegraphics[alt={},max width=\textwidth]{bd023a69-cc46-4b6f-a48e-cc548fa9259d-033_140_358_1387_802}
\captionsetup{labelformat=empty}
\caption{Рис. 1.14}
\end{center}
\end{figure}

трубки делала процесс течения адиабатическим. Давления газа по обе стороны перегородки ( \(p_{1}\) и \(p_{2}\) ) поддерживались постоянными.

При стационарном течении с одной стороны перегородки устанавливалась постоянная температура \(T_{1}\), а с другой стороны - \(T_{2}\). Эти

температуры и измерялись в опытах. Изменение температуры - приращение \(\Delta T\) - при таком течении и называют эффектом Джоуля-Томсона.

Сначала попытаемся выяснить, что можно ожидать относительно \(\Delta T\). Для этого мысленно выделим слева от перегородки объем \(V_{1}\) моля газа. После прохождения через перегородку эта порция газа займет объем \(V_{2}\). Согласно 1 -му началу термодинамики с учетом того, что в этом процессе \(Q=0\), приращение внутренней энергии \(\Delta U\) данной порции газа равно работе \(A^{\prime}\) внешних сил:

\[
\Delta U=A^{\prime}=p_{1} V_{1}-p_{2} V_{2} .
\]

При малом перепаде давлений

\[
\Delta U=C_{V} \Delta T+\Delta U_{\text {вз }},
\]

где \(\Delta U_{\text {вз }}\) - приращение энергии взаимодействия моля молекул газа. Имея в виду, что \(\Delta U=A^{\prime}\), получим из двух предыдущих формул, что


\begin{equation*}
\Delta T=\frac{\Delta U-\Delta U_{\mathrm{B} 3}}{C_{V}} \tag{1.45}
\end{equation*}


\begin{figure}[h]
\begin{center}
  \includegraphics[alt={},max width=\textwidth]{bd023a69-cc46-4b6f-a48e-cc548fa9259d-034_269_353_941_104}
\captionsetup{labelformat=empty}
\caption{Рис. 1.15}
\end{center}
\end{figure}

Из (1.45) следует, что эффект (знак \(\Delta T\) ) зависит от соотношения между приращением внутренней энергии \(\Delta U\), равной работе внешних сил, и приращением энергии взаимодействия \(\Delta U_{\text {вз }}\). Это поясняет рис. 1.15, где величина вертикального отрезка характеризует собственную кинетическую энергию газа \(\tilde{K}\), а значит и температуру \(T\). В частности, если \(\Delta U=\Delta U_{\text {вз }}\), то \(\Delta T=0\).

Довольно громоздкий расчет дает возможность найти зависимость давления \(p_{1}\) от \(T_{1}\), при которых эффект \((\Delta T)\) равен нулю. Соответствующая зависимость \(p_{1}\left(T_{1}\right)\) - ее называют кривой инверсии - показана на рис. 1.16. Начальные состояния ( \(p_{1}\), \(T_{1}\) ) под кривой инверсии приводят к по-


температуры и измерялись в опытах. Изменение температуры — приращение T — при таком течении и называют эффектом Джоуля–Томсона.

Сначала попытаемся выяснить, что можно ожидать относительно T. Для этого мысленно выделим слева от перегородки объем V1 моля газа. После прохождения через перегородку эта порция газа займет объем V2. Согласно 1-му началу термодинамики с учетом того, что в этом процессе


\begin{figure}[h]
\begin{center}
  \includegraphics[alt={},max width=\textwidth]{bd023a69-cc46-4b6f-a48e-cc548fa9259d-034_271_412_1472_75}
\captionsetup{labelformat=empty}
\caption{Рис. 1.16}
\end{center}
\end{figure}

Более точный расчет \(\Delta T\) для частного случая приведен в задаче 1.11.

\section*{Задачи}
1.1. Уравнение процесса. Найти максимально возможную температуру одного моля идеального газа, совершающего процесс \(p=p_{0}-\alpha V^{2}\), где \(p_{0}\) и \(\alpha-\) положительные постоянные.

Решение. Для этого следует сначала найти зависимость \(T(V)\), а затем из условия \(\mathrm{d} T / \mathrm{d} V=0\) определим \(T_{\text {макс }}\). Итак, данный в условии процесс с помощью уравнения состояния \(p V=R T\) перепишем в виде


\begin{equation*}
R T=p_{0} V-\alpha V^{3} . \tag{1}
\end{equation*}


Дифференцируем это уравнение по \(V\) :


\begin{equation*}
R \frac{\mathrm{~d} T}{\mathrm{~d} V}=p_{0}-3 \alpha V^{2}=0 . \tag{2}
\end{equation*}


Отсюда \(V_{m}\), соответствующее максимуму \(T\), равно \(V_{m}=\sqrt{p_{0} / 3 \alpha}\). Подстановка этого выражения в (1) дает

\[
T_{\text {макс }}=\frac{2 p_{0}}{3 R} \sqrt{\frac{p_{0}}{3 \alpha}} .
\]

1.2. Работа над газом. В вертикальном цилиндре под невесомым поршнем \(P\) находится один моль идеального газа при температуре \(T\) (рис. 1.17). Пространство над поршнем сообщается с атмосферой. Какую работу необходимо совершить, чтобы, медленно поднимая поршень, изотермически увеличить объем газа под ним в \(n\) раз?

Решение. По определению, работа, совершаемая силой \(F\), есть


\begin{equation*}
A^{\prime}=\int_{h_{1}}^{h_{2}} F(h) \mathrm{d} h, \tag{1}
\end{equation*}


\begin{figure}[h]
\begin{center}
  \includegraphics[alt={},max width=\textwidth]{bd023a69-cc46-4b6f-a48e-cc548fa9259d-035_370_235_1044_925}
\captionsetup{labelformat=empty}
\caption{Рис. 1.17}
\end{center}
\end{figure}

где силу \(F\) можно выразить через давление газа \(p\) под поршнем и \(p_{0}\) - над поршнем. А именно, из условия медленности сохраняется баланс сил:


\begin{equation*}
F+p S=p_{0} S, \tag{2}
\end{equation*}


где \(S\) - площадь сечения поршня. Подстановка \(F\) из (2) в (1) дает


\begin{equation*}
A^{\prime}=\int\left(p_{0}-p\right) S \mathrm{~d} h=\int p_{0} \mathrm{~d} V-\int \frac{R T}{V} \mathrm{~d} V . \tag{3}
\end{equation*}


В результате интегрирования в пределах от \(V\) до \(n V\) получим

\[
A^{\prime}=p_{0} V(n-1)-R T \ln n,
\]

или с учетом \(p_{0} V=R T\)

\[
A^{\prime}=R T(n-1-\ln n) .
\]

1.3. Политропический процесс. Идеальный газ с показателем адиабаты \(\gamma\) расширили в \(\eta\) раз по закону \(p=\alpha V\), где \(\alpha\) - постоянная. Первоначальный объем газа \(V_{1}\). Найти:\\
a) приращение внутренней энергии газа;\\
б) работу, совершенную газом;\\
в) молярную теплоемкость газа в этом процессе.

Р еш ен и е. а) По определению


\begin{equation*}
\Delta U=C_{V}\left(T_{2}-T_{1}\right)=\frac{C_{V}}{R}\left(p_{2} V_{2}-p_{1} V_{1}\right) . \tag{1}
\end{equation*}


Имея в виду, что \(V_{2} / V_{1}=\eta\) и \(p\) c \(V\), перепишем (1) так:

\[
\Delta U=\alpha V_{1}^{2}\left(\eta^{2}-1\right) /(\gamma-1) ;
\]

б) \(A=\int p \mathrm{~d} V=\frac{p_{1}}{V_{1}} \frac{V_{2}^{2}-V_{1}^{2}}{2}=\frac{1}{2} \alpha V_{1}^{2}\left(\eta^{2}-1\right)\);\\
в) согласно (1.13) решение сводится к нахождению производной \(\mathrm{d} V / \mathrm{d} T\). Из уравнения процесса и уравнения состояния идеального газа следует, что

\[
\alpha V^{2}=R T .
\]

Дифференцируя по \(T\), получаем \(2 \alpha V(\mathrm{~d} V / \mathrm{d} T)=R\), или

\[
p(\mathrm{~d} V / \mathrm{d} T)=R / 2 .
\]

Следовательно

\[
C=C_{V}+\frac{R}{2}=\frac{R}{2} \frac{\gamma+1}{\gamma-1} .
\]

1.4. Показать, что процесс, при котором работа идеального газа пропорциональна соответствующему приращению его внутренней энергии, описывается уравнением \(p V^{n}=\) const, где \(n\) - постоянная.

Р е ш е н и е. Согласно условию \(\mathrm{d}^{\prime} A=a \mathrm{~d} U\), где \(a\) - некоторая постоянная. Раскроем это выражение:


\begin{equation*}
p \mathrm{~d} V=a \frac{R}{\gamma-1} \mathrm{~d} T \tag{1}
\end{equation*}


Перейдем к переменным \(p, V\), взяв дифференциал уравнения \(p V=R T\) :


\begin{equation*}
R \mathrm{~d} T=p \mathrm{~d} V+V \mathrm{~d} p . \tag{2}
\end{equation*}


Подстановка (2) в (1) дает


\begin{equation*}
V \mathrm{~d} p=\frac{1-\alpha}{\alpha} p \mathrm{~d} V \tag{3}
\end{equation*}


где \(\alpha=a /(\gamma-1)\). Разделив (3) на \(p V\), получим


\begin{equation*}
\frac{\mathrm{d} p}{p}=\frac{1-\alpha}{\alpha} \frac{\mathrm{d} V}{V} \tag{4}
\end{equation*}


Интегрирование этого уравнения приводит к следующей формуле

\[
\ln p=\frac{1-\alpha}{\alpha} \ln V+\text { const. }
\]

В результате потенцирования получим

\[
p V^{n}=\mathrm{const}
\]

где \(n\) - некоторая постоянная, здесь \(n=1-(\gamma-1) / a\).\\
1.5. Идеальный газ с показателем адиабаты \(\gamma\) расширяют так, что сообщенное газу тепло равно убыли его внутренней энергии. Найти:\\
а) молярную теплоемкость газа в этом процессе;\\
б) уравнение процесса в переменных \(T, V\).

P е ш е н и е. а) Согласно условию \(\mathrm{d}^{\prime} Q=-\mathrm{d} U\) имеем

\[
C=\frac{\mathrm{d}^{\prime} Q}{\mathrm{~d} T}=\frac{-\mathrm{d} U}{\mathrm{~d} T}=-C_{V}=-\frac{R}{\gamma-1}<0 .
\]

б) Исходим из соотношения (1.13):

\[
C=C_{V}+p \frac{\mathrm{~d} V}{\mathrm{~d} T}=-C_{V}
\]

Отсюда

\[
2 C_{V}=-p \frac{\mathrm{~d} V}{\mathrm{~d} T}=-\frac{R T}{V} \frac{\mathrm{~d} V}{\mathrm{~d} T}
\]

Это приводит к уравнению

\[
\frac{\mathrm{d} T}{T}=-\frac{R}{2 C_{V}} \frac{\mathrm{~d} V}{V},
\]

интегрирование которого дает

\[
\ln T=-\frac{R}{2 C_{V}} \ln V+\text { const. }
\]

Отсюда следует, что \(T V^{R / 2 C_{V}}=\) const, или

\[
T V^{(\gamma-1) / 2}=\text { const. }
\]

1.6. Один моль идеального газа с известным значением \(C_{V}\) находится в левой половине цилиндра (рис. 1.18). Справа от поршня вакуум. В отсутствие газа поршень находится вплотную к левому торцу цилиндра, и пружина в этом положе-

\begin{figure}[h]
\begin{center}
  \includegraphics[alt={},max width=\textwidth]{bd023a69-cc46-4b6f-a48e-cc548fa9259d-038_157_415_739_138}
\captionsetup{labelformat=empty}
\caption{Рис. 1.18}
\end{center}
\end{figure}

нии не деформирована. Боковые стенки цилиндра и поршень адиабатные. Трения нет. Газ нагревают через левый торец цилиндра. Найти теплоемкость газа в этом квазистатическом процессе.

P е ш е н и е. Согласно формуле (1.13) надо найти второе слагаемое, т. е. \(p(\mathrm{~d} V / \mathrm{d} T)\). Поступим так. Пусть жесткость пружины \(\varkappa\), тогда, если сжатие пружины равно \(x\), то в процессе должно выполняться равенство \(x x=p S\), где \(S\) - площадь сечения цилиндра (и поршня). Умножив обе части этого равенства на \(S\), получим


\begin{equation*}
\varkappa V=p S^{2}, \tag{1}
\end{equation*}


где \(V\) - объем газа. Выразим в (1) давление \(p\) через \(T\) и \(V\) с помощью уравнения состояния идеального газа. Тогда получим


\begin{equation*}
V^{2}=\left(R S^{2} / \varkappa\right) T . \tag{2}
\end{equation*}


Дифференцируем это равенство по \(T\) :


\begin{equation*}
2 V \frac{\mathrm{~d} V}{\mathrm{~d} T}=\frac{R S^{2}}{\varkappa}=R \frac{V}{p}, \tag{3}
\end{equation*}


где последнее равенство написано на основании (1). Из уравнения \((3)\) имеем

\[
p \frac{\mathrm{~d} V}{\mathrm{~d} T}=\frac{R}{2},
\]

и искомая теплоемкость

\[
C=C_{V}+R / 2 .
\]

1.7. Колебания. В закрытом с обоих торцов горизонтальном цилиндре, заполненном идеальным газом с показателем адиабаты \(\gamma\), находится поршень массы \(m\) с площадью сечения \(S\). В положении равновесия давление газа равно \(p_{0}\) и поршень делит цилиндр на две одинаковые части, каждая объемом \(V_{0}\). Найти частоту \(\omega\) малых колебаний поршня около положения равновесия, считая процесс адиабатическим и трение ничтожно малым.\\
Решении е. Запишем основное уравнение динамики для случая, когда поршень находится, например, справа от положения равновесия на расстоянии \(x\) (рис. 1.19):

\begin{figure}[h]
\begin{center}
  \includegraphics[alt={},max width=\textwidth]{bd023a69-cc46-4b6f-a48e-cc548fa9259d-039_189_445_584_715}
\captionsetup{labelformat=empty}
\caption{Рис. 1.19}
\end{center}
\end{figure}


\begin{equation*}
m \ddot{x}=-\left(p_{2}-p_{1}\right) S . \tag{1}
\end{equation*}


Из условия \(p V^{\gamma}=p_{0} V_{0}^{\gamma}\) следует, что


\begin{equation*}
p_{2}-p_{1}=p_{0} V_{0}^{\gamma}\left(\frac{1}{V_{2}^{\gamma}}-\frac{1}{V_{1}^{\gamma}}\right)=p_{0} V_{0}^{\gamma} \frac{V_{1}^{\gamma}-V_{2}^{\gamma}}{V_{0}^{2 \gamma}}, \tag{2}
\end{equation*}


где принято во внимание, что \(V_{1}\) мало отличается от \(V_{2}\), поэтому в знаменателе \(V_{1}^{\gamma} V_{2}^{\gamma}\) заменено на \(V_{0}^{2 \gamma}\). Теперь учтем, что


\begin{gather*}
V_{1}^{\gamma}=\left(V_{0}+S x\right)^{\gamma} \approx V_{0}^{\gamma}\left(1+\frac{\gamma S x}{V_{0}}\right),  \tag{3}\\
V_{2}^{\gamma}=\left(V_{0}-S x\right)^{\gamma} \approx V_{0}^{\gamma}\left(1-\frac{\gamma S x}{V_{0}}\right) .
\end{gather*}


Тогда (2) можно преобразовать так:


\begin{equation*}
p_{2}-p_{1}=p_{0} \frac{2 \gamma S x}{V_{0}} . \tag{4}
\end{equation*}


Подстановка этого выражения в исходное (1) дает


\begin{equation*}
m \ddot{x}=-\left(2 \gamma p_{0} S^{2} / V_{0}\right) x . \tag{5}
\end{equation*}


Мы пришли к уравнению гармонических колебаний \(\ddot{x}+\omega^{2} x=0\), из которого следует, что искомая частота

\[
\omega=S \sqrt{2 \gamma p_{0} / m V_{0}} .
\]

1.8. Степени свободы молекулы. Найти число атомов в молекуле газа, у которого при «замораживании» колебательных степеней свободы постоянная адиабаты \(\gamma\) увеличивается в \(\eta=1,20\) раза.\\
Решение. Исходим из условия


\begin{equation*}
\eta=\gamma_{3} / \gamma, \tag{1}
\end{equation*}


где \(\gamma_{3}\) - постоянная адиабаты после «замораживания» колебательных степеней свободы. Дальнейшее зависит от того, какие молекулы - линейные или нелинейные. Пусть число атомов в молекуле равно \(N\). Для линейных молекул число колебательных степеней свободы \(z_{\text {кол }}=3 N-5\) и число \(i=5+2 z_{\text {кол }}=6 N-5\), а после «замораживания» \(i_{3}=5\). Тогда согласно (1.40)


\begin{equation*}
\eta=\frac{\gamma_{3}}{\gamma}=\frac{\left(i_{3}+2\right) / i_{3}}{(i+2) / i}=\frac{(5+2) / 5}{(6 N-5+2) /(6 N-5)}=\frac{7}{5} \frac{6 N-5}{6 N-3} . \tag{2}
\end{equation*}


Зная значение \(\eta\), находим

\[
N=\frac{(15 / 7) \eta-5}{(30 / 7) \eta-6}=2,84 .
\]

Это невозможно, значит расчет надо вести, считая, что молекулы нелинейные.

Поступая аналогично, получим

\[
\eta=4(N-1) /(3 N-2),
\]

откуда

\[
N=(2 \eta-4) /(3 \eta-4)=4 .
\]

Таким образом, мы имеем дело с нелинейными молекулами, состоящими из четырех атомов.\\
1.9. Ван-дер-ваальсовский газ. Получить для моля этого газа уравнение адиабаты в переменных \(T, V\), если известна его молярная теплоемкость \(C_{V}\).

P e ш е н и е. Согласно первому началу и уравнению (1.44)

\[
\mathrm{d}^{\prime} Q=\mathrm{d} U+p \mathrm{~d} V=C_{V} T+\frac{a}{V^{2}} \mathrm{~d} V+\left(\frac{R T}{V-b}-\frac{a}{V^{2}}\right) \mathrm{d} V=0
\]

Отсюда

\[
C_{V} \frac{\mathrm{~d} T}{T}=-\frac{R \mathrm{~d} V}{V-b}
\]

Интегрируя это уравнение, получаем

\[
\ln T^{C_{V} / R}=-\ln (V-b)+\mathrm{const},
\]

или

\[
\ln \left[T^{C_{V} / R}(V-b)\right]=\text { const. }
\]

Таким образом, уравнение адиабаты имеет вид

\[
T(V-b)^{R / C_{V}}=\text { const. }
\]

1.10. Определить для ван-дер-ваальсовского газа разность молярных теплоемкостей \(C_{p}-C_{V}\).\\
Решени е. По определению теплоемкости

\[
C_{p}=\left(\frac{\mathrm{d}^{\prime} Q}{\mathrm{~d} T}\right)_{p}=\left(\frac{\partial U}{\partial T}\right)_{p}+p\left(\frac{\partial V}{\partial T}\right)_{p}
\]

Учитывая, что \(U=C_{V} T-a / V\), получим


\begin{equation*}
C_{p}=C_{V}+\left(p+\frac{a}{V^{2}}\right)\left(\frac{\partial V}{\partial T}\right)_{p} \tag{1}
\end{equation*}


Найдем ( \(\partial V / \partial T)_{p}\). Для этого продифференцируем по \(T\) уравнение Ван-дер-Ваальса (1.42). В результате получим


\begin{equation*}
\left(\frac{\partial V}{\partial T}\right)_{p}=\frac{R}{\left(p+\frac{a}{V^{2}}\right)-\frac{2 a}{V^{2}}(V-b)} \tag{2}
\end{equation*}


Подстановка (2) в (1) приводит к искомому результату:

\[
C_{p}-C_{V}=\frac{R}{1-\frac{2 a(V-b)^{2}}{R T V^{3}}}
\]

1.11. Эффект Джоуля-Томсона. Считая газ до расширения ван-дер-ваальсовским, а после расширения (за пористой перегородкой) - идеальным, найти приращение температуры газа в результате такого расширения. До расширения газ имел молярный объем \(V_{1}\) и температуру \(T_{1}\). Известны также молярная теплоемкость \(C_{p}\) и постоянные Ван-дер-Ваальса данного газа.\\
P e ш e н и е. Согласно 1-му началу термодинамики в данном процессе


\begin{equation*}
Q=\Delta U+A=U_{2}-U_{1}+p_{2} V_{2}-p_{1} V_{1}=0 . \tag{1}
\end{equation*}


Теперь учтем, что согласно условиям задачи


\begin{equation*}
U_{1}=C_{V} T_{1}-a / V_{1}, \quad U_{2}=C_{V} T_{2} . \tag{2}
\end{equation*}


Из уравнений состояния моля ван-дер-ваальсовского и идеального газов следует:


\begin{equation*}
p_{1} V_{1}=R T_{1}+b p_{1}-\frac{a}{V_{1}}+\frac{a b}{V_{1}^{2}}, \quad p_{2} V_{2}=R T_{2} . \tag{3}
\end{equation*}


После подстановки выражений (2) и (3) в формулу (1) и сокращений получим

\[
T_{2}-T_{1} \approx \frac{1}{C_{p}}\left(\frac{R T_{1} b}{V_{1}-b}-\frac{2 a}{V_{1}}\right) .
\]

Здесь принято приближенно, что равенство \(C_{p}=C_{V}+R\) справедливо и для ван-дер-ваальсовского газа, хотя это не совсем так (см. результат решения предыдущей задачи).

\section*{Статистическая физика. Распределения Максвелла и Больцмана}
\section*{§ 2.1. Вероятность. Средние значения}
Статистическая физика- это раздел физики, в котором изучают свойства макросистем, исходя из индивидуальных свойств составляющих макросистему частиц и взаимодействий между ними. Описание движения каждой частицы макросистемы (а их порядка \(10^{22} \div 10^{23}\) ) - задача совершенно немыслимая. Вместо этого статистическая физика оперирует со средними значениями параметров очень большого числа частиц. Колоссальное число частиц в макросистеме приводит, несмотря на очевидный хаос, к появлению новых, статистических закономерностей. Их изучение и делает возможным описание макросистем на основе сведений о свойствах отдельных частиц.

О вероятности. Основу статистической физики составляет теория вероятностей. Исходные понятия этой теории - событие и вероятность.

Событие - это, например, выпадение одного из шести номеров при бросании игрального кубика. Или при измерении скорости молекул газа: разбив возможные значения скоростей на отдельные интервалы \(\Delta v_{i}(i=1,2, \ldots)\) и обнаружив, что скорость молекулы попала в \(i\)-й интервал, мы говорим об \(i\)-м событии.

В дальнейшем нас будут интересовать лишь такие события, которые являются:


Статистическая физика— это раздел физики, в котором изучают свойства макросистем, исходя из индивидуальных свойств составляющих макросистему частиц и взаимодействий между ними. Описание движения каждой частицы макросистемы (а их порядка 1022  1023) — задача совершенно немыслимая. Вместо этого статистическая физика оперирует со средними значениями параметров оч


\begin{enumerate}
  \item случайньлми, т. е. событиями, условия наступления которых по тем или иным причинам неизвестны и которые поэтому нельзя заранее с уверенностью предсказать;
  \item равновозможными или равновероятными - для которых нет никаких оснований ожидать, что при испытаниях они будут вести себя по-разному (например, при бросании игрального кубика или монеты, но никак не бутерброда с маслом).
\end{enumerate}

Вероятность данного случайного события характеризуется кратностью его повторения. Если в \(N\) случаях \(i\)-е событие происходит \(N_{i}\) раз, то вероятностью \(P_{i}\) этого события называют величину


\begin{equation*}
P_{i}=\lim _{N \rightarrow \infty} \frac{N_{i}}{N} \tag{2.1}
\end{equation*}


Так как на практике \(N\) всегда конечно, то для вычисления вероятности стараются, чтобы \(N\) и \(N_{i}\) были достаточно большими. Тогда можно считать, что


\begin{equation*}
P_{i} \approx N_{i} / N . \tag{2.2}
\end{equation*}


Ясно, что сумма вероятностей всех возможных результатов измерений равна единице:


\begin{equation*}
\sum P_{i}=\sum\left(N_{i} / N\right)=1 \tag{2.3}
\end{equation*}


Теперь обратимся к вычислению вероятностей сложных событий. Рассмотрим две основные теоремы: о сложении и умножении вероятностей на примере игрального кубика.


\begin{enumerate}
  \item Теорема сложения вероятностей. Если в результате \(N\) бросаний кубика в \(N_{i}\) случаях выпадет число \(i\), а в \(N_{k}\) случаях - \(k\), то вероятность выпадения \(i\) или \(k\) равна
\end{enumerate}


\begin{equation*}
P(i \text { или } k)=\frac{N_{i}+N_{k}}{N}=P_{i}+P_{k} . \tag{2.4}
\end{equation*}


Это значит, что при бросании кубика вероятность выпадения, скажем, 2 или 5 равна \(P=\frac{1}{6}+\frac{1}{6}=\frac{1}{3}\).

В общем случае эта теорема утверждает: вероятности несовместимьх событий складываются.

Пример. При бросании игрального кубика вероятность:\\
a) выпадения четной цифры равна \(\frac{1}{6}+\frac{1}{6}+\frac{1}{6}=\frac{1}{2}\).\\
б) того, что не выпадет 2 , равна \(\frac{5}{6}\).\\
2. Теорема умножения вероятностей. Найдем вероятность того, что при двух бросаниях кубика выпадет последовательно \(i\) и \(k\). Рассмотрим \(N\) двойных бросаний. Пусть первый из каждой пары бросков дал \(i\) в \(N_{i}\) случаях (так что \(P_{i} \approx N_{i} / N\) ). Теперь

выделим из этих \(N_{i}\) случаев те \(N_{k}\) событий, когда второй бросок кубика давал \(k\) (так что \(P_{k}=N_{k} / N_{i}\) ). Тогда искомая вероятность


\begin{equation*}
P(i, \text { затем } k)=\frac{N_{i}}{N} \frac{N_{k}}{N_{i}}=P_{i} \cdot P_{k} . \tag{2.5}
\end{equation*}


Значит, вероятность того, что при двух бросаниях кубика выпадут, допустим, сначала 2 , а затем 5 , равна \(\frac{1}{6} \cdot \frac{1}{6}=\frac{1}{36}\).

В общем случае теорема умножения вероятностей утверждает: вероятность совмещения двух или нескольких событий равна произведению вероятностей каждого их них в отдельности.

Пример. Вероятность того, что при двух бросаниях кубика:\\
а) выпадут две 5, равна \(\frac{1}{6} \cdot \frac{1}{6}=\frac{1}{36}\);\\
б) не выпадет ни одной 5 , равна \(\frac{5}{6} \cdot \frac{5}{6}=\frac{25}{36}\);\\
в) выпадет одна 5 , равна ( \(\frac{1}{6} \cdot \frac{5}{6}\) ) \(+\left(\frac{5}{6} \cdot \frac{1}{6}\right)=\frac{10}{36}\).

Заметим, что \(P_{\mathrm{a}}+P_{6}+P_{\mathrm{b}}=1\), как и должно быть.\\
Средние значения случайных величин. Зная вероятности появления различных результатов измерения дискретной величины \(x\), можно найти их среднее значение \(\langle x\rangle\). По определению среднего


\begin{equation*}
\langle x\rangle=\frac{1}{N} \sum N_{i} x_{i}=\sum P_{i} x_{i} . \tag{2.6}
\end{equation*}


Функция распределения. Рассмотрим случай, когда случайная величина \(x\) имеет непрерывный характер (например, скорости молекул). Для этого разобьем всю область изменения \(x\) на отдельные интервалы и будем считать число попаданий случайной величины в тот или иной интервал. Интервалы должны быть во избежание заметных флуктуаций достаточно большими, чтобы в каждом интервале число попаданий было \(N_{i} \gg 1\) и чтобы с помощью (2.2) можно было определить вероятность попадания случайной величины в данный интервал. Вместе с тем, интервалы должны быть достаточно небольшими, чтобы более детально характеризовать распределение величины \(x\).

Итак, мы имеем достаточно большое число достаточно небольших интервалов и, допустим, нам известна вероятность \(P_{x}\) попадания в тот или иной интервал \(\Delta x\). Сама величина \(\Delta P_{x}\)

весьма мала, поэтому в качестве характеристики случайной величины берут отношение \(\Delta P_{x} / \Delta x\), которое для достаточно малых \(\Delta x\) не зависит от величины самого интервала \(\Delta x\).

Это отношение при \(\Delta x \rightarrow 0\) называют функцией распределения \(f(x)\) случайной величины \(x\) :


\begin{equation*}
f(x)=\lim _{\Delta x \rightarrow 0} \frac{\Delta P_{x}}{\Delta x}=\frac{\mathrm{d} P_{x}}{\mathrm{~d} x} . \tag{2.7}
\end{equation*}


Видно, что функции распределения \(f(x)\) можно приписать смысл плотности вероятности, т. е. вероятности интересующей нас величины оказаться в единичном интервале вблизи значения \(x\).

В разных случаях функция распределения имеет совершен-


\begin{figure}[h]
\begin{center}
  \includegraphics[alt={},max width=\textwidth]{bd023a69-cc46-4b6f-a48e-cc548fa9259d-046_334_519_713_68}
\captionsetup{labelformat=empty}
\caption{Рис. 2.1}
\end{center}
\end{figure}


\begin{equation*}
\mathrm{d} P_{x}=f(x) \mathrm{d} x . \tag{2.8}
\end{equation*}


Вероятность того, что величина \(x\) попадает в интервал \((a, b)\) :


\begin{equation*}
P=\int_{a}^{b} f(x) \mathrm{d} x \tag{2.9}
\end{equation*}


Ясно, что вероятность того, что величина \(x\) может принять хотя бы какое-нибудь значение (достоверное событие), равна единице. Это называют условием нормировки:


\begin{equation*}
\int f(x) \mathrm{d} x=1, \tag{2.10}
\end{equation*}


где интегрирование производится по всему интервалу возможных значений величины \(x\). Из этого условия следует, что вся площадь под кривой \(f(x)\) равна единице (см. рис. 2.1). Заметим, что (2.10) является аналогом формулы (2.3).

Средние значения. Среднее значение величины \(x\) можно найти, зная ее нормированную на единицу функцию распределения \(f(x)\). Обратимся к формуле (2.6). Она справедлива и для случая, когда интервал изменения величины \(x\) будет разбит на небольшие участки. Уменьшая участки, мы должны в конце концов заменить \(P_{i}\) на \(\mathrm{d} P\) и \(\sum\) - на интеграл \(\int\). Тогда


\begin{equation*}
\langle x\rangle=\int x \mathrm{~d} P=\int x f(x) \mathrm{d} x, \tag{2.11}
\end{equation*}


где интегрирование проводится по интересующему нас интервалу значений \(x\). Аналогичные формулы справедливы для любой функции \(\varphi(x)\), например

\[
\left\langle x^{2}\right\rangle=\int x^{2} f(x) \mathrm{d} x .
\]

Пример. Идеальный газ находится в сферическом сосуде радиуса \(R\). Найдем распределение расстояний \(r\) молекул от центра сосуда и среднее значение \(\langle r\rangle\).

Выделим мысленно тонкий концентрический шаровой слой радиуса \(r\) и толщиной \(\mathrm{d} r\). Сначала найдем вероятность того, что молекулы попадают в этот слой. Она равна доле молекул в этом слое: \(\mathrm{d} P_{r}=\mathrm{d} N_{r} / N\). В силу равномерного распределения молекул это отношение равно отношению соответствующих объемов:

\[
\mathrm{d} P_{r}=\frac{\mathrm{d} V}{V}=\frac{4 \pi r^{2} \mathrm{~d} r}{(4 \pi / 3) R^{3}}=\frac{3 r^{2} \mathrm{~d} r}{R^{3}} .
\]

Согласно (2.7) искомая функция распределения

\[
f(r)=\mathrm{d} P_{r} / \mathrm{d} r=3 r^{2} / R^{3} .
\]

Среднее значение \(\langle r\rangle\) находим по формуле (2.11):

\[
\langle r\rangle=\int_{0}^{R} r \frac{3 r^{2}}{R^{3}} \mathrm{~d} r=\frac{3}{R^{3}} \frac{R^{4}}{4}=\frac{3}{4} R .
\]

Флуктуации. Вероятность случайного события и экспериментально наблюдаемая доля результатов, когда событие осуществляется, - это не одно и то же. Последняя (доля результатов) испытывает случайные отклонения от предсказываемой вероятности. Именно такого рода отклонения происходят в любых макросистемах. Эти отклонения и обусловливают флуктуации.

Согласно теории вероятности, с увеличением числа \(N\) испытаний относительная флуктуация любой величины уменьшается по закону \(1 / \sqrt{N}\). Именно грандиозность числа \(N\) молекул и объясняет, почему макроскопические законы, получаемые на основе статистических представлений о движении частиц макросистемы, оказываются точными.

В дальнейшем мы часто будем использовать понятие бесконечно малого объема \(\mathrm{d} V\) макросистемы. Под этим будет пониматься такой объем, размеры которого ничтожны по сравнению с размерами самой макросистемы, но все же намного превосходящие характерный размер ее микростроения. Каждая бесконечно малая область, предполагается, содержит число частиц \(\mathrm{d} N\) настолько большое, что относительной флуктуацией их можно пренебречь.

Демонстрация статистического распределения. Великолепная демонстрация статистического характера распределения может быть показана на так называемой доске Гальтона. Она представляет собой вертикальную панель, в которую равномерно и достаточно густо вмонтированы горизонтальные стержни - точки на рис. 2.2. Внизу под решеткой из стержней расположены узкие одинаковые вертикальные ячейки. Вверху над стержнями помещена воронка, в которую можно сыпать, например, зерна пшена. И все это (воронка, стержни и ячейки) прикрыто с передней стороны стеклом.

Если бросить в воронку одно зер-

\begin{figure}[h]
\begin{center}
  \includegraphics[alt={},max width=\textwidth]{bd023a69-cc46-4b6f-a48e-cc548fa9259d-048_499_390_1230_70}
\captionsetup{labelformat=empty}
\caption{Рис. 2.2}
\end{center}
\end{figure}

нышко, то при падении оно испытает много столкновений со стержнями, после чего попадет в одну из ячеек. В какую именно - предсказать совершенно невозможно из-за множества случайных факторов, влияющих на его движение. Можно говорить лишь о вероятности попадания зернышка в ту или иную ячейку. Естественно ожидать, что попадание в центральные ячейки более вероятно, чем в крайние. Опыт это подтверждает: если сыпать пшено в воронку непрерывно, то в цен-

тральные ячейки действительно попадает больше всего зернышек.

При очень большом числе зернышек возникает вполне определенная статистическая закономерность распределения их по ячейкам, как показано на рис. 2.2. Повторив опыт, обнаружим то же распределение, ту же закономерность (это распределение называют нормальньιм законом Гаусса).

Таким образом, мы имеем наглядную, убедительную (и невольно вызывающую удивление) демонстрацию того, как проявляют себя статистические законы через большое число случайных событий.

\section*{§ 2.2. Распределение Максвелла}
Закон распределения по скоростям молекул газа, находящегося в термодинамическом равновесии, был найден Максвеллом (1859). Ход его рассуждений довольно сложен и полностью приводить его мы не будем, а ограничимся в основном рассмотрением подхода к решению этой проблемы, а также физического смысла закона Максвелла и некоторых его следствий.

Следуя Максвеллу, представим себе пространство скоростей с прямоугольными координатными осями, по которым будем откладывать значения проекций \(v_{x}, v_{y}, v_{z}\) отдельных молекул. Тогда скорости каждой молекулы будет соответствовать точка в этом пространстве - конец вектора v. Из-за столкновений молекул положения точек будут стремительно* меняться, но их распределение в целом будет оставаться неизменным, поскольку макросистема находится в термодинамическом (статистическом) равновесии.

Вследствие равноправности всех направлений движения расположение точек относительно начала координат будет сферически симметричным. Поэтому плотность точек может зависеть только от модуля скорости \(v\) (но не от \(\mathbf{v}\) ).

Итак, пусть макросистема (газ) содержит \(N\) молекул. Выделим в некоторой точке - конце вектора \(\mathbf{v}\) - малый объем

\footnotetext{\begin{itemize}
  \item При нормальных условиях каждая молекула газа испытывает порядка \(10^{9}\) столкновений в секунду. С такой частотой непредсказуемо меняется модуль и направление ее скорости.
\end{itemize}
}\begin{figure}[h]
\begin{center}
  \includegraphics[alt={},max width=\textwidth]{bd023a69-cc46-4b6f-a48e-cc548fa9259d-050_426_431_141_80}
\captionsetup{labelformat=empty}
\caption{Рис. 2.3}
\end{center}
\end{figure}

\(\mathrm{d} v_{x} \mathrm{~d} v_{y} \mathrm{~d} v_{z}\) (рис. 2.3, где ось \(v_{z}\) направлена на нас). Этот малый объем должен быть таким, чтобы число \(\mathrm{d} N\) молекул в нем было достаточно большим (во избежание заметных флуктуаций). Если, например, макросистема содержит \(N \sim 10^{20}\) молекул, то число \(\mathrm{d} N \sim 10^{6}\) на много порядков меньше \(N\), и это дает нам право рассматривать \(\mathrm{d} N\) как физически бесконечно малую величину, относительные флуктуации которой пренебрежимо малы. В дальнейшем под \(\mathrm{d} N\) мы будем понимать среднее число молекул в соответствующем нитервале той или иной величины.

Относительное число точек (молекул) в этом объеме \(\mathrm{d} N / N\), или другими словами, вероятность \(\mathrm{d} P\) того, что скорость молекулы, т. е. конец вектора \(\mathbf{v}\), попадет в этот объем, можно записать так:


\begin{equation*}
\mathrm{d} P\left(v_{x}, v_{y}, v_{z}\right)=\frac{\mathrm{d} N\left(v_{x}, v_{y}, v_{z}\right)}{N}=f(v) \mathrm{d} v_{x} \mathrm{~d} v_{y} \mathrm{~d} v_{z}, \tag{2.12}
\end{equation*}


где \(f(v)\) имеет смысл объемной плотности вероятности.\\
Вероятность же того, что молекула (точка) будет иметь проекции скорости в интервале ( \(v_{x}, v_{x}+\mathrm{d} v_{x}\) ), есть


\begin{equation*}
\mathrm{d} P\left(v_{x}\right)=\frac{\mathrm{d} N\left(v_{x}\right)}{N}=\varphi\left(v_{x}\right) \mathrm{d} v_{x}, \tag{2.13}
\end{equation*}


где \(\varphi\left(v_{x}\right)\) - функция распределения по \(v_{x}\). Выражение (2.13) это по существу интеграл (2.12) по \(v_{y}\) и \(v_{z}\), т. е. относительное число молекул (точек) в тонком плоском слое от \(v_{x}\) до \(v_{x}+\mathrm{d} v_{x}\).

Вероятности того, что молекула имеет проекции скорости в интервалах ( \(\left.v_{x}, v_{x}+\mathrm{d} v_{x}\right),\left(v_{y}, v_{y}+\mathrm{d} v_{y}\right)\) и ( \(v_{z}, v_{z}+\mathrm{d} v_{z}\) ), являются независимыми (это было строго доказано), поэтому в соответствии с теоремой об умножении вероятностей независимых событий можно записать


\begin{align*}
\mathrm{d} P\left(v_{x}, v_{y}, v_{z}\right) & =\mathrm{d} P\left(v_{x}\right) \mathrm{d} P\left(v_{y}\right) \mathrm{d} P\left(v_{z}\right)= \\
& =\varphi\left(v_{x}\right) \varphi\left(v_{y}\right) \varphi\left(v_{z}\right) \mathrm{d} v_{x} \mathrm{~d} v_{y} \mathrm{~d} v_{z} . \tag{2.14}
\end{align*}


Из соображения равноправия осей \(v_{x}, v_{y}\) и \(v_{z}\) ясно, что функции φ должны одинаковым образом зависеть от соответствующих проекций скоростей. Сопоставив (2.14) с (2.12), находим


\begin{equation*}
f(v)=\varphi\left(v_{x}\right) \cdot \varphi\left(v_{y}\right) \cdot \varphi\left(v_{z}\right) . \tag{2.15}
\end{equation*}


Опуская дальнейшие преобразования (с учетом условия нормировки), приведем окончательные результаты:


\begin{equation*}
\varphi\left(v_{x}\right)=\left(\frac{m}{2 \pi k T}\right)^{1 / 2} \exp \left(-\frac{m v_{x}^{2}}{2 k T}\right), \tag{2.16}
\end{equation*}


аналогичный вид имеют функции \(\varphi\left(v_{y}\right)\) и \(\varphi\left(v_{z}\right)\). И тогда согласно (2.15)


\begin{equation*}
f(v)=\left(\frac{m}{2 \pi k T}\right)^{3 / 2} \exp \left(-\frac{m v^{2}}{2 k T}\right) . \tag{2.17}
\end{equation*}


График функции \(\varphi\left(v_{x}\right)\) изображен на рис. 2.4. Он совпадает с гауссовой кривой погрешностей. Площадь тонированной полоски на рис. 2.4 - это вероятность того, что проекция скорости молекулы лежит в интервале, ( \(v_{x}, v_{x}+\mathrm{d} v_{x}\) ).

\begin{figure}[h]
\begin{center}
  \includegraphics[alt={},max width=\textwidth]{bd023a69-cc46-4b6f-a48e-cc548fa9259d-051_326_586_903_574}
\captionsetup{labelformat=empty}
\caption{Рис. 2.4}
\end{center}
\end{figure}

Функция (2.16) нормирована на единицу, т. е. площадь под кривой \(\varphi\left(v_{x}\right)\)


\begin{equation*}
\int_{-\infty}^{+\infty} \varphi\left(v_{x}\right) \mathrm{d} v_{x}=1 \tag{2.18}
\end{equation*}


Интегрирование в пределах от \(-\infty\) до \(+\infty\) не означает, что в газе есть молекулы с такими большими скоростями. Это следует рассматривать только как вычислительный прием. Молекул с весьма большими скоростями очень мало, и они практически не вносят никакого вклада в нормировочный интеграл. Это и позволяет записывать такие пределы.

\section*{Распределение молекул по модулю скорости}
Найдем вероятность или относительное число молекул, модуль скорости которых заключен в интервале ( \(v, v+\mathrm{d} v\) ). Таким молекулам соответствуют все точки, попадающие в шаровой слой с радиусами \(v\) и \(v+\mathrm{d} v\) (рис. 2.5). Объем этого слоя равен произведению поверхности слоя на его толщину, т. е. \(4 \pi v^{2} \mathrm{~d} v\), объемная же плотность вероятности \(f(v)\) во всех точках слоя одинакова. Следовательно, согласно теореме сложения вероятностей, вероятность попадания в этот слой


\begin{equation*}
\mathrm{d} P=f(v) \cdot 4 \pi v^{2} \mathrm{~d} v . \tag{2.19}
\end{equation*}


\begin{figure}[h]
\begin{center}
  \includegraphics[alt={},max width=\textwidth]{bd023a69-cc46-4b6f-a48e-cc548fa9259d-052_463_505_653_68}
\captionsetup{labelformat=empty}
\caption{Рис. 2.5}
\end{center}
\end{figure}

\begin{figure}[h]
\begin{center}
  \includegraphics[alt={},max width=\textwidth]{bd023a69-cc46-4b6f-a48e-cc548fa9259d-052_388_507_730_609}
\captionsetup{labelformat=empty}
\caption{Рис. 2.6}
\end{center}
\end{figure}

Величина \(\mathrm{d} P / \mathrm{d} v\) - мы ее обозначим \(F(v)\) - характеризует искомую вероятность, т. е. \(F(v)=4 \pi v^{2} f(v)\). Учитывая (2.17), получим:


\begin{equation*}
F(v)=4 \pi\left(\frac{m}{2 \pi k T}\right)^{3 / 2} v^{2} \exp \left(-\frac{m v^{2}}{2 k T}\right) . \tag{2.20}
\end{equation*}


Эта формула представляет собой закон распределения Максвелла по модулю скорости. Вид функции \(F(v)\) показан на рис. 2.6. Эта функция тоже нормирована на единицу:


\begin{equation*}
\int_{0}^{\infty} F(v) \mathrm{d} v=1 . \tag{2.21}
\end{equation*}


На рис. 2.6 пунктиром представлена «конструкция» (сомножители) функции \(F(v)\), один из сомножителей которой \(f(v)\). Заметим, что в отличие от \(F(v)\) площадь под кривой \(f(v)\) физического смысла не имеет.

Следует отметить, что полученные Максвеллом распределения по скоростям не зависят ни от структуры молекул, ни от того, как они взаимодействуют друг с другом. Поэтому они применимы не только к газам, но и к другим агрегатным состояниям вещества.

Характерные скорости. К ним относятся три скорости: наиболее вероятная \(v_{\text {вер }}\), средняя \(\langle υ\rangle\) и среднеквадратичная \(v_{\text {кв }}\).

Наиболее вероятной скорости соответствует максимум функции распределения \(F(v)\). Эта скорость определяется из условия \(\mathrm{d} F / \mathrm{d} v=0\), откуда следует


\begin{equation*}
v_{\text {вер }}=\sqrt{2 \frac{k T}{m}}=\sqrt{2 \frac{R T}{M}} . \tag{2.22}
\end{equation*}


Средняя скорость по определению*


\begin{equation*}
\langle v\rangle=\int_{0}^{\infty} v F(v) \mathrm{d} v=\sqrt{\frac{8}{\pi} \frac{k T}{m}}=\sqrt{\frac{8}{\pi} \frac{R T}{M}} . \tag{2.23}
\end{equation*}


Среднеквадратичная скорость \(v_{\text {кв }}=\sqrt{\left\langle v^{2}\right\rangle}\); она находится из условия*

\[
\left\langle v^{2}\right\rangle=\int_{0}^{\infty} v^{2} F(v) \mathrm{d} v=3 \frac{k T}{m}
\]

откуда следует, что


\begin{equation*}
v_{\text {кв }}=\sqrt{3 \frac{k T}{m}}=\sqrt{3 \frac{R T}{M}} \tag{2.24}
\end{equation*}


Этот результат можно получить и без интегрирования, а как следствие формулы (1.31).

\begin{itemize}
  \item При вычислениях использовались табличные значения интегралов:
\end{itemize}

\[
\int_{0}^{\infty} x^{3} \mathrm{e}^{-\alpha x^{2}} \mathrm{~d} x=\frac{1}{2 \alpha^{2}}, \quad \int_{0}^{\infty} x^{4} \mathrm{e}^{-\alpha x^{2}} \mathrm{~d} x=\frac{3 \sqrt{\pi}}{8 \alpha^{5 / 2}}
\]

В качестве примера приведем среднюю скорость молекул азота \(\mathrm{N}_{2}\) при \(T=300 \mathrm{~K}:\langle υ\rangle \approx 480 \mathrm{~m} / \mathrm{c}\). Эта величина имеет порядок скорости звука в азоте, \(v_{3 в}=\sqrt{\gamma k T / m}=350 \mathrm{~m} / \mathrm{c}\).

Приведенные характерные скорости отличаются друг от друга в пропорции

\[
v_{\text {вер }}:\langle v\rangle: v_{\text {кв }}=1: 1,13: 1,22 .
\]

Качественно это показано на рис. 2.6.\\
Зависимость распределения от \(\boldsymbol{T}\). Подставив значение \(v_{\text {вер }}\) из (2.22) в формулу (2.20), получим, что


\begin{equation*}
F_{\text {макс }} \sim \sqrt{m / T} . \tag{2.25}
\end{equation*}


В соответствии с этим результатом для разных температур \(T_{1}<T_{2}<T_{3}\) кривые распределения \(F(v)\) будут иметь вид, показанный на рис. 2.7. Видно, что с увеличением \(T\) максимум функции \(F(v)\) смещается в сторону больших скоростей, а его величина уменьшается согласно (2.25). При этом площадь под всеми тремя кривыми остается равной единице. Кривые на рис. 2.7 можно рассматривать и иначе - как соответствующие разным массам молекул газа при одной и той же температуре, причем \(m_{1}>m_{2}>m_{3}\).

\begin{figure}[h]
\begin{center}
  \includegraphics[alt={},max width=\textwidth]{bd023a69-cc46-4b6f-a48e-cc548fa9259d-054_366_578_1052_324}
\captionsetup{labelformat=empty}
\caption{Рис. 2.7}
\end{center}
\end{figure}

\section*{Формула Максвелла в приведенном виде}
Решение ряда задач удобнее проводить, если выражать скорости \(v\) молекул в относительных единицах - единицах наиболее вероятной скорости \(v_{\text {вер }}\). Тогда относительная скорость молекулы


\begin{equation*}
u=v / v_{\text {вер }} . \tag{2.26}
\end{equation*}


При переходе к распределению по другой переменной (здесь от \(v к u\) ) будем исходить из вероятности \(\mathrm{d} P\) пребывания молекулы в интервале скоростей ( \(v, v+\mathrm{d} v\) ), которая равна \(\mathrm{d} P= =F(v) \mathrm{d} v\). Значение этой вероятности не изменится, если правую часть равенства разделить и умножить на \(\mathrm{d} u\). Тогда можно записать

\[
\mathrm{d} P=F(v) \frac{\mathrm{d} v}{\mathrm{~d} u} \mathrm{~d} u=\mathscr{T}(u) d u .
\]

Таким образом, мы переходим от интервала ( \(v, v+\mathrm{d} v\) ) к соответствующему ему интервалу ( \(u, u+\mathrm{d} u\) ).

Согласно (2.26) \(v=u v_{\text {вер }}\) и \(\mathrm{d} v / \mathrm{d} u=v_{\text {вер }}\). После подстановки этих формул в выражение для \(\mathrm{d} P\) получим


\begin{equation*}
\mathscr{F}(u)=(4 / \sqrt{\pi}) u^{2} \exp \left(-u^{2}\right) . \tag{2.27}
\end{equation*}


В таком виде распределение Максвелла является универсальным: оно не зависит ни от температуры, ни от рода газа.

Пример. Найдем относительное число молекул \(\mathrm{d} N / N\) со скоростями, отличающимися от наиболее вероятной не более, чем на \(\eta=1 \%\).

В данном случае \(u=1\), и мы можем согласно (2.27) записать

\[
\frac{\mathrm{d} N}{N}=\mathscr{T}(u) \mathrm{d} u=\frac{4}{\sqrt{\pi}} \cdot 1^{2} \mathrm{e}^{-1} 2 \cdot 0,01=\frac{0,08}{\mathrm{e} \sqrt{\pi}}=1,65 \cdot 10^{-2},
\]

где \(\mathrm{d} u=2 \eta\), поскольку на \(\eta \%\) отклонения могут быть как в одну, так и в другую сторону.

\section*{Распределение по энергиям молекул}
Имеется в виду функция распределения молекул по их кинетическим энергиям \(\varepsilon\) поступательного движения. Для нахождения этой функции - обозначим ее \(\Phi(\varepsilon)\) - мы, как обычно, будем исходить из того, что вероятность \(\mathrm{d} P\) пребывания молекулы в интервале скоростей ( \(v, v+\mathrm{d} v\) ) равна \(\mathrm{d} P=F(v) \mathrm{d} v\). Разделив и умножив правую часть этого равенства на d \(\varepsilon\), получим


\begin{equation*}
\mathrm{d} P=F(v) \frac{\mathrm{d} v}{\mathrm{~d} \varepsilon} \mathrm{~d} \varepsilon=\Phi(\varepsilon) \mathrm{d} \varepsilon . \tag{2.28}
\end{equation*}


Теперь учтем, что \(\varepsilon=m v^{2} / 2\) и \(\mathrm{d} \varepsilon / \mathrm{d} v=m v\) c \(\sqrt{\varepsilon}\). Подстановка этих формул в (2.28) приводит к следующему выражению для искомой функции:

\[
\Phi(\varepsilon) \sim \varepsilon \mathrm{e}^{-\varepsilon / k T} / \sqrt{\varepsilon},
\]

или


\begin{equation*}
\Phi(\varepsilon)=A \sqrt{\varepsilon} \mathrm{e}^{-\varepsilon / k T}, \tag{2.29}
\end{equation*}


где \(A\) - нормировочный множитель, \(A=(2 / \sqrt{\pi})(k T)^{-3 / 2}\). График этой функции показан на рис. 2.8. Наиболее вероятная энергия находится из условия \(\mathrm{d} \Phi / \mathrm{d} \varepsilon=0\) :


\begin{equation*}
\varepsilon_{\text {вер }}=k T / 2 . \tag{2.30}
\end{equation*}


\begin{figure}[h]
\begin{center}
  \includegraphics[alt={},max width=\textwidth]{bd023a69-cc46-4b6f-a48e-cc548fa9259d-056_330_449_709_70}
\captionsetup{labelformat=empty}
\caption{Рис. 2.8}
\end{center}
\end{figure}

Следует обратить внимание на то, что \(\varepsilon_{\text {вер }} \neq \varepsilon\) ( \(v_{\text {вер }}\) ). Это связано с тем, что функция \(\Phi(\varepsilon)\) получена из \(F(v)\) путем умножения последней не на константу, а на \(\mathrm{d} v / \mathrm{d} \varepsilon\), которое зависит от \(\varepsilon\). Именно это приводит \(к\) «перекашиванию» функции \(\Phi(\varepsilon)\) относительно \(F(v)\) и смещению максимумов данных функций.

Пример. Найдем с помощью (2.29) среднее значение кинетической энергии поступательного движения молекул газа при температуре \(T\).

Согласно определению среднего

\[
\langle\varepsilon\rangle=\int_{0}^{\infty} \varepsilon \Phi(\varepsilon) \mathrm{d} \varepsilon=A \int_{0}^{\infty} \varepsilon^{3 / 2} \mathrm{e}^{-\varepsilon / k T} \mathrm{~d} \varepsilon .
\]

Воспользовавшись табличным значением интеграла

\[
\int_{0}^{\infty} x^{3 / 2} \mathrm{e}^{-\alpha x} \mathrm{~d} x=\frac{3 \sqrt{\pi}}{4 \alpha^{5 / 2}}
\]

и выражением для коэффициента \(A\), получим

\[
\langle\varepsilon\rangle=\frac{2}{\sqrt{\pi}} \frac{1}{(k T)^{3 / 2}} \cdot \frac{3 \sqrt{\pi}}{4(1 / k T)^{5 / 2}}=\frac{3}{2} k T,
\]

что и следовало ожидать.


Теперь учтем, что  = mv2/2 и d d / v mv  T . Подстановка этих формул в (2.28) приводит к следующему выражению для искомой функции: .() T e / / kT , или .() = A  e–/kT, (2.29) где А — нормировочный множитель, A kT   ( / )( ) /

3 2 " . График этой функции показан на рис. 2.8. Наиболее вероятная энергия находится из условия d./d  0: вер = kT/2. (2.30) Следует обратить внимание н


\section*{Дополнительные замечания}
\begin{enumerate}
  \item При статистическом подходе не имеет смысла вопрос: какова вероятность (или сколько молекул) имеют вполне определенную скорость. Речь может идти только о числе молекул в заданном интервале скоростей. Поэтому на вопрос: каких молекул больше - с наиболее вероятной скоростью или средней - можно смело отвечать: одинаково (в смысле \(0=0\) ).
\end{enumerate}

Это относится, разумеется, и к энергиям ɛ.\\
2. При подсчете вероятности или числа молекул в заданном интервале скоростей (или энергий) не всегда следует прибегать к интегрированию. Если интервал скоростей очень мал (по сравнению с самой скоростью), то решение сводится просто к умножению:


Дополнительные замечания

\begin{enumerate}

  \item При статистическом подходе не имеет смысла вопрос: какова вероятность (или сколько молекул) имеют вполне определенную скорость. Речь может идти только о числе молекул в заданном интервале скоростей. Поэтому на вопрос: каких молекул больше — с наиболее вероятной скоростью или средней — можно смело отвечать: одинаково (в смысле 0  0). Это относится, разумеется, и к энерг

\end{enumerate}



\begin{equation*}
\Delta P \approx F(v) \cdot \Delta v . \tag{2.31}
\end{equation*}


\begin{enumerate}
  \setcounter{enumi}{2}
  \item Следует иметь в виду, что все функции распределения величины размерные. Например, размерность \(\varphi\left(v_{x}\right)\) и \(F(v)\) равна с/м согласно (2.18) и (2.21), а размерность \(f(v)-\mathrm{c}^{3} / \mathrm{m}^{3}\) согласно (2.15).
\end{enumerate}

\section*{§ 2.3. Опытная проверка распределения Максвелла}
Опишем два наиболее точных эксперимента, осуществленных с целью проверки распределения Максвелла по скоростям.

В опыте Ламмерта (1929) в объеме \(V\) (рис. 2.9, \(a\) ) находится газ в равновесном состоянии. Выходящий из отверстия в стенке объема \(V\) молекулярный пучок проходит коллиматор \(K\) из последовательных отверстий, который образует почти параллельный пучок. Далее пучок попадает на устройство \(C\), сортирую-

\begin{figure}[h]
\begin{center}
  \includegraphics[alt={},max width=\textwidth]{bd023a69-cc46-4b6f-a48e-cc548fa9259d-057_285_1042_1486_92}
\captionsetup{labelformat=empty}
\caption{Рис. 2.9}
\end{center}
\end{figure}

щее молекулы по скоростям, и детектор \(D\) для регистрации молекул после сортировки.

Устройство \(C\) представляет собой два вращающихся диска (на одной оси) со щелями вдоль радиусов. Если щели повернуты на угол φ относительно друг друга, то при угловой скорости \(\omega\) диски повернутся на угол \(\varphi\) в течение промежутка времени \(\Delta t=\varphi / \omega\). Поэтому через обе щели, расстояние между которыми \(l\), пройдут молекулы со скоростью \(v=l / \Delta t=l \omega / \varphi\).

Меняя угловую скорость \(\omega\) или угол \(\varphi\) между радиальными щелями, можно выделить из пучка молекулы разных скоростей. Улавливая детектором эти молекулы в течение одинакового времени, можно найти их относительное количество в пучке.

В другом опыте левая часть установки ( \(V, K\) на рис. 2.9, \(a\) ), формирующая параллельный пучок молекул, остается той же. Но селектор \(C\) и детектор \(D\) совмещены во вращающемся цилиндре со щелью \(S\) (см. рис. 2.9, б). Когда щель \(S\) попадает в падающий пучок \(C\), через нее в цилиндр входит порция молекул. Молекулы с разными скоростями достигают противоположной стенки цилиндра с различным запаздыванием по отношению к моменту прохождения щели \(S\) пучком \(P\) и поэтому попадают на различные участки \(D\) противоположной стенки цилиндра. Измерив степень почернения различных участков \(D\), можно определить распределение молекул в пучке по скоростям.

Разумеется, все эти опыты проводились в условиях высокого вакуума и, кроме того, с учетом различия распределения молекул по скоростям \(v\) в пучке и в объеме \(V\).

Результаты этих и других опытов оказались в полном согласии с законом распределения, установленным Максвеллом.

\section*{§ 2.4. Распределение Больцмана}
В отсутствие внешних сил средняя концентрация \(n\) молекул газа в состянии термодинамического равновесия всюду одинакова. Если же газ находится во внешнем силовом поле, ситуация становится иной.

Рассмотрим, например, поведение молекул газа, находящегося под действием силы тяжести. Если бы не было теплового движения, то все молекулы «упали» бы на поверхность Земли. Наличие же теплового движения мешает этому. В результате

совместного действия этих двух факторов устанавливается некоторое равновесие, и концентрация молекул становится зависящей от высоты. Как? Это и предстоит нам выяснить.

Пусть газ находится во внешнем поле потенциальных (консервативных) сил, действующих для простоты в одном направлении и зависящих только от координаты \(z\). При тепловом равновесии температура \(T\) должна быть одинакова по всей толщине газа, иначе бы возникли потоки тепла, и состояние газа не было бы равновесным.

Для определенности будем считать, что силы внешнего поля направлены вниз, а ось \(Z\) - вверх (рис. 2.10). Выделим мысленно бесконечно узкий слой газа толщиной \(\mathrm{d} z\) с площадью основания столба, равной единице ( \(S=1\) ). Запишем условие равновесия этого слоя, используя гидростатический подход. На слой \(\mathrm{d} z\) действует направленная

\begin{figure}[h]
\begin{center}
  \includegraphics[alt={},max width=\textwidth]{bd023a69-cc46-4b6f-a48e-cc548fa9259d-059_354_443_536_713}
\captionsetup{labelformat=empty}
\caption{Рис. 2.10}
\end{center}
\end{figure}

вверх сила, обусловленная разностью давлений \(\mathrm{d} p(\mathrm{~d} p<0)\), и сила, действующая вниз со стороны внешнего поля. При равновесии должно соблюдаться равенство


\begin{equation*}
\mathrm{d} p=n \mathrm{~d} z \cdot F_{z}, \tag{2.32}
\end{equation*}


где \(F_{z}\) - проекция внешней силы, действующей на каждую молекулу. Заметим, что левая и правая части этого равенства являются отрицательными.

Из механики известно, что \(F_{z}=-\partial U / \partial z\), где \(U\) - потенциальная энергия молекулы во внешнем поле. Поэтому (2.32) можно переписать так:


\begin{equation*}
\mathrm{d} p=-n \mathrm{~d} U . \tag{2.33}
\end{equation*}


Считая газ идеальным, т. е. подчиняющимся формуле \(p=n k T\), представим левую часть (2.33) в виде \(\mathrm{d} p=\mathrm{d} n \cdot k T\). Тогда эта формула примет вид \(\mathrm{d} n \cdot k T=-n \mathrm{~d} U\), или


\begin{equation*}
\frac{\mathrm{d} n}{n}=-\frac{\mathrm{d} U}{k T} . \tag{2.34}
\end{equation*}


Проинтегрировав последнее уравнение, получим


\begin{equation*}
\ln \frac{n}{n_{0}}=-\frac{U-U_{0}}{k T} . \tag{2.35}
\end{equation*}


Будем считать, что \(U_{0}=0\), где \(n=n_{0}\), тогда


\begin{equation*}
n=n_{0} \mathrm{e}^{-U / k T} . \tag{2.36}
\end{equation*}


Этот закон и выражает распределение Больимана.\\
С помощью (2.36) можно найти число молекул в интересующем нас элементарном объеме \(\mathrm{d} V\) :


\begin{equation*}
\mathrm{d} N=n \mathrm{~d} V . \tag{2.37}
\end{equation*}


При этом следует иметь в виду. что объем \(\mathrm{d} V\) может иметь, вообще говоря, не любую форму. Обязательным является выполнение условия: во всех точках объема \(\mathrm{d} V\) концентрация \(n\) должна быть одинаковой.

Перед тем, как обсудить полученный закон (2.36) и выяснить его возможности, напомним, что приведенный вывод формулы (2.36) является чисто гидростатическим: в нем мы по сути рассматривали газ как сплошную среду, отвлекаясь от его молекулярной структуры. Это допустимо лишь для достаточно плотных газов при наличии большого числа столкновений. Необходимо, чтобы средний свободный пробег молекул между последовательными столкновениями был мал по сравнению с толщиной \(\mathrm{d} z\) слоя. Только в этом случае имеет смысл говорить о давлении, которое действует на слой \(\mathrm{d} z\) со стороны соседних слоев. И тем не менее приведенный вывод привел к верному результату*.\\
Пример. Узкую трубку длины \(l\), один торец которой запаян, вращают с постоянной угловой скоростью \(\omega\) в горизонтальной плоскости вокруг вертикальной оси \(O O\), проходящей через открытый конец трубки (рис. 2.11). Это происходит в газе, состоящем из молекул массы \(m\), при температуре \(T\). Концентрация

\footnotetext{\begin{itemize}
  \item Существует и молекулярно-кинетический вывод закона распределения Больцмана, свободный от недостатков, присущих приведенному выводу, но мы не будем его приводить, поскольку он значительно сложнее.
\end{itemize}
}молекул у открытого конца трубки равна \(n_{0}\). Найдем концентрацию молекул у запаянного торца.

Газ в трубке находится в поле центробежных сил инерции \(F=m \omega^{2} r\) (имеется в виду система отсчета, связанная с враща-

\begin{figure}[h]
\begin{center}
  \includegraphics[alt={},max width=\textwidth]{bd023a69-cc46-4b6f-a48e-cc548fa9259d-061_285_443_145_713}
\captionsetup{labelformat=empty}
\caption{Рис. 2.11}
\end{center}
\end{figure}

ющейся трубкой). В этом случае

\[
U_{0}-U=\int_{0}^{l} m \omega^{2} r \mathrm{~d} r=m \omega^{2} l^{2} / 2,
\]

где \(U_{0}\) и \(U\) - потенциальные энергии, соответствующие открытому и закрытому торцам трубки. В результате получим

\[
n=n_{0} \exp \left(\frac{U_{0}-U}{k T}\right)=n_{0} \exp \left(\frac{m \omega^{2} l^{2}}{2 k T}\right) .
\]

Полезно самостоятельно подумать, что изменится, если:\\
а) ось вращения проходит через закрытый конец трубки;\\
б) трубка открыта с обоих концов.

Вернемся к формуле (2.36). Рассмотрим подробнее случай изотермической атмосферы в однородном поле сил тяжести. В этом случае \(U=m g z\), где \(m\)-масса молекулы, и распределение Больцмана принимает вид:


\begin{equation*}
n=n_{0} \mathrm{e}^{-m g z / k T} . \tag{2.38}
\end{equation*}


На рис. 2.12 показаны два графика этого распределения, 1 и 2. График 2 соответствует более высокой температуре (по сравнению с графиком 1). Произведение \(n(z) \mathrm{d} z\) равно числу молекул в слое толщиной \(\mathrm{d} z\) на высоте \(z\) в вертикальном столбе, площадь сечения которого равна единице ( \(S=1\) ). Площадь под кривыми 1 и 2 на рис. 2.12 равна полному числу мо-

\begin{figure}[h]
\begin{center}
  \includegraphics[alt={},max width=\textwidth]{bd023a69-cc46-4b6f-a48e-cc548fa9259d-061_410_489_1351_671}
\captionsetup{labelformat=empty}
\caption{Рис. 2.12}
\end{center}
\end{figure}

лекул в таком бесконечно высоком столбе. Отсюда следует, что площади под кривыми 1 и 2 одинаковы в данном случае.

Если газ представляет собой смесь разных газов, то в состоянии термодинамического равновесия концентрация \(n\) этих газов должна убывать с высотой экспоненциально с различной «скоростью» - в зависимости от масс молекул. Более крутая экспонента 1 на рис. 2.12 соответствует более тяжелым молекулам.

Для земной атмосферы резкого изменения состава газа с высотой не наблюдается. Известно также, что с высотой температура понижается, а это противоречит требованию одинаковости температуры в равновесном столбе газа (во избежание конвекционных тепловых потоков). Все это укаэывает на то, что земная атмосфера не находится в состоянии статистического равновесия.

\section*{Определение Перреном постоянной Авогадро}
С этой целью Перрен использовал закон распределения Больцмана (2.38). Определив постоянную \(k\), можно было вычислить и постоянную Авогадро \(N_{A}=R / k\). Но измерить массу молекулы не менее трудно, чем постоянную \(k\). Эту трудность Перрен преодолел, поняв, что роль молекул в (2.38) могут играть достаточно малые, но макроскопические частицы, размер и массу которых можно было измерить. Чтобы распределение таких частиц было не очень «крутым» по высоте, Перрен поместил их в жидкость, плотность которой немного меньше плотности вещества частиц. Тогда поле тяжести будет сильно ослаблено архимедовой подъемной силой, и распределение частиц по высоте в эмульсии окажется приемлемым для проведения измерений.

Одна из трудностей состояла в получении таких частиц, причем одинакового размера и формы. Для этого были использованы частицы гуммигута (особая смола), они имели сферическую форму. Для отбора частиц одинакового размера Перрен использовал многократное центрифугирование. В результате были получены одинаковые частицы гуммигута диаметром менее 0,5 мкм.

С эмульсией из таких частиц и была выполнена Перреном (1908-1909) серия тончайших экспериментов по измерению за-

висимости концентрации частиц гуммигута от высоты. Измерения проводились с помощью микроскопа (рис. 2.13). Непосредственно подсчитывалось число частиц в видимом объеме, ограниченном по диаметру полем зрения микроскопа, а по высоте - глубиной резкости объектива ( \(\sim 1\) мкм). Вся толщина эмульсии была \(\sim 100\) мкм.

Поле зрения пришлось уменьшить насто-

\begin{figure}[h]
\begin{center}
  \includegraphics[alt={},max width=\textwidth]{bd023a69-cc46-4b6f-a48e-cc548fa9259d-063_325_287_141_873}
\captionsetup{labelformat=empty}
\caption{Рис. 2.13}
\end{center}
\end{figure}

лько, чтобы одновременно было видно не более нескольких частиц. Большее число частиц просто невозможно было пересчитать, поскольку они непрерывно хаотично движутся. Для получения необходимой точности Перрену пришлось заглянуть в микроскоп через определенные интервалы времени несколько тысяч (!) раз. Всего таким образом было сосчитано 13000 частиц (на четырех уровнях).

Трудности этим не исчерпывались. А как, например, измерить диаметр частиц? Микроскоп здесь помочь не мог: размер частиц был меньше длины волны света, поэтому видеть частицы было можно, но измерить их диаметр нельзя. И все же Перрену удалось измерить их диаметр, и не одним а тремя (!) способами.

Удалось преодолеть и другие трудности (не будем их перечислять) Рассказанного вполне достаточно, чтобы по достоинству оценить уникальность и виртуозность экспериментов Перрена.

Полученные им значения постоянной Авогадро лежали в пределах \((6,5 \div 7,2) \cdot 10^{23}\) моль \(^{-1}\), что находится в хорошем согласии со значениями, полученными впоследствии другими, более точными методами. Работы Перрена доказали применимость распределения Больцмана не только к молекулам, но и к макрочастицам, а также подтвердили экспериментально и сам закон распределения Больцмана.

Соответствующий расчет постоянной \(k\) дан в задаче 2.8.

\section*{Барометрическая формула}
Умножив обе части распределения (2.38) на \(k T\), получим согласно (2.33), что давление


\begin{equation*}
p=p_{0} \mathrm{e}^{-M g z / R T}, \tag{2.39}
\end{equation*}


где \(M\) - молярная масса, \(R\) - универсальная газовая постоянная. Это так называемая барометрическая формула. Она строго спра-


где M— молярная масса, R — универсальная газовая постоянная.

Это так называемая барометрическая формула. Она строго справедлива для идеального газа, температура которого не зависит от высоты (изотермическая атмосфера). На рис. 2.14 показаны два графика (2.39) при разных температурах: Т2 > Т1.

Следует обратить внимание на то, что в отличие от распределений n(z) (см. рис. 2.12), кривые p(z) на рис.


\begin{figure}[h]
\begin{center}
  \includegraphics[alt={},max width=\textwidth]{bd023a69-cc46-4b6f-a48e-cc548fa9259d-064_420_538_254_70}
\captionsetup{labelformat=empty}
\caption{Рис. 2.14}
\end{center}
\end{figure}

Рассмотрим попутно несколько заслуживающих внимания вопросов.


\begin{enumerate}
  \item О поведении центра масс газа в поле тяжести. При интерпретации ситуации, изображенной на рис. 2.14 , мы приходим к выводу, что в результате повышения температуры от \(T_{1}\) до \(T_{2}\) центр масс газа перемещается вверх. Возникает вопрос: под действием какой внешней силы? На первый взгляд вроде ничего не изменилось - ни сила тяжести, ни сила реакции со стороны поверхности Земли (ведь давление \(p_{0}\) осталось прежним). На самом же деле в процессе нагревания газа равновесие нарушается, вторая сила оказывает-
\end{enumerate}

\begin{figure}[h]
\begin{center}
  \includegraphics[alt={},max width=\textwidth]{bd023a69-cc46-4b6f-a48e-cc548fa9259d-064_501_427_1264_68}
\captionsetup{labelformat=empty}
\caption{Рис. 2.15}
\end{center}
\end{figure}

ся большей по модулю. Она направлена вверх, она-то и вызывает перемещение центра масс.

Точнее, процесс происходит, например, так, как показано на рис. 2.15 , где \(T(t)\) - это график изменения температуры со временем \(t\), а под ним \(p_{0}(t)\) - соответствующий график изменения давления \(p_{0}\). Вре́менное увеличение давления (1) вызывает ускорение центра масс \(C\) газа. Затем, после равномерного подъема точки \(C\)

следует вре́менное уменьшение давления (2). Последнее приводит к остановке центра масс, и температура газа стабилизуется - при том же значении \(p_{0}\), что и в начале процесса.\\
2. О толщине земной атмосферы. Представим (2.39) в виде \(p=p_{0} \mathrm{e}^{-z / h}\), где \(h=R T / M g\) - это высота, на которой давление убывает в е раз. Значение \(h\) и играет роль характерной толщины атмосферы. При \(M=20\) г/моль и \(T=280\) К величина \(h \approx 8\) км. По сравнению с радиусом Земли атмосфера - тонкая пленочка (что и позволяет при получении барометрической формулы считать ускорение \(g\) не зависящим от высоты).\\
3. О массе атмосферы Земли. Сначала вычислим число молекул с помощью (2.38). Возьмем на поверхности Земли площадку с площадью \(S=1\) и рассмотрим столб воздуха над этой площадкой. В слое толщиной \(\mathrm{d} z\) на высоте \(z\) находится число молекул

\[
\mathrm{d} N=n_{0} \mathrm{e}^{-m g z / k T} \mathrm{~d} z .
\]

Интегрируя это выражение по \(z\) от 0 до \(\infty\), находим полное число молекул в столбе:

\[
N=n_{0} k T / m g .
\]

Затем умножим \(N\) на массу \(m\) одной молекулы и на площадь поверхности Земли \(4 \pi R^{2}\). В результате найдем, что масса \(M\) атмосферы

\[
M=N m \cdot 4 \pi R^{2}=\frac{n_{0} k T}{g} 4 \pi R^{2}=\frac{p_{0}}{g} 4 \pi R^{2}=5,3 \cdot 10^{18} \text { кг. }
\]

Этот вопрос можно решить и проще, рассуждая так. Поскольку атмосфера Земли в целом находится в равновесии, то можно считать, что сила тяжести, действующая на газ в каждом вертикальном столбе единичного сечения, уравновешивается силой реакции со стороны поверхности Земли, т.е. давлением \(p_{0}=10^{5} \Pi а\) ( 1 атм). Умножив \(p_{0}\) на поверхность Земли, получим \(M g\), где \(M\) - искомая масса. Отсюда \(M=5,3 \cdot 10^{18}\) кг.\\
4. Рассеяние атмосферы. Строго говоря, атмосфера Земли не является равновесной. Атмосфера непрерывно рассеивается, хотя для Земли этот процесс идет очень медленно. Рассеивание обусловлено тем, что в процессах соударения молекул в верх-

них слоях атмосферы неизбежно возникают молекулы, скорости которых оказываются больше второй космической. И таким молекулам иногда «удается» без столкновений покинуть атмосферу Земли. За все время существования Земля потеряла очень малую часть своей атмосферы.\\
5. «Парадокс»: почему в поле тяжести при движении молекул вверх их кинетическая энергия уменьшается, а температура остается прежней, т. е. средняя кинетическая энергия не меняется, а при движении вниз кинетическая энергия всех молекул увеличивается, а средняя их энергия остается той же? Этот «парадокс» был разъяснен уже самим Максвеллом. Суть в том, что при движении вверх молекулы действительно замедляются, но при этом наиболее медленные молекулы выбывают из потока частиц. При движении же вниз, наоборот, молекулы не только ускоряются, но одновременно их поток пополняется более медленными молекулами. В результате средняя скорость теплового движения молекул остается неизменной. Сила тяжести меняет лишь концентрацию молекул на разных высотах, но не температуру газа. И закон распределения Больцмана как раз и выводится из условия, чтобы температура газа оставалась всюду одной и той же.

\section*{Распределение Больцмана при дискретных уровнях}
Полученное Больцманом распределение (2.36) относится к случаям, когда молекулы находятся во внешнем поле и их потенциальная энергия \(U\) может изменяться непрерывно. Из (2.36) видно, что с ростом \(U\) концентрация частиц уменьшается.

Больцман обобщил закон (2.36) на случай распределения, зависящего от внутренней энергии \(E_{i}\) молекулы (атома). Известно, что величина \(E_{i}\) в этом случае может принимать лишь дискретный ряд дозволенных значений, и соответствующее распределение Больцмана


\begin{equation*}
N \sim \exp \left(-E_{i} / k T\right) \tag{2.40}
\end{equation*}


записывают так:


\begin{equation*}
\frac{N_{2}}{N_{1}}=\frac{g_{2}}{g_{1}} \exp \left(-\frac{E_{2}-E_{1}}{k T}\right), \tag{2.41}
\end{equation*}


где 1 и 2 - два произвольных (интересующих нас) уровня, \(N_{2} / N_{1}\) - отношение числа частиц на этих уровнях, которым отвечают внутренние энергии \(E_{2}\) и \(E_{1}, g\) - кратность вырождения каждого уровня. Например, кратность вырождения энергетического уровня атома водорода с главным квантовым числом \(n\) есть \(g=2 n^{2}\); кратность вырождения колебательного уровня двухатомной молекулы \(g=1\), а у вращательных уровней \(g=2 r+1\), где \(r\) - вращательное квантовое число.

Именно в таком виде распределение Больцмана для дискретного спектра используется наиболее часто.

Пример. Макросистема состоит из \(N\) частиц, которые могут находиться в двух состояниях, 1 и 2 , с внутренними энергиями \(E_{1}\) и \(E_{2}\), причем \(E_{2}>E_{1}\). Известно, что \(g_{1}=g_{2}\). Найдем зависимость от температуры \(T\) системы среднего числа частиц \(N_{2}\) в состоянии 2.

В данном случае

\[
N_{1}+N_{2}=N, \quad N_{2} / N_{1}=\mathrm{e}^{-\Delta E / k T},
\]

где \(\Delta E=E_{2}-E_{1}\). Исключив \(N_{1}\)\\
из этих двух уравнений, получим

\[
N_{2}=\frac{N}{1+\mathrm{e}^{\Delta E / k T}} .
\]

На рис. 2.16 приведен график

\begin{figure}[h]
\begin{center}
  \includegraphics[alt={},max width=\textwidth]{bd023a69-cc46-4b6f-a48e-cc548fa9259d-067_350_459_847_701}
\captionsetup{labelformat=empty}
\caption{Рис. 2.16}
\end{center}
\end{figure}

\section*{Закон распределения Максвелла-Больцмана}
Распределения Максвелла и Больцмана являются составными частями единого распределения, называемого распределением Гиббса (этот вопрос подробно рассматривается в спецкурсах по статистической физике, и мы ограничимся только упоминанием этого факта).

Оба рассмотренных нами распределения можно объединить в одно - закон распределения Максвелла-Больцмана. Согласно этому закону число \(\mathrm{d} N\) молекул в объеме \(\mathrm{d} V\) с радиус-векто-

ром \(\mathbf{r}\), имеющих проекции скорости в интервалах ( \(v_{x}, v_{x}+\mathrm{d} v_{x}\) ), где \(x=x, y, z\), определяются как

\[
\mathrm{d} N=n(\mathbf{r}) \mathrm{d} V \cdot \mathrm{~d} P\left(v_{x}, v_{y}, v_{z}\right)
\]

где \(n(\mathbf{r}) \mathrm{d} V\) - число молекул в объеме \(\mathrm{d} V\), а \(\mathrm{d} P\) - та доля этих молекул, проекции скоростей которых лежат в вышеуказанных интервалах. Остается учесть, что \(n(\mathbf{r})\) определяется формулой (2.36), а \(\mathrm{d} P\) - формулой (2.12): \(\mathrm{d} P=f(v) \mathrm{d} v_{x} \mathrm{~d} v_{y} \mathrm{~d} v_{z}\), где \(f(v)\) - это выражение (2.17).

В результате получим:


\begin{equation*}
\mathrm{d} N=A \mathrm{e}^{-(U+K) / k T} \mathrm{~d} V \mathrm{~d} v_{x} \mathrm{~d} v_{y} \mathrm{~d} v_{z} . \tag{2.42}
\end{equation*}


Здесь нормировочный множитель \(A=n_{0}(m / 2 \pi k T)^{3 / 2}\), потенциальная энергия \(U=U(\mathbf{r})\), кинетическая энергия \(K=m v^{2} / 2\), \(v^{2}=v_{x}^{2}+v_{y}^{2}+v_{z}^{2}\), элемент объема \(\mathrm{d} V=\mathrm{d} x \mathrm{~d} y \mathrm{~d} z\).

В качестве примера см. задачу 2.10.

\section*{Задачи}
2.1. Функция распределения вероятностей. Распределение вероятностей некоторой величины \(x\) описывается формулой \(f(x) \sim \sqrt{x}\) в интервале ( \(0, a\) ). Вне этого интервала \(f(x)=0\). Найти:\\
a) наиболее вероятное и среднее значения величины \(x\);\\
б) вероятность нахождения \(x\) в интервале ( \(0, a / 2\) ).

\begin{figure}[h]
\begin{center}
  \includegraphics[alt={},max width=\textwidth]{bd023a69-cc46-4b6f-a48e-cc548fa9259d-068_332_467_1190_138}
\captionsetup{labelformat=empty}
\caption{Рис. 2.17}
\end{center}
\end{figure}

Решение. а) Наиболее вероятное значение \(x\) соответствует максимуму функции \(f(x)\). Из рис. 2.17 сразу видно, что \(x_{\text {вер }}=a\). Это случай, не требующий выполнения условия \(\mathrm{d} f / \mathrm{d} x=0\).

Среднее значение \(x\) по определению есть


\begin{equation*}
\langle v\rangle=\int_{0}^{a} x f(x) \mathrm{d} x=\int_{0}^{a} x A \sqrt{x} \mathrm{~d} x, \tag{*}
\end{equation*}


где \(A\) - нормировочный множитель. Его находим из условия

\[
\int_{0}^{a} f(x) \mathrm{d} x=A \int_{0}^{a} \sqrt{x} \mathrm{~d} x=1
\]

откуда \(A=3 / 2 a^{3 / 2}\). Подстановка значения \(A\) в (*) и интегрирование дают в результате

\[
\langle x\rangle={ }^{3} /{ }_{5} a .
\]

б) По определению искомая вероятность

\[
P=\int_{0}^{a / 2} A \sqrt{x} \mathrm{~d} x=1 / \sqrt{8}=0,353 .
\]

2.2. Плотность распределения местонахождения частиц по плоскости зависит от расстояния \(r\) до точки 0 как

\[
f(r)=A(1-r / a), \quad \mathrm{M}^{-2},
\]

если \(r \leqslant a\). Здесь \(a\) задано, \(A\) - некоторая неизвестная постоянная. Найти:\\
а) наиболее вероятное расстояние \(r_{\text {вер }}\) частиц от точки 0 ;\\
б) среднее расстояние частиц от точки 0 .

Р е шен и е. а) Вероятность \(\mathrm{d} P\) нахождения частиц в интервале ( \(r, r+\mathrm{d} r\) ) равна произведению \(f(r)\) на площадь кольца радиуса \(r\) и шириной \(\mathrm{d} r\), т. е.


\begin{equation*}
\mathrm{d} P=f(r) \cdot 2 \pi r \mathrm{~d} r \tag{1}
\end{equation*}


Отсюда плотность вероятности в расчете на единицу ширины кольца равна \(\mathrm{d} P / \mathrm{d} r\). Обозначив эту величину как \(F(r)\), запишем


\begin{equation*}
F(r)=f(r) \cdot 2 \pi r=2 \pi A r(1-r / a) . \tag{2}
\end{equation*}


Функции \(f(r)\) и \(F(r)\) показаны на рис. 2.18. Наиболее вероятное расстояние \(r_{\text {вер }}\) находим из условия \(\mathrm{d} F / \mathrm{d} r=0\), откуда \(r_{\text {вер }}=a / 2\).\\
б) По определению


\begin{equation*}
\langle r\rangle=\int_{0}^{a} r F(r) \mathrm{d} r \tag{3}
\end{equation*}


где функция \(F(r)\) должна быть нормирована на единицу - для определения \(A\) :

\begin{figure}[h]
\begin{center}
  \includegraphics[alt={},max width=\textwidth]{bd023a69-cc46-4b6f-a48e-cc548fa9259d-069_390_502_1246_658}
\captionsetup{labelformat=empty}
\caption{Рис. 2.18}
\end{center}
\end{figure}

\[
\int F(r) d r=2 \pi A \int_{0}^{a} r(1-r / a) \mathrm{d} r=1
\]

Отсюда \(A=3 / \pi a^{2}\). Теперь остается взять интеграл (3). В результате получим \(\langle r\rangle=a / 2\). Т. е. в данном случае \(\langle r\rangle=r_{\text {вер }}\).\\
2.3. Распределение Максвелла. Найти с помощью распределения Максвелла среднее значение модуля проекции скорости \(\langle | \boldsymbol{v}_{\boldsymbol{x}}| \rangle\), если температура газа \(T\) и масса каждой молекулы \(m\).\\
Решени е. По определению искомая величина


\begin{equation*}
\langle | v_{x}| \rangle=\int_{-\infty}^{+\infty}\left|v_{x}\right| \varphi\left(v_{x}\right) \mathrm{d} v_{x} \tag{1}
\end{equation*}


Поскольку подынтегральная функция всюду положительна и симметрична относительно начала координат ( \(v_{x}=0\) ), то интеграл (1) можно записать так:


\begin{equation*}
\langle | v_{x}| \rangle=2 \int_{0}^{\infty} v_{x} \varphi\left(v_{x}\right) \mathrm{d} v_{x} \tag{2}
\end{equation*}


После подстановки функции \(\varphi\left(v_{x}\right)\) и некоторых преобразований получим


\begin{equation*}
\langle | v_{x}| \rangle=\left(\frac{2 k T}{\pi m}\right)^{1 / 2} \int_{0}^{\infty} \mathrm{e}^{-\eta} \mathrm{d} \eta \tag{3}
\end{equation*}


где введена новая переменная \(\eta=m v_{x}^{2} / 2 k T\). Последний интеграл равен единице, и мы имеем

\[
\langle | v_{x}| \rangle=(2 k T / \pi m)^{1 / 2} .
\]

2.4. Газ состоит из молекул, масса каждой из которых равна \(m\). При какой температуре \(T\) этого газа число молекул со скоростями в заданном малом интервале ( \(v_{0}, v_{0}+\delta v\) ) будет максимально? Какова наиболее вероятная скорость молекул, соответствующая такой температуре?

Р е ш е н и е. Первый вопрос означает, что надо найти \(T\), при которой функция (2.20) будет максимальной для заданной скорости \(v\). Зависимость этой функции от \(T\) имеет вид


\begin{equation*}
F(T) \sim T^{-3 / 2} \exp \left(-m v_{0}^{2} / 2 k T\right) . \tag{1}
\end{equation*}


Максимум функции \(F(T)\) находим из условия \(\partial F / \partial T=0\), откуда следует


\begin{equation*}
T_{0}=m v_{0}^{2} / 3 k . \tag{2}
\end{equation*}


При этой температуре


\begin{equation*}
v_{\text {вер }}=\sqrt{2 / 3} v_{0} . \tag{3}
\end{equation*}


Заметим, что скорость \(v_{0}\) при температуре \(T\) совпадает со среднеквадратичной. Рассмотренную здесь ситуацию поясняет рис. 2.19 , из которого видно, что при \(T_{0}\) функция \(F\left(v_{0}\right)=\) макс. Нумерация \(1,2,3,4\) соответствует распределениям при температурах \(T_{1}<T_{2}\left(T_{0}\right)<T_{3}<T_{4}\).

\begin{figure}[h]
\begin{center}
  \includegraphics[alt={},max width=\textwidth]{bd023a69-cc46-4b6f-a48e-cc548fa9259d-071_364_510_493_393}
\captionsetup{labelformat=empty}
\caption{Рис. 2.19}
\end{center}
\end{figure}

2.5. Газ из молекул, масса каждой из которых \(m\), находится при температуре \(T\). Найти функцию распределения молекул по дебройлевским длинам волн \(f(\lambda)\) и наиболее вероятное значение \(\lambda\) при данной температуре. Изобразить примерный график функции \(f(\lambda)\).

Решение. Преобразуем выражение \(\mathrm{d} P=F(v) \mathrm{d} v\) следующим образом:

\[
\mathrm{d} P=F(v) \frac{\mathrm{d} v}{\mathrm{~d} \lambda} \mathrm{~d} \lambda=f(\lambda) \mathrm{d} \lambda .
\]

Отсюда видно, что


\begin{equation*}
f(\lambda)=F(v) \frac{\mathrm{d} v}{\mathrm{~d} \lambda} . \tag{1}
\end{equation*}


Далее учтем связь между скоростью \(v\) и длиной волны \(\lambda\), а именно \(\lambda=h / m v\). После соответствующих преобразований выражения (1) получим:


\begin{equation*}
f(\lambda)=C \lambda^{-4} \exp \left(-a / \lambda^{2}\right), \tag{2}
\end{equation*}


где \(C=4 \pi\left(h^{2} / 2 \pi m k T\right)^{3 / 2}\) и \(a=h^{2} / 2 m k T\). Знак минус в выражении (2) мы опустили как несущественный, он означает только, что с ростом скорости \(v\) длина волны \(\lambda\) уменьшается.

Наиболее вероятное значение \(\lambda_{\text {вер }}\) находим из условия \(\mathrm{d} f / \mathrm{d} \lambda=0\), откуда следует, что

\[
\lambda_{\text {Bep }}=\frac{h}{2 \sqrt{m k T}} .
\]

\begin{figure}[h]
\begin{center}
  \includegraphics[alt={},max width=\textwidth]{bd023a69-cc46-4b6f-a48e-cc548fa9259d-072_373_443_327_141}
\captionsetup{labelformat=empty}
\caption{Рис. 2.20}
\end{center}
\end{figure}

Имея в виду, что \(f(\lambda) \rightarrow 0\) как при \(\lambda \rightarrow \infty\), так и при \(\lambda \rightarrow 0\) (в последнем случае это решается с помощью правила Лопиталя), а также то, что при \(\lambda \rightarrow 0\) производная \(\mathrm{d} f / \mathrm{d} \lambda \rightarrow \infty\), мы приходим к выводу, что график функции \(f(\lambda)\) должен иметь вид, показанный на рис. 2.20.\\
2.6. Вычислить с помощью распределения Максвелла число \(v\) молекул газа, падающих ежесекундно на единичную площадь стенки сосуда. Температура газа \(T\), концентрация молекул \(n\) и масса каждой молекулы \(m\) предполагаются известными.

Решение. Направим ось \(X\) перпендикулярно к единичной площадке стенки сосуда (в произвольном месте). Тогда произведение \(\mathrm{d} n\left(v_{x}\right) \cdot v_{x}\) - это число молекул в цилиндре длиной \(v_{x}\) единичной площади сечения, имеющих проекции скорости в интервале ( \(v_{x}\), \(v_{x}+\mathrm{d} v_{x}\) ). Оно равно числу \(\mathrm{d} v\) ударов о стенку ежесекундно этой группы молекул:


\begin{equation*}
\mathrm{d} v=\mathrm{d} n\left(v_{x}\right) \cdot v_{x} . \tag{1}
\end{equation*}


Согласно (2.13) \(\mathrm{d} n\left(v_{x}\right)=n \varphi\left(v_{x}\right) \mathrm{d} v_{x}\), где \(\varphi\left(v_{x}\right)\) - функция распределения Максвелла по \(v_{x}\) (2.16).

Интегрирование (1) по всем \(v_{x}\) дает искомый результат:


\begin{equation*}
v=\int v_{x} \mathrm{~d} n\left(v_{x}\right)=C n \int \exp \left(-m v_{x}^{2} / 2 k T\right) v_{x} \mathrm{~d} v_{x}, \tag{2}
\end{equation*}


где \(C=(m / 2 \pi k T)^{1 / 2}\). Введем новую переменную \(\eta=m v_{x}^{2} / 2 k T\). Тогда (2) примет вид


\begin{equation*}
v=C n \frac{k T}{m} \int_{0}^{\infty} \mathrm{e}^{-\eta} \mathrm{d} \eta=n \sqrt{\frac{k T}{2 \pi m}} \tag{3}
\end{equation*}


Последнее выражение можно представить как


\begin{equation*}
v={ }^{1} /{ }_{4} n\langle v\rangle, \tag{4}
\end{equation*}


где \(\langle v\rangle=\sqrt{8 k T / \pi m}\). Полученный результат (4) соответствует формуле (1.27). Таким образом, мы получили точное значение коэффициента в формуле (1.27): не \(1 / 6\), а \(1 / 4\).\\
2.7. Определить с помощью распределения Максвелла давление, оказываемое газом на стенку, если температура газа \(T\) и концентрация молекул \(n\).\\
Решение. Направим ось \(X\) перпендикулярно к стенке сосуда (в произвольном месте). Каждая молекула передает при падении на стенку импульс \(m v_{x}\). В единице объема содержится \(\mathrm{d} n=n \varphi\left(v_{x}\right) \mathrm{d} v_{x}\) молекул с проекциями скоростей в интервале ( \(v_{x}, v_{x}+\mathrm{d} v_{x}\) ). Передаваемый этими молекулами импульс в единицу времени равен


\begin{equation*}
m v_{x} \cdot\left(\mathrm{~d} n \cdot v_{x}\right), \tag{1}
\end{equation*}


где величина в круглых скобках - это число соответствующих молекул в объеме цилиндра длиной \(v_{x}\) единичного сечения, или число ударов, испытываемых стенкой ежесекундно. Тогда полное давление (суммарный импульс в секунду) определяется как


\begin{equation*}
p=2 \int m v_{x} \cdot v_{x} \mathrm{~d} n\left(v_{x}\right)=2 \int_{0}^{\infty} m v_{x}^{2} n \varphi\left(v_{x}\right) \mathrm{d} v_{x} \tag{2}
\end{equation*}


где двойка перед интегралом учитывает тот факт, что импульс передается стенке при падении молекул и такой же - при отражении. Подстановка в последний интеграл выражения \(\varphi\left(v_{x}\right)\) согласно (2.16) и введение новой переменной \(\chi=v_{x} \sqrt{m / 2 k T}\) позволяет формулу (2) представить в виде


\begin{equation*}
p=\frac{4 n k T}{\sqrt{\pi}} \int_{0}^{\infty} \chi^{2} \exp \left(-\chi^{2}\right) \mathrm{d} \chi \tag{3}
\end{equation*}


Этот интеграл табличный, он равен \(\sqrt{\pi} / 4\), поэтому в результате

\[
p=n k T
\]

что и следовало ожидать.\\
2.8. Распределение Больцмана. При наблюдении в микроскоп взвешенных частиц гуммигута обнаружено, что среднее число их в тонких слоях 1 и 2 , расстояние между которыми по высоте равно \(h\), отличается друг от друга в \(\eta\) раз \(\left(\eta=N_{1} / N_{2}\right)\). Температура сре-

ды \(T\), диаметр частиц \(d\) и их плотность на \(\Delta \rho\) больше плотности окружающей жидкости. Найти по этим данным постоянную Больцмана.

Решение. Согласно распределению Больцмана отношение частиц в слоях 1 и 2 определяется как


\begin{equation*}
N_{2} / N_{1}=\exp \left[-\left(m-m_{0}\right) g h / k T\right] \tag{1}
\end{equation*}


где учтено, что частицы находятся в поле, образованном совместным действием силы тяжести ( \(m g\) ) и силы Архимеда ( \(-m_{0} g\) ). Преобразуем разность масс в круглых скобках выражения (1):


\begin{equation*}
m-m_{0}=\Delta \rho \cdot V=(\pi / 6) d^{3} \Delta \rho \tag{2}
\end{equation*}


После подстановки (2) в (1) соотношение (1) представим в виде


\begin{equation*}
\ln \frac{N_{1}}{N_{2}}=\frac{\pi d^{3} g h \Delta \rho}{6 k T} \tag{3}
\end{equation*}


откуда, имея в виду, что \(N_{1} / N_{2}=\eta\), следует

\[
k=\frac{\pi d^{3} g h \Delta \rho}{6 T \ln \eta}
\]

2.9. Потенциальная энергия молекул газа в некотором центральном потенциальном поле зависит от расстояния \(r\) до центра поля как \(U=a r^{2}\), где \(a\) - положительная постоянная. Температура газа \(T\), концентрация молекул в центре поля \(n_{0}\). Найти:\\
a) число молекул \(\mathrm{d} N\) с потенциальной энергией в интервале \((U, U+\mathrm{d} U)\);\\
б) наиболее вероятное значение потенциальной энергии молекул.\\
\(P\) ешение. а) Молекулы с такими значениями \(U\) будут находиться в сферическом слое объемом \(\mathrm{d} V=4 \pi r^{2} \mathrm{~d} r\). Тогда


\begin{equation*}
\mathrm{d} N=n_{0} \mathrm{e}^{-U / k T} \cdot 4 \pi r^{2} \mathrm{~d} r \tag{1}
\end{equation*}


Поскольку \(U=a r^{2}\), то \(\mathrm{d} U=2 a r \mathrm{~d} r\) и формулу (1) можно представить в виде


\begin{equation*}
\mathrm{d} N=\left(2 \pi n_{0} / a^{3 / 2}\right) \mathrm{e}^{-U / k T} \sqrt{U} \mathrm{~d} U \tag{2}
\end{equation*}


б) Решение сводится к нахождению \(U\), при котором функция \(\mathrm{d} N / \mathrm{d} U\) в уравнении (2) имеет максимум. Обозначив \(\mathrm{d} N / \mathrm{d} U\) как \(f(U)\), запишем условие, при котором \(f=\) макс (из вида этой функции будет видно, что мы имеем дело именно с максимумом). Итак, вычислив производную \(\mathrm{d} f / \mathrm{d} U\) и приравняв полученный peзультат к нулю, получим

\[
U_{\text {вер }}=k T / 2 .
\]

2.10. Распределение Максвелла-Больцмана. Газ из молекул массы \(m\) находится в центральном поле, где потенциальная энергия молекул равна \(U(r), r\) - расстояние от центра поля. Температура газа \(T\), концентрация молекул в центре поля \(n_{0}\). Найти число молекул \(\delta N\) в тонком сферическом слое ( \(r, r+\delta r\) ) со скоростями, отличающимися от наиболее вероятной не более, чем на \(\delta \eta\)-часть ( \(\delta \eta \ll 1\) ).\\
Решение. Число молекул в данном слое


\begin{equation*}
\delta N_{0}=n_{0} \mathrm{e}^{-U / k T} 4 \pi r^{2} \delta r . \tag{1}
\end{equation*}


Из них доля молекул, скорости которых отличаются от \(v_{\text {вер }}\) не более, чем на \(\delta υ\), определяется формулой


\begin{equation*}
\delta P=F(v) \delta v=4 \pi\left(\frac{m}{2 \pi k T}\right)^{3 / 2} v^{2} \exp \left(-\frac{m v^{2}}{2 k T}\right) \delta v, \tag{2}
\end{equation*}


где вместо \(v\) надо подставить \(v_{\text {вер }}=\sqrt{2 k T / m}\) и вместо \(\delta v\) - величину \(v_{\text {вер }} \frac{\delta v}{v_{\text {вер }}}=v_{\text {вер }} \cdot 2 \delta \eta\). Тогда (2) примет вид


\begin{equation*}
\delta P=4 \pi\left(\frac{m}{2 \pi k T}\right)^{3 / 2}\left(\frac{2 k T}{m}\right)^{3 / 2} \mathrm{e}^{-1} 2 \delta \eta=\frac{8 \mathrm{e}^{-1}}{\sqrt{\pi}} \delta \eta . \tag{3}
\end{equation*}


Искомое значение \(\delta N\) равно произведению (1) на (3):

\[
\delta N=\delta N_{0} \cdot \delta P=32 \sqrt{\pi} n_{0} \mathrm{e}^{-E / k T} r^{2} \delta r \delta \eta,
\]

где \(E=U+k T\).\\
2.11. Дискретное распределение Больцмана. Найти отношение количеств молекул водорода на первых возбужденных колебательном и вращательном уровнях при температуре газа \(T=880\) К. Собственная частота колебаний молекулы \(\omega=0,83 \cdot 10^{15} \mathrm{c}^{-1}\), ее момент инерции \(I=4,6 \cdot 10^{-39} г \cdot \mathrm{~cm}^{2}\).

Решение. Воспользуемся формулами (1.41) для дозволенных значений вращательной и колебательной энергии. Для большей ясности изобразим интересующие нас уровни (рис.2.21). Согласно распределению Больцмана (2.41),

\begin{figure}[h]
\begin{center}
  \includegraphics[alt={},max width=\textwidth]{bd023a69-cc46-4b6f-a48e-cc548fa9259d-076_320_406_304_159}
\captionsetup{labelformat=empty}
\caption{Рис. 2.21}
\end{center}
\end{figure}

\[
\begin{aligned}
& \frac{N_{\text {кол }}}{N_{\text {вр }}}=\frac{1}{2 r+1}\left(-\frac{\hbar \omega-\hbar^{2} I}{k T}\right)= \\
& \quad=\frac{1}{3} \exp \left[-\frac{\hbar\left(\omega-\hbar^{2} I\right)}{k T}\right]=3,1 \cdot 10^{-4}
\end{aligned}
\]

\section*{Второе начало термодинамики. Энтропия}
\section*{§ 3.1. Второе начало термодинамики}
Первое начало термодинамики представляет собой по сути обобщение закона сохранения энергии на тепловые явления. Оно устанавливает количественные соотношения между превращениями одних видов энергии в другие.

В отличие от него второе начало определяет условия, при которых возможны эти превращения, а также возможные направления протекания процессов. Оказывается, не все процессы, разрешенные первым началом, возможны.

Существует несколько формулировок второго начала.


Первое начало термодинамики представляет собой по сути обобщение закона сохранения энергии на тепловые явления.

Оно устанавливает количественные соотношения между превращениями одних видов энергии в другие.

В отличие от него второе начало определяет условия, при которых возможны эти превращения, а также возможные направления протекания процессов. Оказывается, не


\begin{enumerate}
  \item Клаузиус (1850): невозможен самопроизвольный переход тепла от менее к более нагретому телу, или невозможны процессы, единственным конечным результатом которых был бы переход тепла от менее к более нагретому телу.
\end{enumerate}

Тот факт, что, например, в холодильнике совершается переход тепла от холодильной камеры в комнату, не противоречит этому утверждению, поскольку этот процесс не является самопроизвольным: для его осуществления потребляется электрическая энергия.\\
2. Кельвин (1851): невозможны процессь, единственным конечным результатом которых было бы превращение тепла целиком в работу.

Казалось бы, что этому противоречит, например, процесс изотермического расширения идеального газа, где все полученное газом тепло превращается в работу. Однако это не единственный конечный результат процесса: при этом происходит изменение объема газа.

Заметим, что слово единственный в обеих формулировках является весьма существенным, без него они теряют смысл.

Приведенные формулировки второго начала эквивалентны, из одной неизбежно следует другая. В самом деле, если бы

можно было осуществить процесс, запрещенный по Кельвину, то тепло, отнятое от какого-либо тела, можно было полностью превратить в работу, а затем, превратив эту работу целиком в тепло (трением), передать это тепло другому телу с более высокой температурой. В результате мы имели бы процесс, невозможный по Клаузиусу.

Если бы не второе начало, можно было легко решить энергетическую проблему - построить двигатель, который отнимал бы тепло из океанов и целиком превращал его в работу. Подобный двигатель по своим практическим последствиям представлял бы перпетуум-мобиле 2 -го рода (в отличие от «вечного двигателя» - перпетуум-мобиле 1-го рода). При современном потреблении энергии человечеством температура океанов за 1000 лет уменьшилась бы не более, чем на один кельвин.

Это позволяет перефразировать формулировку Кельвина так: перпетуум-мобиле 2-го рода невозможен, или невозможно создать тепловой двигатель с КПД \(\eta=1\). Напомним, КПД теплового двигателя \(\eta=A / Q\), где \(Q\) - сообщенное двигателю тепло, \(A\) - произведенная им работа.

Любой тепловой двигатель работает по замкнутому циклу,

\begin{figure}[h]
\begin{center}
  \includegraphics[alt={},max width=\textwidth]{bd023a69-cc46-4b6f-a48e-cc548fa9259d-078_346_386_1016_68}
\captionsetup{labelformat=empty}
\caption{Рис. 3.1}
\end{center}
\end{figure}

Пусть \(Q_{1}\) - поглощенное тепло, а \(Q_{2}^{\prime}\) - отдаваемое тепло ( \(Q_{2}^{\prime}>0\) ). Опыт показывает, что тепло \(Q_{2}\) неизбежно существует в любом тепловом двигателе (как тепловой «шлак»). По первому началу за цикл приращение внутренней энергии рабочего вещества \(\Delta U=0\), поэтому \(A=Q_{1}-Q_{2}^{\prime}\). Эффективность теплового двигателя определяют его КПД:


\begin{equation*}
\eta=\frac{A}{Q_{1}}=\frac{Q_{1}-Q_{2}^{\prime}}{Q_{1}}=1-\frac{Q_{2}^{\prime}}{Q_{1}} . \tag{3.1}
\end{equation*}


Опыт показывает, что всегда \(\eta<1\). Значение \(\eta=1\) запрещено вторым началом.

Проблема необратимости процессов. По существу все процессы в макросистемах являются необратимыми (строго говоря, таковыми являются и процессы, которые мы называли обратимыми - это идеализация, удобная для решения многих важных вопросов).

Возникает принципиальный вопрос: в чем причина необратимости? Это выглядит особенно странно, если учесть, что все законы механики обратимы во времени. И тем не менее, никто не видел, чтобы, например, разбившаяся ваза самопроизвольно восстановилась из осколков. Этот процесс можно наблюдать, если предварительно засняв на пленку, просмотреть ее в обратном направлении, но никак не в действительности.

Загадочными становятся и запреты, устанавливаемые вторым началом термодинамики.

Решение этой сложной проблемы пришло с открытием новой термодинамической величины - энтропии - и раскрытием ее физического смысла.

\section*{§ 3.2. Энтропия}
Открытие Клаузиусом энтропии - одно из самых удивительных, сделанных «на кончике пера», т. е. чисто теоретически. Несмотря на это обстоятельство и отсутствие приборов, которые бы измеряли энтропию вещества, это понятие оказалось необычайно плодотворным. Энтропия \(S\) вводится через ее элементарное приращение как


\begin{equation*}
\mathrm{d} S=\frac{\mathrm{d}^{\prime} Q}{T} . \tag{3.2}
\end{equation*}


Следует обратить внимание на особенность этой формулы. Как мы знаем, d'Q не есть приращение какой-то функции, но после деления на температуру \(T\), оказывается, получается приращение некоторой функции (энтропии). В отличие от теплоты, энтропия такая же функция состояния как температура, внутренняя энергия или давление. Полученное системой тепло \(Q\) зависит от процесса перехода из начального состояния в конечное, приращение же энтропии \(\Delta S\) совершенно не зависит от процесса, а только от начального и конечного состояний.

В интегральной форме соотношение (3.2) имеет вид


\begin{equation*}
S_{2}-S_{1}=\int_{1}^{2} \frac{\mathrm{~d}^{\prime} Q}{T}, \tag{3.3}
\end{equation*}


при этом не играет роли, какой именно процесс перевел систему из состояния 1 в состояние 2 . Процесс может быть даже необратимым. Важно лишь, чтобы состояния 1 и 2 были равновесными, расчет же с помощью (3.3) может проводиться по любому обратимому процессу между состояниями 1 и 2.

Знак равенства в формуле (3.2), а значит и в (3.3), относится только к равновесным (квазистатическим) процессам. В случае же неравновесных процессов знак «=» заменяется на «>» Это объясняется тем, что при неравновесных процессах энтропия получает, кроме приращения \(\mathrm{d}^{\prime} Q / T\), еще и положительное приращение \(\mathrm{d} S_{\mathrm{H}}\), обусловленное самой необратимостью процесса*.

Введение таким образом энтропии \(S\) означает, что мы можем вычислять только разность энтропий, но нельзя сказать, чему равна энтропия в каждом из состояний, т. е. энтропия этими формулами может быть определена с точностью до прибавления произвольной аддитивной постоянной.


В интегральной форме соотношение (3.2) имеет вид S S Q T

   d , (3.3) при этом не играет роли, какой именно процесс перевел систему из состояния 1 в состояние 2. Процесс может быть даже необратимым. Важно лишь, чтобы состояния 1 и 2 были равновесными, расчет же с помощью (3.3) может проводиться по любому обратимому процессу между состояниями 1 и 2.

Знак равенства в формуле (3.2), а зн


\section*{Свойства энтропии}
\begin{enumerate}
  \item Итак, энтропия - функиия состояния. Если процесс проводят вдоль адиабаты, то энтропия системы не меняется. Значит адиабаты - это одновременно и изоэнтропь. Каждой более
\end{enumerate}

\begin{figure}[h]
\begin{center}
  \includegraphics[alt={},max width=\textwidth]{bd023a69-cc46-4b6f-a48e-cc548fa9259d-080_362_388_1250_68}
\captionsetup{labelformat=empty}
\caption{Рис. 3.2}
\end{center}
\end{figure}

\footnotetext{\begin{itemize}
  \item При этом под \(T\) подразумевают температуру теплового резервуара, от которого интересующая нас система получает тепло \(\mathrm{d}^{\prime} Q\).
\end{itemize}
}выглядит система адиабат, можно легко ответить на вопрос о приращении энтропии при проведении любого процесса между интересующими нас равновесными состояниями 1 и 2.\\
2. Энтропия - величина аддитивная: энтропия макросистемы равна сумме энтропий ее отдельных частей.\\
3. Одно из важнейших свойств энтропии заключается в том, что энтропия замкнутой (т. е. теплоизолированной) макросистемы не уменьшается - она либо возрастает, либо остается постоянной*.

Приниип возрастания энтропии замкнутых систем представляет собой еще одну формулировку второго начала термодинамики.

Величина возрастания энтропии в замкнутой макросистеме может служить мерой необратимости процессов, протекающих в системе. В предельном случае, когда процессы имеют обратимый характер, энтропия замкнутой макросистемы не меняется.

Пример. Идеальный газ, находящийся в некотором состоянии, адиабатически (т. е. без теплообмена) расширили до объема \(V\). Одинаково ли будет установившееся давление газа в конечном состоянии (в объеме \(V\) ), если процесс расширения а) обратимый, б) необратимый?

В соответствии с принципом возрастания энтропии в замкнутой системе при необратимом процессе энтропия должна увеличиться. Значит установившееся состояние будет соответствовать точке на более высокой адиабате, т. е. давление будет больше.

Заметим, что с самого начала введение понятия энтропии \(S\) было поставлено в прямую связь с необратимостью. Оказывается, все самопроизвольно протекающие процессы в природе от теплообмена до химических реакций - протекают так, что энтропия возрастает. Необходимо специальное взаимодействие с окружающей средой, чтобы препятствовать возрастанию энтропии в макросистеме. Наиболее ярким примером могут служить все живые существа.


\footnotetext{\begin{itemize}
  \item Заметим, что так ведет себя энтропия только замкнутых макросистем. Если же система не замкнута, то ее энтропия может как увеличиваться, так и уменьшаться.
\end{itemize}
}
4. Теорема Нернста (1906). Эта теорема утверждает, что npu приближении температуры к абсолютному нулю энтропия макросистемь также стремится к нулю:

\begin{equation*}
S \rightarrow 0 \text { при } T \rightarrow 0, \tag{3.4}
\end{equation*}


и мы можем вычислять абсолютное значение энтропии по формуле


\begin{equation*}
S(p, T)=\int_{0}^{T} \frac{C_{p}(T) \mathrm{d} T}{T} \tag{3.5}
\end{equation*}


Отсюда следует, что при \(T \rightarrow 0\) теплоемкость \(C_{p}\) всех макросистем должна тоже стремиться к нулю (иначе интеграл не будет сходиться).

Обратим внимание на то, что энтропия по своей сущности всегда определена с точностью до произвольной аддитивной постоянной. Поэтому слева в формуле (3.4) следовало бы писать \(S-S_{0}\). Tо, что мы считаем \(S_{0}=0\) - это не более как произвольное соглашение.

Теорема Нернста не может быть логически выведена из первых двух начал, поэтому ее часто называют третьим началом термодинамики.


то называют третьим началом термодинамики.

\begin{enumerate}

  \item Основное уравнение термодинамики. Оно представляет собой объединение энтропии с первым началом. Подставив в (1.6) выражение dQ  TdS из (3.2), получим для обратимых процессов: TdS = dU + pdV. (3.6) Это уравнение имеет многочисленные применения.

  \item Энтропия идеального газа. Пусть начальное и конечное состояния, 1 и 2, газа определяются параметрами р1, V1 и p2, V2. Согласно (3.6) элементарное приращение энтропии газа с учетом того, что dU  CV dT и pV  RT, определяется как d d d S C T T R V V V   . (3.7)

\end{enumerate}


\section*{§ 3.3. О вычислении и применении энтропии}
\begin{enumerate}
  \item Основное уравнение термодинамики. Оно представляет собой объединение энтропии с первым началом. Подставив в (1.6) выражение \(\mathrm{d}^{\prime} Q=T \mathrm{~d} S\) из (3.2), получим для обратимых процессов:
\end{enumerate}


\begin{equation*}
T \mathrm{~d} S=\mathrm{d} U+p \mathrm{~d} V . \tag{3.6}
\end{equation*}


Это уравнение имеет многочисленные применения.\\
2. Энтропия идеального газа. Пусть начальное и конечное состояния, 1 и 2 , газа определяются параметрами \(p_{1}, V_{1}\) и \(p_{2}\), \(V_{2}\). Согласно (3.6) элементарное приращение энтропии газа с учетом того, что \(\mathrm{d} U=C_{V} \mathrm{~d} T\) и \(p V=v R T\), определяется как


\begin{equation*}
\mathrm{d} S=C_{V} \frac{\mathrm{~d} T}{T}+v R \frac{\mathrm{~d} V}{V} . \tag{3.7}
\end{equation*}


Взяв дифференциал логарифма от \(v R T=p V\), получим


\begin{equation*}
\frac{\mathrm{d} T}{T}=\frac{\mathrm{d} p}{p}+\frac{\mathrm{d} V}{V}, \tag{3.8}
\end{equation*}


и формуле (3.7) можно придать симметричный вид:


\begin{equation*}
\mathrm{d} S=C_{V} \frac{\mathrm{~d} p}{p}+C_{p} \frac{\mathrm{~d} V}{V}, \tag{3.9}
\end{equation*}


где учтено, что \(C_{p}=C_{V}+v R\). Проинтегрировав последнее выражение, получим в результате


Взяв дифференциал логарифма от RT  pV, получим d d d T T p p V V  , (3.8) и формуле (3.7) можно придать симметричный вид: d d d S C p p C V V V p  , (3.9) где учтено, что Cp  CV + R. Проинтегрировав последнее выражение, получим в результате S S C p p C V V V p

  ln ln . (3.10)

\begin{enumerate}

  \item Приращение энтропии при необратимом процессе между двумя равновесными состояниями 1 и 2. Непо

\end{enumerate}



\begin{equation*}
S_{2}-S_{1}=C_{V} \ln \frac{p_{2}}{p_{1}}+C_{p} \ln \frac{V_{2}}{V_{1}} . \tag{3.10}
\end{equation*}


\begin{enumerate}
  \setcounter{enumi}{2}
  \item Приращение энтропии при необратимом процессе между двумя равновесными состояниями 1 и 2 . Непосредственно считать энтропию по необратимому процессу совершенно невозможно. Но энтропия - функция состояния. Этим мы и воспользуемся, проведя между состояниями 1 и 2 какой-нибудь обратимый процесс, ничего общего не имеющий с реальным необратимым процессом. Обычно выбирают такой обратимый процесс, по которому расчет проще.
\end{enumerate}

Пример. Пусть в одном из двух теплоизолированных сосудов, соединенных трубкой с закрытым вентилем, находится один моль идеального газа, а в другом сосуде - вакуум (рис. 3.3). Объемы сосудов \(V_{1}\) и \(V_{2}\). Вентиль открыли, газ заполнил оба сосуда и пришел в состояние термодинамического равновесия. Найдем приращение энтропии в этом процессе.

\begin{figure}[h]
\begin{center}
  \includegraphics[alt={},max width=\textwidth]{bd023a69-cc46-4b6f-a48e-cc548fa9259d-083_184_382_1210_776}
\captionsetup{labelformat=empty}
\caption{Рис. 3.3}
\end{center}
\end{figure}

Ясно, что процесс расширения газа необратимый. При этом процесс прошел без теплообмена ( \(Q=0\) ) и без совершения работы ( \(A=0\) ). Значит, по первому началу термодинамики \(\Delta U=0\), т. е. конечная температура равна начальной. Это позволяет провести расчет приращения энтропии по обратимому изотермическому процессу:


Взяв дифференциал логарифма от RT  pV, получим d d d T T p p V V  , (3.8) и формуле (3.7) можно придать симметричный вид: d d d S C p p C V V V p  , (3.9) где учтено, что Cp  CV + R. Проинтегрировав последнее выражение, получим в результате S S C p p C V V V p

  ln ln . (3.10)

\begin{enumerate}

  \item Приращение энтропии при необратимом процессе между двумя равновесными состояниями 1 и 2. Непо

\end{enumerate}


\[
\Delta S=\int \frac{\mathrm{d}^{\prime} Q}{T}=\int \frac{p \mathrm{~d} V}{T}=\int R \frac{\mathrm{~d} V}{V}=R \ln \frac{V_{1}+V_{2}}{V_{1}}>0 .
\]

\begin{enumerate}
  \setcounter{enumi}{3}
  \item Возрастание энтропии при смешении газов. Пусть в двух половинах теплоизолированного сосуда объемом \(V\) находятся два идеальных газа, 1 и 2 , разделенных перегородкой. Температура, давление и число молей \(v\) в обеих половинах одинаково. После удаления перегородки начинается необратимый процесс смешения газов. В конце концов он прекращается, и система приходит в равновесное состояние, в котором оба газа равномерно перемешаны. Температура в конечном состоянии будет такая же, так как система теплоизолирована и газы идеальные.
\end{enumerate}

Используя результат предыдущего примера, находим, что при \(V_{1}=V_{2}\) приращение энтропии каждого газа \(\Delta S_{1,2}=v R \ln 2\), т. е. суммарное приращение энтропии системы

\[
\Delta S=2 v R \ln 2 .
\]

Приращение \(\Delta S>0\), что естественно, поскольку процесс смешения существенно необратимый (обратный процесс - саморазделение смеси двух газов - совершенно невероятен).

Последняя формула приводит к выводу, называемому парадоксом Гиббса. Допустим, что газы 1 и 2 тождественны. Тогда после удаления перегородки энтропия увеличивается, хотя ясно, что конечное состояние системы ничем не отличается от начального. В этом суть парадокса.

Для понимания описанной ситуации существенно заметить, что последняя формула получена только для случая, когда газы 1 и 2 различны. Для тождественных газов приведенные рассуждения не применимы. Для них \(\Delta S=0\).

Таким образом, формула \(\Delta S=2 v R \ln 2\) справедлива только при смешивании различных газов, хотя бы это различие и было сколь угодно малым. Возникающая здесь трудность с предельным переходом в действительности не существует, поскольку число различных типов атомов конечно, и такой предельный переход просто невозможен.\\
5. Цикл Карно. Рассмотренный Карно тепловой двигатель состоял из нагревателя с температурой \(T_{1}\), холодильника с \(T_{2}\) и рабочего тела, т. е. устройства, способного получать тепло и совершать работу (рис. 3.4). Под рабочим телом пока будем понимать идеальный газ в цилиндре с поршнем.

Карно рассмотрел цикл из двух изотерм и двух адиабат (рис. 3.5). При изотермическом расширении \(1-2\) газ находится в контакте с нагревателем ( \(T_{1}\) ). Пусть при этом газ получает тепло \(Q_{1}\). На изотерме \(3-4\) газ отдает тепло \(Q_{2}^{\prime}\) холодильнику ( \(T_{2}\) ). В соответствии с (3.1) КПД двигателя

\begin{figure}[h]
\begin{center}
  \includegraphics[alt={},max width=\textwidth]{bd023a69-cc46-4b6f-a48e-cc548fa9259d-085_305_288_141_824}
\captionsetup{labelformat=empty}
\caption{Рис. 3.4}
\end{center}
\end{figure}


\begin{equation*}
\eta=1-\frac{Q_{2}^{\prime}}{Q_{1}} . \tag{3.11}
\end{equation*}


Данный цикл является обратимым (если его проводить бесконечно медленно). Он может быть проведен в обратном направлении, и при этом газ совершает отрицательную работу, нагреватель получает обратно тепло \(Q_{1}\), холодильник отдает газу тепло \(Q_{2}^{\prime}\), которое он получил в прямом цикле. Именно так в принципе работает любой бытовой холодильник.

\begin{figure}[h]
\begin{center}
  \includegraphics[alt={},max width=\textwidth]{bd023a69-cc46-4b6f-a48e-cc548fa9259d-085_287_415_865_96}
\captionsetup{labelformat=empty}
\caption{Рис. 3.5}
\end{center}
\end{figure}

\begin{figure}[h]
\begin{center}
  \includegraphics[alt={},max width=\textwidth]{bd023a69-cc46-4b6f-a48e-cc548fa9259d-085_297_392_855_760}
\captionsetup{labelformat=empty}
\caption{Рис. 3.6}
\end{center}
\end{figure}

Дальнейшие рассуждения проще всего провести, изобразив цикл Карно не на диаграмме \(p-V\), а на диаграмме \(S-T\) (энтропия - температура). На этой диаграмме цикл Карно имеет вид прямоугольника (рис. 3.6). Изотермы изображаются прямыми \(1-2\) и \(3-4\), адиабаты - прямыми \(2-3\) и \(4-1\). Согласно ( 3.3 ) полученное тепло \(Q_{1}=T_{1}\left(S_{2}-S_{1}\right)\) и равно площади под отрезком \(1-2\). Отданное холодильнику тепло \(Q_{2}^{\prime}=T_{2}\left(S_{2}-S_{1}\right)\) и равно площади под отрезком \(4-3\). При этом площадь прямоугольника, т. е. \(Q_{1}-Q_{2}^{\prime}\), равна работе \(A\), совершаемой двигателем за цикл. Подставив выражения \(Q_{1}\) и \(Q_{2}^{\prime}\) в формулу (3.1), получим, что КПД цикла Карно


\begin{equation*}
\eta=1-\frac{T_{2}}{T_{1}} . \tag{3.12}
\end{equation*}


При выводе этой формулы не делалось никаких предположений о свойствах рабочего вещества и устройстве теплового двигателя. Отсюда следует знаменитая теорема Карно:

КПД обратимьх двигателей, работающих по циклу Карно, зависит только от температур \(T_{1}\) и \(T_{2}\) - нагревателя и холодильника, но не зависит ни от устройства двигателя, ни от рода рабочего вещества.\\
Пример. Выясним, в каком случае КПД цикла Карно повышается больше: при увеличении температуры нагревателя на \(\Delta T\) или при уменьшении температуры холодильника на такую же величину.\\
С этой целью возьмем частные производные по \(T_{1}\) и \(T_{2}\) выражения (3.12) для КПД:

\[
\frac{\partial \eta}{\partial T_{1}}=\frac{T_{2}}{T_{1}^{2}}=\frac{T_{2}}{T_{1}} \frac{1}{T_{1}}, \quad \frac{\partial \eta}{\partial T_{2}}=-\frac{1}{T_{1}} .
\]

Так как \(T_{2}<T_{1}\), то \(\partial \eta / \partial T_{1}<\left|\partial \eta / \partial T_{2}\right|\). Значит, при уменьшении температуры холодильника КПД цикла повышается больше. Заметим, что этот вопрос можно решить и с помощью диаграммы \(T-S\).

Можно показать (мы опускаем доказательство), что КПД любого необратимого теплового двигателя, работающего с теми же нагревателем и холодильником, всегда меньше, чем у двигателя, работающего по обратимому циклу Карно:


\begin{equation*}
\eta_{\text {необр }}<\eta_{\text {обр }} \tag{3.13}
\end{equation*}


где \(\eta_{\text {обр }}\) определяется формулой (3.12).\\
6. О других циклах. Решение вопроса о КПД других циклов часто сильно упрощается, если его рассматривать в переменных \(T\) - \(S\). Во многих циклах встречаются участки - изохоры и изобары. Выясним, как они выглядят на диаграмме \(T\) - \(S\).

Пусть начальное состояние газа определяется значениями \(T_{0}, S_{0}\). Повысим температуру до \(T\) сначала изохорически. Это значит, что \(\mathrm{d} S=C_{V} \mathrm{~d} T / T\), откуда


\begin{equation*}
S-S_{0}=C_{V} \ln \frac{T}{T_{0}}, \quad T=T_{0} \exp \left(\frac{S-S_{0}}{C_{V}}\right) . \tag{3.14}
\end{equation*}


Если же процесс проводить изобарически, то вместо \(C_{V}\) надо брать \(C_{p}\), и мы получим


\begin{equation*}
T=T_{0} \exp \left(\frac{S-S_{0}}{C_{p}}\right) . \tag{3.15}
\end{equation*}


Графики функций \(T(S)\) имеют вид экспонент, и поскольку \(\mathrm{C}_{V}<C_{p}\), изохора ( \(V\) ) идет круче (рис. 3.7).

Полученные результаты (3.14) и (3.15) для \(T(S)\) можно обобщить на любые политропические процессы:

\[
T \sim \exp \left(S / C_{n}\right),
\]

\begin{figure}[h]
\begin{center}
  \includegraphics[alt={},max width=\textwidth]{bd023a69-cc46-4b6f-a48e-cc548fa9259d-087_347_394_365_766}
\captionsetup{labelformat=empty}
\caption{Рис. 3.7}
\end{center}
\end{figure}

где \(C_{n}\) - теплоемкость политропического процесса, \(n\) - показатель политропы. Из этого выражения, в частности, следует, что при \(C_{n}<0\) (когда \(1<n<\gamma\) ) зависимость \(T(S)\) имеет вид убывающей с ростом \(S\) экспоненты (пунктирная кривая на рис. 3.7).

Следует также иметь в виду, что в любом политропическом процессе \(\mathrm{d} S=C_{n} \mathrm{~d} T / T\), и значит


\begin{equation*}
S_{2}-S_{1}=C_{n} \ln \left(T_{2} / T_{1}\right) . \tag{3.16}
\end{equation*}


Пример. Идеальный газ совершает прямой цикл, состоящий из изотермы, политропы и адиабаты, причем изотермический процесс происходит при максимальной температуре цикла. Найдем КПД такого цикла, если температура \(T\) в его пределах изменяется в ч раз.

Изобразим данный цикл на диаграмме \(T-S\) (рис. 3.8). По определению, КПД выражает формула (3.1). Решение, таким образом, сводится к нахождению отношения \(Q_{2}^{\prime} / Q_{1}\).

\begin{figure}[h]
\begin{center}
  \includegraphics[alt={},max width=\textwidth]{bd023a69-cc46-4b6f-a48e-cc548fa9259d-087_398_455_1250_695}
\captionsetup{labelformat=empty}
\caption{Рис. 3.8}
\end{center}
\end{figure}

Мы знаем, что

\[
Q_{1}=T_{2}\left(S_{2}-S_{1}\right), \quad Q_{2}^{\prime}=C_{n}\left(T_{2}-T_{1}\right)
\]

(напомним, \(Q_{2}^{\prime}\) должно быть положительным). Отсюда с учетом (3.16) получим

\[
\frac{Q_{2}^{\prime}}{Q_{1}}=\frac{C_{n}\left(T_{2}-T_{1}\right)}{T_{2}\left(S_{2}-S_{1}\right)}=\frac{T_{2}-T_{1}}{T_{2} \ln \left(T_{2} / T_{1}\right)}=\frac{\tau-1}{\tau \ln \tau},
\]

и КПД цикла

\[
\eta=1-\frac{\tau-1}{\tau \ln \tau} .
\]

\section*{§ 3.4. Статистический смысл второго начала термодинамики}
Вернемся в вопросу: как необратимость реальных процессов может быть согласована с обратимостью законов, управляющих движением каждой частицы. Ответ на этот вопрос был найден на основании представления о вероятности отдельных состояний макросистемы и статистического подсчета наиболее вероятных состояний.

Состояние макросистемы может быть охарактеризовано заданием, например, таких макропараметров как объем, давление, температура, число молей и др. В этом случае говорят, что задано макросостояние.

Состояние же макросистемы, охарактеризованное настолько детально, что оказываются заданными состояния всех молекул, называют микросостоянием.

Любое макросостояние может быть реализовано различными способами или различными микросостояниями. Число различных микросостояний, соответствующих данному макросостоянию, называют статистическим весом макросостояния.

Некоторое представление о статистическом весе можно получить, рассмотрев способы, которыми молекулы могут распределяться между двумя половинками сосуда с газом (отвле-

\begin{figure}[h]
\begin{center}
  \includegraphics[alt={},max width=\textwidth]{bd023a69-cc46-4b6f-a48e-cc548fa9259d-088_241_390_1512_68}
\captionsetup{labelformat=empty}
\caption{Рис. 3.9}
\end{center}
\end{figure}

каясь ради простоты от влияния на состояние газа распределением молекул по скоростям). Возьмем, например, случай, когда в сосуде, мысленно разделенным на две одинаковые половины \(A\) и \(B\) (рис. 3.9), находится число молекул \(N=4\). Перенумеруем их: \(1,2,3,4\).

Каждая молекула с равной вероятностью может находиться как в половине \(A\), так и в половине \(B\) сосуда. Значит, вероятность того, что любая молекула окажется в половине \(A\), равна \(1 / 2\). Пребывание в половине \(A\) одной молекулы и одновременно другой - события статистически независимые. Поэтому вероятность одновременного пребывания в половине \(A\) двух молекул равна произведению вероятностей, т. е. \((1 / 2)^{2}\), трех молекул - \((1 / 2)^{3}\), четырех - \((1 / 2)^{4}\).

Подобные рассуждения приводят к выводу, что вероятность любого размещения четырех молекул также равна \((1 / 2)^{4}\). Каждое размещение - это некоторое микросостояние системы, и вероятность \(P\) каждого из них одинакова и равна \((1 / 2)^{4}\).

Все возможные распределения четырех молекул по половинам \(A\) и \(B\) сосуда приведены в табл. 3.1. Из нее видно, какие возможны «макросостояния» (они отделены друг от друга горизонтальными линиями) и сколькими способами (микросостояниями) реализуется каждое «макросостояние».

\begin{table}[h]
\begin{center}
\captionsetup{labelformat=empty}
\caption{Таблица 3.1}
\begin{tabular}{|l|l|l|l|}
\hline
\multicolumn{2}{|c|}{Способы реализации макросостояний} & \multirow{2}{*}{Число способов \(\Omega\)} & \multirow{2}{*}{Обычная вероятность \(P\)} \\
\hline
\(A\) & \(B\) &  &  \\
\hline
- & 1, 2, 3, 4 & 1 & 1/16 \\
\hline
1 & 2, 3, 4 &  & \multirow{4}{*}{4/16} \\
\hline
2 & 1, 3, 4 & 4 &  \\
\hline
3 & 1, 2, 4 &  &  \\
\hline
4 & 1, 2, 3 &  &  \\
\hline
1, 2 & 3, 4 &  & \multirow{6}{*}{6/16} \\
\hline
1, 3 & 2, 4 &  &  \\
\hline
1, 4 & 2, 3 &  &  \\
\hline
2, 3 & 1, 4 &  &  \\
\hline
2, 4 & 1, 3 &  &  \\
\hline
3, 4 & 1, 2 &  &  \\
\hline
1, 2, 3 & 4 &  & \multirow{4}{*}{4/16} \\
\hline
1, 2, 4 & 3 &  &  \\
\hline
1, 3, 4 & 2 & 4 &  \\
\hline
2, 3, 4 & 1 &  &  \\
\hline
1, 2, 3, 4 & - & 1 & 1/16 \\
\hline
\multicolumn{4}{|c|}{Всего микросостояний \(2^{4}=16\)} \\
\hline
\end{tabular}
\end{center}
\end{table}

Под макросостояниями будем понимать состояния, при которых в половине \(A\) сосуда находится, скажем, одна молекула (любая), а в половине \(B\) - три молекулы. Из таблицы видно, что такое макросостояние реализуется четырьмя микросостояниями. Значит, статистический вес данного макросостояния \(\Omega=4\), а обычная вероятность \(P=4 / 16\).

Макросостояние, у которого в обеих половинах сосуда находится одинаковое число молекул (по две), реализуется с помощью шести микросостояний, и статистический вес, а также вероятность такого макросостояния максимальны.

Пусть первоначально все четыре молекулы находились в половине \(A\) сосуда. С течением времени некоторые из молекул, двигаясь беспорядочно, окажутся в половине \(B\). Это будет означать, что «газ» расширился. В дальнейшем «макросостояние» системы будет меняться случайным образом, но чаще будет осуществляться то из них, которому соответствует максимальное значение статистического веса \(\Omega\). Может произойти так, что случайно все четыре молекулы снова окажутся в половине \(A\) сосуда. Это означает, что в рассматриваемом случае (всего четыре молекулы) вполне возможно, что «газ», сперва расширившись, затем самопроизвольно сожмется. Процесс расширения оказался обратимым.

При большой скорости движения молекул различные состояния системы быстро следуют друг за другом и, очевидно, придется не очень долго ждать, чтобы расширившийся «газ» сам собой снова сжался. Однако так обстоит дело только при очень малом числе молекул.

Вследствие статистической независимости поведения молекул идеального газа вероятность любого микросостояния равна произведению вероятностей пребывания молекулы в любой половине сосуда. Для \(N\) частиц это произведение \(P=(1 / 2)^{N}\). Отсюда следует, что полное число возможных микросостояний системы равно \(2^{N}\). При \(N=10\) получим \(P \approx 10^{-3}\). Если в течение длительного времени фиксировать распределение молекул через равные промежутки времени, то на каждые 1000 случаев в среднем придется один случай, когда все 10 молекул будут находиться, например, только в половине \(A\) сосуда.

Но при \(N=100\) мы получим \(P \sim 10^{-30}\) ! В макросистемах мы имеем дело с очень большим числом частиц. Скажем, если \(N=10^{20}\), то для вероятности получается невообразимо малая величина \(P \sim 10^{-3} \cdot 10^{19}\). Такова вероятность, что газ самопроизвольно сожмется в одной половине сосуда. Ясно, что при таком значении вероятности этим событием можно пренебречь - оно практически неосуществимо.

Вместе с тем статистический вес (и вероятность) макросстстояния, при котором молекулы распределятся равномерно по обеим половинам сосуда, стремительно растет с ростом числа молекул \(N\).

Таким образом, мы приходим к важному выводу: предоставленная самой себе макросистема стремится переходить от менее вероятных состояний к более вероятным. В этом суть необратимости. В принципе необратимый процесс возможен, но вероятность его ничтожно мала.

Таким образом, второе начало термодинамики, указывающее на необратимость перехода работы в тепло, обусловлено тем, что переход теплоты в работу означает переход от более вероятного состояния к менее вероятному.

Подчеркнем: понятие необратимости процессов имеет смысл только для макросистем. К совокупности небольшого числа частиц это понятие не применимо. Это мы видели на примере системы из четырех молекул - самопроизвольное сжатие такого «газа» вполне возможно, т. е. процесс обратим.

В состоянии равновесия число молекул в обеих половинах сосуда можно считать одинаковым с тем большей точностью, чем больше число молекул. При этом неизбежные флуктуации становятся ничтожно малыми, и ими можно спокойно пренебречь (речь идет, разумеется, об относительных флуктуациях).

\section*{§ 3.5. Энтропия и вероятность}
Если макросистема находится в неравновесном состоянии, то она самопроизвольно будет переходить в состояние с большей вероятностью - равновесное. Вместе с тем, согласно второму началу термодинамики все самопроизвольные процессы в замкнутых макросистемах сопровождаются возрастанием энтропии.

Поэтому можно ожидать, что между энтропией \(S\) макросистемы в каждом состоянии и вероятностью \(P\) того же состояния должна существовать определенная связь. Эта идея, высказанная Больцманом, оправдалась и оказалась весьма плодотворной.

Для нахождения этой связи рассмотрим аналогично примеру на стр. 83 необратимый процесс расширения идеального газа в пустоту. Пусть данный газ первоначально находится

\begin{figure}[h]
\begin{center}
  \includegraphics[alt={},max width=\textwidth]{bd023a69-cc46-4b6f-a48e-cc548fa9259d-092_270_471_453_68}
\captionsetup{labelformat=empty}
\caption{Рис. 3.10}
\end{center}
\end{figure}

в объеме \(V_{1}\) теплоизолированного сосуда объемом \(V_{0}\). От остальной части сосуда он отделен перегородкой (рис. 3.10). Перегородку практически мгновенно перемещают из положения 1 в 2 , газ расширяется в пустоту до объема \(V_{2}\) и приходит в равновесное состояние.\\
В данном случае (газ идеальный) работу газ не совершает ( \(A=0\) ), переданное газу тепло \(Q=0\), следовательно, по первому началу приращение внутренней энергии \(\Delta U=0\), т. е. температура конечного и начального состояний одинакова.

Поскольку энтропия - функция состояния, то ее приращение в процессе \(V_{1} \rightarrow V_{2}\) можно вычислить по обратимому процессу, например, изотермическому. В изотермическом процессе согласно (1.25) \(Q=A=v R T \ln \left(V_{2} / V_{1}\right)\) и


\begin{equation*}
\Delta S=Q / T=v R \ln \left(V_{2} / V_{1}\right)=k N \ln \left(V_{2} / V_{1}\right), \tag{3.14}
\end{equation*}


где \(N\) - число молекул в газе.\\
Теперь обратимся к вероятностям. В рассмотренном процессе распределение по скоростям в начальном и конечном состояниях одинаково: оно зависит только от температуры \(T\), которая не изменилась. Пространственное же распределение молекул стало более «свободным», а значит и более вероятным. В самом деле, вероятность нахождения одной молекулы газа в объеме \(V_{1}\) равна, очевидно, \(V_{1} / V_{0}\). Вероятность же всех \(N\) молекул собраться в объеме \(V_{1}\) равна \(\left(V_{1} / V_{0}\right)^{N}\). Обозначим эту вероятность как \(P_{1}\). Соответственно \(\left(V_{2} / V_{0}\right)^{N}\) - как \(P_{2}\). Тогда отношение этих вероятностей


\begin{equation*}
P_{2} / P_{1}=\left(V_{2} / V_{1}\right)^{N}, \tag{3.15}
\end{equation*}


и приращение энтропии (3.14) можно записать как


\begin{equation*}
\Delta S=k N \ln \frac{V_{2}}{V_{1}}=k \ln \left(\frac{V_{2}}{V_{1}}\right)^{N}=k \ln \frac{P_{2}}{P_{1}} . \tag{3.16}
\end{equation*}


Поскольку вероятность макросистемы пропорциональна ее статистическому весу, т.е. \(P \sim \Omega\), последнюю формулу представим так:


\begin{equation*}
\Delta S=k \ln \left(\Omega_{2} / \Omega_{1}\right), \tag{3.17}
\end{equation*}


и мы приходим к знаменитой формуле Больцмана


\begin{equation*}
S=k \ln \Omega, \tag{3.18}
\end{equation*}


из которой и следует (3.17).\\
Заметим, что приведенные здесь рассуждения не претендуют на вывод формулы (3.18), а представляют собой скорее некоторые пояснения. Строгий вывод этой формулы дается в теоретической физике, где, кстати, показывается, что (3.18) относится не только к равновесным состояниям, но и к неравновесным.

Теперь предположим, что макросистема состоит из двух практически не взаимодействующих подсистем, одна из которых находится в состоянии 1 с энтропией \(S_{1}\) и статистическим весом \(\Omega_{1}\), а другая - в состоянии 2 с энтропией \(S_{2}\) и статистическим весом \(\Omega_{2}\).

Число способов (микросостояний), которыми может реализоваться рассматриваемое состояние макросистемы, равно произведению чисел способов, т. е. \(\Omega_{1}\) и \(\Omega_{2}\), которыми могут быть осуществлены состояния каждой из подсистем в отдельности:


\begin{equation*}
\Omega=\Omega_{1} \cdot \Omega_{2} . \tag{3.19}
\end{equation*}


Отсюда следует, что \(S=k \ln \left(\Omega_{1} \Omega_{2}\right)=k \ln \Omega_{1}+k \ln \Omega_{2}=S_{1}+S_{2}\), как и должно быть, поскольку энтропия - величина аддитивная. Это, как видно, согласуется с (3.18).

Принцип возрастания энтропии со статистической точки зрения привел Больцмана к фундаментальному выводу:

все замкнутые макросистемы стремятся переходить от состояний менее вероятных к более вероятным.

При этом сама энтропия \(S\) характеризует степень беспоряд\(к а\) в макросистеме: состояниям с бо́льшим беспорядком отвечает бо́льшая вероятность (или статистический вес \(\Omega\) ), чем у более упорядоченного состояния, а поэтому и бо́льшее значение энтропии.

Возрастание энтропии с ростом беспорядка в макросистеме - этот вопрос решается далеко не так просто, как может показаться. Об этом свидетельствует хотя бы следующий пример.

Пример. На рис. 3.11 показаны распределения Максвелла молекул по

\begin{figure}[h]
\begin{center}
  \includegraphics[alt={},max width=\textwidth]{bd023a69-cc46-4b6f-a48e-cc548fa9259d-094_334_386_556_209}
\captionsetup{labelformat=empty}
\caption{Рис. 3.11}
\end{center}
\end{figure}

На первый взгляд, энтропия макросистемы при \(T_{2}\) будет больше: при большей температуре молекулы больше разбросаны по скоростям, значит, беспорядок больше. Однако этот вывод оказывается не однозначным. В самом деле, ведь мы не знаем как произошло нагревание, в результате какого процесса. Это поясняет рис. 3.7, где на диаграмме \(T-S\) кроме изохоры ( \(V\) ) и изобары ( \(p\) ), показан пунктиром один из множества процессов, в результате которого с ростом \(T\) энтропия \(S\) уменьшается. И связано это с тем, что, характеризуя степень беспорядка, мы должны учитывать не только распределение (разброс) молекул по скоростям, но и по пространству. Для «пунктирного» политропического процесса объем макросистемы с ростом \(T\) уменьшается, значит, «пространственный» беспорядок становится меньше. Последний в данном случае и превалирует.

В результате энтропия макросистемы при повышении ее температуры от \(T_{1}\) до \(T_{2}\) может как увеличиться, так и уменьшиться. Без дополнительных условий этот вопрос однозначно решить невозможно.

С этим связана и необратимость реальных самопроизвольных тепловых процессов: они протекают так, что беспорядок в макросистеме увеличивается. С этим связан и тот факт, что любой вид энергии переходит в конце концов во внутреннюю

энергию, т. е. в состояние, при котором «хаос» в макросистеме максимален. Это состояние является равновесным, его энтропия \(S=\) макс.

Каково бы ни было первоначальное состояние макросистемы (например, газа), будучи теплоизолированной она неизбежно переходит в состояние, при котором распределение молекул по скоростям будет максвелловским, а во внешнем поле еще и больцмановским.

Энтропия и судьба Вселенной. Принцип возрастания энтропии приводит к мысли (Клаузиус), что энтропия Вселенной приближается к максимуму, по достижении которого во Вселенной прекратятся какие бы то ни были процессы. Должно наступить абсолютно равновесное состояние, в котором никакие процессы уже невозможны. Наступит тепловая смерть Вселенной.

В связи с этой концепцией Больцманом была высказана так называемая флуктуационная гипотеза. Больцман не отрицал применимость принципа возрастания энтропии ко всей Вселенной в целом (а такие сомнения высказывались), но он обратил внимание на статистическую природу этого закона. Поэтому отступления от термодинамического равновесия Вселенной флуктуации - не только возможны, но и неизбежны. Сейчас мы имеем дело с гигантской флуктуацией. Она должна исчезнуть. Тогда наступит тепловая смерть Вселенной. Однако через некоторое время снова возникнет гигантская флуктуация, и Вселенная выйдет из состояния тепловой смерти. Затем опять все повторится, и так без конца.

В настоящее время установлено, что вывод о «тепловой смерти» Вселенной и первоначальные попытки его опровержения являются несостоятельными, поскольку в них не учитывалось влияние тяготения. Выяснилось, что из-за тяготения однородное изотермическое распределение вещества во Вселенной не соответствует максимуму энтропии, поскольку такое состояние не является наиболее вероятным. Вселенная нестационарна - она расширяется, и первоначально однородное вещество распадается под действием сил тяготения, образуя скопления галактик, сами галактики, звезды и т. д. Эти процессы происходят с ростом энтропии - в соответствии со вто-

рым началом термодинамики. И ниоткуда не следует, что эти процессы приведут к однородному изотермическому состоянию Вселенной, т. е. к «тепловой смерти» Вселенной.

\section*{§ 3.6. Термодинамические соотношения}
В исходную формулу (3.2), являющуюся определением энтропии, подставим выражение для \(d^{\prime} Q\) из первого начала термодинамики:


\begin{equation*}
\mathrm{d} S=\frac{1}{T} \mathrm{~d} U+\frac{p}{T} \mathrm{~d} V \tag{3.20}
\end{equation*}


Это выражение наводит на мысль рассматривать энтропию как функцию двух переменных - внутренней энергии \(U\) и объема \(V\), т.е. \(S(U, V)\). Тогда в соответствии с общим правилом определения дифференциала функции нескольких переменных


\begin{equation*}
\mathrm{d} S=\left(\frac{\partial S}{\partial U}\right)_{V} \mathrm{~d} U+\left(\frac{\partial S}{\partial V}\right)_{U} \mathrm{~d} V \tag{3.21}
\end{equation*}


Из сравнения (3.21) с (3.20) видно, что


\begin{equation*}
\left(\frac{\partial S}{\partial U}\right)_{V}=\frac{1}{T}, \quad\left(\frac{\partial S}{\partial V}\right)_{U}=\frac{p}{T} \tag{3.22}
\end{equation*}


Эти равенства начинают собой длинный ряд различных термодинамических соотношений, которые получаются при переходе от энтропии \(S\) к другим функциям (так называемым термодинамическим потенциалам), а от переменных \(U\) и \(V\) к другим переменным.

Представим энтропию как функцию переменных \(T\) и \(V\), т.е. \(S(T, V)\). Дифференциал внутренней энергии в этих переменных


\begin{equation*}
\mathrm{d} U=\left(\frac{\partial U}{\partial T}\right)_{V} \mathrm{~d} T+\left(\frac{\partial U}{\partial V}\right)_{T} \mathrm{~d} V . \tag{3.23}
\end{equation*}


Подставив это выражение в (3.20), получим

\[
\mathrm{d} S=\frac{1}{T}\left(\frac{\partial U}{\partial T}\right)_{V} \mathrm{~d} T+\left[\frac{1}{T}\left(\frac{\partial U}{\partial V}\right)_{T}+\frac{p}{T}\right] \mathrm{d} V
\]

откуда следует, что


\begin{equation*}
\frac{1}{T}\left(\frac{\partial U}{\partial T}\right)_{V}=\left(\frac{\partial S}{\partial T}\right)_{V}, \quad \frac{1}{T}\left(\frac{\partial U}{\partial V}\right)_{T}+\frac{p}{T}=\left(\frac{\partial S}{\partial V}\right)_{T} . \tag{3.24}
\end{equation*}


Если исключить энтропию \(S\) из этих равенств, то можно получить соотношение между более привычными переменными. Для этого продифференцируем первое равенство (3.24) по \(V\), а второе - по \(T\). Из равенства правых частей - смешанных производных,

\[
\frac{\partial^{2} S}{\partial T \partial V}=\frac{\partial^{2} S}{\partial V \partial T},
\]

получим

\[
\frac{1}{T} \frac{\partial^{2} U}{\partial T \partial V}=\frac{\partial}{\partial T}\left[\frac{1}{T}\left(\frac{\partial U}{\partial V}\right)_{T}+\frac{p}{T}\right]_{V} .
\]

После выполнения дифференцирования и сокращения получим:


\begin{equation*}
\frac{\partial}{\partial T}\left(\frac{p}{T}\right)=\frac{1}{T^{2}} \frac{\partial U}{\partial V} . \tag{3.25}
\end{equation*}


Чтобы не возникло впечатления, что термодинамические соотношения - это просто какие-то математические упражнения в частных производных, покажем на следующем примере, какие глубокие физические результаты можно извлечь из этих соотношений.

Пример. Зная, что давление теплового излучения \(p=u / 3\), где \(u=U / V\) - плотность излучения, являющаяся функцией только \(T\), найдем уравнение состояния этого излучения. Подставим приведенные выражения для \(p\) и \(u\) в (3.25):

\[
\frac{\partial}{\partial T}\left(\frac{u}{3 T}\right)=\frac{1}{T^{2}} \frac{\partial}{\partial V}(u V),
\]

откуда \(\partial u / \partial T=4 u / T\), или \(\partial u / u=4 \partial T / T\). Это дифференциальное уравнение представим в виде

\[
\partial \ln \left(u / T^{4}\right)=0 .
\]

Значит, \(u / T^{4}=\) const, т. е. \(u\) c \(T^{4}\). Мы получили важный peзультат, подтвержденный экспериментально.

В заключение затронем вопрос, связанный с так называемыми термодинамическими потенциалами. Так называют функции состояния, на использовании которых основаны все расчеты в термодинамике. Существует множество таких функций, но лишь некоторые играют важную роль. Мы ограничимся двумя - внутренней энергией \(U\) и свободной энергией \(F\), поскольку другие нам не понадобятся.

Внутренняя энергия \(\boldsymbol{U}\). С этой функцией мы уже познакомились. Представим первое начало термодинамики с учетом того, что \(\mathrm{d}^{\prime} Q=T \mathrm{~d} S\), в виде


\begin{equation*}
\mathrm{d} U=T \mathrm{~d} S-p \mathrm{~d} V . \tag{3.26}
\end{equation*}


Это соотношение позволяет утверждать, что


\begin{equation*}
\left(\frac{\partial U}{\partial S}\right)_{V}=T, \quad\left(\frac{\partial U}{\partial V}\right)_{S}=-p . \tag{3.27}
\end{equation*}


Из того же первого начала следует, что при отсутствии теплообмена с окружающей средой работа, совершаемая макросистемой, равна убыли внутренней энергии:


\begin{equation*}
A=-\Delta U . \tag{3.28}
\end{equation*}


Это соотношение, кстати, может служить термодинамическим определением внутренней энергии.

Свободная энергия \(\boldsymbol{F}\). Перепишем (3.26) в виде


\begin{equation*}
\mathrm{d}^{\prime} A=-\mathrm{d} U+T \mathrm{~d} S . \tag{3.29}
\end{equation*}


При изотермическом процессе ( \(T=\) const) работа \(\mathrm{d}^{\prime} A\) может быть определена как


\begin{equation*}
\mathrm{d}^{\prime} A=-\mathrm{d}(U-T S)=-\mathrm{d} F, \tag{3.30}
\end{equation*}


где \(F\) и есть свободная энергия:


\begin{equation*}
F=U-T S . \tag{3.31}
\end{equation*}


Таким образом, работа, совершаемая макросистемой при изотермическом процессе, равна убыли свободной энергии:


\begin{equation*}
A=-\Delta F . \tag{3.32}
\end{equation*}


При изотермических процессах свободная энергия \(F\) играет роль потенциальной энергии: ее убыль равна производимой работе, и равновесному состоянию макросистемы соответствует \(F=\) мин.

Представим \(F\) как функцию двух переменных. Для этого возьмем дифференциал от (3.31) и примем во внимание (3.26). Тогда


\begin{equation*}
\mathrm{d} F=\mathrm{d} U-\mathrm{d}(T S)=T \mathrm{~d} S-p \mathrm{~d} V-T \mathrm{~d} S-S \mathrm{~d} T=-p \mathrm{~d} V-S \mathrm{~d} T . \tag{3.33}
\end{equation*}


Отсюда следует, что


\begin{equation*}
\left(\frac{\partial F}{\partial V}\right)_{T}=p, \quad\left(\frac{\partial F}{\partial T}\right)_{V}=-S . \tag{3.34}
\end{equation*}


Аналогичные соотношения можно извлечь и из других термодинамических потенциалов.

\section*{Задачи}
3.1. Энтропия. Один моль идеального газа с известным значением теплоемкости \(C_{V}\) совершает процесс, при котором его энтропия \(S\) зависит от температуры \(T\) как \(S=\alpha / T\), где \(\alpha\) - постоянная. Температура газа изменилась от \(T_{1}\) до \(T_{2}\). Найти:\\
a) молярную теплоемкость газа как функцию \(T\);

б\} количество теплоты, сообщенной газу.\\
Решение. а) Молярная теплоемкость в данном процессе

\[
C=\frac{\mathrm{d}^{\prime} Q}{\mathrm{~d} T}=\frac{T \mathrm{~d} S}{\mathrm{~d} T}=-\frac{\alpha}{T} .
\]

Видно, что если \(\alpha>0\), то \(C<0\), и наоборот.\\
б) \(Q=\int C \mathrm{~d} T=-\alpha \int_{T_{1}}^{T_{2}} \frac{\mathrm{~d} T}{T}=\alpha \ln \left(T_{1} / T_{2}\right)\).\\
3.2. В результате политропического процесса объем идеального газа уменьшился в \(v\) раз. При этом работа, совершенная над газом, \(A^{\prime}= =2 \Delta U\), где \(\Delta U\) - приращение его внутренней энергии. Найти приращение энтропии в этом процессе.\\
\(P\) ешение. По определению \(\Delta S=\int d^{\prime} Q / T\). Найдем \(d^{\prime} Q\). Согласно первому началу термодинамики в данном случае


\begin{equation*}
Q=\Delta U-A^{\prime}=-\Delta U=-C_{V} \mathrm{~d} T . \tag{1}
\end{equation*}


Следовательно, \(\mathrm{d}^{\prime} Q=-C_{V} \mathrm{~d} T\) и


\begin{equation*}
\Delta S=-\int_{T_{1}}^{T_{2}} C_{V} \frac{\mathrm{~d} T}{T}=-C_{V} \ln \left(\frac{T_{2}}{T_{1}}\right) . \tag{2}
\end{equation*}


Остается выяснить, чему равно отношение \(T_{2} / T_{1}\). Из (1) следует, что теплоемкость данного политропического процесса \(C_{n}=-C_{V}\). Но \(C_{n}\) согласно (1.21) равно разности \(C_{V}-R /(n-1)\). Это значит, что \(2 C_{V}=R /(n-1)\). Отсюда


\begin{equation*}
n-1=(\gamma-1) / 2 . \tag{3}
\end{equation*}


Имея в виду, что уравнение политропического процесса \(T V^{n-1}=\) = const, получим

\[
T_{2} / T_{1}=\left(V_{2} / V_{1}\right)^{n-1}=\nu^{(\gamma-1) / 2} .
\]

Таким образом, уравнение (2) примет вид

\[
\Delta S=-\frac{R}{\gamma-1} \ln v^{(\gamma-1) / 2}=-\frac{R}{2} \ln v .
\]

3.3. Один моль ван-дер-ваальсовского газа, имевший объем \(V_{1}\) и температуру \(T_{1}\), перевели в состояние с объемом \(V_{2}\) и температурой \(T_{2}\). Найти соответствующее приращение энтропии газа, считая его молекулярную теплоемкость \(C_{V}\) и ван-дер-ваальсовскую поправку \(b\) известными.

Решении е. Искомая величина

\[
\Delta S=\int \frac{\mathrm{d} U+p \mathrm{~d} V}{T} .
\]

Используя выражения для внутренней энергии \(U\) ван-дер-ваальсовского газа (1.44) и для давления \(p\) из уравнения его состояния (1.42), получим после сокращений

\[
\Delta S=\int_{T_{1}}^{T_{2}} C_{V} \frac{\mathrm{~d} T}{T}+\int_{V_{1}}^{V_{2}} \frac{R \mathrm{~d} V}{V-b}=C_{V} \ln \frac{T_{2}}{T_{1}}+R \ln \frac{V_{2}-b}{V_{1}-b} .
\]

3.4. Идеальный газ с показателем адиабаты \(\gamma\) совершает процесс по закону \(p=p_{0}-\alpha V\), где \(p_{0}\) и \(\alpha\) - положительные постоянные, \(V\) объем газа. При каком значении объема \(V_{m}\) энтропия газа окажется максимальной?

P е ш е н и е. Энтропия будет максимальной в точке, где \(\mathrm{d} S=0\). Воспользовавшись формулой (3.9), запишем


\begin{equation*}
C_{V} \frac{\mathrm{~d} p}{p}+C_{p} \frac{\mathrm{~d} V}{V}=0, \tag{1}
\end{equation*}


откуда следует, что


\begin{equation*}
\frac{\mathrm{d} p}{\mathrm{~d} V}=-\frac{C_{p}}{C_{V}} \frac{p}{V}=-\gamma \frac{p_{0}-\alpha V}{V} . \tag{2}
\end{equation*}


Кроме того, из уравнения процесса производная


\begin{equation*}
\mathrm{d} p / \mathrm{d} V=-\alpha . \tag{3}
\end{equation*}


Подстановка (3) в (2) приводит к уравнению


\begin{equation*}
\gamma\left(\frac{p_{0}}{V}-\alpha\right)=\alpha, \tag{4}
\end{equation*}


откуда объем \(V_{m}\), отвечающий максимуму энтропии \(S\), равен

\[
V_{m}=\frac{\gamma p_{0}}{\alpha(\gamma+1)}
\]

3.5. Теплоизолированный сосуд разделен перегородкой на две части так, что объем одной из них в \(n\) раз больше объема другой. В меньшей части находится \(v_{1}\) моля одного газа, а в большей части \(v_{2}\) моля другого газа. Температура газов одинакова. Перегородку удалили, и газы перемешались. Найти приращение энтропии \(\Delta S\) макросистемы, считая газы идеальными.

Р еш ени е. Из аддитивности энтропии следует, что \(\Delta S=\Delta S_{1}+ +\Delta S_{2}\), где оба приращения \(\Delta S_{1}\) и \(\Delta S_{2}\) найдем, используя изотермический процесс расширения каждого газа. Напомним, что это по существу чисто математический прием, основанный на том, что \(S\) является функцией состояния. Итак,

\[
\Delta S_{1}=\int \frac{p \mathrm{~d} V}{T}=v_{1} R \ln \frac{V_{1}+V_{2}}{V_{1}}, \quad \Delta S_{2}=v_{2} R \ln \frac{V_{2}+V_{1}}{V_{2}} .
\]

Имея в виду, что \(V_{2} / V_{1}=n\), получим

\[
\Delta S=\Delta S_{1}+\Delta S_{2}=v_{1} R \ln (n+1)+v_{2} R \ln (n+1 / n) .
\]

3.6. Два одинаковых теплоизолированных сосуда, соединенные трубкой с вентилем, содержат по одному молю одного и того же идеального газа. Температура газа в одном сосуде \(T_{1}\), в другом \(T_{2}\). Молярная теплоемкость газа равна \(C_{V}\). Вентиль открыли, и газ пришел в новое состояние равновесия. Найти приращение \(\Delta S\) энтропии газа в этом процессе. Показать, что \(\Delta S>0\).

Р е ш е н и е. По условию \(Q=0, A=0\), значит согласно первому началу и \(\Delta U=0\), т. е. \(U_{1}+U_{2}=U\), откуда


\begin{equation*}
T_{1}+T_{2}=2 T \tag{1}
\end{equation*}


где \(T\) - установившаяся температура. Зная начальное и конечное равновесные состояния, для подсчета \(\Delta S\) возьмем наиболее простой обратимый процесс - изохорический:


\begin{equation*}
\Delta S=C_{V} \ln \frac{T}{T_{1}}+C_{V} \ln \frac{T}{T_{2}}+\Delta S_{0}, \tag{2}
\end{equation*}


где \(\Delta S_{0}\) - приращение энтропии за счет перемешивания газов. Но при перемешивании тождественных газов \(\Delta S_{0}=0\) (см. парадокс Гиббса, § 3.3).

Перепишем (2) с учетом (1) в виде


\begin{equation*}
\Delta S=C_{V} \ln \frac{T^{2}}{T_{1} T_{2}}=C_{V} \ln \frac{\left(T_{1}+T_{2}\right)^{2}}{4 T_{1} T_{2}} . \tag{3}
\end{equation*}


Докажем теперь, что \(\Delta S>0\). Для этого поступим так:

\[
\begin{gathered}
\left(T_{1}+T_{2}\right)^{2}=T_{1}^{2}+T_{2}^{2}+2 T_{1} T_{2}-2 T_{1} T_{2}+2 T_{1} T_{2}= \\
=\left(T_{1}-T_{2}\right)^{2}+4 T_{1} T_{2}
\end{gathered}
\]

Тогда выражение под логарифмом в (3) будет иметь вид


\begin{equation*}
\frac{\left(T_{1}+T_{2}\right)^{2}}{4 T_{1} T_{2}}=\frac{\left(T_{1}-T_{2}\right)^{2}}{4 T_{1} T_{2}}+1 . \tag{4}
\end{equation*}


Отсюда видно, что оно больше единицы, значит его логарифм больше нуля и \(\Delta S>0\).\\
3.7. Цикл Карно. Один моль идеального газа из жестких двухатомных молекул совершает цикл Карно. Температура нагревателя \(T_{1}=400\) К. Найти КПД цикла, если при адиабатическом сжатии затрачивается работа \(A^{\prime}=2,0\) кДж.

Р е ш е н и е. При адиабатическом сжатии \(Q=0\), следовательно, совершаемая над газом работа идет на приращение внутренней энергии газа:


\begin{equation*}
A^{\prime}=\Delta U=C_{V}\left(T_{1}-T_{2}\right)=C_{V} T_{1}\left(1-\frac{T_{2}}{T_{1}}\right) . \tag{*}
\end{equation*}


Величина в последних круглых скобках - это КПД η. Поэтому из (*) следует, что

\[
\eta=A^{\prime} / C_{V} T_{1} .
\]

Поскольку \(C_{V}=(i / 2) R\), то

\[
\eta=2 A^{\prime} / i R T_{1}=0,24,
\]

где \(i=5\) (молекула жесткая).\\
3.8. КПД цикла. Один моль идеального газа с постоянной адиабаты \(\gamma\) совершает цикл, состоящий из изохоры, изотермы и изобары, причем изотермическое расширение происходит при максимальной температуре цикла. Температура в пределах цикла изменяется в \(\tau\) раз. Найти КПД цикла.\\
Решение. Изобразим данный цикл на диаграмме \(T-S\) (рис. 3.12). Буквами \(V\) и \(p\) отмечены изохорический и изобарический процессы (см. рис. 3.7). Согласно формуле (3.1) решение сводится к нахождению отношения \(Q_{2}^{\prime} / Q_{1}\). Выпишем сначала их отдельно:


\begin{align*}
Q_{2}^{\prime} & =C_{p}\left(T_{2}-T_{1}\right) \\
Q_{1} & =C_{V}\left(T_{2}-T_{1}\right)+T_{2}\left(S_{3}-S_{2}\right) \tag{1}
\end{align*}


Определим \(S_{3}-S_{2}\) с помощью фор-

\begin{figure}[h]
\begin{center}
  \includegraphics[alt={},max width=\textwidth]{bd023a69-cc46-4b6f-a48e-cc548fa9259d-103_415_461_375_701}
\captionsetup{labelformat=empty}
\caption{Рис. 3.12}
\end{center}
\end{figure}


\begin{equation*}
S_{3}-S_{1}=C_{p} \ln \left(T_{2} / T_{1}\right), \quad S_{2}-S_{1}=C_{V} \ln \left(T_{2} / T_{1}\right) \tag{2}
\end{equation*}


откуда их разность


\begin{equation*}
S_{3}-S_{2}=\left(C_{p}-C_{V}\right) \ln \tau=R \ln \tau . \tag{3}
\end{equation*}


Отношение \(Q_{2}^{\prime} / Q_{1}\) согласно приведенным формулам можно привести к виду

\[
\frac{Q_{2}^{\prime}}{Q_{1}}=\frac{\gamma}{1+\frac{(\gamma-1) \tau}{\tau-1} \ln \tau} .
\]

Искомый КПД цикла

\[
\eta=1-\frac{\gamma}{1+\frac{\gamma-1}{\tau-1} \tau \ln \tau}
\]

Видно, что в отличие от цикла Карно КПД данного цикла зависит от рода вещества ( \(\gamma\) ).\\
3.9. Статистический вес. Один моль идеального газа, состоящего из одноатомных молекул, находится в сосуде при температуре \(T_{0}=300\) К. Как и во сколько раз изменится статистический вес этой макросистемы, если ее нагреть изохорически на \(\Delta T=1,0\) К?

P e ш e н и е. Исходя из формулы (3.18), имеем


\begin{equation*}
\Delta S=k \ln \left(\Omega / \Omega_{0}\right) . \tag{1}
\end{equation*}


Видно, что решение сводится к нахождению \(\Delta S\) :


\begin{equation*}
\Delta S=\int \frac{\mathrm{d}^{\prime} Q}{T}=C_{V} \ln \frac{T}{T_{0}}=\frac{i}{2} R \ln \left(1+\frac{\Delta T}{T_{0}}\right) . \tag{2}
\end{equation*}


Из сопоставления формул (1) и (2) получим


\begin{equation*}
\ln \frac{\Omega}{\Omega_{0}}=\frac{i}{2} N_{A} \ln \left(1+\frac{\Delta T}{T_{0}}\right) \approx \frac{i}{2} N_{A} \frac{\Delta T}{T_{0}}=3 \cdot 10^{21} . \tag{3}
\end{equation*}


Здесь учтено, что \(\Delta T / T_{0} \ll 1\). Из последней формулы находим:


Р е ш е н и е. Исходя из формулы (3.18), имеем S = k ln(8/80). (1) Видно, что решение сводится к нахождению S: S Q T C T T i R T T V           d ln ln

 . (2) Из сопоставления формул (1) и (2) получим ln ln



3 10         i N T T i N T T A A  . (3) Здесь учтено, что T/T0 I 1. Из последней формулы находим:

/ ,

3 10 1 3 10


\[
\Omega / \Omega_{0}=\mathrm{e}^{3 \cdot 10^{21}}=10^{1,3 \cdot 10^{21}}
\]

\begin{itemize}
  \item величина чудовищно грандиозная даже при таких скромных изменениях температуры.\\
3.10. Найти статистический вес \(\Omega\) наиболее вероятного распределения \(N=10\) одинаковых молекул по двум одинаковым половинам сосуда. Чему равна вероятность такого распределения?
\end{itemize}

Решение. Рассмотрим такое макросостояние, при котором в одной (первой) половине сосуда находится \(n\) молекул, а в другой (второй) \(N-n\). Найдем число способов (микросостояний), которыми может быть реализовано такое макросостояние. Это число равно числу перестановок \(N\) молекул, т. е. \(N\) ! Однако перестановки, при которых изменяется только порядок размещения в первой половине (т. е. \(n\) !) не приводят к новому микросостоянию (напомним, что отдельные микросостояния отличаются только номерами молекул, попадающих в разные половины сосуда).

Каждому из возможных распределений остальных молекул во второй половине сосуда соответствует \(n\) ! различных размещений молекул в первой половине. Поэтому, разделив полное число перестановок \(N\) ! на \(n\) !, мы исключим перестановки, которые отличаются лишь способом размещения молекул в первой половине сосуда.

Разделив затем \(N!/ n!\) на ( \(N-n\) )!, мы исключим перестановки во второй половине сосуда. В результате число распределений моле-

кул по обеим половинам сосуда, отличающихся лишь номерами молекул в обеих половинах, определяется как

\[
\Omega(n, \quad N-n)=\frac{N!}{n!(N-n)!} .
\]

Эта величина есть число микросостояний, соответствующих макросостоянию \((n, N-n)\). При наиболее вероятном распределении \(n=N / 2\), и мы имеем

\[
\Omega_{\text {Bep }}=\frac{N!}{[(N / 2)!]^{2}}=252 .
\]

Соответствующая вероятность

\[
P=\Omega_{\text {вер }} / 2^{N}=25 \%
\]

где \(2^{N}\) - полное число микросостояний.\\
3.11. Свободная энергия. Найти энтропию одного моля азота ( \(\gamma=1,4\) ) при температуре \(T_{1}=300 К\), если при обратимом адиабатическом сжатии его в \(v=5,0\) раза приращение свободной энергии оказалось равным \(\Delta F=-48,5\) кДж. Газ считать идеальным.\\
Решение. Распишем приращение свободной энергии в данных условиях:


\begin{equation*}
\Delta F=\Delta U-S\left(T_{2}-T_{1}\right)=\left(C_{V}-S\right)\left(T_{2}-T_{1}\right) . \tag{1}
\end{equation*}


При адиабатическом процессе ( \(T V^{\gamma-1}=\) const)


\begin{equation*}
T_{2} / T_{1}=\left(V_{1} / V_{2}\right)^{\gamma-1}=v^{\gamma-1} \tag{2}
\end{equation*}


Совместное решение уравнений (1) и (2) дает

\[
S=C_{V}-\frac{\Delta F}{T_{2}-T_{1}}=\frac{R}{\gamma-1}-\frac{\Delta F}{T_{1}\left(v^{\gamma-1}-1\right)}=20,8+179 \text { Дж } / \text { К }
\]

Таким образом, \(S=0,20\) кДж/К.

\section*{Квантовые статистики и их применения}
\section*{§ 4.1. Квантовые статистики}
В квантовой физике, как и в статистической, закономерности имеют вероятностный, статистический характер. Однако есть и принципиальное отличие: в квантовой физике статистический (вероятностный) подход лежит в самой природе микрочастиц, в их волновых свойствах.

Согласно квантовой теории все микрочастицы подразделяют на два класса, которым соответствуют две квантовые статистики:


В квантовой физике, как и в статистической, закономерности имеют вероятностный, статистический характер. Однако есть и принципиальное отличие: в квантовой физике статистический (вероятностный) подход лежит в самой природе микрочастиц, в их волновых свойствах.

Согласно квантовой теории все микрочастицы подразделяют на два класса, которым соответствуют две квантовые стати


\begin{enumerate}
  \item частицы с полуцельл спином, их называют фермионами; они подчиняются статистике Ферми-Дирака;
  \item частицы с иельлм спином - бозонью; они подчиняются статистике Бозе-Эйнштейна.
\end{enumerate}

Других возможностей квантовая теория не допускает. Нет частиц, подчиняющихся классической статистике Больцмана. Последняя является приближенным предельным случаем, в который переходят при определенных условиях эти две квантовые статистики*.

Во всех трех статистиках (классической, Бозе-Эйнштейна и Ферми-Дирака) допустимые микросостояния считаются равновероятными. Но различие их - в способах определения микросостояний и статистических весов. В статистике Больцмана считается, что даже тождественные частицы принципиально различимы. В квантовых же статистиках, наоборот, считается, что тождественные частицы принципиально неразличимь.

В статистике Ферми-Дирака в каждом квантовом состоянии может находиться не более одной частицы (принцип Паули), а в статистике Бозе-Эйнштейна - любое число частиц.

\footnotetext{\begin{itemize}
  \item Физическая природа различия этих двух квантовых статистик вытекает из принципа неразличимости тождественных частиц, согласно которому существуют два типа волновых \(\Psi\)-функций, описывающих состояние тождественных частиц, - симметричные и антисимметричные.
\end{itemize}
}Различие статистик поясняет табл. 4.1, где показано как в каждой из них размещаются две тождественные частицы \(a\) и \(b\) по трем квантовым состояниям (клеткам).

\begin{table}[h]
\begin{center}
\captionsetup{labelformat=empty}
\caption{Таблица 4.1}
\begin{tabular}{|l|l|l|l|l|l|l|l|l|l|l|l|l|l|}
\hline
\multicolumn{6}{|c|}{Статистика Больцмана} & \multicolumn{4}{|c|}{Статистика Бозе-Эйнштейна} & \multicolumn{4}{|c|}{Статистика Ферми-Дирака} \\
\hline
 & \includegraphics[max width=\textwidth, alt={}]{bd023a69-cc46-4b6f-a48e-cc548fa9259d-107_64_190_484_110}
 &  & \multicolumn{3}{|c|}{\(a\)} & \multicolumn{4}{|c|}{\includegraphics[max width=\textwidth, alt={}]{bd023a69-cc46-4b6f-a48e-cc548fa9259d-107_68_188_480_616}
} & \multicolumn{4}{|c|}{\includegraphics[max width=\textwidth, alt={}]{bd023a69-cc46-4b6f-a48e-cc548fa9259d-107_64_182_484_911}
} \\
\hline
\includegraphics[max width=\textwidth, alt={}]{bd023a69-cc46-4b6f-a48e-cc548fa9259d-107_25_32_586_185}
 & \(b\) &  & \includegraphics[max width=\textwidth, alt={}]{bd023a69-cc46-4b6f-a48e-cc548fa9259d-107_68_190_559_342}
 &  & \includegraphics[max width=\textwidth, alt={}]{bd023a69-cc46-4b6f-a48e-cc548fa9259d-107_64_186_561_346}
 & \multicolumn{4}{|c|}{\includegraphics[max width=\textwidth, alt={}]{bd023a69-cc46-4b6f-a48e-cc548fa9259d-107_66_186_561_618}
} & \multicolumn{4}{|c|}{\includegraphics[max width=\textwidth, alt={}]{bd023a69-cc46-4b6f-a48e-cc548fa9259d-107_66_186_561_911}
} \\
\hline
a & \begin{tabular}{l}
□ \\
\(b\) \\
\end{tabular} &  & \begin{tabular}{l}
□ \\
\(b a\) \\
\end{tabular} &  &  & □ & \includegraphics[max width=\textwidth, alt={}]{bd023a69-cc46-4b6f-a48e-cc548fa9259d-107_62_48_637_618}
 &  &  &  &  &  & \includegraphics[max width=\textwidth, alt={}]{bd023a69-cc46-4b6f-a48e-cc548fa9259d-107_70_190_635_911}
 \\
\hline
\multicolumn{10}{|c|}{\includegraphics[max width=\textwidth, alt={}]{bd023a69-cc46-4b6f-a48e-cc548fa9259d-107_68_572_712_228}
} & \multicolumn{4}{|c|}{\multirow{2}{*}{}} \\
\hline
 &  &  & \includegraphics[max width=\textwidth, alt={}]{bd023a69-cc46-4b6f-a48e-cc548fa9259d-107_68_186_787_228}
 & \multicolumn{6}{|c|}{\includegraphics[max width=\textwidth, alt={}]{bd023a69-cc46-4b6f-a48e-cc548fa9259d-107_70_186_787_618}
} &  &  &  &  \\
\hline
 &  &  & □ & \multicolumn{6}{|c|}{\includegraphics[max width=\textwidth, alt={}]{bd023a69-cc46-4b6f-a48e-cc548fa9259d-107_70_192_862_616}
} &  &  &  &  \\
\hline
\end{tabular}
\end{center}
\end{table}

Видно, что в статистике Больцмана всех микросостояний девять и вероятность каждого из них равна \(1 / 9\).

В статистиках же Бозе-Эйнштейна и Ферми-Дирака состояния в первых трех парах распределения Больцмана неразличимы, и каждая пара рассматривается как одно состояние. Частицы \(a\) и \(b\) принципиально неразличимы, поэтому они обозначены просто точками. Для бозонов число микросостояний равно шести, и вероятность каждого из них \(1 / 6\). Для фермионов последние три распределения статистики Бозе-Эйнштейна невозможны (приниип Паули). Остается только три микросостояния, и вероятность каждого из них равна \(1 / 3\).

Фазовые ячейки. Основная задача квантовых статистик это нахождение соответствующих им функций распределения частиц по тем или иным параметрам (например, по энергиям), а также определение средних значений этих параметров, характеризующих наиболее вероятное макросостояние всей системы частиц.

Для описания состояния системы частиц рассматривают воображаемое шестимерное пространство, каждая точка которого

характеризуется шестью координатами: \(x, y, z, p_{x}, p_{y}, p_{z}\). Это так называемое фазовое пространство. Состояние системы определяется тем, как распределены в этом пространстве точки, изображающие состояния всех \(N\) частиц системы. При этом нужно учесть присущий частицам корпускулярно-волновой дуализм, согласно которому неопределенности координаты \(x\) и соответствующей проекции импульса \(p_{x}\) могут быть определены только с неопределенностью \(\delta x\) и \(\delta p_{x}\), произведение которых, согласно приниипу неопределенностей Гейзенберга, \(\delta x \cdot \delta p_{x} \geqslant h\) (здесь \(h=2 \pi \hbar\) ). Аналогично и для других пар: \(y\) и \(p_{y}\), \(z\) и \(p_{z}\).

Поэтому естественно считать, что данному состоянию частицы в фазовом пространстве соответствует не точка, а фазовая ячейка, объем которой


\begin{equation*}
\delta \Lambda=\delta x \cdot \delta y \cdot \delta z \cdot \delta p_{x} \cdot \delta p_{y} \cdot \delta p_{z} \approx h^{3} . \tag{4.1}
\end{equation*}


Распределение частиц по таким фазовым ячейкам есть предельно подробное квантовое описание состояния системы. Нас будет интересовать наиболее вероятное распределение частиц по фазовым ячейкам. Решение этой задачи достаточно сложно, и его нет смысла здесь приводить. Мы приведем лишь окончательные результаты - распределения частиц по энергиям \(\varepsilon_{i}\).

Квантовые распределения. Эти распределения представляют собой функции \(f\left(\varepsilon_{i}\right)\), определяющие средние числа частиц в одной фазовой ячейке с энергией \(\varepsilon_{i}\), или функции заполнения ячеек:

\[
\begin{array}{c|c|}
\text { для фермионов } & f\left(\varepsilon_{i}\right)=\frac{1}{\mathrm{e}^{\left(\varepsilon_{i}-\mu\right) / k T}+1} . \\
\text { для бозонов } & f\left(\varepsilon_{i}\right)=\frac{1}{\mathrm{e}^{\left(\varepsilon_{i}-\mu\right) / k T}-1} . \tag{4.3}
\end{array}
\]

Здесь \(\mu\) - так называемый химический потенииал (некоторая характерная энергия, значение которой можно найти из условия нормировки: суммарное число частиц во всех фазовых ячейках должно быть равно полному числу \(N\) частиц макросистемы).

Остановимся подробнее на особенностях этих распределений.


Остановимся подробнее на особенностях этих распределений.

\begin{enumerate}

  \item Для фермионов функция f(i) не может быть больше единицы, а для бозонов ее значение может быть любым (f j 0).

  \item Если f I 1, то в знаменателях обоих распределений можно пренебречь единицей, и формула переходят в f A i kT kT i i ( ) ( ) / / <     e e , (4.4) т. е. в распределение Больцмана (А — нормировочный коэффициент). Значит

\end{enumerate}


\begin{enumerate}
  \item Для фермионов функция \(f\left(\varepsilon_{i}\right)\) не может быть больше единицы, а для бозонов ее значение может быть любым ( \(f \geqslant 0\) ).
  \item Если \(f \ll 1\), то в знаменателях обоих распределений можно пренебречь единицей, и формула переходят в
\end{enumerate}


\begin{equation*}
f\left(\varepsilon_{i}\right)=\mathrm{e}^{\left(\mu-\varepsilon_{i}\right) / k T}=A \mathrm{e}^{-\varepsilon_{i} / k T}, \tag{4.4}
\end{equation*}


т. е. в распределение Больцмана ( \(A\) - нормировочный коэффициент). Значит, классическое распределение Больцмана справедливо лишь тогда, когда малы «числа заполнения» фазовых ячеек,- при условии \(\left\langle N_{i}\right\rangle \ll 1\). Особо отметим, что в этом случае речь идет о совпадении формул, а отнюдь не о том, что изменяется поведение частиц (фермионы остаются фермионами, бозоны - бозонами).\\
3. В макросистеме уровни энергии \(\varepsilon_{i}\) частиц квазинепрерывны (расположены очень плотно). Поэтому индекс \(i\) у \(\varepsilon_{i}\) можно опустить.\\
4. Для бозонов значения \(\mu\) в (4.3) не могут быть положительными, иначе при \(\varepsilon_{i}<\mu\) окажется, что \(f<0\), а это лишено физического смысла. Таким образом, для бозонов \(\mu \leqslant 0\). У макросистем с переменным числом бозонов (к числу которых относятся, например, фотоны) \(\mu=0\), и формула (4.3) переходит в


\begin{equation*}
f=\frac{1}{\mathrm{e}^{\varepsilon_{i} / k T}-1} . \tag{4.5}
\end{equation*}


Для фермионов подобного ограничения не существует.\\
Число фазовых ячеек. До сих пор мы имели дело с функцией \(f(\varepsilon)\), характеризующей, напомним, среднее число частиц с энергией ε в одной фазовой ячейке. Для дальнейших целей необходимо найти число \(\mathrm{d} Z\) фазовых ячеек в интервале энергий ( \(ε, ~ ε+\mathrm{d} ε\) ).

Чтобы определить \(\mathrm{d} Z\), найдем сначала соответствующий объем \(\mathrm{d} \Lambda\) фазового шестимерного пространства. Для этого в импульсной части фазового пространства выделим шаровой слой радиусом, равным импульсу \(p\) частицы, и толщиной \(\mathrm{d} p\). Его объем равен \(4 \pi p^{2} \mathrm{~d} p\). Умножив его на объем \(V\) координатной

части фазового пространства (это объем макросистемы), получим искомый элемент объема \(\mathrm{d} \Lambda\) фазового пространства:


\begin{equation*}
\mathrm{d} \Lambda=4 \pi p^{2} \mathrm{~d} p \cdot V \tag{4.6}
\end{equation*}


Число \(\mathrm{d} Z\) фазовых ячеек в этом элементе объема получим, разделив \(\mathrm{d} \Lambda\) на объем одной фазовой ячейки, равный \(h^{3}\) согласно (4.1). Кроме того, в дальнейшем нас будет интересовать число фазовых ячеек, приходящихся на единицу объема обычного пространства, поэтому будем считать, что \(V=1\). Таким образом, число фазовых ячеек в расчете на единицу объема, занимаемого газом, будет равно


\begin{equation*}
\mathrm{d} Z=\frac{4 \pi}{h^{3}} p^{2} \mathrm{~d} p \tag{4.7}
\end{equation*}


Эта величина имеет размерность \(\mathrm{cm}^{-3}\).\\
Переход от импульсов к энергиям зависит от природы частиц. Это будет конкретизировано в дальнейшем.

Распределение частиц. Зная число \(\mathrm{d} Z\) фазовых ячеек в интервале энергий ( \(\varepsilon, \varepsilon+d \varepsilon\) ) и среднее число частиц в каждой ячейке, т. е. функцию заполнения \(f\), мы можем найти число частиц \(\mathrm{d} n\) в данном интервале энергий (в расчете на единицу объема газа):


\begin{equation*}
\mathrm{d} n=\gamma f \mathrm{~d} Z, \tag{4.8}
\end{equation*}


где \(\gamma\) - числовой коэффициент порядка единицы; как мы увидим далее, он связан со спецификой частиц идеального газа.

\section*{§ 4.2. Распределение Ферми-Дирака для электронов в металле}
Свободные электроны в металле. Электропроводность металлов обусловлена, как известно, наличием в них электронов, которые мы называем свободньлми. Они не связаны с конкретными атомами и могут практически свободно перемещаться в пределах образца. В первом приближении свободные электроны можно рассматривать как идеальный газ из фермионов в прямоугольной потенциальной яме.

Прежде всего рассмотрим поведение электронного газа при температуре \(T=0\). В этом случае функция (4.2) принимает следующие значения:


\begin{equation*}
f(\varepsilon \leqslant \mu)=1, \quad f(\varepsilon>\mu)=0 . \tag{4.9}
\end{equation*}


Соответствующий график показан на рис. 4.1, из которого видно, что заполнены все состояния с энергией \(\varepsilon<\mu\), а состояния с \(\varepsilon>\mu\) оказываются незанятыми.

Состояния квантованы, и энергетические уровни являются дискрет-

\begin{figure}[h]
\begin{center}
  \includegraphics[alt={},max width=\textwidth]{bd023a69-cc46-4b6f-a48e-cc548fa9259d-111_273_430_371_718}
\captionsetup{labelformat=empty}
\caption{Рис. 4.1}
\end{center}
\end{figure}

В статистике Ферми-Дирака, которой подчиняется электронный газ, принимается во внимание принцип Паули (согласно ему в каждом квантовом состоянии может находиться не более одного электрона). Отсюда следует, что даже между свободными электронами существует какое-то взаимодействие. Однако это взаимодействие не является силовым, а представляет собой сугубо квантовый эффект, чуждый классическим представлениям.

Энергия Ферми. В рассматриваемом случае ( \(T=0\) ) величину \(\mu\) называют энергией или уровнем Ферми: \(\varepsilon_{F}=\mu\). Эта энергия является максимальной, которую могут иметь свободные электроны в металле при \(T=0\). Найдем \(\varepsilon_{F}\).

Для этого сначала определим число квантовых состояний (фазовых ячеек) \(\mathrm{d} Z\) для свободных электронов с энергиями в интервале ( \(\varepsilon, \varepsilon+\mathrm{d} \varepsilon\) ). Связь между энергией и импульсом электрона это \(\varepsilon=p^{2} / 2 m\). Отсюда \(p=\sqrt{2 m ε}\) и \(\mathrm{d} p=\sqrt{m / 2 \varepsilon} \mathrm{~d} \varepsilon\). Поэтому \(\mathrm{d} Z\) согласно (4.7), т. е. в расчете на единицу объема электронного газа, примет вид


\begin{equation*}
\mathrm{d} Z=\alpha \sqrt{\varepsilon} \mathrm{d} \varepsilon, \tag{4.10}
\end{equation*}


где введено обозначение


\begin{equation*}
\alpha=2 \pi(2 m)^{3 / 2} / h^{3} . \tag{4.11}
\end{equation*}


\begin{figure}[h]
\begin{center}
  \includegraphics[alt={},max width=\textwidth]{bd023a69-cc46-4b6f-a48e-cc548fa9259d-112_299_515_145_68}
\captionsetup{labelformat=empty}
\caption{Рис. 4.2}
\end{center}
\end{figure}

Зависимость числа квантовых состояний на единицу энергии, \(\mathrm{d} Z / \mathrm{d} \varepsilon\), от энергии \(\varepsilon\) представлена графически на рис. 4.2.

Чтобы определить число свободных электронов \(\mathrm{d} n\) в интервале энергий ( \(\varepsilon, \varepsilon+d \varepsilon\) ), надо умножить соответствующее число \(\mathrm{d} Z\) фазовых ячеек, т. е. (4.10), на среднее число этих электронов в одной ячейке - на функцию заполнения \(f\) :


\begin{equation*}
\mathrm{d} n=2 f \mathrm{~d} Z . \tag{4.12}
\end{equation*}


Коэффициент 2 появился в связи с тем, что в каждой фазовой ячейке могут расположиться два электрона (фермиона) с противоположно направленными спинами. В нашем случае ( \(T=0\) ) свободные электроны заполняют полностью ( \(f=1\) ) все квантовые состояния с энергиями \(\varepsilon \leqslant \mu\), и мы имеем


\begin{equation*}
\mathrm{d} n=2 \alpha \sqrt{\varepsilon} \mathrm{~d} \varepsilon . \tag{4.13}
\end{equation*}


\begin{figure}[h]
\begin{center}
  \includegraphics[alt={},max width=\textwidth]{bd023a69-cc46-4b6f-a48e-cc548fa9259d-112_302_497_1020_70}
\captionsetup{labelformat=empty}
\caption{Рис. 4.3}
\end{center}
\end{figure}

График функции распределения свободных электронов по энергиям, т.е. \(\mathrm{d} n / \mathrm{d} \varepsilon\) в зависимости от энергии ɛ, показан на рис. 4.3. Площадь под этим графиком равна концентрации \(n\) свободных электронов:


\begin{equation*}
n=\int_{0}^{\mu} \mathrm{d} n(\varepsilon)=\frac{4}{3} \alpha \mu^{3 / 2} . \tag{4.14}
\end{equation*}


Отсюда находим \(\mu\) - максимальное значение энергии свободных электронов при \(T=0\). Эта величина и есть энергия или уровень Ферми:


\begin{equation*}
\varepsilon_{F}=\frac{h^{2}}{8 m}\left(\frac{3 n}{\pi}\right)^{2 / 3}, \tag{4.15}
\end{equation*}


где, напомним, \(h=2 \pi \hbar\). При получении этого выражения учтено, что постоянная \(\alpha\) определяется формулой (4.11).

Оценим значение \(\varepsilon_{F}\). Концентрация \(n\) свободных электронов в металлах находится в пределах от \(10^{22}\) до \(10^{23} \mathrm{~cm}^{-3}\). Для среднего значения \(n=5 \cdot 10^{22}\) получим

\[
\varepsilon_{F}=0,8 \cdot 10^{-11} \text { эрг }=5 \text { эВ. }
\]

Пример. Определим распределение свободных электронов в металле по скоростям и максимальную скорость их при \(T=0\).

Ясно, что число \(\mathrm{d} n_{\varepsilon}\) свободных электронов в интервале энергий ( \(\varepsilon, \varepsilon+\mathrm{d} \varepsilon\) ) должно равняться их числу \(\mathrm{d} n_{v}\) в соответствующем интервале скоростей ( \(v, v+\mathrm{d} v\) ), т. е. \(\mathrm{d} n_{v}=\mathrm{d} n_{\varepsilon}\). Имея в виду связь между \(\varepsilon\) и \(v\), а именно \(\varepsilon=m v^{2} / 2\), и формулу (4.13) для \(\mathrm{d} n_{\varepsilon}\), получим

\[
\mathrm{d} n_{v}=2 \alpha \sqrt{m v^{2} / 2} m v \mathrm{~d} v .
\]

Остается воспользоваться выражением (4.11) для \(\alpha\), и мы найдем:

\[
\mathrm{d} n_{v}=8 \pi(m / h)^{3} v^{2} \mathrm{~d} v,
\]

т. е. плотность распределения \(\mathrm{d} n_{v} / \mathrm{d} v \sim v^{2}\).

Поскольку металл не конкретизирован, ограничимся оценкой максимальной скорости, полагая энергию Ферми равной значению, которое приведено выше ( 5 эВ). Тогда

\[
\begin{aligned}
v_{\text {макс }} & =\sqrt{2 \varepsilon_{F} / m}=\sqrt{2 \cdot 5 \cdot 1,6 \cdot 10^{-12} /\left(0,9 \cdot 10^{-27}\right)}= \\
& =1,34 \cdot 10^{8} \mathrm{~cm} / \mathrm{c}=1,34 \cdot 10^{3} \mathrm{KM} / \mathrm{c} .
\end{aligned}
\]

Средняя энергия свободных электронов. При \(T=0\) имеем следующее выражение:


\begin{equation*}
\langle\varepsilon\rangle=\frac{1}{n} \int_{0}^{\varepsilon_{F}} \varepsilon \mathrm{~d} n_{\varepsilon}=\frac{1}{n} \int_{0}^{\varepsilon_{F}} 2 \alpha \varepsilon^{3 / 2} \mathrm{~d} \varepsilon=\frac{3}{5} \varepsilon_{F}, \tag{4.16}
\end{equation*}


где использованы формулы (4.13) и (4.11). При значении \(\varepsilon_{F}=5\) эВ \(\quad\langle\varepsilon\rangle=3\) эВ.

Классическому газу с такой средней энергией соответствовала бы, согласно формуле \(\langle\varepsilon\rangle=\frac{3}{2} k T\), температура \(T \sim 5 \cdot 10^{4}\) К. Эта

температура во много раз превосходит температуру плавления любого металла.

Идеальный газ, распределение частиц по энергиям которого сильно отличается от классического, называют вырожденным, что мы и имеем в рассмотренном случае.

\begin{figure}[h]
\begin{center}
  \includegraphics[alt={},max width=\textwidth]{bd023a69-cc46-4b6f-a48e-cc548fa9259d-114_307_467_365_68}
\captionsetup{labelformat=empty}
\caption{Рис. 4.4}
\end{center}
\end{figure}

Энергия Ферми при \(\boldsymbol{T}>\mathbf{0}\). Распределение Ферми-Дирака несколько размывается в окрестности уровня Ферми (рис. 4.4). Это происходит вследствие взаимодействия свободных электронов с тепловым движением атомов. Так как средняя энергия теплового движения атомов имеет порядок \(k T\), то область размывания функции имеет тот же порядок \(k T\). Аналогично деформируется и функция распределения свободных электронов по энергиям, т. е. \(\mathrm{d} n / \mathrm{d} \varepsilon\) (рис. 4.5).

Таким образом, при нагрева-

\begin{figure}[h]
\begin{center}
  \includegraphics[alt={},max width=\textwidth]{bd023a69-cc46-4b6f-a48e-cc548fa9259d-114_348_509_915_68}
\captionsetup{labelformat=empty}
\caption{Рис. 4.5}
\end{center}
\end{figure}

Именно в этом ключ к разгадке «странного» поведения электронного газа, вклад которого в теплоемкость практически не заметен, и теплоемкость кристалла практически зависит только от колебаний атомов решетки (к этому вопросу мы еще вернемся в конце § 4.5).

Имея в виду, что в модели свободных электронов последние находятся в прямоугольной потенциальной яме, распределение электронов изображают для наглядности так, как показано на рис. 4.6. Здесь \(U\) - глубина ямы, \(A\) - работа выхода (наимень-

шая энергия, которую надо сообщить электрону для удаления его из металла, \(A=U-\varepsilon_{F}\) ). Спектр энергетических уровней дискретный (практически квазинепрерывный). Тонирована часть спектра, заполненного свободными электронами.

\begin{figure}[h]
\begin{center}
  \includegraphics[alt={},max width=\textwidth]{bd023a69-cc46-4b6f-a48e-cc548fa9259d-115_281_437_141_715}
\captionsetup{labelformat=empty}
\caption{Рис. 4.6}
\end{center}
\end{figure}

Энергия Ферми, как показывает расчет, несколько зависит от температуры:


\begin{equation*}
\varepsilon_{F}(T)=\varepsilon_{F}(0)\left[1-\frac{\pi^{2}}{12}\left(\frac{k T}{\varepsilon_{F}(0)}\right)^{2}\right], \tag{4.17}
\end{equation*}


где \(\varepsilon_{F}(0)\) - уровень Ферми при \(T=0\), определяемый формулой (4.15). И более общее определение уровня Ферми гласит: это энергия, при которой распределение Ферми-Дирака (4.2) принимает значение \(f=1 / 2\).

\section*{§ 4.3. О зонной теории. Электропроводность}
Энергетические зоны в кристаллах. Свободные электроны в металле - это модель поведения электронов в первом приближении. Уточнение этой модели сводится к учету того, что электроны на самом деле движутся в периодическом поле, и это приводит к появлению нового явления. А именно, квазинепрерывный энергетический спектр электронов распадается на ряд разрешенных и запрещенных зон.

Каждая из разрешенных зон состоит из \(N\) близко расположенных уровней, число которых равно количеству атомов в кристалле. Ширина зон ( \(3 \div 5\) эВ) не зависит от размеров кристалла. Происхождение этих зон связано с энергетическими уровнями изолированных атомов. При образовании кристалла каждый уровень атома распадается на \(N\) уровней. Особенно сильному расщеплению подвергаются вышележащие энергетические уровни атома, и в частности, уровень с внешним валентным электроном. В результате возникает особенно интересующая нас валентная зона, уровни которой заполнены наполови-

ну, если в соответствующем атоме был один валентный электрон. Наполовину, потому что на каждом уровне могут располагаться по два электрона с противоположно направленными спинами. Если же валентных электронов у атома два, то валентная зона будет заполнена полностью. Впрочем, возможны случаи, когда две зоны, образованные из двух соседних уровней атома, перекрываются, и образуется зона, содержащая \(2 N\) уровней.

Зону, расположенную над валентной, называют свободной.\\
В зависимости от степени заполнения валентной зоны электронами и ширины \(\Delta E\) запрещенной зоны кристаллы подразделяют на металль, полупроводники и диэлектрики (рис. 4.7). На рисунке тонирована та часть энергетического спектра, которая полностью заполнена электронами (по два на уровень). Это распределение соответствует температуре \(T=0\). При этой температуре все электроны совершают внутренние квантовые движения, никак при этом себя не проявляя.

\begin{figure}[h]
\begin{center}
  \includegraphics[alt={},max width=\textwidth]{bd023a69-cc46-4b6f-a48e-cc548fa9259d-116_479_990_903_118}
\captionsetup{labelformat=empty}
\caption{Рис. 4.7}
\end{center}
\end{figure}

В металлах, как мы уже знаем, достаточно сообщить электронам, находящимся на верхних уровнях, небольшую энергию (повышая температуру или прикладывая внешнее электрическое поле \(\mathbf{E}\) ), чтобы перевести их на более высокие уровни, где они проявляют себя или в теплоемкости или в электропроводности.

У полупроводников дело обстоит иначе, и электроны могут себя так или иначе проявить лишь в случае, если им будет со-

общена энергия, превышающая энергию \(\Delta E\) запрещенной зоны, чтобы перевести их с верхних уровней заполненной валентной зоны в свободную зону (у полупроводников \(\Delta E\) порядка нескольких десятых эВ). Свободная зона станет для таких электронов зоной проводимости. Одновременно могут себя проявлять и электроны на верхних уровнях валентной зоны, поскольку эти уровни частично освобождаются.

\section*{Электропроводность металлов и полупроводников}
Напомним, электропроводность \(\sigma\) - это величина, связывающая плотность ј электрического тока и напряженность \(\mathbf{E}\) в локальном законе Ома \(\mathbf{j}=\sigma \mathbf{E}\).

Для металлов квантовая теория приводит к следующему выражению электропроводности:


\begin{equation*}
\sigma=\frac{n e^{2} \tau}{m}, \tag{4.18}
\end{equation*}


где \(n\) - концентрация свободных электронов, \(m\) - масса электрона, \(\tau\) - некоторое характерное время, которое иногда называют временем релаксации. Это время характеризует в частности процесс установления равновесия между электронами и решеткой, нарушенного, например, внезапным включением внешнего поля Е.

Может показаться, что \(\sigma\) в (4.18) зависит от поведения всех свободных электронов. Это не так. Участвуют только те из них, которые имеют энергию вблизи уровня Ферми. А это малая часть всех свободных электронов, и их доля учитывается величиной \(\tau\).

Анализ выражения (4.18) показывает, что σ ం \(1 / T\). Это расходится с результатами классической теории, где \(\sigma\) ა \(1 / \sqrt{T}\). Эксперимент подтверждает вывод квантовой теории.

Физической причиной электрического сопротивления является рассеяние электронных волн на примесях и дефектах решетки, а также на ее тепловых колебаниях.

У полупроводников электропроводность возникает при \(T \neq 0\) в результате перехода электронов с верхних уровней валентной зоны в свободную зону. При этом в валентной зоне освобождается такое же число состояний на верхних уровнях - их назы-

вают «дырками». Когда говорят о перемещении дырок, то имеют в виду не реальную частицу, а тот факт, что ее движение отражает характер движения всей совокупности электронов в верхних уровнях валентной зоны. Это удобное понятие, и им

\begin{figure}[h]
\begin{center}
  \includegraphics[alt={},max width=\textwidth]{bd023a69-cc46-4b6f-a48e-cc548fa9259d-118_507_558_326_68}
\captionsetup{labelformat=empty}
\caption{Рис. 4.8}
\end{center}
\end{figure}

широко пользуются.

Распределение электронов по уровням в свободной и валентной зонах описывается функцией Ферми-Дирака (4.2). Графически это показано на рис. 4.8. Соответствующий расчет дает, что уровень Ферми при этом расположен посредине запрещенной зоны, как это и показано на рисунке.

Уровни свободной зоны (с электронами) находятся на «хвосте» распределения \(f(\varepsilon)\). Это означает, что \(\varepsilon-\varepsilon_{F} \approx \Delta E / 2\). С учетом последнего соотношения и того, что \(\Delta E \gg k T\), вероятность \(f(\varepsilon)\) заполнения уровней в свободной зоне, т. е. формулу (4.4), где \(\mu=\varepsilon_{F}\), можно записать как


\begin{equation*}
f(\varepsilon) \sim \mathrm{e}^{-\Delta E / 2 k T} . \tag{4.19}
\end{equation*}


Число электронов, перешедших в свободную зону, а значит и число образовавшихся дырок, будет пропорционально \(f(\varepsilon)\). Эти электроны и дырки и являются носителями тока.

Электропроводность \(\sigma\) пропорциональна числу носителей, значит и \(\sigma\) c \(f(\varepsilon)\), т.е.


\begin{equation*}
\sigma \circlearrowleft \mathrm{e}^{-\Delta E / 2 k T} . \tag{4.20}
\end{equation*}


Отсюда следует, что электропроводность полупроводников быстро растет с увеличением температуры.

Типичными полупроводниками являются германий и кремний.

Проводимость, которую мы рассмотрели, называют coбственной, она свойственна чистым полупроводникам. Кроме нее существует и широко используется примесная проводи-

мость. Она возникает, когда в чистом полупроводнике некоторые атомы замещают другими (примесью). В результате в запрещенной зоне возникают добавочные уровни, расположенные близко или к свободной зоне (донорные) или - к потолку валентной зоны (акцепторы). Это способствует существенному увеличению электропроводности полупроводника. Но в детали этого вопроса мы углубляться не будем, поскольку радикально новых идей здесь нет.

\section*{§ 4.4. Распределение Бозе-Эйнштейна для фотонного газа}
Равновесное тепловое излучение в замкнутой полости представляет собой совокупность стоячих электромагнитных волн с дискретными частотами. Попытки теоретически объяснить наблюдаемое распределение спектральной плотности излучения по частотам с классической точки зрения оказались несостоятельными и породили так называемую «проблему теплового излучения». В 1900 г. она была решена Планком путем введения в процесс взаимодействия излучения с веществом идеи квантования.

Эйнштейн сделал следующий шаг. Он предположил, и это подтвердилось экспериментом, что само излучение представляет собой фотонный газ, газ идеальный. У фотонов спин равен единице. Значит это бозоны, а они подчиняются статистике Бозе-Эйнштейна.

Число фотонов в полости не сохраняется, оно зависит от температуры. А для систем с переменным числом бозонов, как было сказано в § 4.1, химический потенциал \(\mu=0\), и функция (4.3) принимает вид (4.5), т. е.


\begin{equation*}
f=\frac{1}{\mathrm{e}^{\varepsilon / k T}-1} . \tag{4.21}
\end{equation*}


Для фотонов \(\varepsilon=h v\) и \(p=h v / c\), поэтому число квантовых состояний (фазовых ячеек) в интервале частот ( \(v, v+\mathrm{d} v\) ) в расчете на единицу объема фотонного газа равно согласно (4.7)


\begin{equation*}
\mathrm{d} Z=\frac{4 \pi}{c^{3}} v^{2} \mathrm{~d} v . \tag{4.22}
\end{equation*}


Графики функций \(f\) и \(\mathrm{d} Z / \mathrm{d} v\) для фотонного газа представлены на рис. 4.9 и рис. 4.10. Следует обратить внимание на то, что обе функции ведут себя с ростом частоты \(v\) взаимно противоположно: \(f\) убывает, а \(\mathrm{d} Z / \mathrm{d} \nu\) растет.

\begin{figure}[h]
\begin{center}
  \includegraphics[alt={},max width=\textwidth]{bd023a69-cc46-4b6f-a48e-cc548fa9259d-120_359_416_338_135}
\captionsetup{labelformat=empty}
\caption{Рис. 4.9}
\end{center}
\end{figure}

\begin{figure}[h]
\begin{center}
  \includegraphics[alt={},max width=\textwidth]{bd023a69-cc46-4b6f-a48e-cc548fa9259d-120_359_363_338_702}
\captionsetup{labelformat=empty}
\caption{Рис. 4.10}
\end{center}
\end{figure}

В соответствии с формулой (4.8) число фотонов с частотами в интервале ( \(v, v+\mathrm{d} v\) ) равно


\begin{equation*}
\mathrm{d} n=2 f \mathrm{~d} Z=\frac{8 \pi v^{2}}{c^{3}} \frac{\mathrm{~d} v}{\mathrm{e}^{h v / k T}-1} . \tag{4.23}
\end{equation*}


Коэффициент 2 появился в связи с двумя независимыми поляризациями излучения во взаимно перпендикулярных плоскостях. Другими словами, он указывает на две возможные поперечные поляризации фотона. Напомним, что в случае электронов этот коэффициент учитывал две возможные «ориентации» спина электрона.

График распределения фотонов по частотам, т. е. \(\mathrm{d} n / \mathrm{d} v\), показан на рис. 4.11. Площадь под кривой равна полному числу \(n\) фотонов в расчете на единицу объема фотонного газа.

\begin{figure}[h]
\begin{center}
  \includegraphics[alt={},max width=\textwidth]{bd023a69-cc46-4b6f-a48e-cc548fa9259d-120_341_684_1449_272}
\captionsetup{labelformat=empty}
\caption{Рис. 4.11}
\end{center}
\end{figure}

Теперь перейдем к спектральной плотности энергии излучения (фотонного газа): \(u_{v}=\mathrm{d} u / \mathrm{d} v\), где \(\mathrm{d} u=h v \cdot \mathrm{~d} n\). В результате получим:


\begin{equation*}
u_{v}=\frac{8 \pi h v^{3}}{c^{3}} \frac{1}{\mathrm{e}^{h v / k T}-1} . \tag{4.24}
\end{equation*}


Это знаменитая формула План\(к а\). Её открытие и интерпретация положили начало созданию квантовой физики. График этой зависимости от частоты \(v\) показан на рис. 4.12.

При переходе от \(v\) и \(h\) к циклической частоте \(\omega=2 \pi \nu\) и \(\hbar=h / 2 \pi\) надо учесть, что \(u_{v} \mathrm{~d} v=u_{\omega} \mathrm{d} \omega\). Тогда формула Планка приобретает вид:

\begin{figure}[h]
\begin{center}
  \includegraphics[alt={},max width=\textwidth]{bd023a69-cc46-4b6f-a48e-cc548fa9259d-121_388_511_425_653}
\captionsetup{labelformat=empty}
\caption{Рис. 4.12}
\end{center}
\end{figure}


\begin{equation*}
u_{\omega}=\frac{\hbar \omega^{3}}{\pi^{2} c^{3}} \frac{1}{\mathrm{e}^{\hbar \omega / k T}-1} . \tag{4.25}
\end{equation*}


Вернемся к формуле (4.24), графики которой при разных температурах представлены на рис. 4.12 , где \(T_{1}<T_{2}<T_{3}\). Площадь под каждой из этих кривых равна полной плотности энергии \(u\) при соответствующей температуре. Выясним, как эта величина \((u)\) зависит от \(T\). Для этого ради упрощения преобразований представим (4.24) в виде


\begin{equation*}
u_{v}=v^{3} F(v / T), \tag{4.26}
\end{equation*}


где \(F\) - функция, вид которой до открытия Планка был неизвестен. Кстати, в таком виде формула (4.26) была получена Вином и получила название формуль Вина.

Воспользовавшись (4.26), запишем:


\begin{equation*}
u=\int_{0}^{\infty} u_{v} \mathrm{~d} v=\int_{0}^{\infty} v^{3} F\left(\frac{v}{T}\right) \mathrm{d} v=T^{4} \int_{0}^{\infty} x^{3} F(x) \mathrm{d} x, \tag{4.27}
\end{equation*}


где введена новая переменная \(x=v / T\). Последний интеграл представляет собой некоторую постоянную \(a\), и мы приходим

к выводу, что


\begin{equation*}
u=a T^{4} . \tag{4.28}
\end{equation*}


Закон Стефана-Больцмана. Вместо плотности энергии излучения \(u\) удобнее пользоваться понятием энергетической светимости \(M\), которая выражает поток энергии излучения с единицы поверхности по всем направлениям в пределах телесного угла \(2 \pi\). Можно показать, что обе эти величины связаны соотношением


\begin{equation*}
M=\frac{c}{4} u . \tag{4.29}
\end{equation*}


Остается учесть (4.28), и мы получим


\begin{equation*}
M=\sigma T^{4} . \tag{4.30}
\end{equation*}


Эта формула и выражает закон Стефана-Больцмана. Здесь \(\sigma\) - постоянная Стефана-Больцмана. С помощью формулы Планка можно найти ее зависимость от постоянных \(c, h, k\) и ее числовое значение:


\begin{equation*}
\sigma=5,67 \cdot 10^{-8} \mathrm{BT} /\left(\mathrm{M}^{2} \cdot \mathrm{~K}^{4}\right) . \tag{4.31}
\end{equation*}


Если в стенках полости с равновесным тепловым излучением (фотонным газом) сделать небольшое отверстие, то можно экспериментально исследовать спектральный состав выходящего через это отверстие излучения. Это было проделано для разных температур полости. Результаты оказались в прекрасном соответствии с формулой Планка и законом Стефана-Больцмана.

В данном случае отверстие в стенке полости играет роль так называемого черного тела. Его особенностью является способность полностью поглощать падающее на него извне излучение. Термин черное не следует понимать буквально: при повышении температуры отверстие начинает «краснеть», а затем может стать и белым. После этого не должно вызывать особого удивления, если в качестве примера черного тела будет названо... Солнце. Дело в том, что спектральный состав его излучения очень близок к описываемому формулой Планка (и Солнце при этом полностью поглощает падающее на него излучение).

Закон смещения Вина. При теоретических исследованиях спектральный состав излучения удобнее характеризовать по частотам, в экспериментальных же - по длинам волн. Имея в виду соотношение \(u_{v} \mathrm{~d} v=-u_{\lambda} \mathrm{d} \lambda\), формулу (4.26) и то, что \(v=c / \lambda\), запишем:


\begin{equation*}
u_{\lambda}=-u_{v} \frac{\mathrm{~d} \nu}{\mathrm{~d} \lambda}=\frac{1}{\lambda^{3}} \mathscr{T}(\lambda T) \frac{1}{\lambda^{2}}=\frac{1}{\lambda^{5}} \mathscr{T}(\lambda T) . \tag{4.32}
\end{equation*}


Наличие знака минус в исходной формуле связано с тем, что с ростом частоты ( \(\mathrm{d} \nu>0\) ) длина волны уменьшается ( \(\mathrm{d} \lambda<0\) ).

Найдем теперь длину волны \(\lambda_{m}\), соответствующую максимуму функции \(u_{\lambda}\). Это значит, надо решить уравнение \(\mathrm{d} u_{\lambda} / \mathrm{d} \lambda=0\) :

\[
\frac{\mathrm{d} u_{\lambda}}{\mathrm{d} \lambda}=\frac{1}{\lambda^{5}} T \frac{\mathrm{~d} \mathscr{T}}{\mathrm{~d} \lambda}-\frac{5}{\lambda^{6}} \mathscr{T}=\frac{1}{\lambda^{6}}\left(\lambda T \frac{\mathrm{~d} \mathscr{T}}{\mathrm{~d} \lambda}-5 \mathscr{F}\right)=0 .
\]

Выражение в скобках есть некоторая функция Ф(λT). При длине волны \(\lambda_{m}\), соответствующей максимуму функции \(u_{\lambda}\), функция Ф(λT) должна обратиться в нуль: \(\Phi\left(\lambda_{m} T\right)=0\). Решение последнего уравнения приводит к некоторому значению \(b\) величины \(\lambda_{m} T\). Таким образом, можно записать, что


\begin{equation*}
T \lambda_{m}=b . \tag{4.33}
\end{equation*}


Это и есть закон смещения Вина. Значение постоянной \(b\) можно найти экспериментально или с помощью формулы Планка:


\begin{equation*}
b=0,29 \text { см } \cdot \text { К. }^{\circ} \tag{4.34}
\end{equation*}


Из формулы (4.33) видно, что с ростом температуры длина волны \(\lambda_{m}\) уменьшается, а значит, частота \(v_{m}\) увеличивается, как показано на рис. 4.12. Заметим только, что \(v_{m} \neq c / \lambda_{m}\), поскольку \(v_{m}\) соответствует распределению по частотам, а \(\lambda_{m}\) — по длинам волн.

С помощью закона смещения Вина легко определить температуру \(T\) электромагнитного излучения (или его источника), спектр которого соответствует формуле Планка. Так находят, например, температуру звезд.

Пример. Определим длину волны \(\lambda_{m}\), соответствующую максимуму функции распределения \(u_{\lambda}\) теплового излучения, энергетическая светимость которого \(M=30 \mathrm{Br} / \mathrm{cm}^{2}\).

Для этого следует использовать закон смещения Вина, температура \(T\) в котором связана с величиной \(M\) законом Стефа-на-Больцмана. В результате получим:

\[
\begin{gathered}
\lambda_{m}=b \sqrt[4]{\sigma / M}=0,29 \cdot 10^{-2} \sqrt[4]{5,67 \cdot 10^{-8} /\left(30 \cdot 10^{4}\right)}= \\
=1,9 \cdot 10^{-6} \text { м }=1,9 \text { мкм. }
\end{gathered}
\]

\section*{§ 4.5. Теплоемкость твердых тел}
Классическая модель. В основе классической теории теплоемкости твердых тел (кристаллов) лежит закон равнораспределения энергии по степеням свободы. Твердое тело рассматривают как систему \(N\) независимьιx друг от друга атомов, имеющих по три колебательных степени свободы. На каждую из них приходится в среднем энергия \(k T(k T / 2\) в виде кинетической и \(k T / 2\) в виде потенциальной). Имея в виду, что число колебательных степеней свободы равно \(3 N\), получим, что внутренняя энергия одного моля атомов \(U=3 N_{A} k T=3 R T\). Отсюда молярная теплоемкость


\begin{equation*}
C=\partial U / \partial T=3 R . \tag{4.35}
\end{equation*}


\begin{figure}[h]
\begin{center}
  \includegraphics[alt={},max width=\textwidth]{bd023a69-cc46-4b6f-a48e-cc548fa9259d-124_358_495_1149_68}
\captionsetup{labelformat=empty}
\caption{Рис. 4.13}
\end{center}
\end{figure}

В этом суть закона Дюлонга и Пти, который утверждает, что молярная теплоемкость всех химически простых твердых тел одинакова и равна \(3 R\). Этот закон выполняется достаточно хорошо только при сравнительно высоких температурах. Опыт показывает, что при низких температурах теплоемкость тел убывает (рис. 4.13), стремясь к нулю при \(T \rightarrow 0\) по закону \(C\) c \(T^{3}\).

Модель Эйнштейна. Трудности, на которые натолкнулась классическая теория в вопросе о теплоемкости твердых тел, устранила квантовая теория. В первоначальной модели, разработанной Эйнштейном, кристалл рассматривался как система\\
\(N\) атомов, каждый из которых является квантовым гармоническим осциллятором. Предполагалось, что колебания атомов происходят независимо друг от друга с одинаковой частотой \(v\).

Энергия квантового гармонического осциллятора дискретна:


\begin{equation*}
\varepsilon_{v}=(v+1 / 2) h v, \quad v=0,1,2, \ldots \tag{4.36}
\end{equation*}


Можно показать (см. решение задачи 4.8), что средняя энергия такого осциллятора


\begin{equation*}
\langle\varepsilon\rangle=\frac{h v}{2}+\frac{h v}{\mathrm{e}^{h v / k T}-1} . \tag{4.37}
\end{equation*}


Первое слагаемое здесь - это так называемая нулевая энергия данного осциллятора. Она не зависит от \(T\) и не имеет отношения к тепловому движению. Поэтому в теории теплоемкости тел ее можно опустить и выражение для внутренней энергии одного моля будет иметь вид:


\begin{equation*}
U=3 N_{A}\langle\varepsilon\rangle=3 N_{A} \frac{h v}{\mathrm{e}^{h v / k T}-1} . \tag{4.38}
\end{equation*}


Теперь можно найти молярную теплоемкость кристаллической решетки:


\begin{equation*}
C=\frac{\partial U}{\partial T}=\frac{3 R(h v / k T)^{2}}{\left(\mathrm{e}^{h v / k T}-1\right)^{2}} \mathrm{e}^{h v / k T} . \tag{4.39}
\end{equation*}


Это выражение называют формулой Эйнштейна.\\
При высоких температурах ( \(k T \gg h \nu\) ) она переходит в формулу (4.35). При низких же температурах ( \(k T \ll h v\) ) можно пренебречь единицей в знаменателе, и


\begin{equation*}
C=3 R(h v / k T)^{2} \mathrm{e}^{-h v / k T} . \tag{4.40}
\end{equation*}


При \(T \rightarrow 0\) и \(C \rightarrow 0\), и ход кривой \(C(T)\) в общем почти совпадает с результатами опыта (см. рис. 4.13). Это был серьезный успех квантовой теории*.

\footnotetext{\begin{itemize}
  \item Заметим, что теория Эйнштейна применима и к колебательной теплоемкости газов, состоящих из молекул.
\end{itemize}
}Вместе с тем, предсказываемое теорией поведение теплоемкости при \(T \rightarrow 0\) не очень согласуется с наблюдаемым. По Эйнштейну зависимость \(C(T)\) должна иметь экспоненциальный характер, а опыт дает, что \(C\) ა \(T^{3}\). При других температурах формула Эйнштейна также согласуется с экспериментальными данными только в качественном отношении.

Эти расхождения связаны не с существом квантовой теории, а с чрезмерным упрощением самой модели твердого тела, т. е. с предположением, что все атомы колеблются независимо друг от друга и с одинаковой частотой. Это понимал и сам Эйнштейн, он же указал, в каком направлении следует развивать квантовую модель.

Модель Дебая. В этой модели кристаллическая решетка рассматривается как связанная система взаимодействующих атомов. Колебания такой системы - результат наложения многих гармонических колебаний с различными частотами. Под гармоническим осциллятором той или иной частоты теперь надо понимать колебания не отдельного атома, а всей системы в целом. Задача сводится к нахождению спектра частот этих осцилляторов. Это весьма сложно. Дебай сильно упростил задачу. Он обратил внимание на то, что при низких температурах основной вклад в теплоемкость вносят колебания (осцилляторы) низких частот, которым соответствуют малые кванты энергии \(h v\). Практически только такие колебания и возбуждены ври низких температурах. Низкочастотный же спектр колебаний решетки может быть рассчитан достаточно точно, и вычисления оказываются довольно простыми. Таким путем Дебай построил теорию теплоемкости твердых тел, особенно хорошо согласующуюся с опытом при низких температурах. Из теории следовало, что при \(T \rightarrow 0\) действительно \(C\) ɔ \(T^{3}\).

Мы не будем воспроизводить подробно рассуждения Дебая. Чтобы получить результаты, к которым пришел Дебай, мы поступим иначе.

Фононы. Ситуация, с которой мы встретились (кристалл, объем которого заполнен стоячими упругими волнами - квантовыми осцилляторами), аналогична той, которую мы имели в случае полости, заполненной электромагнитными квантовыми осцилляторами. Там, по идее Эйнштейна, оказалось возмож-

ным представить электромагнитное излучение в полости в виде фотонного газа. Это наводит на мысль поступить подобным же образом и в случае упругих волн.

Упругие колебания (осцилляторы) в кристалле имеют квантовые свойства, проявляющиеся в том, что существует наименьшая порция - квант энергии с частотой \(v\). Это позволяет сопоставить упругой волне с частотой \(v\) квазичастиць - фононьι, распространению которых со скоростью \(v\) соответствует упругая волна.

Введение понятия фонона, как выяснилось, является плодотворным приемом, значительно облегчающим рассуждения. Оно также весьма эффективно в математическом отношении, так как математические приемы вычисления различных величин, связанных с фононами, аналогичны соответствующим вычислениям, относящимся к фотонам.

Фонон характеризуют энергией ɛ и импульсом \(p\) :


\begin{equation*}
\varepsilon=h v, \quad p=h v / v, \tag{4.41}
\end{equation*}


где \(v\) - скорость фонона (скорость волны), \(v\) - его частота. Импульс имеет направление, совпадающее с направлением распространения упругой волны*.

Таким образом, подобно тому, как квантование электромагнитных волн приводит \(\kappa\) фотонам, квантование упругих волн - к фононам.

Выяснилось, что фонону следует приписать спин, равный нулю. Значит фононы - это бозоны и подчиняются статистике Бозе-Эйнштейна. Фононы могут рождаться и исчезать, при этом число их не сохраняется (оно зависит от температуры \(T\) ), поэтому для фононного газа химический потенциал \(\mu=0\), и функция \(f\) заполнения фазовых ячеек определяется формулой (4.5).

Колебательная энергия решетки. Колебательную энергию \(U\) кристаллической решетки можно рассматривать как энергию фононного газа. Повторив рассуждения, которые приводят к


го газа. Повторив рассуждения, которые приводят к Квантовые статистики и их применения

* Импульс фонона обладает своеобразными свойствами: при взаимодействии фононов друг с другом их импульс может дискретными порциями передаваться кристаллической решетке и, следовательно, не сохраняться. В связи с этим импульс фонона называют квазиимпульсом.


\footnotetext{\begin{itemize}
  \item Импульс фонона обладает своеобразными свойствами: при взаимодействии фононов друг с другом их импульс может дискретными порциями передаваться кристаллической решетке и, следовательно, не сохраняться. В связи с этим импульс фонона называют квазиимпульсом.
\end{itemize}
}формуле (4.8), определяющей число частиц (в расчете на единицу объема) в интервале энергий ( \(\varepsilon, \varepsilon+d \varepsilon\) ), и учитывая, что мы имеем дело с бозонами, запишем:


\begin{equation*}
\mathrm{d} n=\gamma f \mathrm{~d} Z, \tag{4.42}
\end{equation*}


где согласно (4.5), (4.7) и (4.41)


\begin{equation*}
f=\frac{1}{\mathrm{e}^{h v / k T}-1}, \quad \mathrm{~d} Z=\frac{4 \pi p^{2} \mathrm{~d} p}{h^{3}}=\frac{4 \pi v^{2} \mathrm{~d} v}{v^{3}} . \tag{4.43}
\end{equation*}


В твердых телах могут распространяться три волны: продольная и поперечные с двумя взаимно ортогональными поляризациями. Их скорости несколько отличаются друг от друга, поэтому под \(v\) имеется в виду их средняя скорость. В соответствии с наличием трех волн, в (4.2) коэффициент \(\gamma=3\). Другими словами, он учитывает три возможные поляризации фононов.

Энергия фононного газа в интервале частот ( \(v, v+\mathrm{d} v\) ) равна произведению энергии одного фонона на их число в данном интервале частот: \(\mathrm{d} U=h v \mathrm{~d} n\), где \(\mathrm{d} n\) определяется формулами (4.42) и (4.43). Остается проинтегрировать полученное выражение по всем возможным частотам:


\begin{equation*}
U=\int h v \mathrm{~d} n=\frac{12 \pi h}{v^{3}} \int_{0}^{v_{\text {макс }}} \frac{v^{3} \mathrm{~d} v}{\mathrm{e}^{h v / k T}-1}, \tag{4.44}
\end{equation*}


где \(v_{\text {макс }}\) - верхняя граница возможных частот фононов. Для определения этой частоты приходится вводить довольно искусственное (как и в рассуждениях Дебая) условие. А именно, полное число квантовых состояний фононного газа, т. е. фазовых ячеек с учетом трех возможных поляризаций фононов, должно равняться числу степеней свободы \(3 n_{0}\left(n_{0}\right.\) - концентрация атомов):


\begin{equation*}
3 n_{0}=\int_{0}^{\nu_{\text {макс }}} 3 \mathrm{~d} Z=\frac{12 \pi}{v^{3}} \frac{v_{\text {макс }}^{3}}{3}=\frac{4 \pi}{v^{3}} v_{\text {макс }}^{3}, \tag{4.45}
\end{equation*}


где использовано выражение (4.43) для \(\mathrm{d} Z\). Таким образом,


\begin{equation*}
v_{\text {макс }}=v \sqrt[3]{3 n_{0} / 4 \pi} . \tag{4.46}
\end{equation*}


Отметим, что для упругой волны соответствующая этой частоте длина волны оказывается равной


\begin{equation*}
\lambda_{\text {мин }}=\frac{v}{v_{\text {макс }}}=\sqrt[3]{\frac{4 \pi}{3 n_{0}}} \sim 2 d, \tag{4.47}
\end{equation*}


поскольку \(n_{0}=1 / d^{3}, d\) - период решетки. Этот результат согласуется с тем, что волны с \(\lambda<2 d\) не имеют физического смысла (рис. 4.14). Это служит разумным оправданием условия (4.45).

\begin{figure}[h]
\begin{center}
  \includegraphics[alt={},max width=\textwidth]{bd023a69-cc46-4b6f-a48e-cc548fa9259d-129_184_451_387_709}
\captionsetup{labelformat=empty}
\caption{Рис. 4.14}
\end{center}
\end{figure}

Учитывая (4.46), перепишем выражение (4.44) для энергии \(U\) единиць объема фононного газа в виде


\begin{equation*}
U=\frac{9 n_{0} h}{v_{\text {макс }}^{3}} \int_{0}^{v_{\text {макс }}} \frac{v^{3} \mathrm{~d} v}{\mathrm{e}^{h v / k T}-1} . \tag{4.48}
\end{equation*}


Теплоемкость кристалла. Зная \(U(T)\), находим, что теплоемкость единицы объема кристалла


\begin{equation*}
C=\frac{\partial U}{\partial T}=\frac{9 n_{0} h^{2}}{k T^{2} v_{\text {макс }}^{3}} \int_{0}^{v_{\text {макс }}} \frac{\mathrm{e}^{h v / k T} v^{4} \mathrm{~d} v}{\left(\mathrm{e}^{h v / k T}-1\right)^{2}} . \tag{4.49}
\end{equation*}


Введем так называемую характеристическую температуру Дебая \(Θ\), определяемую условием


\begin{equation*}
h v_{\text {макс }}=k \Theta, \tag{4.50}
\end{equation*}


а также новую переменную \(x=h v / k T\). Тогда выражение для теплоемкости (4.49) примет вид


\begin{equation*}
C=9 n_{0} k\left(\frac{T}{\Theta}\right)^{3} \int_{0}^{x_{m}} \frac{\mathrm{e}^{x} x^{4} \mathrm{~d} x}{\left(\mathrm{e}^{x}-1\right)^{2}}, \tag{4.51}
\end{equation*}


где \(x_{m}=h v_{\text {макс }} / k T=\Theta / T\). Выражение (4.51) называют формулой Дебая.

Отметим еще, что дебаевская температура \(\Theta\) указывает для каждого твердого тела область температур ( \(T<\Theta\) ), где становится существенным квантование энергии колебаний.


вится существенным квантование энергии колебаний.

Квантовые статистики и их применения


\section*{Поведение теплоемкости в предельных случаях}
\begin{enumerate}
  \item При \(T \ll \Theta\) можно приближенно считать, что верхний предел интеграла (4.51) \(x_{m} \rightarrow \infty\). Тогда интеграл будет представлять собой некоторое число, и мы видим, что в этом случае
\end{enumerate}


\begin{equation*}
C \sim T^{3} . \tag{4.52}
\end{equation*}


Этот результат называют законом \(T^{3}\) Дебая. Именно такую зависимость \(C\) от \(T\) и наблюдают во многих случаях, при \(T \ll \Theta\).\\
2. При \(T \gg \Theta\), т. е. при \(h v_{\text {макс }} / k T \ll 1\), выражение (4.48) для \(U\) можно упростить, считая \(\mathrm{e}^{h v / k T} \approx 1+h v / k T\). Тогда


\begin{equation*}
U=\frac{9 n_{0} h}{v_{\text {макс }}^{3}} \int_{0}^{v_{\text {макс }}} \frac{k T}{h v} v^{3} \mathrm{~d} v=3 n_{0} k T . \tag{4.53}
\end{equation*}


Для моля кристалла заменяем \(n_{0}\) на \(N_{A}\), и получим, что для молярной теплоемкости кристалла

\[
C=3 R,
\]

как и должно быть в соответствии с законом Дюлонга и Пти.\\
Резюме. Зависимость \(C(T)\), рассчитанная по формуле (4.51), т. е. по формуле Дебая, очень хорошо описывает экспериментальные результаты, например, для алюминия и меди. Но полученные соотношения не являются универсальными. Во-первых, они хорошо передают зависимость \(C(T)\) только для химически простых тел с простой кристаллической решеткой. К телам с более сложной структурой формула Дебая не применима. Это связано с тем, что у таких тел спектр колебаний (распределение фононов по квантовым состояниям) оказывается очень сложным.

Мы не собираемся вникать в эти детали. Для нас важен главный вывод: квантовый подход - это единственный путь к решению подобных проблем.

Роль электронного газа в теплоемкости кристалла. Расчет показывает, что при \(T \neq 0\) средняя энергия свободного электрона в металле имеет вид


\begin{equation*}
\langle\varepsilon\rangle=\frac{3}{5} \varepsilon_{F}\left[1+\frac{5 \pi^{2}}{12}\left(\frac{k T}{\varepsilon_{F}}\right)^{2}\right], \tag{4.54}
\end{equation*}


где \(\varepsilon_{F}\) - энергия Ферми при \(T=0\). Тогда молярная теплоемкость электронного газа


\begin{equation*}
C_{\text {эл }}=N_{A} \frac{\partial\langle\varepsilon\rangle}{\partial T}=\frac{\pi^{2}}{2} \frac{k T}{\varepsilon_{F}} R . \tag{4.55}
\end{equation*}


По закону Дюлонга и Пти молярная теплоемкость решетки при нормальных условиях \(C_{\text {реш }}=3 R\). Тогда отношение электронной теплоемкости к решеточной при нормальных условиях будет равно


\begin{equation*}
\frac{C_{\text {эл }}}{C_{\text {реш }}}=\frac{\pi^{2}}{6} \frac{k T}{\varepsilon_{F}} . \tag{4.56}
\end{equation*}


Поскольку при рассматриваемых условиях \(k T \ll \varepsilon_{F}\), то это означает, что теплоемкость металлов за счет свободных электронов пренебрежимо мала. Напомним, это обусловлено тем, что при обычных температурах в тепловом движении принимает участие лишь небольшая часть общего числа свободных электронов - только те электроны, энергия которых лежит вблизи уровня Ферми. Таким образом, поведение вырожденного электронного газа резко отличается от поведения обычного газа, его степени свободы оказываются в основном «замороженными».

Заметим, что при достаточно низких температурах ситуация становится обратной: теплоемкость электронного газа превосходит решеточную, поскольку последняя уменьшается \(\circlearrowleft T^{3}\).

\section*{Задачи}
4.1. Свободные электроны в металле. Найти функцию распределения \(\mathrm{d} n / \mathrm{d} \lambda\) свободных электронов в металле при \(T=0\) по дебройлевским длинам волн (в расчете на единицу объема).

Решение. Число свободных электронов в интервале дебройлевских длин волн ( \(\lambda, \lambda+\mathrm{d} \lambda\) ) должно равняться их числу в соответствующем интервале энергий ( \(\varepsilon, \varepsilon+\mathrm{d} \varepsilon\) ), т.е. \(\mathrm{d} n_{\lambda}=\mathrm{d} n_{\varepsilon}\). Имея в виду формулу де-Бройля \(\lambda=h / p\) и связь между \(\varepsilon\) и \(\lambda\), а именно \(\varepsilon= =p^{2} / 2 m=h^{2} / 2 m \lambda^{2}\), а также формулу (4.13), получим:

\[
\mathrm{d} n_{\lambda}=2 \alpha \sqrt{\varepsilon} \mathrm{~d} \varepsilon=\sqrt{2} \alpha \frac{h^{3} \mathrm{~d} \lambda}{m^{3 / 2} \lambda^{4}} .
\]

Знак минус здесь опущен: он не существен. Раскрыв с помощью (4.11) значение \(\alpha\), найдем

\[
\mathrm{d} n_{\lambda}=8 \pi \frac{\mathrm{~d} \lambda}{\lambda^{4}} .
\]

Заметим, что это распределение ограничено наименьшей дебройлевской длиной волны \(\lambda_{\text {мин }}\), значение которой следует из энергии Ферми \(\varepsilon_{F}\) :

\[
\lambda_{\text {мин }}=\frac{h}{\sqrt{2 m \varepsilon_{F}}}=\frac{2 \pi}{\left(3 \pi^{2} n\right)^{1 / 3}} .
\]

Полагая решетку металла кубической с периодом \(a\) и содержащую по одному свободному электрону на атом, получим \(n=1 / a^{3}\) и

\[
\lambda_{\text {мин }} \approx 2 a .
\]

4.2. Дискретность спектра свободных электронов. Вычислить интервал между соседними энергетическими уровнями свободных электронов в металле при \(T=0\) вблизи уровня Ферми. Считать, что концентрация свободных электронов \(n=2 \cdot 10^{22} \mathrm{~cm}^{-3}\) и объем металла \(V=1 \mathrm{~cm}^{3}\).

Решение. Исходим из формулы (4.13), где учтено, что в каждой фазовой ячейке находятся два электрона с взаимно противоположными спинами. Поэтому \(\delta n=2\), и мы имеем

\[
\delta \varepsilon=\frac{1}{\alpha \sqrt{\varepsilon}} .
\]

После подстановки в эту формулу значения \(\alpha\) (4.11) и \(\varepsilon=\varepsilon_{F}\) из (4.15) получим

\[
\delta \varepsilon=\frac{h^{2}}{2 \pi m}\left(\frac{\pi}{3 n}\right)^{1 / 3} \approx 2 \cdot 10^{-22} \text { эВ. }
\]

Соответствующую дискретность уровней обнаружить просто невозможно. Именно поэтому спектр энергетических уровней свободных электронов можно рассматривать как непрерывный (точнее квазинепрерывный).\\
4.3. Давление электронного газа. Вычислить давление электронного газа на стенки металла при \(T=0\), считая, что концентрация свободных электронов \(n=2,5 \cdot 10^{22} \mathrm{~cm}^{-3}\).

Решения. Согласно уравнению идеальных газов для давления (1.30)

\[
p=\frac{2}{3} n\langle\varepsilon\rangle,
\]

где \(\langle\varepsilon\rangle=(3 / 5) \varepsilon_{F}\) в соответствии с формулой (4.16). После подстановки выражения (4.15) для \(\varepsilon_{F}\) получим

\[
p=\frac{(3 / \pi)^{2 / 3}}{20} \frac{h^{2} n^{5 / 3}}{m}=5 \cdot 10^{9} \text { Па } \approx 5 \cdot 10^{4} \text { атм. }
\]

4.4. Термодинамика фотонного газа. Замкнутая полость объемом \(V=1,0\) л заполнена тепловым излучением (фотонным газом), температура которого \(T=1000\) К. Найти его\\
a) теплоемкость \(C_{V}\) и б) энтропию \(S\).

Решение. а) По определению \(C_{V}=(\partial U / \partial T)_{V}\). Решение сводится к нахождению внутренней энергии \(U\). Это легко сделать, зная, что плотность \(u\) энергии излучения определяется формулами \((4.29)\) и (4.30): \(u=4 M / c=4 \sigma T^{4} / c\). Тогда \(U=u V\) и

\[
C_{V}=\frac{16}{c} \sigma T^{3} V=3 \cdot 10^{-9} \text { Дж } / \text { К. }
\]

б) В соответствии с теоремой Нернста (3.5)

\[
S=\int_{0}^{T} \frac{\mathrm{~d} U}{T}=\frac{16}{3 c} \sigma T^{3} V=1 \cdot 10^{9} \text { ДжК. }
\]

4.5. Найти уравнение адиабаты (в переменных \(T, V\) ) для равновесного фотонного газа, имея в виду, что его давление \(p\) зависит от плотности энергии \(u\) как \(p=u / 3\).\\
Решени е. По определению адиабатического процесса

\[
\mathrm{d}^{\prime} Q=\mathrm{d} U+p \mathrm{~d} V=0
\]

В нашем случае это выглядит так:

\[
\mathrm{d}(u V)+(u / 3) \mathrm{d} V=0 .
\]

Отсюда следует, что

\[
\frac{\mathrm{d} u}{u}+\frac{4}{3} \frac{\mathrm{~d} V}{V}=0
\]

и \(\ln u+(4 / 3) \ln V=\) const, или \(\ln \left(u V^{4 / 3}\right)=\) const. Поскольку \(u\) ა \(T^{4}\), то в результате получим

\[
V T^{3}=\text { const. }
\]

4.6. Излучение Солнца по своему спектральному составу близко к излучению черного тела, для которого максимум излучения приходится на длину волны \(\lambda_{m}=0,48\) мкм. Найти массу, теряемую Солнцем ежесекундно за счет этого излучения.\\
Р еш ен и е. Поток излучения с поверхности Солнца равен \(\Phi=4 \pi R^{2} M\), где \(R\) - радиус Солнца, \(M\) - энергетическая светимость. Согласно закону Стефана-Больцмана (4.30) и закону смещения Вина (4.33), \(M=\sigma T^{4}=\sigma b^{4} / \lambda_{m}^{4}\). Поток \(\Phi\) - это энергия, теряемая Солнцем ежесекундно. Разделив \(\Phi\) на \(c^{2}\), получим соответствующую массу в кг/с. Таким образом,

\[
\left|\frac{\mathrm{d} m}{\mathrm{~d} t}\right|=\frac{4 \pi R^{2} \sigma b^{4}}{c^{2} \lambda_{m}^{4}}=5,1 \cdot 10^{9} \text { кг/с. }
\]

Это гигантская, на первый взгляд тревожная, величина. Но масса Солнца \(m=2 \cdot 10^{30}\) кг. При такой интенсивности излучения масса Солнца уменьшится, скажем, на \(1 \%\) за время

\[
t=\frac{m}{|\mathrm{~d} m / \mathrm{d} t|} \approx 4 \cdot 10^{18} \mathrm{c} \approx 10^{11} \text { лет. }
\]

Так что ситуация не безнадежная.\\
4.7. О квантовых осцилляторах. Система квантовых осцилляторов с частотой \(v\) находится при температуре \(T\). С какой вероятностью \(P_{v}\) можно обнаружить в этой системе осциллятор с энергией \(\varepsilon_{v}=(v+1 / 2) h v\) ? Квантовое число \(v\) задано.\\
Решение. Искомая вероятность по определению есть


\begin{equation*}
P_{v}=N_{v} / N \tag{1}
\end{equation*}


где \(N_{v}=A \exp \left(-\varepsilon_{v} / k T\right), N\) - полное число осцилляторов с частотой \(v\). Раскроем правую часть равенства (1):


\begin{equation*}
P_{v}=\frac{A \exp \left(-\varepsilon_{v} / k T\right)}{\sum_{v=0}^{\infty} A \exp \left(-\varepsilon_{v} / k T\right)}=\frac{\mathrm{e}^{-v x}}{1+\mathrm{e}^{-x}+\mathrm{e}^{-2 x}+\ldots}, \tag{2}
\end{equation*}


где \(x=h v / k T\). Сумма членов бесконечной геометрической прогрессии равна \(1 /\left(1-\mathrm{e}^{-x}\right)\), поэтому (2) в результате примет вид

\[
P_{v}=\mathrm{e}^{-v x}\left(1-\mathrm{e}^{-x}\right)
\]

4.8. Средняя энергия квантового осциллятора. Имеется система, состоящая из невзаимодействующих квантовых осцилляторов с одной и той же частотой \(v\). Энергия каждого осциллятора может принимать значения \(\varepsilon_{v}=(v+1 / 2) h v\), где \(v=0,1,2, \ldots\) Используя распределение Больцмана, показать, что средняя энергия таких осцилляторов

\[
\langle\varepsilon\rangle=\frac{h v}{2}+\frac{h v}{\mathrm{e}^{h v / k T}-1}
\]

Решение. Согласно распределению Больцмана число осцилляторов с энергией, соответствующей квантовому числу \(v\), равно

\[
N_{v}=A \exp \left(-\varepsilon_{v} / k T\right)
\]

где \(A\) - нормировочный коэффициент. Тогда средняя энергия < \(\epsilon\) > определяется формулой


\begin{equation*}
\langle\varepsilon\rangle=\frac{\sum \varepsilon_{v} N_{v}}{\sum N_{v}}=\frac{\sum \varepsilon_{v} \exp \left(-\alpha \varepsilon_{v}\right)}{\sum \exp \left(-\alpha \varepsilon_{v}\right)}, \tag{1}
\end{equation*}


где \(\alpha=1 / k T\). Здесь суммирование проводится по \(v\) от 0 до \(\infty\), и делается это так:


\begin{equation*}
\langle\varepsilon\rangle=-\frac{\partial}{\partial \alpha} \ln \sum \exp \left(-\alpha \varepsilon_{v}\right) . \tag{2}
\end{equation*}


Известно, что сумма членов бесконечной геометрической прогрессии \(\Sigma=a_{1} /(1-q)\), где \(a_{1}\) — первый член прогрессии, \(q\) — ее знаменатель. В нашем случае это


\begin{equation*}
\sum \exp \left(-\alpha \varepsilon_{v}\right)=\frac{\mathrm{e}^{-\alpha h \nu / 2}}{1-\mathrm{e}^{-\alpha h \nu}} \tag{3}
\end{equation*}


Остается взять производную по \(\alpha\) от (3) и не забыть знак минус. В результате получим приведенное в условии выражение для \(\langle\varepsilon\rangle\).\\
4.9. Фононы. Оценить максимальные значения энергии и импульса фонона в меди, дебаевская температура которой \(\Theta=330\) К и плотность \(\rho=8,9 г / \mathrm{cm}^{3}\).

Решение. Максимальную энергию фонона находим с помощью (4.50):

\[
\varepsilon_{\text {макс }}=k \Theta=3 \cdot 10^{-2} \text { эВ. }
\]

Для максимального импульса имеем


\begin{equation*}
p_{\text {макс }}=h / \lambda_{\text {мин }} . \tag{*}
\end{equation*}


Согласно (4.47) \(\lambda_{\text {мин }} \approx 2 d\), где \(d\) находим из условия \(n \approx 1 / d^{3}\) и \(n=\left(N_{A} / M\right) \rho, M\) - молекулярная масса. Отсюда

\[
d \approx \sqrt[3]{M / N_{A} \rho}
\]

В итоге формула (*) примет вид

\[
p_{\text {макс }} \approx h / 2 d=(h / 2) \sqrt[3]{\rho N_{A} / M}=1,4 \cdot 10^{-19} \text { г.см/с. }
\]

4.10. Оценить давление фононного газа в меди, дебаевская температура которой \(\Theta=330\) К. Концентрация атомов меди равна \(n_{0}= =0,84 \cdot 10^{23} \mathrm{~cm}^{-3}\).

Решение. Давление идеального газа (в нашем случае фононного) согласно (1.30) равно


\begin{equation*}
p={ }^{2} /{ }_{3} U, \tag{*}
\end{equation*}


где \(U\) - энергия единицы объема фононного газа, определяемая формулой (4.48). Введем новую переменную \(x=h v / k T\) и преобразуем (*) к виду

\[
p=\frac{6 n_{0} k^{4} T^{4}}{h^{3} \nu_{\text {макс }}^{3}} \int_{0}^{x} \frac{x^{3} \mathrm{~d} x}{\mathrm{e}^{x}-1}
\]

Теперь учтем, что \(T=\Theta\), согласно (4.49) \(x_{m}=1\) и значение интеграла (табличное) равно 0,225 . В результате получим:

\[
p=6 n_{0} k \Theta \cdot 0,225 \approx 5 \cdot 10^{8} \Pi а=5 \cdot 10^{3} \text { атм. }
\]

4.11. Кристалл состоит из \(N_{0}\) одинаковых атомов. Его дебаевская температура равна \(\Theta\). Найти число \(\mathrm{d} N\) фононов в интервале частот ( \(v, v+\mathrm{d} v\) ) при температуре \(T\) кристалла.\\
Решение. Согласно формулам (4.42) и (4.43) для объема \(V\) кристалла имеем:


\begin{equation*}
\mathrm{d} N=\gamma f \mathrm{~d} Z \cdot V=3 \frac{1}{\mathrm{e}^{h \nu / k T}-1} \frac{4 \pi \nu^{2} \mathrm{~d} \nu}{v^{3}} V . \tag{1}
\end{equation*}


Исключим из этой формулы скорость \(v\) с помощью соотношений (4.46) и (4.50):


\begin{equation*}
v^{3}=\frac{4 \pi v_{\text {макс }}^{3}}{3 n_{0}}=\frac{4 \pi}{3 n_{0}}\left(\frac{k \Theta}{h}\right)^{3} \tag{2}
\end{equation*}


После подстановки (2) в (1) получим

\[
\mathrm{d} N=9 N_{0}\left(\frac{h}{k \Theta}\right)^{3} \frac{v^{2} d v}{\mathrm{e}^{h v / k T}-1}
\]

где учтено, что \(n_{0} V=N_{0}\).\\
4.12. При нагревании кристалла меди массы \(m\) от \(T_{1}\) до \(T_{2}\) ему было сообщено количество теплоты \(Q\). Найти дебаевскую температуру Ө для меди, если известно, что молярная теплоемкость кристалла при этих температурах зависит от \(T\) как

\[
C=\frac{12 \pi^{4}}{5} \frac{R}{\Theta^{3}} T^{3}
\]

Р e ш e н и . По определению искомая теплота

\[
Q=\int_{T_{1}}^{T_{2}} v C(T) \mathrm{d} T=\frac{3 \pi^{4}}{5} \frac{m R}{M \Theta^{3}}\left(T_{2}^{4}-T_{1}^{4}\right)
\]

где \(v\) - количество вещества ( \(m / M\) ). Отсюда

\[
\Theta=\sqrt[3]{\frac{3 \pi^{4}}{5} \frac{m R}{M Q}\left(T_{2}^{4}-T_{1}^{4}\right)}
\]

\section*{Состояния вещества}
\section*{§ 5.1. Изотермы Ван-дер-Ваальса}
Теоретические изотермы. Анализ изотерм ван-дер-ваальсовского газа дает возможность получить весьма существенные результаты. На рис. 5.1 показаны три наиболее характерные изо-


\begin{figure}[h]
\begin{center}
  \includegraphics[alt={},max width=\textwidth]{bd023a69-cc46-4b6f-a48e-cc548fa9259d-138_400_487_532_68}
\captionsetup{labelformat=empty}
\caption{Рис. 5.1}
\end{center}
\end{figure}

Еще интересней ведет себя изотерма при \(T_{1}\) : она содержит волнообразный участок, между точками \(A\) и \(B\) которого \(\partial p / \partial V>0\). Очевидно, такого не может быть, чтобы с изотермическим увеличением объема росло бы и давление.

Экспериментальные изотермы. Что же дает эксперимент? Его результаты представлены на рис. 5.2. Первое впечатление


\begin{figure}[h]
\begin{center}
  \includegraphics[alt={},max width=\textwidth]{bd023a69-cc46-4b6f-a48e-cc548fa9259d-138_426_479_1329_68}
\captionsetup{labelformat=empty}
\caption{Рис. 5.2}
\end{center}
\end{figure}

исходит в цилиндре. При достижении объема \(V_{г}\) цилиндр заполнен одним газом, но при дальнейшем сжатии наряду с газом появится жидкость, количество которой постепенно будет расти, а давление оставаться постоянным. Когда объем достигнет \(V_{\text {ж }}\), он целиком окажется заполненным жидкостью.

Таким образом, на участке \(C D\) мы наблюдаем так называемый фазовый переход вещества из газообразного состояния в жидкое. Это позволяет сделать вывод: если фазовый переход происходит изотермически, то он совершается при \(p\) = const. Оказывается, это общее свойство всех фазовых переходов: не только газообразной фазы в жидкую (и наоборот), но и жидкой фазы в твердую и т. д.

Вообще, фазой называют физически однородную часть вещества, отделенную от других частей системы границей раздела. Например, вода и лед, вода и ртуть, графит и алмаз, но вода и спирт - это одна фаза.

Вернемся к рис. 5.2, где показаны экспериментальные изотермы. Всем горизонтальным изотермам в области под пунктирной кривой (она слабо тонирована) соответствуют двухфазные состояния - жидкость с паром, находящиеся в равновесии друг с другом. Такой пар называют насыщенным.

Левее двухфазной области расположена область, соответствующая одной фазе - жидкости (она тонирована сильнее). Здесь изотермы идут очень круто, что отвечает малой сжимаемости жидкости. Правее двухфазной области вещество находится в газообразном состоянии. Причем все состояния вне этой двухфазной области неплохо описываются уравнением Ван-дер-Ваальса. Это говорит о том, что данное уравнение описывает состояние вещества не только в газообразном, но и в жидком состоянии*.

Таким образом, если газ начинают сжимать, например, по изотерме 1 (см. рис. 5.2), то он переходит в двухфазное состояние (горизонтальный участок изотермы \(C D\) ) жидкость + насыщенный пар. Объем жидкой фазы растет по мере приближения к точке \(C\), после которой остается одна фаза - жидкость. Но это не всё.


исходит в цилиндре. При достижении объема Vг цилиндр заполнен одним газом, но при дальнейшем сжатии наряду с газом появится жидкость, количество которой постепенно будет расти, а давление оставаться постоянным. Когда объем достигнет Vж, он целиком окажется заполненным жидкостью.

Таким образом, на участке CD мы наблюдаем так называемый фазовый переход вещества из газообразного состояния в жидкое. Э


\footnotetext{\begin{itemize}
  \item Заметим, что горизонтальный участок реальной изотермы (он изображен пунктиром на рис. 5.1) расположен на таком уровне, что охватываемые волнообразной изотермой площади под пунктирной прямой и над ней одинаковы.
\end{itemize}
}Метастабильные состояния. Более тщательные эксперимен-


Метастабильные состояния. Более тщательные эксперименты (вещество очищалось от примесей и процесс проводился осторожнее) позволили обнаружить, что при изотермическом сжатии газа его можно довести до состояния B (рис. 5.3), а при расширении жидкости — до состояния А. Т. е. реализуются участки, предсказываемые уравнением Ван-дер-Ваальса.

Особенности этих участков в том, что каждая точка их соответст


\begin{figure}[h]
\begin{center}
  \includegraphics[alt={},max width=\textwidth]{bd023a69-cc46-4b6f-a48e-cc548fa9259d-140_394_566_201_68}
\captionsetup{labelformat=empty}
\caption{Рис. 5.3}
\end{center}
\end{figure}

Особенности этих участков в том, что каждая точка их соответствует неравновесному состоянию при данной температуре. Эти состояния называют метастабильньлми. При малейшем возмущении вещества в этих состояниях происходит распад его на две фазы, соответствующие равновесному состоянию (это показано тонкими вертикальными стрелками на рисунке).

Участок \(B D\) описывает состояния, которые называют пересыщенным паром, а участок \(C A\) - перегретой жидкостью. Пересыщенный пар можно получить, подвергнув непересыщенный пар резкому, т. е. адиабатному расширению. Пар при этом охлаждается и становится пересыщенным. Это используют в камере Вильсона - приборе для наблюдения следов (треков) заряженных частиц. Последние, пролетая через пересыщенный пар, оставляют цепочку ионов, на которых и происходит конденсация пара в результате распада метастабильного состояния. В итоге образуется цепочка капелек - след частицы.

Перегретая жидкость также нашла применение - в пузырьковых жидководородных камерах. Заряженная частица, пролетая через перегретую жидкость, также оставляет цепочку ионов, на которых происходит парообразование. Возникает цепочка пузырьков - след пролетевшей частицы.

Кипение. Нагревание жидкости при постоянном внешнем давлении сопровождается парообразованием. Сначала оно происходит спокойно и только с поверхности жидкости. Такой процесс называют испарением. По достижении же определенной температуры - температуры кипения - образование пара на-

чинается и внутри жидкости (благодаря наличию множества мельчайших пузырьков газа, в которые и начинает происходить интенсивное испарение). Процесс становится бурным - начинается кипение. По существу, кипение есть особый вид испарения.

Таким образом, кипение возможно только тогда, когда внутри жидкости имеются пузырьки газа. Если же таковых нет, кипение становится невозможным, и жидкость можно нагреть выше температуры кипения - она становится перегретой, т. е. оказывается в метастабильном состоянии.

\section*{§ 5.2. Фазовые переходы. Диаграмма состояний}
Фазовые переходы. Различают фазовые переходы 1-го и 2-го рода. Первые сопровождаются теплотой перехода. Это то количество теплоты, которое необходимо сообщить веществу, чтобы изотермически-изобарически перевести его из одной фазы в другую. Этот процесс характеризуют удельной

\begin{enumerate}
  \item теплотой испарения (конденсации) \(q_{12}\), Дж/кг;
  \item теплотой плавления (кристаллизации) \(q_{\text {пл }}\), Дж/кг.
\end{enumerate}

Заметим, что теплоту испарения твердого тела называют теплотой сублимации.

Фазовые же переходы 2 -го рода происходят без теплообмена. Это, например, изменение кристаллической модификации, переход в сверхпроводящее состояние, в сверхтекучее состояние у жидкого гелия, переход ферромагнетизма в парамагнетизм. Этим мы и ограничимся, основное же внимание сосредоточим на фазовых переходах 1-го рода.

Уравнение Клапейрона-Клаузиуса. Известно, что две любые фазы вещества могут находиться в равновесии лишь при определенном давлении, зависящем от температуры. Например, для испарения (конденсации) кривая зависимости \(p(T)\) показана на рис. 5.4 , где \(K\) - критическая точка, слева от этой кривой - жидкая фаза, справа - газообразная. Представляет практический интерес знать аналитический вид этой кривой.

Общий вид зависимости \(p(T)\), но в дифференциальной форме, был получен из термодинамических соображений. А именно, можно найти наклон кривой, т. е. \(\mathrm{d} p / \mathrm{d} T\) (большего получить термодинамика не позволяет). Делается это так.

\begin{figure}[h]
\begin{center}
  \includegraphics[alt={},max width=\textwidth]{bd023a69-cc46-4b6f-a48e-cc548fa9259d-142_374_431_145_68}
\captionsetup{labelformat=empty}
\caption{Рис. 5.4}
\end{center}
\end{figure}

\begin{figure}[h]
\begin{center}
  \includegraphics[alt={},max width=\textwidth]{bd023a69-cc46-4b6f-a48e-cc548fa9259d-142_418_527_141_633}
\captionsetup{labelformat=empty}
\caption{Рис. 5.5}
\end{center}
\end{figure}

Рассмотрим элементарный цикл Карно в области фазового перехода (рис. 5.5). Здесь \(V_{1}^{\prime}\) и \(V_{2}^{\prime}\) - удельные объемы фазы 1 и фазы 2, т. е. объемы единицы массы, \(\mathrm{m}^{3} / к г\).

Работа, совершаемая рабочим веществом за цикл, равна площади, ограниченной этим циклом (стр. 14), т. е. \(\mathrm{d} p \cdot\left(V_{2}^{\prime}-V_{1}^{\prime}\right)\). Представим теперь данный цикл на диаграмме \(T\) - \(S\). Ясно, что работа за цикл при этом не изменится и тоже будет равна площади, ограниченной циклом (стр. 85), а именно \(\mathrm{d} T \cdot\left(S_{2}^{\prime}-S_{1}^{\prime}\right)\), где в скобках - значения удельной энтропии тех же фаз вещества. Таким образом,


\begin{equation*}
\mathrm{d} p \cdot\left(V_{2}^{\prime}-V_{1}^{\prime}\right)=\mathrm{d} T \cdot\left(S_{2}^{\prime}-S_{1}^{\prime}\right) . \tag{5.1}
\end{equation*}


С другой стороны, приращение удельной энтропии при изотермическом фазовом переходе \(1 \rightarrow 2\) равно согласно (3.3) выражению


\begin{equation*}
S_{2}^{\prime}-S_{1}^{\prime}=\frac{q_{12}}{T}, \tag{5.2}
\end{equation*}


где \(q_{12}\) - удельная теплота данного фазового перехода. Из последних формул видно, что


\begin{equation*}
\frac{\mathrm{d} p}{\mathrm{~d} T}=\frac{q_{12}}{T\left(V_{2}^{\prime}-V_{1}^{\prime}\right)} . \tag{5.3}
\end{equation*}


Это и есть уравнение Клапейрона-Клаузиуса. Следует обратить внимание на порядок символов 1 и 2 соответствующих фаз.

Видно, что знак производной, характеризующей наклон кривой \(p(T)\), зависит от того, как изменяются удельные объемы при поглощении тепла - возрастают или уменьшаются. При испарении жидкости или твердого тела удельный объем всегда возрастает, поэтому производная \(\mathrm{d} p / \mathrm{d} T\) соответствующих кривых может быть только положительной. Заметим попутно, что кривую испарения твердого тела называют кривой сублимации.

При плавлении удельный объем подавляющего числа веществ возрастает, поэтому \(\mathrm{d} p / \mathrm{d} T>0\). Однако у некоторых веществ, к числу которых принадлежит вода, объем жидкой фазы меньше объема твердой фазы ( \(V_{2}^{\prime}<V_{1}^{\prime}\) ), и \(\mathrm{d} p / \mathrm{d} T<0\).

Диаграмма состояний. Так называют плоскость \(p, T\) (рис. 5.6), разделенную на три области: твердую фазу (т), жидкую (ж) и газообразную (г). Границами этих соприкасающихся фаз являются кривые сублимаиии, испарения и плавления, характеризующие двухфазные равновесные состояния. В точке \(T p\), ее называют тройной, в равновесии находятся три фазы. Пунктиром показана кривая плавления для веществ, у которых

\begin{figure}[h]
\begin{center}
  \includegraphics[alt={},max width=\textwidth]{bd023a69-cc46-4b6f-a48e-cc548fa9259d-143_405_497_715_663}
\captionsetup{labelformat=empty}
\caption{Рис. 5.6}
\end{center}
\end{figure}

На рис. 5.6 показана типичная диаграмма состояний. Подобные диаграммы строят экспериментально для разных веществ. Они полезны в том отношении, что позволяют предсказывать фазовые превращения в тех или иных процессах. Например, мы видим, что кривая испарения заканчивается в критической точке \(K\). Поэтому возможен переход из области жидкой фазы в газообразную путем обхода точки \(K\) «сверху» (без пересечения кривой испарения). В этом случае переход из одной фазы в другую совершается непрерывно, через последовательность однофазных состояний.

Кроме того, кривая плавления может встретить на своем «пути» другую тройную точку \(T p^{\prime}\) (см. рис. 5.6), являющуюся

началом вилки: левая кривая будет являться границей между двумя твердыми (но разными) фазами, а правая - кривой плавления новой твердой фазы. В дальнейшие детали мы углубляться не будем.

Завершим этот обзор представлением фазовой диаграммы в

\begin{figure}[h]
\begin{center}
  \includegraphics[alt={},max width=\textwidth]{bd023a69-cc46-4b6f-a48e-cc548fa9259d-144_418_546_377_70}
\captionsetup{labelformat=empty}
\caption{Рис. 5.7}
\end{center}
\end{figure}

объемном виде. В параметрах \(p, T, V\) она будет выглядеть так, как показано на рис. 5.7. Это сложная поверхность, каждая точка которой отвечает определенному равновесному состоянию - или однофазному, или двухфазному, или трехфазному. Непрерывными линиями показаны изотермы, параллельные части их (параллельные оси \(V\) ) соответствуют двухфазным состояниям, а одна - трехфазному состоянию.

Достаточно посмотреть на объемную фазовую диаграмму (рис. 5.7) справа налево (вдоль оси \(V\) ), и можно узнать диаграмму на рис. 5.6, а если - вдоль оси \(T\), то - диаграмму, правая часть которой показана на рис. 5.2 (полезно самостоятельно изобразить и ее левую часть).

Представленная на рис. 5.2. фазовая диаграмма соответствует веществам, удельный объем которых при затвердевании уменьшается (таких веществ подавляющее большинство). В противном же случае, например, для системы вода-лед, левая часть объемной фазовой диаграммы имеет иной, более сложный вид, но мы на этом останавливаться не будем.

\section*{§ 5.3. Жидкое состояние}
Если вещество находится в газовой фазе, то молекулы его движутся почти свободно, поскольку их энергия взаимодействия значительно меньше кинетической энергии (речь идет о средних значениях этих величин). При этом газ заполняет все доступное для него пространство (например, объем сосуда).

В жидкой фазе молекулы находятся вплотную друг к другу, но как и в газе, обладают большой подвижностью и расположены неупорядоченно. Энергия взаимодействия молекул сравнима с их кинетической энергией, и это проявляется в том, что жидкость занимает определенный объем. В расположении частиц жидкости наблюдается так называемый ближний порядок, т. е. по отношению к любой частице расположение ближайших частиц является упорядоченным. В отличие от жидкости, в кристалле имеет место дальний порядок.

В связи с отсутствием в жидкости дальнего порядка она, как правило, не обнаруживает анизотропии, характерной для кристаллов. В жидкостях с удлиненными молекулами наблюдается одинаковая ориентация их в пределах значительного объема, что приводит к появлению анизотропии ряда свойств. Это так называемые жидкие кристалль. Жидкие кристаллы обладают очень важными оптическими свойствами, что обеспечило их многочисленные применения и большой интерес к их изучению. На свойства жидких кристаллов сильное влияние оказывают электрические и магнитные поля. Изучение этих влияний предмет интенсивных научных исследований, а получаемые результаты широко используются в практике (управление световыми потоками, цифровые указатели на жидких кристаллах, дисплеи и др.).

Жидкость занимает промежуточное положение между газом и кристаллами. Это связано с тем, что жидкое состояние является наиболее сложным по своим свойствам. Поэтому его теория развита гораздо менее, чем у газообразного и кристаллического состояний. До сих пор нет вполне законченной и общепризнанной теории жидкостей.

Одна из моделей поведения молекул в жидкости, предложенная Я. И. Френкелем, выглядит так. Каждая молекула в течение некоторого времени колеблется около определенного положения равновесия. Затем молекула меняет свое местоположение, скачком перемещаясь в новое положение равновесия. Таким образом молекула медленно перемещается внутри жидкости, странствуя по всему ее объему. При повышении температуры сильно возрастает подвижность молекул и вязкость жидкости уменьшается.

Построение теории жидкостей наталкивается на принципиальную трудность, связанную с тем, что взаимодействие молекул приводит к зависимости энергии системы от их взаимного положения. Учет этого обстоятельства делает невозможным вычисление равновесного распределения молекул. Преодолеть эту трудность пока не удалось. Привлечение к решению этой проблемы даже самых быстродействующих компьютеров не позволяет рассчитывать модель системы с большим числом частиц.

Несмотря на отсутствие теории жидкостей, некоторые важные свойства их изучены достаточно полно и представляют существенный практический интерес. В первую очередь это поверхностное натяжение и связанные с ним явления.

Поверхностное натяжение. Поверхностный слой жидкости находится в особых условиях. Дело в том, что радиус молекулярного действия \(r_{0} \sim 10^{-7} \mathrm{~cm}\) (несколько междуатомных расстояний), и это приводит к тому, что в поверхностном слое такой толщины на молекулы действуют силы, направленные вглубь жидкости. Это легко понять из рис. 5.8.

\begin{figure}[h]
\begin{center}
  \includegraphics[alt={},max width=\textwidth]{bd023a69-cc46-4b6f-a48e-cc548fa9259d-146_229_941_980_145}
\captionsetup{labelformat=empty}
\caption{Рис. 5.8}
\end{center}
\end{figure}

Здесь показано, что происходит на границе раздела жидкости со своим паром (при обычных условиях). Видно, что по мере приближения молекулы (жирной точки) к границе раздела действующая на нее сила со стороны других молекул увеличивается. По мере же удаления молекулы от жидкости эта сила уменьшается и на расстоянии \(r_{0}\) практически обращается в нуль.

Слева на рис. 5.8 изображен график хода потенциальной энергии \(U_{\text {п }}\) молекулы в области, прилегающей к границе раздела. Этот график указывает на то, что молекулы жидкости находятся по существу в потенциальной яме, для выхода из кото-

рой (преодоления потенциального барьера) им необходима избыточная кинетическая энергия*.

Таким образом, выход молекулы из глубины жидкости на поверхность сопровождается увеличением её потенциальной энергии - за счет кинетической энергии (выходящие на поверхность молекулы «остывают»). Значит, поверхностный слой обладает дополнительной потенциальной энергией, которая вместе с поверхностью стремится уменьшиться, и жидкость, предоставленная самой себе в условиях невесомости, принимает форму шара (поверхность его минимальна). «Стремление» уменьшить свою поверхность проявляется в возникновении сил поверхностного натяжения, касательных к поверхности.

Наличие поверхностного натяжения эффектно демонстрируется с помощью мыльных пленок. Для этого берут проволочный П-образный каркас с подвижной перемычкой длины \(l\) (рис. 5.9). Замкнутый контур затягивают пленкой, которая сразу же, стремясь уменьшить свою поверхность, начинает поднимать перемычку. Чтобы воспрепятствовать этому, приходится прикладывать внешнюю силу \(F\).

Обозначим силу \(f_{\text {ед }}\) поверхностного натя-

\begin{figure}[h]
\begin{center}
  \includegraphics[alt={},max width=\textwidth]{bd023a69-cc46-4b6f-a48e-cc548fa9259d-147_368_283_671_877}
\captionsetup{labelformat=empty}
\caption{Рис. 5.9}
\end{center}
\end{figure}

жения, приходящуюся на единицу длины контура, через \(\alpha\) :


\begin{equation*}
f_{\mathrm{eg}}=\alpha, \mathrm{H} / \mathrm{M} . \tag{5.4}
\end{equation*}


Величину а называют поверхностным натяжением.\\
Медленно перемещая перемычку вниз на расстояние \(\mathrm{d} x\), сила \(F\), равная \(2 \alpha l\), совершает работу \(2 \alpha l \mathrm{~d} x\). Коэффициент 2 появился из-за того, что пленка имеет два поверхностных слоя. Сама жидкость совершает над перемычкой работу, равную \(\mathrm{d}^{\prime} A=-2 \alpha l \mathrm{~d} x=-\alpha \mathrm{d} S\), где \(\mathrm{d} S\) - приращение площади поверхностного слоя. При этом дополнительное количество молекул переходит из глубины жидкости в поверхностный слой, что сопровождается, как было сказано выше, охлаждением жидко-


\footnotetext{\begin{itemize}
  \item Вот почему для испарения жидкости требуется подвод тепла. Скорости молекул жидкости распределены по закону Максвелла, и покинуть ее могут только наиболее быстрые молекулы. В результате их ухода жидкость охлаждалась бы без подвода тепла, и испарение становилось бы все слабее.
\end{itemize}
}сти. Однако если этот процесс проводить медленно, он протекает изотермически, и это дает возможность иной интерпретации величины \(\alpha\).

Ранее (§ 3.6) мы установили, что работа системы в изотермическом процессе совершается за счет убыли свободной энергии \(F\). Тогда полученное выше выражение \(\mathrm{d}^{\prime} A=-\alpha \mathrm{d} S\) с учетом равенства \(\mathrm{d}^{\prime} A=-\mathrm{d} F\) приводит к выводу, что


\begin{equation*}
\alpha=\mathrm{d} F / \mathrm{d} S, \quad \text { Дж } / \mathrm{m}^{2} . \tag{5.5}
\end{equation*}


Т. е. поверхностное натяжение \(\alpha\) представляет собой свободную энергию единицы площади поверхностного слоя. Заметим, что с ростом температуры натяжение \(\alpha\) уменьшается, обращаясь в нуль при критической температуре (при этой температуре граница между жидкостью и насыщенным паром исчезает).

Поверхностное натяжение \(\alpha\) зависит от свойств жидкости и варьируется в довольно широких пределах. У большинства жидкостей при \(20^{\circ} \mathrm{C}\) а имеет порядок от \(10^{-2}\) до \(10^{-1} \mathrm{H} / \mathrm{m}\) или Дж/м² (см. Приложение № 6).

В состоянии равновесия свободная энергия, а значит и поверхность жидкости, должна быть минимальной. Это открывает великолепную возможность чисто экспериментально и просто с помощью пленок находить форму поверхности, натянутой на контур любой пространственной конфигурации, площадь которой была бы минимальной. Математическое решение этой практически важной задачи сопряжено со значительными трудностями.

С поверхностным натяжением связан ряд явлений, представляющих сугубо практический интерес.

Давление под изогнутой поверхностью. Рассмотрим простейший случай, когда поверхность жидкости сферическая

\begin{figure}[h]
\begin{center}
  \includegraphics[alt={},max width=\textwidth]{bd023a69-cc46-4b6f-a48e-cc548fa9259d-148_277_568_1468_68}
\captionsetup{labelformat=empty}
\caption{Рис. 5.10}
\end{center}
\end{figure}

(рис. 5.10). Выделим мысленно на этой поверхности с радиусом кривизны \(R\) небольшой кружок радиуса \(r\) ( \(r \ll R\) ). Этот участок поверхности растягивают силы поверхностного натяжения, направленные почти радиально\\
(точнее, в виде зонтика). Благодаря этим силам в жидкости возникает дополнительное давление \(\Delta p=f_{\text {ед }}^{\prime} \cdot 2 \pi r / \pi r^{2}\), или


\begin{equation*}
\Delta p=\frac{2 \alpha}{R}, \tag{5.6}
\end{equation*}


где учтено, что \(f_{\text {ед }}^{\prime}=f_{\text {ед }} \cdot \theta=\alpha \cdot r / R\). Здесь \(R\) - величина алгебраическая: если центр кривизны \(O\) поверхности находится внутри жидкости (как на рисунке), то \(R>0\), если же вне жидкости, то поверхность будет не выпуклая, а вогнутая и \(R<0\). Соответственно в первом случае дополнительное давление \(\Delta p>0\), а во втором \(\Delta p<0\) (рис. 5.11).

\begin{figure}[h]
\begin{center}
  \includegraphics[alt={},max width=\textwidth]{bd023a69-cc46-4b6f-a48e-cc548fa9259d-149_257_850_649_189}
\captionsetup{labelformat=empty}
\caption{Рис. 5.11}
\end{center}
\end{figure}

Это явление приводит, например, к повышению давления внутри пузырьков с воздухом радиуса \(R=5\) мкм на \(\Delta p= =0,3\) атм.

Лаплас обобщил формулу (5.6) на поверхность любой формы. Формула Лапласа выглядит так:


\begin{equation*}
\Delta p=\alpha\left(\frac{1}{R_{1}}+\frac{1}{R_{2}}\right), \tag{5.7}
\end{equation*}


где \(R_{1}\) и \(R_{2}\) - радиусы кривизны в двух взаимно перпендикулярных плоскостях, пересечение которых совпадает с нормалью к поверхности жидкости в интересующей нас точке. Оба радиуса кривизны - величины алгебраические. Правило знаков для них то же, что было приведено выше. Например, для седлообразной поверхности радиусы кривизны имеют противоположные знаки.

Явления на границах между средами. Твердые тела, как и жидкости, по тем же причинам обладают поверхностной энергией и натяжением. При этом, если речь идет о границе раздела двух сред, то следует иметь в виду, что поверхностная энер-

гия на границе раздела зависит от свойств обеих сред. Другими словами, надо рассматривать суммарную поверхностную энергию \(\alpha_{12}\) двух сред.

Рассмотрим каплю жидкости на поверхности твердого тела (рис. 5.12). Для равновесия необходимо, чтобы все силы, дейст-

\begin{figure}[h]
\begin{center}
  \includegraphics[alt={},max width=\textwidth]{bd023a69-cc46-4b6f-a48e-cc548fa9259d-150_295_441_379_68}
\captionsetup{labelformat=empty}
\caption{Рис. 5.12}
\end{center}
\end{figure}

вующие на элемент контура, перпендикулярный рисунку в точке \(O\), уравновешивались. Эти силы состоят из сил поверхностного натяжения \(\quad \alpha_{12}, \quad \alpha_{23}, \quad \alpha_{13}\), действующих вдоль границ раздела между средами. Равнодействующая этих сил \(\alpha_{12}+\alpha_{23}+\alpha_{13} \quad\) уравновешивается молекулярными силами твердого тела, которые направлены вниз. Поэтому равновесие обеспечивается равенством нулю суммарной проекции сил на горизонтальное направление, или \(\alpha_{13}= =\alpha_{12} \cos \theta+\alpha_{23}\). Отсюда


\begin{equation*}
\cos \theta=\frac{\alpha_{13}-\alpha_{23}}{\alpha_{12}}, \tag{5.8}
\end{equation*}


где \(\theta\) - так называемый краевой угол. Его обычно выбирают в области, занятой жидкостью (как на рис. 5.12).

Если \(\theta<\pi / 2\), то говорят, что жидкость смачивает поверхность твердого тела. При \(\theta \rightarrow 0\) имеет место полное смачивание. Оно будет и при условии, когда правая часть (5.8) окажется больше единицы, т. е. при условии

\[
\alpha_{13}>\alpha_{12}+\alpha_{23} .
\]

Если \(\theta>\pi / 2\), то жидкость не смачивает поверхность. При \(\theta \rightarrow \pi\) мы имеем полное несмачивание. Оно будет наблюдаться и при условии, когда правая часть (5.8) окажется менее -1 , т. е. при условии

\[
\alpha_{23}>\alpha_{13}+\alpha_{12} .
\]

На рис. 5.13 приведены примеры смачивания ( \(\theta<\pi / 2\) ) и несмачивания ( \(\theta>\pi / 2\) ) - при соприкосновении жидкости с горизонтальной и вертикальной поверхностями твердого тела.

\begin{figure}[h]
\begin{center}
  \includegraphics[alt={},max width=\textwidth]{bd023a69-cc46-4b6f-a48e-cc548fa9259d-151_261_959_137_135}
\captionsetup{labelformat=empty}
\caption{Рис. 5.13}
\end{center}
\end{figure}

Капиллярные явления. Наличие краевого угла приводит к искривлению поверхности жидкости вблизи стенок сосуда. Изогнутые поверхности называют менисками. Если узкую трубку (капилляр) погрузить одним концом в жидкость, то в результате смачивания уровень жидкости в ней будет выше, чем в сосуде, а при несмачивании - ниже (рис. 5.14). Все явления, связанные с искривлением поверхности жидкости, называют капиллярными.

Найдем разность уровней \(h\) между жидкостью в капилляре радиуса \(r\) и в сосуде, полагая известными поверхностное натяжение \(\alpha\) жидкости (на границе с атмосферой), ее плотность \(\rho\) и краевой угол \(\theta\).

Разность уровней \(h\) должна быть такой, чтобы гидростатический вес столба жидкости единичного сечения уравновешивался дополнительным давлением \(\Delta p\) под мениском. Будем считать, что мениск в капилляре имеет сферическую форму (это очень близко к реальности). Тогда согласно (5.6) с учетом того, что \(R=r / \cos \theta\) - это видно из рис. 5.15 - получим

\begin{figure}[h]
\begin{center}
  \includegraphics[alt={},max width=\textwidth]{bd023a69-cc46-4b6f-a48e-cc548fa9259d-151_249_279_641_881}
\captionsetup{labelformat=empty}
\caption{Рис. 5.14}
\end{center}
\end{figure}

\begin{figure}[h]
\begin{center}
  \includegraphics[alt={},max width=\textwidth]{bd023a69-cc46-4b6f-a48e-cc548fa9259d-151_392_273_958_887}
\captionsetup{labelformat=empty}
\caption{Рис. 5.15}
\end{center}
\end{figure}


\begin{equation*}
\Delta p=\frac{2 \alpha \cos \theta}{r}, \tag{5.9}
\end{equation*}


и при равновесии \(\Delta p=\rho g h\), поэтому


\begin{equation*}
\rho g h=-\frac{2 \alpha \cos \theta}{r} . \tag{5.10}
\end{equation*}


Отсюда следует, что разность уровней


\begin{equation*}
h=\frac{2 \alpha \cos \theta}{\rho g r} . \tag{5.11}
\end{equation*}


Для смачивающей жидкости ( \(\theta<\pi / 2\) ) \(h>0\), т. е. уровень жидкости в капилляре поднимается, для несмачивающей жидкости \((\theta>\pi / 2) h<0\), значит, уровень в капилляре опускается.

Пример. Найдем разность \(\Delta h\) уровней ртути в двух сообщающихся вертикальных стеклянных капиллярах 1 и 2 , радиусы сечения которых \(r_{1}\) и \(r_{2}\), причем \(r_{1}<r_{2}\). Краевой угол равен \(\theta\).

\begin{figure}[h]
\begin{center}
  \includegraphics[alt={},max width=\textwidth]{bd023a69-cc46-4b6f-a48e-cc548fa9259d-152_376_302_437_211}
\captionsetup{labelformat=empty}
\caption{Рис. 5.16}
\end{center}
\end{figure}

Ртуть по отношению к стеклу является несмачивающей жидкостью, поэтому картина будет выглядеть как на рис. 5.16. На уровне, отмеченном пунктиром, давления в обоих капиллярах одинаковы. Это значит, что разность дополнительных давлений должна уравновешиваться гидростатическим давлением столба жидкости высотой \(\Delta h\), т.е.

\[
\Delta p_{1}-\Delta p_{2}=\rho g \Delta h .
\]

Остается подставить в это выражение формулу (5.9), и мы получим

\[
\Delta h=\frac{2 \alpha|\cos \theta|}{\rho g}\left(\frac{1}{r_{1}}-\frac{1}{r_{2}}\right) .
\]

\section*{§ 5.4. Кристаллическое состояние}
Строение кристаллов. Подавляющее большинство твердых тел имеет кристаллическое строение, т. е. упорядоченное расположение частиц (атомов, ионов, молекул), из которых они образованы. С такой структурой кристаллов связано свойство анизотропии, которая проявляется в зависимости ряда физических свойств от направления.

Эти свойства не всегда проявляются, поскольку кристаллические тела встречаются, как правило, в виде поликристаллов - совокупности множества беспорядочно расположенных мелких кристалликов. Искусственно с помощью специальных условий можно получить из раствора большие монокристаллы. Последние, впрочем, встречаются и в природе у некоторых минералов.

Симметрия кристаллов. Кристаллической решетке присущи различные виды симметрии, т. е. свойства решетки совпадать с самой собой при некоторых мысленных пространственных перемещениях (например, поступательных и поворотах). Остановимся несколько подробнее на этих видах симметрии.

Любая идеальная кристаллическая решетка прежде всего характеризуется трансляционной симметрией. Это означает, что каждой решетке можно сопоставить три некомпланарных элементарных вектора \(\mathbf{a , b , c}\), при мысленном совмещении вдоль которых решетка совемещается сама с собой. Параллелепипед, построенный на векторах \(\mathbf{a , b , c}\), называют элементарной ячейкой (рис. 5.17), а модули этих векторов - периодами идентичности решетки вдоль соответствующих направлений (или постоянными решетки).

\begin{figure}[h]
\begin{center}
  \includegraphics[alt={},max width=\textwidth]{bd023a69-cc46-4b6f-a48e-cc548fa9259d-153_338_342_471_818}
\captionsetup{labelformat=empty}
\caption{Рис. 5.17}
\end{center}
\end{figure}

Элементарную ячейку можно выбрать различными способами. Если она включает наименьшее число атомов, то ее называют примитивной ячейкой. Примитивная ячейка - это элементарная ячейка наименьшего объема. Ее можно выбрать так, чтобы во всех вершинах соответствующего ей параллелепипеда находились только узлы (атомы) решетки. Однако обычно вместо примитивной выбирают элементарную ячейку с большим числом атомов, но обладающую той же симметрией, что и вся решетка.

Если решетка совмещается сама с собой при мысленном повороте вокруг некоторой оси на угол \(360^{\circ} / n\), где \(n\) - число совмещений за полный поворот, то эту ось называют осью симметрии \(n\)-го порядка. Кроме тривиальной оси 1 -го порядка, возможны только оси симметрии, порядок которых равен \(2,3,4\) и 6 . Им соответствуют углы поворота \(180,120,90\) и \(60^{\circ}\). Заметим, что симметрия 5 -го порядка у кристаллов не существует.

Кроме этих видов симметрии, возможны и другие. Более детальное описание и систематика их дается в курсах кристаллографии.

В заключение приведем некоторые характеристики кубических решеток. К ним относятся простая (примитивная), объем-

\begin{figure}[h]
\begin{center}
  \includegraphics[alt={},max width=\textwidth]{bd023a69-cc46-4b6f-a48e-cc548fa9259d-154_337_1112_138_57}
\captionsetup{labelformat=empty}
\caption{Рис. 5.18}
\end{center}
\end{figure}

ноцентрированная (оц) и гранецентрированная (гц) - они показаны на рис. 5.18. Примитивная ячейка простой кубической решетки содержит только один атом (на ячейку), однако элементарная ячейка объемноцентрированной решетки - два атома, а элементарная ячейка гранецентрированной решетки четыре атома. Чтобы это понять, надо учесть, что каждый атом в вершинах угла ячейки принадлежит восьми прилегающим ячейкам, т. е. на каждую из них приходится \(1 / 8\) данного атома. В кубической же ячейке 8 углов, отсюда \(8 \cdot 1 / 8=1\). Аналогично рассуждая, можно понять, почему в гранецентрированной решетке на элементарную ячейку приходится четыре атома.

Индексы Миллера. Решетчатая структура кристаллов приводит к необходимости выделять и определенным образом обозначать различные кристаллографические плоскости - плоскости, в которых находится множество атомов решетки. Практическое значение имеют лишь плоскости с наибольшим числом атомов на единицу площади (поверхностной плотностью).

Для этого пользуются специальной системой координат, связанной с кристаллом так, что координатные оси \(X, Y, Z\) берут вдоль ребер элементарной ячейки, а начало координат - в одном из узлов решетки. Координаты в такой системе измеряют в единицах, равных межатомным расстояниям, т. е. в единицах соответственно \(a, b, c\). Положение какой-либо плоскости однозначно определяется координатами трех точек этой плоскости, например, тех, в которых она пересекает три координатные оси. Если плоскость \(S\) (рис. 5.19) пересекает ось \(X\) в точке на расстоянии от начала \(O\) в \(4 a\), ось \(Y\) - в \(1 b\) и ось \(Z\) - в \(2 c\), то

положение плоскости задается тройкой чисел: \(4,1,2\).

Однако принято пользоваться для обозначения плоскостей в кристалле не этими числами, а особыми индексами Миллера, которые определяются следующим образом:

\begin{enumerate}
  \item находим координаты трех точек пересечения плоскости с коор-
\end{enumerate}

\begin{figure}[h]
\begin{center}
  \includegraphics[alt={},max width=\textwidth]{bd023a69-cc46-4b6f-a48e-cc548fa9259d-155_323_467_141_695}
\captionsetup{labelformat=empty}
\caption{Рис. 5.19}
\end{center}
\end{figure}

динатными осями (в единицах постоянных решетки);\\
2) обратные значения полученных чисел приводим к общему знаменателю и знаменатель отбрасываем. Числители полученных дробей и есть индексы Миллера \(h, k, l\), которые заключают в круглые скобки ( \(h k l\) ).

Так, например, для только что рассмотренной плоскости, пересекающей оси координат в точках \(4,1,2\), обратные значеня данных координат будут \(1 / 4,1,1 / 2\). Общий знаменатель этих дробей равен 4 , и индексы Миллера оказываются равными 1, 4, 2. Данные числа записывают в круглых скобках, так что интересующая нас плоскость символически обозначается как (142).

Этот набор индексов определяет, очевидно, не одну плоскость, а все семейство параллельных плоскостей. Если плоскость параллельна одной из координатных осей, т. е. пересекает ее в бесконечности, то соответствующий индекс Миллера равен нулю.

На рис. 5.20 приведены индексы Миллера некоторых наиболее важных плоскостей кубической решетки, т. е. плоскостей с малыми индексами.

\begin{figure}[h]
\begin{center}
  \includegraphics[alt={},max width=\textwidth]{bd023a69-cc46-4b6f-a48e-cc548fa9259d-155_367_1101_1404_64}
\captionsetup{labelformat=empty}
\caption{Рис. 5.20}
\end{center}
\end{figure}

\begin{figure}[h]
\begin{center}
  \includegraphics[alt={},max width=\textwidth]{bd023a69-cc46-4b6f-a48e-cc548fa9259d-156_285_269_145_72}
\captionsetup{labelformat=empty}
\caption{Рис. 5.21}
\end{center}
\end{figure}

Если плоскость проходит через начало координат (рис. 5.21), то для определения отрезков, отсекаемых на осях координат, надо сместить эту плоскость параллельно самой себе на некоторую величину и затем определить соответствующие отрезки, например, \(1,-1, \infty\), или \(-1,1, \infty\). Переходя к миллеровским индексам данной плоскости, находим (110) или (110). Поясним символ 1 : при пересечении плоскости с координатной осью в отрицательной ее части соответствующий отрезок, а значит и миллеровский индекс будут отрицательными. Для краткости знак минус пишут над индексом.

Кристаллографические плоскости играют большую роль в методах рентгено- и нейтроноструктурного анализов кристаллов.

Межплоскостное расстояние \(d\) для простой кубической peшетки определяется формулой

\[
d=\frac{a}{\sqrt{h^{2}+k^{2}+l^{2}}},
\]

где \(a\) - постоянная решетки, \(h, k, l\) - миллеровские индексы рассматриваемой системы плоскостей. Эта формула получается так. Плоскость ( \(h k l\) ), ближайшая к началу координат, взятому в одном из узлов решетки, отсекает на осях координат отрезки \(a / h, a / k, a / l\). Расстояние от начала координат до этой плоскости равно \(d\). Обозначив углы между нормалью к плоскости и осями координат \(X, Y, Z\) соответственно \(\alpha, \beta, \gamma\), получим:

\[
\cos \alpha=\frac{d}{a / h}, \quad \cos \beta=\frac{d}{a / k}, \quad \cos \gamma=\frac{d}{a / l} .
\]

Остается учесть, что сумма квадратов этих косинусов равна единице, и мы приходим к формуле для \(d\).

Физические типы кристаллов. Различают четыре типа кристаллов в зависимости от природы частиц в узлах решетки и от характера сил взаимодействия между частицами. Это ионные, атомные, металлические и молекулярные.


Если плоскость проходит через начало координат (рис. 5.21), то для определения отрезков, отсекаемых на осях координат, надо сместить эту плоскость параллельно самой себе на некоторую величину и затем определить соответствующие отрезки, например, 1, –1, , или –1, 1, . Переходя к миллеровским индексам данной плоскости, находим (110) или (110). Поясним символ 1: при пересечении плоскости с координа


\begin{enumerate}
  \item Ионные кристалль. У них в узлах кристаллической решетки находятся ионы противоположных знаков. Силы взаимодействия между ними в основном электростатические. Такую связь называют гетерогенной (или ионной). Пример ионной решетки - кристалл NaCl. У него кубическая решетка с чередующимися ионами \(\mathrm{Na}^{+}\)и \(\mathrm{Cl}^{-}\).
  \item Атомные кристалль. В узлах решетки этих кристаллов расположены нейтральные атомы. Связь между ними называют гомеополярной (или ковалентной). Соответствующие силы взаимодействия имеют также электрический характер. Но их объяснение может быть дано только на основе квантовой теории.
\end{enumerate}

Гомеополярная связь осуществляется электронными парами - в ней участвует по одному электрону от каждого атома. Гомеополярная связь имеет направленный характер (в отличие от гетерополярной), т. е. воздействие направлено на тот атом, с которым у данного атома имеется совместная электронная пара. Данная связь может осуществляться только валентными электронами.

Типичные примеры атомных кристаллов - это графит и алмаз. Оба они состоят из атомов углерода, но резко отличаются кристаллическим строением. В отличие от алмаза, атомы углерода в графите расположены слоями, и это оказывает сильное влияние на различие свойств графита и алмаза.\\
3. Металлические кристалль. В узлах решетки этих кристаллов расположены положительные ионы. Пространство между ними заполнено электронным газом, который компенсирует силы отталкивания между ионами, удерживая их вместе. В противном случае решетка просто распалась бы. При этом и электронное облако удерживается в пределах решетки и не может ее покинуть.\\
4. Молекулярные кристалль. В узлах решетки этих кристаллов расположены определенным образом ориентированные молекулы. Силы взаимодействия между молекулами в кристалле имеют ту же природу, что и силы притяжения между молекулами реального газа. Поэтому их называют ван-дер-ваальсовскими силами.

О теплоемкости кристаллов. Данный вопрос достаточно подробно рассмотрен в § 4.5 как с классической точки зрения (закон Дюлонга и Пти), так и с квантовой (модели Эйнштейна и Дебая, фононы).

Этим мы и ограничимся, оставляя за рамками книги многие вопросы, представляющие как научный, так и практический интерес. Физика твердого тела продолжает интенсивно развиваться, занимая одно из приоритетных направлений науки.

\section*{§ 5.5. Плазма}
Повседневно мы встречаемся с веществом, находящемся в трех состояниях - газообразном, жидком и твердом. Вместе с тем существует и четвертое состояние вещества - плазма.

Плазма - это ионизированный квазинейтральный газ. Квазинейтральность газа означает, что число зарядов противоположных знаков в нем почти одинаково. Отношение числа ионизированных атомов к их полному числу в том же объеме называют степенью ионизации плазмь \(\alpha\). В зависимости от значения степени ионизации, говорят о слабой, сильной и полностью ионизированной плазме. Плазмой может быть состояние с \(\alpha \sim 10^{-3}\).

Сначала плазма вызывала интерес как особый проводник электрического тока и как источник света (газоразрядные лампы). Положение существенно изменилось в связи с пониманием роли плазмы во Вселенной, подавляющая часть которой (звезды и межзвездная среда) являются плазмой, а также в связи с возможностью создания принципиально новых источников энергии - управляемого термоядерного синтеза, преобразования энергии плазмы непосредственно в электрическую (в магнитогидродинамических генераторах).

Отметим основные особенности и свойства плазмы.


О теплоемкости кристаллов. Данный вопрос достаточно подробно рассмотрен в § 4.5 как с классической точки зрения (закон Дюлонга и Пти), так и с квантовой (модели Эйнштейна и Дебая, фононы).

Этим мы и ограничимся, оставляя за рамками книги многие вопросы, представляющие как научный, так и практический интерес. Физика твердого тела продолжает интенсивно развиваться, занимая одно из приоритетных напра


\begin{enumerate}
  \item Дебаевский радиус \(\boldsymbol{r}_{\boldsymbol{D}}\) - это характерный линейный размер области зарядовой декомпенсации в плазме. Из расчета следует, что
\end{enumerate}


\begin{equation*}
r_{D}=\sqrt{k T / \varkappa n \mathrm{e}^{2}}, \tag{5.12}
\end{equation*}


где \(\varkappa=1 / \varepsilon_{0}\) (СИ) и \(4 \pi\) (СГС), \(k\) - постоянная Больцмана, \(T\) температура, \(n\) - концентрация электронов. Дебаевский ради-

ус характеризует расстояние, на котором плазма экранирует себя от локального избытка заряда. Это надо понимать так.

Допустим, в плазме оказался некоторый заряд \(q>0\). Ближайшие к нему электроны плазмы будут им притягиваться, а ионы, наоборот, отталкиваться. В результате вокруг заряда \(q\) возникнет сферически симметричное электронное облако. Оно будет экранировать заряд \(q\) от расположенной вокруг плазмы, так что на некотором удалении от заряда \(q\) поле, образованное этим зарядом и электронным облаком, будет исчезающе малым. Это расстояние практически равно радиусу электронного облака, т. е. размеру области, в пределах которой сказывается действие заряда \(q\). Радиус этой области порядка дебаевского радиуса \(r_{D}\).

Оказывается, плазма экранирует не только действие помещенного в нее заряда, но и внешнее электрическое поле. Последнее проникает в плазму на расстояние тоже порядка дебаевского радиуса.\\
2. Условие квазинейтральности. Чтобы плазма сохраняла квазинейтральность, ее линейные размеры должны намного превосходить дебаевский радиус \(r_{D}\). И только при соблюдении квазинейтральности плазма ведет себя как связанный коллектив заряженных частиц. Это значит, что в плазме одновременно взаимодействует громадное число частиц, в отличие от обычных газов, где частицы взаимодействуют друг с другом в основном только при столкновениях. Этим свойством плазма обязана дальнодействию кулоновских сил, вовлекающих во взаимодействие множество частиц. С этим связана возможность разнообразных коллективных колебаний плазмы.\\
3. Плазменные колебания. В отличие от обычного газа, где тепловые флуктуации развиваются беспорядочно и могут заполнить весь объем газа, в плазме флуктуационное нарушение нейтральности жестко локализовано в достаточно малом объеме. При нарушении электронейтральности в какой-либо области плазмы в ней возникает электрическое поле и следствием этого - продольные колебания зарядов с некоторой частотой \(\omega_{\text {пл }}\). Ее называют плазменной частотой. Расчет показывает (см. решение задачи 5.11), что для электронов


\begin{equation*}
\omega_{п л}=\sqrt{\varkappa n \mathrm{e}^{2} / m}, \tag{5.13}
\end{equation*}


где \(\varkappa=1 / \varepsilon_{0}\) в СИ и \(4 \pi\) в СГС, \(m\) - масса электрона, \(n\) - их концентрация.

Эти так называемые плазменные колебания довольно быстро затухают в результате столкновений электронов с ионами. Они не имеют волнового характера, т. е. не распространяются по плазме. Плазменные колебания - это тот механизм, с помощью которого в плазме происходит выравнивание плотностей разноименных зарядов.\\
4. Температура плазмы. Большое различие масс электронов и ионов делает возможным существование квазиравновесных состояний плазмы, которые могут быть охарактеризованы двумя температурами - электронной \(T_{e}\) и ионной \(T_{i}\). Например, при получении плазмы с помощью газового разряда энергия от источника питания передается в основном электронам как наиболее подвижным частицам. Ионы же набирают энергию во вторичных процессах столкновений с электронами. Необходимое для выравнивания энергий число столкновений должно быть не менее нескольких тысяч на каждый ион.

Если такую плазму предоставить самой себе, то сначала установится максвелловское распределение скоростей электронов, затем - ионов. Возникает квазиравновесное состояние, в котором электроны будут иметь температуру \(T_{e}\), а ионы \(-T_{i}\), причем \(T_{e} \gg T_{i}\). Затем в результате обмена энергиями между электронами и ионами устанавливается максвелловское распределение для всей плазмы, т. е. общая температура электронов и ионов. Такую плазму называют изотермической.\\
5. Идеальная плазма. Так называют плазму, у которой средняя потенциальная энергия взаимодействия частиц мала по сравнению с их средней кинетической энергией. Поэтому тепловое движение частиц в плазме и идеальном газе обладает большим сходством, и термодинамические свойства такой плазмы достаточно хорошо описываются уравнением состояния идеального газа: \(p=n k T\).\\
6. Проводимость плазмы \(\sigma\). Она описывается той же формулой (4.18), где \(\tau\) - время релаксации плазменных электронов. Подстановка числовых значений входящих в эту формулу величин приводит к простому выражению для оценки проводимости плазмы (в СИ):


\begin{equation*}
\sigma \approx 10^{-3} T^{3 / 2}, \mathrm{OM}^{-1} \mathrm{M}^{-1} . \tag{5.14}
\end{equation*}


Благодаря быстрому увеличению σ плазмы с ростом температуры \(T\) достаточно разогретая плазма становится хорошим проводником.\\
7. Плазма в магнитном поле. Наиболее важный практический интерес, который представляет физика плазмы, связан с решением проблемы управляемого термоядерного синтеза. Для осуществления термоядерных реакций водородную плазму необходимо нагреть до температуры не менее \(10^{8}\) К. Поддержание столь высокой температуры невозможно при соприкосновении плазмы со стенками камеры, в которой она находится. Избежать этого дает возможность так называемая магнитная термоизоляиия: плазму помещают в сильное магнитное поле (например, тороидальное), препятствующее ионам и электронам перемещаться в поперечном направлении и уходить на стенки камеры. Известно, что заряженные частицы в сильном магнитном поле движутся по винтовым траекториям, которые «обвивают» линии вектора В.

Основная трудность в решении этой проблемы связана с получением устойчивой плазмы. Проблема оказалась чрезвычайно сложной и до сих пор окончательно не решена.

\section*{Задачи}
5.1. Фазовый переход. Некоторую массу вещества, взятого в состоянии насыщенного пара, изотермически сжали в \(n\) раз по объему. Какую часть \(\eta\) конечного объема занимает жидкая фаза, если удельные объемы насыщенного пара и жидкости отличаются друг от друга в \(N\) раз \((N>n)\) ?\\
Решение. Изобразим для наглядности происходящее на рис. 5.22 , где \(N=V_{\text {п }} / V_{\text {ж }}\) и \(n=V_{\text {п }} / V^{\prime}, V^{\prime}\) - конечный объем системы. Интересующая нас величина \(\eta=V_{\text {ж }} / V^{\prime}\). В конечном состоянии отметим все величины штрихом.

\begin{figure}[h]
\begin{center}
  \includegraphics[alt={},max width=\textwidth]{bd023a69-cc46-4b6f-a48e-cc548fa9259d-161_386_523_1383_641}
\captionsetup{labelformat=empty}
\caption{Рис. 5.22}
\end{center}
\end{figure}

Составим баланс масс и объемов обеих фаз в конечном состоянии


\begin{equation*}
m=m_{\Im}^{\prime}+m_{\Pi}^{\prime}, \quad V^{\prime}=V_{\Im}^{\prime}+V_{\Pi}^{\prime}=V_{\Im}^{\prime}+\frac{m-m_{\Im}}{m} V_{\Pi} . \tag{1}
\end{equation*}


Разделим последнее выражение на \(V^{\prime}\) :


\begin{equation*}
1=\frac{V_{\text {ж }}^{\prime}}{V^{\prime}}+\left(1-\frac{V_{\text {ж }}^{\prime}}{V_{\text {ж }}}\right) \frac{V_{\text {п }}}{V^{\prime}}=\frac{V_{\text {ж }}^{\prime}}{V^{\prime}}+\frac{V_{\text {п }}}{V^{\prime}}-\frac{V_{\text {ж }}^{\prime}}{V^{\prime}} \frac{V_{\text {п }}}{V_{\text {ж }}} . \tag{2}
\end{equation*}


Перепишем это выражение через \(\eta, n\) и \(N\) :


\begin{equation*}
1=\eta+n-\eta N . \tag{3}
\end{equation*}


Отсюда следует, что

\[
\eta=(n-1) /(N-1) .
\]

5.2. Критическое состояние. Определить для критического состояния вещества, ван-дер-ваальсовские постоянные которого \(a\) и \(b\) известны, значения следующих критических величин: молярного объема \(V_{\text {мкр }}, T_{\text {кр }}\) и \(p_{\text {кр }}\).\\
Решение. Из уравнения Ван-дер-Ваальса (1.42) следует, что


\begin{equation*}
p=\frac{R T}{V_{M}-b}-\frac{a}{V_{M}^{2}} . \tag{1}
\end{equation*}


В критической точке \(K\) производная \(\partial p / \partial V=0\), т.е.


\begin{equation*}
\frac{\partial p}{\partial V}=-\frac{R T}{\left(V_{M}-b\right)^{2}}+\frac{2 a}{V_{M}^{3}}=0 . \tag{2}
\end{equation*}


Кроме того, критическая изотерма в точке \(K\) испытывает перегиб. Это значит, что в точке \(K\) вторая производная \(\partial^{2} p / \partial V^{2}=0\), или


\begin{equation*}
\frac{\partial^{2} p}{\partial V^{2}}=\frac{2 R T}{\left(V_{M}-b\right)^{3}}-\frac{6 a}{V_{M}^{4}}=0 . \tag{3}
\end{equation*}


Исключив \(R T\) из уравнений (2) и (3), получим:


\begin{equation*}
V_{\text {Mкр }}=3 b . \tag{4}
\end{equation*}


Подстановка (4) в (2) дает:


\begin{equation*}
T_{\text {кр }}=\frac{8 a}{27 b R} . \tag{5}
\end{equation*}


Подставив, наконец, (4) и (5) в формулу (1), получим:

\[
p_{\text {кр }}=\frac{a}{27 b^{2}} .
\]

Отметим попутно, что, как легко показать, связь между этими критическими величинами имеет вид

\[
p_{\text {кр }} V_{M \text { кр }}=\frac{3}{8} R T_{\text {кр }} .
\]

5.3. Энтропия и фазовые переходы. Лед с начальной температурой \(T_{1}\), равной температуре таяния, сначала изотермически превратили в воду, а затем при температуре кипения \(T_{2}\) - в пар. Найти приращение \(\Delta S\) удельной энтропии системы в этом процессе.

Решение. Данный процесс состоит из трех частей: таяния (плавления), нагрева образовавшейся воды от \(T_{1}\) до \(T_{2}\) и превращения ее в пар. Соответственно можно записать в расчете на единицу массы:

\[
\Delta S=\frac{q_{\text {пл }}}{T_{1}}+\int_{T_{1}}^{T_{2}} c \frac{\mathrm{~d} T}{T}+\frac{q_{\text {пар }}}{T_{2}}=\frac{q_{\text {пл }}}{T_{1}}+c \ln \frac{T_{2}}{T_{1}}+\frac{q_{\text {пар }}}{T_{2}},
\]

где \(q_{\text {пл }}\) и \(q_{\text {пар }}\) - удельные теплоты плавления и парообразования, \(c\) - удельная теплоемкость воды.\\
5.4. Уравнение Клапейрона-Клаузиуса. Давление \(p\) насыщенного пара ртути зависит от температуры \(T\) по закону

\[
\ln p=-\frac{a}{T}-b \ln T+\mathrm{const},
\]

где \(a\) и \(b\) - положительные постоянные. Найти молярную теплоту испарения ртути как функцию температуры, \(q(T)\).\\
Решение. Воспользуемся уравнением Клапейрона-Клаузиуса (5.3). Учитывая, что молярный объем пара значительно больше, чем у жидкости ( \(V_{п} \gg V_{\text {ж }}\) ), запишем


\begin{equation*}
\mathrm{d} p / \mathrm{d} T=q / T V \tag{1}
\end{equation*}


Теперь найдем производную \(\mathrm{d} p / \mathrm{d} T\) из уравнения, приведенного в условии задачи:


\begin{equation*}
\frac{1}{p} \frac{\mathrm{~d} p}{\mathrm{~d} T}=\frac{a}{T^{2}}-b \frac{1}{T} . \tag{2}
\end{equation*}


Совместное решение уравнений (1) и (2) дает

\[
q=T V \frac{\mathrm{~d} p}{\mathrm{~d} T}=R(a-b T)
\]

5.5. Водяной пар, заполняющий пространство под поршнем в цилиндре, сжимают так, что он все время остается насыщенным, находясь на грани конденсации. Полагая для простоты, что удельная теплота парообразования равна \(q\) и не зависит от температуры \(T\), найти молярную теплоемкость \(C\) пара в данном процессе как функцию \(T\). Пар считать идеальным газом.

P е ш е н и е. Согласно (1.13) прежде всего найдем, что


\begin{equation*}
p \frac{\mathrm{~d} V}{\mathrm{~d} T}=p \frac{\mathrm{~d}}{\mathrm{~d} T}\left(\frac{R T}{p}\right)=R-\frac{R T}{p} \frac{\mathrm{~d} p}{\mathrm{~d} T} \tag{1}
\end{equation*}


Следовательно, теплоемкость \(C\) есть


\begin{equation*}
C=C_{V}+R-\frac{R T}{p} \frac{\mathrm{~d} p}{\mathrm{~d} T} \tag{2}
\end{equation*}


Указанный в условии задачи процесс на диаграмме \(p-T\) идет по кривой парообразования. Для него справедливо уравнение Кла-пейрона-Клаузиуса:


\begin{equation*}
\frac{\mathrm{d} p}{\mathrm{~d} T} \approx \frac{q}{T V_{\Pi}^{\prime}}=\frac{M q}{T V_{\Pi}} \tag{3}
\end{equation*}


где \(M\) - молярная масса, \(V_{\text {п }}\) - молярный объем пара (объемом жидкости мы пренебрегаем).\\
Подстановка (3) в (2) с учетом идеальности газа дает

\[
C=C_{p}-M q / T
\]

5.6. Капиллярные явления. В дне стеклянного сосуда со ртутью имеется малое круглое отверстие радиуса \(r\). При какой толщине слоя ртути она не будет вытекать через это отверстие?

Решение. Из формулы (5.6) следует, что максимальная толщина слоя ртути возможна при условии, что радиус кривизны мени-

\begin{figure}[h]
\begin{center}
  \includegraphics[alt={},max width=\textwidth]{bd023a69-cc46-4b6f-a48e-cc548fa9259d-164_308_409_1429_147}
\captionsetup{labelformat=empty}
\caption{Рис. 5.23}
\end{center}
\end{figure}

ска \(R=\) мин. Из рис. 5.23 видно, что это будет при условии \(R=r\). В этом случае поверхностное натяжение \(\alpha\) направлено вертикально вверх, и соответствующая суммарная сила будет максимальной. Таким образом

\[
h \leqslant 2 \alpha / \rho g r,
\]

где \(\rho\) - плотность ртути.\\
5.7. На мыльном пузыре радиуса \(a\) «сидит» пузырь радиуса \(b\). Имея в виду, что \(b<a\), найти радиус \(R\) кривизны пленки, их разделяющей (рис. 5.24). Каковы углы между пленками в месте их соприкосновения?

Решение. Запишем выражение для избыточного давления в обоих пузы-

\begin{figure}[h]
\begin{center}
  \includegraphics[alt={},max width=\textwidth]{bd023a69-cc46-4b6f-a48e-cc548fa9259d-165_267_379_141_775}
\captionsetup{labelformat=empty}
\caption{Рис. 5.24}
\end{center}
\end{figure}

рях:


\begin{equation*}
\Delta p_{a}=2 \frac{2 \alpha}{a}, \quad \Delta p_{b}=2 \frac{2 \alpha}{b}, \tag{1}
\end{equation*}


где коэффициент 2 учитывает тот факт, что пленка имеет два поверхностных слоя. Из условия равновесия на границе раздела обоих пузырей следует, что


\begin{equation*}
2 \frac{2 \alpha}{R}=\Delta p_{b}-\Delta p_{a} . \tag{2}
\end{equation*}


Совместно решая уравнения (1) и (2), находим

\[
R=\frac{a b}{a-b} .
\]

Далее, сумма трех сил поверхностного натяжения при равновесии должна равняться нулю (рис. 5.25), откуда \(\theta=120^{\circ}\).

\begin{figure}[h]
\begin{center}
  \includegraphics[alt={},max width=\textwidth]{bd023a69-cc46-4b6f-a48e-cc548fa9259d-165_330_286_875_864}
\captionsetup{labelformat=empty}
\caption{Рис. 5.25}
\end{center}
\end{figure}

5.8. Вертикальный капилляр привели в соприкосновение с поверхностью воды. Какое количество тепла выделится при поднятии воды по капилляру? Смачивание считать полным, поверхностное натяжение равно \(\alpha\).

Решение. Работа \(A\), совершаемая силами поверхностного натяжения при поднятии жидкости, идет на сообщение ей потенциальной и кинетической энергии:

\[
A=\Delta \Pi+K .
\]

Если бы не было сил трения, уровень жидкости в капилляре совершал бы гармонические колебания около равновесного положения. Благодаря трению, кинетическая энергия \(K\) переходит во внутреннюю энергию, т. е. выделяется тепло \(Q=K\).

Учитывая формулу (5.11), запишем выражение для работы \(A\) :

\[
A=\alpha \cdot 2 \pi r h=4 \pi \alpha^{2} / \rho g .
\]

Приращение же потенциальной энергии

\[
\Delta \Pi=m g h / 2=2 \pi \alpha^{2} / \rho g,
\]

где \(m=\pi r^{2} h \rho\).\\
В результате получим \(Q=A-\Delta \Pi=2 \pi \alpha^{2} / \rho g\).\\
5.9. Получить формулу (5.11) для \(h\), пользуясь энергетическими соображениями и не делая каких-либо предположений о форме мениска.

\begin{figure}[h]
\begin{center}
  \includegraphics[alt={},max width=\textwidth]{bd023a69-cc46-4b6f-a48e-cc548fa9259d-166_469_504_613_135}
\captionsetup{labelformat=empty}
\caption{Рис. 5.26}
\end{center}
\end{figure}

Решени е. Равновесное положение мениска должно соответствовать минимуму энергии \(E\) системы жидкость-капилляр. Эта энергия слагается из поверхностной энергии и потенциальной энергии жидкости в поле тяготения. Мысленно представим себе, что уровень жидкости в капилляре поднялся на \(\mathrm{d} h\) (рис. 5.26, a). При этом поверхность соприкосновения жидкости со стенкой капилляра увеличится на \(2 \pi r \mathrm{~d} h\), и энергия получает приращение


\begin{equation*}
\mathrm{d} E=2 \pi r \mathrm{~d} h\left(\alpha_{\text {тж }}-\alpha_{\text {тг }}\right)+\left(\pi r^{2} \mathrm{~d} h \rho\right) g h, \tag{1}
\end{equation*}


где первое слагаемое - приращение свободной энергии, второе приращение потенциальной энергии столба жидкости. Далее следует учесть, что при равновесии (рис. 5.26, б)


\begin{equation*}
\alpha_{\text {тГ }}=\alpha_{\text {тж }}+\alpha_{\text {жг }} \cos \theta, \tag{2}
\end{equation*}


и вблизи равновесия выражение (1) равно нулю, \(\mathrm{d} E=0\). В результате получим


\begin{equation*}
2 \pi r \alpha_{\text {ж }} \cos \theta=\left(\rho \pi r^{2} h\right) g . \tag{3}
\end{equation*}


Отсюда и следует формула (5.11).\\
5.10. Теплота образования поверхностного слоя. Рассмотрев цикл Карно для пленки жидкости, показать, что при изотермическом процессе теплота, необходимая для образования единицы пло-

щади поверхностного слоя

\[
q=-T \frac{\mathrm{~d} \alpha}{\mathrm{~d} T}
\]

где \(\mathrm{d} \alpha / \mathrm{d} T\) - производная поверхностного натяжения по температуре.

Решение. Проведем с пленкой цикл Карно, у которого температура нагревателя \(T_{1}\) отличается на очень малое \(\delta T\) от температуры \(T_{2}\) холодильника. Этот цикл изображен на диаграмме \(\alpha-S\) (рис. 5.27), где \(S\) - площадь пленки. Здесь учтено, что с ростом температуры \(\alpha\) уменьшается, поэтому на рисунке цикл Карно выглядит как «перевернутый». Но ход рассуждений от этого не меняется.

\begin{figure}[h]
\begin{center}
  \includegraphics[alt={},max width=\textwidth]{bd023a69-cc46-4b6f-a48e-cc548fa9259d-167_347_444_433_718}
\captionsetup{labelformat=empty}
\caption{Рис. 5.27}
\end{center}
\end{figure}

При растяжении пленке сообщается теплота


\begin{equation*}
Q_{1}=q \cdot \Delta S \tag{1}
\end{equation*}


За цикл пленка совершает работу \(A=A_{12}+A_{34}\). Учитывая, что в изотермическом процессе работа равна убыли свободной энергии, т. е. \(A=-\Delta F=-\alpha \Delta S\), запишем выражение для работы как


\begin{equation*}
\mathrm{d}^{\prime} A=-\alpha_{1} \Delta S-\alpha_{2}(-\Delta S)=\left(\alpha_{2}-\alpha_{1}\right) \Delta S=\mathrm{d} \alpha \cdot \Delta S \tag{2}
\end{equation*}


По теореме Карно, учитывая (1) и (2), получим:

\[
\frac{\mathrm{d}^{\prime} A}{Q_{1}}=\frac{T_{1}-T_{2}}{T_{1}}=\frac{-\mathrm{d} T}{T}=\frac{\mathrm{d} \alpha \cdot \Delta S}{q \cdot \Delta S} .
\]

Из последнего равенства приходим к выражению в тексте задачи.\\
5.11. Плазменные колебания. Под действием некоторой причины электронная компонента плазмы, имеющая форму плоского слоя, сместилась на некоторое расстояние \(x\) перпендикулярно слою (рис. 5.28). Вследствие этого возникли поверхностные заряды (как на плоском конденсаторе) и соответствующая возвращающая сила, что привело к возбуждению плазменных колебаний. Найти их частоту \(\omega\), если концентрация электронов равна \(n\).

\begin{figure}[h]
\begin{center}
  \includegraphics[alt={},max width=\textwidth]{bd023a69-cc46-4b6f-a48e-cc548fa9259d-167_289_310_1479_852}
\captionsetup{labelformat=empty}
\caption{Рис. 5.28}
\end{center}
\end{figure}

Решение. На электронную компоненту плазмы в этом случае будет действовать сила \(\mathbf{F}=q \mathbf{E}\), где \(q\) - суммарный заряд электронов ( \(q<0\) ). Если суммарная масса этих электронов равна \(m\), то согласно основному уравнению динамики


\begin{equation*}
m \ddot{x}=q E_{x}, \tag{1}
\end{equation*}


где \(E_{x}\) - проекция напряженности электрического поля, создаваемого зарядами на поверхности слоя. Известно, что, как и в конденсаторе с поверхностной плотностью заряда \(\sigma\),


\begin{equation*}
E_{x}=\frac{\sigma}{\varepsilon_{0}}=\frac{e n x}{\varepsilon_{0}}, \tag{2}
\end{equation*}


где \(e\) - элементарный заряд, \(n\) - концентрация электронов, \(n x\) - число электронов на единицу площади заряженного поверхностного слоя. Заметим, что в рассматриваемый момент (см. рис. 5.28) \(E_{x}>0\) и \(x>0\). Подставим (2) в (1) и учтем, что \(m=N m_{e}\) и \(q=N(-e)\). Тогда (1) можно преобразовать к виду

\[
\ddot{x}+\frac{n e^{2}}{\varepsilon_{0} m_{e}} x=0 .
\]

Таким образом, мы пришли к уравнению гармонических колебаний с циклической частотой

\[
\omega=\sqrt{n e^{2} / \varepsilon_{0} m_{e}} .
\]

\section*{Глава 6}
\section*{Неравновесные макросистемы}
В предыдущих главах основное внимание было уделено равновесным макросистемам. Однако, строго говоря, равновесные системы - это идеализация. Повседневно нам приходится встречаться только с неравновесными системами*. Физика неравновесных систем развивается, имея большие перспективы. Мы не будем затрагивать общие вопросы этой области науки. Ограничимся рассмотрением лишь нескольких случаев, представляющих особый практический интерес. Это неравновесные системы, с которыми связаны создание лазеров, а также потоки некоторых физических величин (явления переноса).

\section*{§ 6.1. Инверсная среда. Лазеры}
Вынужденное (индуцированное) излучение. Первоначально считалось, что между энергетическими уровнями атомов происходят два вида переходов: спонтанные (самопроизвольные) с более высоких на более низкие уровни и те, которые происходят под действием излучения (вынужденные) с более низких на более высокие уровни. Переходы первого рода приводят к спонтанному испусканию фотонов, переходы второго рода обусловливают поглощение излучения веществом.

Эйнштейн (1918) пришел к выводу, что этих двух видов излучения недостаточно для объяснения равновесия между излучением и веществом. В самом деле, вероятность спонтанных переходов определяется лишь внутренними свойствами атомов и поэтому не зависит от интенсивности падающего излучения, вероятность же вынужденных переходов с более низких уровней зависит как от свойств атомов, так и от интенсивности падающего излучения.


\footnotetext{\begin{itemize}
  \item Системы, очень далекие от равновесных, называют сильно неравновесными. \(К\) их числу, между прочим, относятся биологические (живые) системы, для которых такое состояние является условием существования (за счет активного взаимодействия с окружающей средой).
\end{itemize}
}Для установления равновесия при произвольной интенсивности падающего излучения необходимы переходы, вероятность которых возрастала бы с увеличением интенсивности излучения, т. е. переходы, связанные с испусканием фотонов под действием излучения. Возникающее при таких переходах излучение было названо вынужденным или индуцированным. Это было весьма важное открытие, которое, увы, не сразу было оценено по достоинству.

Для упрощения дальнейших рассуждений рассмотрим два энергетических уровня атомов (молекул) среды, между которыми, по Эйнштейну, возможны три типа процессов: спонтанное излучение, поглощение и вынужденное (индуцированное) излу-

\begin{figure}[h]
\begin{center}
  \includegraphics[alt={},max width=\textwidth]{bd023a69-cc46-4b6f-a48e-cc548fa9259d-170_193_399_665_92}
\captionsetup{labelformat=empty}
\caption{Рис. 6.1}
\end{center}
\end{figure}

чение (рис. 6.1). На рисунке они отмечены как cn, погл и инд. Далее будем считать (это не принципиально), что кратность вырождения этих уровней \(g_{1}=g_{2}=1\), т. е. уровни невырожденные.\\
Теперь введем «вероятности» переходов \(P\), понимая под этим число переходов ежесекундно в расчете на один атом:


\begin{equation*}
P_{21}^{\text {инд }}=B_{21} u_{v}, \quad P_{12}^{\text {погл }}=B_{12} u_{v}, \tag{6.1}
\end{equation*}


где \(B_{21}\) и \(B_{12}\) - так называемые коэффициенты Эйнштейна, \(u_{v}\) - спектральная плотность энергии излучения, соответствующая энергии \(h v\) перехода между данными уровнями.

Исходя из термодинамических соображений, Эйнштейн установил, что \(P_{21}^{\text {инд }}=P_{12}^{\text {погл }}\). Это значит, что коэффициенты


\begin{equation*}
B_{21}=B_{12} . \tag{6.2}
\end{equation*}


Обратимся к прохождению излучения через вещество. Cогласно закону Бугера интенсивность \(I\) пучка излучения при этом изменяется с расстоянием \(x\) как


\begin{equation*}
I=I_{0} \mathrm{e}^{-x x}, \tag{6.3}
\end{equation*}


где \(x\) - коэффициент поглощения (другого процесса здесь не предполагалось). Учтем, однако, роль индуцированного излучения и его уникальные свойства:


\begin{enumerate}
  \item его направление точно совпадает с направлением вынуждающего излучения;
  \item это же относится к частоте, фазе и поляризации, т. е. индуцированное излучение когерентно с вынуждающим, причем в высокой степени.
\end{enumerate}

Все это означает, что вынуждающее излучение, проходя через среду, будет не только поглощаться, но и пополняться за счет индуцированного излучения. При прохождении слоя среды толщиной \(\mathrm{d} x\) убыль интенсивности пучка с частотой \(v\) определяется как


\begin{equation*}
-\mathrm{d} I_{v}=P_{12} h v \cdot N_{1} \mathrm{~d} x-P_{21} h v \cdot N_{2} \mathrm{~d} x, \tag{6.4}
\end{equation*}


где первое слагаемое соответствует поглощению, второе - индуцированному излучению, \(N_{1}\) и \(N_{2}\) - концентрации атомов на уровнях 1 и 2 , а их произведения с \(\mathrm{d} x\) - это число таких атомов на единицу площади поверхности слоя. Каждый из этих атомов или поглощает в единицу времени энергию \(P_{12} h v\), или вносит в пучок энергию \(P_{21} h v\).

Поскольку пучок параллельный, \(u_{v}=I_{v} / c\), и мы можем переписать выражение (6.4) с учетом (6.1) так:


\begin{equation*}
-\mathrm{d} I_{v}=\left(N_{1} B_{12}-N_{2} B_{21}\right) \frac{h v}{c} I_{v} \mathrm{~d} x . \tag{6.5}
\end{equation*}


Интегрируя, получаем


\begin{equation*}
I_{v}=I_{0} \mathrm{e}^{-x^{\prime} x}, \tag{6.6}
\end{equation*}


где коэффициент


\begin{equation*}
\chi^{\prime}=\left(N_{1} B_{12}-N_{2} B_{21}\right) h v / c . \tag{6.7}
\end{equation*}


Вынеся \(N_{1} B_{12}\) за скобку и учтя, что \(B_{12}=B_{21}\), приходим к выводу, что


\begin{equation*}
x^{\prime} \rightharpoonup\left(1-\frac{N_{2}}{N_{1}}\right) . \tag{6.8}
\end{equation*}


В обычных условиях \(N_{2}<N_{1}\) и даже с учетом индуцированного излучения \(x^{\prime}>0\), т. е. имеет место ослабление пучка.

А что, если среду сделать активной - чтобы число атомов на уровне 2 оказалось больше, чем на уровне 1 ? Тогда вклад в интенсивность за счет индуцированного излучения окажется больше, чем поглощение, коэффициент \(\varkappa^{\prime}\) станет отрицательным и интенсивность пучка (6.6) усилится при прохождении такой среды. Ее называют инверсной. Эта мысль была высказана В. А. Фабрикантом (1940), но на нее не было обращено должного внимания. Создание инверсных систем казалось бесперспективным.

И все же открытие состоялось. В 1954 г. Н. Г. Басов и А. М. Прохоров и независимо от них Ч. Таунс, используя индуцированное излучение, создали квантовый генератор в микроволновом диапазоне, названный мазером. А в 1960 г. Мейман создал лазер - квантовый генератор в оптическом диапазоне (лазер на рубине). В дальнейшем началось стремительное совершенствование этих новых необычных источников света.

Трехуровневая система. Основная трудность в практическом осуществлении данной идеи - это создание инверсной, т. е. резко неравновесной среды. Рассмотрим, как была решена эта проблема на примере трехуровневой системьь.

Для создания среды с инверсной заселенностью уровней необходимо затрачивать энергию на преодоление процессов, восстанавливающих равновесное состояние среды. Это делается с помощью так называемой накачки. Разработаны различные способы накачки, один из которых - световой - мы и рассмотрим. Именно он и был использован в рубиновом лазере. Здесь активной средой является цилиндрический монокристалл рубина, представляющий собой окись алюминия \(\mathrm{Al}_{2} \mathrm{O}_{3}\), в которой часть атомов Al замещена атомами хрома Cr.

Энергию накачки в этом лазере создают с помощью газораз-

\begin{figure}[h]
\begin{center}
  \includegraphics[alt={},max width=\textwidth]{bd023a69-cc46-4b6f-a48e-cc548fa9259d-172_285_413_1476_68}
\captionsetup{labelformat=empty}
\caption{Рис. 6.2}
\end{center}
\end{figure}

рядных ламп, работающих в импульсном режиме. Длительность вспышки \(\sim 10^{-3}\) с, а мощность накачки в одном импульсе превышает десятки мегаватт. До вспышки атомы Cr находятся в основном состоянии с энергией \(\mathscr{E}_{1}\) (рис. 6.2). В результате накачки атомы Cr переходят в возбужденное состоя-

ние с энергией \(\mathscr{E}_{2}\). Время жизни атомов \(\operatorname{Cr}\) на уровне \(\mathscr{E}_{2}\) меньше \(10^{-7}\) с. При этом атомы \(\operatorname{Cr}\) переходят не в основное состояние \(\mathscr{E}_{1}\) (вероятность такого перехода весьма мала), а в состояние с энергией \(\mathscr{E}_{3}\). Такой переход происходит без испускания фотонов, а избыток энергии передается непосредственно от атома Cr кристаллической решетке - на возбуждение ее колебаний.

Уровень \(\mathscr{E}_{3}\) является метастабильным (это существенно). Время жизни атомов Cr в этом состоянии сравнительно большое - порядка \(10^{-3}\) с. В течение этого времени среда пребывает в активном состоянии*. Так образуется инверсия заселенностей уровней \(\mathscr{E}_{3}\) и \(\mathscr{E}_{1}\). Теперь достаточно, чтобы в результате спонтанного перехода \(\mathscr{E}_{3} \rightarrow \mathscr{E}_{1}\) появился фотон, имеющий направление вдоль оси рубинового цилиндра, и индуцированное излучение приведет к лавинообразному размножению фотонов.

Обратная связь. Генератор не будет действовать без обратной связи, которая заключается в том, что часть усиленного излучения остается в активной среде и подвергается повторному когерентному усилению. В лазерах для этого используют оптические резонаторы - системы из двух зеркал (рис. 6.3). Индуцированное излучение за счет отражений многократно проходит активную среду. Одно из зеркал ( \(3_{2}\) ) делают слабо пропускающим для вывода части излучения наружу, в результате чего и

\begin{figure}[h]
\begin{center}
  \includegraphics[alt={},max width=\textwidth]{bd023a69-cc46-4b6f-a48e-cc548fa9259d-173_140_558_1012_580}
\captionsetup{labelformat=empty}
\caption{Рис. 6.3}
\end{center}
\end{figure}

формируется лазерный пучок.

Лазер начнет генерировать, если усиление излучения при одном проходе между зеркалами (туда и обратно) будет превосходить потери энергии из-за излучения, вышедшего за один проход через зеркало \(3_{2}\). Это усиление не может быть меньше некоторого значения, называемого пороговым. Последнее в значительной степени зависит от степени пропускания зеркала \(3_{2}\) : чем она больше, тем больше должно быть и пороговое усиление активной среды.


ние с энергией E2. Время жизни атомов Cr на уровне E2 меньше 10–7 с. При этом атомы Cr переходят не в основное состояние E1 (вероятность такого перехода весьма мала), а в состояние с энергией E3. Такой переход происходит без испускания фотонов, а избыток энергии передается непосредственно от атома Сr кристаллической решетке — на возбуждение ее колебаний.

Уровень E3 является метастабильным (это сущ


\footnotetext{\begin{itemize}
  \item Поскольку такая среда является заведомо неравновесной, распределение Больцмана (2.21) к ней не применимо. Но формально его можно применять и в этом случае, если ввести понятие отрицательной температуры, \(T<0\), что зачастую и делают. Это позволяет кратко и удобно характеризовать состояние инверсной среды.
\end{itemize}
}Особенности лазерного излучения. Это излучение обладает свойствами, которых нет ни у одного нелазерного источника. Оно отличается высокой степенью монохроматичности, направленности и когерентности. Так, в лучших газовых гелий-неоновых лазерах расходимость пучка лучей достигает теоретического предела \(\delta \theta \sim \lambda / D \sim 20\) угл. сек. Относительная ширина спектральной линии в излучении таких лазеров \(\delta v / v \sim 10^{-11} \div 10^{-12}\). Кроме того, лазеры позволяют создавать излучения, интенсивности которых достигают огромных значений. Они соответствуют эффективной температуре, превышающей температуру Солнца в \(10^{11}-10^{12}\) раз!

В отличие от рубинового лазера, газовые лазеры могут работать не только в импульсном, но и в непрерывном режиме.

К настоящему времени разработано большое количество разнообразных лазеров, и трудно назвать области, где бы они не использовались. Но это требует особого разговора, выходящего за рамки нашей задачи - продемонстрировать идею создания резко неравновесных систем и возможности их использования.

\section*{§ 6.2. Явления переноса}
Выведенная из состояния равновесия, любая макросистема стремится вернуться в равновесное состояние. При этом растет энтропия, значит этот процесс необратим. Нарушение равновесия сопровождается возникновением потоков или частиц, или тепла, или электрического заряда и др. Соответствующие процессы называют явлениями переноса. Все они являются необратимыми.

Предметом нашего внимания будут три явления переноса: диффузия, внутреннее трение и теплопроводность (причем в условиях, когда отклонения от равновесия малы). Наша программа будет состоять из двух частей. Сначала мы приведем эмпирические уравнения этих процессов - они применимы к любым средам (газообразным, жидким и твердым). Затем получим молекулярно-кинетический вывод данных уравнений для газов, который позволит нам раскрыть содержание коэффициентов, характеризующих соответствующие явления.

В дальнейшем придется использовать понятие потока той или иной физической величины через интересующую нас по-

верхность \(S\). Напомним, поток - величина скалярная и алгебраическая. Его знак зависит от выбора положительного «направления»: с одной стороны поверхности \(S\) к другой или наоборот. Положительное направление обычно выбирают произвольно (за исключением замкнутых поверхностей, где по соглашению его выбирают наружу области, ограниченной этой поверхностью).

Мы будем рассматривать потоки в основном через плоские поверхности \(S\), перпендикулярные оси \(X\), выбирая положительное «направление» поверхности \(S\) совпадающим с ортом оси \(X\). Если физическая величина будет переноситься через \(S\) в направлении оси \(X\), будем считать соответствующий поток положительным, если же в обратном направлении, то - отрицательным.

Любое явление переноса связано с неодинаковостью в пространстве некоторой величины. Например, поток тепла возникает в случае неодинаковости температуры в разных точках среды. На эту особенность потоков следует обратить внимание. Та же температура - это характеристика системы в целом, а здесь мы говорим, что она разная. Приходится вводить понятие локального равновесия. В состоянии локального равновесия среда в каждой малой части своего объема находится в тепловом равновесии, однако равновесие между различными частями отсутствует.

Под малостью имеют в виду объем, размер которого намного превышает, например, среднее расстояние между соседними молекулами. При этом число частиц в таком объеме должно быть макроскопическим, чтобы можно было применять макроскопические параметры состояния теплового равновесия.

Теперь перейдем к эмпирическим уравнениям процессов переноса.

Диффузия. Так называют взаимопроникновение вещества в различных смесях, обусловленное тепловым движением молекул. Пусть смесь содержит две компоненты с парциальны-

\begin{figure}[h]
\begin{center}
  \includegraphics[alt={},max width=\textwidth]{bd023a69-cc46-4b6f-a48e-cc548fa9259d-175_381_514_1405_645}
\captionsetup{labelformat=empty}
\caption{Рис. 6.4}
\end{center}
\end{figure}

дой компоненты стремится выравняться, возникают потоки массы обеих компонент, направленные в сторону уменьшения их плотностей. Экспериментально было установлено выражение для плотности потока массы \(i\)-й компоненты:


\begin{equation*}
j_{M_{i}}=-D \frac{\partial \rho_{i}}{\partial x}, \kappa \Gamma /\left(\mathrm{c} \cdot \mathrm{M}^{2}\right), \tag{6.9}
\end{equation*}


где \(D\) - коэффициент диффузии. Знак минус обусловлен тем, что поток \(i\)-й компоненты противоположен производной \(\partial \rho / \partial x\) - ее называют градиентом плотности (см. рис. 6.4).

Внутреннее трение. Из механики известно, что сила трения между двумя слоями жидкости или газа, отнесенная к единице площади поверхности раздела слоев, равна


\begin{equation*}
f=\eta\left|\frac{\partial u}{\partial x}\right|, \mathrm{H} / \mathrm{m}^{2}, \tag{6.10}
\end{equation*}


где \(\eta\) - коэффициент вязкости (вязкость), производная \(\partial u / \partial x\) - градиент скорости - характеризует степень изменения скорости жидкости или газа в направлении оси \(X\), перпендикулярном направлению движения слоев.

Согласно 2 -му закону Ньютона взаимодействие двух слоев с силой \(f\) можно рассматривать как процесс передачи в единицу времени импульса. Тогда (6.10) можно представить как


\begin{equation*}
j_{p}=-\eta \frac{\partial u}{\partial x}, \mathrm{H} / \mathrm{m}^{2}, \tag{6.11}
\end{equation*}


где \(j_{p}\) - импульс, передаваемый ежесекундно от слоя к слою

\begin{figure}[h]
\begin{center}
  \includegraphics[alt={},max width=\textwidth]{bd023a69-cc46-4b6f-a48e-cc548fa9259d-176_394_459_1379_68}
\captionsetup{labelformat=empty}
\caption{Рис. 6.5}
\end{center}
\end{figure}

через единицу площади поверхности, т. е. плотность потока импульса. Знак минус обусловлен тем, что поток импульса противоположен по направлению градиенту \(\partial u / \partial x\) (рис. 6.5). На рисунке показаны силы, действующие в плоскости площадки \(S\) : левая \(f\) - сила, с которой действуют слои справа от площадки \(S\) (они движутся в дан-

ном случае быстрее), правая \(f\) - сила, с которой действуют слои слева от \(S\). Эти силы взаимно противоположны и одинаковы по модулю. Вопрос, куда действует сила в плоскости \(S\), не имеет смысла, пока не указано, со стороны каких слоев на какие.

Теплопроводность. Опыт показывает, что если в среде создать вдоль оси градиент температуры \(\partial T / \partial x\), то возникает поток тепла, плотность которого


\begin{equation*}
j_{Q}=-\varkappa \frac{\partial T}{\partial x}, \mathrm{Br} / \mathrm{m}^{2}, \tag{6.12}
\end{equation*}


где \(x\) - коэффициент теплопроводности (теплопроводность). Знак минус стоит по той же причине: плотность потока противоположна по направлению градиенту \(\partial T / \partial x\) (рис. 6.6).

\begin{figure}[h]
\begin{center}
  \includegraphics[alt={},max width=\textwidth]{bd023a69-cc46-4b6f-a48e-cc548fa9259d-177_356_416_605_728}
\captionsetup{labelformat=empty}
\caption{Рис. 6.6}
\end{center}
\end{figure}

Пример. Один конец стержня, заключенного в теплоизолированную оболочку, поддерживается при температуре \(T_{1}\), а другой конец при \(T_{2}\). Сам стержень состоит из двух частей, длины которых \(l_{1}\) и \(l_{2}\) и теплопроводности \(\varkappa_{1}\) и \(\varkappa_{2}\). Найдем температуру \(T\) поверхности соприкосновения этих частей стержня.

Ясно, что поток тепла в данном случае должен быть одинаков в любом сечении стержня. Отсюда следует, что


\begin{equation*}
\varkappa_{1}\left(\frac{\partial T}{\partial x}\right)_{1}=\varkappa_{2}\left(\frac{\partial T}{\partial x}\right)_{2} . \tag{1}
\end{equation*}


Поскольку стержень предполагается цилиндрическим, то температура в каждой его части меняется линейно с координатой \(x\), и вместо производной в (1) можно записать отношение разности температур на концах каждой части стержня к ее длине. Тогда (1) примет вид


\begin{equation*}
\varkappa_{1} \frac{T-T_{1}}{l_{1}}=\varkappa_{2} \frac{T_{2}-T}{l_{2}} . \tag{2}
\end{equation*}


Эта запись сделана в предположении, что \(T_{1}<T<T_{2}\) (можно было предположить и наоборот, это не существенно). Из (2) следует, что

\[
T=\frac{\varkappa_{1} T_{1} / l_{1}+\varkappa_{2} T_{2} / l_{2}}{\varkappa_{1} / l_{1}+\varkappa_{2} / l_{2}} .
\]

В заключение отметим еще раз:


\begin{enumerate}
  \item потоки всех величин являются алгебраическими. Их знак зависит от направления оси \(X\). Достаточно обратить положительное направление этой оси на противоположное, и знак потока изменится;
  \item во всех трех явлениях переноса направления плотностей потоков противоположны градиентам соответствующих величин. Это означает, что потоки всегда направлены в сторону уменьшения величин \(\rho, u, T\), т. е. против их градиентов. Таким образом, для потоков существенны градиенты величин, имеющих тенденцию выравниваться.
\end{enumerate}

Теперь перейдем к решению второй части нашей программы - обоснованию эмпирических законов переноса с молеку-лярно-кинетической точки зрения, причем только для газов.

\section*{Предварительные понятия}
\begin{enumerate}
  \item Эффективный диаметр молекуль. Так называют рассто-
\end{enumerate}

\begin{figure}[h]
\begin{center}
  \includegraphics[alt={},max width=\textwidth]{bd023a69-cc46-4b6f-a48e-cc548fa9259d-178_292_348_1167_68}
\captionsetup{labelformat=empty}
\caption{Рис. 6.7}
\end{center}
\end{figure}

яние \(d\), на которое сближаются центры двух молекул при столкновении (рис. 6.7). Величина \(d\) несколько зависит от энергии молекул, это следует из рис. 1.12, а значит и от температуры. С ее увеличением \(d\) уменьшается. Но в дальнейшем этим мы будем пренебрегать.

Площадь, ограниченная штриховой окружностью на рис. 6.7, называют эффективным сечением б молекулы:


Эта запись сделана в предположении, что Т1 < Т < Т2 (можно было предположить и наоборот, это не существенно). Из (2) следует, что T T l T l l l  k k k k

/ / / / .

В заключение отметим еще раз:

\begin{enumerate}

  \item потоки всех величин являются алгебраическими. Их знак зависит от направления оси X. Достаточно обратить положительное направление этой оси на противоположное, и знак потока измени

\end{enumerate}



\begin{equation*}
\sigma=\pi d^{2} . \tag{6.13}
\end{equation*}


\begin{enumerate}
  \setcounter{enumi}{1}
  \item Средняя длина свободного пробега. За секунду молекула проходит в среднем путь, равный ее средней скорости «υ〉. Если при этом она претерпевает в среднем \(v\) столкновений, то сред-
\end{enumerate}

няя длина свободного пробега молекулы


\begin{equation*}
\lambda=\langle v\rangle / v . \tag{6.14}
\end{equation*}


Чтобы определить \(v\) ,проследим за поведением некоторой дви- жущейся молекулы,используя рис.6.7.Пусть это будет левая молекула,и движется она перпендикулярно плоскости рисунка. Нетрудно сообразить,что она испытает за единицу времени сто- лько столкновений,сколько встретится на ее пути молекул,цен- тры которых окажутся в пределах объема цилиндра радиуса \(d\) . Ясно,конечно,что при каждом столкновении цилиндр будет ис- пытывать «излом».Таким образом,это будет «ломаный» ци- линдр,объем которого \(V\) практически равен произведению сред- него пути молекулы за одну секунду,т.е.«υ〉,на площадь попе- речного сечения цилиндра \(\sigma=\pi d^{2}\) .Заметим,что так можно поступить,если среднее расстояние между «изломами» значите- льно больше диаметра молекулы \(d\) ,т.е.при условии \(\lambda \gg d\) ,что мы и предполагаем.Это позволяет пренебречь частями объема цилиндра,приходящимися на его изломы.

Если все это так,то среднее число \(v\) столкновений молекулы ежесекундно равно произведению объема \(V\) ,равного \(\langle v\rangle \sigma\) ,на концентрацию \(n\) молекул:


\begin{equation*}
v=\langle v\rangle \sigma n, \tag{6.15}
\end{equation*}


и средняя длина свободного пробега(6.14)равна


\begin{equation*}
\lambda=\frac{1}{\sigma n} . \tag{6.16}
\end{equation*}


Более строгое рассмотрение вопроса о числе столкновений \(v\) приводит к необходимости замены средней скорости 〈U〉 на среднюю относительную скорость \(\left\langle v_{\text {отн }}\right\rangle\) сталкивающихся молекул,которая,как показывает расчет,в \(\sqrt{2}\) больше,чем \(\langle υ\rangle\) . Тогда уточненные формулы(6.15)и(6.16)будут иметь вид


\begin{equation*}
v=\sqrt{2}\langle v\rangle \sigma n, \quad \lambda=\frac{1}{\sqrt{2} \sigma n} . \tag{6.17}
\end{equation*}


Эти величины отличаются от приближенных формул(6.15)и (6.16)числовым коэффициентом,близким к единице.

В дальнейшем мы собираемся проводить оценочные расчеты, поэтому множителем \(\sqrt{2}\) будем пренебрегать как не существенным.

Пример. Оценим для азота \(\mathrm{N}_{2}\) при нормальных условиях значение \(\lambda\), среднее расстояние \(l\) между молекулами и число столкновений \(v\) молекулы в секунду.

Сначала найдем концентрацию молекул в этих условиях:

\[
n=\frac{N_{A}}{V_{M}} \approx \frac{6 \cdot 10^{23} \text { моль }^{-1}}{22,4 \cdot 10^{3} \text { см }^{3} / \text { моль }}=2,7 \cdot 10^{19} \mathrm{~cm}^{-3},
\]

где \(V_{M}\) - молярный объем газа. Примем эффективный диаметр молекулы азота равным \(2 \cdot 10^{-8} \mathrm{~cm}\). Тогда согласно (6.16)

\[
\lambda=\frac{1}{\pi\left(2 \cdot 10^{-8}\right)^{2} \cdot 2,7 \cdot 10^{19}}=3 \cdot 10^{-5} \text { см. }
\]

Сравним эту величину со средним расстоянием между молекулами. Из условия \(\langle l\rangle^{3}=1 / n\), получим

\[
\langle l\rangle=1 / \sqrt[3]{n}=1 / \sqrt[3]{2,7 \cdot 10^{19}}=3,3 \cdot 10^{-7} \text { см, }
\]

т. е. \(\lambda\) превышает < \(l\) > на два порядка. Число столкновений ежесекундно согласно (6.15) равно

\[
v=\langle v\rangle \pi d^{2} n \approx 4,8 \cdot 10^{4} \cdot \pi \cdot\left(2 \cdot 10^{-8}\right)^{2} \cdot 2,7 \cdot 10^{19}=1,6 \cdot 10^{9} \mathrm{c}^{-1},
\]

где \(\langle υ\rangle=480 \mathrm{~m} / \mathrm{c}\) (см. стр. 54). Таким образом, число столкновений составляет порядка миллиарда в секунду!

\section*{§ 6.3. Молекулярно-кинетическая интерпретация явлений переноса}
В этом параграфе мы рассмотрим явления переноса в газах с молекулярно-кинетической точки зрения. Соответствующие расчеты будут иметь оценочный характер. Еще раз напомним (об этом уже говорилось ранее), что оценочный подход - это то, с чего обычно начинается создание теории. Главное достоинство такого подхода состоит в простоте и акценте на физической стороне явления, не заслоненной громоздкими вычислениями и преобразованиями. Разумеется, оценочный подход ни в коей мере не может претендовать на получение точных резу-

льтатов, но различие заключается только в числовых коэффициентах.

Итак, будем исходить из предельно упрощенной модели, которой мы уже пользовались ранее и убедились, что она дает неплохие результаты. Повторим: ввиду полной хаотичности теплового движения молекул будем считать, что молекулы движутся по трем направлениям \(X, Y\) и \(Z\), так что на каждое направление в одну сторону плотность потока молекул составляет


\begin{equation*}
j=\frac{1}{6}\langle v\rangle n, \tag{6.18}
\end{equation*}


где \(n\) - концентрация молекул. Эти потоки и являются переносчиками определенных физических величин \(G\). Плотность потока величины \(G\) будем обозначать \(j_{G}\).

Далее будем считать, что через интересующую нас площадку \(S\) молекулы будут переносить то значение величины \(G\), которое они имели на расстоянии \(\lambda\) от площадки \(S\). Т. е. будем предполагать, что последнее соударение молекулы испытывают на этом расстоянии от \(S\).

Теперь перейдем к рассмотрению с помощью этой модели явлений переноса и начнем с вывода общего уравнения переноса, не зависящего от времени.

Общее уравнение переноса. Пусть величина \(G\) характеризует определенное молекулярное свойство, отнесенное \(\kappa\) одной молекуле. Это может быть энергия, импульс, электрический заряд и др. Ясно, что при наличии градиента величины \(G\) должен возникнуть поток в сторону ее уменьшения.

Пусть величина \(G\) меняется только в направлении оси \(X\), например, так, как показано на рис. 6.8. Площадку \(S\) будут пронизывать молекулы, движущиеся во встречных направлениях, их плотности потоков обозначим \(j^{\prime} u j^{\prime \prime}\). Причем это существенно - они должны быть равны друг другу ( \(j^{\prime}=j^{\prime \prime}\) ), чтобы не возникало газодинамических потоков и чтобы все процессы сво-

\begin{figure}[h]
\begin{center}
  \includegraphics[alt={},max width=\textwidth]{bd023a69-cc46-4b6f-a48e-cc548fa9259d-181_400_465_1361_695}
\captionsetup{labelformat=empty}
\caption{Рис. 6.8}
\end{center}
\end{figure}

дились только к переносу величины \(G\). Тогда для результирующей плотности потока величины \(G\) можно (см. рис. 6.8) записать:


\begin{equation*}
j_{G}=j \cdot G^{\prime}-j \cdot G^{\prime \prime}=\frac{1}{6}\langle v\rangle n_{0}\left(G^{\prime}-G^{\prime \prime}\right) . \tag{6.19}
\end{equation*}


Благодаря малости \(\lambda\) разность значений \(G^{\prime \prime}-G^{\prime}\) представим в виде


\begin{equation*}
G^{\prime \prime}-G^{\prime}=\frac{\partial G}{\partial x} 2 \lambda . \tag{6.20}
\end{equation*}


С учетом этой формулы выражение (6.19) запишем так:


\begin{equation*}
j_{G}=-\frac{1}{3}\langle v\rangle \lambda n_{0} \frac{\partial G}{\partial x} . \tag{6.21}
\end{equation*}


Это и есть общее уравнение переноса для любой величины \(G\). Здесь \(n_{0}\) - концентрация молекул, 〈υ〉- их средняя тепловая скорость. Значения этих величин берутся в сечении \(S\).

Применим это уравнение к трем наиболее интересным явлениям переноса, связанным с диффузией, вязкостью и теплопроводностью.

Диффузия. Ограничимся рассмотрением самодиффузии, т. е. процессом перемешивания (взаимопроникновения) молекул одного сорта. Макроскопически самодиффузию наблюдать нельзя: из-за тождественности молекул она не может проявляться ни в одном явлении. Для наблюдения этого процесса часть молекул газа надо как-то «пометить». Практически это можно сделать с помощью так называемых «меченых» атомов: смесь газов берут из двух изотопов одного и того же элемента, один из которых радиоактивен. Тогда процесс диффузии можно наблюдать, регистрируя радиоактивное излучение радиоизотопа. Можно также взять смесь двух различных газов, молекулы которых почти одинаковы по массе и размерам (такие, например, как \(\mathrm{N}_{2}\) и CO ). В этом случае у обеих компонент газа будут одинаковы как средние скорости, так и длины свободного пробега, т. е. \(\langle v\rangle\) и \(\lambda\).

Чтобы отсутствовали газокинетические потоки и перемешивание молекул происходило только за счет диффузии, необхо-

димо (так мы и будем считать), чтобы суммарная концентрация \(n_{0}\) обеих компонент смеси не зависела от координаты в направлении оси \(X\), вдоль которой происходит этот процесс (рис. 6.9).

Пусть концентрация молекул 1-го сорта зависит от коор-


димо (так мы и будем считать), чтобы суммарная концентрация n0 обеих компонент смеси не зависела от координаты в направлении оси X, вдоль которой происходит этот процесс (рис. 6.9).

Пусть концентрация молекул 1-го сорта зависит от координаты x как n1(х). Учитывая, что величина G в уравнении (6.21) есть характеристика переносимого количества, отнесенного к одной молекуле, имеем G = n1/n0, где n0 —


\begin{figure}[h]
\begin{center}
  \includegraphics[alt={},max width=\textwidth]{bd023a69-cc46-4b6f-a48e-cc548fa9259d-183_374_511_141_637}
\captionsetup{labelformat=empty}
\caption{Рис. 6.9}
\end{center}
\end{figure}

\[
G=n_{1} / n_{0},
\]

где \(n_{0}\) - равновесная концентрация (см. рис. 6.9). Тогда уравнение (6.21) в данном случае примет вид


\begin{equation*}
j_{n_{1}}=-\frac{1}{3}\langle v\rangle \lambda \frac{\partial n_{1}}{\partial x} . \tag{6.22}
\end{equation*}


Сравнив это выражение с эмпирической формулой (6.9), находим, что коэффициент самодиффузии


\begin{equation*}
D=\frac{1}{3}\langle v\rangle \lambda . \tag{6.23}
\end{equation*}


Рассуждения, приведшие нас к формуле (6.22), в равной мере справедливы и для другой компоненты смеси. Значит, коэффициент \(D\) одинаков для обеих компонент.

Более строгий расчет приводит к такой же формуле для \(D\), но с несколько большим числовым коэффициентом (в \(1,2 \div 1,5\) раза для разных газов).

Единицей измерения коэффициента \(D\) является \(\mathrm{m}^{2} / \mathrm{c}\). При нормальных условиях коэффициент \(D\) для кислорода и азота в вохдухе имеет порядок \(10^{-5} \mathrm{~m}^{2} / \mathrm{c}\).

Мы ограничимся рассмотрением самодиффузии, поскольку взаимная диффузия разных газов требует весьма громоздких преобразований, но принципиально нового ничего не дает. Как показывают расчеты и измерения, коэффициент взаимной диф-

фузии \(D_{12}\) разных газов при нормальных условиях имеет тот же порядок \(10^{-5} \mathrm{~m}^{2} / \mathrm{c}\).

Вязкость (внутреннее трение). Это явление возникает в тех случаях, когда на хаотическое тепловое движение молекул на-

\begin{figure}[h]
\begin{center}
  \includegraphics[alt={},max width=\textwidth]{bd023a69-cc46-4b6f-a48e-cc548fa9259d-184_370_441_326_70}
\captionsetup{labelformat=empty}
\caption{Рис. 6.10}
\end{center}
\end{figure}


\begin{equation*}
j_{p}=-\frac{1}{3}\langle v\rangle \lambda \rho \frac{\partial u}{\partial x}, \tag{6.24}
\end{equation*}


где \(\rho=m n_{0}\) - плотность газа. Сопоставив это уравнение с эмпирической формулой (6.11), находим выражение для вязкости:


\begin{equation*}
\eta=\frac{1}{3}\langle v\rangle \lambda \rho . \tag{6.25}
\end{equation*}


Более точный расчет дает несколько большее значение для числового коэффициента: не \(1 / 3\), а 0,49 .

Единицей вязкости в СИ является паскаль-секунда (Па•с), а в системе СГС - пуаз (П). Связь между ними: 1 Па•с \(=10\) П.

При нормальных условиях вязкость газов \(\eta \sim 10^{-5}\) Па.с.\\
Теплопроводность. В этом явлении величиной \(G\) в \((6.21)\) является средняя энергия теплового движения, приходящаяся на одну молекулу. Из теоремы о равнораспределении энергии по степеням свободы имеем \(G=(i / 2) k T\), и тогда плотность потока тепла


\begin{equation*}
j_{Q}=-\frac{1}{3}\langle v\rangle \lambda n_{0} \frac{i}{2} k \frac{\partial T}{\partial x} . \tag{6.26}
\end{equation*}


Для упрощения этой формулы введем удельную теплоемкость \(c_{V}\). Для этого обратим внимание на то, что \((i / 2) k\) - это теплоем-

кость при постоянном объеме, рассчитанная на одну молекулу. Произведение данной величины на концентрацию \(n_{0}\) дает теплоемкость единицы массы \(c_{V}\), умноженную на плотность газа \(\rho\). Таким образом, учитывая, что \((i / 2) k \cdot n_{0}=c_{V} \rho\), перепишем (6.26) в виде


\begin{equation*}
j_{Q}=-\frac{1}{3}\langle v\rangle \lambda \rho c_{V} \frac{\partial T}{\partial x} . \tag{6.27}
\end{equation*}


Из сравнения этого выражения с формулой (6.12) видим, что теплопроводность


\begin{equation*}
\varkappa=\frac{1}{3}\langle v\rangle \lambda \rho c_{V}, \tag{6.28}
\end{equation*}


где, повторим, \(c_{V}\) - удельная теплоемкость, отнесенная к единице массы, Дж/(К•кг). Более точные вычисления числового коэффициента в (6.28) представляют большие трудности, но полученные результаты оказываются того же порядка, что и \(1 / 3\).

Единицей теплопроводности является Вт/(м•К).\\
При заданной концентрации \(n_{0}\) теплопроводность зависит в основном от средней скорости 〈υ〉. Из-за этого легкие газы обладают значительно большей теплопроводностью, чем тяжелые, поскольку \(\langle v\rangle \circlearrowleft 1 / \sqrt{m}\). Например, при нормальных условиях кислород имеет теплопроводность \(0,024 \mathrm{Br} /(\mathrm{m} \cdot \mathrm{K})\), а водород 0,176 Вт/(м⋅К).

Анализ коэффициентов переноса. Прежде всего выпишем для удобства сопоставления и анализа все три коэффициента рассмотренных явлений переноса:



\begin{align*}
D & =\frac{1}{3}\langle v\rangle \lambda \\
\eta & =\frac{1}{3}\langle v\rangle \lambda \rho  \tag{6.29}\\
\varkappa & =\frac{1}{3}\langle v\rangle \lambda \rho c_{V} .
\end{align*}


\begin{enumerate}
  \item Определив по эмпирическим формулам коэффициенты \(D\), \(\eta\) и \(\varkappa\), мы имеем возможность с помощью формул (6.29) вычислить \(\lambda\) и диаметр \(d\) молекул. При этом следует иметь в виду, что полученные значения заметно зависят от того, на основании какого коэффициента их вычисляют (поэтому в таблицах это оговаривается).
  \item Все три коэффициента, \(D, \eta\) и \(\varkappa\), с ростом температуры \(T\) увеличиваются, так как \(\langle v\rangle \circlearrowleft \sqrt{T}\).
  \item Поскольку \(\lambda \sim 1 / n\), а \(n \sim \rho\), то как вязкость \(\eta\), так и теплопроводность \(\varkappa\) не зависят от концентрации, а значит и от давления (при неизменной температуре).
\end{enumerate}

На первый взгляд этот вывод кажется странным и в свое время послужил поводом к тому, чтобы подвергнуть сомнению развиваемые молекулярно-кинетические представления. Однако при более внимательном рассмотрении выяснилось, что здесь все в порядке. Действительно, уменьшая давление, мы уменьшаем концентрацию молекул, но при этом одновременно растет \(\lambda\), а значит и различие в значениях величины \(G\), переносимой каждой молекулой в противоположных направлениях. Тем самым «парадокс» был разрешен, и это явилось очередным триумфом молекулярно-кинетической теории.

Все так, и тем не менее, уменьшая давление \(p\), мы обнаруживаем, что с какого-то момента коэффициенты \(\eta\) и \(x\) начинают зависеть от давления. Это связано с тем, что с уменьшением давления растет \(\lambda\) и наступает момент, когда \(\lambda\) становится сравнимым с характерным размером сосуда, в котором находится газ (например, с расстоянием между дисками, колеблющимися относительно друг друга). После этого коэффициенты \(\eta\) и \(\varkappa\) начинают уменьшаться за счет уменьшения плотности \(\rho\), т.е. концентрации \(n\), и мы вступаем в новую область - в область вакуумных явлений.

Уравнения переноса, зависящие от времени. Приведенные выше расчеты и результаты относятся к так называемым стационарным задачам, когда распределение интересующей нас величины \(G\) зависит только от координат. Но процессы переноса (выравнивания величины \(G\) ) зависят и от времени. Это обстоятельство приводит к необходимости решать нестационарные задачи, учитывающие зависимость величины \(G\) как от координат, так и от времени. В качестве примера приведем соответствующее одномерное дифференциальное уравнение для теплопроводности:

\[
\frac{\partial T}{\partial t}=\alpha \frac{\partial^{2} T}{\partial x^{2}} .
\]

Для решения подобных уравнений необходимо знать начальные и граничные условия. Если они заданы и известен коэффициент \(\alpha\), то зада-

ча является чисто математической, ее решение подробно рассматривается в курсе математической физики.

\section*{Ультраразреженные газы (вакуум)}
Когда длина свободного пробега \(\lambda\) превышает характерный размер \(l\) сосуда, говорят, что достигнут вакуум, и газ находится в состоянии ультраразрежения. Под характерным понимают тот размер сосуда (или в сосуде), который определяет интересующее нас явление. Вакуум - понятие относительное: условие \(\lambda>l\) может иметь место в малых порах даже при атмосферном давлении.

Поведение ультраразреженного газа существенно отличается от поведения газов при обычных условиях. В условиях вакуума теряет смысл говорить о давлении одной части газа на другую, а это значит, что внутреннее трение отсутствует. Имеет смысл говорить только о силе трения, испытываемого движущимся телом.

Теряет также смысл введенное нами понятие теплопроводности, как процесса передачи энергии от одного слоя газа к другому (тепловая диффузия). Следует говорить только о теплообмене между телами.

Из сказанного следует, что при выводе уравнений переноса величины \(G\) (импульса или энергии) надо остановиться на формуле (6.19), где под \(G^{\prime}\) и \(G^{\prime \prime}\) следует понимать значения величины \(G\), соответствующие двум телам. Формула (6.20) в случае ультраразреженного газа уже не имеет смысла.

Теперь должно быть понятным, почему в опыте с крутильными колебаниями диска 1 (рис. 6.11, где диск 2 неподвижен) в сосуде \(C\) при уменьшении давления в нем сначала коэффициент затухания не меняется, а начиная с некоторого значения давления начинает уменьшаться. Это «граничное» давление соответствует концентрации \(n\) молекул, при которой их длина свободного пробега \(\lambda\) становится равной характерному размеру - расстоянию между дисками, и коэффициент \(\eta\) уменьшается из-за уменьшения концентрации \(n\).

\begin{figure}[h]
\begin{center}
  \includegraphics[alt={},max width=\textwidth]{bd023a69-cc46-4b6f-a48e-cc548fa9259d-187_450_368_1319_792}
\captionsetup{labelformat=empty}
\caption{Рис. 6.11}
\end{center}
\end{figure}

Становится также понятным, почему и насколько надо эвакуировать объем между стенками сосуда Дюара, чтобы как можно более снизить коэффициент теплопроводности \(\varkappa\). Здесь характерным размером является расстояние между стенками сосуда.

\section*{Задачи}
6.1. Диффузия. Два сосуда 1 и 2 одинакового объема \(V\) соединены трубкой большой длины \(l\) и малой площади \(S\) поперечного сечения. В начальный момент \(t=0\) в одном сосуде имеется газ 1 с концентрацией \(n_{10}\), а в другом - газ 2 с концентрацией \(n_{20}\). Давления и температуры в обоих сосудах одинаковы. Найти концентрацию \(n_{1}^{(1)}\) в первом сосуде как функцию времени \(t\), считая коэффициент диффузии \(D\) известным.

Решение. Одинаковость давлений и температур в обоих сосудах означает, что концентрации газов 1 и 2 тоже одинаковы, и процесс смешения газов будет происходить только за счет диффузии. В момент \(t\) из условия сохранения частиц газа 1 имеем


\begin{equation*}
n_{1}^{(1)}+n_{1}^{(2)}=n_{10} . \tag{1}
\end{equation*}


Поток частиц газа 1 равен убыли этих частиц в сосуде 1 за единицу времени, т. е.


\begin{equation*}
j_{1} S=-\partial N_{1} / \partial t \tag{2}
\end{equation*}


Преобразуем это выражение с помощью формулы (6.9) для плотности потока частиц:


\begin{equation*}
D \frac{n_{1}^{(1)}-n_{1}^{(2)}}{l} S=-V \frac{\partial n_{1}^{(1)}}{\partial t} . \tag{3}
\end{equation*}


Чтобы исключить \(n_{1}^{(2)}\), представим разность концентраций газа 1 на концах трубки с помощью (1) как


\begin{equation*}
n_{1}^{(1)}-n_{1}^{(2)}=n_{1}^{(1)}-\left(n_{10}-n_{1}^{(1)}\right)=2 n_{1}^{(1)}-n_{10} . \tag{4}
\end{equation*}


Кроме того, введем для упрощения преобразований обозначение \(\alpha=2 D S / V l\). Тогда уравнение (3) примет вид

\[
\frac{\partial n_{1}^{(1)}}{\partial t}+\alpha n_{1}^{(1)}=\frac{\alpha}{2} n_{10} .
\]

Это неоднородное дифференциальное уравнение. Его решение

\[
n_{1}^{(1)}=\left(n_{10} / 2\right)\left(1+\mathrm{e}^{-\alpha t}\right) .
\]

Видно, что при \(t \rightarrow \infty\) газ 1 распределится поровну между обоими сосудами.\\
6.2. Вязкость. Два одинаковых параллельных диска, оси которых совпадают, расположены на расстоянии \(h\) друг от друга. Радиус каждого диска равен \(R\), причем \(R \gg h\). Один диск вращают с небольшой угловой скоростью \(\omega\), другой диск неподвижен. Найти момент сил трения, действующий на каждый диск, если вязкость среды между дисками равна \(\eta\).\\
Решение. В условиях задачи неявно предполагается, что краевыми эффектами можно пренебречь, а малость угловой скорости ω означает, что упорядоченное движение слоев среды между дисками имеет ламинарный характер (без турбулентности). Кроме того, не сказано, относительно какой оси надо найти момент сил трения. Но это не существенно, поскольку в данном случае результирующая сила, действующая на

\begin{figure}[h]
\begin{center}
  \includegraphics[alt={},max width=\textwidth]{bd023a69-cc46-4b6f-a48e-cc548fa9259d-189_301_367_690_795}
\captionsetup{labelformat=empty}
\caption{Рис. 6.12}
\end{center}
\end{figure}

каждый диск, равна нулю, и поэтому (как доказывается в механике) искомый момент не зависит от выбора оси. Ради простоты будем проводить расчет относительно оси дисков.

Сначала найдем момент силы, действующий на малый элемент \(\mathrm{d} S\) (рис. 6.12), площадь которого \(\mathrm{d} S=r \mathrm{~d} \varphi \cdot \mathrm{~d} r\) :


\begin{equation*}
\delta N=r \mathrm{~d} F_{\text {тр }}=r \eta \frac{\partial v}{\partial z} r \mathrm{~d} \varphi \mathrm{~d} r . \tag{1}
\end{equation*}


В нашем случае производная \(\partial v / \partial r=v / h=\omega r / h\). Учитывая это, найдем момент сил, действующих на бесконечно тонкое кольцо с радиусами \(r\) и \(r+\mathrm{d} r\) :


\begin{equation*}
\mathrm{d} N=\int_{\varphi=0}^{2 \pi} \delta N=2 \pi \eta \frac{\omega}{h} r^{3} \mathrm{~d} r . \tag{2}
\end{equation*}


Проинтегрировав последнее выражение по \(r\) от 0 до \(R\), получим:

\[
N=\pi \eta \omega R^{4} / 2 h
\]

6.3. Согласно формуле Пуазейля поток жидкости плотностью \(\rho\) через поперечное сечение трубы равен

\[
\mu=\frac{\pi \rho a^{4}}{8 \eta} \frac{p_{1}-p_{2}}{l}, \text { кг/с, }
\]

где \(a\) и \(l\) - радиус и длина трубы, \(p_{1}-p_{2}\) - разность давлений на ее торцах, причем \(p_{1}>p_{2}\). Воспользовавшись этой формулой, определить вязкость \(\eta\) газа. Известны величины \(\mu, a, p_{1}, p_{2}, l\), молярная масса \(M\) и температура \(T\) газа. Газ считать идеальным.\\
Р е ш е н и е. Пусть ось \(X\) направлена вдоль оси трубы. Тогда в слое толщины \(\mathrm{d} x\) можно считать газ несжимаемым и применить к нему формулу Пуазейля:


\begin{equation*}
\mu=\frac{\pi \rho a^{4}}{8 \eta}\left(-\frac{\partial p}{\partial x}\right) . \tag{1}
\end{equation*}


Перепишем это выражение, имея в виду, что для идеального газа плотность \(\rho=p M / R T\),


\begin{equation*}
\mu=-A p \frac{\partial p}{\partial x}, \tag{2}
\end{equation*}


где \(A=π a^{4} M / 8 \eta R T\). Разделив в (2) переменные \(p\) и \(x\), проинтегрируем полученное уравнение и найдем:


\begin{equation*}
\mu l=-A\left(p_{2}^{2}-p_{1}^{2}\right) / 2 . \tag{3}
\end{equation*}


Остается раскрыть коэффициент \(A\), и мы получим

\[
\eta=\frac{\pi a^{4} M}{16 \mu R T} \frac{p_{1}^{2}-p_{2}^{2}}{l} .
\]

6.4. Теплопроводность. Стержень длины \(l\) с теплоизолирующей боковой поверхностью состоит из материала, теплопроводность которого изменяется с температурой как \(\varkappa=\alpha / T\), где \(\alpha\) - постоянная. Торцы стержня поддерживаются при температурах \(T_{1}\) и \(T_{2}\). Найти зависимость \(T(x)\), где \(x\) - расстояние от торца с температурой \(T_{1}\).\\
Р е ш ен и е. Плотность потока тепла в стационарных условиях не должна зависеть от \(x\), т.е.


\begin{equation*}
\frac{\alpha}{T} \frac{\partial T}{\partial x}=C, \tag{1}
\end{equation*}


где \(C\) - константа. Из этого уравнения следует


\begin{equation*}
\frac{\mathrm{d} T}{T}=\frac{C}{\alpha} \mathrm{~d} x, \quad \ln \frac{T}{T_{1}}=\frac{C}{\alpha} x \tag{2}
\end{equation*}


Для нахождения константы \(C\) учтем, что при \(x=l\) температура \(T=T_{2}\) и \(\ln \left(T_{2} / T_{1}\right)=(C / \alpha) l\), откуда

\[
C=\frac{\alpha}{l} \ln \frac{T_{2}}{T_{1}}
\]

В результате из второй формулы (2) получим

\[
T=T_{1} \mathrm{e}^{C x / \alpha}=T_{1} \exp \left(\frac{x}{l} \ln \frac{T_{2}}{T_{1}}\right)=T_{1}\left(T_{2} / T_{1}\right)^{x / l} .
\]

6.5. Два металлических тела 1 и 2 с теплоемкостями \(C_{1}\) и \(C_{2}\) соединены между собой однородным стержнем длины \(l\) с площадью поперечного сечения \(S\) и достаточно малой теплопроводностыо \(\varkappa\). Вся система теплоизолирована. В момент \(t=0\) разность температур между телами 1 и 2 равна \(\left(T_{2}-T_{1}\right)_{0}\). Пренебрегая теплоемкостью стержня, найти разность температур \(T_{2}-T_{1}\) между этими телами в зависимости от времени \(t\).\\
Решение. Исходим из того, что количество тепла, передаваемого от более нагретого тела к менее нагретому за время \(\mathrm{d} t\), определяется как


\begin{equation*}
j_{Q} S \mathrm{~d} t=-C_{2} \mathrm{~d} T_{2}, \quad j_{Q} S \mathrm{~d} t=C_{1} \mathrm{~d} T_{1} \tag{1}
\end{equation*}


где в правой части равенств записано тепло, отдаваемое за время \(\mathrm{d} t\) более нагретым телом ( \(-C_{2} \mathrm{~d} T_{2}\) ), и тепло, получаемое за это же время менее нагретым телом ( \(C_{1} \mathrm{~d} T_{1}\) ). Принимая во внимание формулу (6.12), согласно которой в нашем случае \(j_{Q}=\varkappa\left(T_{2}-T_{1}\right) / l\), разделим первое и второе равенства (1) соответственно на \(C_{2}\) и \(C_{1}\), а затем сложим отдельно левые и правые части полученных выражений. В результате получим


\begin{equation*}
\varkappa \frac{T_{2}-T_{1}}{l} S\left(\frac{1}{C_{1}}+\frac{1}{C_{2}}\right) \mathrm{d} t=-\mathrm{d}\left(T_{2}-T_{1}\right) \tag{2}
\end{equation*}


Введем обозначения \(\tau=T_{2}-T_{1}\) и \(\alpha=\varkappa S\left(1 / C_{1}+1 / C_{2}\right) / l\), тогда (2) примет вид \(\alpha \tau \mathrm{d} t=-\mathrm{d} \tau\), или

\[
\frac{\mathrm{d} \tau}{\tau}=-\alpha \mathrm{d} t
\]

Проинтегрировав это уравнение, получим \(\ln \left(\tau / \tau_{0}\right)=-\alpha t\), или \(\tau=\tau_{0} \mathrm{e}^{-\alpha t}\). Переходя к первоначальным параметрам, имеем

\[
T_{2}-T_{1}=\left(T_{2}-T_{1}\right)_{0} \mathrm{e}^{-\alpha t},
\]

где \(\alpha=\left(1 / C_{1}+1 / C_{2}\right) \varkappa S / l\).\\
6.6. Найти распределение температуры в пространстве между двумя концентрическими сферами с радиусами \(R_{1}\) и \(R_{2}\), заполненном однородным теплопроводящим веществом, если температуры сфер \(T_{1}\) и \(T_{2}\), а \(R_{2}>R_{1}\).

Решение. Запишем поток тепла через промежуточную концентрическую сферу радиуса \(r: \dot{Q}=-\varkappa(\partial T / \partial r) 4 \pi r^{2}\). Эта величина в стационарном случае не зависит от \(r\), поэтому


\begin{equation*}
r^{2}(\partial T / \partial r)=C, \tag{1}
\end{equation*}


где \(C\) - константа. Из (1) следует


\begin{equation*}
\mathrm{d} T=C \frac{\mathrm{~d} r}{r^{2}}, \quad T-T_{1}=C\left(\frac{1}{R_{1}}-\frac{1}{r}\right) . \tag{2}
\end{equation*}


\begin{figure}[h]
\begin{center}
  \includegraphics[alt={},max width=\textwidth]{bd023a69-cc46-4b6f-a48e-cc548fa9259d-192_394_408_907_155}
\captionsetup{labelformat=empty}
\caption{Рис. 6.13}
\end{center}
\end{figure}

Константу \(C\) находим из того условия, что при \(r=R_{2}\) температура \(T=T_{2}\), значит


\begin{equation*}
C=\frac{T_{2}-T_{1}}{1 / R_{1}-1 / R_{2}} . \tag{3}
\end{equation*}


Подстановка (3) в (2) дает искомый результат:

\[
T=T_{1}+\frac{T_{2}-T_{1}}{1 / R_{1}-1 / R_{2}}\left(\frac{1}{R_{1}}-\frac{1}{r}\right) .
\]

Попутно рассмотрим, какой вид имеет зависимость \(T(r)\) в тех случаях, когда \(T_{2}>T_{1}\) и \(T_{2}<T_{1}\).

В первом случае согласно (2) \(C>0\) и \(\partial T / \partial r=C / r^{2}\), т. е. с ростом \(r\) наклон кривой \(T(r)\) уменьшается. Во втором случае \(C<0\), наклон кривой \(T(r)\) «отрицательный» и уменьшается по модулю. Оба случая показаны на рис. 6.13.\\
6.7. Постоянный электрический ток течет по проводу, радиус сечения которого \(R\) и теплопроводность \(\varkappa\). В единице объема провода выделяется тепловая мощность \(w\). Найти распределение температуры в

проводе, если установившаяся температура на его поверхности равна \(T_{0}\).

Р е ш е н и е. Рассмотрим цилиндрический коаксиальный слой \(r\), \(r+\mathrm{d} r\). В расчете на единицу длины этого слоя в него входит поток тепла \(\left(j_{Q} \cdot 2 \pi r\right)_{r}\), выходит \(\left(j_{Q} \cdot 2 \pi r\right)_{r+\mathrm{d} r}\), внутри же слоя выделяется количество тепла \(w 2 \pi r \mathrm{~d} r\). Составим баланс тепла:


\begin{equation*}
\left(r j_{Q}\right)_{r+\mathrm{d} r}-\left(r j_{Q}\right)_{r}=w r \mathrm{~d} r . \tag{1}
\end{equation*}


Это уравнение можно представить как


\begin{equation*}
\mathrm{d}\left(r j_{Q}\right)=w r \mathrm{~d} r, \tag{2}
\end{equation*}


или, учитывая (6.12), - в виде


\begin{equation*}
-\mathrm{d}\left(\varkappa r \frac{\mathrm{~d} T}{\mathrm{~d} r}\right)=w r \mathrm{~d} r \tag{3}
\end{equation*}


Проинтегрировав это выражение, получим



\begin{equation*}
-\varkappa r \frac{\mathrm{~d} T}{\mathrm{~d} r}=w \frac{r^{2}}{2} \tag{4}
\end{equation*}


Разделим переменные \(T, r\) и еще раз проинтегрируем:

\[
-\int_{T}^{T_{0}} d T=\frac{w}{2 \varkappa} \int_{r}^{R} r \mathrm{~d} r, \quad T=T_{0}+\frac{w}{4 \varkappa}\left(R^{2}-r^{2}\right)
\]

6.8. Длина свободного пробега. Азот находится при нормальных условиях. При какой частоте колебаний длина звуковой волны будет равна средней длине свободного пробега молекул данного газа? Эффективный диаметр молекулы азота \(d=0,37 \mathrm{HM}\).

Решение. Искомая частота \(v=v / \lambda\), где \(v\) - скорость звуковой волны, \(\lambda\) - длина волны. Последняя, по условию, должна быть равной величине (6.17). А скорость волны \(v=\sqrt{\gamma R T / M}\). В результате подстановки в исходное выражение получим

\[
v=\pi d^{2} p N_{A} \sqrt{2 \gamma / M R T}=5,5 \cdot 10^{9} \text { Гц. }
\]

6.9. Вязкость. Тонкостенный цилиндр \(C\) (рис. 6.14) массы \(m\), радиуса \(a\) и длины \(l\) подвешен на упругой нити и находится в зазоре между двумя закреплен-

\begin{figure}[h]
\begin{center}
  \includegraphics[alt={},max width=\textwidth]{bd023a69-cc46-4b6f-a48e-cc548fa9259d-193_496_186_1269_964}
\captionsetup{labelformat=empty}
\caption{Рис. 6.14}
\end{center}
\end{figure}

ными цилиндрами, заполненном газом \(X\) с молярной массой \(M\) и температурой \(T\). Поверхность цилиндра \(C\) равноудалена от поверхностей закрепленных цилиндров на расстояние \(\Delta a\), причем \(\Delta a \ll a\). Цилиндр \(C\) совершает затухающие крутильные колебания вокруг вертикальной оси \(O O\) с временем релаксации \(\tau\). Найти эффективное сечение \(\sigma\) атомов \(X\).

P е ш е н и е. Чтобы найти \(\sigma\), надо знать вязкость \(\eta\), а ее можно определить из уравнения динамики для крутильных колебаний:


\begin{equation*}
I \ddot{\varphi}=-\gamma \varphi-N_{\text {тр } z}, \tag{1}
\end{equation*}


где \(I=m a^{2}\) - момент инерции относительно оси \(O O, \gamma\) - коэффициент упругости нити, \(N_{\text {тр } z}\) - момент сил трения, действующий на цилиндр \(C\).

Вычислим этот момент. На единицу площади цилиндра \(C\) действует сила \(f=2 \eta(\partial u / \partial r)\). Появление коэффициента 2 связано с тем, что у цилиндра \(C\) две поверхности. В нашем случае (малый зазор) \(\partial v / \partial r=v / \Delta a=\dot{\varphi} a / \Delta a\), и \(f=2 \eta(a / \Delta a) \dot{\varphi}\). Тогда


\begin{equation*}
N_{\text {тр } z}=a F_{\text {тр }}=a f S=\alpha \dot{\varphi}, \tag{2}
\end{equation*}


где \(S\) - площадь цилиндра \(C(S=2 \pi a l)\), а коэффициент


\begin{equation*}
\alpha=4 \pi \eta l a^{3} / \Delta a . \tag{3}
\end{equation*}


Теперь представим уравнение (1) в стандартном виде:


\begin{equation*}
\ddot{\varphi}+2 \beta \dot{\varphi}+\omega_{0}^{2} \varphi=0, \quad 2 \beta=\alpha / I . \tag{4}
\end{equation*}


Напомним, время релаксации \(\tau\) - это время, за которое амплитуда колебаний уменьшается в е раз. Оно связано с коэффициентом затухания \(\beta\) как \(\tau=1 / \beta\). Поэтому из второго выражения (4) с учетом (3) находим, что вязкость


\begin{equation*}
\eta=\frac{m \Delta a}{2 \pi l a \tau} . \tag{5}
\end{equation*}


И наконец, имея в виду, что \(\eta\) определяется выражением (6.25), а \(\lambda\) - уточненной формулой (6.17), получим:

\[
\sigma=\frac{4}{3} \frac{\tau l a}{m \Delta a N_{A}} \sqrt{\pi R T M}
\]

6.10. Теплопроводность. Пространство между двумя большими горизонтальными пластинами заполнено гелием, диаметр атомов которого равен \(d\). Расстояние между пластинами \(h\). Нижняя пластина поддерживается при температуре \(T_{1}\), верхняя - при \(T_{2}\), причем \(T_{2}>T_{1}\). Давление газа нормальное. Найти плотность потока тепла.

Ре шен и е. Ясно, что поток тепла направлен вниз. Плотность потока дается формулой (6.12):


\begin{equation*}
j_{Q}=-\varkappa \frac{\partial T}{\partial x} \tag{1}
\end{equation*}


Теплопроводность \(\varkappa\) определяется формулой (6.28):


\begin{equation*}
\varkappa=\frac{1}{3}\langle v\rangle \rho c_{V}=\alpha \sqrt{T}, \tag{2}
\end{equation*}


где, как следует из преобразований, коэффициент


\begin{equation*}
\alpha=(R / \pi)^{3 / 2} / d^{2} N_{A} M^{1 / 2} \tag{3}
\end{equation*}


\(R\) - универсальная газовая постоянная, \(M\) - молярная масса.\\
Подстановка (2) в (1) приводит к уравнению


\begin{equation*}
j_{Q} \mathrm{~d} x=-\alpha \sqrt{T} \mathrm{~d} T \tag{4}
\end{equation*}


Проинтегрировав это выражение, получим:

\[
j_{Q}=\frac{2 \alpha}{3 l}\left(T_{2}^{3 / 2}-T_{1}^{3 / 2}\right) .
\]

\section*{1. Единицы физических величин в СИ и СГС}
\begin{center}
\begin{tabular}{|l|l|l|l|}
\hline
\multirow{2}{*}{Величина} & \multicolumn{2}{|c|}{Единица величины} & \multirow{2}{*}{Отношение \(\frac{\text { ед. СИ }}{\text { ед. СГС }}\)} \\
\hline
 & СИ & СГС &  \\
\hline
Скорость \(v\) & м/с & см/с & \(10^{2}\) \\
\hline
Частота колебаний \(v\) & Гц & Гц & 1 \\
\hline
Круговая частота \(\omega\) & \(\mathrm{c}^{-1}\) & \(\mathrm{c}^{-1}\) & 1 \\
\hline
Угловая скорость \(\omega\) & рад/с & рад/с & 1 \\
\hline
Macca \(m\) & кг & г & \(10^{3}\) \\
\hline
Плотность \(\rho\) & кг/м \({ }^{3}\) & г/cM \({ }^{3}\) & \(10^{-3}\) \\
\hline
Сила \(F\) & H & дин & \(10^{5}\) \\
\hline
Давление \(p\) & Па & дин/см \({ }^{2}\) & 10 \\
\hline
Импульс \(p\) & кг•м/с & г.см/с & \(10^{5}\) \\
\hline
Момент силы \(N\) & \(\mathbf{H} \cdot \mathbf{m}\) & дин•см & \(10^{7}\) \\
\hline
Энергия \(E\), работа \(A\), Количество тепла \(Q\) & Дж & эрг & \(10^{7}\) \\
\hline
Теплоемкость \(C\) & Дж/К & эрг/К & \(10^{7}\) \\
\hline
Энтропия \(S\) & Дж/К & эрг/к & \(10^{7}\) \\
\hline
\multirow{3}{*}{\begin{tabular}{l}
Плотность \(\quad\) массы \(j_{m}\) потока \(\left\{\right.\) импульса \(j_{p}\) \\
тепла \(j_{Q}\) \\
\end{tabular}} & кг/(с•M \({ }^{2}\) ) & \(\boldsymbol{\Gamma} \boldsymbol{/} \boldsymbol{(} \mathbf{c} \boldsymbol{\cdot} \mathbf{c m}^{\mathbf{2}} \boldsymbol{)}\) & \(10^{-1}\) \\
\hline
 & \(\mathrm{H} /\left(\mathrm{c} \cdot \mathrm{m}^{2}\right)\) & дин/(c • \(\mathrm{cm}^{2}\) ) & 10 \\
\hline
 & Вт/m \({ }^{2}\) & эрг/(c•cm \({ }^{2}\) ) & \(10^{3}\) \\
\hline
Коэффициент диффузии \(D\) & \(\mathrm{m}^{2} / \mathrm{c}\) & cm \({ }^{2}\) /c & \(10^{4}\) \\
\hline
Вязкость \(\eta\) & Па•с & \(\Pi\) (пуаз) & 10 \\
\hline
Теплопроводность \(\varkappa\) & Вт/(м•К) & эрг/(с•см•К) & \(10^{5}\) \\
\hline
\end{tabular}
\end{center}

\begin{table}[h]
\begin{center}
\captionsetup{labelformat=empty}
\caption{2. Десятичные приставки к названиям единиц}
\begin{tabular}{|c|c|}
\hline
\(\Gamma-\) гига, \(10^{9}\) & м - милли, \(10^{-3}\) \\
М - мега, \(10^{6}\) & мк - микро, \(10^{-6}\) \\
к - кило, \(10^{3}\) & н - нано, \(10^{-9}\) \\
с - санти, \(10^{-2}\) & п - пико, \(10^{-12}\) \\
\hline
\end{tabular}
\end{center}
\end{table}

\section*{3. Греческий алфавит}
\begin{center}
\begin{tabular}{|l|l|l|}
\hline
А, \(\alpha\) - альфа & I, i - йота & P, \(\rho\) - po \\
\hline
B, \(\beta\) - бета & \(\mathrm{K}, \varkappa\) - каппа & \(\Sigma, \sigma\) - сигма \\
\hline
Г, \(\gamma\) - гамма & \(\Lambda, \lambda\) - ламбда & T, \(\tau\) - тау \\
\hline
\(\Delta, \delta\) - дельта & М, μ - мю & Y, \(v\) - ипсилон \\
\hline
Ε, \(\varepsilon\) - эпсилон & \(\mathrm{N}, v\) - ню & \(\Phi, φ\) - фи \\
\hline
\(\mathrm{Z}, \zeta\) - дзета & \(\Xi, \xi\) - кси & Х, \(\chi\) - хи \\
\hline
H, η - эта & О, о - омикрон & \(\Psi, ψ\) - пси \\
\hline
\(\Theta, \theta, \vartheta\) - тета & П, \(\pi\) - пи & \(\Omega, \omega\) - омега \\
\hline
\end{tabular}
\end{center}

\section*{4. Плотности веществ}
\begin{center}
\begin{tabular}{|l|l|l|l|}
\hline
Газ (при нормальных условиях) & \(\rho, к г / \mathrm{M}^{3}\) & Твердое вещество & \(\rho, \quad г / \mathrm{cm}^{3}\) \\
\hline
Азот & 1,25 & Алмаз & 3,5 \\
\hline
Водород & 0,09 & Алюминий & 2,7 \\
\hline
Воздух & 1,293 & Вольфрам & 19,1 \\
\hline
Кислород & 1,43 & Графит & 1,6 \\
\hline
Метан & 0,72 & Железо (сталь) & 7,8 \\
\hline
 &  & Золото & 19,3 \\
\hline
Углекислый газ & 1,98 & Лед & 0,916 \\
\hline
Хлор & 3,21 & Медь & 8,9 \\
\hline
Жидкость & \(\rho, г / \mathrm{cm}^{3}\) & Молибден & 10,2 \\
\hline
 &  & Натрий & 0,97 \\
\hline
Вода & 1,00 & Олово & 7,4 \\
\hline
Глицерин & 1,26 & Платина & 21,5 \\
\hline
 &  & Свинец & 11,3 \\
\hline
Керосин & 0,80 & Серебро & 10,5 \\
\hline
Ртуть & 13,6 & Титан & 4,5 \\
\hline
Спирт & 0,79 & Уран & 19,0 \\
\hline
Эфир & 0,72 & Цинк & 7,0 \\
\hline
\end{tabular}
\end{center}

\begin{enumerate}
  \setcounter{enumi}{4}
  \item Постоянные газов
\end{enumerate}

\begin{center}
\begin{tabular}{|l|l|l|l|l|}
\hline
Газ & \(\gamma=\frac{C_{p}}{C_{V}}\) & Вязкость, \(\eta\), мкПа•с & Теплопроводность \(\varkappa\), мВт/(м•К) & Диаметр молекулы \(d\), нм \\
\hline
He & 1,67 & 18,9 & 141,5 & 0,20 \\
\hline
Ar & 1,67 & 22,1 & 16,2 & 0,35 \\
\hline
\(\mathrm{H}_{2}\) & 1,41 & 8,4 & 168,4 & 0,27 \\
\hline
\(\mathrm{N}_{2}\) & 1,40 & 16,7 & 24,3 & 0,37 \\
\hline
\(\mathrm{O}_{2}\) & 1,40 & 19,2 & 24,4 & 0,35 \\
\hline
\(\mathrm{CO}_{2}\) & 1,30 & 14,0 & 23,2 & 0,40 \\
\hline
\(\mathrm{H}_{2} \mathrm{O}\) & 1,32 & 9,0 & 15,8 & 0,30 \\
\hline
Воздух & 1,40 & 17,2 & 24,1 & 0,35 \\
\hline
\end{tabular}
\end{center}

Постоянные \(\boldsymbol{a}\) и \(\boldsymbol{b}\) Ван-дер-Ваальса

\begin{center}
\begin{tabular}{|c|c|c|c|c|c|c|}
\hline
Газ & Ar & \(\mathrm{H}_{2}\) & \(\mathrm{~N}_{2}\) & \(\mathrm{O}_{2}\) & \(\mathrm{CO}_{2}\) & \(\mathrm{H}_{2} \mathrm{O}\) \\
\hline
\(a\), Па \({ }^{6} \mathrm{M}^{6} /\) моль \(^{2}\) & 0,132 & 0,024 & 0,137 & 0,137 & 0,367 & 0,554 \\
\(b, 10^{-6} \mathrm{M}^{3} /\) моль & 32 & 27 & 39 & 32 & 43 & 30 \\
\hline
\end{tabular}
\end{center}

\begin{enumerate}
  \setcounter{enumi}{5}
  \item Постоянные жидкостей
\end{enumerate}

\begin{center}
\begin{tabular}{|l|l|l|l|l|}
\hline
Жидкость & Вязкость η, мПа•с & Поверхностное натяжение \(α\), мН/м & Удельная теплоемкость \(c\), Дж/(г•Κ) & Удельная теплота парообразования \(q\), Дж/(г•Κ) \\
\hline
Вода & 10 & 73 & 4,18 & 2250 \\
\hline
Глицерин & 1500 & 66 & 2,42 & - \\
\hline
Ртуть & 16 & 470 & 0,14 & 284 \\
\hline
Спирт & 12 & 24 & 2,42 & 853 \\
\hline
\end{tabular}
\end{center}

\begin{enumerate}
  \setcounter{enumi}{6}
  \item Постоянные твердых тел
\end{enumerate}

\begin{center}
\begin{tabular}{|l|l|l|l|l|}
\hline
Вещество & Удельная теплоемкость \(c\), Дж/(г.К) & Дебаевская температура \(\Theta\), К & Температура плавления \(t,{ }^{\circ} \mathrm{C}\) & Удельная теплота плавления \(q\), Дж/г \\
\hline
Алюминий & 0,90 & 374 & 660 & 321 \\
\hline
Железо & 0,46 & 467 & 1535 & 270 \\
\hline
Лед & 2,00 & - & 0 & 333 \\
\hline
Медь & 0,39 & 329 & 1083 & 175 \\
\hline
Свинец & 0,13 & 89 & 328 & 25 \\
\hline
Серебро & 0,23 & 210 & 960 & 88 \\
\hline
\end{tabular}
\end{center}

\section*{8. Фундаментальные физические константы}
\begin{center}
\begin{tabular}{|l|l|}
\hline
Скорость света в вакууме & \(c=3,00 \cdot 10^{8} \mathrm{~m} / \mathrm{c}\) \\
\hline
Гравитационная постоянная & \(\gamma=6,67 \cdot 10^{-11} \mathrm{M}^{3} /\left(\mathrm{KN} \cdot \mathrm{c}^{2}\right)\) \\
\hline
Ускорение свободного падения & \(g=9,81 \mathrm{~m} / \mathrm{c}^{2}\) \\
\hline
Постоянная Авогадро & \(N_{A}=6,02 \cdot 10^{23}\) моль \(^{-1}\) \\
\hline
Стандартный объем моля газа & \(V_{0}=22,41\) л/моль \\
\hline
Постоянная Лошмидта & \(n_{0}=2,7 \cdot 10^{19} \mathrm{~cm}^{-3}\) \\
\hline
Молярная газовая постоянная & \(R=8,31\) Дж/(К•моль) \\
\hline
Постоянная Больцмана & \(k=1,38 \cdot 10^{-23}\) Дж/К \\
\hline
Элементарный заряд & \(e=\left\{\begin{array}{l}1,60 \cdot 10^{-19} \text { Кл } \\ 4,80 \cdot 10^{-10} \text { СГСЭ }\end{array}\right.\) \\
\hline
Масса электрона & \(m_{e}=0,91 \cdot 10^{-30}\) кг \\
\hline
Масса протона & \(m_{p}=1,67 \cdot 10^{-27}\) кг \\
\hline
Постоянная Планка & \(\hbar=1,055 \cdot 10^{-34}\) Дж•с \\
\hline
Постоянная Стефана-Больцмана & \(\sigma=5,67 \cdot 10^{-8} \mathrm{Bt} /\left(\mathrm{M}^{2} \cdot \kappa^{4}\right)\) \\
\hline
Постоянная смещения Вина & \(b=0,29 \mathrm{~cm} \cdot \boldsymbol{K}^{-}\) \\
\hline
Атомная единица массы & 1 а.е.м. \(=1,66 \cdot 10^{-27}\) кг \\
\hline
Электрическая постоянная & \(\varepsilon_{0}=0,885 \cdot 10^{-11} \Phi / \mathrm{m}\) \\
\hline
 & \(1 / 4 \pi \varepsilon_{0}=9 \cdot 10^{9} \mathrm{M} / \Phi\) \\
\hline
Магнитная постоянная & \(\mu_{0}=1,257 \cdot 10^{-6} \Gamma \mathrm{H} / \mathrm{m}\) \\
\hline
 & \(\mu_{0} / 4 \pi=10^{-7} \Gamma \mathrm{H} / \mathrm{m}\) \\
\hline
\end{tabular}
\end{center}

\section*{Предметный указатель}
Аддитивность 12,81\\
Адиабата 19\\
Анализ коэффициентов переноса 185\\
Анизотропия 152\\
Атомы «меченые» 182\\
Бозоны 106\\
Вакуум 187\\
Вероятность 43\\
Вес статистический 88\\
Волна дебройлевская 71\\
Время релаксации 10, 117\\
Вязкость 176, 184\\
Газ Ван-дер-Ваальса 30

\begin{itemize}
  \item вырожденный 114
  \item идеальный 15
  \item ультраразреженный 187
  \item фононный 127
  \item фотонный 119
  \item электронный 110, 111, 130
\end{itemize}

Гипотеза о равнораспределении энергии 26

\begin{itemize}
  \item о тепловой смерти Вселенной 95
  \item флуктуационная Больцмана 95
\end{itemize}

Градиент плотности 176\\
Давление внутреннее 31

\begin{itemize}
  \item газа на стенку 24
  \item дополнительное 149
  \item критическое 162
  \item под изогнутой поверхностью 148
  \item фононного газа 136
  \item фотонного газа 25
  \item электронного газа 132
\end{itemize}

Диаграмма состояний 143, 144\\
Диаметр молекулы эффективный 178\\
Диффузия 175, 183

\begin{itemize}
  \item тепловая 187
\end{itemize}

Диэлектрики 116\\
Длина свободного пробега средняя 178\\
Доска Гальтона 48\\
«Дырки» 118\\
Жидкость 144

\begin{itemize}
  \item перегретая 141
\end{itemize}

Закон Бугера 170

\begin{itemize}
  \item Дюлонга и Пти 124
  \item кубов Дебая 126, 130
  \item распределения Больцмана 60
  \item 
  \begin{itemize}
    \item Максвелла 52
  \end{itemize}
  \item 
  \begin{itemize}
    \item Максвелла- Больцмана 67,68
  \end{itemize}
  \item смещения Вина 123
  \item Стефана- Больцмана 122
\end{itemize}

Значения средние 45,47

\begin{itemize}
  \item 
  \begin{itemize}
    \item случайных величин 45
  \end{itemize}
\end{itemize}

Зона валентная 115

\begin{itemize}
  \item проводимости 117
  \item свободная 116
\end{itemize}

Зоны энергетические в кристалле 115\\
Излучение индуцированное (вынужденное) 170

\begin{itemize}
  \item когерентное 171
\end{itemize}

Изотерма критическая 138\\
Изотермы Ван-дер-Ваальса 138

\begin{itemize}
  \item экспериментальные 138
\end{itemize}

Изоэнтропы 80\\
Индексы Миллера 154, 155\\
Интенсивность пучка 171\\
Испарение 140\\
Камера Вильсона 140

\begin{itemize}
  \item пузырьковая жидководородная 140
\end{itemize}

Капилляр 151\\
Квазиимпульс 127\\
Квазичастицы 127\\
Кипение 140\\
Колебания плазменные 159\\
Количество вещества 15

\begin{itemize}
  \item теплоты 12
\end{itemize}

Конденсация пара 140\\
Коэффициент вязкости 176

\begin{itemize}
  \item диффузии 176
  \item полезного действия 78,85
  \item самодиффузии 183
  \item теплопроводности 177
\end{itemize}

Коэффициенты Эйнштейна 170\\
Кратность вырождения 67, 170\\
Кривая инверсии 34

\begin{itemize}
  \item испарения 143
  \item плавления 143
  \item сублимации 143
\end{itemize}

Кристаллы атомные 157

\begin{itemize}
  \item жидкие 145
  \item ионные 157
  \item металлические 157
  \item молекулярные 157
\end{itemize}

Лазер 172\\
Мазер 172\\
Макросистема 7\\
Макросостояние 88\\
Масса атмосферы Земли 65

\begin{itemize}
  \item молярная 15
\end{itemize}

Мениск 151\\
Металлы 116\\
Метод статистический 21\\
Микросостояние 88\\
Модель жидкости 145

\begin{itemize}
  \item теплоемкости Дебая 126
  \item 
  \begin{itemize}
    \item классическая 124
  \end{itemize}
  \item 
  \begin{itemize}
    \item Эйнштейна 124
  \end{itemize}
\end{itemize}

Молекулы линейные 27

\begin{itemize}
  \item нелинейные 27
\end{itemize}

Моль 16

Нагреватель 84\\
Накачка 172\\
Натяжение поверхностное 146\\
Начало термодинамики второе 77,81

\begin{itemize}
  \item 
  \begin{itemize}
    \item первое 11-13
  \end{itemize}
  \item 
  \begin{itemize}
    \item третье 82
  \end{itemize}
\end{itemize}

Нуль температуры абсолютный 9\\
Объем фазовой ячейки 108\\
Опыт Ламмерта 57

\begin{itemize}
  \item Перрена 62, 63
\end{itemize}

Особенности лазерного излучения 174\\
Ось симметрии 153\\
Парадокс Гиббса 84\\
Параметры термодинамические 9\\
Пар насыщенный 139

\begin{itemize}
  \item пересыщенный 140
\end{itemize}

Переходы фазовые 139, 141\\
Период идентичности 153\\
Перпетуум-мобиле 2-го рода 78\\
Плазма 158

\begin{itemize}
  \item в магнитном поле 161
  \item идеальная 160
  \item изотермическая 160
\end{itemize}

Плоскости кристаллические 155\\
Плотность вероятности 46

\begin{itemize}
  \item 
  \begin{itemize}
    \item объемная 50
  \end{itemize}
  \item потока импульса 176
  \item 
  \begin{itemize}
    \item массы 176
  \end{itemize}
  \item 
  \begin{itemize}
    \item молекул 181
  \end{itemize}
  \item 
  \begin{itemize}
    \item тепла 177
  \end{itemize}
\end{itemize}

Поликристаллы 152\\
Политропа 19\\
Полупроводники 116\\
Порядок ближний 145

\begin{itemize}
  \item дальний 145
\end{itemize}

Постоянная Авогадро 16

\begin{itemize}
  \item адиабаты 17, 28
  \item Больцмана 24
  \item газовая молярная 15
  \item закона смещения Вина 123
  \item Лошмидта 199
  \item Планка 29, 199
  \item Стефана- Больцмана 122
\end{itemize}

Постоянные Ван-дер-Ваальса 31

\begin{itemize}
  \item решетки 153
\end{itemize}

Потенциал химический 108\\
Потенциалы термодинамические 97\\
Поток 174\\
Принцип возрастания энтропии 81

\begin{itemize}
  \item неопределенности Гейзенберга 108
  \item неразличимости тождественных частиц 106
  \item Паули 106, 111
\end{itemize}

Проблема необратимости процессов 79

\begin{itemize}
  \item устойчивости плазмы 161
\end{itemize}

Проводимость плазмы 160

\begin{itemize}
  \item примесная 118
  \item собственная 118
\end{itemize}

Пространство скоростей 49

\begin{itemize}
  \item фазовое 108
\end{itemize}

Процесс 10

\begin{itemize}
  \item адиабатический 18
  \item изобарический 17
  \item изотермический 19
  \item квазистатический 10
  \item круговой 14,78
  \item неравновесный 10
  \item обратимый 11
  \item политропический 19
  \item равновесный 10
\end{itemize}

Работа 13

\begin{itemize}
  \item выхода 114
  \item газа при политропическом процессе 21
\end{itemize}

Равновесие локальное 175

\begin{itemize}
  \item термодинамическое 9
  \item фаз 141
\end{itemize}

Радиус дебаевский 158

\begin{itemize}
  \item молекулярного действия 146
\end{itemize}

Распределение Бозе- Эйнщтейна 119, 120

\begin{itemize}
  \item Больцмана 60
  \item 
  \begin{itemize}
    \item при дискретных уровнях 66
  \end{itemize}
  \item квантовых частиц 110
  \item Максвелла- Больцмана 67,68
  \item 
  \begin{itemize}
    \item в приведенном виде 55
  \end{itemize}
  \item 
  \begin{itemize}
    \item по модулям скорости 52
  \end{itemize}
  \item 
  \begin{itemize}
    \item по проекциям скорости 51
  \end{itemize}
  \item по энергиям молекул 56
  \item Ферми- Дирака 110, 111, 112
  \item фононов 128
  \item фотонов 119, 120
  \item электронов 111, 112
\end{itemize}

Распределения квантовые 108\\
Рассеяние атмосферы 65\\
Расстояние межплоскостное 156\\
Резонаторы оптические 173\\
Решетка кристаллическая 152\\
Самодиффузия 182\\
Светимость энергетическая 122\\
Свойства энтропии 80,81\\
Связь гетерогенная (ионная) 157

\begin{itemize}
  \item гомеополярная (ковалентная) 157
  \item обратная 173
\end{itemize}

Сечение молекулы эффективное 178\\
Симметрия кристаллов 153

\begin{itemize}
  \item трансляции 153
\end{itemize}

Система неравновесная 10, 169

\begin{itemize}
  \item сильно неравновесная 169
  \item трехуровневая 172
\end{itemize}

Скорости молекул характерные 53\\
Скорость наиболее вероятная 53

\begin{itemize}
  \item среднеквадратичная 53
  \item средняя 53
\end{itemize}

Смачивание 150\\
Смерть Вселенной тепловая 95\\
Смысл температуры физический 24

\begin{itemize}
  \item второго начала статистический 88
\end{itemize}

Соотношения термодинамические 96\\
Состояние жидкое 144

\begin{itemize}
  \item критическое 162
  \item метастабильное 140
  \item неравновесное 10, 169
  \item равновесное 10
  \item системы 9
\end{itemize}

Спектр квазинепрерывный 111\\
Среда активная 172

\begin{itemize}
  \item инверсная 172
\end{itemize}

Статистика Бозе - Эйнштейна 106

\begin{itemize}
  \item Ферми- Дирака 106
\end{itemize}

Статистики квантовые 106\\
Степени свободы 25,27\\
Степень беспорядка 94

\begin{itemize}
  \item ионизации плазмы 158
\end{itemize}

Строение кристаллов 152\\
Тело черное 122\\
Теорема Карно 86

\begin{itemize}
  \item Нернста 82
  \item сложения вероятностей 44
  \item умножения вероятностей 44
\end{itemize}

Температура 9

\begin{itemize}
  \item кипения 140
  \item критическая 138
  \item отрицательная ( \(T<0\) ) 173
  \item плазмы 160
  \item термодинамическая 9
  \item характеристическая Дебая 129
\end{itemize}

Теория молекулярно-кинетическая 21

\begin{itemize}
  \item теплоемкости Дебая 129
  \item 
  \begin{itemize}
    \item Эйнштейна 125
  \end{itemize}
\end{itemize}

Теплоемкость 16

\begin{itemize}
  \item идеального газа 16
  \item кристалла 129
  \item молярная 16
  \item при политропических процессах 20
  \item 
  \begin{itemize}
    \item постоянном давлении 17
  \end{itemize}
  \item 
  \begin{itemize}
    \item 
    \begin{itemize}
      \item объеме 17
    \end{itemize}
  \end{itemize}
  \item удельная 16
  \item электронного газа 130
\end{itemize}

Теплопроводность 177 , 185\\
Теплопередача 12\\
Теплота 12

\begin{itemize}
  \item испарения 141
  \item конденсации 141
  \item кристаллизации 141
  \item перехода 141
  \item 
  \begin{itemize}
    \item удельная испарения 141
  \end{itemize}
  \item 
  \begin{itemize}
    \item 
    \begin{itemize}
      \item конденсации 141
    \end{itemize}
  \end{itemize}
  \item 
  \begin{itemize}
    \item 
    \begin{itemize}
      \item кристаллизации 138
    \end{itemize}
  \end{itemize}
  \item 
  \begin{itemize}
    \item 
    \begin{itemize}
      \item плавления 141
    \end{itemize}
  \end{itemize}
  \item плавления 141
  \item сублимации 141
\end{itemize}

Термодинамика 7

\begin{itemize}
  \item фотонного газа 133
\end{itemize}

Термоизоляция магнитная 161\\
Типы кристаллов физические 157\\
Толщина земной атмосферы 65\\
Точка критическая \(138,141,143\)

\begin{itemize}
  \item тройная 9,143
\end{itemize}

Трение внутреннее 176\\
Угол краевой 150\\
Уравнение (адиабаты) Пуассона 19

\begin{itemize}
  \item Ван-дер-Ваальса 30
  \item кинетической теории газов 24
  \item Клапейрона 15
  \item Клапейрона- Клаузиуса 142
  \item переноса общее 182
  \item 
  \begin{itemize}
    \item зависящее от времени 186
  \end{itemize}
  \item самодиффузии 182
  \item состояния идеального газа 15, 25
  \item термодинамики основное 82
\end{itemize}

Уровень (энергия) Ферми 112

\begin{itemize}
  \item энергии метастабильный 173
\end{itemize}

Уровни акцепторные и донорные 119

\begin{itemize}
  \item энергии квазинепрерывные 109, 111
\end{itemize}

Усиление пороговое 173\\
Условие квазинейтральности 159

\begin{itemize}
  \item нормировки 46
\end{itemize}

Фаза 139\\
Фермионы 106\\
Физика квантовая 8

\begin{itemize}
  \item молекулярная 7
\end{itemize}

Флуктуации 47\\
Фононы 126\\
Формула барометрическая 63

\begin{itemize}
  \item Больцмана (для энтропии) 93
  \item Вина 121
  \item Дебая 129
  \item Лапласа 149
  \item Максвелла в приведенном виде 54,55
  \item Планка 29, 114
  \item Пуазейля 190
  \item Эйнштейна (для теплоемкости) 125
\end{itemize}

Фотоны 119\\
Функция процесса 16

\begin{itemize}
  \item распределения 45
  \item состояния 12, 79, 80
\end{itemize}

Холодильник 77,84\\
Цикл Карно 84, 85

\begin{itemize}
  \item круговой 14
\end{itemize}

Циклы 14, 78, 85\\
Частота плазменных колебаний 159\\
Число степеней свободы 26

\begin{itemize}
  \item ударов молекул о стенку 22
  \item фазовых ячеек 110
\end{itemize}

Шкала Кельвина 9

\begin{itemize}
  \item температур термодинамическая 9
  \item Цельсия 9
\end{itemize}

Электроны свободные 110\\
Электропроводность металлов 117

\begin{itemize}
  \item полупроводников 118
\end{itemize}

Энергия ван-дер-ваальсова газа 32

\begin{itemize}
  \item внутренняя 11, 28, 98
  \item вращательная 29
  \item колебательная 29
  \item нулевая 125
  \item поверхностного слоя 147
  \item решетки колебательная 127
  \item свободная 98, 148
  \item средняя квантового осциллятора 125, 135
  \item 
  \begin{itemize}
    \item свободных электронов 113
  \end{itemize}
  \item Ферми 112
  \item 
  \begin{itemize}
    \item при \(T>0115\)
  \end{itemize}
\end{itemize}

Энтропия 79

\begin{itemize}
  \item абсолютное значение 82
  \item и вероятность 91,92
  \item идеального газа 82,83
  \item как мера необратимости 81
\end{itemize}

Эффект Джоуля - Томсона 33, 34\\
Явления под изогнутой поверхностью 148

\begin{itemize}
  \item капиллярные 151
  \item на границах между средами 150
  \item переноса 174
\end{itemize}

Ячейка простая (примитивная) 153

\begin{itemize}
  \item фазовая 108
  \item элементарная 153
\end{itemize}

Минимальные системные требования определяются соответствующими требованиями программы Adobe Reader версии не ниже 11-й для платформ Windows, Mac OS, Android, iOS, Windows Phone и BlackBerry; экран 10"

\section*{Учебное электронное издание}
Серия: «Технический университет. Общая физика»\\
Иродов Игорь Евгеньевич\\
ФИЗИКА МАКРОСИСТЕМ. ОСНОВНЫЕ ЗАКОНЫ\\
Учебное пособие\\
Подписано к использованию 19.03.15.\\
Формат \(125 \times 200\) мм\\
Издательство «БИНОМ. Лаборатория знаний»\\
125167, Москва, проезд Аэропорта, д. 3\\
Телефон: (499) 157-5272\\
e-mail: \href{mailto:info@pilotLZ.ru}{info@pilotLZ.ru}, \href{http://www.pilotLZ.ru}{http://www.pilotLZ.ru}

\begin{itemize}
  \item 
\end{itemize}

\begin{itemize}
  \item 
\end{itemize}

\begin{itemize}
  \item 
\end{itemize}

\begin{itemize}
  \item 
\end{itemize}

\begin{itemize}
  \item 
\end{itemize}

\begin{itemize}
  \item 
\end{itemize}

\begin{itemize}
  \item 
\end{itemize}

\begin{itemize}
  \item 
\end{itemize}

\begin{itemize}
  \item 
\end{itemize}

\begin{itemize}
  \item 
\end{itemize}

\begin{itemize}
  \item 
\end{itemize}

\begin{itemize}
  \item 
\end{itemize}

\begin{itemize}
  \item 
\end{itemize}

\begin{itemize}
  \item 
\end{itemize}

\begin{itemize}
  \item 
\end{itemize}

\begin{itemize}
  \item 
\end{itemize}


\end{document}

% ════════════════════════════════════════════════════════════════════
% ДУТУУ ТЕКСТҮҮД — TeX-д байрлал олдоогүй тул эцэст нэмэгдсэн
% ════════════════════════════════════════════════════════════════════

% ── PDF стр. 1–4 (458 үг) ──
УЧЕБНОЕ ПОСОБИЕ
Москва
БИНОМ. Лаборатория знаний
6-е издание (электронное)
2 0 1 5
И. Е. Иродов
основные
законы
ФИЗИКА
МАКРОСИСТЕМ


% ── PDF стр. 16 (31 үг) ──
Моль — это количество вещества, содержащее число частиц,
равное постоянной Авогадро:
NA = 6,022 · 10
23 моль
–1.
Молю соответствует масса — молярная масса, — разная для
различных газов. Эти массы приведены в периодической системе элементов, где у каждого элемента первое число — порядковый номер, а второе — молярная масса в г/моль.
С молекулярной точки зрения идеальный газ состоит из молекул, взаим


% ── PDF стр. 22 (30 үг) ──
Первый шаг на этом пути — выбор модели данной макросистемы. Модель, естественно, выбирают сначала простейшую и
с помощью нее проводят оценочный расчет*. Полученные результаты сравнивают с экспериментом. При неудовлетворительном расхождении с опытом модель усложняют, выясняют, как
это отразится на результатах и т. д. Это типичный путь познания природы.
Простейшей моделью обладает идеальный газ. Буд


% ── PDF стр. 30 (62 үг) ──
льной степени свободы* (полностью — нет, поскольку раньше
наступает диссоциация молекулы на два атома, участок 3).
Объяснение зависимости СV (T), показанной на рис. 1.11, —
это серьезный успех квантовой теории.
§1.7. Газ Ван-дер-Ваальcа
Уравнение Ван-дер-Ваальса. До сих пор мы не выходили за
рамки модели идеального газа (рVM  RT). Однако опыт вынуждает уточнить эту модель, поскольку с ростом давл


% ── PDF стр. 49–50 (30 үг) ──
тральные ячейки действительно попадает больше всего зернышек.
При очень большом числе зернышек возникает вполне определенная статистическая закономерность распределения их по
ячейкам, как показано на рис. 2.2. Повторив опыт, обнаружим
то же распределение, ту же закономерность (это распределение
называют нормальным законом Гаусса).
Таким образом, мы имеем наглядную, убедительную (и невольно вызываю


% ── PDF стр. 60 (26 үг) ──
Проинтегрировав последнее уравнение, получим
ln n
n
U
U
kT
0
0
 

.
(2.35)
Будем считать, что U0  0, где n  n0, тогда
n = n0e–U/kT.
(2.36)
Этот закон и выражает распределение Больцмана.
С помощью (2.36) можно найти число молекул в интересующем нас элементарном объеме dV:
dN = n dV.
(2.37)
При этом следует иметь в виду. что объем dV может иметь,
вообще говоря, не любую форму. Обязательным явля


% ── PDF стр. 106 (34 үг) ──
§ 4.1. Квантовые статистики
В квантовой физике, как и в статистической, закономерности имеют вероятностный, статистический характер. Однако
есть и принципиальное отличие: в квантовой физике статистический (вероятностный) подход лежит в самой природе микрочастиц, в их волновых свойствах.
Согласно квантовой теории все микрочастицы подразделяют
на два класса, которым соответствуют две квантовые стати


% ── PDF стр. 114 (40 үг) ──
ргиям, т. е. dn/d (рис. 4.5).
Таким образом, при нагревании металла энергию могут изменить
только
те
свободные
электроны, которые находятся
на самых верхних уровнях, примыкающих
к
уровню
Ферми.
Основная же масса свободных
электронов
на
более
низких
энергетических уровнях остается в прежнем состоянии и поглощать энергию при нагревании
не будет.
Именно в этом ключ к разгадке «странного» поведения электронного газа, вклад которого в теплоемкость практически не
заметен, и теплоемкость кристалла практически зависит только от колебаний атомов решетки (к этому вопросу мы еще вернемся в конце § 4.5).
Имея в виду, что в модели свободных электронов последние
находятся в прямоугольной потенци


% ── PDF стр. 184 (32 үг) ──
фузии D12 разных газов при нормальных условиях имеет тот
же порядок 10–5 м2/с.
Вязкость (внутреннее трение). Это явление возникает в тех
случаях, когда на хаотическое тепловое движение молекул накладывается упорядоченное движение. Пусть скорость u упорядоченного движения зависит только от
координаты
х,
как
показано
на
рис. 6.10. В этом случае через единичную площадку S будет происходить перенос им
