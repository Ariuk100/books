% This LaTeX document needs to be compiled with XeLaTeX.
\documentclass[10pt]{article}
\usepackage[utf8]{inputenc}
\usepackage{ucharclasses}
\usepackage{graphicx}
\usepackage[export]{adjustbox}
\graphicspath{ {./images/} }
\usepackage{amsmath}
\usepackage{amsfonts}
\usepackage{amssymb}
\usepackage[version=4]{mhchem}
\usepackage{stmaryrd}
\usepackage{caption}
\usepackage{multirow}
\usepackage{fvextra, csquotes}
\usepackage{hyperref}
\hypersetup{colorlinks=true, linkcolor=blue, filecolor=magenta, urlcolor=cyan,}
\urlstyle{same}
\usepackage[fallback]{xeCJK}
\usepackage{polyglossia}
\usepackage{fontspec}
\usepackage{newunicodechar}
\IfFontExistsTF{Noto Serif CJK JP}
{\setCJKmainfont{Noto Serif CJK JP}}
{\IfFontExistsTF{STSong}
  {\setCJKmainfont{STSong}}
  {\IfFontExistsTF{Droid Sans Fallback}
    {\setCJKmainfont{Droid Sans Fallback}}
    {\setCJKmainfont{SimSun}}
}}

\setmainlanguage{russian}
\setotherlanguages{greek}
\IfFontExistsTF{CMU Serif}
{\newfontfamily\lgcfont{CMU Serif}}
{\IfFontExistsTF{DejaVu Sans}
  {\newfontfamily\lgcfont{DejaVu Sans}}
  {\newfontfamily\lgcfont{Georgia}}
}
\setDefaultTransitions{\lgcfont}{}
\newfontfamily\cyrillicfont{CMU Serif}
\setDefaultTransitions{\cyrillicfont}{}

\title{ВОЛНОВЫЕ ПРОЦЕССЫ \\
 OCHOBHWE 3aKOHSI }

\author{Иродов И. Е.}
\date{}


%New command to display footnote whose markers will always be hidden
\let\svthefootnote\thefootnote
\newcommand\blfootnotetext[1]{%
  \let\thefootnote\relax\footnote{#1}%
  \addtocounter{footnote}{-1}%
  \let\thefootnote\svthefootnote%
}

%Overriding the \footnotetext command to hide the marker if its value is `0`
\let\svfootnotetext\footnotetext
\renewcommand\footnotetext[2][?]{%
  \if\relax#1\relax%
    \ifnum\value{footnote}=0\blfootnotetext{#2}\else\svfootnotetext{#2}\fi%
  \else%
    \if?#1\ifnum\value{footnote}=0\blfootnotetext{#2}\else\svfootnotetext{#2}\fi%
    \else\svfootnotetext[#1]{#2}\fi%
  \fi
}

\newunicodechar{⊸}{\ifmmode\multimap\else{$\multimap$}\fi}
\newunicodechar{↓}{\ifmmode\downarrow\else{$\downarrow$}\fi}

\begin{document}
\maketitle
\captionsetup{singlelinecheck=false}
\begin{center}
\includegraphics[max width=\textwidth, alt={}]{950fefba-d83b-4c8f-96e7-0bd5a7eb90be-001_692_872_971_303}
\end{center}

\section*{ВОЛНОВЫЕ ПРОЦЕССЫ }
\section*{основные законы}
7-е издание (электронное)

Рекомендовано учебно-методическим объединением в области «Ядерные физика и технологии» в качестве учебного пособия для студентов физических специальностей высших учебных заведений

И83 Волновые процессы. Основные законы [Электронный ресурс] / И. Е. Иродов. - 7-е изд. (эл.). - Электрон. текстовые дан. (1 файл pdf : 265 с.). - М. : БИНОМ. Лаборатория знаний, 2015. - Систем. требования: Adobe Reader XI ; экран \(10^{\prime \prime}\).

ISBN 978-5-9963-2738-6\\
Данное учебное пособие содержит теоретический материал (основные идеи волновых процессов), а также разбор многочисленных примеров и задач, где показано, как (по мнению автора) надо подходить к их решению. Задачи тесно связаны с основным текстом и часто являются его развитием и дополнением. Материал книги, насколько возможно, освобожден от излишней математизации - основной акцент перенесен на физическую сторону рассматриваемых явлений.

Для студентов физических специальностей вузов.\\
УДК 535.12(075)\\
ББК 22.343я7

Деривативное электронное издание на основе печатного аналога: Волновые процессы. Основные законы / И. Е. Иродов. -7-е изд. - М. : БИНОМ. Лаборатория знаний, 2015. - 263 с. : ил. - ISBN 978-5-9963-1897-1.

В соответствии со ст. 1299 и 1301 ГК РФ при устранении ограничений, установленных техническими средствами защиты авторских прав, правообладатель вправе требовать от нарушителя возмещения убытков или выплаты компенсации

\section*{Содержание}
Предисловие ..... 5\\
Принятые обозначения ..... 6\\
Часть I. Волны . ..... 7\\
Глава 1. Упругие волны ..... 7\\
§ 1.1. Уравнение волны. ..... 7\\
§ 1.2. Волновые уравнения ..... 12\\
§ 1.3. Скорость упругих волн ..... 16\\
§ 1.4. Энергия упругой волны ..... 20\\
§ 1.5. Стоячие волны ..... 25\\
§ 1.6. Звуковые волны. ..... 29\\
§ 1.7. Эффект Доплера для звуковых волн ..... 33\\
Задачи . ..... 36\\
Глава 2. Электромагнитные волны ..... 42\\
§ 2.1. Волновое уравнение электромагнитной волны ..... 42\\
§ 2.2. Плоская электромагнитная волна ..... 44\\
§ 2.3. Стоячая электромагнитная волна ..... 48\\
§ 2.4. Энергия электромагнитной волны ..... 50\\
§ 2.5. Импульс электромагнитной волны ..... 52\\
§ 2.6. Эффект Доплера для электромагнитных волн ..... 54\\
§ 2.7. Излучение диполя ..... 58\\
Задачи . ..... 61\\
Часть II. Волновая оптика ..... 66\\
Глава 3. Вступление ..... 66\\
§ 3.1. Световая волна ..... 66\\
§ 3.2. Электромагнитная волна на границе раздела ..... 68\\
§ 3.3. Геометрическая оптика ..... 71\\
§ 3.4. Фотометрические величины ..... 80\\
Задачи . ..... 84\\
Глава 4. Интерференция света ..... 92\\
§ 4.1. Интерференция световых волн ..... 92\\
§ 4.2. Когерентность ..... 97\\
§ 4.3. Интерференционные схемы ..... 105\\
§ 4.4. Интерференция при отражении от тонких пластинок ..... 111\\
§ 4.5. Интерферометр Майкельсона ..... 121\\
§ 4.6. Многолучевая интерференция ..... 123\\
Задачи ..... 126\\
Глава 5. Дифракция света ..... 133\\
§ 5.1. Принцип Гюйгенса-Френеля ..... 133\\
§ 5.2. Дифракция Френеля от круглого отверстия ..... 136\\
§ 5.3. Дифракция Френеля от полуплоскости и щели ..... 145\\
§ 5.4. Дифракция Фраунгофера ..... 152\\
§ 5.5. Дифракция Фраунгофера от круглого отверстия ..... 154\\
§ 5.6. Дифракция Фраунгофера от щели ..... 159\\
§ 5.7. Дифракционная решетка ..... 162\\
§ 5.8. Дифракционная решетка как спектральный прибор. ..... 170\\
§ 5.9. Дифракция от пространственной решетки ..... 174\\
§ 5.10. О голографии . ..... 177\\
Задачи ..... 181\\
Глава 6. Поляризация света ..... 189\\
§ 6.1. Общие сведения о поляризации ..... 189\\
§ 6.2. Поляризация при отражении и преломлении ..... 193\\
§ 6.3. Поляризация при двойном лучепреломлении ..... 197\\
§ 6.4. Суперпозиция поляризованных волн ..... 201\\
§ 6.5. Интерференция поляризованных волн ..... 209\\
§ 6.6. Искусственное двойное лучепреломление ..... 214\\
§ 6.7. Вращение направления линейной поляризации ..... 217\\
Задачи ..... 221\\
Глава 7. Взаимодействие света с веществом ..... 229\\
§ 7.1. Дисперсия света ..... 229\\
§ 7.2. Классическая теория дисперсии ..... 230\\
§ 7.3. Групповая скорость ..... 235\\
§ 7.4. Поглощение света ..... 239\\
§ 7.5. Рассеяние света ..... 241\\
Задачи ..... 244\\
Приложения. ..... 251

\begin{enumerate}
  \item Поведение плоской волны на границе двух диэлектриков ..... 251
  \item Формула сферической преломляющей поверхности. ..... 252
  \item Излучение Вавилова-Черенкова. ..... 253
  \item Единицы физических величин ..... 255
  \item Десятичные приставки к названиям единиц ..... 255
  \item Греческий алфавит ..... 255
  \item Единицы величин в СИ и системе Гаусса ..... 256
  \item Основные формулы электродинамики в СИ и гауссовой системе ..... 257
  \item Некоторые физические константы ..... 258\\
Предметный указатель ..... 259
\end{enumerate}

\section*{Предисловие}
Основная идея предлагаемой книги - органически совместить в одном учебном пособии изложение принципов теории и практику решения задач. С этой целью в каждой главе сначала излагается теория соответствующего вопроса (с иллюстрацией на конкретных примерах), а затем дается разбор ряда задач, где показывается, к а к, по мнению автора, следует подходить к их решению. Задачи тесно связаны с основным текстом, часто являются его развитием и дополнением, поэтому работа над ними должна проводиться параллельно с изучением основного материала. Кроме того, предлагаемый набор задач должен, по замыслу автора, дать возможность учащемуся дополнительно обдумать ряд важных вопросов и помочь представить (даже если многие задачи не решать, а просто прочитать их условия) большой диапазон приложения изучаемых идей.

При изложении теоретического материала автор стремился исключить из текста все второстепенное, с тем чтобы сконцентрировать внимание читателя на основньιx законах волновых процессов и, в частности, на вопросах наиболее трудных для понимания. Стремление изложить основные идеи кратко, доступно и вместе с тем достаточно корректно побудило автора насколько возможно освободить материал от излишней математизации и перенести основной акцент на физическую сторону рассматриваемых явлений. Именно поэтому из двух форм представления комплексной амплитуды автор предпочел векторную, как более простую и наглядную. С той же целью широко использованы различные модельные представления, упрощающие факторы, частные случаи, соображения симметрии и др.

Изложение ведется в СИ. Вместе с тем, учитывая достаточно широкое использование системы Гаусса, в Приложении дана сводка основных единиц и наиболее важных формул в СИ и в системе Гаусса.

Курсивом выделены важнейшие положения и термины. Петит используется для материала повышенной трудности и относительно громоздких расчетов (этот материал при первом чтении можно безболезненно опустить), а также для примеров и задач.

В данном издании материал книги дополнен двумя параграфами: § 1.6. Звуковые волны и § 3.4. Фотометрические величины. Дополнены также Приложения. Внесены некоторые уточнения, а также исправлены замеченные ошибки и опечатки.

Книга как учебное пособие рассчитана на студентов вузов с расширенной программой по физике (в рамках курса общей физики). Она может быть полезной и преподавателям вузов.\\
И. Иродов.

\section*{Принятые обозначения}
Векторы обозначены полужирным прямым шрифтом (например, \(\mathbf{v}\), \(\mathbf{E )}\); та же буква курсивом и светлым шрифтом ( \(v, E\) ) означает модуль вектора.

Средние величины отмечены скобками < >, например <λ>, <Π>.\\
Символы перед величинами означают:\\
\(\Delta\) - конечное приращение величины, т. е. разность ее конечного и начального значений, например \(\Delta \varphi=\varphi_{2}-\varphi_{1}, \Delta \mathbf{E}=\mathbf{E}_{2}-\mathbf{E}_{1}\);\\
d - дифференциал (бесконечно малое приращение), например, \(\mathrm{d} \varphi, \mathrm{dk}\).\\
\(\delta\) - элементарное значение величины, например \(\delta \lambda\);\\
ა - знак пропорциональности;\\
\(\sim\) - величина порядка... ( \(\lambda \sim 10^{-8}\) см).\\
Орты - единичные векторы:\\
\(\mathbf{e}_{x}, \mathbf{e}_{y}, \mathbf{e}_{z}\) (или \(\mathbf{i}, \mathbf{j}, \mathbf{k}\) ) - орты декартовых координат;\\
\(\mathbf{e}_{r}\) - орт радиуса-вектора;\\
\(\mathbf{n}\) - орт нормали к элементу поверхности;\\
\(\tau\) - орт касательной к контуру или границе раздела.\\
Производная по времени от произвольной функции \(x\) обозначена \(\mathrm{d} x / \mathrm{d} t\) или точкой над функцией, \(\dot{x}\). То же для второй производной: \(\mathrm{d}^{2} x / \mathrm{d} t^{2}\) или \(\ddot{x}\).

Интегралы любой кратности обозначены одним-единственным знаком \(\int\) и различаются лишь обозначением элемента интегрирования: \(\mathrm{d} V\) - элемент объема, \(\mathrm{d} \mathbf{S}\) - элемент поверхности, \(\mathrm{d} \mathbf{l}\) - элемент контура. Знак \(\oint\) обозначает интегрирование по замкнутой поверхности или по замкнутому контуру.

Векторный оператор \(\nabla\) (набла). Операции с ним обозначены так:\\
\(\nabla \varphi\) - градиент \(\varphi(\operatorname{grad} \varphi)\),\\
\(\nabla \cdot \mathbf{E}\) - дивергенция \(\mathbf{E}(\operatorname{div} \mathbf{E})\),\\
\(\boldsymbol{\nabla} \times \mathbf{E}\) - ротор \(\mathbf{E}(\operatorname{rot} \mathbf{E})\).

\section*{Часть 1 \\
 Волны}
Глава 1\\
Упругие волны

\section*{§ 1.1. Уравнение волны}
Упругой волной называют процесс распространения возмущения в упругой среде. При этом происходит распространение именно возмущения частиц среды, но сами частицы испытывают движения около своих положений равновесия. Среду будем рассматривать как сплошную и непрерывную, отвлекаясь от ее атомистического строения.

Различают волны продольные и поперечные, в зависимости от того, движутся ли частицы около своих положений равновесия вдоль или поперек направления распространения волны.

Уравнение волны. Несмотря на большое разнообразие физических процессов, вызывающих волны, их образование происходит по общему принципу. Возмущение, происшедшее в какой-нибудь точке \(A\) среды в некоторый момент времени, проявляется спустя определенное время на интересующем нас расстоянии от точки \(A\), т. е. передается с определенной скоростью.

Рассмотрим для простоты распространение возмущения вдоль длинного натянутого шнура, с которым совместим ось \(X\). Мы можем представить возмущение \(\xi\) - смещение элементов шнура из положения равновесия - как функцию координаты \(x\) и времени \(t\), т.е. \(\xi=f(x, t)\). Легко видеть, что распространение возмущения со скоростью \(v\) в положительном направлении оси \(X\) изобразится той же функцией \(f\), если в ее аргумент \(x\) и \(t\) будут входить в виде комбинации ( \(v t-x\) ) или ( \(t-x / v\) ). Действительно, такое строение аргумента показывает, что значение функции \(f\), которое она имела в точке \(x\) в момент \(t\), будет в дальней-

шем сохраняться, если \(v t-x=\) const. Но это так и есть, поскольку именно при этом условии \(\mathrm{d} x / \mathrm{d} t=v\).

Итак, любая функция от аргумента ( \(v t-x\) ) или ( \(t-x / v\) ) выражает распространение возмущения со скоростью \(v\) :


\begin{equation*}
\xi(x, t)=f(t-x / v) . \tag{1.1}
\end{equation*}


Это и есть уравнение волны, распространяющейся в положительном направлении оси \(X\). Волна же, распространяющаяся в отрицательном направлении \(X\), описывается уравнением


\begin{equation*}
\xi(x, t)=f(t+x / v) . \tag{1.2}
\end{equation*}


Особую роль среди различных волн играет гармоническая волна. Во многих отношениях это простейшее волновое движение и его выделенность связана с особыми свойствами гармонических осцилляторов. Уравнение гармонической волны имеет вид


\begin{equation*}
\xi(x, t)=a \cos \omega(t-x / v), \tag{1.3}
\end{equation*}


где \(a\) - амплитуда волны, \(\omega\) - циклическая (круговая) частота колебаний частиц среды ( \(\mathrm{c}^{-1}\) ). Эта волна периодична во времени и пространстве, поскольку сама функция периодична и ее период равен \(2 \pi\). Из периодичности во времени \(\omega \Delta t=2 \pi\) находим \(\Delta t=2 \pi / \omega\). Этот промежуток времени называют периодом колебаний:


\begin{equation*}
T=2 \pi / \omega . \tag{1.4}
\end{equation*}


Из периодичности в пространстве \(\omega \Delta x / v=2 \pi\) находим \(\Delta x= =2 \pi v / \omega=v T\). Расстояние \(\Delta x\) называют длиной волны \(\lambda\). Таким образом, длина волны - это расстояние между ближайшими точками среды, колеблющимися с разностью фаз \(2 \pi\). Другими словами, это расстояние, на которое распространяется волна за время, равное периоду колебаний \(T\) :


\begin{equation*}
\lambda=v T . \tag{1.5}
\end{equation*}


Поскольку \(T=1 / v\), где \(v\) - частота колебаний (Гц), формулу (1.5) можно представить и так:


\begin{equation*}
\lambda=v / v . \tag{1.6}
\end{equation*}


Уравнение гармонической волны (1.3) принято записывать в симметричном более удобном и простом виде. Для этого внесем ω в скобку, тогда

\[
\omega t-\omega x / v=\omega t-k x,
\]

где \(k=\omega / v=2 \pi / T v\), или


\begin{equation*}
k=2 \pi / \lambda . \tag{1.7}
\end{equation*}


Величину \(k\) называют волновым числом.\\
Тогда уравнение (1.3) примет следующий симметричный вид:


\begin{equation*}
\xi=a \cos (\omega t-k x) . \tag{1.8}
\end{equation*}


Отметим, что фигурирующая выше скорость \(v\) - это фазовая скорость волны,


\begin{equation*}
v=\omega / k, \tag{\(\prime\)}
\end{equation*}


т. е. скорость, с которой распространяется определенное значение фазы волны - величины в скобках формул (1.1), (1.2), (1.8). Именно фаза характеризует определенное состояние движения частиц среды при прохождении волны.

Вообще говоря, в фазу волны должна быть включена и начальная фаза \(\alpha\), определяемая выбором начал отсчета \(x\) и \(t\). В случае одной волны всегда можно добиться того, чтобы \(\alpha\) была равна нулю, что мы и предполагаем. При совместном же действии нескольких волн это сделать, как правило, не удается.

До сих пор предполагалось, что волна распространяется в непоглощающей упругой среде, поэтому ее амплитуда \(a=\) const. C учетом же поглощения амплитуда волны, как показывает опыт, уменьшается с расстоянием \(x\) по закону \(a=a_{0} \mathrm{e}^{-\gamma x}\), где \(\gamma\) - коэффициент затухания волны ( \(\mathrm{m}^{-1}\) ), и уравнение волны будет иметь вид:


\begin{equation*}
\xi=a_{0} \mathrm{e}^{-\gamma x} \cos (\omega t-k x) . \tag{1.9}
\end{equation*}


Уравнение плоской волны. Уравнения (1.1), (1.2), (1.8) описывают и плоскую волну в упругой среде. В плоской волне волновые поверхности (где точки среды колеблются в одинаковой

фазе) имеют вид плоскостей. Когда говорят, что плоская волна распространяется вдоль оси \(X\), то это надо понимать так, что ее волновые поверхности (плоскости) перпендикулярны этой оси.

Если же плоская волна распространяется в произвольном направлении, которое характеризуется единичным вектором \(\mathbf{n}\) (рис. 1.1), то


\begin{equation*}
\xi=f(t-l / v)=f(t-\mathrm{rn} / v), \tag{1.10}
\end{equation*}


где rn \(=x \cos \alpha+y \cos \beta+z \cos \gamma, \alpha, \beta, \gamma\) - углы между вектором n и осями координат.

\begin{figure}[h]
\begin{center}
  \includegraphics[alt={},max width=\textwidth]{950fefba-d83b-4c8f-96e7-0bd5a7eb90be-011_320_494_630_369}
\captionsetup{labelformat=empty}
\caption{Рис. 1.1}
\end{center}
\end{figure}

Для гармонической волны \(\cos \omega(t-\mathbf{n r} / v)=\cos (\omega t-\mathbf{r n} \omega / v)\) и


\begin{equation*}
\xi=a \cos (\omega t-\mathrm{kr}), \tag{1.11}
\end{equation*}


где k - волновой вектор:


\begin{equation*}
\mathrm{k}=\frac{\omega}{v} \mathrm{n}=\frac{2 \pi}{\lambda} \mathrm{n} . \tag{1.12}
\end{equation*}


До сих пор, говоря о фазовой скорости \(v\), мы имели в виду (как обычно и предполагают, если нет оговорок) скорость распространения данной фазы в направлении волнового вектора \(\mathbf{k}\), т. е. \(v=\omega / k\) согласно (1.8'). А как обстоит дело, если нас интересует скорость ее распространения в другом направлении, составляющем, например, угол φ с вектором \(\mathbf{k}\) ? Для ответа на этот вопрос воспользуемся уравнением волны (1.11).

Из условия, что фаза (выражение в скобках) должна быть постоянной, т. е. \(\omega t-\mathbf{k r}=\) const, следует после дифференцирования по \(t\)


\begin{equation*}
\omega=\mathbf{k v}_{\varphi}, \tag{1.12'}
\end{equation*}


где \(\mathbf{v}_{\varphi}=\mathrm{d} \mathbf{r} / \mathrm{d} t\) - скорость распространения фазы в интересующем нас направлении. Из (1.12') следует, что \(\omega=k v_{\varphi} \cos \varphi\) и


\begin{equation*}
v_{\varphi}=\frac{\omega / k}{\cos \varphi}=\frac{v}{\cos \varphi}, \tag{\(\prime\prime\)}
\end{equation*}


где \(v=\omega / k\) - фазовая скорость в направлении вектора \(\mathbf{k}\), т. е. при \(\varphi=0\).

Таким образом, если фазовая скорость плоской волны в направлении вектора \(\mathbf{k}\) равна \(v\), то в направлениях осей \(X, Y, Z\), с которыми вектор \(\mathbf{k}\) составляет углы \(\alpha, \beta, \gamma\) соответственно, скорости распространения данной фазы будут равны*:

\[
v_{\alpha}=v / \cos \alpha, \quad v_{\beta}=v / \cos \beta, \quad v_{\gamma}=v / \cos \gamma .
\]

Это позволяет утверждать, что фазовая скорость \(v\) не является вектором. В противном случае мы имели бы \(v_{x}=v \cos \alpha\) и т. д.

Из последних трех выражений следует, что при достаточно малом значении какого-либо из углов \(\alpha, \beta, \gamma\) соответствующая фазовая скорость может оказаться больше скорости света \(c\). Это не должно вызывать недоумения, поскольку такие фазовые скорости не связаны ни со скоростью частиц, ни со скоростью переноса информации, и теории относительности не противоречат.

При распространении волны в поглощающей среде в уравнения (1.10) и (1.11) нужно добавить экспоненциальный множитель \(\mathrm{e}^{-\gamma l}=\mathrm{e}^{-\gamma \mathbf{n r}}\).

Сферическая и цилиндрическая волны. В однородной изотропной среде продольная волна от точечного источника представляет собой сферически расходящееся возмущение вида


\begin{equation*}
\xi=\frac{1}{r} f(t-r / v), \tag{1.13}
\end{equation*}


где \(r\) - расстояние от точечного источника. В частности, если источник возбуждает продольные монохроматические колебания, то предыдущее уравнение принимает вид


\begin{equation*}
\xi=\frac{a_{0}}{r} \cos (\omega t-k r) \tag{1.14}
\end{equation*}


\footnotetext{\begin{itemize}
  \item Этотже результат можно получить и из более общей формулы (1.10) для плоской волны. Надо только учесть, что под \(v\) здесь понимается фазовая скорость в направлении орта \(\mathbf{n}\).
\end{itemize}
}где \(a_{0}\) - постоянная, \(a_{0} / r\) - амплитуда волны. Ее волновые поверхности являются сферическими. Отметим, что в выражении (1.14) стоит именно \(k\) (волновое число), а не волновой вектор \(\mathbf{k}\), как для плоской гармонической волны.

Если учитывать поглощение среды, то в формулы (1.13) и (1.14) следует добавить множитель \(\mathrm{e}^{-\gamma r}\).

Интересно, что при прохождении сферической волны в каждой точке среды всегда наблюдаются как сгущения, так и разряжения (в отличие от плоской волны, которая может состоять только из одних сгущений или разряжений).

Другой важный вид симметричной волны - цилиндрическая, расходящаяся например от источников, равномерно распределенных вдоль оси в однородной среде. Структура цилиндрической волны значительно сложнее сферической, и ее форма не повторяет временного поведения функции источника, как в случае сферической, - волна тянет за собой длинный «шлейф». И только на больших расстояниях \(R\) от источника (больших по сравнению с характерным параметром данной волны) ее можно представить в виде


\begin{equation*}
\xi=\frac{1}{\sqrt{R}} f(t-R / v) . \tag{1.15}
\end{equation*}


В частности, монохроматическая расходящаяся волна на расстояниях \(R\), значительно превышающих ее длину волны, имеет вид


\begin{equation*}
\xi=\frac{a}{\sqrt{R}} \cos (\omega t-k R), \tag{1.16}
\end{equation*}


где \(a\) - постоянная. Цилиндрическая волна, как и сферическая, непременно должна содержать как сгущения, так и разряжения.

\section*{§ 1.2. Волновые уравнения}
Линейное волновое уравнение. Аналогично основному уравнению динамики, которое описывает все возможные движения материальной точки, и здесь, в области волновых процессов, существуют уравнения, являющиеся обобщенным выражением

волн, независимо от их конкретного вида. Это дифференциальные уравнения в частных производных, связывающие изменения функций, характеризующих волну, во времени и пространстве.

Найдем эту связь для волн типа \(\xi=f(t-x / v)\). Обозначим фазу волны буквой \(\varphi\), т.е. \(\varphi=t-x / v\). Тогда


\begin{equation*}
\frac{\partial \xi}{\partial t}=\frac{\partial \xi}{\partial \varphi} \frac{\partial \varphi}{\partial t}=\xi_{\varphi}^{\prime} \cdot 1 ; \quad \frac{\partial \xi}{\partial x}=\frac{\partial \xi}{\partial \varphi} \frac{\partial \varphi}{\partial x}=\xi_{\varphi}^{\prime}\left(-\frac{1}{v}\right)=-\xi_{\varphi}^{\prime} / v . \tag{1.17}
\end{equation*}


Сопоставив полученные выражения, получим


\begin{equation*}
\frac{\partial \xi}{\partial x}=-\frac{1}{v} \frac{\partial \xi}{\partial t} . \tag{1.18}
\end{equation*}


Это уравнение справедливо, к сожалению, только для волн, распространяющихся в положительном направлении оси \(X\). Для волн, распространяющихся в отрицательном направлении оси \(X\), справа, как нетрудно проверить, должен стоять знак «+».

Таким образом, можно написать


\begin{equation*}
\frac{\partial \xi}{\partial x}=\mp \frac{1}{v} \frac{\partial \xi}{\partial t}, \tag{1.19}
\end{equation*}


где знаки «-» и «+» относятся только к тем волнам, которые распространяются соответственно в положительном или отрицательном направлении оси \(X\).

Уравнение (1.19) является простейшим волновым уравнением. Во многих случаях оно оказывается весьма полезным.

Выясним физический смысл производных, входящих в это волновое уравнение. Производная по времени \(\partial \xi / \partial t=u_{x}\) - это проекция скорости частицы среды, движущейся около своего положения равновесия, а \(\partial \xi / \partial x=\varepsilon\) - относительная деформация среды. Последнее надо пояснить.

Выделим мысленно малый (по сравнению с изменением профиля волны) цилиндрический элемент среды \(\Delta x\) (рис. 1.2) вдоль направления распространения волны. При прохождении продольной волны

\begin{figure}[h]
\begin{center}
  \includegraphics[alt={},max width=\textwidth]{950fefba-d83b-4c8f-96e7-0bd5a7eb90be-014_204_441_1581_711}
\captionsetup{labelformat=empty}
\caption{Рис. 1.2}
\end{center}
\end{figure}

этот элемент будет смещаться и деформироваться. Например, левый его торец переместится на \(\xi\), а правый - на \(\xi+\Delta \xi\). По определению, относительная деформация


\begin{equation*}
\varepsilon=\lim _{\Delta x \rightarrow 0} \frac{\Delta \xi}{\Delta x}=\frac{\partial \xi}{\partial x} . \tag{1.20}
\end{equation*}


Эта величина алгебраическая, она может быть больше нуля (растяжение), равна нулю и меньше нуля (сжатие).

Пример. Продольная волна распространяется в длинном стержне (ось \(X)\). В некоторый момент смещения частиц из положения равновесия \(\xi(x)\) имеют вид как на рис. 1.3. Зная, что волна распространяется в положительном направлении оси \(X\), найдем (качественно) зависимость скорости частиц среды в этот момент от координаты \(x\).

\begin{figure}[h]
\begin{center}
  \includegraphics[alt={},max width=\textwidth]{950fefba-d83b-4c8f-96e7-0bd5a7eb90be-015_374_598_794_383}
\captionsetup{labelformat=empty}
\caption{Рис. 1.3}
\end{center}
\end{figure}

Мы знаем, что \(\partial \xi / \partial t\) зависит от \(\partial \xi / \partial x\), согласно уравнению (1.19). Имея в виду, что производная \(\partial \xi / \partial x\) в каждой точке характеризует наклон (или крутизну) кривой \(\xi(x)\), изобразим график \(\partial \xi / \partial x\) как функцию \(x\) (штриховой линией).\\
Поскольку волна распространяется в положительном направлении оси \(X\), в уравнении (1.19) должен быть знак «-». Это означает, что график \(\partial \xi / \partial t(x)\) будет «зеркальным» по отношению к графику \(\partial \xi / \partial x\). Он изображен точечной кривой.\\
Общее волновое уравнение. Уравнение (1.19) соответствует волне, распространяющейся или в положительном направлении оси \(X\) (знак «-»), или в отрицательном (знак «+»). Можно, однако, получить уравнение, справедливое для волны любого направления, а также и для суперпозиции таких волн. Для этого продифференцируем выражения (1.17) еще раз по \(t\) и по \(x\) соответственно:

\[
\begin{gathered}
\frac{\partial^{2} \xi}{\partial t^{2}}=\frac{\partial}{\partial t}\left(\frac{\partial \xi}{\partial t}\right)=\frac{\partial}{\partial t}\left(\xi_{\varphi}^{\prime}\right)=\frac{\partial \xi_{\varphi}^{\prime}}{\partial \varphi} \frac{\partial \varphi}{\partial t}=\xi_{\varphi}^{\prime \prime} \\
\frac{\partial^{2} \xi}{\partial x^{2}}=-\frac{1}{v} \frac{\partial}{\partial x}\left(\xi_{\varphi}^{\prime}\right)=-\frac{1}{v} \frac{\partial \xi_{\varphi}^{\prime}}{\partial \varphi} \frac{\partial \varphi}{\partial x}=-\frac{1}{v} \xi_{\varphi}^{\prime \prime}\left(-\frac{1}{v}\right)=\frac{1}{v^{2}} \xi_{\varphi}^{\prime \prime} .
\end{gathered}
\]

Из сопоставления этих выражений получим:


\begin{equation*}
\frac{\partial^{2} \xi}{\partial x^{2}}=\frac{1}{v^{2}} \frac{\partial^{2} \xi}{\partial t^{2}} \tag{1.21}
\end{equation*}


где \(v\) - фазовая скорость. Это одномерное волновое уравнение 2 -го порядка в частных производных. Ему удовлетворяют как возмущения вида (1.1), (1.2), так и более общее решение


\begin{equation*}
\xi=f_{1}(t-x / v)+f_{2}(t+x / v), \tag{1.22}
\end{equation*}


где \(f_{1}\) и \(f_{2}\) - произвольные функции, соответствующие волнам, распространяющимся в противоположных направлениях оси \(X\).

Заметим, что волновые уравнения (1.19) и (1.21) справедливы для однородных изотропных сред, затухание в которых пренебрежимо мало*, и при условии \(\xi \ll \lambda\).

Обобщение уравнения (1.21) на трехмерный случай приводит к волновому уравнению вида


\begin{equation*}
\nabla^{2} \xi=\frac{1}{v^{2}} \frac{\partial^{2} \xi}{\partial t^{2}} \tag{1.23}
\end{equation*}


где \(\nabla^{2} \xi=\frac{\partial^{2} \xi}{\partial x^{2}}+\frac{\partial^{2} \xi}{\partial y^{2}}+\frac{\partial^{2} \xi}{\partial z^{2}}, \nabla^{2}\) - оператор Лапласа.\\
Уравнение (1.23) можно получить так. Обратимся сначала к уравнению (1.21). В нем \(v\) - это фазовая скорость волны, распространяющейся вдоль оси \(X\). Это значит, что в случае плоской волны, распро-

\footnotetext{\begin{itemize}
  \item При наличии затухания одномерное волновое уравнение имеет вид
\end{itemize}

\[
\frac{\partial^{2} \xi}{\partial x^{2}}+2 \gamma \frac{\partial \xi}{\partial x}+\gamma^{2} \xi=\frac{1}{v^{2}} \frac{\partial^{2} \xi}{\partial t^{2}}
\]
}где \(\gamma\) - коэффициент затухания волны.

страняющейся в произвольном направлении, как было отмечено на с. 11 под формулой ( \(1.12^{\prime \prime}\) ), фазовая скорость в направлении оси \(X\) равна \(v / \cos \alpha\). Тогда волновые уравнения вдоль осей \(X, Y, Z\) будут иметь следующий вид

\[
\frac{\partial^{2} \xi}{\partial x^{2}}=\frac{\cos ^{2} \alpha}{v^{2}} \frac{\partial^{2} \xi}{\partial t^{2}}, \quad \frac{\partial^{2} \xi}{\partial y^{2}}=\frac{\cos ^{2} \beta}{v^{2}} \frac{\partial^{2} \xi}{\partial t^{2}}, \quad \frac{\partial^{2} \xi}{\partial z^{2}}=\frac{\cos ^{2} \gamma}{v^{2}} \frac{\partial^{2} \xi}{\partial t^{2}},
\]

где \(\alpha, \beta, \gamma\) - углы между волновым вектором \(\mathbf{k}\) и осями координат. Сложив отдельно левые и правые части этих уравнений и учитывая, что

\[
\cos ^{2} \alpha+\cos ^{2} \beta+\cos ^{2} \gamma=1
\]

мы и приходим к уравнению (1.23), где, напомним, \(v=\omega / k\).\\
Отметим, что выражения для сферических и цилиндрических волн также являются решением волнового уравнения (1.23), надо только вторые производные по координатам, т. е. \(\nabla^{2}\) - оператор Лапласа записать в соответствующей системе координат (сферической или цилиндрической).

Волновые уравнения (1.21) и (1.23) играют весьма важную роль в теории волновых процессов. Если мы, исходя из законов механики при изучении некоторого явления, придем, например, к уравнению вида (1.23), то сразу можно утверждать, что имеем дело с волновым процессом, скорость распространения \(v\) которого легко найти из сопоставления полученного уравнения с (1.23). В дальнейшем мы этим и воспользуемся.

\section*{§ 1.3. Скорость упругих волн}
Скорость волны в тонком стержне. Под тонким имеется в виду стержень, толщина которого мала по сравнению с длиной волны \(\lambda\). При малых продольных деформациях стержня справедлив закон Гука:


\begin{equation*}
\sigma=E \varepsilon, \tag{1.24}
\end{equation*}


где \(\sigma\) - напряжение \(\left(\mathrm{H} / \mathrm{m}^{2}\right), E\) - модуль Юнга \((П a), \varepsilon=\partial \xi / \partial x\). Заметим, что \(\sigma\), как и \(\varepsilon\), величина алгебраическая, и знаки \(\sigma\) и \(\varepsilon\) всегда одинаковы: при растяжении - положительные, при сжатии - отрицательные.

\begin{figure}[h]
\begin{center}
  \includegraphics[alt={},max width=\textwidth]{950fefba-d83b-4c8f-96e7-0bd5a7eb90be-018_162_564_155_334}
\captionsetup{labelformat=empty}
\caption{Рис. 1.4}
\end{center}
\end{figure}

Рассмотрим малый элемент стержня \(\Delta x \ll \lambda\) в момент, когда при прохождении волны он оказался, например, в растянутом состоянии (рис. 1.4). Применим к этому элементу 2 -й закон Ньютона:

\[
\rho \Delta x S \cdot \ddot{\xi}=F_{x}(x+\Delta x)+F_{x}(x),
\]

где \(\rho\) - плотность материала стержня, \(S\) - площадь его поперечного сечения. В данный момент, как видно из рисунка, \(F_{x}(x+\Delta x)>0\), а \(F_{x}(x)<0\). Соответствующие же значения б в сечениях \(x\) и \(x+\Delta x\) положительные (растяжение!). Поэтому правую часть уравнения можно переписать так:

\[
F_{x}(x+\Delta x)+F_{x}(x)=S \sigma(x+\Delta x)-S \sigma(x)=S \frac{\partial \sigma}{\partial x} \Delta x,
\]

где учтено, что слева \(F_{x}\) и \(\sigma\) имеют разные знаки (это будет и при сжатии). Тогда уравнение движения после сокращения на \(\Delta x \cdot S\) примет вид \(\rho \ddot{\xi}=\partial \sigma / \partial x\). Остается учесть (1.24), после чего получим окончательно:


\begin{equation*}
\rho \frac{\partial^{2} \xi}{\partial t^{2}}=E \frac{\partial^{2} \xi}{\partial x^{2}} . \tag{1.25}
\end{equation*}


Мы пришли, таким образом, к волновому уравнению. Это позволяет утверждать, что в стержне будет распространяться продольная волна, скорость \(v\) которой легко определить, сопоставив полученное выражение с (1.23):


\begin{equation*}
v=\sqrt{E / \rho} . \tag{1.26}
\end{equation*}


Заметим, что для не тонкого стержня выражение для \(v\) имеет более сложный вид и значение \(v\) оказывается больше, чем в случае тонкого стержня.

Можно показать, что скорость поперечных упругих волн в неограниченной изотропной твердой среде


\begin{equation*}
v=\sqrt{G / \rho}, \tag{1.27}
\end{equation*}


где \(G\) - модуль сдвига среды, \(\rho\) - ее плотность.

\begin{figure}[h]
\begin{center}
  \includegraphics[alt={},max width=\textwidth]{950fefba-d83b-4c8f-96e7-0bd5a7eb90be-019_301_483_153_375}
\captionsetup{labelformat=empty}
\caption{Рис. 1.5}
\end{center}
\end{figure}

Скорость волны в гибком шнуре. Найдем уравнение малых поперечных колебаний натянутого шнура, исходя из основного уравнения динамики. На малый элемент 12 шнура (рис. 1.5) слева действует сила натяжения \(F\). Ее вертикальная проекция \(F_{\xi}(x)=-F \sin \alpha\). При малых смещениях \(\sin \alpha=\operatorname{tg} \alpha=\partial \xi / \partial x\), и мы можем записать

\[
F_{\xi}(x)=-F \partial \xi / \partial x .
\]

Аналогичное выражение для проекции силы (только со знаком «+») можно записать и для правого конца элемента 12. Результирующая этих двух проекций сил


\begin{equation*}
\left(F \frac{\partial \xi}{\partial x}\right)_{2}+\left(-F \frac{\partial \xi}{\partial x}\right)_{1}=\frac{\partial}{\partial x}\left(F \frac{\partial \xi}{\partial x}\right) \mathrm{d} x . \tag{1.28}
\end{equation*}


Пренебрегая изменением силы \(F\) вдоль шнура (это справедливо для малых смещений при колебаниях), правую часть предыдущего выражения можно переписать так: \(F\left(\partial^{2} \xi / \partial x^{2}\right) \mathrm{d} x\). Если линейная плотность шнура (масса единицы его длины) равна \(\rho_{1}\), то по второму закону Ньютона \(\rho_{1} \mathrm{~d} x \cdot \ddot{\xi}=F \xi_{x}^{\prime \prime} \mathrm{d} x\), или


\begin{equation*}
\frac{\partial^{2} \xi}{\partial t^{2}}=\frac{F}{\rho_{1}} \frac{\partial^{2} \xi}{\partial x^{2}} . \tag{1.29}
\end{equation*}


Из сравнения с (1.23) находим выражение для скорости волны в шнуре:


\begin{equation*}
v=\sqrt{F / \rho_{1}} . \tag{1.30}
\end{equation*}


Скорость звука в жидкостях и газах. Формулу (1.26) можно использовать для вычисления скорости продольных волн в жидкостях и газах. Действительно, вырезав мысленно канал в на-

правлении распространения плоской волны, мы можем повторить все рассуждения, приведшие нас к этой формуле. Остается только выяснить, что в этом случае играет роль модуля Юнга \(E\).

При продольных волнах в среде возникают сжатия и разряжения отдельных слоев, и закон Гука в данном случае - связь избыточного давления \(\Delta p\) с относительным изменением длины элемента \(\Delta x\) цилиндрического канала \(\Delta \xi / \Delta x\) - примет вид \(\Delta p=-E \Delta \xi / \Delta x\), где знак минус связан с тем, что приращения давления \(\Delta p\) и длины \(\Delta \xi\) противоположны по знаку. Умножив числитель и знаменатель на площадь поперечного сечения канала, получим


\begin{equation*}
\Delta p=-E \Delta V / V \tag{1.31}
\end{equation*}


где \(\Delta V / V\) - относительное приращение объема рассматриваемого элемента. Перейдя к пределу, получим


\begin{equation*}
E=-V \mathrm{~d} p / \mathrm{d} V \tag{1.32}
\end{equation*}


Объем \(V\) элемента \(\Delta x\) и его плотность меняются при прохождении волны, но их произведение, т. е. масса \(\rho V=\) const. Отсюда \(\mathrm{d} \rho / \rho=-\mathrm{d} V / V\), значит


\begin{equation*}
\mathrm{d} V=-V \mathrm{~d} \rho / \rho . \tag{1.33}
\end{equation*}


После подстановки этого выражения в (1.32) получим \(E=\rho \mathrm{d} p / \mathrm{d} \rho\), и скорость волны - формула (1.26) - примет вид


\begin{equation*}
v=\sqrt{\mathrm{d} p / \mathrm{d} \rho} \tag{1.34}
\end{equation*}


Это выражение справедливо для волн в жидкостях и газах.\\
Опыт показывает, что при распространении звука в газе связь между давлением и объемом определяется уравнением


\begin{equation*}
p V^{\gamma}=\mathrm{const}, \tag{1.35}
\end{equation*}


где \(\gamma\) - так называемая постоянная адиабаты, равная отношению теплоемкостей газа при постоянных давлении и объеме, \(\gamma=C_{p} / C_{V}\) - величина, характерная для каждого газа. Запишем дифференциал натурального логарифма выражения (1.35):

\[
\frac{\mathrm{d} p}{p}+\gamma \frac{\mathrm{d} V}{V}=0
\]

откуда \(\mathrm{d} p / \mathrm{d} V=-\gamma p / V\), и формула (1.32) принимает вид


\begin{equation*}
E=\gamma p . \tag{1.36}
\end{equation*}


Таким образом, скорость звуковой волны в газе


\begin{equation*}
v=\sqrt{\gamma p / \rho} \tag{1.37}
\end{equation*}


Это выражение можно преобразовать к более удобному для расчетов виду, если учесть уравнение состояния идеального газа \(p V=(m / M) R T\), где, напомним, \(m\) - масса газа, \(M\) - его молярная масса. Из уравнения состояния определим плотность как \(\rho=m / V=p M / R T\), и уравнение (1.37) станет таким:


\begin{equation*}
v=\sqrt{\gamma R T / M}, \tag{1.38}
\end{equation*}


где \(R\) - универсальная газовая постоянная.

\section*{§ 1.4. Энергия упругой волны}
Плотность энергии упругой волны. Прежде всего найдем выражение для плотности упругой (потенциальной) энергии растянутого (или сжатого) стержня. Приложим к торцу стержня, другой конец которого закреплен, растягивающую силу \(F(x)\) и будем медленно увеличивать ее от 0 до значения \(F_{0}\). Удлинение стержня при этом будет меняться от 0 до \(x\). По закону Гука \(F(x)=x x\), где \(x\) - коэффициент упругости. Работа силы \(F(x)\) в этом процессе

\[
A=\int_{0}^{x} F(x) \mathrm{d} x=\varkappa \int_{0}^{x} x \mathrm{~d} x=\frac{\varkappa x^{2}}{2}
\]

Эта работа идет на увеличение упругой энергии \(U\) стержня, значит


\begin{equation*}
U=x x^{2} / 2 \tag{1.39}
\end{equation*}


Плотность же упругой энергии \(w_{\text {п }}=U / S l\), где \(S\) и \(l\) - площадь поперечного сечения и длина стержня. Преобразуем выражение (1.39), учитывая, что \(x x=F=\sigma S, \sigma=E \varepsilon\) и \(\varepsilon=x / l\), тогда

\[
U=\frac{F x}{2}=\frac{\sigma S \cdot \varepsilon l}{2}=\frac{E \varepsilon^{2}}{2} S l .
\]

Отсюда видно, что плотность упругой энергии


\begin{equation*}
w_{\text {п }}=E \mathrm{e}^{2} / 2 \tag{1.40}
\end{equation*}


При прохождении продольной волны в стержне каждая единица его объема обладает как потенциальной энергией упругой деформации \(w_{\text {п }}\), так и кинетической энергией \(w_{\text {к }}\). Плотность полной энергии


\begin{equation*}
w=w_{\text {к }}+w_{\text {п }}=\rho \dot{\xi}^{2} / 2+E \varepsilon^{2} / 2 . \tag{1.41}
\end{equation*}


Для тонкого стержня \(E=\rho v^{2}\), согласно (1.26), и выражение (1.41) можно переписать так:


\begin{equation*}
w=\frac{\rho}{2}\left[\left(\frac{\partial \xi}{\partial t}\right)^{2}+v^{2}\left(\frac{\partial \xi}{\partial x}\right)^{2}\right] . \tag{1.42}
\end{equation*}


Как следует из волнового уравнения (1.19), оба слагаемых равны друг другу, т. е. плотности кинетической и упругой энергии одинаковы и изменяются синфазно. Поэтому мы имеем в результате


\begin{equation*}
w=\rho \dot{\xi}^{2} . \tag{1.43}
\end{equation*}


В частности, для гармонической волны \(\xi=a \cos (\omega t-k x)\)


\begin{equation*}
w=\rho a^{2} \omega^{2} \sin ^{2}(\omega t-k x) . \tag{1.44}
\end{equation*}


Соответствующее распределение \(w(x)\) вдоль стержня в некоторый момент показано на рис. 1.6.

\begin{figure}[h]
\begin{center}
  \includegraphics[alt={},max width=\textwidth]{950fefba-d83b-4c8f-96e7-0bd5a7eb90be-022_333_633_1355_298}
\captionsetup{labelformat=empty}
\caption{Рис. 1.6}
\end{center}
\end{figure}

Среднее значение плотности энергии за период (или за время значительно большее периода колебаний) равно


\begin{equation*}
\langle w\rangle=\frac{1}{2} \rho a^{2} \omega^{2}, \tag{1.45}
\end{equation*}


поскольку среднее значение квадрата синуса равно \(1 / 2\).\\
В заключение отметим, что полученные выражения справедливы и для упругих волн в жидкостях и газах.

Плотность потока энергии. Так как энергия перемещается в среде вместе с возмущением, вводят понятие потока энер гии Ф. Это количество энергии, переносимое волной через определенную поверхность \(S\) в единицу времени:


\begin{equation*}
\Phi=\mathrm{d} W / \mathrm{d} t, \tag{1.46}
\end{equation*}


где \(\mathrm{d} W\) - энергия, переносимая через данную поверхность за время \(\mathrm{d} t\).

Поток энергии в разных точках поверхности \(S\) может иметь различную интенсивность. Для характеристики этого обстоятельства вводят понятие плотности потока энергии. Это поток энергии через единичную площадку, перпендикулярную к направлению переноса энергии:


\begin{equation*}
j=d \Phi / d S_{\perp}, \tag{1.47}
\end{equation*}


где \(\mathrm{d} \Phi=\mathrm{d} W / \mathrm{d} t\), а \(\mathrm{d} W\) - это энергия, заключенная внутри косого цилиндра (рис. 1.7) с основанием

\begin{figure}[h]
\begin{center}
  \includegraphics[alt={},max width=\textwidth]{950fefba-d83b-4c8f-96e7-0bd5a7eb90be-023_263_396_1250_68}
\captionsetup{labelformat=empty}
\caption{Рис. 1.7}
\end{center}
\end{figure}

площадью \(\mathrm{d} S\) и образующей длиной \(v \mathrm{~d} t\), где \(v\) - скорость переноса энергии (возмущения). Размеры этого цилиндра должны быть настолько малы, чтобы во всех его точках плотность энергии \(w\) была бы одинаковой. Тогда \(\mathrm{d} W=w \mathrm{~d} V, \mathrm{~d} V\) - объем данного цилиндра, и мы можем записать:

\[
\mathrm{d} W=w v \mathrm{~d} t \mathrm{~d} S \cos \varphi=w v \mathrm{~d} t \mathrm{~d} S_{\perp} .
\]

С учетом этого соотношения выражение (1.47) примет вид:


\begin{equation*}
j=w v . \tag{1.48}
\end{equation*}


Для определения плотности потока и его направления вводят вектор Умова ј:


\begin{equation*}
\mathbf{j}=w \mathbf{v}, \tag{1.49}
\end{equation*}


где \(\mathbf{v}\) - вектор скорости, нормальный к волновой поверхности в данном месте*. Для гармонической волны \(\mathbf{v}=(\omega / k) \mathbf{n}\).

В случае монохроматической волны вектор \(\mathbf{j}\), как и плотность энергии, изменяется со временем по закону квадрата синуса (1.44). Поэтому среднее по времени значение вектора Умова с учетом (1.45) можно записать как


\begin{equation*}
\langle\mathbf{j}\rangle=\frac{1}{2} \rho a^{2} \omega^{2} \mathbf{v} . \tag{1.50}
\end{equation*}


Это выражение справедливо для волн любого вида - плоской, сферической, цилиндрической, затухающих и др.

Среднее по времени значение плотности потока энергии называют интенсивностью волны: \(I=\langle j\rangle\).

Для монохроматической волны


\begin{equation*}
I=\frac{1}{2} \rho a^{2} \omega^{2} v, \tag{\(\prime\)}
\end{equation*}


где, напомним, \(a\) - амплитуда волны (например, для сферической волны с затуханием \(a=\left(a_{0} / r\right) \mathrm{e}^{-\gamma r}\) ).

Зная вектор Умова во всех точках интересующей нас поверхности \(S\), можно найти поток энергии сквозь эту поверхность. Для этого разобьем мысленно поверхность \(S\) на элементарные участки \(\mathrm{d} S\). Поток энергии через этот участок, согласно (1.47), есть

\[
\mathrm{d} \Phi=j \mathrm{~d} S_{\perp}=j \mathrm{~d} S \cos \varphi=\mathbf{j} \mathrm{d} \mathbf{S}=j_{n} \mathrm{~d} S
\]

где \(j_{n}\) - проекция вектора \(\mathbf{j}\) на нормаль \(\mathbf{n}\) к элементу поверхности \(\mathrm{d} S\) (см. рис. 1.7). Тогда полный поток энергии сквозь поверхность \(S\)


\begin{equation*}
\Phi=\int_{S} \mathbf{j} \mathrm{~d} \mathbf{S}=\int_{S} j_{n} \mathrm{~d} S, \tag{1.51}
\end{equation*}


здесь \(\mathrm{dS}=\mathbf{n d S}\). Выражение (1.51) означает, что поток энергии равен потоку вектора \(\mathbf{j}\) сквозь эту поверхность \(S\).

\footnotetext{\begin{itemize}
  \item Здесь \(v_{x}=(\omega / k) \cos \alpha, \alpha\) - угол между вектором \(\mathbf{j}\) и ортом \(\mathbf{i}\) оси \(X\). Это не фазовая скорость вдоль оси \(X\).
\end{itemize}
}Пример. Убедимся, что амплитуда \(a\) сферической волны действительно пропорциональна \(1 / r\).\\
Для этого найдем среднее значение потока энергии сквозь волновую поверхность радиуса \(r\). Поскольку во всех точках этой поверхности <j> одинаково, то определение среднего потока сводится просто к умножению 〈j〉 на площадь сферы:

\[
\langle\Phi\rangle=\langle j\rangle 4 \pi r^{2} \circlearrowleft a_{r}^{2} r^{2} .
\]

Если энергия волны не поглощается средой, то \(\langle\Phi\rangle\) не должно зависеть от \(r\), а значит \(a_{r}^{2} r^{2}=\) const. Отсюда и следует, что \(a_{r} \sim 1 / r\).\\
Необходимо отметить, что полученное выражение (1.49) справедливо только для бегущей волны. Если же мы имеем дело с более сложным образованием - суперпозицией (наложением) нескольких продольных волн, выражению для вектора Умова (1.49) следует придать другой вид:


\begin{equation*}
\mathbf{j}=-\sigma \mathbf{u} \tag{1.52}
\end{equation*}


где \(\sigma\) - напряжение (или избыточное давление), \(\mathbf{u}\) - скорость частиц среды (не скорость волны!). Это выражение справедливо для жидких и газообразных сред, для твердых же сред, строго говоря, - только в случае тонкого стержня или тонкой пластины.

Выражение (1.52) можно получить так. Пусть возмущение распространяется, например, в положительном направлении оси \(X\). Тогда векторное равенство (1.49) в проекции на ось \(X\) примет вид \(j_{x}=w v\). Так как, согласно (1.43), \(w=\rho \dot{\xi}^{2}\) и \(v=\sqrt{E / \rho}\), то

\[
j_{x}=\frac{w}{v} v^{2}=\frac{1}{v} \rho\left(\frac{\partial \xi}{\partial t}\right)^{2} \frac{E}{\rho}=\left(\frac{1}{v} \frac{\partial \xi}{\partial t}\right) \frac{\partial \xi}{\partial t} E .
\]

Выражение в последних скобках, согласно (1.19), равно \(-\partial \xi / \partial x\) (для волны, распространяющейся в положительном направлении оси \(X\) ).

Значит


\begin{equation*}
j_{x}=-E \frac{\partial \xi}{\partial x} \frac{\partial \xi}{\partial t}=-\sigma u_{x} \tag{1.53}
\end{equation*}


откуда и следует (1.52).

Отметим, что для волны, распространяющейся в положительном направлении оси \(X\), в любой момент величины \(\sigma\) и \(u_{x}\) противоположны по знаку, согласно (1.18), и значит все время \(j_{x}>0\), как и должно быть в данном случае.

\section*{§ 1.5. Стоячие волны}
Уравнение стоячей волны. При распространении в упругой среде одновременно нескольких волн возникает их наложение, причем волны не возмущают друг друга: колебания частиц среды оказываются векторной суммой колебаний, которые совершали бы частицы при распространении каждой из волн в отдельности. Это называют принципом суперпозиции (наложения) волн.

Рассмотрим практически важный случай, когда две гармонические волны с одинаковыми частотой \(\omega\) и амплитудой \(a\) распространяются в противоположных направлениях оси \(X\) :

\[
\xi_{1}=a \cos (\omega t-k x) \quad \text { и } \quad \xi_{2}=a \cos (\omega t+k x) .
\]

Чтобы не усложнять формулы, начала отсчета времени и координаты выбраны так, чтобы добавочные фазы \(\alpha_{1}\) и \(\alpha_{2}\) были равны нулю.

Суперпозиция этих волн дает:


\begin{equation*}
\xi=\xi_{1}+\xi_{2}=A \cos k x \cdot \cos \omega t, \quad \text { где } A=2 a . \tag{1.54}
\end{equation*}


Это и есть уравнение стоячей волны. Видно, что ее частота та же, т. е. ω, а амплитуда равна \(|A \cos k x|\) и, в отличие от бегущей гармонической волны, зависит от \(x\). В точках, где \(|\cos k x|=1\), мы имеем максимумы - пучности, а где \(\cos k x=0\), - минимумы узльl. Период \(|\cos k x|\) равен \(\pi\), поэтому \(k \Delta x=\pi\) и \(\Delta x=\pi / k=\lambda / 2\). Т. е. интервалы между соседними пучностями или узлами равны половине длины волны (рис. 1.8, где показаны крайние смещения \(\xi\) через половину периода).

\begin{figure}[h]
\begin{center}
  \includegraphics[alt={},max width=\textwidth]{950fefba-d83b-4c8f-96e7-0bd5a7eb90be-026_188_471_1577_379}
\captionsetup{labelformat=empty}
\caption{Рис. 1.8}
\end{center}
\end{figure}

Между двумя соседними узлами все точки среды колеблются синфазно, при переходе же через узел фаза изменяется на \(\pi\), т. е. колебания по разные стороны от узла (в пределах полуволны) происходят в противофазе. Узлы смещения как бы разделяют среду на автономные области, в которых гармонические колебания совершаются независимо. Никакой передачи движения из одной области к другой, а значит и перетекания энергии через узлы не происходит. Другими словами, нет никакого распространения возмущения вдоль оси \(X\). Именно поэтому возмущения, описываемые формулой (1.54), и называют стоячей волной.

Энергия стоячей волны. Переходя к распределению энергии в стоячей волне, определим сначала с помощью (1.54) выражение для скорости \(\dot{\xi}\) частиц среды и ее относительной деформации \(\varepsilon=\partial \xi / \partial x\) :


\begin{align*}
\dot{\xi} & =-A \omega \cos k x \cdot \sin \omega t, \\
\varepsilon & =-A k \sin k x \cdot \cos \omega t . \tag{1.55}
\end{align*}


Видно, что обе величины, \(\dot{\xi}\) и \(\varepsilon\), тоже стоячие волны, причем они сдвинуты относительно друг друга по фазе на \(\pi / 2\) - как в пространстве, так и во времени. Кроме того, узлы и пучности скорости \(\dot{\xi}\) частиц среды совпадают с узлами и пучностями их смещения \(\xi\). Узлы же и пучности деформации \(\varepsilon\) совпадают соответственно с пучностями и узлами смещения. Это показано на рис. 1.9 для моментов \(t=0\) и \(t=T / 4\), здесь узлы смещения отмечены жирными точками. В момент \(t=0\), когда \(\xi\) и \(\varepsilon\) ста-

\begin{figure}[h]
\begin{center}
  \includegraphics[alt={},max width=\textwidth]{950fefba-d83b-4c8f-96e7-0bd5a7eb90be-027_487_524_1294_356}
\captionsetup{labelformat=empty}
\caption{Рис. 1.9}
\end{center}
\end{figure}

новятся максимальными, скорость \(\dot{\xi}\) обращается в нуль, и наоборот ( \(t=T / 4\) ).

Соответственно происходят превращения энергии стоячей волны: то полностью в потенциальную (упругую), то полностью в кинетическую (аналогичное происходит при колебаниях маятника). На рис. 1.10 показано распределение плотности энергии в моменты \(t=0\) и \(t=T / 4\). В процессе колебаний происходит перетекание энергии от каждого узла к соседним с ним пучностям и обратно. Средний же по времени поток энергии в любом сечении стоячей волны равен нулю.

\begin{figure}[h]
\begin{center}
  \includegraphics[alt={},max width=\textwidth]{950fefba-d83b-4c8f-96e7-0bd5a7eb90be-028_321_660_597_284}
\captionsetup{labelformat=empty}
\caption{Рис. 1.10}
\end{center}
\end{figure}

Пример. В тонком стержне установилась продольная стоячая волна вида \(\xi=a \sin k x \cdot \sin \omega t\). Найдем проекцию вектора Умова на ось \(X\), взятую вдоль стержня.\\
Воспользовавшись формулой (1.52), запишем:

\[
\begin{aligned}
j_{x}=-E \frac{\partial \xi}{\partial x} \frac{\partial \xi}{\partial t} & =-E(a k \cos k x \cdot \sin \omega t)(a \omega \sin k x \cdot \cos \omega t)= \\
& =-\frac{1}{4} E a^{2} k \omega \sin 2 k x \cdot \sin 2 \omega t
\end{aligned}
\]

Видно, что \(j_{x}\) периодически меняет знак, а значит вектор \(\mathbf{j}\) направление. Но в любом сечении \(\left\langle j_{x}\right\rangle=0\), а в сечениях, где \(2 k x=n \pi, n\) - целое число, \(j_{x}=0\) постоянно. Эти сечения отстоят друг от друга на \(\lambda / 4\).\\
Колебания струны (стержня). В натянутой струне, закрепленной с обоих концов, при возбуждении какого-либо произвольного поперечного возмущения возникнет довольно сложное нестационарное движение. Стационарное же движение в виде стоячей волны возможно лишь при вполне определенных частотах. Это связано с тем, что на закрепленных концах струны должны выполняться определенные граничные условия: в них смещение \(\xi\) все время должно равняться нулю. Значит, если в стру-

не возбуждается стоячая волна, то концы струны должны быть ее узлами. Отсюда следует, что на длине струны \(l\) должно укладываться целое число \(n\) полуволн: \(l=n \cdot \lambda / 2\). Из этого условия находим возможные длины волн:


\begin{equation*}
\lambda_{n}=2 l / n, \quad n=1,2, \ldots \tag{1.56}
\end{equation*}


Соответствующие частоты


\begin{equation*}
v_{n}=\frac{v}{\lambda_{n}}=\frac{v}{2 l} n, \tag{1.57}
\end{equation*}


где \(v\) - фазовая скорость волны, определяемая, согласно (1.30), силой \(F\) натяжения струны и линейной плотностью \(\rho_{1}\), т. е. массой единицы ее длины.

Частоты \(v_{n}\) называют собственными частотами струны. Частоту \(v_{1}(n=1)\) называют основной частотой, остальные \(v_{2}, v_{3}, \ldots\) обертонами. Гармонические колебания с частотами (1.57) называют собственными колебаниями, или гармониками. В общем случае колебания струны представляют собой суперпозицию различных гармоник (спектр).

Колебания струны примечательны тем, что в рамках классической физики возникает дискретный спектр одной из величин (частоты). Такая дискретность для классической физики является исключением, в отличие от квантовой физики.

Приведенные выше соображения относятся не только к струне, но и к стержням, закрепленным различным образом в середине, на одном конце и т. д. Отличие заключается лишь в том, что свободный конец стержня является пучностью. Это касается как поперечных, так и продольных колебаний.\\
Пример. Найдем собственные частоты стержня, закрепленного на одном конце, если длина стержня \(l\), модуль Юнга материала стержня \(E\) и его плотность \(\rho\).\\
Поскольку свободный конец стержня должен быть пучностью, на длине стержня установится целое число полуволн и еще четверть волны, т. е. \(l=n \lambda / 2+\lambda / 4=(2 n+1) \lambda / 4\). Отсюда найдем возможные значения \(\lambda_{n}\), а затем, учитывая (1.26), и собственные частоты:

\[
v_{n}=v / \lambda_{n}=\sqrt{E / \rho}(2 n+1) / 4 l, \quad n=0,1,2, \ldots
\]

\section*{§ 1.6. Звуковые волны}
Звуковые волны (или просто звук) - это распространяющийся в упругой среде волновой процесс, воспринимаемый человеческим ухом в диапазоне частот от \(\sim 20\) Гц до 20 кГц. Упругие волны с частотами, меньшими 20 Гц, называют инфразвуком, а волны с частотами, превышающими 20 кГц, ультразвуком. Инфра- и ультразвуки человеческое ухо не слышит.

Звук различают по высоте, тембру и громкости.\\
Высота звука определяется его частотой: чем больше частота, тем выше звук.

Тембр звука определяется характером колебаний. Лишь в редких случаях звук представляет собой гармонические колебания, обычно он является наложением гармонических колебаний с определенным набором амплитуд и частот - акустический спектр. Этот спектр может состоять из непрерывного набора частот (сплошной спектр) или из набора колебаний дискретных частот \(v_{1}, v_{2}, \ldots\) (линейчатый спектр). Состав этого спектра и определяет тембр звука.

Громкость звука - это величина слухового ощущения, позволяющая располагать все звуки по шкале от тихих до громких. При неизменной частоте и форме колебаний громкость звука растет с увеличением его интенсивности \(I\), Вт/ \(\mathrm{M}^{2}\). Чтобы вызвать ощущение звука, его интенсивность должна превышать некоторую минимальную величину - порог сльшшимости. Этот порог различен для разных частот (рис. 1.11). Наиболее чувствительно человеческое ухо к колебаниям с частотами \(0,5 \div 5,0\) кГц. В этом диапазоне порог слышимости соответствует интенсивности звука \(I_{\text {пор }} \sim 10^{-12} \mathrm{Br} / \mathrm{M}^{2}\).

При интенсивности порядка \(1 \div 10 \mathrm{Br} / \mathrm{m}^{2}\) колебания перестают восприниматься как звук, вызывая в ушах ощущение давления и боли. Значение интенсивонсти, при котором это происходит, называют порогом болевого ощущения. Этот порог также зависит от частоты колебаний (см. рис. 1.11).

В связи с тем, что субъективно оцениваемая громкость звука возрастает гораздо медленнее роста интенсивности \(I\) звука, в качестве исходной используют шкалу \(L\) интенсивности звука, которую определяют как логарифм отношения интенсив-

\begin{figure}[h]
\begin{center}
  \includegraphics[alt={},max width=\textwidth]{950fefba-d83b-4c8f-96e7-0bd5a7eb90be-031_638_915_169_153}
\captionsetup{labelformat=empty}
\caption{Рис. 1.11}
\end{center}
\end{figure}

ности \(I\) звука к интенсивности \(I_{0}\), принимаемой за нулевой уровень \(L\) :


\begin{equation*}
L_{\mathrm{E}}=\lg \frac{I}{I_{0}}, \tag{1.58}
\end{equation*}


где \(L_{\text {Б }}\) - выраженный в белах (Б) уровень интенсивности \(I\) звука относительно \(I_{0}\). Условно считают, что \(I_{0}\) находится вблизи порога слышимости при частоте 1 кГц и имеет значение \(I_{0}=10^{-12} \mathrm{Bt} / \mathrm{M}^{2}\). Так что при \(I=I_{0}\) согласно (1.58) \(L_{\mathrm{E}}=0\).

Обычно пользуются не белами, а в 10 раз меньшей единицей - децибелом (дБ). В этих единицах уровень интенсивности звука


\begin{equation*}
L_{\text {дБ }}=10 \lg \frac{I}{I_{0}} . \tag{1.59}
\end{equation*}


Отношение двух интенсивностей ( \(I_{1} / I_{2}\) ) также может быть выражено в децибелах:


\begin{equation*}
\Delta L_{12}=10 \lg \frac{I_{1}}{I_{2}} \tag{1.60}
\end{equation*}


С помощью этой формулы можно представить в децибелах уменьшение интенсивности (затухание) звука на интересующем нас расстоянии. Например, затухание в 20 дБ соответствует уменьшению интенсивности \(I\) в 100 раз. Эту шкалу (в дБ) интенсивности используют и в различного рода измерительных приборах (таких, например, как звуковой генератор), в линиях передачи сигналов и др.

Для характеристики громкости \(\Gamma\) звука при разных частотах вводят единицу измерения фон. Условно считают, что уровень громкости \(\Gamma\) звука с частотой 1 кГц в фонах численно равен уровню интенсивности \(L\) в децибелах (см. рис. 1.11). На этом рисунке показаны кривые - это уровни равной громкости в фонах. Видно, что, например, звук частоты 100 Гц, соответствующий 50 дБ, на слух воспринимается таким же громким, как и звук частоты 1 кГц, соответствующий 40 дБ. Отсюда следует, что оба звука имеют одинаковую громкость 40 фон.

Диапазон интенсивности \(I\), вызывающий у нас звуковое ощущение, составляет при частоте \(\sim 1\) кГц от \(10^{-12}\) до \(\sim 1 \mathrm{Bt} / \mathrm{m}^{2}\). Это соответствует уровням громкости \(\Gamma\) от 0 до 120 фон. Таким образом, рис. 1.11 по существу представляет «окно» в окружающий нас мир звуков.

В табл. 1.1 приведены приближенные значения уровней громкости для некоторых звуков с частотой, близкой к 1 кГц.

\begin{table}[h]
\begin{center}
\captionsetup{labelformat=empty}
\caption{Таблица 1.1}
\begin{tabular}{|l|l|}
\hline
Звук & Уровень громкости \(\Gamma\), фон \\
\hline
Тихий шепот & 30 \\
\hline
Нормальный разговор & 50 \\
\hline
Громкая речь & 70 \\
\hline
Оркестр фортиссимо & 100 \\
\hline
Шум самолетного мотора на расстоянии 5 m & 120 \\
\hline
\end{tabular}
\end{center}
\end{table}

Рассмотрим два примера, связанные с громкостью звука.

Пример 1. Интенсивность звука с частотой \(v=150\) Гц уменьшается от значения \(10^{-4}\) до \(10^{-9}\) Вт/м \({ }^{2}\). Определим, на сколько фон уменьшилась при этом громкость звука.\\
С помощью рис. 1.11 находим:

\[
80-20=60 \text { фон. }
\]

Пример 2. Имеются три одинаковых источника звука частоты 1 кГц, расположенных в одном месте. Если действует только один из них, то в некоторой точке громкость звука \(\Gamma_{1}=50\) фон. Найдем громкость звука в этой точке, если одновременно действуют все три источника.\\
Имея в виду, что при частоте 1 кГц громкость \(\Gamma\) звука совпадает с уровнем интенсивности \(L\) в децибелах, находим согласно (1.59), что для одного источника

\[
\Gamma_{1}=50=10 \lg \frac{I}{I_{0}}
\]

Если же одновременно действуют три данных источника, то

\[
\Gamma_{3}=10 \lg \frac{3 I}{I_{0}}
\]

Из этих формул следует, что

\[
\Gamma_{3}-\Gamma_{1}=10 \lg \frac{3 I}{I} \approx 4,8 \text { фон и } \Gamma_{3} \approx 54,8 \text { фон. }
\]

Энергия, соответствующая обычным звукам ( \(\sim 60\) дБ), весьма мала. Так, чтобы нагреть таким звуком стакан воды до кипения, потребовались бы тысячи лет.

Ультразвуковые волны (с частотой \(>20\) кГц), в отличие от обычного звука, могут быть получены в виде направленных пучков. Такие пучки нашли широкое применение для сигнализации и локации (определения объектов и расстояний до них) в воде, в дефектоскопии, ульразвуковых исследованиях (УЗИ) и др.

Интенсивность и давление звука. Интенсивность \(I\) звука для плоской гармонической бегущей волны связана с амплитудой звукового давления \(\delta p_{m}\) следующим соотношением:


\begin{equation*}
I=\frac{\left(\delta p_{m}\right)^{2}}{2 \rho v} \tag{1.61}
\end{equation*}


где \(\rho\) - плотность среды, \(v\) - скорость звука в ней.

Это выражение можно получить так. Из формулы \(\mathrm{d} p / p+\gamma \mathrm{d} V / V=0\) (см. с. 19) с учетом того, что \(\mathrm{d} V / V=\partial \xi / \partial x\), следует

\[
d p=-\gamma p \frac{\partial \xi}{\partial x}
\]

Для плоской гармонической волны \(\xi=a \cos (\omega t-k x)\) производная \(\partial \xi / \partial x=-a k \sin (\omega t-k x)\), поэтому

\[
\mathrm{d} p=\gamma p a k \sin (\omega t-k x) .
\]

Отсюда амплитуда звукового давления

\[
\delta p_{m}=\gamma p a k
\]

и амплитуда \(a=\delta p_{m} / \gamma p k\). Подставив это выражение в (1.50), получим:

\[
I=\langle j\rangle=\frac{1}{2} \rho \frac{\left(\delta p_{m}\right)^{2}}{(\gamma p k)^{2}} \omega^{2} v
\]

Остается учесть, что \(\omega^{2} / k^{2}=v^{2}, \gamma p=\rho v^{2}\) согласно (1.37), и мы придем к выражению (1.61).

Имея в виду, что \(I \sim\left(\delta p_{m}\right)^{2}\), формулу (1.59) можно представить как


\begin{equation*}
L_{\text {дБ }}=20 \lg \frac{\delta p_{m}}{\delta p_{m 0}} \tag{1.62}
\end{equation*}


где \(\delta p_{m 0}\) - условно-пороговая амплитуда звукового давления. Для воздуха при нормальных условиях \(\delta p_{m 0} \approx 2 \cdot 10^{-5}\) Па. Так, например, для частоты \(\sim 1\) кГц при нормальном разговоре громкостью 50 фон \(\delta p_{m} \approx 6,3 \cdot 10^{-3}\) Па, а при громкости 100 фон \(\delta p_{m} \approx 2\) Па.

\section*{§ 1.7. Эффект Доплера для звуковых волн}
Пусть источник, находящийся в газе или жидкости, испускает короткие импульсы с частотой \(v\). Если источник и приемник покоятся относительно средь, в которой распространяется волна, то частота воспринимаемых приемником импульсов будет равна частоте \(v\) источника. Если же источник, или приемник, или оба движутся относительно среды, то частота \(v^{\prime}\), воспринимаемая приемником, вообще говоря, оказывается отличной от частоты источника: \(v^{\prime} \neq v\). Это явление называют эффектом Доплера.

Сначала рассмотрим случай, когда источник \(S\) и приемник \(P\) движутся вдоль проходящей через них прямой с постоянными скоростями \(u\) и \(u^{\prime}\) соответственно (относительно среды).

Если бы двигался только источник навстречу приемнику, испуская импульсы с периодом \(T=1 / v\), то за это время очередной импульс пройдет относительно среды расстояние \(\lambda=v T\), где \(v\) - скорость волн в среде, и пока будет испущен следую-

\begin{figure}[h]
\begin{center}
  \includegraphics[alt={},max width=\textwidth]{950fefba-d83b-4c8f-96e7-0bd5a7eb90be-035_258_374_463_70}
\captionsetup{labelformat=empty}
\caption{Рис. 1.12}
\end{center}
\end{figure}

щий импульс, источник «нагонит» предыдущий импульс на расстояние \(u T\). Таким образом, расстояние между импульсами в среде станет равным \(\lambda^{\prime}=v T-u T\) (рис. 1.12), и воспринимаемая неподвижным приемником частота (число импульсов за единицу времени)


\begin{equation*}
v^{\prime}=\frac{v}{\lambda^{\prime}}=\frac{v}{T(v-u)} . \tag{1.63}
\end{equation*}


Если же движется и приемник (пусть тоже навстречу источнику), то импульсы относительно приемника будут иметь скорость \(v+u^{\prime}\), и число воспринимаемых за единицу времени импульсов


\begin{equation*}
v^{\prime}=\frac{v+u^{\prime}}{T(v-u)}=v \frac{v+u^{\prime}}{v-u} \tag{1.64}
\end{equation*}


Нетрудно сообразить, что при движении как источника, так и приемника в противоположных направлениях, знаки перед \(u^{\prime}\) и \(u\) надо поменять на обратные. Еще раз подчеркнем, что скорости \(u^{\prime}\) и \(u\) - это скорости приемника и источника относительно средь.

Как видно из приведенных рассуждений, эффект Доплера является следствием «уплотнения» (или разряжения) импульсов, обусловленным движением источника и приемника.

Формулу (1.64) целесообразнее записать в иной форме, более общей и более простой для запоминания и использования:


\begin{equation*}
v^{\prime}=v \frac{v-u_{x}^{\prime}}{v-u_{x}}, \tag{1.65}
\end{equation*}


где \(u_{x}^{\prime}\) и \(u_{x}\) - проекции скоростей приемника и источника на ось \(X\), проходящую через них и положительное направление

которой совпадает с направлением распространения импульсов, т. е. от источника \(S\) к приемнику \(P\).

Прежде чем продолжить обсуждение возможностей выражения (1.65), приведем два простых примера.\\
Пример 1. Источник \(S\) и приемник \(P\) удаляются друг от друга по одной прямой в противоположные стороны относительно среды со скоростями \(u\) и \(u^{\prime}\). Частота источника \(v\), скорость сигналов в среде \(v\). Найдем частоту \(v^{\prime}\), воспринимаемую приемником.

В данном случае проекция скорости приемника на ось \(X\) есть \(u_{x}^{\prime}=u^{\prime}\), а проекция скорости источника \(u_{x}=-u\). Подставив эти величины в формулу (1.65), получим

\[
v^{\prime}=v \frac{v-u^{\prime}}{v+u^{\prime}}
\]

Пример 2. Источник \(S\), испускающий сигналы с частотой \(v\), движется с постоянной скоростью \(u_{S}\) относительно приемника \(P\), установленного на башне (рис. 1.13). При этом воздушная масса перемещается относительно земной поверхности вправо с постоянной скоростью \(u_{0}\) (ветер). Скорость звука в воздухе \(v\). Найдем частоту \(v^{\prime}\), воспринимаемую приемником.

\begin{figure}[h]
\begin{center}
  \includegraphics[alt={},max width=\textwidth]{950fefba-d83b-4c8f-96e7-0bd5a7eb90be-036_231_507_1050_361}
\captionsetup{labelformat=empty}
\caption{Рис. 1.13}
\end{center}
\end{figure}

Имея в виду, что в формулу (1.65) входят скорости относительно среды, запишем: проекция скорости приемника \(u_{x}^{\prime}=-u_{0}\), а проекция скорости источника \(u_{x}=u_{S}-u_{0}\). Обе проекции взяты, как должно быть, на ось \(X\), направленную вправо. Остается подставить эти проекции в формулу (1.65), и мы получим:

\[
v^{\prime}=v \frac{v-\left(-u_{0}\right)}{v-\left(u_{S}-u_{0}\right)}=v \frac{v+u_{0}}{v-\left(u_{S}-u_{0}\right)} .
\]

Вернемся к обсуждению возможностей формулы (1.65). Оказывается, эта формула при определенных дополнительных

условиях может быть использована и в более сложных случаях, а именно, когда источник и приемник движутся не по одной прямой и с изменяющимися во времени скоростями \(\mathbf{u}(t)\) и \(\mathbf{u}^{\prime}(t)\). В этих случаях необходимо учитывать так называемый эффект запаздывания. Поскольку скорость передачи сигналов конечна, воспринимаемая приемником частота \(v^{\prime}\) в момент \(t\) будет обусловлена приходом в этот момент сигналов, испущенных источником в предшествующий момент \(t^{\prime}=t-\tau\), где \(\tau\) время, необходимое для прохождения расстояния \(l\) от источника в момент \(t-\tau\) до приемника в момент \(t\), т. е. \(\tau=l / v\). В качестве примеров могут служить задачи 1.9 и 1.10.

Вместе с тем, в некоторых случаях эффектом запаздывания можно пренебречь - это при условии, что скорости источника и приемника значительно меньше скорости звука (и при разумных расстояниях между источником и приемником).

Пусть источник \(S\) и приемник \(P\) движутся, например, так, как показано на рис. 1.14, со скоростями \(u\) и \(u^{\prime} \ll v\).

\begin{figure}[h]
\begin{center}
  \includegraphics[alt={},max width=\textwidth]{950fefba-d83b-4c8f-96e7-0bd5a7eb90be-037_148_582_911_324}
\captionsetup{labelformat=empty}
\caption{Рис. 1.14}
\end{center}
\end{figure}

Тогда в формулу (1.65) следует подставить

\[
u_{x}=u \cos \alpha, \quad u_{x}^{\prime}=u^{\prime} \cos \alpha^{\prime} .
\]

В приведенном на рисунке случае \(u_{x}>0\), а \(u_{x}^{\prime}<0\).\\
В заключение этой главы отметим, что многие вопросы, касающиеся особенностей волновых процессов (отражение, преломление, интерференция, дифракция, дисперсия и др.) будут рассмотрены далее на примере электромагнитных волн и в разделе «Волновая оптика».

\section*{Задачи}
1.1. Волна смещений частиц среды имеет вид \(\xi=a \sin (\alpha t-\beta x)\), где \(a\), \(\alpha, \beta\) - положительные постоянные. Найти отношение амплитуды скорости частиц среды к скорости волны.

Решение. Скорость частиц \(\partial \xi / \partial t=a \alpha \cos (\alpha t-\beta x)\), где \(a \alpha\) - амплитуда скорости ( \(u_{m}\) ). Скорость волны находим из условия \(\alpha t-\beta x=\)\\
\(=\) const. Продифференцировав это выражение по \(t\), получим: \(\dot{x}=\alpha / \beta\). Искомое отношение \(u_{m} / \dot{x}=a \beta\).\\
1.2. Точечный изотропный источник испускает гармонические звуковые колебания. Найти коэффициент затухания \(\gamma\) волны, если амплитуда колебаний частиц среды на расстоянии \(r\) от источника в \(\eta\) раз меньше, нежели на расстоянии \(r_{0}\).\\
Решение. Волна, испускаемая точечным источником, сферическая. Ее амплитуда \(a=(\alpha / r) \exp (-\gamma r)\). По условию,\\
\(\eta=a\left(r_{0}\right) / a(r)=\left(r / r_{0}\right) \exp \left[\gamma\left(r-r_{0}\right)\right]\). Отсюда

\[
\gamma=\ln \left(\eta r_{0} / r\right) /\left(r-r_{0}\right) .
\]

1.3. Найти волновой вектор \(\mathbf{k}\) плоской волны с частотой \(\omega\), если ее фазовые скорости в положительных направлениях осей \(X, Y, Z\) равны \(v_{1}, v_{2}, v_{3}\).\\
Решение. Волновой вектор \(\mathbf{k}=k \mathbf{n}\), где \(\mathbf{n}\) - орт нормали к волновой поверхности: \(\mathbf{n}=\mathbf{e}_{x} \cos \alpha+\mathbf{e}_{y} \cos \beta+\mathbf{e}_{z} \cos \gamma, \alpha, \beta, \gamma\) - углы между вектором \(\mathbf{n}\) и ортами осей координат. Остается учесть, что \(k=\omega / v, v\) - фазовая скорость вдоль вектора \(\mathbf{k}, \cos \alpha=v / v_{1}\), \(\cos \beta=v / v_{2}, \cos \gamma=v / v_{3}\).\\
В результате получим:

\[
\mathbf{k}=\omega\left(\mathbf{e}_{x} / v_{1}+\mathbf{e}_{y} / v_{2}+\mathbf{e}_{z} / v_{3}\right) .
\]

1.4. Поток энергии. Точечный изотропный источник звука мощности \(P\) находится в центре круглого полого цилиндра радиуса \(R\) и высоты \(h\). Найти средний по времени поток энергии, падающей на боковую поверхность цилиндра, полагая, что его стенки полностью поглощают звук, т. е. нет отражений.\\
Решении е. Сначала найдем поток энергии dФ, падающий на бесконечно узкую кольцевую полоску, отстоящую

\begin{figure}[h]
\begin{center}
  \includegraphics[alt={},max width=\textwidth]{950fefba-d83b-4c8f-96e7-0bd5a7eb90be-038_390_382_1135_772}
\captionsetup{labelformat=empty}
\caption{Рис. 1.15}
\end{center}
\end{figure}

на расстояние \(z\) от средней плоскости (рис. 1.15):


\begin{equation*}
\mathrm{d} \Phi=j_{n} \mathrm{~d} S=j \cos \alpha \mathrm{~d} S=\left(P / 4 \pi r^{2}\right) \cos \alpha \cdot 2 \pi R \mathrm{~d} z \tag{1}
\end{equation*}


Поскольку \(z=R \operatorname{tg} \alpha, \mathrm{~d} z=R \mathrm{~d} \alpha / \cos ^{2} \alpha\). Кроме того, \(r=R / \cos \alpha\).

После подстановки выражений для \(\mathrm{d} z\) и \(r\) в (1) получим:

\[
\mathrm{d} \Phi=(P / 2) \cos \alpha \mathrm{d} \alpha .
\]

Остается проинтегрировать это уравнение по \(\alpha\) от 0 до \(\alpha_{0}\), соответствующего краю цилиндра, и умножить на 2 , ибо такой же вклад дает и нижняя половина цилиндра. В итоге:

\[
\Phi=P \sin \alpha_{0}=P / \sqrt{1+(2 R / h)^{2}} .
\]

1.5. Найти звуковую мощность \(P\) точечного изотропного источника, если на расстоянии \(r\) от него интенсивность звука равна \(I\) и коэффициент затухания волны \(\gamma\).

P е ш е н и е. Поток звуковой энергии сквозь сферу радиуса \(r\), в центре которой находится источник, равен


\begin{equation*}
\Phi=4 \pi r^{2} I=A \mathrm{e}^{-2 \gamma r}, \tag{1}
\end{equation*}


где учтено, что интенсивность \(I\) пропорциональна квадрату амплитуды волны (1.50'), \(A\) - некоторая постоянная, не зависящая от \(r\). При \(r=0\) поток \(\Phi=\Phi_{0}=P\). Поэтому \(A=P\), и из (1) следует, что

\[
P=\Phi \mathrm{e}^{2 \gamma r}=4 \pi r^{2} I \mathrm{e}^{2 \gamma r} .
\]

1.6. Стоячая волна. Стержень длины \(l\) из материала, модуль Юнга которого \(E\) и плотность \(\rho\), закреплен на одном конце, другой - свободен. Найти число \(N\) продольных собственных колебаний этого стержня в диапазоне частот от \(v_{1}\) до \(v_{2}\).

Решение. Поскольку закреплен только один конец, то это значит, что на нем будет узел, а на свободном конце - пучность. Следовательно, на длине стержня должно укладываться, вообще говоря, целое число полуволн и одна четверть волны:

\[
l=m \lambda / 2+\lambda / 4=(2 m+1) \lambda / 4, \quad m=0,1,2, \ldots
\]

Длина волны \(\lambda=v / v\), где \(v=\sqrt{E / \rho}\). Поэтому из первой формулы возможные значения \(m\) будут следующие:

\[
m=2 l v / \sqrt{E / \rho}-1 / 2 .
\]

Подставив в это выражение значения \(v\), равные \(v_{1}\) и \(v_{2}\), найдем соответствующие \(m_{1}\) и \(m_{2}\) (они будут не целочисленными). Тогда искомое число собственных колебаний

\[
N=\left[m_{2}-m_{1}\right],
\]

где квадратные скобки означают, что надо взять целое число от величины \(m_{2}-m_{1}\).\\
1.7. Суперпозиция волн. В упругой однородной среде распространяются две плоские волны, одна вдоль оси \(X\), другая вдоль оси \(Y\) :

\[
\xi_{1}=a \cos (\omega t-k x), \quad \xi_{2}=a \cos (\omega t-k y) .
\]

Найти характер движения частиц среды в плоскости \(X, Y\), если обе волны поперечные и направление колебаний одинаково.

Решение. В тех точках, где разность фаз \(k(y-x)\) равна кратному \(2 \pi\), колебания будут происходить в фазе. Отсюда

\[
y=x \pm m \lambda, \quad m=0,1,2, \ldots,
\]

т. е. максимумы амплитуды колебаний, равные \(2 a\), будут располагаться вдоль таких прямых (рис. 1.16). Вдоль прямых \(y=x \pm(m+1 / 2) \lambda\)

\begin{figure}[h]
\begin{center}
  \includegraphics[alt={},max width=\textwidth]{950fefba-d83b-4c8f-96e7-0bd5a7eb90be-040_330_384_526_752}
\captionsetup{labelformat=empty}
\caption{Рис. 1.16}
\end{center}
\end{figure}

1.8. Эффект Доплера. Неподвижный источник испускает звук частоты \(v_{0}\). Найти частоту звука, отраженного от стенки, которая удаляется от источника с постоянной скоростью \(u\). Скорость звука \(v\). Считать, что \(u \ll v\).

Решение. Рассмотрим процесс отражения звука в две фазы. Сначала стенка играет роль приемника и воспринимаемая ею частота, согласно (1.65), равна


\begin{equation*}
v=v_{0}(v-u) / v . \tag{1}
\end{equation*}


На второй стадии стенка играет роль удаляющегося источника звука с частотой \(v\), поэтому частота отраженного звука


\begin{equation*}
v^{\prime}=v v /(v+u) . \tag{2}
\end{equation*}


Подставив (1) в (2), получим

\[
v^{\prime}=v_{0} \frac{v-u}{v+u} \approx v_{0}(1-2 u / v),
\]

где учтено, что \(u \ll v\).\\
1.9. Эффект запаздывания. Источник коротких звуковых импульсов с частотой \(v_{0}\) и приемник находятся в одной точке. В момент \(t=0\) источник начинает удаляться от приемника с постоянным ускорением \(a\). Найти частоту импульсов, воспринимаемых приемником в момент \(t\), если скорость звука равна \(v\).

Решение. Здесь следует учесть эффект запаздывания. Это значит, что воспринимаемые в момент \(t\) импульсы были испущены источником в предшествующий момент \(t^{\prime}\). Поэтому, согласно (1.65),


\begin{equation*}
v(t)=v_{0} v /\left(v+u_{t^{\prime}}\right) . \tag{1}
\end{equation*}


Скорость источника в момент \(t^{\prime}\) равна \(u_{t^{\prime}}=a t^{\prime}\), где \(t^{\prime}\) найдем из условия


\begin{equation*}
a t^{\prime 2} / 2=v\left(t-t^{\prime}\right), \tag{2}
\end{equation*}


т. е. путь, пройденный источником к моменту \(t^{\prime}\), равен произведению скорости звука на время запаздывания полученных сигналов. Определив \(t^{\prime}\) из (2), найдем \(u_{t^{\prime}}\) и с помощью (1) - искомую частоту:

\[
v(t)=v_{0} / \sqrt{1+2 a t / v} .
\]

1.10. Источник звука \(S\), собственная частота которого \(v_{0}\), движется равномерно по прямой, отстоящей от неподвижного наблюдателя \(P\) на расстояние \(l\) (рис. 1.17). Скорость источника составляет \(\eta\)-ю часть скорости звука. Найти:\\
a) частоту звука, воспринимаемую наблюдателем в момент, когда источник окажется в точке \(O\);\\
б) расстояние между источником и наблюдателем в момент, когда воспринимаемая наблюдателем частота \(v=v_{0}\).

\begin{figure}[h]
\begin{center}
  \includegraphics[alt={},max width=\textwidth]{950fefba-d83b-4c8f-96e7-0bd5a7eb90be-041_303_512_1442_359}
\captionsetup{labelformat=empty}
\caption{Рис. 1.17}
\end{center}
\end{figure}

Решение. а) Ясно, что в этот момент наблюдателя достигнут сигналы, испущенные источником, когда он находился еще в некоторой точке \(S\) (рис. 1.17). Ее положение должно быть таким, чтобы за время, пока источник движется со скоростью \(u\) до точки \(O\), сигнал со скоростью \(v\) достиг бы точки \(P\). Отсюда проекция скорости источника на направление \(S P\) равна \(u_{x}=u \cos \alpha\), где \(\cos \alpha=u / v=\eta\). Искомая частота, согласно (1.65),

\[
v=v_{0} v /(v-u \cos \alpha)=v_{0} /\left(1-\eta^{2}\right) .
\]

б) Источник, от которого дошел сигнал с частотой \(v=v_{0}\), должен был находиться в точке \(O\). Пока звук дошел до точки \(P\) (за время \(\tau=l / v\) ), сам источник переместился вправо на расстояние \(x=u \tau\). Из рис. 1.17 следует, что искомое расстояние

\[
r=\sqrt{l^{2}+(u \tau)^{2}}=l \sqrt{1+\eta^{2}}
\]

\section*{Глава 2}
\section*{Электромагнитные волны}
\section*{§ 2.1. Волновое уравнение электромагнитной ВОЛНЫ}
Из уравнений Максвелла следует, как мы сейчас убедимся, важный вывод о существовании принципиально нового физического явления: электромагнитное поле способно существовать самостоятельно - без электрических зарядов и токов. При этом изменение его состояния (возмущение поля) обязательно имеет волновой характер. Поля такого рода называют электромагнитными волнами. В вакууме эти волны распространяются со скоростью, равной скорости света \(c\).

Рассмотрим однородную нейтральную непроводящую среду с проницаемостями \(\varepsilon\) и \(\mu\), где \(\mathbf{D}=\varepsilon \varepsilon_{0} \mathbf{E}\) и \(\mathbf{B}=\mu \mu_{0} \mathbf{H}\).

Сразу же договоримся, что для упрощения записи формул мы будем оставлять только \(\varepsilon_{0}\) и \(\mu_{0}\) (как в вакууме). При переходе же к указанной среде достаточно в формулы, содержащие \(\varepsilon_{0}\) и \(\mu_{0}\), приписать \(\varepsilon\) к \(\varepsilon_{0}\) (т. е. \(\varepsilon \varepsilon_{0}\) ) и \(\mu-\kappa \mu_{0}\) (т. е. \(\mu \mu_{0}\) ). Упростив таким образом приведенные выше соотношения, запишем


\begin{equation*}
\mathbf{D}=\varepsilon_{0} \mathbf{E}, \quad \mathbf{B}=\mu_{0} \mathbf{H} . \tag{2.1}
\end{equation*}


Поскольку в данном случае плотности зарядов и токов равны нулю ( \(\rho=0\) и \(\mathbf{j}=0\) ), уравнения Максвелла будут иметь вид:

\[
\begin{array}{cc}
\nabla \times \mathbf{E}=-\frac{\partial \mathbf{B}}{\partial t}, & \nabla \times \mathbf{H}=\frac{\partial \mathbf{D}}{\partial t}, \\
\nabla \cdot \mathbf{B}=0, & \nabla \cdot \mathbf{D}=0, \tag{2.3}
\end{array}
\]

где уравнения (2.2) выражают роторы \(\mathbf{E}\) и \(\mathbf{H}\), а уравнения (2.3) дивергенции В и D.

Так как любые волновые процессы должны подчиняться волновому уравнению, связывающему вторые производные по вре-

мени и координатам, попытаемся придти к нему с помощью написанных выше уравнений Максвелла. Для этого продифференцируем второе уравнение из (2.2) по времени и затем используем первое уравнение:


\begin{equation*}
\varepsilon_{0} \frac{\partial^{2} \mathbf{E}}{\partial t^{2}}=\frac{\partial}{\partial t}(\boldsymbol{\nabla} \times \mathbf{H})=\boldsymbol{\nabla} \times \frac{\partial \mathbf{H}}{\partial t}=-\frac{1}{\mu_{0}} \boldsymbol{\nabla} \times(\boldsymbol{\nabla} \times \mathbf{E}) . \tag{2.4}
\end{equation*}


Двойное векторное произведение в правой части можно преобразовать с помощью векторного равенства


\begin{equation*}
\mathbf{a} \times(\mathbf{b} \times \mathbf{c})=\mathbf{b}(\mathbf{a c})-\mathbf{c}(\mathbf{a b}), \tag{2.5}
\end{equation*}


\begin{itemize}
  \item правило «бац минус цаб». Заменив в этой формуле a и b оператором \(\nabla\) и положив \(\mathbf{c}=\mathbf{E}\), запишем
\end{itemize}


\begin{equation*}
\nabla \times(\nabla \times \mathbf{E})=\nabla(\nabla \cdot \mathbf{E})-\mathbf{E}(\nabla \cdot \nabla) \ldots ? \tag{2.5'}
\end{equation*}


Получилось что-то не то - это касается последнего члена \(\mathbf{E}(\boldsymbol{\nabla} \cdot \boldsymbol{\nabla})\). В таком виде он все еще остался оператором. Ошибка в том, что мы не проявили осторожности и не выдержали нужного порядка «сомножителей». Вернувшись обратно к формуле (2.5), легко увидеть, что последний член с равным правом можно записать в виде (ab)c. И тогда последний член в (2.5') следует записать как ( \(\nabla \cdot \nabla\) )E или \(\nabla^{2} \mathbf{E}\), после чего (2.5') примет правильный вид

\[
\nabla \times(\nabla \times \mathbf{E})=\nabla(\nabla \cdot \mathbf{E})-\nabla^{2} \mathrm{E},
\]

где \(\nabla^{2}\) - оператор Лапласа. Остается учесть, что в нашем случае \(\nabla \cdot \mathbf{E}=0\), и тогда двойное векторное произведение в (2.4) оказывается равным \(-\nabla^{2} \mathbf{E}\). В результате мы приходим к волновому уравнению для поля Е. Прежде чем его выписать, заметим, что подобным образом мы можем получить аналогичное волновое уравнение для поля H.

Таким образом, мы приходим к идентичным волновым уравнениям для полей \(\mathbf{E}\) и \(\mathbf{H}\) :


\begin{equation*}
\nabla^{2} \mathbf{E}=\varepsilon_{0} \mu_{0} \frac{\partial^{2} \mathbf{E}}{\partial t^{2}}, \quad \nabla^{2} \mathbf{H}=\varepsilon_{0} \mu_{0} \frac{\partial^{2} \mathbf{H}}{\partial t^{2}} . \tag{2.6}
\end{equation*}


Здесь коэффициент перед второй производной по времени есть не что иное как величина, обратная квадрату скорости распространения волны, согласно (1.23). В вакууме эта скорость равна


\begin{equation*}
c=1 / \sqrt{\varepsilon_{0} \mu_{0}}, \tag{2.7}
\end{equation*}


Если же электромагнитная волна распространяется в однородной среде с проницаемостями \(\varepsilon\) и \(\mu\), то ее скорость


\begin{equation*}
v=1 / \sqrt{\varepsilon \varepsilon_{0} \mu \mu_{0}}=c / \sqrt{\varepsilon \mu} . \tag{\(\prime\)}
\end{equation*}


В дальнейшем мы будем рассматривать в основном немагнитные среды ( \(\mu=1\) ), поэтому предыдущая формула принимает еще более простой вид


\begin{equation*}
v=c / \sqrt{\varepsilon} . \tag{2.8}
\end{equation*}


Оказалось, что \(c=3 \cdot 10^{8} \mathrm{~m} / \mathrm{c}\), т. е. совпадает со скоростью света в вакууме. Это и дало основание Максвеллу предположить задолго до экспериментального подтверждения, что свет представляет собой электромагнитные волны.

\section*{§ 2.2. Плоская электромагнитная волна}
Перепишем уравнения Максвелла (2.2) и (2.3) в форме более удобной для дальнейшего анализа, имея в виду, что роторы \(\mathbf{E}\) и Н можно представить в виде определителей (как векторное произведение двух векторов):


\begin{gather*}
\nabla \times \mathbf{E}=\left|\begin{array}{ccc}
\mathbf{e}_{x} & \mathbf{e}_{y} & \mathbf{e}_{z} \\
\frac{\partial}{\partial x} & \frac{\partial}{\partial y} & \frac{\partial}{\partial z} \\
E_{x} & E_{y} & E_{z}
\end{array}\right|=-\mu_{0} \frac{\partial \mathbf{H}}{\partial t}, \quad \nabla \times \mathbf{H}=\left|\begin{array}{ccc}
\mathbf{e}_{x} & \mathbf{e}_{y} & \mathbf{e}_{z} \\
\frac{\partial}{\partial x} & \frac{\partial}{\partial y} & \frac{\partial}{\partial z} \\
H_{x} & H_{y} & H_{z}
\end{array}\right|=\varepsilon_{0} \frac{\partial \mathbf{E}}{\partial t},  \tag{2.9}\\
\frac{\partial}{\partial x} E_{x}+\frac{\partial}{\partial y} E_{y}+\frac{\partial}{\partial z} E_{z}=0, \quad \frac{\partial}{\partial x} H_{x}+\frac{\partial}{\partial y} H_{y}+\frac{\partial}{\partial z} H_{z}=0, \tag{2.10}
\end{gather*}


где \(\mathbf{e}_{x}, \mathbf{e}_{y}, \mathbf{e}_{z}\) - орты осей \(X, Y, Z\).\\
Установим основные свойства электромагнитной волны на примере плоской волны. Направим ось \(X\) перпендикулярно волновым поверхностям. При этом \(\mathbf{E}\) и \(\mathbf{H}\), а значит и их проек-

ции на оси \(Y\) и \(Z\), не будут зависеть от координат \(y\) и \(z\), т. е. соответствующие производные по \(y\) и \(z\) будут равны нулю. Поэтому уравнения (2.9) и (2.10) упрощаются (останутся только производные по \(x\) ) и принимают вид:

\[
\begin{array}{rlrl}
\mathbf{0} & =\mu_{0} \frac{\partial H_{x}}{\partial t}, & 0 & =\varepsilon_{0} \frac{\partial E_{x}}{\partial t} \\
\frac{\partial E_{z}}{\partial x} & =\mu_{0} \frac{\partial H_{y}}{\partial t}, & -\frac{\partial H_{z}}{\partial x} & =\varepsilon_{0} \frac{\partial E_{y}}{\partial t}  \tag{2.11}\\
& \frac{\partial E_{y}}{\partial x} & =-\mu_{0} \frac{\partial H_{z}}{\partial t}, & \ddots \\
\frac{\partial E_{x}}{\partial x} & =0, & \frac{\partial H_{y}}{\partial x} & =\varepsilon_{0} \frac{\partial E_{z}}{\partial t} \\
\frac{\partial E_{x}}{\partial x} & =0 .
\end{array}
\]

Из условий \(\partial E_{x} / \partial x=0\) и \(\partial E_{x} / \partial t=0\) следует, что \(E_{x}\) не зависит ни от \(x\), ни от \(t\) (аналогично и для \(H_{x}\) ). Это значит, что отличные от нуля \(E_{x}\) и \(H_{x}\) могут быть обусловлены лишь постоянными однородными полями, накладывающимися на поле волны. А для переменного поля плоской волны \(E_{x}=0\) и \(H_{x}=0\), т. е. векторы \(\mathbf{E}\) и \(\mathbf{H}\) перпендикулярны направлению распространения волны - оси \(X\). Значит, электромагнитная волна является поперечной.

Кроме того, оказывается, векторы Е и Н в электромагнитной волне взаимно ортогональны. Чтобы убедиться в этом, объединим средние уравнения (2.11), содержащие, например, \(E_{y}\) и \(H_{z}\) (они подчеркнуты и соединены сплошными линиями), в пару:


\begin{equation*}
\frac{\partial E_{y}}{\partial x}=-\mu_{0} \frac{\partial H_{z}}{\partial t}, \quad \frac{\partial H_{z}}{\partial x}=-\varepsilon_{0} \frac{\partial E_{y}}{\partial t}, \tag{2.12}
\end{equation*}


(можно было бы взять и другую пару, содержащую производные \(E_{z}\) и \(H_{y}\), - она подчеркнута пунктирной линией). Из этих уравнений видно, что изменение во времени, скажем, магнитного поля, направленного вдоль оси \(Z\), порождает электрическое поле \(E_{y}\) вдоль оси \(Y\). Изменение во времени поля \(E_{y}\) в свою очередь порождает поле \(H_{z}\) и т. д. Ни поля \(E_{z}\), ни поля \(H_{y}\) при этом не возникает. А это и значит, что \(\mathbf{E} \perp \mathbf{H}\).

Связь мгновенных значений Е и Н. В нашем случае, когда плоская волна распространяется в вакууме вдоль оси \(X\), например, в ее положительном направлении,


\begin{equation*}
E_{y}=E_{y}(t-x / c), \quad H_{z}=H_{z}(t-x / c), \tag{2.13}
\end{equation*}


где \(E_{y}\) и \(H_{z}\) - некоторые функции, характеризующие форму волны. Введя обозначение \(\varphi=t-x / c\), найдем производные \(E_{y}\) по \(x\) и \(H_{z}\) по \(t\) - в соответствии с (2.12):

\[
\frac{\partial E_{y}}{\partial x}=\frac{\partial E_{y}}{\partial \varphi} \frac{\partial \varphi}{\partial x}=\frac{\partial E_{y}}{\partial \varphi}\left(-\frac{1}{c}\right) ; \quad \frac{\partial H_{z}}{\partial t}=\frac{\partial H_{z}}{\partial \varphi} \frac{\partial \varphi}{\partial t}=\frac{\partial H_{z}}{\partial \varphi} \cdot 1 .
\]

Подставив эти выражения в первое уравнение (2.12), получим:

\[
\frac{1}{c} \frac{\partial E_{y}}{\partial \varphi}=\mu_{0} \frac{\partial H_{z}}{\partial \varphi},
\]

или с учетом того, что \(c=1 / \sqrt{\varepsilon_{0} \mu_{0}}\),

\[
\sqrt{\varepsilon_{0}} \frac{\partial E_{y}}{\partial \varphi}=\sqrt{\mu_{0}} \frac{\partial H_{z}}{\partial \varphi} .
\]

Отсюда следует, что \(\sqrt{\varepsilon_{0}} E_{y}=\sqrt{\mu_{0}} H_{z}+\) const, где произвольная константа обусловлена наличием постоянного электрического и магнитного полей. Нас интересует только переменное поле, поэтому const \(=0\), в результате мы получим для однородной среды с проницаемостью \(\varepsilon\) и \(\mu\) следующее соотношение:


\begin{equation*}
\sqrt{\varepsilon \varepsilon_{0}} E_{y}=\sqrt{\mu \mu_{0}} H_{z} . \tag{2.14}
\end{equation*}


Это выражение означает, что \(\mathbf{E}\) и \(\mathbf{H}\) не только взаимно ортогональны, но и составляют правовинтовую систему с направлением распространения: мы ведь рассмотрели случай, когда волна распространяется в положительном направлении оси \(X\) (рис. 2.1). Кроме того, \(\mathbf{E}\) и \(\mathbf{H}\), согласно (2.14), изменяются при этом синфазно: \(E_{y}\) и \(H_{z}\) одинаковы в каждый момент по знаку, одновременно обращаются в нуль и одновременно достигают максимума, что и показано на рис. 2.2 - мгновенная картина в некоторый момент.

\begin{figure}[h]
\begin{center}
  \includegraphics[alt={},max width=\textwidth]{950fefba-d83b-4c8f-96e7-0bd5a7eb90be-048_366_344_153_172}
\captionsetup{labelformat=empty}
\caption{Рис. 2.1}
\end{center}
\end{figure}

\begin{figure}[h]
\begin{center}
  \includegraphics[alt={},max width=\textwidth]{950fefba-d83b-4c8f-96e7-0bd5a7eb90be-048_367_452_153_659}
\captionsetup{labelformat=empty}
\caption{Рис. 2.2}
\end{center}
\end{figure}

Заметим, что если бы мы рассмотрели волну, распространяющуюся в отрицательном направлении оси \(X\), то \(E_{y}\) и \(H_{z}\) изменялись бы в противофазе ( \(\sqrt{\varepsilon \varepsilon_{0}} E_{y}=-\sqrt{\mu \mu_{0}} H_{z}\) ). Однако по-прежнему оба вектора, \(\mathbf{E}\) и \(\mathbf{H}\), составляли бы правовинтовую систему с направлением распространения. Это же относится и к случаю, когда вектор \(\mathbf{E}\) направлен вдоль оси \(Z\), а вектор \(\mathbf{H}\) - вдоль оси \(Y\), т. е. их проекции \(E_{z}\) и \(H_{y}\).

Выяснив эти детали, индексы \(y\) и \(z\) у проекций векторов \(\mathbf{E}\) и H можно не писать (как это обычно и делают). Поэтому уравнение, например, плоской бегущей гармонической волны - она представляет особый интерес - записывают так:


\begin{equation*}
E=E_{m} \cos (\omega t-k x), \quad H=H_{m} \cos (\omega t-k x), \tag{2.15}
\end{equation*}


где знак минус в скобках означает, что волна распространяется в положительном направлении оси \(X\). В этих выражениях \(\omega\) круговая (циклическая) частота колебаний, \(k\) - волновое число ( \(k=2 \pi / \lambda, \lambda\) - длина волны).

Заметим, что когда говорят, что плоская волна распространяется, например, в положительном направлении оси \(X\), то это означает, что с этим направлением совпадает ее волновой вектор k или, другими словами, ее волновые поверхности ортогональны оси \(X\). Но при этом колебания распространяются очевидно и в других направлениях.

Пример. Плоская гармоническая электромагнитная волна распространяется в вакууме так, что ее волновой вектор \(\mathbf{k}\) перпендикулярен оси \(Z\) и составляет угол \(\alpha=60^{\circ}\) с ортом оси \(X\). Найдем скорость распространения колебаний вдоль оси \(X\).

Изобразив рисунок, аналогичный рис. 1.1, найдем, что искомая скорость \(v=c / \cos \alpha=2 c\) ! Полученный результат не противоречит теории относительности: фазовая скорость может быть любой, в отличие от скорости сигнала, которая не может быть больше \(c\) - скорости света в вакууме.

Теперь рассмотрим пример, который относится к формуле (2.14) - тоже на связь амплитуд электрической и магнитной составляющих волны, но не в скалярном, а в векторном виде.

Пример. В вакууме распространяется гармоническая плоская электромагнитная волна, электрическая составляющая которой имеет вид

\[
\mathbf{E}=\mathbf{e}_{z} E_{m} \cos (\omega t-\mathbf{k r}) .
\]

Найдем вектор-амплитуду магнитной составляющей этой волны, \(\mathbf{H}_{m}\).\\
Видно, что данная волна распространяется в направлении вектора \(\mathbf{k}\). Значит, три вектора, \(\mathbf{E}_{m}, \mathbf{H}_{m}, \mathbf{k}\) должны составлять правовинтовую систему (см. рис. 2.1). Отсюда следует, что вектор \(\mathbf{H}_{m}\) должен быть сонаправлен с вектором \([\mathbf{k E}]\), направление которого совпадает с ортом \(\left[\mathbf{n}_{k} \mathbf{e}_{z}\right]\), где орт \(\mathbf{n}_{k}=\mathbf{k} / k\). Остается найти модуль вектора \(\mathbf{H}_{m}\), т. е. воспользоваться формулой (2.14): \(H_{m}=\sqrt{\varepsilon_{0} / \mu_{0}} E_{m}\). В результате получим:

\[
\mathbf{H}_{m}=\left[\mathbf{n}_{k} \mathbf{e}_{z}\right] \sqrt{\varepsilon_{0} / \mu_{0}} E_{m} .
\]

\section*{§ 2.3. Стоячая электромагнитная волна}
В § 1.5 было показано, что стоячую упругую волну можно представить как результат суперпозиции двух одинаковых волн, бегущих навстречу друг другу. Это относится и к электромагнитным волнам. Однако надо учесть, что электромагнитная волна характеризуется не одним вектором, а двумя взаимно ортогональными векторами \(\mathbf{E}\) и \(\mathbf{H}\).

Пусть волна распространяется в положительном направлении оси \(X\) и описывается уравнениями


\begin{equation*}
E_{y}=E_{m} \cos (\omega t-k x), \quad H_{z}=H_{m} \cos (\omega t-k x) . \tag{2.16}
\end{equation*}


\begin{figure}[h]
\begin{center}
  \includegraphics[alt={},max width=\textwidth]{950fefba-d83b-4c8f-96e7-0bd5a7eb90be-050_457_949_149_141}
\captionsetup{labelformat=empty}
\caption{Рис. 2.3}
\end{center}
\end{figure}

Уравнения волны, распространяющейся в обратном направлении, можно получить из (2.16), если заменить в скобках минусы на плюсы и учесть, что векторы \(\mathbf{E}, \mathbf{H}, \mathbf{k}\) должны составлять правую тройку. Это поясняет рис. 2.3, где слева ( \(a\) ) \(\mathbf{E}\) и \(\mathbf{H}\) меняются в фазе - волна (2.16), а справа \(\mathbf{E}\) и \(\mathbf{H}\) - в противофазе (во встречной волне). Последнее означает, что перед \(E_{m}\) или \(H_{m}\) должен появиться знак минус. Итак, уравнения встречной волны будут иметь вид:


\begin{equation*}
E_{y}=E_{m} \cos (\omega t+k x), \quad H_{z}=-H_{m} \cos (\omega t+k x) . \tag{2.17}
\end{equation*}


В результате суперпозиции этих двух встречных волн, (2.16) и (2.17), получим:


\begin{equation*}
E_{y}=2 E_{m} \cos k x \cdot \cos \omega t, \quad H_{z}=2 H_{m} \sin k x \cdot \sin \omega t . \tag{2.18}
\end{equation*}


Это и есть уравнения стоячей электромагнитной волны. Они состоят из двух стоячих волн - электрической и магнитной. Видно, что в этой волне колебания векторов Е и Н сдвинуты по фазе на \(\pi / 2\) как в пространстве, так и во времени. Если в некоторый момент \(E_{y}\) во всех точках имело максимальное значение и при этом \(H_{z}=0\), то через четверть периода картина будет обратной: \(H_{z}\) достигнет всюду максимальных значений со сдвигом в пространстве на \(\lambda / 4\), а \(E_{y}\) обратится в нуль. Таким образом, в процессе колебаний электрическое поле постепенно переходит в магнитное, магнитное - в электрическое и т. д. (рис. 2.4). Поскольку колебания векторов \(\mathbf{E}\) и \(\mathbf{H}\) происхо-

\begin{figure}[h]
\begin{center}
  \includegraphics[alt={},max width=\textwidth]{950fefba-d83b-4c8f-96e7-0bd5a7eb90be-051_404_632_153_298}
\captionsetup{labelformat=empty}
\caption{Рис. 2.4}
\end{center}
\end{figure}

дят не в фазе, соотношение (2.14) оказыватеся справедливым только для амплитудных значений \(E_{m}\) и \(H_{m}\) стоячей волны:


\begin{equation*}
E_{m} \sqrt{\varepsilon \varepsilon_{0}}=H_{m} \sqrt{\mu \mu_{0}} . \tag{2.19}
\end{equation*}


\section*{§ 2.4. Энергия электромагнитной волны}
С электромагнитной волной связан перенос энергии. Плотность потока энергии можно найти с помощью формулы (1.48) как произведение плотности энергии \(w\) на скорость волны \(v\).

В обычной изотропной среде с проницаемостями \(\varepsilon\) и \(\mu\) плотность энергии электромагнитного поля равна сумме плотностей энергии электрического и магнитного полей:


\begin{equation*}
w=\frac{\varepsilon \varepsilon_{0} E^{2}}{2}+\frac{\mu \mu_{0} H^{2}}{2} . \tag{2.20}
\end{equation*}


В данной среде справедливо соотношение (2.14) между \(E\) и \(H\), а это означает, что плотность электрической энергии в бегущей волне равна плотности магнитной энергии. Поэтому (2.20) можно записать так:


\begin{equation*}
w=\varepsilon \varepsilon_{0} E^{2}=\sqrt{\varepsilon \varepsilon_{0} \mu \mu_{0}} E H=E H / v, \tag{2.21}
\end{equation*}


где \(v\) - скорость волны, (2.7').\\
Умножив \(w\) на \(v\), получим плотность потока энергии:


\begin{equation*}
\Pi=w v=E H . \tag{2.22}
\end{equation*}


Векторы \(\mathbf{E}\) и \(\mathbf{H}\) взаимно ортогональны и образуют с направлением распространения волны правовинтовую систему. Значит, направление вектора \([\mathbf{E H}]\) совпадает с направлением переноса энергии, а модуль этого вектора равен \(E H\). Поэтому вектор плотности потока электромагнитной энергии \(\boldsymbol{\Pi}\) можно представить как


\begin{equation*}
\boldsymbol{\Pi}=[\mathbf{E H}] . \tag{2.23}
\end{equation*}


Вектор П называют вектором Пойнтинга.\\
В случае бегущей гармонической электромагнитной волны (2.15) плотность энергии, согласно (2.21), равна

\[
w=\varepsilon \varepsilon_{0} E_{m}^{2} \cos ^{2}(\omega t-k x) .
\]

Плотность же потока энергии, как следует из (2.22),


\begin{equation*}
\Pi=w v=\sqrt{\varepsilon \varepsilon_{0} / \mu \mu_{0}} E_{m}^{2} \cos ^{2}(\omega t-k x), \tag{2.24}
\end{equation*}


где учтено, что скорость \(v\) определяется формулой (2.7').\\
Интенсивность \(I\) такой волны равна, по определению, среднему значению плотности потока энергии: \(I=\langle П\rangle\). Принимая во внимание, что при усреднении (2.24) среднее значение квадрата косинуса равно \(1 / 2\), получим


\begin{equation*}
I=\sqrt{\varepsilon \varepsilon_{0} / \mu \mu_{0}} E_{m}^{2} / 2 . \tag{2.25}
\end{equation*}


Обратим внимание, что \(I\) пропорционально квадрату амплитуды, \(I\) c \(E_{m}^{2}\).

Пример. В вакууме распространяется плоская гармоническая линейно поляризованная электромагнитная волна частоты \(\omega\). Интенсивность волны равна \(I\). Найдем амплитудное значение плотности тока смещения в этой волне.\\
По определению, плотность тока смещения \(\mathbf{j}_{\mathrm{cm}}=\partial \mathbf{D} / \partial t\), где \(\mathbf{D}=\varepsilon_{0} \mathbf{E}\). Пусть \(\mathbf{E}=\mathbf{E}_{m} \cos (\omega t-k x)\), тогда амплитудное значение плотности тока смещения \(j_{\text {см макс }}=\varepsilon_{0} \omega E_{m}\). Остается найти \(E_{m}\). Это делается с помощью формулы (2.25):

\[
E_{m}=\sqrt{2 I \sqrt{\mu_{0} / \varepsilon_{0}}},
\]

\begin{verbatim}
и мы получим из предыдущих двух формул, что
    \(j_{\text {см макс }}=\omega \sqrt{2 \varepsilon_{0} I / c}\),
где \(c=1 / \sqrt{\varepsilon_{0} \mu_{0}}\).
\end{verbatim}

В стоячей электромагнитной волне энергия переходит из чисто электрической, имеющей максимумы в пучностях \(\mathbf{E}\), в магнитную с максимумами в пучностях вектора Н, т. е. смещенным в пространстве на \(\lambda / 4\). Это аналогично поведению гармонического осциллятора, например математического маятника, где энергия переходит из чисто потенциальной (в крайнем положении) в кинетическую (в положении равновесия), и наоборот.

Отметим, что если волна представляет собой наложение двух бегущих волн со взаимно перпендикулярными плоскостями поляризации (направлением колебаний вектора \(\mathbf{E}\) ), то ее интенсивность независимо от особенностей этих волн будет равна сумме интенсивностей складываемых волн. Действительно, \(\mathbf{E}=\mathbf{E}_{1}+\mathbf{E}_{2}\), а интенсивность \(I \sim\left\langle E^{2}\right\rangle=\left\langle E_{1}^{2}+2 \mathbf{E}_{1} \mathbf{E}_{2}+E_{2}^{2}\right\rangle\). Поскольку \(\mathbf{E}_{1} \perp \mathbf{E}_{2}\), скалярное произведение \(\mathbf{E}_{1} \mathbf{E}_{2}=\mathbf{0}\), и мы имеем \(I=I_{1}+I_{2}\).

\section*{§ 2.5. Импульс электромагнитной волны}
Перенос энергии электромагнитной волной сопровождается и переносом импульса. Согласно теории относительности, импульс объекта с нулевой массой покоя, движущегося со скоростью света, \(p=W / c\), где \(W\) - его энергия (это по существу верно и для электромагнитной волны как потока фотонов). Поскольку в случае электромагнитной волны масса покоя «объекта» равна нулю, связь между энергией и импульсом будет такой же:


\begin{equation*}
p=w / c, \tag{2.26}
\end{equation*}


где \(p\) и \(w\) - плотности импульса и энергии, т. е. величины, отнесенные к единице объема. Умножив числитель и знаменатель правой части равенства (2.26) на \(c\), получим в числителе, согласно (2.22), плотность потока энергии ( \(w c\) ), которая в свою очередь равна модулю вектора Пойнтинга. Таким образом, в векторном виде


\begin{equation*}
\mathbf{p}=[\mathbf{E H}] / c^{2} . \tag{2.27}
\end{equation*}


Если падающая нормально на поверхность некоторого тела электромагнитная волна полностью поглощается этим телом, то единице площади поверхности тела сообщается за промежуток времени \(\mathrm{d} t\) импульс, заключенный в цилиндре с площадью сечения, равной единице, и высотой \(c \mathrm{~d} t\), т. е. \(\mathrm{d} p=(w / c) c \mathrm{~d} t\). Но импульс, сообщаемый единице поверхности в единицу времени, \(\mathrm{d} p / \mathrm{d} t\), равен давлению \(p^{*}\) на поверхность тела. Поэтому для поглощающей поверхности давление \(p^{*}=w, \mathrm{H} / \mathrm{m}^{2}\). В случае гармонической волны эта величина пульсирует с достаточно большой частотой, и практически представляет интерес лишь ее среднее значение по времени:


\begin{equation*}
p^{*}=\langle w\rangle . \tag{2.28}
\end{equation*}


Для идеально отражающей поверхности давление будет в два раза больше.

Рассмотрим более детально механизм передачи импульса телу, т. е. как возникает давление. Электрическое поле волны возбуждает в теле ток плотности \(\mathbf{j}=\sigma \mathbf{E}\), а магнитное поле волны будет действовать на \(\mathbf{j}\) в соответствии с законом Ампера - с силой, объемная плотность которой равна


\begin{equation*}
\mathbf{F}_{\mathrm{e} \mu}=[\mathbf{j B}]=\sigma[\mathbf{E B}], \tag{2.29}
\end{equation*}


откуда следует, что сила направлена в сторону распространения волны.

Надо иметь в виду, что электромагнитная волна оказывает давление не только внутри вещества (при условии, что удельная проводимость \(\sigma \neq 0\) ), но и при отражении от поверхности, так что


\begin{equation*}
p^{*}=\langle w\rangle(1+R), \tag{2.30}
\end{equation*}


где \(R\) - коэффициент отражения, т. е. отношение интенсивности отраженной волны к интенсивности падающей.

Давление, вычисленное по формуле (2.28), оказывается в обычных условиях очень малым. Например, солнечный свет оказывает давление порядка \(10^{-5}\) Па (атмосферное давление \(\approx 10^{5}\) Па). Измерить такое давление экспериментально очень трудно. Впервые это удалось П. Н. Лебедеву (в 1900 г.). Его измерения дали значение, согласующееся с теорией с точностью

до \(20 \%\). Позднее эти измерения повторил Герлах (в 1923 г.), достигнув точности до \(2 \%\).

Тот факт, что электромагнитное поле обладает импульсом, предписывает при составлении баланса импульсов частиц учитывать и импульс электромагнитного поля. Только при этом с законом сохранения импульса будет все в порядке.

\section*{§ 2.6. Эффект Доплера для электромагнитных ВОЛН}
Рассмотренное в § 1.7 изменение частоты звуковых сигналов, обусловленное эффектом Доплера, определяется скоростями движения источника и приемника относительно средь, являющейся носителем звуковых волн. Для электромагнитных же волн особой среды, которая служила бы их носителем, нет. Поэтому доплеровское смещение частоты электромагнитных волн (сигналов) определяется только скоростью источника \(o m\) носительно приемника.

Пусть в \(K\)-системе отсчета находится неподвижный приемник \(P\) (рис. 2.5). К нему с релятивистской скоростью \(v\) приближается \(S\) - источник периодических электромагнитных (или

\begin{figure}[h]
\begin{center}
  \includegraphics[alt={},max width=\textwidth]{950fefba-d83b-4c8f-96e7-0bd5a7eb90be-055_211_383_1060_104}
\captionsetup{labelformat=empty}
\caption{Рис. 2.5}
\end{center}
\end{figure}

световых) сигналов. В \(K^{\prime}\)-системе отсчета, связанной с источником, сигналы испускаются с частотой \(v_{0}\) (собственная частота). Найдем частоту \(v\), с которой воспринимаются эти сигналы приемником.\\
Промежуток времени между двумя последовательными сигналами (импульсами) в \(K^{\prime}\)-системе, связанной с источником, равен \(T_{0}=1 / v_{0}\). Поскольку источник движется со скоростью \(v\), то соответствующий промежуток времени в \(K\)-системе, согласно «эффекту замедления хода движущихся часов», будет больше, а именно


\begin{equation*}
T=T_{0} / \sqrt{1-\beta^{2}}, \quad \beta=v / c \tag{2.31}
\end{equation*}


Расстояние между соседними импульсами в \(K\)-системе


\begin{equation*}
\lambda=c T-v T=(c-v) T_{0} / \sqrt{1-\beta^{2}} . \tag{2.32}
\end{equation*}


Поэтому воспринимаемая приемником частота \(\nu=c / \lambda\), или


\begin{equation*}
v=v_{0} \frac{\sqrt{1-\beta^{2}}}{1-v / c} . \tag{2.33}
\end{equation*}


Если источник приближается (как в нашем случае), то \(v>v_{0}\), если же удаляется, то \(v<v_{0}\) (в этом случае знак перед \(v\) меняется на противоположный). Полученная формула (2.33) соответствует продольному эффекту Доплера.

Как видно из приведенного вывода, эффект Доплера для электромагнитных волн является следствием двух явлений: замедления хода движущихся часов (корень в числителе последней формулы) и «уплотнения» (или разряжения) импульсов, связанного с изменением расстояния между источником и приемником - это учтено в первом равенстве формулы (2.32).

Рассмотрим и более общий случай: в \(K\)-системе источник \(S\) движется со скоростью \(\mathbf{v}\), составляющей угол \(\alpha\) с линией наблюдения (рис. 2.6). В этом случае в формуле (2.33) следует за-

\begin{figure}[h]
\begin{center}
  \includegraphics[alt={},max width=\textwidth]{950fefba-d83b-4c8f-96e7-0bd5a7eb90be-056_201_497_951_367}
\captionsetup{labelformat=empty}
\caption{Рис. 2.6}
\end{center}
\end{figure}

менить \(v\) на \(v_{x}=v \cos \alpha\), где \(v_{x}\) - проекция вектора \(\mathbf{v}\) на ось \(X\), положительное направление которой взято от \(S\) к \(P\). Тогда


\begin{equation*}
v=v_{0} \frac{\sqrt{1-\beta^{2}}}{1-v_{x} / c} \tag{2.34}
\end{equation*}


В процессе движения источника \(S\) проекция скорости \(v_{x}\), вообще говоря, меняется, поэтому необходимо учесть эффект запаздывания. Воспринимаемая приемником \(P\) частота \(v\) в момент \(t\) будет обусловлена сигналами, испущенными источником \(S\) в предшествующий момент \(t^{\prime}=t-l / c\), где \(l\) - расстояние от источника \(S\) до \(P\) в момент \(t^{\prime}\). Поэтому значение \(v_{x}\) надо брать в момент \(t^{\prime}\). Итак, частоте \(v(t)\) соответствует \(v_{x}\left(t^{\prime}\right)\).

В отличие от акустического эффекта Доплера, при \(\alpha=90^{\circ} \left(v_{x}=0\right)\) наблюдается поперечный эффект Доплера:


\begin{equation*}
v=v_{0} \sqrt{1-\beta^{2}} \tag{2.35}
\end{equation*}


при котором воспринимаемая приемником частота оказывается всегда меньше собственной частоты источника: \(v<v_{0}\). Поперечный эффект является прямым следствием замедления хода движущихся часов. Этот эффект значительно слабее продольного: он зависит от \(v / c\) не в первой степени, а во второй, т. е. является квадратичным относительно \(v / c\). Поэтому экспериментально его можно наблюдать, проводя измерения перпендикулярно, например, пучку излучающих атомов, имеющему очень малый угол расходимости (чтобы практически исключить продольный эффект).

В нерелятивистском случае, когда \(v \ll c\), вместо (2.31) можно считать, что \(T=T_{0}\), поэтому формула (2.34) не будет содержать корня \(\sqrt{1-\beta^{2}}\), и тогда воспринимаемая частота


\begin{equation*}
v=v_{0} /\left(1-v_{x} / c\right) \approx v_{0}\left(1+v_{x} / c\right) . \tag{2.36}
\end{equation*}


Отсюда относительное изменение частоты \(\left(v-v_{0}\right) / v_{0}\) равно


\begin{equation*}
\Delta v / v=v_{x} / c \tag{2.37}
\end{equation*}


При \(v_{x}>0\) (источник приближается) \(\Delta v / v>0\), если же \(v_{x}<0\) (источник удаляется), то \(\Delta v / v<0\). При \(v_{x}=0\) и \(\Delta v / v=0\).

Эффект Доплера нашел многочисленные практические применения. С его помощью определяют, например, скорость излучающих атомов в пучке, угловую скорость вращения Солнца. На эффекте Доплера основаны радиолокационные методы измерения скорости самолетов, ракет, автомашин и др. Именно этот эффект позволил открыть двойные звезды (системы, состоящие из двух звезд, движущихся вокруг общего центра масс) - объекты, которые невозможно разрешить даже самыми мощными телескопами. С помощью эффекта Доплера Хаббл (1929 г.) обнаружил явление, названное космологическим красным смещением: линии в спектре излучения внегалактических объектов смещены в сторону бо́льших длин волн, т.е. в красноволновую

часть спектра. Оно свидетельствует о том, что внегалактические объекты удаляются от нашей Галактики со скоростями, пропорциональными расстоянию до них.

Рассмотрим в заключение два примера на применение эффекта Доплера. Но предварительно преобразуем формулу (2.37) от частот к длинам волн. Частота \(\nu=c / \lambda\), отсюда малое приращение частоты \(\Delta v=-\left(c / \lambda^{2}\right) \Delta \lambda\). Подставив обе эти формулы в (2.37), получим


\begin{equation*}
\frac{\Delta \lambda}{\lambda}=-\frac{v_{x}}{c}=-\frac{v}{c} \cos \alpha \tag{2.37'}
\end{equation*}


где \(\alpha\) - угол между скоростью \(\mathbf{v}\) и направлением наблюдения.\\
Пример 1. Одна из спектральных линий, испускаемых возбужденными ионами \(\mathrm{He}^{+}\)в состоянии покоя, имеет длину волны \(\lambda\). Если эту линию наблюдать под углом \(\alpha\) к пучку данных ионов, то обнаруживается ее доплеровское смещение \(\Delta \lambda<0\), причем \(|\Delta \lambda| \ll \lambda\). Определим скорость ионов в пучке.\\
Так как \(|\Delta \lambda| \ll \lambda\), то это значит, что ионы движутся с нерелятивистской скоростью и справедливо соотношение (2.37'). Условие же \(\Delta \lambda<0\) означает согласно (2.37'), что \(\cos \alpha>0\), т. е. угол \(\alpha<\pi / 2\). Искомая скорость

\[
v=\frac{c|\Delta \lambda|}{\lambda \cos \alpha}
\]

Пример 2. При наблюдении спектральной линии \(\lambda=0,51\) мкм в направлениях на противоположные края солнечного диска на его экваторе обнаружили различие в длинах волн на \(\delta \lambda=8,0\) пм. Найдем период \(T\) вращения Солнца вокруг собственной оси.

Так как данные края диска движутся при вращении Солнца в противополжных направлениях с одинаковой скоростью \(v\), то доплеровское смещение этой линии будет одинаково по модулю, но противоположно по знаку. Поэтому суммарная разность смещенных длин волн равна удвоенному доплеровскому смещению:

\[
\frac{\delta \lambda}{\lambda}=2 \frac{v}{c}=2 \frac{\omega R}{c}=\frac{2}{c} \frac{2 \pi R}{T},
\]

где \(\omega\) - угловая скорость Солнца, \(R\) - его радиус ( \(\approx 7 \cdot 10^{8} \mathrm{M}\) ).

Отсюда следует, что период вращения Солнца

\[
T=\frac{4 \pi R \lambda}{c \delta \lambda}=25 \text { сут. }
\]

\section*{§ 2.7. Излучение диполя}
Возбуждение электромагнитных волн какой-либо системой называют излучением этих волн, а саму систему - излучающей системой. Поле электромагнитной волны называют полем излучения.

Согласно представлениям классической электродинамики электромагнитные волны в вакууме возбуждаются электрическими зарядами, движущимися с ускорением. Простейшей излучающей системой является осциллирующий электрический диполь, момент р которого изменяется с течением времени, элементарный вибратор.

Если излучающая система электронейтральна, а ее размеры малы по сравнению с длиной \(\lambda\) излучаемых волн, то в точках, отстоящих от системы на расстояниях \(r \gg \lambda\) - в так называемой волновой зоне, - поле излучения близко к полю излучения осциллятора, имеющего такой же электрический момент, как и вся излучающая система.

Рассмотрим некоторые закономерности излучения линейного гармонического осциллятора - электрического диполя, размер которого \(l \ll \lambda\), а момент \(\mathbf{p}\) изменяется во времени по закону


\begin{equation*}
\mathbf{p}=\mathbf{p}_{m} \cos \omega t, \tag{2.38}
\end{equation*}


где \(\mathbf{p}_{m}\) - амплитудное значение \(\mathbf{p}\). Все дальнейшее относится к вакууму, где длина волны \(\lambda\) излучения связана с частотой \(\omega\) соотношением \(\lambda=2 \pi c / \omega\).

Напомним, электрическое поле постоянного диполя спадает при удалении от него по закону \(E \sim 1 / r^{3}\). В случае же осциллирующего диполя дело обстоит иначе. В непосредственной близости от диполя картина электромагнитного поля очень сложна. Она сильно упрощается в волновой зоне: быстро спадающее статическое поле практически исчезает и остается только поле излучения от осциллирующих зарядов - расходящаяся сфери-

ческая волна с той же частотой, что и у осциллятора. Амплитуда волны (это доказывается в электродинамике) уменьшается с ростом расстояния \(r\) от диполя как


\begin{equation*}
E_{m} \circlearrowleft H_{m} \text { ⊸ } \frac{1}{r} \sin \vartheta, \tag{2.39}
\end{equation*}


где \(\vartheta\) - угол между осью диполя и ради-усом-вектором \(\mathbf{r}\) точки, где наблюдается поле (рис. 2.7). Из этого рисунка видно, что вектор E в каждой точке волновой зоны направлен по касательной к меридиану, а вектор H - по касательной к параллели, причем так, что в каждый момент векторы \(\mathbf{E}\) и \(\mathbf{H}\) составляют правую тройку с вектором Пойнтинга \(\boldsymbol{\Pi}=[\mathbf{E H}]\).

\begin{figure}[h]
\begin{center}
  \includegraphics[alt={},max width=\textwidth]{950fefba-d83b-4c8f-96e7-0bd5a7eb90be-060_334_304_443_834}
\captionsetup{labelformat=empty}
\caption{Рис. 2.7}
\end{center}
\end{figure}

Факт существования электромагнитного поля, амплитуда которого убывает с расстоянием как \(1 / r\), - поля излучения весьма важен. Наличие именно такого поля позволяет осуществлять передачи на большие расстояния, видеть звезды.

Интенсивность электромагнитной волны, т. е. среднее значение плотности потока энергии \(\langle\Pi\rangle\), пропорционально произведению \(E_{m} H_{m}\), значит, согласно (2.39),


\begin{equation*}
I=\langle\Pi\rangle \diamond \frac{1}{r^{2}} \sin ^{2} \vartheta . \tag{2.40}
\end{equation*}


Зависимость \(I(\vartheta)\) наглядно изображают с помощью диаграммь направленности излучения диполя (рис. 2.8). Здесь длина отрезка \(O O^{\prime}\), отсекаемого на луче под углом Э, дает интенсивность излучения под этим углом. Видно, что максимум излучения происходит в экваториальной плоскости ( \(\vartheta=\pi / 2\) ), а вдоль

\begin{figure}[h]
\begin{center}
  \includegraphics[alt={},max width=\textwidth]{950fefba-d83b-4c8f-96e7-0bd5a7eb90be-060_311_374_1295_779}
\captionsetup{labelformat=empty}
\caption{Рис. 2.8}
\end{center}
\end{figure}

оси ( \(\vartheta=0\) ) диполь не излучает совсем это важный вывод.

Как показывает теория, мощность излучения \(P\) диполя, т. е. энергия, излучаемая в единицу времени по всем направлениям,

пропорциональна квадрату второй производной дипольного момента по времени и определяется формулой


\begin{equation*}
P=\alpha \ddot{\mathbf{p}}^{2}, \tag{2.41}
\end{equation*}


где \(\alpha=\mu_{0} / 6 \pi c\) (СИ) или \(2 / 3 c^{3}\) (СГС). Зная зависимость \(\mathbf{p}\) от \(t\), формула (2.38), получим:


\begin{equation*}
P=\alpha \omega^{4} p_{m}^{2} \cos ^{2} \omega t \tag{2.42}
\end{equation*}


Следовательно, средняя по времени мощность излучения диполя


\begin{equation*}
\langle P\rangle=(\alpha / 2) \omega^{4} p_{m}^{2} . \tag{2.43}
\end{equation*}


Это важный результат: средняя мощность излучения осциллирующего диполя зависит от квадрата его амплитуды и очень сильно от частоты (как \(\omega^{4}\) ). Отсюда следует, что, например, радиостанции должны использовать высокие частоты, а излучение линий передач переменного тока промышленной частоты оказывается незначительным.

Формула (2.41) справедлива также для излучения заряда \(q\), движущегося ускоренно. В самом деле, дипольный момент можно представить так: \(\mathbf{p}=q \mathbf{l}=q\left(\mathbf{r}_{+}-\mathbf{r}_{-}\right)\), где \(\mathbf{r}_{+}\)и \(\mathbf{r}_{-}\)- радиу-сы-векторы зарядов \(q\) и \(-q\). Отсюда

\[
\ddot{\mathbf{p}}=q\left(\ddot{\mathbf{r}}_{+}-\ddot{\mathbf{r}}_{-}\right)=q\left(\mathbf{a}_{+}-\mathbf{a}_{-}\right),
\]

и если заряд, например \(q\), покоится, а движется только заряд \(-q\), то

\[
\ddot{\mathbf{p}}=-q \mathbf{a}_{-} .
\]

После подстановки этого выражения в формулу (2.41) найдем:


\begin{equation*}
P_{t}=\alpha q^{2} \mathbf{a}_{t^{\prime}}^{2}, \tag{2.44}
\end{equation*}


где \(\alpha\) - тот же коэффициент, что и в формуле (2.41).\\
Это знаменитая формула для мощности излучения заряда, движущегося с ускорением. Индексы \(t\) и \(t^{\prime}\) показывают, что мощность \(P\) в момент \(t\) определяется ускорением заряда, которое

он имеет в более ранний момент \(t^{\prime}=t-l / c\) (эффект запаздывания). И еще, формула (2.44), как следует из теории, справедлива лишь для зарядов, движущихся с малыми скоростями ( \(v \ll c\) ).

В качестве примера можно привести заряженные частицы, движущиеся в циклических ускорителях (бетатроне, циклотроне и др.). Здесь обнаруживается естественный предел для энергии ускоряемой частицы, когда энергия, сообщаемая частице за период, становится равной энергии излучения.

Другой пример - излучение электрона в атоме. По классическим представлениям электрон в атоме совершает колебания, т. е. движется с ускорением и, значит, излучает. Расчет показывает, что время \(\tau\), за которое амплитуда колебаний электрона уменьшается в е раз, порядка \(10^{-8}\) с. Это время называют средним временем жизни возбужденного атома, или временем излучения. Точный (квантовый) расчет приводит практически к тому же значению этого времени.

Следует обратить внимание на то, что заряд, колеблющийся с частотой \(\omega\), излучает монохроматическую электромагнитную волну с той же частотой \(\omega\). Если же заряд движется с произвольным ускорением, то его излучение представляет собой спектр различных частот.

И последнее, заряд, движущийся в вакууме с постоянной скоростью, не излучает. В этом легко убедиться и непосредственно. Достаточно перейти в систему отсчета, где заряд покоится (а такой заряд не излучает) и затем воспользоваться принципом относительности: если этого явления (излучения) нет в одной системе отсчета, его нет и в других, по отношению к которым заряд движется*.

\section*{Задачи}
2.1. Плоская электромагнитная волна \(\mathbf{E}=\mathbf{E}_{m} \cos (\omega t-\mathbf{k r})\) распространяется в вакууме. Найти вектор \(\mathbf{H}\) как функцию времени в точке с радиусом-вектором \(\mathbf{r}=0\).

\footnotetext{\begin{itemize}
  \item Это относится только к движению в вакууме. Если же заряд движется с постоянной скоростью в среде, то в случае, когда его скорость превышает фазовую скорость электромагнитных волн в этой среде, наблюдается излучение Вавило-ва-Черенкова (см. Приложение).
\end{itemize}
}Р е шени е. Искомый вектор \(\mathbf{H}=\mathbf{H}_{m} \cos \omega t\), где \(\mathbf{H}_{m}\) можно найти из условия, что векторы \(\mathbf{E}_{m}, \mathbf{H}_{m}\) и \(\mathbf{k}\) составляют правую тройку, откуда \(\mathbf{H}_{m} \uparrow \uparrow\left[\mathbf{k} \mathbf{E}_{m}\right]\). Кроме того, \(\sqrt{\varepsilon_{0}} E_{m}=\sqrt{\mu_{0}} H_{m}\). Поэтому \(\mathbf{H}_{m}=H_{m} \mathbf{e}_{H}\), где \(\mathbf{e}_{H}\) - орт вектора \(\mathbf{H}_{m}\), равный \(\left[\mathbf{k} \mathbf{E}_{m}\right] / k E_{m}\), и, следовательно, \(H_{m}=\sqrt{\varepsilon_{0} / \mu_{0}} E_{m} \cdot\left[\mathbf{k} \mathbf{E}_{m}\right] / k E_{m}\). В результате получим:

\[
\mathbf{H}=\frac{1}{k} \sqrt{\frac{\varepsilon_{0}}{\mu_{0}}}\left[\mathbf{k} \mathbf{E}_{m}\right] \cos \omega t
\]

2.2. Вектор Пойнтинга. В вакууме распространяется плоская электромагнитная волна, амплитуда электрической составляющей которой равна \(E_{m}\). Найти среднюю за период колебаний плотность потока энергии.\\
Решение. Модуль вектора Пойнтинга, \(\Pi=E H\), с учетом (2.14) примет вид \(\Pi=\sqrt{\varepsilon_{0} / \mu_{0}} E_{m}^{2} \cos ^{2} \omega t\). Отсюда

\[
\langle\Pi\rangle=\sqrt{\varepsilon_{0} / \mu_{0}} E_{m}^{2} / 2
\]

где принято во внимание, что \(\left\langle\cos ^{2} \omega t\right\rangle=1 / 2\).\\
2.3. В вакууме вдоль оси \(X\) распространяются две плоские электромагнитные волны, электрические составляющие которых изменяются по закону \(\mathbf{E}_{1}=\mathbf{E}_{0} \cos (\omega t-k x)\) и \(\mathbf{E}_{2}=\mathbf{E}_{0} \cos (\omega t-k x+\varphi)\). Найти среднее значение плотности потока энергии.\\
Решение. Исходим из соотношений \(\Pi=E H\) и \(\sqrt{\varepsilon_{0}} E=\sqrt{\mu_{0}} H\). Отсюда \(\langle I\rceil\rangle=\sqrt{\varepsilon_{0} / \mu_{0}}\left\langle E^{2}\right\rangle\), где в любой

\begin{figure}[h]
\begin{center}
  \includegraphics[alt={},max width=\textwidth]{950fefba-d83b-4c8f-96e7-0bd5a7eb90be-063_267_348_1156_152}
\captionsetup{labelformat=empty}
\caption{Рис. 2.9}
\end{center}
\end{figure}

точке \(E^{2}=E_{m}^{2} \cos ^{2} \omega t\) (начальная фаза не существенна). Найдем \(E_{m}^{2}\). Согласно векторной (или фазовой) диаграмме эта амплитуда вектора \(\mathbf{E}\) является суммой векторов \(\mathbf{E}_{0}\), разность фаз между которыми равна \(\varphi\) (рис. 2.9). Из теоремы косинусов имеем \(E_{m}^{2}=2(1+\cos \varphi) E_{0}^{2}\). В результате искомая величина

\[
\langle\Pi\rangle=\sqrt{\varepsilon_{0} / \mu_{0}}(1+\cos \varphi) E_{0}^{2},
\]

где учтено, что \(\left\langle\cos ^{2} \omega t\right\rangle=1 / 2\).\\
2.4. В вакууме распространяются две плоские электромагнитные волны, одна - вдоль оси \(X\), другая - вдоль оси \(Y\) :

\[
\mathbf{E}_{1}=\mathbf{E}_{0} \cos (\omega t-k x), \quad \mathbf{E}_{2}=\mathbf{E}_{0} \cos (\omega t-k y),
\]

где вектор \(\mathbf{E}_{0}\) направлен параллельно оси \(Z\). Найти среднее значение плотности потока энергии в точках плоскости \(y=x\).\\
Решение. В плоскости \(y=x\) векторы \(\mathbf{E}_{1}\) и \(\mathbf{E}_{2}\) будут колебаться в фазе с амплитудой \(E_{m}=2 E_{0}\). Векторы же \(\mathbf{H}_{1}\) и \(\mathbf{H}_{2}\) - тоже в фазе, но под углом \(\pi / 2\) друг к другу, поэтому амплитуда результирующего вектора \(H_{m}=\sqrt{2} H_{0}\), где \(H_{0}\) связано с \(E_{0}\) соотношением \(\sqrt{\mu_{0}} H_{0}=\sqrt{\varepsilon_{0}} E_{0}\). Поэтому среднее значение плотности потока энергии в плоскости \(y=x\) равно

\[
\langle\Pi\rangle=\left\langle 2 E_{0} \cdot H_{0} \sqrt{2}\right\rangle=\sqrt{2 \varepsilon_{0} / \mu_{0}} E_{0}^{2},
\]

где учтено, что \(\left\langle\cos ^{2} \omega t\right\rangle=1 / 2\).\\
2.5. Стоячая волна. Пусть электрическая составляющая стоячей электромагнитной волны имеет вид \(E_{y}=E_{m} \cos k x \cdot \cos \omega t\). Найти с помощью уравнений (2.12) выражение для магнитной составляющей этой волны \(H_{z}(x, t)\).\\
Р e ш e н и е. Согласно второму уравнению (2.12),

\[
\partial H_{z} / \partial x=-\varepsilon_{0} \partial E_{y} / \partial t=\varepsilon_{0} \omega E_{m} \cos k x \cdot \sin \omega t .
\]

Проинтегрировав это выражение по \(x\), получим:

\[
H_{z}=\varepsilon_{0} c E_{m} \sin k x \cdot \sin \omega t+\mathrm{const},
\]

где \(c=\omega / k=1 / \sqrt{\varepsilon_{0} \mu_{0}}\). Нас интересует только переменное поле, поэтому const \(=0\). Учитывая связь (2.19) между амплитудами \(E_{m}\) и \(H_{m}\), найдем:

\[
H_{z}=H_{m} \sin k x \cdot \sin \omega t,
\]

т. е. \(H_{z}\) имеет вид тоже стоячей волны, но сдвинутой в пространстве (по \(x\) ) на четверть волны и по времени - на четверть периода.\\
2.6. Эффект Доплера. Радиолокатор работает на частоте \(v_{0}\). Найти скорость \(v\) приближающегося самолета, если частота биений между сигналами передатчика и отраженными от самолета равна \(\Delta v\) (в месте расположения локатора).\\
Решение. Здесь мы имеем дело с нерелятивистским эффектом Доплера. Частота сигналов, воспринимаемая самолетом как приемником, к которому приближается локатор, согласно (2.36), равна

\[
v=v_{0} /(1-v / c)=v_{0}(1+v / c),
\]

поскольку \(v_{x}=v\). Сигналы такой же частоты \(v\) самолет и отражает - уже как движущийся источник. Поэтому приемник локатора принимает сигналы с частотой

\[
v^{\prime}=v(1+v / c)=v_{0}(1+v / c)^{2} \approx v_{0}(1+2 v / c)
\]

Частота биений \(\Delta v=v^{\prime}-v_{0}=2 v_{0} v / c\). Отсюда \(v=c \Delta v / 2 v_{0}\).\\
2.7. С какой скоростью должна была бы двигаться автомашина, чтобы красный свет светофора, \(\lambda_{0} \approx 0,70\) мкм, воспринимался как зеленый, \(\lambda \approx 0,55\) мкм (анекдот о Вуде)?\\
Р е ш е н и е. Согласно (2.34), \(v=v_{0} \sqrt{(1+\beta) /(1-\beta)}\), \(\beta=v / c\).\\
Отсюда

\[
\beta=\frac{\left(v / v_{0}\right)^{2}-1}{\left(v / v_{0}\right)^{2}+1}=\frac{\left(\lambda_{0} / \lambda\right)^{2}-1}{\left(\lambda_{0} / \lambda\right)^{2}+1} .
\]

В результате находим \(v=\beta c \approx 7 \cdot 10^{4}\) км \(/ \mathrm{c}\) !\\
2.8. Эффект запаздывания. Источник \(S\), испускающий электромагнитные сигналы с частотой \(v_{0}\), движется с релятивистской скоростью \(v\) по прямой, отстоящей на некоторое расстояние от неподвижного наблюдателя \(P\) (рис.2.10). Найти частоту сигналов, воспринимаемых наблюдателем в момент, когда:\\
a) источник окажется в точке \(O\);\\
б) наблюдатель увидит его в точке \(O\).

Решение. а) В этот момент в точку \(P\) должны прийти сигналы,

\begin{figure}[h]
\begin{center}
  \includegraphics[alt={},max width=\textwidth]{950fefba-d83b-4c8f-96e7-0bd5a7eb90be-065_271_350_1232_163}
\captionsetup{labelformat=empty}
\caption{Рис. 2.10}
\end{center}
\end{figure}

испущенные источником \(S\) левее точки \(O\), когда его скорость составляла некоторый угол \(\alpha\) с прямой \(S P\). Этот угол должен удовлетворять условию \(\cos \alpha=v \tau / c \tau=v / c\), где \(\tau\) - время, за которое источник из точки \(S\) достигнет точки \(O\). За это же время сигналы достигнут \(P\). Тогда, согласно (2.34),

\[
v=v_{0} \frac{\sqrt{1-\beta^{2}}}{1-(v / c) \cos \alpha}=v_{0} \frac{\sqrt{1-\beta^{2}}}{1-\beta \cdot \beta}=\frac{v_{0}}{\sqrt{1-\beta^{2}}}, \beta=v / c
\]

б) В этом случае будет наблюдаться чисто поперечный эффект Доплера: \(v=v_{0} \sqrt{1-\beta^{2}}\).\\
2.9. Излучение диполя. Электромагнитная волна, излучаемая диполем, распространяется в вакууме так, что в волновой зоне на луче, перпендикулярном оси диполя, на расстоянии \(r\) от него интенсивность равна \(I_{0}\). Найти среднюю мощность \(P\) излучения диполя.

Решение. Прежде всего найдем мощность \(\mathrm{d} P\) излучения, проходящего через кольцевую полоску \(d S\) на сфере радиуса \(r\) (рис. 2.11). Площадь этой полоски равна произведению ее длины \(2 \pi r \sin \vartheta\) на ширину \(r\) d \(\vartheta\). Учитывая, кроме того, что, согласно (2.40), интенсивность под углом Э относится к интенсивности под углом \(\vartheta= =90^{\circ}\) как \(I / I_{0}=\sin ^{2} \vartheta\), запишем

\[
\mathrm{d} P=I \mathrm{~d} S=I_{0} \sin ^{2} \vartheta \cdot 2 \pi r \sin \vartheta r \mathrm{~d} \vartheta .
\]

\begin{figure}[h]
\begin{center}
  \includegraphics[alt={},max width=\textwidth]{950fefba-d83b-4c8f-96e7-0bd5a7eb90be-066_348_322_371_826}
\captionsetup{labelformat=empty}
\caption{Рис. 2.11}
\end{center}
\end{figure}

Проинтегрировав это выражение по \(\vartheta\) от 0 до \(\pi\), получим:

\[
P=(8 \pi / 3) r^{2} I_{0} .
\]

По виду данной формулы может показаться, что \(P\) зависит от \(r\). Но это не так, поскольку \(I_{0} \circlearrowleft 1 / r^{2}\).\\
2.10. Постоянный по модулю электрический диполь \(p\) вращают с постоянной угловой скоростью \(\omega\) вокруг оси, перпендикулярной оси диполя и проходящей через его середину. Найти мощность излучения такого диполя.

Решение. Согласно (2.41), вопрос сводится к определению второй производной вектора \(\mathbf{p}\) по времени. За время \(\mathrm{d} t\) модуль приращения вектора \(\mathbf{p}\) равен \(|\mathrm{d} \mathbf{p}|=p \omega \mathrm{~d} t\), значит \(\mathrm{d} \mathbf{p}=[\omega \mathbf{p}] \mathrm{d} t\) и \(\dot{\mathbf{p}}=[\omega \mathbf{p}]\). Из последнего равенства видно, что при вращении постоянного по модулю вектора \(\mathbf{p}\) его производная по времени \(\dot{\mathbf{p}}\) определяется как векторное произведение \(\boldsymbol{\omega}\) на вектор \(\mathbf{p}\). Это справедливо для любого вектора, в частности и для \(\dot{\mathbf{p}}\). Его производная по времени, т. е. \(\ddot{\mathbf{p}}=[\omega \dot{\mathbf{p}}]=[\omega[\omega \mathbf{p}]]\), или по модулю \(|\ddot{\mathbf{p}}|=\omega^{2} p\). После подстановки этого выражения в (2.41) получим

\[
P=\left(\mu_{0} / 6 \pi c\right) p^{2} \omega^{4} .
\]

\section*{Часть 11}
\section*{Волновая оптика}
\section*{Глава 3}
\section*{Вступление}
\section*{§ 3.1. Световая волна}
Шкала электромагнитных волн. В этом разделе будет рассмотрен круг явлений, в основе которых лежит волновая природа света. Различают несколько видов электромагнитных волн: радиоволны, оптический диапазон, рентгеновское и гам-ма-излучения. В дальнейшем нас будет интересовать главным образом оптический диапазон длин волн. Его подразделяют на

\[
\begin{array}{ll}
\text { ультрафиолетовое излучение... } & \lambda=0,01 \div 0,40 \text { мкм, } \\
\text { видимое излучение (свет)....... } & \lambda=0,40 \div 0,76 \text { мкм, }  \tag{3.1}\\
\text { инфракрасное излучение....... } & \lambda=0,76 \text { мкм } \div 1 \text { мм. }
\end{array}
\]

Соответствующие длины волн указаны в вакууме.\\
Показатель преломления. Электромагнитная волна характеризуется векторами \(\mathbf{E}\) и \(\mathbf{H}\). Поскольку практически все действия света связаны с вектором \(\mathbf{E}\), принято говорить о световом векторе, имея в виду вектор E.

Модуль амплитуды светового вектора мы будем обозначать буковой \(A\) или \(E_{m}\).

Показатель преломления \(n\) некоторой среды определяют как


\begin{equation*}
n=c / v \tag{3.2}
\end{equation*}


где \(c\) - скорость света в вакууме, \(v\) - фазовая скорость в данной среде. Сопоставление с формулой (2.8) дает:


\begin{equation*}
n=\sqrt{\varepsilon}, \tag{3.3}
\end{equation*}


что справедливо для подавляющего большинства прозрачных веществ, у которых \(\mu \approx 1\).

Последняя формула связывает оптические свойства вещества с его электрическими свойствами. Заметим, что \(\varepsilon\) зависит от частоты электромагнитной волны. Этим объясняется дисперсия све\(m a\), т. е. зависимость \(n\) (или \(v\) ) от частоты (или длины волны).

Показатель преломления \(n\) характеризует оптическую плотность среды. Среду с бо́льшим показателем преломления называют оптически более плотной.

В веществе длина волны \(\lambda^{\prime}=v / v=c / n v=\lambda / n\). Таким образом, длина волны света в среде с показателем преломления \(n\) равна


\begin{equation*}
\lambda^{\prime}=\lambda / n . \tag{3.4}
\end{equation*}


Интенсивность волны. Световую волну характеризуют интенсивностью I - это модуль среднего по времени значения плотности потока энергии. Плотность потока электромагнитной энергии определяется, как мы уже знаем, вектором Пойнтинга П как


\begin{equation*}
I=\langle\Pi\rangle \circlearrowleft E_{m} H_{m} . \tag{3.5}
\end{equation*}


Согласно (2.14), \(H_{m} \circlearrowleft \sqrt{\varepsilon} E_{m}=n E_{m}\), поэтому формулу (3.5) можно записать так:


\begin{equation*}
I \sim n E_{m}^{2}=n A^{2} . \tag{3.6}
\end{equation*}


Линии, ортогональные волновым поверхностям, называют лучами. Вектор Пойнтинга направлен в каждой точке по касательной к лучу. Это, однако, относится только к изотропным средам.

Виды световых волн. Световые волны являются электромагнитными, поэтому они поперечны. Однако обычно они не обнаруживают асимметрии относительно направления распространения. Это связано с тем, что в свете, испускаемом обычными источниками - этот свет называют естественным - колебания светового вектора происходят поочередно в самых разных направлениях, перпендикулярных направлению распространения.

По классическим представлениям излучение светящегося тела (газа) слагается из волн, испускаемых его атомами. Излучение отдельного атома продолжается порядка \(10^{-8}\) с и представляет со-

бой, как говорят, цуг волн протяженностью в среднем порядка 3 м. Излучив, атом через некоторое время, придя в возбужденное состояние, излучает опять и т. д. Одновременно излучает множество атомов. Порожденные ими цуги волн, налагаясь друг на друга, образуют испускаемую телом световую волну. Направления колебаний для каждого цуга ориентированы случайным образом. Поэтому в результирующей световой волне колебания светового вектора происходят в разных направлениях с равной вероятностью. Это надо понимать так, что при прохождении световой волны через некоторую точку колебания светового вектора быстро и

\begin{figure}[h]
\begin{center}
  \includegraphics[alt={},max width=\textwidth]{950fefba-d83b-4c8f-96e7-0bd5a7eb90be-069_243_237_619_84}
\captionsetup{labelformat=empty}
\caption{Рис. 3.1}
\end{center}
\end{figure}

беспорядочно сменяют друг друга. Но в пределах некоторого короткого времени мы имеем дело со световым вектором, направление колебаний которого сохраняется, затем направление колебаний меняется на другое и т. д. При этом модуль светового вектора остается неизменным. Условно это изображают как на рис. 3.1, где направление распространения волны перпендикулярно плоскости рисунка.

Свет, в котором направление колебаний светового вектора упорядочено каким-либо образом, называют поляризованным. Если колебания светового вектора происходят только в одной плоскости, свет называют плоско- (или линейно-) поляризованным.

Если конец светового вектора описывает эллипс, то такой свет называют эллиптически-поляризованным (в частности, поляризованным по кругу).

Создание принципиально нового источника света - лазера позволило получить плоско-поляризованный свет с высокой степенью монохроматичности. Использование такого источника света сильно упростило экспериментальное решение многих вопросов, связанных с интерференцией, дифракцией и др.

\section*{§ 3.2. Электромагнитная волна на границе раздела}
Соотношения между амплитудами и фазами. Выясним, что происходит при падении плоской электромагнитной волны на границу раздела двух однородных изотропных прозрачных диэлектриков, магнитная проницаемость которых равна единице\\
( \(\mu=1\) ). Известно, что при этом возникают отраженная и преломленная волны. Ограничимся рассмотрением частного, но практически важного случая, когда волна падает нормально на границу раздела диэлектриков с показателями преломления \(n_{1}\) и \(n_{2}\) (результаты, которые мы получим, оказывается, справедливы и при наклонном падении волны).

Обозначим электрическую составляющую в падающей, отраженной и преломленной волнах соответственно через \(\mathbf{E}, \mathbf{E}^{\prime}\) и \(\mathbf{E}^{\prime \prime}\), а магнитную составляющую - через \(\mathbf{H}, \mathbf{H}^{\prime}\) и \(\mathbf{H}^{\prime \prime}\). Из соображений симметрии ясно, что колебания векторов \(\mathbf{E}, \mathbf{E}^{\prime}\) и \(\mathbf{E}^{\prime \prime}\) происходят в одной плоскости. Это же относится и к векторам \(\mathbf{H}, \mathbf{H}^{\prime}\) и \(\mathbf{H}^{\prime \prime}\). На рис. 3.2 показаны относительное расположение этих векторов в непосредственной близости от границы раздела и направления распространения всех трех волн, обозначенные векторами \(\mathbf{k}, \mathbf{k}^{\prime}\) и \(\mathbf{k}^{\prime \prime}\). Дальнейший расчет покажет, насколько эта картина соответствует действительности.

\begin{figure}[h]
\begin{center}
  \includegraphics[alt={},max width=\textwidth]{950fefba-d83b-4c8f-96e7-0bd5a7eb90be-070_491_386_544_722}
\captionsetup{labelformat=empty}
\caption{Рис. 3.2}
\end{center}
\end{figure}

Воспользуемся граничными условиями для тангенциальных составляющих векторов \(\mathbf{E}\) и \(\mathbf{H}\) :


\begin{equation*}
E_{1 y}=E_{2 y}, \quad H_{1 z}=H_{2 z} . \tag{3.7}
\end{equation*}


Перепишем эти условия для нашего случая:


\begin{align*}
& E_{y}+E_{y}^{\prime}=E_{y}^{\prime \prime}  \tag{3.8}\\
& H_{z}+H_{z}^{\prime}=H_{z}^{\prime \prime} \tag{3.9}
\end{align*}


Согласно (2.14), \(H_{z} v \sqrt{\varepsilon} E_{y}=n_{1} E_{y}, H_{z}^{\prime \prime} v n_{2} E_{y}^{\prime \prime}\), но \(H_{z}^{\prime} v-n_{1} E_{y}^{\prime}\), поскольку проекции \(E_{y}^{\prime}\) и \(H_{z}^{\prime}\) в отраженной волне имеют противоположные знаки (см. рис. 3.2). Поэтому равенство (3.9) можно переписать так: \(n_{1} E_{y}-n_{1} E_{y}^{\prime}=n_{2} E_{y}^{\prime \prime}\), или


\begin{equation*}
E_{y}-E_{y}^{\prime}=\frac{n_{2}}{n_{1}} E_{y}^{\prime \prime} . \tag{3.10}
\end{equation*}


Решив совместно уравнения (3.8) и (3.10), получим выражения для \(E_{y}^{\prime}\) и \(E_{y}^{\prime \prime}\) через \(E_{y}\), которые в векторной форме имеют вид:


\begin{equation*}
\mathbf{E}^{\prime}=\frac{n_{1}-n_{2}}{n_{1}+n_{2}} \mathbf{E}, \quad \quad \mathbf{E}^{\prime \prime}=\frac{2 n_{1}}{n_{1}+n_{2}} \mathbf{E} \tag{3.11}
\end{equation*}


Отсюда следует, что:

\begin{enumerate}
  \item Вектор \(\mathbf{E}^{\prime \prime}\) всегда сонаправлен с вектором \(\mathbf{E}\), т. е. оба вектора колеблются синфазно - при прохождении волны через границу раздела фаза не претерпевает скачка.
  \item Это же относится и к векторам \(\mathbf{E}^{\prime}\) и \(\mathbf{E}\), но при условии, что \(n_{1}>n_{2}\), т. е. если волна переходит в оптически менее плотную среду. В случае же, когда \(n_{1}<n_{2}\), дробь в выражении (3.11) для \(\mathbf{E}^{\prime}\) оказывается отрицательной, а это означает, что направление вектора \(\mathbf{E}^{\prime}\) противоположно направлению вектора \(\mathbf{E}\), т. е. колебания вектора \(\mathbf{E}^{\prime}\) происходят в противофазе с колебаниями вектора \(\mathbf{E}\) (этому соответствует рис. 3.2). Другими словами, npu отражении волны от оптически более плотной среды ее фаза изменяется скачком на π.
\end{enumerate}

Эти результаты мы будем использовать в дальнейшем при изучении интерференции волн, отраженных от обеих поверхностей тонких пластинок.

Коэффициенты отражения и пропускания. Вопрос об этих коэффициентах мы рассмотрим для случая нормального падения световой волны на границу раздела двух прозрачных диэлектриков. Ранее мы выяснили, что интенсивность \(I\) гармонической волны, согласно (3.6), пропорциональна \(\sqrt{\varepsilon} E_{m}^{2}\), или \(I \sim n E_{m}^{2}\). Коэффициент отражения, по определению, есть \(\rho=I^{\prime} / I=n_{1} E_{m}^{\prime 2} / n_{1} E_{m}^{2}\). После подстановки отношения \(E_{m}^{\prime} / E_{m}\) из первой формулы (3.11), найдем:


\begin{equation*}
\rho=\frac{I^{\prime}}{I}=\left(\frac{n_{1}-n_{2}}{n_{1}+n_{2}}\right)^{2} . \tag{3.12}
\end{equation*}


Обратим внимание на то, что \(\rho\) не зависит от направления падающей волны на границу раздела: из среды 1 в среду 2 , или наоборот.

Аналогично находим и коэффициент пропускания \(\tau\) как отношение \(I^{\prime \prime} / I\). Согласно (3.6), \(I^{\prime \prime} / I=n_{2} E_{m}^{\prime \prime 2} / n_{1} E_{m}^{2}\). Остается учесть вторую формулу из (3.11), и мы получим, что коэффициент пропускания


\begin{equation*}
\tau=\frac{I^{\prime \prime}}{I}=\frac{4 n_{1} n_{2}}{\left(n_{1}+n_{2}\right)^{2}} . \tag{3.13}
\end{equation*}


Нетрудно убедиться в том, что сумма обоих коэффициентов \(\rho+\tau=1\), как и должно быть.

При нормальном (или близком к нему) падении световой волны на границу раздела воздух ( \(n_{1} \approx 1\) ) и стекло ( \(n_{2} \approx 1,5\) ) получим \(\rho=0,04\), т.е. отражается около \(4 \%\) падающего света.

Заметим (это важно!), что если свет падает не по нормали к границе раздела, то коэффициент пропускания должен определяться через отношение не интенсивностей, а потоков, поскольку сечения падающего и преломленного пучков в этом случае различны, в отличие от падающего и отраженного. В связи с этим падающий поток \(\Phi=\Phi^{\prime}+\Phi^{\prime \prime}\), но \(I \neq I^{\prime}+I^{\prime \prime}\). Итак, в общем случае \(\tau=\Phi^{\prime \prime} / \Phi\).

\section*{§ 3.3. Геометрическая оптика}
Длины световых волн, воспринимаемых глазом, очень малы (менее 1 мкм). Поэтому распространение света во многих случаях можно рассматривать, отвлекаясь от его волновой природы, и считать, что свет распространяется вдоль лучей.

В пределе ( \(\lambda \rightarrow 0\) ) законы оптики можно сформулировать на языке геометрии. Соответствующий раздел оптики называют геометрической (или лучевой) оптикой.

Основу геометрической оптики составляют три закона:

\begin{enumerate}
  \item Закон прямолинейного распространения света (в однородной среде).
  \item Закон отражения света: угол отражения \(\vartheta^{\prime}\) света равен его углу падения Э,
\end{enumerate}


\begin{equation*}
\vartheta^{\prime}=\vartheta, \tag{3.14}
\end{equation*}


и оба луча лежат в одной плоскости с нормалью к поверхности раздела, ее называют плоскостью падения.\\
3. Закон преломления света (закон Снелла): при преломлении света на границе раздела двух изотропных сред с показателями преломления \(n_{1}\) и \(n_{2}\) выполняется условие


\begin{equation*}
n_{1} \sin \vartheta_{1}=n_{2} \sin \vartheta_{2} . \tag{3.15}
\end{equation*}


Отсюда, в частности, следует, что при падении света на оптически менее плотную среду ( \(n_{2}<n_{1}\) ) угол \(\vartheta_{2}\) может достигнуть \(90^{\circ}\). Соответствующий угол падения называют предельным:


\begin{equation*}
\sin \vartheta_{1 \text { пр }}=n_{2} / n_{1} . \tag{3.16}
\end{equation*}


Для углов падения \(\vartheta_{1}>\vartheta_{1 \text { пр }}\) наблюдается полное внутреннее отражение. Это явление нашло достаточно широкое практическое применение (призмы полного внутреннего отражения, световоды и др.).

Заметим, что законы отражения и преломления света могут быть получены как следствие поведения электромагнитной волны на границе раздела двух диэлектриков с учетом граничных условий для векторов E, H и D, B (см. приложение 1).

Принцип Ферма. Этот принцип также может быть положен в основу геометрической оптики (вместо перечисленных выше трех законов). Прежде чем его сформулировать, рассмотрим путь \(1-2\), который проходит луч света в неоднородной среде

\begin{figure}[h]
\begin{center}
  \includegraphics[alt={},max width=\textwidth]{950fefba-d83b-4c8f-96e7-0bd5a7eb90be-073_169_390_1153_68}
\captionsetup{labelformat=empty}
\caption{Рис. 3.3}
\end{center}
\end{figure}

(рис. 3.3). Участок пути \(\mathrm{d} s\) свет проходит за время \(\mathrm{d} t=\mathrm{d} s / v\), где \(v\) - скорость света в данном месте. Так как \(v= =c / n\), то \(\mathrm{d} t=n \mathrm{~d} s / c\), и время \(\tau\) для прохождения светом пути \(1-2\) равно


\begin{equation*}
\tau=\frac{1}{c} \int_{1}^{2} n \mathrm{~d} s=\frac{L}{c}, \tag{3.17}
\end{equation*}


где \(L=\int n \mathrm{~d} s\) называют оптической длиной пути. В однородной среде \(L=n s\), где \(s\) - геометрический путь. Заметим, что оптическая длина пути \(L\) эквивалентна по времени распространения света в вакууме, т. е. время распространения света на пути \(s\) в веществе со скоростью \(v\) такое же, как в вакууме со скоростью \(c\) на пути \(L\).

Принцип Ферма утверждает: свет распространяется по такому пути, оптическая длина которого минимальна. Точнее, оптическая длина пути должна быть экстремальной, т. е. либо минимальной, либо максимальной, либо стационарной одинаковой для всех возможных путей. В последнем случае все пути света между двумя точками оказываются согласно (3.17) таутохронными (требующими для своего прохождения одинакового времени).

Убедимся в том, что из принципа Ферма действительно следуют все три закона геометрической оптики.

\begin{enumerate}
  \item Пусть свет попадает из точки 1 в точку 2 (рис. 3.4). Здесь возможны сразу два случая: прямой путь \(1-2\) и при отражении от границы раздела двух однородных сред на пути \(1 O 2\). Среда, в которой распространяется луч, однородна, поэтому минимальность оптической длины пути сводится к минимальности его геометрической длины.
\end{enumerate}

Из рис. 3.4 видно, что прямой путь \(1-2\) действительно минимальный: ближайшие к нему будут длиннее. То же можно сказать и о пути 1O2. Он оказывается тоже минимальным при \(\vartheta^{\prime}=\vartheta\) (вспомогательная точка \(1^{\prime}\) - это зеркальное изображение точки 1). Длина любого из ближайших путей, например, \(1 O^{\prime} 2=1^{\prime} O^{\prime} 2\), будет, как видно из рисунка, больше.

\begin{figure}[h]
\begin{center}
  \includegraphics[alt={},max width=\textwidth]{950fefba-d83b-4c8f-96e7-0bd5a7eb90be-074_419_420_822_744}
\captionsetup{labelformat=empty}
\caption{Рис. 3.4}
\end{center}
\end{figure}

Здесь следует обратить внимание на то, что в разных средах (ситуациях) возможен не единственный путь луча, удовлетворяющий принципу Ферма, а несколько.\\
2. Получим с помощью принципа Ферма закон преломления света на границе раздела двух однородных прозрачных сред с показателями преломления \(n_{1}\) и \(n_{2}\).

Для этого найдем точку \(C\) (рис. 3.5), в которой должен преломиться луч, распространяясь от точки \(A\) к точке \(B\), чтобы оптическая длина пути \(L\) была экстремальной. Пусть отрезок \(A^{\prime} B^{\prime}=b\), тогда, как видно из рисунка,


\begin{equation*}
L=n_{1} s_{1}+n_{2} s_{2}=n_{1} \sqrt{a_{1}^{2}+x^{2}}+n_{2} \sqrt{a_{2}^{2}+(b-x)^{2}} . \tag{3.18}
\end{equation*}


\begin{figure}[h]
\begin{center}
  \includegraphics[alt={},max width=\textwidth]{950fefba-d83b-4c8f-96e7-0bd5a7eb90be-075_434_526_153_354}
\captionsetup{labelformat=empty}
\caption{Рис. 3.5}
\end{center}
\end{figure}

Продифференцируем это выражение по \(x\) и приравняем производную нулю (условие экстремума):


\begin{equation*}
\frac{d L}{d x}=\frac{n_{1} x}{\sqrt{a_{1}^{2}+x^{2}}}-\frac{n_{2}(b-x)}{\sqrt{a_{2}^{2}+(b-x)^{2}}}=n_{1} \frac{x}{s_{1}}-n_{2} \frac{b-x}{s_{2}}=0 . \tag{3.19}
\end{equation*}


Множители при \(n_{1}\) и \(n_{2}\) равны соответственно \(\sin \vartheta_{1}\) и \(\sin \vartheta_{2}\). Таким образом, получаем \(n_{1} \sin \vartheta_{1}=n_{2} \sin \vartheta_{2}\), что и должно быть согласно (3.15).

Преломление на сферической поверхности. Выясним, как формируется изображение предмета с помощью преломления световых лучей на сферической поверхности, отделяющей однородные среды с показателями преломления \(n\) и \(n^{\prime}\) (рис. 3.6, где \(n^{\prime}>n\) ). Пусть \(S\) - одна из точек предмета, а \(S^{\prime}\) - ее изображение. Эти точки сопряженные: поместив источник в точку \(S^{\prime}\), его изображение получим в точке \(S\).

Назовем прямую \(S S^{\prime}\) оптической осью - она проходит через точку \(S\) и центр кривизны \(C\) сферической поверхности радиуса \(R\). Четкое изображение в данном случае можно получить только

\begin{figure}[h]
\begin{center}
  \includegraphics[alt={},max width=\textwidth]{950fefba-d83b-4c8f-96e7-0bd5a7eb90be-075_342_816_1494_207}
\captionsetup{labelformat=empty}
\caption{Рис. 3.6}
\end{center}
\end{figure}

с помощью параксиальных лучей - лучей, составляющих малый угол с оптической осью (порядка нескольких градусов). Как мы увидим далее, только в этом случае оптические пути всех лучей, выходящих из точки \(S\) и приходящих в точку \(S^{\prime}\), оказываются одинаковыми. Значит, система в этой области является таутохронной, и все колебания приходят в точку \(S^{\prime}\) в фазе - необходимое условие для получения чёткого изображения.

Для общности получаемых соотношений примем следующее правило знаков:

\begin{enumerate}
  \item все расстояния будем отсчитывать от точки \(O\) - «вершины» сферической поверхности (см. рис. 3.6);
  \item расстояния, отсчитываемые от точки \(O\) против хода лучей, будем считать отрицательными (на рисунке \(s<0\) ), а по ходу лучей - положительными ( \(s^{\prime}>0, R>0\) ).
\end{enumerate}

Довольно кропотливый расчет (он приведен в приложении 2) приводит к следующей связи положений сопряженных точек \(S\) и \(S^{\prime}\), т. е. расстояний \(s\) и \(s^{\prime}\) :


\begin{equation*}
\frac{n^{\prime}}{s^{\prime}}-\frac{n}{s}=\frac{n^{\prime}-n}{R} . \tag{3.20}
\end{equation*}


Это соотношение является исходным (основным) при расчете всех оптических систем, содержащих много сферических поверхностей раздела (сложные объективы, микроскопы и др.). Величину в правой части (3.20) называют оптической силой Ф. Таким образом, для сферической поверхности


\begin{equation*}
\Phi=\frac{n^{\prime}-n}{R} . \tag{3.21}
\end{equation*}


Оптическую силу измеряют в Әuоптриях (дптр), 1 дптр \(=1 / \mathrm{m}\). Оптическая сила характеризует преломляющие свойства системы, причем Ф не зависит от направления хода лучей, поскольку при этом меняется знак как в числителе, так и в знаменателе (3.21).

Итак, с помощью (3.20) можно найти положение изображения \(S^{\prime}\), т. е. расстояние \(s^{\prime}\), если известны оптическая сила \(Φ\) и положение источника \(S\). В случае \(s^{\prime}>0\) изображение будет действительным, если же \(s^{\prime}<0\), то - мнимьлм, и оно будет нахо-

диться с той же стороны от сферической поверхности, что и источник \(S\). Соответственно систему называют собирающей ( ↓ ) или рассеивающей ( Ү ). Здесь в скобках указаны принятые обозначения их.

Формулу (3.20) можно записать, как мы увидим далее, в более общем виде. С учетом (3.21) она примет вид


\begin{equation*}
\frac{n^{\prime}}{s^{\prime}}-\frac{n}{s}=\Phi \tag{3.22}
\end{equation*}


При \(s \rightarrow \infty\) параксиальные лучи, падающие на сферическую поверхность, будут параллельны друг другу, и изображение окажется в точке, отстоящей от точки \(O\) на расстояние \(s^{\prime}=n^{\prime} / Ф\). Эту точку называют задним фокусом \(F^{\prime}\), а соответствующее расстояние от точки \(O\) - задним фокусным расстоянием \(f^{\prime}\) :


\begin{equation*}
f^{\prime}=n^{\prime} / \Phi . \tag{3.23}
\end{equation*}


Аналогично при \(s^{\prime} \rightarrow \infty\) мы получим \(s=-n /\) Ф. Это расстояние, на котором находится точка \(F\) - передний фокус, называется передним фокусным расстоянием \(f\) :


\begin{equation*}
f=-n / \Phi . \tag{3.24}
\end{equation*}


Из последних двух формул следует, что


\begin{equation*}
\frac{f^{\prime}}{f}=-\frac{n^{\prime}}{n}, \tag{3.25}
\end{equation*}


где знак минус означает, что фокусные расстояния \(f\) и \(f^{\prime}\), а значит и соответствующие фокусы \(F\) и \(F^{\prime}\), находятся по разные стороны от преломляющей поверхности.

Рассмотрим пример на построение изображения.\\
Пример. На рис. 3.7 показаны построения изображения в случаях, когда сферическая поверхность играет роль собирающей ( \(a\) ) и рассеивающей (б) системы. Построения выполнены с помощью «основных» лучей: 1) параллельного относительно оптической оси и 2) проходящего через передний фокус \(F\).\\
Следует обратить внимаине на ход луча, проходящего через точку \(O\) : он испытывает преломление из-за того, что \(n^{\prime} \neq n\) (забегая вперед, заметим, что это относится и к линзе).

\begin{figure}[h]
\begin{center}
  \includegraphics[alt={},max width=\textwidth]{950fefba-d83b-4c8f-96e7-0bd5a7eb90be-078_427_953_149_141}
\captionsetup{labelformat=empty}
\caption{Рис. 3.7}
\end{center}
\end{figure}

Кроме того, из различия фокусных расстояний можно сделать вывод о том, какой показатель преломления, \(n\) или \(n^{\prime}\), больше в данных случаях. А именно, согласно (3.25) в случае:\\
a) \(n^{\prime}>n\) (изображение действительное);\\
б) тоже \(n^{\prime}>n\) (изображение мнимое).

Отношение поперечных размеров изображения ( \(y^{\prime}\) ) и предмета \((y)\) называют линейным или поперечным увеличением \(\beta\) :


\begin{equation*}
\beta=y^{\prime} / y, \tag{3.26}
\end{equation*}


где \(\beta\) - величина алгебраическая: если изображение перевернутое (относительно предмета), то \(y^{\prime}<0\), значит и \(\beta<0\). Выясним с помощью рис. 3.8, как \(\beta\) зависит от расстояний \(s\) и \(s^{\prime}\) и

\begin{figure}[h]
\begin{center}
  \includegraphics[alt={},max width=\textwidth]{950fefba-d83b-4c8f-96e7-0bd5a7eb90be-078_313_754_1153_239}
\captionsetup{labelformat=empty}
\caption{Рис. 3.8}
\end{center}
\end{figure}

показателей преломления \(n\) и \(n^{\prime}\). Для параксиальных лучей, согласно закону преломления (3.15), \(n \vartheta=n^{\prime} \vartheta^{\prime}\) или \(n \frac{y}{-s}=n^{\prime} \frac{-y^{\prime}}{s^{\prime}}\). Тогда (3.26) можно представить так:


\begin{equation*}
\beta=\frac{y^{\prime}}{y}=\frac{s^{\prime} / n^{\prime}}{s / n} . \tag{3.27}
\end{equation*}


В заключение обратим внимание на то, что формулы (3.22)-(3.27), полученные для одной сферической поверхности, справедливы и для сложных оптических систем. Вопрос лишь в том, откуда отсчитывать расстояния \(s, s^{\prime}, f\) и \(f^{\prime}\). В общей теории оптических систем доказывается, что в каждой из них существуют так называемые главные плоскости: передняя \(H\) и задняя \(H^{\prime}\). Именно от этих плоскостей и отсчитываются все расстояния (с учетом того же правила знаков). Но в дальнейшем нас будет интересовать только тонкая линза, у которой главные плоскости совпадают друг с другом и проходят через ее центр. Рассмотрением свойств этой линзы мы и ограничимся.

Тонкая линза. Так называют линзу, толщина \(d\) которой в условиях рассматриваемой задачи пренебрежимо мала. Напомним ее свойства, необходимые для дальнейшего.

\begin{enumerate}
  \item Формула тонкой линзы в воздухе - ее легко получить, применяя (3.20) последовательно к передней и задней (по ходу лучей) поверхностям линзы, радиусы кривизны которых \(R_{1}\) и \(R_{2}\). В результате находим:
\end{enumerate}


\begin{equation*}
\frac{1}{s^{\prime}}-\frac{1}{s}=(n-1)\left(\frac{1}{R_{1}}-\frac{1}{R_{2}}\right), \tag{3.28}
\end{equation*}


где \(n\) - показатель преломления вещества линзы. Справа в этой формуле записана оптическая сила линзы \(\Phi=1 / f^{\prime}\). Причем для тонкой линзы \(\Phi=\Phi_{1}+\Phi_{2}\), где \(\Phi_{1}\) и \(\Phi_{2}\) - оптические силы передней и задней поверхностей линзы. Вывод формулы (3.28) приведен в решении задачи 3.9.

На рис. 3.9 показано построение изображения с помощью основных лучей в случае собирающей линзы. Заметим, что со-

\begin{figure}[h]
\begin{center}
  \includegraphics[alt={},max width=\textwidth]{950fefba-d83b-4c8f-96e7-0bd5a7eb90be-079_330_592_1423_68}
\captionsetup{labelformat=empty}
\caption{Рис. 3.9}
\end{center}
\end{figure}

гласно (3.25) передний и задний фокусы, \(F\) и \(F^{\prime}\), в нашем случае находятся на одинаковом расстоянии от центра \(C\) линзы. В связи этим луч, проходящий через \(C\), не испытывает преломления, остававясь прямым.

Ось \(O O^{\prime}\), проходящая через центры кривизны поверхностей линзы и точку \(C\), называют главной оптической осью линзы.

Правило знаков для отрезков \(s, s^{\prime}\) и радиусов кривизны \(R\) и \(R^{\prime}\) остается прежним. Все лучи, предполагается, параксиальные.

Заметим, что если бы линза находилась в среде, показатель преломления которой был бы разным, например, слева \(n\), а справа \(n^{\prime}\), то переднее и заднее фокусные расстояния были бы не одинаковы. Их отношение определялось бы формулой


\begin{equation*}
f^{\prime} / f=-n^{\prime} / n . \tag{3.29}
\end{equation*}


Знак минус показывает, что оба фокуса, передний \(F\) и задний \(F^{\prime}\), всегда находятся по разные стороны от линзы. Следует также иметь в виду, что при \(n^{\prime} \neq n\) центр линзы оказывается не оптическим центром: луч, проходящий через него, будет испытывать преломление.\\
2. При падении на линзу пучка параллельных лучей под углом к главной оптической оси (на которой находятся центры кривизны поверхностей линзы, а также фокусы \(F\) и \(F^{\prime}\) ) изображение - точка \(S^{\prime}\) - образуется в задней фокальной плоскости линзы (рис. 3.10). И главное: оптические пути всех лучей от плоскости \(T T^{\prime}\), перпендикулярной падающему пучку, до изображения \(S^{\prime}\) будут одинаковы в силу таутохронности. Это мы будем использовать в дальнейшем неоднократно.\\
3. При поступательном смещении параллельного пучка лучей положение точки их схождения \(S^{\prime}\) в фокальной плоскости не меняется. Т. е. при смещении, например, отверстия из положения 1 в положение 2 (рис. 3.11) изображение - точка \(S^{\prime}\) - будет оставаться на прежнем месте. Это мы будем учитывать при изучении действия дифракционной решетки (§ 5.7).

\begin{figure}[h]
\begin{center}
  \includegraphics[alt={},max width=\textwidth]{950fefba-d83b-4c8f-96e7-0bd5a7eb90be-080_301_492_1484_116}
\captionsetup{labelformat=empty}
\caption{Рис. 3.10}
\end{center}
\end{figure}

\begin{figure}[h]
\begin{center}
  \includegraphics[alt={},max width=\textwidth]{950fefba-d83b-4c8f-96e7-0bd5a7eb90be-080_273_407_1512_697}
\captionsetup{labelformat=empty}
\caption{Рис. 3.11}
\end{center}
\end{figure}

\begin{enumerate}
  \setcounter{enumi}{3}
  \item Линейное или поперечное увеличение линзы \(\beta=y^{\prime} / y\), где \(y^{\prime}\) и \(y\) - поперечные размеры изображения и самого предмета (см. рис. 3.9).
\end{enumerate}

Легко видеть, что

\[
\beta=s^{\prime} / s .
\]

На рис. 3.9 величины \(y^{\prime}\) и \(s\) отрицательные, следовательно и \(\beta<0\). Это означает, что изображение перевернутое (относительно предмета).

\section*{§ 3.4. Фотометрические величины}
Раздел оптики, посвященный измерению световых потоков и связанных с ним величин, называют фотометрией. Познакомимся с основными фотометрическими величинами и единицами их измерения.

Кривая видности. В видимом диапазоне длин волн \((0,40 \div \div 0,76\) мкм) действие световой энергии на глаз (световое ощущение) весьма сильно зависит от длины волны. Чувствительность среднего нормального человеческого глаза к световой энергии разной длины волны характеризуют кривой видности (более точное название - кривая относительной спектральной чувствительности). Ее график показан на рис. 3.12, где \(V\) - относительная спектральная чувствительность глаза. Наиболее чувствителен глаз к свету с длиной волны \(\lambda_{m}=0,555\) мкм \(=555\) нм (зеленая часть спектра). Для этой длины волны принято \(V=1\). При одинаковом потоке световой энергии оцениваемая зрительно интенсивность света других длин волн оказывается меньшей. Вне интервала видимых длин волн \(V=0\).

\begin{figure}[h]
\begin{center}
  \includegraphics[alt={},max width=\textwidth]{950fefba-d83b-4c8f-96e7-0bd5a7eb90be-081_424_705_1357_262}
\captionsetup{labelformat=empty}
\caption{Рис. 3.12}
\end{center}
\end{figure}

Пример. Значение, скажем, \(V\left(\lambda_{0}\right)=0,5\) означает, что для получения зрительного ощущения такой же интенсивности, как и при \(V\left(\lambda_{m}\right)=1\), плотность потока световой энергии с длиной волны \(\lambda_{0}\) должна быть вдвое больше, чем при \(\lambda_{m}\).\\
Световой поток. Для характеристики интенсивности света с учетом его способности вызывать зрительное ощущение вводят понятие светового потока Ф. Его определяют в люменах (лм) по формуле


\begin{equation*}
\Phi(л м)=K_{m} V(\lambda) \Phi_{Э}(\text { Вт }), \tag{3.30}
\end{equation*}


где \(K_{m} \approx 683\) лм/Вт, Фэ - поток энергии в ваттах с той же длиной волны \(\lambda\). Из этой формулы следует, что световому потоку в 1 лм с длиной волны \(\lambda=555\) нм ( \(V=1\) ) соответствует поток энергии в \(1,46 \mathrm{mB}\). И наоборот, например, при \(\lambda=0,60\) мкм одному ватту соответствует, согласно (3.30), световой поток \(\Phi=683 \cdot 0,6 \cdot 1=410\) лм.

Для интервала длин волн ( \(\lambda, \lambda+\mathrm{d} \lambda\) ) световой поток

\[
\mathrm{d} \Phi=K_{m} V(\lambda) \varphi(\lambda) \mathrm{d} \lambda,
\]

где \(\varphi(\lambda)=\mathrm{d} \Phi_{\ni} / \mathrm{d} \lambda\) - функция распределения световой энергии. Полный световой поток


\begin{equation*}
\Phi=K_{m} \int_{0}^{\infty} V(\lambda) \varphi(\lambda) \mathrm{d} \lambda . \tag{3.31}
\end{equation*}


Сила света I. По определению, \(I\) - это поток излучения точечного источника, приходящийся на единицу телесного угла:


\begin{equation*}
I=\frac{\mathrm{d} \Phi}{\mathrm{~d} \Omega} \tag{3.32}
\end{equation*}


В общем случае сила света зависит от направления, т. е. \(I(\vartheta, \varphi)\), где \(\vartheta\) и \(\varphi\) - полярный и азимутальный углы в сферической системе координат.

Если источник изотропный, то сила света \(I\) не зависит от направления, и значит


\begin{equation*}
I=\Phi / 4 \pi, \tag{3.33}
\end{equation*}


где Ф - полный световой поток источника.

Для протяженного источника можно говорить о силе света элемента его поверхности \(\mathrm{d} S\).

Единица силы света - кандела (кд), 1 кд \(=1\) лм/ср.\\
Освещенность \(\boldsymbol{E}\). Ее определяют как световой поток, падающий на единицу площади интересующей нас поверхности, то есть


\begin{equation*}
E=\frac{\mathrm{d} \Phi_{\text {пад }}}{\mathrm{d} S} . \tag{3.34}
\end{equation*}


Единица освещенности - люкс (лк):

\[
1 \text { лк }=1 \text { лм } / \mathrm{M}^{2} .
\]

\begin{figure}[h]
\begin{center}
  \includegraphics[alt={},max width=\textwidth]{950fefba-d83b-4c8f-96e7-0bd5a7eb90be-083_427_352_661_70}
\captionsetup{labelformat=empty}
\caption{Рис. 3.13}
\end{center}
\end{figure}

Например, необходимая для нормального чтения освещенность \(\sim 50\) лк.

Определение освещенности, создаваемой точечным источником, сводится к нахождению d \(\Phi_{\text {пад }}\) в (3.34). Согласно (3.32) \(\mathrm{d} \Phi_{\text {пад }}=I \mathrm{~d} \Omega\), где \(\mathrm{d} \Omega\), как видно из рис. 3.13 , равно \(d S \cos \alpha / r^{2}\). Тогда в соответствии с (3.34)


\begin{equation*}
E=I \frac{\cos \alpha}{r^{2}} . \tag{3.35}
\end{equation*}


Светимость М. Эта величина характеризует различные участки протяженного источника. Светимость - это световой поток, испускаемый (отражаемый) единицей площади наружу по всем направлениям (в пределах телесного угла \(2 \pi\) стерадиан):


\begin{equation*}
M=\frac{\mathrm{d} \Phi_{\text {исп }}}{\mathrm{d} S} . \tag{3.36}
\end{equation*}


Единица светимости - люмен на квадратный метр (лм/м2).

Яркость \(L\). Эта величина характеризует излучение (отражение) света элементом поверхности \(\Delta S\) в заданном направлении \((\vartheta, \varphi)\). Яркость определяют как отношение силы света \(\mathrm{d} I\) элемента поверхности \(\Delta S\) в заданном направлении к проекции \(\Delta S\)

на плоскость, перпендикулярную к этому направлению, \(\Delta S_{\perp}\) (рис. 3.14):


\begin{equation*}
L=\frac{I(\vartheta, \varphi)}{\Delta S_{\perp}}=\frac{\mathrm{d} \Phi_{\text {исп }} / \mathrm{d} \Omega}{\Delta S \cdot \cos \vartheta} . \tag{3.37}
\end{equation*}


Единица яркости - кандела на квадратный метр (кд/м²).

Вообще говоря, яркость различна для разных направлений, \(L(\vartheta, φ)\). Как и светимость, яркость используют и для отраженного от данной поверхности света.

Источники, яркость которых не зависит от направления, называют ламбертовскими (подчиняющимися закону Ламберта) или косинусными (световой поток, посылаемый элементом поверхности такого источника, \(d \Phi_{\text {исп }} \circlearrowleft \cos \vartheta\) ). Строго следует закону Ламберта только так называемое

\begin{figure}[h]
\begin{center}
  \includegraphics[alt={},max width=\textwidth]{950fefba-d83b-4c8f-96e7-0bd5a7eb90be-084_483_342_419_822}
\captionsetup{labelformat=empty}
\caption{Рис. 3.14}
\end{center}
\end{figure}

абсолютно черное тело.

Светимость \(M\) и яркость \(L\) ламбертовского источника связаны простым соотношением:


\begin{equation*}
M=\pi L . \tag{3.38}
\end{equation*}


Чтобы убедиться в этом, подставим в выражение

\[
\mathrm{d} \Phi_{\text {исп }}=L \mathrm{~d} \Omega \Delta S \cos \vartheta,
\]

которое следует из (3.37), \(\mathrm{d} \Omega=\mathrm{d} S / r^{2}\), где \(\mathrm{d} S=2 \pi r \sin \vartheta \cdot r \mathrm{~d} \vartheta\) (рис. 3.15), и проинтегрируем полученное соотношение по Э от 0 до \(\pi / 2\), учтя, что \(L\) = const. Тогда

\[
\Delta \Phi_{\text {исп }}=L \Delta S \int_{0}^{\pi / 2} \cos \vartheta \cdot 2 \pi \sin \vartheta \mathrm{~d} \vartheta=\pi L \Delta S .
\]

Разделив этот световой поток на \(\Delta S\), мы и получим (3.38).

\begin{figure}[h]
\begin{center}
  \includegraphics[alt={},max width=\textwidth]{950fefba-d83b-4c8f-96e7-0bd5a7eb90be-084_418_360_1319_800}
\captionsetup{labelformat=empty}
\caption{Рис. 3.15}
\end{center}
\end{figure}

Энергетические величины. Наряду с рассмотренными фотометрическими величинами - силой света \(I\), освещенностью \(E\), светимостью \(M\) и яркостью \(L\) - можно использовать аналогичные им энергетические величины, получаемые с помощью тех же соотношений с заменой лишь светового потока Ф на энергетический поток \(\Phi_{\text {э }}\). В таблице 3.1 представлены основные фотометрические и энергетические величины и единицы их измерения.

\begin{table}[h]
\begin{center}
\captionsetup{labelformat=empty}
\caption{Таблица 3.1}
\begin{tabular}{|l|l|l|}
\hline
\multirow{2}{*}{Величина} & \multicolumn{2}{|c|}{Единица измерения} \\
\hline
 & фотометрическая & энергетическая \\
\hline
Световой поток Ф & люмен (лм) & Вт \\
\hline
Сила света I & кандела (кд) & Вт/ср \\
\hline
Освещенность \(E\) & люкс (лк) & Вт/M \({ }^{2}\) \\
\hline
Светимость \(M\) & лм/M \({ }^{2}\) & Bt/m \({ }^{2}\) \\
\hline
Яркость \(L\) & кд/M \({ }^{2}\) & Вт/(ср \(\cdot \mathrm{m}^{2}\) ) \\
\hline
\end{tabular}
\end{center}
\end{table}

\section*{Задачи}
3.1. Отражение света. Показать, что при отражении от зеркала выполняется условие

\[
\mathbf{e}=\mathbf{e}_{0}-2\left(\mathbf{e}_{0} \mathbf{n}\right) \mathbf{n},
\]

где \(\mathbf{e}\) и \(\mathbf{e}_{0}\) - орты отраженного и падающего лучей, \(\mathbf{n}\) - орт нормали к плоскости зеркала.

Решение. Согласно рис. 3.16, можно записать:

\begin{figure}[h]
\begin{center}
  \includegraphics[alt={},max width=\textwidth]{950fefba-d83b-4c8f-96e7-0bd5a7eb90be-085_353_506_1433_67}
\captionsetup{labelformat=empty}
\caption{Рис. 3.16}
\end{center}
\end{figure}

\[
\mathbf{e}=\mathbf{e}_{0}+\Delta \mathbf{e} .
\]

Найдем вектор \(\Delta \mathbf{e}\). Его модуль, как видно из рисунка, \(|\Delta \mathbf{e}|=2\left|\mathbf{e}_{0}\right| \cos \alpha= =2 \cdot 1 \cdot(-\cos \beta)=-2\left(\mathbf{e}_{0} \mathbf{n}\right)\). И сам вектор

\[
\Delta \mathbf{e}=-2\left(\mathbf{e}_{0} \mathbf{n}\right) \mathbf{n},
\]

что и требовалось показать.\\
3.2. Воспользовавшись формулой из предыдущей задачи, показать, что луч света, последовательно отразившийся от трех взаимно перпендикулярных плоских зеркал, изменит свое направление на прямо противоположное. Это так называемый уголковый отражатель - идеальный объект для радаров.\\
Решение. Согласно формуле из предыдущей задачи можно записать


\begin{equation*}
\mathbf{e}_{1}=\mathbf{e}_{0}-2\left(\mathbf{e}_{0} \mathbf{n}_{1}\right) \mathbf{n}_{1}, \quad \mathbf{e}_{2}=\mathbf{e}_{1}-2\left(\mathbf{e}_{1} \mathbf{n}_{2}\right) \mathbf{n}_{2}, \quad \mathbf{e}_{3}=\mathbf{e}_{2}-2\left(\mathbf{e}_{2} \mathbf{n}_{3}\right) \mathbf{n}_{3} . \tag{1}
\end{equation*}


Сложив отдельно левые и правые части этих равенств, получим после сокращения:


\begin{equation*}
\mathbf{e}_{3}=\mathbf{e}_{0}-2\left[\left(\mathbf{e}_{0} \mathbf{n}_{1}\right) \mathbf{n}_{1}+\left(\mathbf{e}_{1} \mathbf{n}_{2}\right) \mathbf{n}_{2}+\left(\mathbf{e}_{2} \mathbf{n}_{3}\right) \mathbf{n}_{3}\right] . \tag{2}
\end{equation*}


В этом выражении ( \(\mathbf{e}_{1} \mathbf{n}_{2}\) ) = ( \(\mathbf{e}_{0} \mathbf{n}_{2}\) ) - это следует из (1), если первое равенство умножить скалярно на \(\mathbf{n}_{2}\). Поступая аналогично, найдем: \(\left(\mathbf{e}_{2} \mathbf{n}_{3}\right)=\left(\mathbf{e}_{1} \mathbf{n}_{3}\right)=\left(\mathbf{e}_{0} \mathbf{n}_{3}\right)\). После этого (2) можно переписать так:


\begin{equation*}
\mathbf{e}_{3}=\mathbf{e}_{0}-2\left[\left(\mathbf{e}_{0} \mathbf{n}_{1}\right) \mathbf{n}_{1}+\left(\mathbf{e}_{0} \mathbf{n}_{2}\right) \mathbf{n}_{2}+\left(\mathbf{e}_{0} \mathbf{n}_{3}\right) \mathbf{n}_{3}\right] . \tag{3}
\end{equation*}


Величины в круглых скобках - это направляющие косинусы орта \(\mathbf{e}_{0}\) относительно координат с ортами \(\mathbf{n}_{1}, \mathbf{n}_{2}, \mathbf{n}_{3}\). Значит, величина в квадратных скобках есть просто \(\mathbf{e}_{0}\) и \(\mathbf{e}_{3}=-\mathbf{e}_{0}\), что и требовалось доказать.\\
3.3. Свет интенсивности \(I_{0}\) падает нормально на идеально прозрачную пластинку. Считая, что коэффициент отражения каждой поверхности ее \(\rho=0,05\), найти интенсивность \(I\) прошедшего через пластинку света с учетом:\\
a) только однократных отражений;\\
б) многократных отражений.

Решени е. а) После первой поверхности интенсивность будет равна \(I_{0}(1-\rho)\), после второй поверхности

\[
I=I_{0}(1-\rho)^{2} \approx I_{0}(1-2 \rho)=0,9 I_{0} .
\]

б) После прохождения двух поверхностей интенсивность равна \(I_{0}(1-\rho)^{2}\). При этом часть \(I_{0}(1-\rho) \rho\) отражается от второй поверхности и затем, отразившись от первой поверхности, станет равной \(I_{0}(1-\rho) \rho^{2}\). И световая волна при прохождении второй поверхности будет иметь интенсивность \(I_{0}(1-\rho)^{2} \rho^{2}\) - это после двукратного отражения. Следующая волна испытает четырехкратное отра-

жение и выйдет из пластинки с интенсивностью \(I_{0}(1-\rho)^{2} \rho^{4}\). Затем надо учесть вклады от шести-, восьми- и т. д. кратных отражений.

В сумме интенсивность проходящего света можно представить как

\[
I=I_{0}(1-\rho)^{2}\left(1+\rho^{2}+\rho^{4}+\ldots\right)=I_{0}(1-\rho) /(1+\rho) \approx 0,9 I_{0},
\]

где учтено, что сумма геометрической прогрессии \(1+\rho^{2}+\rho^{4}+\ldots= =1 /\left(1-\rho^{2}\right)\).

Полученный результат практически совпадает с найденным в предыдущем пункте (отличие составляет всего \(\Delta I / I \approx \rho^{2}=0,25 \%\) ). Поэтому, как правило, многократными отражениями пренебрегают.\\
3.4. Преломление света. При каком значении угла падения \(\vartheta_{1}\) луч, отраженный от поверхности прозрачного диэлектрика, будет перпендикулярен преломленному лучу, если показатель преломления диэлектрика равен \(n\) ?\\
Решение. В этом случае должны быть выполнены два условия:

\[
\sin \vartheta_{1}=n \sin \vartheta_{2}, \quad \vartheta_{1}^{\prime}+\vartheta_{2}=\pi / 2,
\]

где \(\vartheta_{1}^{\prime}\) - угол отражения, равный углу \(\vartheta_{1}\). Поэтому первое равенство запишем так:

\[
\sin \vartheta_{1}=n \sin \left(\pi / 2-\vartheta_{1}\right)=n \cos \vartheta_{1} .
\]

Отсюда следует, что \(\operatorname{tg} \vartheta_{1}=n\).\\
3.5. Пучок параллельных лучей интенсивности \(I_{0}\) падает на поверхность прозрачного диэлектрика с показателем преломления \(n\). Угол падения \(\vartheta_{1}\) таков, что \(\operatorname{tg} \vartheta_{1}=n\). При этом \(\rho\)-часть светового потока отражается. Найти интенсивность преломленного пучка.

Решение. Пусть пучок параллельных лучей падает на поверхность диэлектрика площади \(S\) (рис. 3.17). Исходим из того, что для потоков должно выполняться равенство

\[
\Phi_{0}=\Phi^{\prime}+\Phi^{\prime \prime}
\]

(именно для потоков, а не для интенсивностей, поскольку сечения падающего и преломленного пучков при наклонном падении не одинаковы).

Теперь представим с помощью рис. 3.17 потоки как произведения интенсивностей на сечения пучков. Тогда предыдущее равенство примет вид:

\begin{figure}[h]
\begin{center}
  \includegraphics[alt={},max width=\textwidth]{950fefba-d83b-4c8f-96e7-0bd5a7eb90be-088_352_493_150_672}
\captionsetup{labelformat=empty}
\caption{Рис. 3.17}
\end{center}
\end{figure}

\[
I_{0} S \cos \vartheta_{1}=\rho I_{0} S \cos \vartheta_{1}+I^{\prime \prime} S \cos \vartheta_{2}
\]

Отсюда \(I^{\prime \prime}=(1-\rho) I_{0} \cos \vartheta_{1} / \cos \vartheta_{2}\). Из условия \(\operatorname{tg} \vartheta_{1}=n\) имеем \(\sin \vartheta_{1}=n \cos \vartheta_{1}\). Сопоставив последнее равенство с выражением для закона преломления (3.15), видим, что \(\cos \vartheta_{1}=\sin \vartheta_{2}\). Отсюда следует, что \(\cos \vartheta_{2}=\sin \vartheta_{1}\). И выражение для искомой интенсивности \(I^{\prime \prime}\) примет вид:

\[
I^{\prime \prime}=(1-\rho) I_{0} / n
\]

3.6. Показать, что при преломлении в призме с малым преломляющим углом \(\theta\) луч отклоняется от своего первоначального направления на угол \(\alpha=(n-1) \theta\) независимо от угла падения, если он также мал.

Решение. Вследствие малости углов падения (и преломления) синусы в законе Снелла (3.15) можно опустить. Тогда условия преломления на передней и задней поверхностях призмы (рис. 3.18) примут вид:


\begin{gather*}
\vartheta_{1}=n \vartheta_{1}^{\prime}, \quad n \vartheta_{2}=\vartheta_{2}^{\prime} \\
\vartheta_{1}^{\prime}+\vartheta_{2}=\theta \tag{1}
\end{gather*}


Искомый угол \(\alpha\), как видно из рисунка, равен


\begin{equation*}
\alpha=\left(\vartheta_{1}-\vartheta_{1}^{\prime}\right)+\left(\vartheta_{2}^{\prime}-\vartheta_{2}\right) . \tag{2}
\end{equation*}


\begin{figure}[h]
\begin{center}
  \includegraphics[alt={},max width=\textwidth]{950fefba-d83b-4c8f-96e7-0bd5a7eb90be-088_502_521_1115_646}
\captionsetup{labelformat=empty}
\caption{Рис. 3.18}
\end{center}
\end{figure}

После подстановки углов из (1)\\
в (2) получим \(\alpha=(n-1) \theta\).\\
3.7. Перед выпуклой поверхностью толстой стеклянной выпукло-плоской линзы толщины \(d\) находится предмет. Его изображение образуется на плоской поверхности линзы, которая служит экраном. Показатель преломления стекла \(n\).

Определить:\\
a) поперечное увеличение \(\beta\), если радиус кривизны выпуклой поверхности линзы равен \(R\);\\
б) освещенность изображения, если яркость предмета \(L\) и диаметр входного отверстия выпуклой поверхности линзы равен \(D\) (потерями света пренебречь).

P e ш e н и е. а) Согласно формуле (3.27)


\begin{equation*}
\beta=\frac{s^{\prime} / n^{\prime}}{s / n}=\frac{d / n}{s} . \tag{1}
\end{equation*}


Задача сводится к нахождению \(s\) - расстояния предмета от выпуклой поверхности линзы. Для этого используем соотношение (3.20). В нашем случае (рис. 3.19) оно примет вид:


\begin{equation*}
\frac{n}{d}-\frac{1}{s}=\frac{n-1}{R} . \tag{2}
\end{equation*}


\begin{figure}[h]
\begin{center}
  \includegraphics[alt={},max width=\textwidth]{950fefba-d83b-4c8f-96e7-0bd5a7eb90be-089_298_639_1015_301}
\captionsetup{labelformat=empty}
\caption{Рис. 3.19}
\end{center}
\end{figure}

Отсюда найдем \(1 / s\) и полученное выражение подставим в (1). В результате получим


\begin{equation*}
\beta=1-\frac{d(n-1)}{n R} . \tag{3}
\end{equation*}


б) Освещенность изображения \(E=\Phi / \Delta S^{\prime}\), где \(\Phi=L \Delta S \Delta \Omega\) - это следует из формулы (3.37), где учтено, что угол \(\vartheta\) мал и \(\cos \vartheta \approx 1\); \(\Delta S^{\prime}\) и \(\Delta S\) - площади изображения и предмета. Таким образом,


\begin{equation*}
E=\frac{L \Delta S \Delta \Omega}{\Delta S^{\prime}}=L\left(\frac{1}{\beta}\right)^{2} \frac{\pi D^{2}}{4 s^{2}} . \tag{4}
\end{equation*}


Заменив увеличение \(\beta\) выражением (1), получим после сокращения на \(s^{2}\)

\[
E=\frac{\pi n^{2} L D^{2}}{4 d^{2}}
\]

3.8. Найти светимость \(M\) поверхности, яркость которой зависит от направления как \(L=L_{0} \cos \vartheta\), где \(\vartheta\) - угол между направлением излучения и нормалью к поверхности.\\
Решение. Исходимиз формулы (3.36), где согласно (3.37)

\[
\mathrm{d} \Phi_{\text {исп }}=L \Delta S \cos \vartheta \mathrm{~d} \Omega .
\]

Учитывая, что \(L=L_{0} \cos \vartheta\) и \(\mathrm{d} \Omega=2 \pi \sin \vartheta \mathrm{~d} \vartheta\), запишем выражение для полного испускаемого потока:

\[
\begin{aligned}
\Delta \Phi_{\text {исп }} & =\int L_{0} \cos \vartheta \cdot \Delta S \cos \vartheta \cdot 2 \pi \sin \vartheta \mathrm{~d} \vartheta= \\
& =2 \pi L_{0} \Delta S \int_{\vartheta=0}^{\pi / 2} \cos ^{2} \vartheta \mathrm{~d}(-\cos \vartheta)=\frac{2}{3} \pi L_{0} \Delta S .
\end{aligned}
\]

Разделив \(\Delta \Phi_{\text {исп }}\) на \(\Delta S\), получим

\[
M=\frac{2}{3} \pi L_{0}
\]

3.9. Тонкая линза. Найти с помощью формулы преломления на сферической границе раздела (3.20) формулу тонкой линзы в воздухе. Решение. Применим указанную формулу последовательно для передней и задней поверхностей линзы. Для передней поверхности (рис. 3.20), полагая, что за ней всюду среда с показателем преломления \(n\), запишем:

\begin{figure}[h]
\begin{center}
  \includegraphics[alt={},max width=\textwidth]{950fefba-d83b-4c8f-96e7-0bd5a7eb90be-090_246_546_1190_616}
\captionsetup{labelformat=empty}
\caption{Рис. 3.20}
\end{center}
\end{figure}

\[
\frac{n}{d+s_{0}}-\frac{1}{s}=\frac{n-1}{R_{1}} .
\]

Для задней поверхности линзы источник \(S_{0}\) находится справа (он мнимый), и мы имеем:

\[
\frac{1}{s^{\prime}}-\frac{n}{s_{0}}=\frac{1-n}{R_{2}}
\]

где \(s_{0}>0\), а \(R_{2}<0\) (в соответствии с правилом знаков).

Сложив отдельно левые и правые части этих равенств с учетом того, что толщина линзы \(d\) пренебрежимо мала, получим:

\[
\frac{1}{s^{\prime}}-\frac{1}{s}=(n-1)\left(\frac{1}{R_{1}}-\frac{1}{R_{2}}\right) .
\]

3.10. Найти построением продолжение хода луча за собирающей и рассеивающей тонкими линзами (рис. 3.21, где \(O O^{\prime}\) - главная оптическая ось, \(F\) и \(F^{\prime}\) - передний и задний фокусы линзы). Обратить внимание на то, что фокусные расстояния, переднее и заднее, разные.

\begin{figure}[h]
\begin{center}
  \includegraphics[alt={},max width=\textwidth]{950fefba-d83b-4c8f-96e7-0bd5a7eb90be-091_291_1012_606_153}
\captionsetup{labelformat=empty}
\caption{Рис. 3.21}
\end{center}
\end{figure}

Решение. а) Прежде всего отметим, что переднее фокусное расстояние больше заднего, а это значит, согласно (3.29), что показатель преломления среды слева от линзы больше, чем справа, \(n>n^{\prime}\). Ход построения: вспомогательные лучи изобразим штриховыми линиями и последовательно перенумеруем (рис. 3.22, \(a\) ), - это позволяет легко восстанавливать последовательность самого построения. Следует обратить внимание на то, что луч, падающий параллельно заданному лучу и проходящий, не преломляясь, через линзу и изображение \(S^{\prime}\), определяет положение замечательной точки \(K\) (ее называют узловой), которая играет

\begin{figure}[h]
\begin{center}
  \includegraphics[alt={},max width=\textwidth]{950fefba-d83b-4c8f-96e7-0bd5a7eb90be-091_361_981_1429_126}
\captionsetup{labelformat=empty}
\caption{Рис. 3.22}
\end{center}
\end{figure}

роль оптического центра линзы: любой луч, проходящий через точку \(K\), не преломляется линзой, например, луч \(K K^{\prime}\). Луч же, проходящий через центр линзы (здесь он не является оптическим центром), преломляется, что легко проверить с помощью построения. Для точечного источника, помещенного в точку \(K\), линзы как бы нет.\\
б) Здесь, согласно (3.29), справа от линзы показатель преломления среды больше, \(n^{\prime}>n\). Ход построения показан на рис. 3.22, б. Вспомогательный луч 1 , параллельный падающему и проходящий через передний фокус \(F\) (он справа от линзы), должен за линзой стать параллельным оси \(O O^{\prime}\), это луч 2. Его точка пересечения с задней фокальной плоскостью 3 (она слева от линзы) определяет положение мнимого изображения - точки \(S^{\prime}\). Продолжение исходного луча за линзой должно проходить через точку \(S^{\prime}\) - луч 4 . Положение узловой точки \(K\) найдем, проведя через точку \(S^{\prime}\) прямую, параллельную исходному лучу. Точка пересечения этой прямой с оптической осью \(O O^{\prime}\) и есть точка \(K\). Любой луч, проходящий через эту точку, линзой не преломляется (например, луч \(K K^{\prime}\) ).

\section*{Интерференция света}
\section*{§ 4.1. Интерференция световых волн}
Интерференция это одно из явлений, где проявляются волновые свойства света.

Когерентность. Рассмотрим суперпозицию двух гармониче-

\begin{figure}[h]
\begin{center}
  \includegraphics[alt={},max width=\textwidth]{950fefba-d83b-4c8f-96e7-0bd5a7eb90be-093_227_326_689_90}
\captionsetup{labelformat=empty}
\caption{Рис. 4.1}
\end{center}
\end{figure}

ских волн одинаковой частоты, которые возбуждают в интересующей нас точке пространства колебания одинакового направления с амплитудами \(A_{1}\) и \(A_{2}\). Если разность фаз этих колебаний равна \(\delta\), то возникает результирующее колебание с амплитудой \(A\), которую легко найти с помощью векторной (или фазовой) диаграммы (рис. 4.1) и теоремы косинусов:


\begin{equation*}
A^{2}=A_{1}^{2}+A_{2}^{2}+2 A_{1} A_{2} \cos \delta . \tag{4.1}
\end{equation*}


Если оба колебания не согласованы друг с другом, т. е. разность фаз \(\delta\) как-то изменяется во времени, то такие колебания называют некогерентными. В том случае, когда \(\delta\) непрерывно изменяется, причем так, что принимает с равной вероятностью любые значения, среднее по времени значение \(\langle\cos \delta\rangle=0\), последнее слагаемое в (4.1) обращается в нуль и остается \(A^{2}=A_{1}^{2}+A_{2}^{2}\). Принимая во внимание, что интенсивность \(I\) пропорциональна квадрату амплитуды, \(I \sim A^{2}\), можно записать


\begin{equation*}
I=I_{1}+I_{2} . \tag{4.2}
\end{equation*}


Это значит, что в данном случае интенсивность результирующего колебания равна сумме интенсивностей, создаваемых каждой из волн в отдельности.

Если же разность фаз \(\delta\) постоянна во времени, то такие колебания (и волны) называют когерентными. В случае суперпози-

ции когерентных волн интенсивность результирующего колебания, согласно (4.1),


\begin{equation*}
I=I_{1}+I_{2}+2 \sqrt{I_{1} I_{2}} \cos \delta \tag{4.3}
\end{equation*}


Последнее слагаемое в этой формуле и в (4.1) называют интерферениионным членом. Рассмотрим его влияние на результирующую интенсивность.

В точках пространства, где \(\cos \delta>0, I>I_{1}+I_{2}\); там же, где \(\cos \delta<0, I<I_{1}+I_{2}\). Другими словами, при суперпозиции когерентных волн происходит перераспределение интенсивности \(I\) в пространстве: в одних местах возникают максимумы, в других минимумы интенсивности. Это явление называют интерференцией волн. Особенно отчетливо (контрастно) интерференция проявляется тогда, когда \(I_{1}=I_{2}\). Тогда, согласно (4.3), \(I=4 I_{1}\) в максимумах и \(I=0\) в минимумах. Для некогерентных волн при \(I_{1}= =I_{2}\) интенсивность \(I\) всюду одинакова и, согласно (4.2), \(I=2 I_{1}\).

Основной принцип интерференционных схем. Интерференция характерна для волн любой природы и сравнительно просто наблюдается на опыте для волн на поверхности воды или для звуковых волн. Наблюдать же интерференцию световых волн можно лишь при определенных условиях.

Дело в том, что свет, испущенный обычными (не лазерными) источниками, не бывает монохроматическим. Такой свет можно рассматривать как хаотичную последовательность отдельных цугов синусоидальных волн. Длительность отдельного цуга порядка \(10^{-8}\) с, поэтому при наложении световых волн от разных источников фазовые соотношения между световыми колебаниями многократно изменяются случайным образом. Источники оказываются некогерентными и достаточно устойчивой картины интерференции не возникает (сменяющие друг друга с весьма большой частотой картины интерференции в дальнейшем нас интересовать не будут, их регистрация требует специальных малоинерционных приемников).

И тем не менее, когерентные световые волны можно получить даже от обычных источников. Общий принцип их получения таков: волну, излучаемую одним источником света, разделяют тем или иньлм способом на две части и затем накладьвают их друг на друга подходящим способом.

Если разность хода этих волн от источника до точки наблюдения не превышает некоторой характерной длины*, то случайные изменения амплитуды и фазы световых колебаний в двух волнах происходят согласованно (когерентно), и мы будем наблюдать интерференционную картину, например систему чередующихся светлых и темных полос.

Как будет видно в дальнейшем, образовавшиеся после разделения во́лны во всех интерференционных схемах можно представить как бы исходящими из двух точечных источников \(S_{1}\) и \(S_{2}\) (действительных или мнимых - это не существенно). Поэтому общий подход к интерпретации получаемых результатов будет единым, с него мы и начнем.

\begin{figure}[h]
\begin{center}
  \includegraphics[alt={},max width=\textwidth]{950fefba-d83b-4c8f-96e7-0bd5a7eb90be-095_447_828_673_201}
\captionsetup{labelformat=empty}
\caption{Рис. 4.2}
\end{center}
\end{figure}

Рассмотрим две волны, исходящие из когерентных источников \(S_{1}\) и \(S_{2}\) (рис. 4.2). В области, где эти волны перекрываются - ее называют зоной интерферениии - должна возникать система чередующихся максимумов и минимумов освещенности, которую можно наблюдать на экране Э.

Обозначим разность расстояний \(r_{2}\) и \(r_{1}\) от источников до интересующей нас точки \(P\) как \(\Delta=r_{2}-r_{1}\). Эту величину называют разностью хода. Если разность хода равна целому числу длин волн, т. е.


\begin{equation*}
\Delta=m \lambda, \quad m=0, \pm 1, \pm 2, \ldots, \tag{4.4}
\end{equation*}


\footnotetext{\begin{itemize}
  \item Ее называют длиной когерентности, но об этом более подробно в следующем параграфе.
\end{itemize}
}где \(m\) - порядок интерферениии, то колебания, возбуждаемые в точке \(P\) обеими волнами, будут происходить в фазе. Таким образом, (4.4) есть условие возникновения интерференционных максимумов. В точках же, для которых \(\Delta\) равно полуцелому числу длин волн, образуются минимумы.

Оптическая разность хода. Рассмотрение многих вопросов значительно упрощается, если вместо пути луча \(s\) и длины волны \(\lambda^{\prime}\) в данной среде использовать понятие оптической длины пути \(L\) и длины волны \(\lambda\) в вакууме. Для монохроматической волны на пути \(\lambda^{\prime}\) возникает отставание по фазе на \(2 \pi\), а на пути \(s\) - на \(\delta\). Отсюда следует, что

\[
\delta=2 \pi \frac{s}{\lambda^{\prime}}=2 \pi \frac{L}{\lambda},
\]

где учтено, что \(s n=L\) и \(\lambda^{\prime} n=\lambda\), согласно (3.4).\\
Это относится и к случаям, когда речь идет о связи разности фаз \(\delta\) с оптической разностью хода \(\Delta\). И в этих случаях


\begin{equation*}
\delta=2 \pi \frac{\Delta}{\lambda} \tag{4.4'}
\end{equation*}


где, повторим, \(\lambda\) - длина волны в вакууме.\\
Таким образом, если волны от источников (см. рис. 4.2) распространяются не в вакууме, а в среде с показателем преломления \(n\), то в формуле (4.4) под \(\Delta\) следует понимать не геометрическую, а оптическую разность хода интерферирующих волн: \(\Delta=n\left(r_{2}-r_{1}\right)\). При этом \(\lambda\) - это по-прежнему длина волны в вакууме.

Ширина интерференционной полосы. В практически важных случаях угол \(\theta \ll 1\) (см. рис. 4.2) и разность хода \(\Delta\) можно записать как \(\Delta=d \cdot \theta\), где \(d\) - расстояние между источниками \(S_{1}\) и \(S_{2}\). А так как \(\theta \approx x / l\), где \(l\) - расстояние от источников до экрана, то для максимумов, согласно (4.4), получим \(d \cdot x_{m} / l=m \lambda\), откуда


\begin{equation*}
x_{m}=m \lambda l / d . \tag{4.5}
\end{equation*}


В точке \(x=0\) расположен максимум, соответствующий нулевой разности хода. Для него порядок интерференции \(m=0\). Это центр интерференционной картины.

При переходе к соседнему максимуму \(m\) меняется на единицу и \(x\) - на величину \(\Delta x\), которую называют шириной интерференционной полось. Таким образом,


\begin{equation*}
\Delta x=\lambda l / d \quad \text { или } \quad \Delta x=\lambda / \psi, \tag{4.6}
\end{equation*}


где \(ψ\) - угол, под которым видны оба источника из центра экрана, \(\psi=d / l\) (см. рис. 4.2).

Из этих формул видно, что для увеличения ширины полосы следует увеличивать \(l\), или уменьшать \(d\), или то и другое, т. е. в конечном счете - уменьшать угловое расстояние \(ψ\) между источниками. Полезно иметь в виду, что размер интерференционной картины обычно не превышает 1 mm , это при расстоянии от источников до экрана порядка нескольких десятков сантиметров.

Практически для получения более яркой интерференционной картины в качестве источников \(S_{1}\) и \(S_{2}\) используют две щели (или изображения исходного источника - щели \(S\) ), и интерференционная картина имеет вид чередующихся светлых и темных полос, параллельных данным щелям.

Распределение интенсивности. Рассмотрим идеализированный случай, когда источники \(S_{1}\) и \(S_{2}\) строго монохроматические. В интересующую нас точку экрана колебания от этих источников будут приходить практически с одинаковой амплитудой, \(A_{1}=A_{2}=A_{0}\). Тогда, согласно (4.1),


\begin{equation*}
A^{2}=2 A_{0}^{2}+2 A_{0}^{2} \cos \delta=2 A_{0}^{2}(1+\cos \delta)=4 A_{0}^{2} \cos ^{2}(\delta / 2) \tag{4.7}
\end{equation*}


где \(\delta\) - разность фаз, которая зависит от разности хода как \(\delta=2 \pi \Delta / \lambda\). В нашем случае (см. рис. 4.2) \(\Delta=d \cdot \theta=d x / l\). Следовательно, \(\delta=2 \pi d x / l \lambda\). Имея в виду, что интенсивность \(I \sim A^{2}\), получим


\begin{equation*}
I=I_{0} \cos ^{2} \eta x \tag{4.8}
\end{equation*}


где \(\eta=\pi d / l \lambda, I_{0}\) — интенсивность в максимумах, в минимумах \(I=0\). Полученное идеализированное распределение интенсивности \(I(x)\) несколько отличается, естественно, от реального, которому соответствует рис. 4.2.

\section*{§ 4.2. Когерентность}
В предыдущем параграфе была описана принципиальная схема для наблюдения интерференции света от обычных (не лазерных) источников. Оказывается, однако, что идея, лежащая в основе таких схем, выражает только необходимые условия, но не достаточные. Для получения интерференционной картины необходимо еще, чтобы были удовлетворены некоторые условия, связанные с особыми свойствами световых волн. Выясним, в чем они заключаются.

Напомним, когерентностью называют согласованное протекание колебательных (волновых) процессов. Степень согласованности называют степенью когерентности: чем лучше согласованность, тем выше степень когерентности.

Различают длину и ширину когерентности*. В чем их существо и различие мы покажем на примере первой экспериментальной установки для демонстрации интерференции, предложенном Юнгом (опыт Юнга). В ней яркий пучок солнечного света освещал узкую щель \(S\) (рис. 4.3). Прошедший через щель свет вследствие дифракции образует расходящуюся волну, которая падает на две узкие щели \(S_{1}\) и \(S_{2}\). Эти щели действуют как вторичные когерентные источники, и исходящие из них дифрагированные волны, перекрываясь, дают на экране Э систему интерференционных полос.

\begin{figure}[h]
\begin{center}
  \includegraphics[alt={},max width=\textwidth]{950fefba-d83b-4c8f-96e7-0bd5a7eb90be-098_406_764_1196_233}
\captionsetup{labelformat=empty}
\caption{Рис. 4.3}
\end{center}
\end{figure}

\footnotetext{\begin{itemize}
  \item Эти характеристики связаны с так называемыми временной и пространственной когерентностями.
\end{itemize}
}Для получения устойчивой во времени интерференционной картины необходимо, чтобы геометрия установки удовлетворяла определенным условиям, связанным со свойствами используемого излучения, а именно с его длиной и шириной когерентности. Или наоборот, для данной геометрии установки обеспечивают определенные значения этих характеристик используемой световой волны. Тем самым достаточность условий будет обеспечена, и устойчивую интерференционную картину можно наблюдать.

Рассмотрим подробнее, что представляют собой длина и ширина когерентности.

Длина когерентности. В опыте Юнга интерференционная картина по мере удаления от ее середины размывается: несколько полос видны, но далее постепенно они исчезают. Почему?

Ответ ясен: потому, что степень когерентности складываемых в этих точках экрана колебаний (волн) постепенно уменьшается, и колебания становятся наконец полностью некогерентными.

Исходя из этого факта, попытаемся объяснить наблюдаемое с помощью следующей наглядной модели. Пусть мы видим, например, первые четыре порядка интерференции ( \(m=4\) ), а затем полосы исчезают (этот переход наблюдается довольно плавным, но мы не будем останавливаться на деталях). Исчезновение полос с \(m>4\) означает, что колебания, пришедшие в соответствующие точки экрана от обеих волн, оказываются уже некогерентными между собой. Т. е. пока их разность хода не превышает \(m=4\) длин волн, колебания в какой-то степени когерентны. Значит, вдоль распространения волны когерентными между собой будут только участки волны в этом интервале (длины). Данный интервал и называют длиной когерентности \(l_{\text {ког }}\). В рассмотренном случае \(l_{\text {ког }}=4 \lambda\). Заметим, что в данных условиях это простейший способ оценки длины когерентности: \(l_{\text {ког }} \approx m \lambda\), где \(m\) - максимальный порядок интерференции, соответствующий еще видимой светлой полосе.

Все это можно схематически представить с помощью рис. 4.4: в падающей на обе щели волне (рис. 4.3) длина когерентности \(l_{\text {ког }}\), щели создают две волны с той же длиной когерентности, но поскольку они достигают разных точек экрана с различными разностями хода, то участки когерентности обеих волн постепенно

\begin{figure}[h]
\begin{center}
  \includegraphics[alt={},max width=\textwidth]{950fefba-d83b-4c8f-96e7-0bd5a7eb90be-100_307_733_153_250}
\captionsetup{labelformat=empty}
\caption{Рис. 4.4}
\end{center}
\end{figure}

сдвигаются относительно друг друга и, начиная с \(m=5\), перестают перекрывать друг друга - складываемые колебания становятся некогерентными и интерференционные полосы исчезают.

Все сказанное, как мы увидим далее, справедливо при условии, что «первичная» щель \(S\) достаточно узка. При расширении этой щели вступает в действие другой эффект.

Найдем выражение, определяющее \(l_{\text {ког }}\). Известно, что строго монохроматический свет - это идеализация. Реальный свет, как бы ни стараться его монохроматизировать, остается в той или иной степени немонохроматическим, представляющим собой набор монохроматических компонент в некотором конечном интервале длин волн ( \(\lambda, \lambda+\Delta \lambda\) ). Примем, что эти монохроматические компоненты равномерно заполняют указанный интервал.

Как показывают формулы (4.6), ширина полос \(\Delta x\) ა \(\lambda\). Изобразим положения максимумов для длин волн, соответствующих крайним значениям спектрального интервала ( \(\lambda, \lambda+\Delta \lambda\) ): сплошными отрезками - для \(\lambda\), пунктирными - для \(\lambda+\Delta \lambda\) (рис. 4.5). Максимумы же от промежуточных длин волн заполняют интервал между крайними максимумами каждого порядка интерференции. В результате промежуточные максимумы, как видно из рисунка, будут постепенно заполнять интервал

\begin{figure}[h]
\begin{center}
  \includegraphics[alt={},max width=\textwidth]{950fefba-d83b-4c8f-96e7-0bd5a7eb90be-100_289_538_1478_346}
\captionsetup{labelformat=empty}
\caption{Рис. 4.5}
\end{center}
\end{figure}

между максимумами соседних порядков для \(\lambda\) и \(\lambda+\Delta \lambda\). А это значит, что результирующие максимумы (нижняя часть рисунка) будут постепенно размываться, и полосы интерференции исчезнут.

С помощью рис. 4.5 можно заключить, что полосы исчезнут там, где \(m(\lambda+\Delta \lambda) \approx(m+1) \lambda\), здесь \(m\) - предельный порядок интерференции, начиная с которого полосы исчезают. Отсюда


\begin{equation*}
m \approx \lambda / \Delta \lambda . \tag{4.9}
\end{equation*}


Величина \(\lambda / \Delta \lambda\) характеризует степень монохроматичности света: чем она больше, тем больше и степень монохроматичности.

Таким образом, мы нашли то значение \(m\), при котором картина интерференции исчезает, т. е. складываемые колебания становятся уже некогерентными. Заметим, что установить точное значение этого \(m\) довольно затруднительно из-за того, что полосы размываются постепенно, впрочем это и не так существенно.

Найденное значение \(m(4.9)\) связано с длиной когерентности как \(l_{\text {ког }} \approx m \lambda\). Отсюда следует, что


\begin{equation*}
l_{\text {ког }} \approx \lambda^{2} / \Delta \lambda . \tag{4.10}
\end{equation*}


Мы видим, что длина когерентности световой волны непосредственно связана со степенью монохроматичности ( \(\lambda / \Delta \lambda\) ): чем больше последняя, тем больше и длина когерентности. Для солнечного света \(l_{\text {ког }} \approx 5 \lambda\), для лучших (не лазерных) источников света удалось получить \(l_{\text {ког }}\) порядка нескольких десятков сантиметров. Лазеры позволили получить излучение с \(l_{\text {ког }}\) порядка сотен метров (и даже нескольких километров!).\\
Пример. На рис. 4.6 показана часть симметричного распределения интенсивности в интерференционной картине от двух щелей (аналог опыта Юнга). Длина волны используемого света \(\lambda \approx 0,5\) мкм. Оценим угловое расстояние между щелями относительно центра экрана, степень монохроматичности используемого света и длину его когерентности.\\
Из данного рисунка видно, что ширина полосы \(\Delta x=0,2\) мм. Значит, искомое угловое расстояние, согласно (4.6), \(\psi \approx \lambda / \Delta x= =0,5 \cdot 10^{-3} \mathrm{~mm} / 0,2 \mathrm{~mm}=2,5 \cdot 10^{-3}\) рад. Степень монохроматичности, согласно (4.9), равна предельному порядку интерференции.

\begin{figure}[h]
\begin{center}
  \includegraphics[alt={},max width=\textwidth]{950fefba-d83b-4c8f-96e7-0bd5a7eb90be-102_303_737_153_246}
\captionsetup{labelformat=empty}
\caption{Рис. 4.6}
\end{center}
\end{figure}

Наибольшему максимуму соответствует \(m=0\) (третий максимум слева). Следующих максимумов (порядков интерференции), как видно из рисунка, -8 . Значит \(\lambda / \Delta \lambda \approx m=8\). Длина когерентности \(l_{\text {ког }} \approx m \lambda=8 \cdot 0,5\) мкм \(=4,0\) мкм.\\
Итак, мы можем утверждать, что для получения интерференционной картины необходимо, чтобы оптическая разность хода складываемых колебаний была меньше длины когерентности:


\begin{equation*}
\Delta<l_{\text {ког }} \cdot \tag{4.11}
\end{equation*}


Это требование касается всех установок, с помощью которых мы хотим наблюдать картину интерференции.

В заключение заметим, что длина когерентности связана с так называемым временем когерентности \(\tau_{\text {ког }}\) - промежутком времени, в течение которого случайные изменения фазы световой волны в данной точке достигают значения порядка \(\pi\). За это время волна распространяется на расстояние порядка \(l_{\text {ког }}=c \tau_{\text {ког }}\).

Ширина когерентности. До сих пор щель \(S\) в опыте Юнга (рис. 4.2) предполагалась весьма узкой (часто говорят - бесконечно узкой). Расширение же щели, как и уменьшение степени монохроматичности света приводит к ухудшению (размытию) интерференционных полос и даже к полному их исчезновению. Чтобы выяснить роль ширины щели \(S\), рассмотрим теперь на примере опыта Юнга другой крайний случай: излучение монохроматическое, щель не узкая.

Интерференционную картину на экране Э (рис. 4.7) можно представить как наложение интерференционных картин от бесконечно узких щелей, на которые мысленно разобьем щель \(S\). Пусть положение максимумов на экране Э от узкой щели, взятой около верхнего края щели \(S\) - точки 1 - таково, как отме-

\begin{figure}[h]
\begin{center}
  \includegraphics[alt={},max width=\textwidth]{950fefba-d83b-4c8f-96e7-0bd5a7eb90be-103_398_661_153_88}
\captionsetup{labelformat=empty}
\caption{Рис. 4.7}
\end{center}
\end{figure}

\begin{figure}[h]
\begin{center}
  \includegraphics[alt={},max width=\textwidth]{950fefba-d83b-4c8f-96e7-0bd5a7eb90be-103_380_330_151_826}
\captionsetup{labelformat=empty}
\caption{Рис. 4.8}
\end{center}
\end{figure}

чено сплошными отрезками на рис. 4.8. А максимумы от узкой щели, взятой около нижнего края щели \(S\) - точки 2 - будут смещены вверх, они отмечены пунктирными отрезками на этом же рисунке. Интервалы между этими максимумами заполнены максимумами от промежуточных узких щелей, расположенных между краями 1 и 2.

При расширении щели \(S\) расстояния между максимумами от ее крайних элементов будут увеличиваться, т. е. интервалы между соседними максимумами от одного края щели будут постепенно заполняться максимумами от остальных элементов щели.

Для простоты будем считать, что в схеме (рис. 4.7) расстояния \(a=b\). Тогда при ширине щели \(s\), равной ширине интерференционной полосы \(\Delta x\), интервал между соседними максимумами от края 1 будет целиком заполнен максимумами от остальных элементов щели, и интерференционные полосы исчезнут.

Итак, при расширении щели \(S\) интерференционная картина постепенно размывается и при некоторой ширине щели практически исчезает.

Это наблюдаемое явление можно объяснить и иначе, а именно: интерференционная картина исчезает вследствие того, что вторичные источники - щели \(S_{1}\) и \(S_{2}\) (рис. 4.7) становятся некогерентными. Сказанное позволяет говорить о ширине когерентности падающей на щели \(S_{1}\) и \(S_{2}\) световой волны - ширине \(h_{\text {ког }}\), на которой отдельные участки волны в достаточной степени когерентны между собой. Во избежание недоразумений уточним: под шириной \(h_{\text {ког }}\) имеется в виду характерное для данной установки расстояние между точками поверхности, перпендикулярной направлению распространения волны.

Найдем формулу для вычисления \(h_{\text {ког }}\). В рассматриваемой схеме опыта Юнга запишем условие, при котором щели \(S_{1}\) и \(S_{2}\) становятся некогерентными источниками: \(h_{\text {ког }} \approx d\), где \(d\) - расстояние между щелями. Кроме того, мы выяснили, что интерференционная картина исчезает, когда ширина щели \(s \approx \Delta x\). Ширинаже полосы \(\Delta x\), согласно (4.6), равна \(\Delta x=\lambda l / d\). Из этих трех равенств получим:

\[
h_{\text {ког }} \approx d=\lambda l / \Delta x \approx \lambda l / s=\lambda /(s / l)=\lambda / \varphi \text {, }
\]

где \(\varphi\) - угловая ширина щели \(S\) относительно диафрагмы с двумя щелями. Итак, ширина когерентности


\begin{equation*}
h_{\text {ког }} \approx \lambda / \varphi . \tag{4.12}
\end{equation*}


Это значит, что ширина когерентности пропорциональна длине волны света и обратно пропорциональна угловой ширине источника относительно интересующего нас места (в опыте Юнга - относительно места расположения двух щелей). Сказанное поясняет рис. 4.9.

\begin{figure}[h]
\begin{center}
  \includegraphics[alt={},max width=\textwidth]{950fefba-d83b-4c8f-96e7-0bd5a7eb90be-104_281_390_730_774}
\captionsetup{labelformat=empty}
\caption{Рис. 4.9}
\end{center}
\end{figure}

Если в качестве источника использовать непосредственно Солнце (его угловой размер \(\varphi \approx 0,01\) рад и \(\lambda \approx 0,5\) мкм), то ширина когерентности, согласно (4.12), \(h_{\text {ког }} \approx 0,05\) мм. Для получения интерференционной картины от двух щелей с помощью такого излучения расстояние между двумя щелями должно быть меньше 0,05 мм, что сделать практически невозможно.

Формула (4.12) по существу лежит в основе метода, предложенного Физо и осуществленного Майкельсоном - по определению угловых размеров звезд путем измерения ширины когерентности \(h_{\text {ког }}\). Попытки провести эти измерения, помещая экран с двумя щелями перед объективом телескопа, оказались безуспешными: полосы интерференции оставались четкими даже при наибольшем расстоянии между этими щелями. Майкельсон преодолел эту трудность с помощью звездного интерферометра (рис. 4.10). Расположенные против щелей зеркала \(3_{0}-3_{0}\) неподвижны, а зеркала \(3-3\) можно одновременно раздвигать, меняя расстояние \(h\) между ними. Видность полос зависит от степени когерентности световых колебаний на зеркалах \(3-3\), в то

\begin{figure}[h]
\begin{center}
  \includegraphics[alt={},max width=\textwidth]{950fefba-d83b-4c8f-96e7-0bd5a7eb90be-105_533_752_150_239}
\captionsetup{labelformat=empty}
\caption{Рис. 4.10}
\end{center}
\end{figure}

время как ширина полос \(\Delta x\) определяется расстоянием между щелями. Постепенно раздвигая зеркала 3 - З, обнаруживают, что при определенном расстоянии \(h\) между ними интерференционная картина исчезает. Это значит, что расстояние \(h\) между этими зеркалами оказалось \(\sim h_{\text {ког }}\). Остается по формуле (4.12) вычислить \(\varphi\). При максимальном расстоянии \(h \approx 6\) м можно было измерить угловой диаметр объекта \(\varphi \approx 0,02\) угл. сек.

Первой звездой, угловой диаметр которой удалось определить, была Бетельгейзе ( 0,047 угл. сек). Измерив кроме того расстояние до нее (по параллаксу), определили диаметр этой звезды-гиганта (он оказался больше диаметра земной орбиты!).

Общие выводы. Для получения устойчивой интерференционной картины с использованием обычных (не лазерных) источников света необходимо исходную световую волну расщепить подходящим способом на две части, которые затем в области перекрытия и дадут систему полос, но... лишь в том случае, если у исходной световой волны:

\begin{enumerate}
  \item длина когерентности \(l_{\text {ког }}\) превышает оптическую разность хода \(\Delta\) складываемых колебаний и
  \item ширина когерентности \(h_{\text {ког }}\) превышает расстояние \(d\) между щелями*.
\end{enumerate}

\footnotetext{\begin{itemize}
  \item Далее мы увидим, что в разных интерференционных схемах под \(d\) надо понимать расстояние между некоторыми характерными лучами в месте расщепления исходной световой волны. См., например, задачу 4.6.
\end{itemize}
}Насколько больше должны быть эти величины, общепринятого соглашения нет. Будем считать, например, вдвое. Тогда можно записать:


\begin{align*}
& l_{\text {ког }} \geqslant 2 \Delta  \tag{4.13}\\
& h_{\text {ког }} \geqslant 2 d \tag{4.14}
\end{align*}


Выполнение этих условий гарантирует получение интерференционной картины с достаточно хорошей видностью полос.

\section*{§ 4.3. Интерференционные схемы}
Рассмотрим несколько интерференционных схем, отличающихся от схемы Юнга бо́льшей светосильностью.

Бипризма Френеля. В этой схеме для разделения исходной световой волны используют двойную призму 5 (бипризму) с малым преломляющим углом \(\theta\) (рис. 4.11). Источником света служит ярко освещенная узкая щель \(S\), параллельная преломляющему ребру бипризмы.

\begin{figure}[h]
\begin{center}
  \includegraphics[alt={},max width=\textwidth]{950fefba-d83b-4c8f-96e7-0bd5a7eb90be-106_424_906_968_163}
\captionsetup{labelformat=empty}
\caption{Рис. 4.11}
\end{center}
\end{figure}

Поскольку преломляющий угол бипризмы очень мал (порядка десятка угловых минут), то, как можно показать (см. задачу 3.6), все лучи отклоняются бипризмой на практически одинаковый угол \(\alpha=(n-1) \theta\). В результате образуются две когерентные волны, как бы исходящие из мнимых источников \(S_{1}\) и \(S_{2}\), лежащих в одной плоскости со щелью \(S\).

Ширину \(\Delta x\) интерференционных полос находим по первой из формул (4.6), учитывая, что в данном случае \(l=a+b\) и рас-

стояние между изображениями \(S_{1}\) и \(S_{2}\) щели \(S\) равно \(d=a \cdot 2 \alpha\). Таким образом,


\begin{equation*}
\Delta x=\frac{\lambda}{2 \alpha}\left(1+\frac{b}{a}\right) . \tag{4.15}
\end{equation*}


Видно, что ширина полос тем больше, чем больше расстояние \(b\) от бипризмы до экрана.

Если же на бипризму падает плоская волна, т. е. \(a \rightarrow \infty\), то


\begin{equation*}
\Delta x=\lambda / 2 \alpha . \tag{4.16}
\end{equation*}


Откуда следует, что ширина полосы в этом случае не зависит от положения экрана (расстояния \(b\) ).

При наблюдении в белом свете центральный максимум (нулевого порядка, \(m=0\) ) получается белым, остальные окрашенными, поскольку \(\Delta x\) г \(\lambda\).

Максимальное число \(N\) возможных полос интерференции на экране, где ширина зоны интерференции \(x=b \cdot 2 \alpha\) (см. рис. 4.11), определяется условием \(N_{\text {макс }}=x / \Delta x\). Отсюда следует с учетом (4.15), что


\begin{equation*}
N_{\text {макс }}=\frac{4 \alpha^{2}}{\lambda} \frac{a b}{a+b} . \tag{4.17}
\end{equation*}


Пример. Убедимся, что для получения интерференционной картины с шириной полос, например, \(\Delta x=0,5\) мм при размерах установки \(a=50 \mathrm{~cm}, b=100 \mathrm{~cm}\) преломляющий угол бипризмы \(\theta\) должен быть весьма малым. Будем считать, что показатель преломления стекла \(n=1,5\) и длина волны света \(\lambda=0,5\) мкм. Из (4.15) следует, если учесть, что угол \(\alpha=(n-1) \theta\) :

\[
\theta=\frac{\lambda(1+b / a)}{2(n-1) \Delta x}=3 \cdot 10^{-3} \text { рад } \approx 10 \text { угл. мин. }
\]

Найдем, кстати, и ширину \(x\) зоны интерференции на экране:

\[
x=b \cdot 2 \alpha=2(n-1) \theta b=3 \text { мм. }
\]

Видно, что \(x \ll b\). Это характерно для многих интерференционных схем, что мы ранее и учитывали, упрощая некоторые расчеты.

В предыдущем параграфе было показано, что условия, подобные рассмотренным нами сейчас для случая бипризмы Френеля, являются необходимыми, но еще не достаточными для получения интерференционной картины. Следует обязательно учесть роль ширины \(s\) щели (она связана с шириной когерентности) и степень монохроматичности \(\lambda / \Delta \lambda\) используемого света (которая связана с длиной когерентности). Оказывается (расчет можно посмотреть в задаче 4.4), для получения интерференционной картины с достаточно хорошей видностью нужно, чтобы ширина \(s\) щели удовлетворяла условию


\begin{equation*}
s \leqslant \frac{\lambda}{4 \alpha}\left(1+\frac{a}{b}\right) \tag{4.18}
\end{equation*}


а степень монохроматичности - условию


\begin{equation*}
\frac{\lambda}{\Delta \lambda} \geqslant \frac{4 \alpha^{2} a b}{\lambda(a+b)} \tag{4.19}
\end{equation*}


где \(\alpha=(n-1) \theta\).\\
Следует обратить внимание на то, что для увеличения ширины \(\Delta x\) интерференционных полос нужно, согласно (4.15), увеличивать отношение \(b / a\). А чтобы использовать более широкую щель \(S\), т. е. добиться бо́льшей светосильности установки, надо, как видно из (4.18), наоборот - увеличивать обратное отношение \(a / b\). Компромисс между этими двумя противоположными требованиями решается обычно экспериментально.

Бизеркала Френеля. Здесь две когерентные световые волны получают при отражении от двух зеркал, плоскости которых образуют между собой небольшой угол \(\alpha\) (рис. 4.12). Источник узкая ярко освещенная щель \(S\), параллельная линии пересечения зеркал. Отраженные от зеркал пучки падают на экран Э и там, где они перекрываются (зона интерференции), возникает интерференционная картина в виде полос, параллельных щели \(S\). Отраженные от зеркал волны распространяются так, как если бы они исходили из мнимых источников \(S_{1}\) и \(S_{2}\), являющихся изображениями щели \(S\).

Найдем ширину \(\Delta x\) интерференционных полос на экране Э. Воспользуемся первой из формул (4.6). В нашем случае \(l=a+b\)

\begin{figure}[h]
\begin{center}
  \includegraphics[alt={},max width=\textwidth]{950fefba-d83b-4c8f-96e7-0bd5a7eb90be-109_447_806_152_213}
\captionsetup{labelformat=empty}
\caption{Рис. 4.12}
\end{center}
\end{figure}

и \(d=2 \alpha \cdot a\), поэтому


\begin{equation*}
\Delta x=\frac{\lambda}{2 \alpha}\left(1+\frac{b}{a}\right) . \tag{4.20}
\end{equation*}


Видно, что ширина полос растет с увеличением расстояния \(b\). Если же на бизеркала падает плоская волна, т. е. \(a \rightarrow \infty\), то


\begin{equation*}
\Delta x=\lambda / 2 \alpha, \tag{4.21}
\end{equation*}


значит ширина полос в этом случае не зависит от расстояния \(b\) положения экрана.

Число возможных полос на экране \(N=x / \Delta x\), где \(x\) - ширина зоны интерференции на экране, \(x=b \cdot 2 \alpha\). Следовательно,


\begin{equation*}
N=\frac{4 \alpha^{2}}{\lambda} \frac{a b}{a+b} . \tag{4.22}
\end{equation*}


Но чтобы все эти полосы были действительно видны (и достаточно хорошо), нужно удовлетворить требованиям (4.13) и (4.14). Не вдаваясь в детали вывода (он аналогичен решению задачи 4.4 для бипризмы Френеля), получим, что ширина \(s\) щели \(S\) должна быть


\begin{equation*}
s \leqslant \frac{\lambda}{4 \alpha}\left(1+\frac{a}{b}\right), \tag{4.23}
\end{equation*}


а степень монохроматичности используемого света


\begin{equation*}
\frac{\lambda}{\Delta \lambda} \geqslant \frac{4 \alpha^{2}}{\lambda} \frac{a b}{a+b} . \tag{4.24}
\end{equation*}


Обращает на себя внимание то, что полученные формулы полностью идентичны с формулами для бипризмы Френеля.

Билинза Бийе. Обычную собирательную линзу разрезают пополам по диаметру, удаляя слой небольшой толщины, и обе половинки ее сдвигают (или немного раздвигают). Такую систему и называют билинзой. Рассмотрим билинзу, у которой толщина удаленного слоя равна \(\delta\), а источник - ярко освещенная щель \(S\) - расположен в плоскости, соединяющей обе половинки билинзы, и находится в ее фокальной плоскости на расстоянии \(f\) от билинзы (рис. 4.13). В этом случае оптический центр \(O_{1}\) верхней половинки 1 билинзы и оптический центр \(O_{2}\) нижней половинки 2 расположены как показано на рисунке, и расстояние между этими оптическими центрами равно толщине удаленного слоя, т. е. \(\delta\). Изобразив пунктиром побочные оптические оси, проходящие через щель \(S\) и оптические центры обеих половинок билинзы, можно построить и ход лучей через эти половинки.

\begin{figure}[h]
\begin{center}
  \includegraphics[alt={},max width=\textwidth]{950fefba-d83b-4c8f-96e7-0bd5a7eb90be-110_376_886_873_173}
\captionsetup{labelformat=empty}
\caption{Рис. 4.13}
\end{center}
\end{figure}

Таким образом, мы видим, что билинза расщепляет падающую на нее световую волну на две части, которые затем частично перекрываются (зона интерференции). На экране Э в области перекрывания волн должна возникнуть при определенных дополнительных условиях интерференционная картина.

Ширину \(\Delta x\) интерференционной полосы можно найти с помощью второй из формул (4.6), для этой цели она более удобна. Имея в виду, что угол между направлениями распространения двух плоских волн, как видно из рис. 4.13, равен \(\alpha=\delta / f\), получим:


\begin{equation*}
\Delta x=\lambda f / \delta . \tag{4.25}
\end{equation*}


Отсюда следует, что ширина полосы в данном случае не зависит от расстояния между экраном и билинзой.

Для подсчета числа полос на экране надо учесть, что зона интерференции здесь имеет вид вытянутого ромба, максимальная ширина \(x_{\text {макс }}\) которого равна половине диаметра \(D\) билинзы: \(x_{\text {макс }}=D / 2\). Поэтому важно знать, в каком месте этого «ромба» находится экран. Если он расположен ближе места, где \(x=x_{\text {макс }}\) (обычно так и бывает), то ширина зоны интерференции на экране будет \(x=b \alpha=b \delta / f\). И число \(N\) возможных полос интерференции окажется \(N=x / \Delta x\), т. е.


\begin{equation*}
N=b \delta^{2} / f^{2} \lambda \tag{4.26}
\end{equation*}


Проведем некоторые оценки, чтобы иметь представление о порядке величин, с которыми приходится иметь дело в подобных интерференционных схемах. Для этого рассмотрим следующий числовой пример.\\
Пример. Пусть билинза, о которой шла речь, имеет фокусное расстояние \(f=400\) мм и диаметр \(D=50 \mathrm{~mm}\). Толщина удаленного слоя \(\delta=1,0\) мм, длина волны света \(\lambda=0,50\) мкм. Щель находится в фокальной плоскости билинзы. Вычислим ширину полосы \(\Delta x\), расстояние \(b\), на котором число полос может быть максимальным, \(N_{\text {макс }}\), и значение \(N_{\text {макс }}\).\\
Согласно (4.25), \(\Delta x=\lambda f / \delta=0,2\) мм. В нашем случае \(\Delta x\) не зависит от расстояния \(b\), поэтому максимальное число полос будет в том месте, где ширина зоны интерференции тоже максимальна и равна, как видно из рис. \(4.13, D / 2\). Из этого рисунка также видно, что \(D / 2 \approx b \cdot \alpha\), где \(\alpha=\delta / f\). Отсюда

\[
b=D / 2 \alpha=D f / 2 \delta=10 \text { м. }
\]

На этом расстоянии можно было бы наблюдать максимальное число полос

\[
N_{\text {макс }}=(D / 2) / \Delta x=D \delta / 2 \lambda f \approx 125 .
\]

Обычно берут расстояние \(b \sim 1 \mathrm{~m}\), на котором можно наблюдать 12 полос.\\
Остается выяснить дополнительные условия, которым должны удовлетворять ширина \(s\) щели \(S\) и степень монохроматичности \(\lambda / \Delta \lambda\) используемого света, чтобы интерференционную картину можно было получить, причем с достаточно хорошей

видностью. Эти условия мы найдем с помощью соотношений (4.14) и (4.9). Предоставив желающим в этом убедиться самостоятельно, выпишем их для нашего случая, когда щель находится в фокальной плоскости билинзы:


\begin{gather*}
s \leqslant \lambda f^{2} / b \delta  \tag{4.27}\\
\lambda / \Delta \lambda \geqslant m_{\text {макс }} \approx b \delta^{2} / 2 \lambda f^{2}, \tag{4.28}
\end{gather*}


где \(m_{\text {макс }}\) - максимальный порядок интерференции на экране, отстоящем на расстояние \(b\) от билинзы (он равен отношению полуширины зоны интерференции к ширине интерференционной полосы).

В заключение следует заметить, что обзор интерференционных схем на этом, разумеется, не ограничивается. На трех рассмотренных схемах мы продемонстрировали общность подхода к расчету интерференционных картин, поучаемых подобными схемами. Из существующих в настоящее время интерференционных схем можно назвать еще и такие: зеркало Ллойда (см. задачу 4.1), интерферометр Рэлея (см. задачу 4.2), звездный интерферометр Майкельсона (см. в конце предыдущего параграфа), интерферометр Маха-Цендера и др. Некоторые из них нашли широкое применение при проведении очень тонких и высокочувствительных измерений.

Параллельно была разработана другая идея получения и интерференции когерентных волн - путем расщепления первичной волны при отражении от двух границ раздела прозрачных диэлектриков. К изучению этого метода мы и переходим.

\section*{§ 4.4. Интерференция при отражении от тонких пластинок}
При отражении от плоскопараллельной пластинки. Пусть на прозрачную плоскопараллельную пластинку падает плоская монохроматическая световая волна, направление распространения которой показано падающим лучом на рис. 4.14. В результате отражений от обеих поверхностей пластинки исходная волна расщепится на две, что и показано лучами 1 и 2. Амплитуды этих волн мало отличаются друг от друга - это важно для получения достаточно контрастной картины интерференции.

\begin{figure}[h]
\begin{center}
  \includegraphics[alt={},max width=\textwidth]{950fefba-d83b-4c8f-96e7-0bd5a7eb90be-113_447_582_153_324}
\captionsetup{labelformat=empty}
\caption{Рис. 4.14}
\end{center}
\end{figure}

Заметим, что, кроме этих двух отраженных волн (1 и 2), возникает еще многократное отражение. Однако их вклад практически пренебрежимо мал (см. задачу 3.3), и мы ограничимся только волнами, возникшими при однократном отражении.

Оптическую разность хода волн 1 и 2 определим, согласно рис. 4.14, как


\begin{equation*}
\Delta=n(A B+B C)-A D, \tag{4.29}
\end{equation*}


где \(n\) - показатель преломления вещества пластинки. Кроме того, видно, что \(A B=B C=2 b / \cos \vartheta^{\prime}\) и \(A D=2 b \operatorname{tg} \vartheta^{\prime} \cdot \sin \vartheta, b-\) толщина пластинки. В результате подстановки этих выражений в (4.29) получим


\begin{equation*}
\Delta=2 n b \cos \vartheta^{\prime} . \tag{4.30}
\end{equation*}


Следует также учесть, что при отражении от верхней поверхности пластинки (от среды, оптически более плотной) в соответствии с (3.11) происходит скачок фазы на \(\pi\) у отраженной волны, т. е., как говорят, «потеря» полуволны ( \(\pm \lambda / 2\) ). Учитывая еще, что \(\sin \vartheta=n \sin \vartheta^{\prime}\), получим


\begin{equation*}
\Delta=2 b \sqrt{n^{2}-\sin ^{2} \vartheta}-\lambda / 2 \tag{4.31}
\end{equation*}


(здесь можно было написать и \(+\lambda / 2\), но это не существенно).\\
Если отраженные волны 1 и 2 когерентны между собой (а мы об этом позаботимся), то максимумы отражения будут на-

блюдаться при условии


\begin{equation*}
2 b \sqrt{n^{2}-\sin ^{2} \vartheta}-\lambda / 2=m \lambda, \tag{4.32}
\end{equation*}


где \(m\) - целое число (порядок интерференции).\\
Меняя угол падения Э, мы будем наблюдать последовательную смену максимумов и минимумов отражения. (Заметим, что при минимуме отражения наблюдается максимум проходящего через пластинку света, и наоборот.) Если бы обе отраженные волны были некогерентными, то такого явления мы не наблюдали бы: по мере увеличения угла падения интенсивность отраженного света монотонно уменьшалась бы.

Теперь выясним условия, при которых отраженные волны будут когерентными и смогут интерферировать, т. е. выполняются соотношения (4.13) и (4.14).

\begin{figure}[h]
\begin{center}
  \includegraphics[alt={},max width=\textwidth]{950fefba-d83b-4c8f-96e7-0bd5a7eb90be-114_239_1098_839_66}
\captionsetup{labelformat=empty}
\caption{Рис. 4.15}
\end{center}
\end{figure}

Проиллюстрируем ситуацию с помощью рис. 4.15. Выделим в падающей волне некоторую область когерентности \(l_{\text {ког }} \cdot h_{\text {ког }}\) (она слегка затенена на рисунке) и проследим за ее дальнейшей судьбой. После расщепления падающей волны расщепится и выделенная область когерентности, причем так, что в отраженных волнах эти области когерентности сместятся относительно друг друга (рис. 4.15, \(a\) ). Если они при этом перекрываются (на рисунке более темный участок), интерференция будет наблюдаться и тем более отчетливо, чем больше степень перекрытия. Нетрудно видеть, что для пластинки с большей толщиной область перекрытия когерентных участков уменьшается (рис. 4.15, б), и интерференция будет наблюдаться все менее отчетливо. Начиная с некоторой толщины пластинки итерференция исчезнет совсем.

Из рис. 4.15 видно, что смещение расчлененных частей области когерентности происходит как вдоль распространения волны (оно не должно превосходить длину когерентности \(l_{\text {ког }}\) ), так и поперек распространения волны (смещение не должно превосходить ширину когерентности \(h_{\text {ког }}\) ). Интерференция будет наблюдаться лишь в том случае, когда будут удовлетворены оба эти условия. Напомним, что для лучшей видности мы договорились брать половины значений \(l_{\text {ког }}\) и \(h_{\text {ког }}\).

Перейдем к расчету. Согласно (4.13), необходимо, чтобы оптическая разность хода \(\Delta \leqslant l_{\text {ког }} / 2\). Следовательно,


\begin{equation*}
2 b \sqrt{n^{2}-\sin ^{2} \vartheta}-\lambda / 2 \leqslant l_{\text {ког }} / 2 . \tag{4.33}
\end{equation*}


Для оценки необходимого значения толщины пластинки \(b\) будем считать, что корень в этом выражении равен величине порядка единицы (что обычно и бывает), а также пренебрежем \(\lambda / 2\). Тогда получим

\[
2 b \leqslant l_{\text {ког }} / 2,
\]

т. е. необходимо, чтобы удвоенная толщина пластинки была не более половины длины когерентности используемого излучения. Например, если \(\lambda=600 \mathrm{HM}\), а \(\Delta \lambda=3\) нм, то толщина пластинки

\[
b \leqslant \lambda^{2} / 4 \Delta \lambda=3 \cdot 10^{4} \text { нм }=30 \text { мкм. }
\]

Далее, поперечный сдвиг частей области когерентности не должен превосходить половины ширины когерентности \(h_{\text {ког }}\). Этот сдвиг, как видно из рис. 4.14 , равен отрезку \(D C\). Значит, необходимо, чтобы \(D C \leqslant h_{\text {ког }} / 2\). Из рис. 4.14 следует, что

\[
D C=2 b \operatorname{tg} \vartheta^{\prime} \cos \vartheta=b \frac{\sin 2 \vartheta}{\sqrt{n^{2}-\sin ^{2} \vartheta}} .
\]

Это смещение существенно зависит от угла падения Э. Чем меньше угол падения, тем меньше смещение \(D C\), тем меньше может быть \(h_{\text {ког }}\). И основную роль в этом случае будет играть длина когерентности. При \(\vartheta=0\) смещение происходит только вдоль распространения волн, поперек - оно равно нулю, и ширина когерентности \(h_{\text {ког }}\) становится практически не существенной.

Обратимся к вопросу, что следует понимать под словами «тонкая» пластинка. Когда говорят, что интерференция происходит при отражении от тонкой пластинки, то имеют в виду, что ее толщина меньше (в той или иной степени) \(l_{\text {ког }}\) и \(h_{\text {ког }}\) (если \(\vartheta \neq 0\) ). Причем - это важно - при нормальном падении интерференция обеспечивается только соотношением между толщиной пластинки и \(l_{\text {ког }}\).

Для солнечного света ( \(l_{\text {ког }} \approx 5 \lambda\) ) пластинка будет тонкой, если ее толщина порядка нескольких длин волн. Длину когерентности можно увеличить с помощью светофильтров, соответственно увеличивается и толщина пластинки, которую мы называем тонкой. Для лазерного же излучения тонкой будет пластинка в десятки сантиметров и метров (в зависимости от длины когерентности излучения используемого лазера).

Итак, мы выяснили, что при падении плоской световой волны на плоскопараллельную тонкую пластинку интенсивность отраженного света зависит от угла падения. Изменяя этот угол, мы будем наблюдать чередование максимумов и минимумов отраженного света. Это можно использовать для получения интерференционной картины в виде привычной системы полос. Достаточно использовать в качестве падающего рассеянный монохроматический свет (он содержит волны, падающие на пластинку одновременно под разными углами), а на пути отраженного света поставить линзу и в ее фокальной плоскости экран (рис. 4.16).

Максимумы на экране будут располагаться в местах, соответствующих условию (4.32). Полоса данного порядка интерференции обусловлена светом, падающим на пластинку под од-

\begin{figure}[h]
\begin{center}
  \includegraphics[alt={},max width=\textwidth]{950fefba-d83b-4c8f-96e7-0bd5a7eb90be-116_390_638_1387_296}
\captionsetup{labelformat=empty}
\caption{Рис. 4.16}
\end{center}
\end{figure}

ним и тем же углом \(\vartheta\), но с разных направлений. Поэтому такие полосы называют полосами равного наклона. При расположении линзы как показано на рисунке, эти полосы имеют вид концентрических колец с центром в ее фокусе \(F\). Порядок интерференции \(m\) растет с уменьшением угла падения \(\vartheta\), и в центре картины он максимален.

Поскольку для наблюдения интерференционной картины в данном случае экран помещают в фокальной плоскости линзы, т. е. так, как его располагают для получения на нем изображения бесконечно удаленных предметов, то говорят, что полосы равного наклона локализованы в бесконечности. Роль линзы и экрана может играть хрусталик и сетчатка глаза. В этом случае для наблюдения полос равного наклона глаз нужно аккомодировать (настраивать) так, как при рассмотрении удаленных предметов.

В белом свете интерференционные полосы окрашены. Поэтому такое явление называют цвета тонких пластинок.

Клиновидные пластинки. Пусть стеклянная пластинка имеет форму клина с углом раствора \(\alpha \ll 1\), и на нее падает плоская монохроматическая световая волна. Теперь отраженные от поверхностей клина световые волны будут распространяться не в одном направлении, а под некоторым углом (рис. 4.17).

Выясним прежде всего, где будет локализована интерференционная картина. Это проще всего сделать с помощью рис. 4.18, на котором показано, что происходит с областью когерентности после расщепления волны при отражении от поверхностей клина. Ясно, что при небольших значения \(l_{\text {ког }}\) и \(h_{\text {ког }}\) область перекрытия когерентных частей отраженных волн локализована в основном вблизи поверхности клина и становится все более узкой по мере перемещения в сторону утолщения клина, постепенно исчезая совсем (рис. 4.18). На рисунке область локализации несколько затемнена.

\begin{figure}[h]
\begin{center}
  \includegraphics[alt={},max width=\textwidth]{950fefba-d83b-4c8f-96e7-0bd5a7eb90be-117_173_365_1564_175}
\captionsetup{labelformat=empty}
\caption{Рис. 4.17}
\end{center}
\end{figure}

\begin{figure}[h]
\begin{center}
  \includegraphics[alt={},max width=\textwidth]{950fefba-d83b-4c8f-96e7-0bd5a7eb90be-117_279_476_1507_649}
\captionsetup{labelformat=empty}
\caption{Рис. 4.18}
\end{center}
\end{figure}

Так как разность хода лучей, отразившихся от различных участков клина, неодинакова, в области локализации интерференции появятся светлые и темные полосы, параллельные ребру клина. Каждая из таких полос возникает в результате отражений от участков клина с одинаковой толщиной, поэтому их на-

\begin{figure}[h]
\begin{center}
  \includegraphics[alt={},max width=\textwidth]{950fefba-d83b-4c8f-96e7-0bd5a7eb90be-118_311_382_153_782}
\captionsetup{labelformat=empty}
\caption{Рис. 4.19}
\end{center}
\end{figure}

зывают полосами равной толщины.

Локализованные вблизи поверхности клина интерференционные полосы можно наблюдать непосредственно глазом, фокусируя его на поверхность клина, либо с помощью линзы, сфокусированной также на поверхность клина. С помощью линзы интерференционную картину с поверхности клина можно отобразить на экран Э, расположенный в плоскости, сопряженной с поверхностью клина (см. рис. 4.19).

Полосы равной толщины можно наблюдать в тонкой клиновидной прослойке воздуха между поверхностями двух прозрачных пластинок. Если направление наблюдения близко к нормальному, то оптическая разность хода лучей, отраженных от поверхностей клина в месте, где ширина зазора равна \(b\), определяется как \(\Delta=2 b-\lambda / 2\). Там, где \(\Delta=(m-1 / 2) \lambda, m=1,2, \ldots\), возникают минимумы. Т. е. темные полосы возникают в тех местах, где толщина зазора \(b=m \lambda / 2\). Переход от одной полосы к соседней (ширина полосы \(\Delta x\) ) соответствует изменению толщины зазора на \(\lambda / 2\). Каждая полоса соответствует определенной толщине зазора. Это используют для проверки качества оптических поверхностей при их шлифовке, наблюдая в отраженном свете прослойку между контролируемой и эталонной поверхностями.

Измеряя ширину полосы \(\Delta x\) в монохроматическом свете с известной длиной волны \(\lambda\), можно измерить углы порядка 0,1 угл. мин и меньше (см. задачу 4.6).

В предыдущих рассуждениях мы принимали во внимание только волны, отраженные от поверхностей воздушного клиновидного зазора, игнорируя волны, отраженные от наружных поверхностей пластинок. Это можно сделать лишь постольку, поскольку толщина пластинок обычно значительно превышает длину когерентности используемого света от обычных источников, и отраженные от этих поверхностей волны оказываются некогерентными.

Кольца Ньютона - это кольцевые полосы равной толщины, наблюдаемые при отражении света от поверхностей зазора меж-

\begin{figure}[h]
\begin{center}
  \includegraphics[alt={},max width=\textwidth]{950fefba-d83b-4c8f-96e7-0bd5a7eb90be-119_282_348_237_68}
\captionsetup{labelformat=empty}
\caption{Рис. 4.20}
\end{center}
\end{figure}

ду стеклянной пластинкой и соприкасающейся с ней выпуклой линзой (рис. 4.20). Волна, отраженная от верхней поверхности линзы, в силу небольшой длины когерентности обычных источников света, некогерентна с волнами, отраженными от поверхностей зазора, и участия в образовании интерференционной картины не принимает. Поэтому мы ее и не будем учитывать.

При нормальном падении света кольца в отраженном свете имеют вид концентрических окружностей с центром в точке соприкосновения линзы с пластинкой. Найдем радиусы \(r\) темных колец (минимумов).

Сначала запишем условие образования темных колец. Они возникают там, где оптическая разность хода \(\Delta\) волн, отраженных от обеих поверхностей зазора, равна нечетному числу полуволн:

\[
\Delta=2 b+\lambda / 2=(2 m+1) \lambda / 2,
\]

где \(\lambda / 2\) связано с «потерей» полуволны при отражении от пластинки и \(m=0,1,2, \ldots\) Отсюда


\begin{equation*}
2 b=m \lambda . \tag{4.34}
\end{equation*}


Далее, согласно теореме Пифагора (см. рис. 4.20), \(r^{2}=R^{2}-(R-b)^{2}\). Учитывая, что \(b \ll R\), получим


\begin{equation*}
r^{2}=2 b R . \tag{4.35}
\end{equation*}


Из (4.34) и (4.35) следует, что радиус \(m\)-го темного кольца


\begin{equation*}
r_{m}=\sqrt{m \lambda R}, \quad m=0,1,2, \ldots \tag{4.36}
\end{equation*}


Заметим, что значению \(m=0\) соответствует минимум темного пятна (не кольца). Аналогичный расчет можно провести и для светлых колец.

Пример. Найдем радиус 5 -го светлого кольца, если радиус кривизны выпуклой поверхности линзы \(R=16 \mathrm{~cm}\) и контакт ее с плоской поверхностью стекла идеальный (в точке). Длина волны света \(\lambda =0,50\) мкм.

Условие максимумов в данном случае имеет вид

\[
2 b+\lambda / 2=m \lambda, \quad m=1,2,3, \ldots,
\]

где \(b\) - толщина зазора в месте \(m\)-го максимума (заметим, что если бы мы взяли \(-\lambda / 2\), то значения \(m\) надо было начинать с нуля). Согласно (4.35), \(2 b=r_{m}^{2} / R\). Из этих двух соотношений следует, что искомый радиус

\[
r_{m}=\sqrt{(m-1 / 2) \lambda R}=0,6 \text { мкм. }
\]

Следует обратить внимание на то, что формула (4.36) справедлива лишь в случае идеального (точечного) контакта сферической поверхности линзы с пластинкой. Но идеальных контактов не бывает, номера колец не равны, вообще говоря, порядку интерференции \(m\), и это обстоятельство необходимо учитывать при расчетах (см. задачу 4.8).

Если линзу постепенно отодвигать от поверхности пластинки, то интерференционные кольца будут стягиваться к центру: это ведь кольца (полосы) равной толщины, а она при этом перемещается к центру.

С помощью колец Ньютона можно с достаточно высокой точностью контролировать качество изготовления, например, сферических поверхностей.

Рассмотрим теперь на конкретном примере вопрос, связанный с причиной локализации колец Ньютона в очень малой области для обычных линз (кольца приходится рассматривать в микроскоп).

Пример. Плосковыпуклая линза, радиус кривизны сферической поверхности которой \(R=60\) мм, соприкасается со стеклянной пластинкой. Оценим радиус наблюдаемой в отраженном свете интерференционной картины, если длина волны света \(\lambda=0,60\) мкм и \(\Delta \lambda=0,06\) мкм. Свет падает практически нормально.\\
При нормальном падении света ограничивать интерференционную картину будет только длина когерентности \(l_{\text {ког }}\). Кольца исчезают при условии \(2 b \approx l_{\text {ког }}\), где \(b\) - ширина зазора в месте исчезновения колец. Согласно (4.35), \(r^{2}=2 b R\), а \(l_{\text {ког }} \approx \lambda^{2} / \Delta \lambda\). Из этих формул получим \(r^{2} / R \approx \lambda^{2} / \Delta \lambda\), откуда

\[
r \approx \lambda \sqrt{R / \Delta \lambda}=0,6 \text { мм. }
\]

Число видимых колец равно, согласно (4.9), \(\lambda / \Delta \lambda \approx 10\). Этот результат можно получить и с помощью (4.36).

Просветление оптики. В ее основе лежит интерференция света при отражении от тонких пластинок. Дело в том, что при прохождении света через каждую преломляющую поверхность линзы отражается примерно \(4 \%\) падающего света. В сложных объективах такие отражения совершаются многократно и суммарная потеря светового потока оказывается весьма ощутимой. Например, в призменном бинокле она составляет свыше \(50 \%\).

В просветленной оптике на каждую поверхность линзы наносят путем напыления тонкую пленку прозрачного диэлектрика с показателем преломления \(n^{\prime} \approx \sqrt{n_{1} n_{2}}\), где \(n_{1}\) и \(n_{2}\) - показатели преломления сред, между которыми находится пленка. При этом условии амплитуды отраженных от обеих поверхностей пленки волн оказываются, согласно (3.11), практически одинаковыми. Толщина же пленки делается такой, чтобы волны, отраженные от обеих поверхностей пленки, оказывались в противофазе, т. е. гасили друг друга.

Пример. Найдем показатель преломления \(n^{\prime}\) пленки и ее толщину \(b\), чтобы отражение света с длиной волны \(\lambda=0,55\) мкм от поверхности стекла с показателем преломления \(n=1,5\) отсутствовало при нормальном падении света.\\
Сначала найдем \(n^{\prime}\), при котором амплитуды волн, отраженных

\begin{figure}[h]
\begin{center}
  \includegraphics[alt={},max width=\textwidth]{950fefba-d83b-4c8f-96e7-0bd5a7eb90be-121_237_292_1089_205}
\captionsetup{labelformat=empty}
\caption{Рис. 4.21}
\end{center}
\end{figure}

от обеих поверхностей пленки, были бы одинаковыми (рис. 4.21). Для этого, согласно (3.11), нужно чтобы \(E_{1}^{\prime} / E=E_{2}^{\prime} / E\). Мы будем пренебрегать многократными отражениями и считать, что амплитуды падающих на обе поверхности пленки волн практически одинаковы. В нашем случае это выглядит так:

\[
\frac{1-n^{\prime}}{1+n^{\prime}} \approx \frac{n^{\prime}-n}{n^{\prime}+n},
\]

откуда \(n^{\prime} \approx \sqrt{n}=1,22\). Твердых веществ с таким малым показателем преломления нет. Трудность может быть преодолена (что и делают) путем использования двухслойных покрытий с соответственно подобранными показателями преломления (в детали мы углубляться не будем).\\
Теперь определим толщину пленки, при которой отраженные волны будут в противофазе. Это значит, что оптическая разность хода этих двух отраженных волн на выходе из пленки

должна быть равна полуцелому числу длин волн:

\[
2 b n^{\prime}=(m+1 / 2) \lambda, \quad m=0,1,2, \ldots
\]

Здесь учтено, что обе волны отражаются от оптически более плотных сред и, значит, одинаково испытывают скачок фазы на \(\pi\) («потерю» полуволны). Из последнего условия находим

\[
b=\frac{(m+1 / 2) \lambda}{2 \sqrt{n}} .
\]

Наименьшая толщина (при \(m=0\) ) равна \(b=\lambda / 4 \sqrt{n}=0,10\) мкм.\\
У обычного света длина когерентности невелика, поэтому пленка должна иметь толщину порядка нескольких длин волн. Это обязывает с особой осторожностью относиться к таким покрытиям: малейшее механическое повреждение пленки разрушает ее действие.

Обычно просветление оптики проводят для средней (жел-то-зеленой) области видимого спектра. Для краев же этого спектра коэффициент отражения заметно отличается от нуля, и объективы кажутся в отраженном свете пурпурными, что соответствует смешению красного и фиолетового цветов.

\section*{§ 4.5. Интерферометр Майкельсона}
Этот интерферометр сыграл фундаментальную роль в развитии науки и техники. С его помощью впервые была измерена длина световой волны, проведено изучение тонкой структуры спектральных линий, выполнено первое прямое сравнение эталонного метра с определенной длиной волны света. С помощью этого интерферометра был осуществлен знаменитый опыт Май-кельсона-Морли, доказавший независимость скорости света от движения Земли.

Упрощенная схема интерферометра Майкельсона показана на рис. 4.22. Монохроматический свет от источника \(S\) падает на разделительную пластинку \(P\), которая состоит из двух одинаковой толщины плоскопараллельных стеклянных пластинок, склеенных друг с другом. Причем одна из склеиваемых поверхностей покрыта полупрозрачным тонким слоем серебра или алюминия.

\begin{figure}[h]
\begin{center}
  \includegraphics[alt={},max width=\textwidth]{950fefba-d83b-4c8f-96e7-0bd5a7eb90be-123_368_725_151_254}
\captionsetup{labelformat=empty}
\caption{Рис. 4.22}
\end{center}
\end{figure}

Пластинка \(P\) разделяет падающий на нее пучок на два взаимно перпендикулярных пучка 1 и 2 одинаковой интенсивности. Пучок 1 , отраженный затем от зеркала \(\zeta_{1}\), вторично падает на пластинку \(P\), где снова разделяется на две части. Одна из них отражается в сторону зрительной трубы \(T\), другая же идет к источнику \(S\) и не представляет интереса.

Пучок 2, прошедший пластинку \(P\), отражается от зеркала \(3_{2}\), возвращается к пластинке \(P\), где опять расчленяется на две части, одна из которых попадает в трубу \(T\).

Таким образом, от одного источника \(S\) получаются два пучка примерно одинаковой амплитуды, которые распространяются после разделительного слоя \(P\) в разных «плечах» интерферометра, затем снова встречаются и создают при условии соблюдения временной и пространственной когерентности интерференционную картину в фокальной плоскости объектива зрительной трубы.

Зеркало \(3_{1}\) неподвижно, а зеркало \(3_{2}\) можно перемещать поступательно и изменять его наклон.

Заменим мысленно зеркало \(3_{1}\) его мнимым изображением \(3_{1}^{\prime}\) (в полупрозрачном «зеркале» \(P\) ). Тогда пучки \(1^{\prime}\) и \(2^{\prime}\) можно рассматривать как возникающие при отражении от прозрачной «пластинки», ограниченной плоскостями \(3_{1}^{\prime}\) и \(3_{2}\), что заметно облегчает дальнейшие рассуждения.

Вид интерференционной картины зависит от юстировки зеркал и от расходимости пучка света, падающего на разделительную пластинку \(P\). Обычно используют два случая.

\begin{enumerate}
  \item Если пучок слегка расходящийся, а плоскости \(3_{2}\) и \(3_{1}^{\prime}\) параллельны, то получаются полось равного наклона, имеющие вид концентрических колец. При поступательном перемеще-
\end{enumerate}

нии зеркала \(3_{2}\) радиусы колец изменяются: когда зеркало \(3_{2}\) приближается к \(3_{1}^{\prime}\), кольца стягиваются к центру, где и исчезают (обратное тому, что наблюдается в случае колец Ньютона). Смещение картины на одну полосу соответствует перемещению зеркала \(3_{2}\) на половину длины волны. Визуально смещение можно оценить с точностью до \(1 / 20\) полосы, но есть методы, позволяющие обнаружить смещение до \(10^{-3}\) полосы. По мере приближения \(3_{2} \kappa 3_{1}^{\prime}\) ширина полос возрастает, и при совпадении \(3_{2}\) с \(3_{1}^{\prime}\) освещенность поля зрения становится равномерной.\\
2. Если пучок от источника \(S\) параллельный, а плоскости \(3_{2}\) и \(3_{1}^{\prime}\) не параллельны, то в поле зрения трубы будут наблюдаться полось равной толщиньι (как от клиновидной пластинки). В месте пересечения \(3_{2}\) и \(3_{1}^{\prime}\) - белый максимум (нулевой порядок интерференции, \(m=0\) ).

При больших расстояниях между \(3_{2}\) и \(3_{1}^{\prime}\) и высокой степени монохроматичности света удавалось с помощью не лазерных источников наблюдать интерференцию очень высокого порядка (около \(10^{6}\) ).

\section*{§ 4.6. Многолучевая интерференция}
При суперпозиции двух плоских когерентных световых волн образуются интерференционные полосы с распределением интенсивности, согласно (4.8), \(I \sim \cos ^{2}(\pi \Delta / \lambda)\), где \(\Delta\) - оптическая разность хода данных волн. Светлые и темные полосы в этом случае одинаковы по ширине.

При суперпозиции же большого числа волн распределение интенсивности в интерференционной картине существенно меняется: образуются узкие максимумы, т. е. резкие светлые полосы, разделенные широкими темными промежутками. Благодаря этому многолучевая интерференция получила важные практические применения.

Большое число когерентных световых волн можно получить, например, при прохождении плоской волны через экран со множеством одинаковых регулярно расположенных отверстий. Распределение интенсивности в соответствующей интерференционной картине будет рассмотрено в § 5.7 на примере дифракционной решетки. Здесь же мы рассмотрим интерференцию при многократных отражениях света от двух паралле-

льных поверхностей. Практически это реализуется в интерферометре Фабри-Перо, который широко используется в спектроскопии высокого разрешения, метрологии и в качестве открытого резонатора лазеров.

Интерферометр Фабри-Перо делают в виде плоскопараллельной стеклянной или кварцевой пластины, на обе поверхности которой нанесены отражающие слои, либо в виде двух пластин, у которых покрытые отражающим слоем поверхности установлены строго параллельно друг другу и разделены воздушным промежутком.

Многократное отражение света от двух параллельных плоскостей приводит к образованию интерференционных полос равного наклона, локализованных в бесконечности или в фокальной плоскости объектива (рис. 4.23). Полосы имеют вид резких светлых концентрических колец с центром в фокусе \(F\) объектива. Максимумы (кольца) тем у́же, чем больше отражательная способность поверностей пластин интерферометра. А она может быть доведена до \(95-98 \%\). С увеличением радиуса ко́льца располагаются все ближе друг к другу.

\begin{figure}[h]
\begin{center}
  \includegraphics[alt={},max width=\textwidth]{950fefba-d83b-4c8f-96e7-0bd5a7eb90be-125_398_556_992_338}
\captionsetup{labelformat=empty}
\caption{Рис. 4.23}
\end{center}
\end{figure}

Слева (рис. 4.23) на интерферометр падает рассеянный свет. В некоторую точку \(P\) фокальной плоскости объектива собираются лучи, которые до объектива образуют с его оптической осью один и тот же угол θ. Далее будем считать для упрощения расчета, что мы имеем дело только с двумя параллельными отражающими поверхностями, т. е. будем предполагать, что всюду показатель преломления \(n=1\) (но это не принципиально). В этом случае разность хода \(\Delta\) двух соседних интерферирующих

лучей (волн) легко можно найти с помощью рис. 4.24:

\[
\Delta=B A C-B D=\frac{2 b}{\cos \theta}-2 b \operatorname{tg} \theta \sin \theta=2 b \cos \theta
\]

Максимумы интенсивности в проходящем свете образуются там, где \(\Delta\) составляет целое число \(m\) длин волн:


\begin{equation*}
2 b \cos \theta=m \lambda . \tag{4.37}
\end{equation*}


Отсюда видно, что с уменьшением угла \(\theta\), т. е. с приближением к центру колец, порядок интерференции \(m\) растет. Расстояние \(b\) между зеркальными поверхностями обычно составляет \(1 \div 100 м м\) (а в специальных случаях и до \(1 м\) ). Поэтому порядки интерференции \(m \approx 2 b / \lambda\) весьма велики: при \(b=5\) мм \(m \approx 20000\).

\begin{figure}[h]
\begin{center}
  \includegraphics[alt={},max width=\textwidth]{950fefba-d83b-4c8f-96e7-0bd5a7eb90be-126_317_422_512_742}
\captionsetup{labelformat=empty}
\caption{Рис. 4.24}
\end{center}
\end{figure}

Из (4.37) следует, что угол \(\theta\) зависит от \(\lambda\). На этом основано использование данного интерферометра в качестве спектрального прибора. Одной из важнейших характеристик любого спектрального прибора является угловая дисперсия \(D=\mathrm{d} \theta / \mathrm{d} \lambda\). Она определяет угловое расстояние между спектральными линиями (одного порядка \(m\) ), отличающимися по длине волны на единицу (например, на 1 нм). Для интерферометра Фабри-Перо на основании (4.37) имеем: \(\mathrm{d} \theta / \mathrm{d} \lambda=-m / 2 b \sin \theta\). И поскольку из (4.37) видно, что \(m / 2 b=\cos \theta / \lambda\), то в результате получим:


\begin{equation*}
\frac{\mathrm{d} \theta}{\mathrm{~d} \lambda}=-\frac{1}{\lambda \operatorname{tg} \theta} \approx-\frac{1}{\lambda \cdot \theta}, \tag{4.38}
\end{equation*}


где знак минус означает, что с ростом \(\lambda\) угол \(\theta\) для максимумов того же порядка убывает.

В (4.38) учтено, что измерения обычно проводят на втором или третьем от центра максимуме (кольце), для которых \(\theta \sim 10^{-2}\) рад \(\approx 0,6^{\circ}\). Для таких углов \(\mathrm{d} \theta / \mathrm{d} \lambda \sim 10\) угл. град/нм, что значительно превышает угловую дисперсию других спектральных приборов и является основным преимуществом интерферометра Фабри-Перо.

Однако область дисперсии \(\Delta \lambda\), т. е. интервал длин волн, в пределах которого не происходит перекрывания спектром другого порядка, очень мала. Например, при \(b=5\) мм и \(\lambda=500\) нм \(\Delta \lambda\) составляет менее \(0,03 \mathrm{HM}\). В этом недостаток данного интерферометра. Но этот интерферометр незаменим при исследовании сверхтонкой структуры спектральных линий.

\section*{Задачи}
4.1. Зеркало Ллойда. В этой интерференционной схеме интерферируют световая волна 1 , исходящая непосредственно из источника \(S\) (узкой ярко освещенной щели), и волна 2 , отраженная от зеркала 3 (рис. 4.25). На экране Э образуется система интерференционных полос. Найти длину волны света, если известно, что расстояние от источника до экрана равно \(l\), ширина интерференционной полосы \(\Delta x\), а после того, как источник \(S\) отодвинули от плоскости зеркала на \(\Delta d\), ширина полос уменьшилась в \(\eta\) раз.

Решение. Согласно (4.6), расстояние \(d\) между источником \(S\) и его мнимым изображением \(S^{\prime}\) равно \(d=\lambda l / \Delta x\). После отодвигания источника \(S\) это расстояние стало

\[
d+2 \Delta d=\lambda l /(\Delta x / \eta)=\eta \lambda l / \Delta x .
\]

Вычтя первое равенство из второго, получим:

\[
\lambda=2 \Delta x \Delta d / l(\eta-1) .
\]

4.2. Интерферометр Рэлея. Его схема показана на рис. 4.26. Здесь \(S\) - узкая щель, освещаемая монохроматическим светом с длиной волны \(\lambda, 1\) и 2 - две одинаковые трубки с воздухом, длина каждой из которых равна \(l\), торцы - прозрачные, Д - диафрагма с двумя щелями. Когда воздух в трубке 1 постепенно заменили газом \(X\), то интерференционная картина на экране Э сместилась вверх на \(N\) полос. Зная показатель преломления \(n_{0}\) воздуха, определить показатель преломления \(n\) газа \(X\).

Р е ш е н и е. Смещение на \(N\) полос означает, что оптическая разность хода \(\Delta\) лучей, падающих на щели, стала равной \(N \lambda\), т. e. \(\ln -\ln _{0}=N \lambda\). Отсюда \(n=n_{0}+N \lambda / l\).

\begin{figure}[h]
\begin{center}
  \includegraphics[alt={},max width=\textwidth]{950fefba-d83b-4c8f-96e7-0bd5a7eb90be-127_150_476_1633_126}
\captionsetup{labelformat=empty}
\caption{Рис. 4.25}
\end{center}
\end{figure}

\begin{figure}[h]
\begin{center}
  \includegraphics[alt={},max width=\textwidth]{950fefba-d83b-4c8f-96e7-0bd5a7eb90be-127_133_526_1644_636}
\captionsetup{labelformat=empty}
\caption{Рис. 4.26}
\end{center}
\end{figure}

Смещение полос вверх свидетельствует о том, что и максимум нулевого порядка сместился вверх. При этом увеличение геометрической длины луча 2 компенсируется увеличением оптической длины луча 1.

Интерферометр Рэлея используется для измерения малых разностей показателей преломления прозрачных веществ (газов и жидкостей).\\
4.3. Интерференция плоских волн. Две одинаковые когерентные плоские световые волны, угол между направлениями распространения которых \(\varphi \ll 1\), падают почти нормально на экран. Показать, что расстояние между соседними максимумами на экране (ширина интерференционной полосы) \(\Delta x=\lambda / \varphi\), где \(\lambda\) - длина волны света.

Р е ш е н и е. Каждая плоская волна имеет вид

\[
E=A \cos (\omega t-\mathbf{k r}+\alpha),
\]

и отличаются они друг от друга только векторами \(\mathbf{k}_{1}\) и \(\mathbf{k}_{2}\), а также дополнительными фазами \(\alpha_{1}\) и \(\alpha_{2}\).\\
В точке экрана с радиусом-вектором \(\mathbf{r}\) разность фаз таких двух волн равна следующему выражению:

\[
\mathbf{k}_{1} \mathbf{r}-\mathbf{k}_{2} \mathbf{r}+\Delta \alpha=\Delta \mathbf{k} \cdot \mathbf{r}+\Delta \alpha,
\]

где \(\Delta \mathbf{k}=\mathbf{k}_{1}-\mathbf{k}_{2}\). Пусть в этой точке экрана образуется максимум, тогда переход к следующему максимуму будет определяться условием


\begin{equation*}
\Delta \mathbf{k} \cdot \Delta \mathbf{r}=2 \pi, \tag{1}
\end{equation*}


где \(\Delta \mathbf{k} \| \Delta \mathbf{r}\). Заменим эти векторы на их модули: \(|\Delta \mathbf{k}|=k \varphi=(2 \pi / \lambda) \varphi\) и \(|\Delta \mathbf{r}|=\Delta x\). В результате из формулы (1) получим \(\Delta x=\lambda / \varphi\).\\
Следует отметить, что приведенный расчет дает возможность найти только ширину \(\Delta x\) интерференционной полосы, но не порядок интерференции \(m\). Последнее же бывает во многих случаях весьма необходимым.\\
4.4. Бипризма Френеля. Найти выражения, определяющие условия для ширины \(s\) щели и степени монохроматичности \(\lambda / \Delta \lambda\), которые обеспечивали бы получение интерференционной картины на всей ширине зоны интерференции (в месте расположения экрана), причем с достаточно хорошей видностью. Расстояния от бипризмы Б до щели и экрана равны соответственно \(a\) и \(b\), преломляющий угол бипризмы \(\theta\), показатель преломления стекла \(n\).

Решение. Для выполнения указанных требований следует, согласно формулам (4.13) и (4.14), обеспечить должные значения длины и ширины когерентности, \(l_{\text {ког }}\) и \(h_{\text {ког }}\).\\
Для получения интерференционных полос надо, чтобы \(h_{\text {ког }}\) в месте расположения бипризмы (где волна расчленяется на две части) превышала вдвое расстояние \(d^{\prime}\) между лучами, которые затем сходятся вблизи центра интерференционной картины на экране. В этом случае складываемые колебания будут достаточно когерентны для создания интерференционных полос с хорошей видностью. Итак, надо чтобы \(h_{\text {ког }} \geqslant 2 d^{\prime}\).\\
Здесь \(h_{\text {ког }} \approx \lambda / \varphi=\lambda /(s / a), s\) - искомая ширина щели. Расстояние же \(d^{\prime}\) найдем с помощью рис. 4.27, откуда следует, что

\[
d^{\prime} / d=b /(a+b),
\]

где \(d=a \cdot 2 \alpha, \alpha\) - угол отклонения луча, после прохождения через бипризму - он одинаков для всех лучей: \(\alpha=(n-1) \theta\) (см. задачу 3.6).

После подстановки выражений для \(h_{\text {ког }}\) и \(d^{\prime}\) в исходную формулу получим

\[
\frac{\lambda a}{b} \geqslant 2 \frac{b}{a+b} a \cdot 2 \alpha,
\]

откуда

\[
s \leqslant \frac{\lambda}{4 \alpha}\left(1+\frac{a}{b}\right)
\]

Теперь второе условие - относительно \(\lambda / \Delta \lambda\). Согласно (4.13), необходимо, чтобы \(l_{\text {ког }} \geqslant 2 \Delta\). Здесь \(l_{\text {ког }} \approx \lambda^{2} / \Delta \lambda\), а \(\Delta\) - оптическая разность хода, которая должна соответствовать получению максимума наибольшего порядка - на краю зоны интерференции в месте расположения экрана. Пусть полуширина зоны интерференции в этом месте равна \(x_{m}\), тогда с помощью рис. 4.28 из подобия

\begin{figure}[h]
\begin{center}
  \includegraphics[alt={},max width=\textwidth]{950fefba-d83b-4c8f-96e7-0bd5a7eb90be-129_272_1103_1478_64}
\captionsetup{labelformat=empty}
\caption{Рис. 4.27}
\end{center}
\end{figure}

Рис. 4.28

треугольников получим:

\[
\Delta / d=x_{m} /(a+b)
\]

где \(d=a \cdot 2 \alpha, x_{m}=b \cdot \alpha\) (см. рис. 4.11). После подстановки выражений для \(l_{\text {ког }}\) и \(\Delta\) в исходную формулу получим:

\[
\frac{\lambda^{2}}{\Delta \lambda} \geqslant \frac{x_{m} d}{a+b}=4 \frac{a b}{a+b} \alpha^{2},
\]

откуда

\[
\frac{\lambda}{\Delta \lambda} \geqslant \frac{4 \alpha^{2}}{\lambda} \frac{a b}{a+b} .
\]

4.5. Интерференция при отражении от тонкой пленки. На поверхности стекла находится тонкая пленка воды. На нее падает свет с длиной волны \(\lambda\) под углом \(\vartheta\) к нормали. Найти скорость, с которой уменьшается толщина пленки (из-за испарения), если интенсивность отраженного света меняется так, что промежуток времени между последовательными максимумами отражения равен \(\Delta t\).

Решение. Из формулы (4.32) следует, что переход к следующему максимуму происходит при условии \(\Delta m=1\) и \(\Delta b=v \Delta t\). Отсюда \(v\) - искомая скорость:

\[
v=\frac{\lambda}{2 \Delta t \sqrt{n^{2}-\sin ^{2} \vartheta}},
\]

где \(n\) - показатель преломления воды.\\
4.6. Интерференция от клина. Свет с длиной волны \(\lambda\) от удаленного точечного источника падает нормально на поверхность стеклянного клина с малым углом раствора. В отраженном свете наблюдают систему интерференционных полос. Расстояние между соседними максимумами на поверхности клина равно \(\Delta x\). Найти:\\
a) угол между гранями клина;\\
б) длину когерентности, если исчезновение интерференционных полос наблюдается на расстоянии \(l\) от вершины клина ( \(l \gg \Delta x\) ).\\
Решение. а) При переходе к соседнему максимуму оптическая разность хода \(\Delta\) должна равняться \(\lambda\), т.е.

\[
2 \Delta x \theta n=\lambda .
\]

Здесь учтено, что угол клина весьма мал. Отсюда

\[
\theta=\lambda / 2 n \Delta x .
\]

б) Запишем условие исчезновения интерференционной картины ( \(l_{\text {ког }} \leqslant \Delta\) ), пренебрегая «потерей» полуволны, поскольку в нашем случае \(l \gg \Delta x\). Тогда

\[
l_{\text {ког }} \approx 2 l \theta n .
\]

Подставив в эту формулу выражение для \(\theta\) из предыдущего пункта, получим:

\[
l_{\text {ког }} \approx \lambda l / \Delta x .
\]

4.7. Кольца Ньютона. Сферическая поверхность плоско-выпуклой линзы соприкасается со стеклянной пластинкой. Пространство между линзой и пластинкой заполнено некоторой прозрачной жидкостью. Известны показатели преломления линзы \(n_{1}\), данной жидкости \(n_{2}\) и пластинки \(n_{3}\), причем \(n_{1}<n_{2}<n_{3}\). Радиус кривизны сферической поверхности линзы равен \(R\). Определить радиус \(N\)-го темного кольца в отраженном свете, длина волны которого \(\lambda\).

Р е ш е н и е. При таком соотношении между показателями преломления «потеря» полуволны (это ведь скачок фазы на \(\pi\) ) происходит при отражении от обеих поверхностей границы жидкость стекло, и они «гасят» друг друга. Поэтому условие образования \(m\)-го темного кольца выглядит так:

\[
2 b n_{2}=(m-1 / 2) \lambda, \quad m=1,2, \ldots N, \ldots
\]

Кроме того, учтем связь (4.35) между зазором \(b\) и радиусом кольца \(r\) :

\[
2 b R=r^{2}
\]

Исключив \(b\) из этих двух уравнений, получим:

\[
r_{N}=\sqrt{(N-1 / 2) \lambda R / n_{2}} .
\]

4.8. Плоско-выпуклая стеклянная линза с радиусом кривизны сфери-

\begin{figure}[h]
\begin{center}
  \includegraphics[alt={},max width=\textwidth]{950fefba-d83b-4c8f-96e7-0bd5a7eb90be-131_312_404_1402_201}
\captionsetup{labelformat=empty}
\caption{Рис. 4.29}
\end{center}
\end{figure}

Решении е. Здесь воспользоваться непосредственно формулой (4.36) мы не можем, так как не знаем порядков интерференции колец: они не совпадают с номерами колец. Но разности порядков интерференции равны разности номеров. Этим мы и воспользуемся. Условие образования темных колец (рис. 4.29), согласно (4.34), запишем как

\[
2(b+\Delta b)=m \lambda
\]

кроме того, геометрическая связь между \(b\) и \(r\), согласно (4.35),

\[
2 b R=r^{2}
\]

С помощью этих двух равенств находим:

\[
r_{2}^{2}-r_{1}^{2}=R\left(2 b_{2}-2 b_{1}\right)=R\left(m_{2}-m_{1}\right) \lambda=R\left(N_{2}-N_{1}\right) \lambda .
\]

Отсюда

\[
\lambda=\frac{d_{2}^{2}-d_{1}^{2}}{4\left(N_{2}-N_{1}\right) R}
\]

4.9. В двухлучевом интерферометре используется некоторая спектральная линия, состоящая из двух близких компонент с длинами волн \(\lambda_{1}=577\) нм и \(\lambda_{2}=579\) нм. При каком наименьшем порядке интерференции видность интерференционной картины будет наихудшей?

\begin{figure}[h]
\begin{center}
  \includegraphics[alt={},max width=\textwidth]{950fefba-d83b-4c8f-96e7-0bd5a7eb90be-132_244_661_1133_286}
\captionsetup{labelformat=empty}
\caption{Рис. 4.30}
\end{center}
\end{figure}

Решение. Изобразим расположение максимумов двух компонент (рис. 4.30). Из рисунка видно, что видность будет наихудшей, когда максимум компоненты \(\lambda_{2}\) окажется посередине между максимумами компоненты \(\lambda_{1}\). Это будет при условии

\[
m \lambda_{2}=m \lambda_{1}+\lambda_{1} / 2
\]

Отсюда

\[
m=\lambda_{1} / 2\left(\lambda_{2}-\lambda_{1}\right) \approx 140 .
\]

4.10. Интерферометр Майкельсона. В нем используют желтую линию натрия, состоящую из двух компонент с длинами волн, равными \(\lambda_{1}=589,0\) нм и \(\lambda_{2}=589,6\) нм. При поступательном перемещении одного из зеркал интерферометра интерференционная картина периодически исчезала (почему?). Найти перемещение зеркала, при котором последовательно появляются наиболее четкие интерференционные картины.\\
Решение. Условие перехода от одной четкой картины к следующей:

\[
(m+1) \lambda_{1}=m \lambda_{2}
\]

где \(m\) - некоторое целое число. При этом условии максимумы от \(\lambda_{1}\) и \(\lambda_{2}\) будут накладываться друг на друга.\\
Соответствующее перемещение \(\Delta b\) зеркала определяется уравнением

\[
2 \Delta b=m \lambda,
\]

где под \(\lambda\) можно понимать как \(\lambda_{1}\), так и \(\lambda_{2}\) (их различие здесь не существенно: оно слишком мало).\\
Из этих двух уравнений получим:

\[
\Delta b \approx \lambda^{2} / 2 \Delta \lambda,
\]

где \(\Delta \lambda=\lambda_{2}-\lambda_{1}\).

\section*{Глава 5}
\section*{Дифракция света}
\section*{§ 5.1. Принцип Гюйгенса-Френеля}
Введение. Под дифракцией света, как и других волновых процессов, понимают любое отклонение от прямолинейного распространения колебаний в среде с резкими неоднородностями (края экранов, отверстия и др.), что связано с отклонениями от законов геометрической оптики. Это приводит к огибанию световыми волнами препятствий и проникновению света в область геометрической тени.

Мы постоянно наблюдаем дифракцию звуковых волн, волн на поверхности воды, радиоволн. Для наблюдения же дифракции световых волн необходимы специальные условия, обусловленные малостью их длин волн \(\lambda\).

Наблюдение дифракции света проводят обычно по такой схеме. На пути световой волны помещают непрозрачную преграду, закрывающую часть световой волны. За преградой располагают экран, на котором при определенных условиях возникает дифракционная картина в виде той или иной системы полос и пятен - максимумов и минимумов освещенности. Исследование распределения интенсивности света на экране и будет являться основной нашей задачей, поскольку дает достаточно обширную информацию как о свойствах самой световой волны, так и о действии того или иного участка незакрытой части преграды (круглых отверстий, щелей и др.).

Первое объяснение дифракции света принадлежит Френелю (1818 г.). Он показал, что количественное описание дифракционных явлений возможно на основе построения Гюйгенса, если его дополнить принципом интерференции вторичных волн.

Вообще говоря, для описания дифракционных явлений не требуется вводить никаких новых принципов. В рамках электромагнитной теории света задача сводится к нахождению решения уравнений Максвелла при определенных граничных усло-

виях. Однако решение такой задачи представляет большие математические трудности. Поэтому в большинстве случаев, представляющих практический интерес, вполне достаточным оказывается приближенный метод решения задачи о распределении интенсивности света, основанный на принципе Гюйген-са-Френеля. Именно этот принцип и основанные на нем простые и наглядные методы расчета мы и возьмем за основу дальнейшего изложения.

Принцип Гюйгенса-Френеля. Этот принцип является основным постулатом волновой теории, описывающим и объясняющим механизм распространения волн, в частности световых. В чем его суть?

\begin{figure}[h]
\begin{center}
  \includegraphics[alt={},max width=\textwidth]{950fefba-d83b-4c8f-96e7-0bd5a7eb90be-135_310_552_679_340}
\captionsetup{labelformat=empty}
\caption{Рис. 5.1}
\end{center}
\end{figure}

Рассмотрим преграду \(N\) (не обязательно плоскую) с некоторым отверстием, через которое проходит свет от точечного монохроматического источника \(P_{0}\) (рис. 5.1). Задача состоит в определении напряженности \(E\) электрического поля в любой точке \(P\) за преградой.

В методе Френеля предполагается, что напряженность \(E\) в точках отверстия такова, как и при отсутствии преграды, и что в точках непосредственно за преградой \(E=0\). Т. е. считается, что существенна только форма отверстия преграды, но не сама преграда. Это предположение, как показал опыт, справедливо, когда размеры отверстия и расстояния до источника и точки наблюдения \(P\) значительно больше длины волны \(\lambda\), т. е. когда отклонения от геометрической оптики довольно малы. Оно нарушается для отверстия, например, щели, ширина которой значительно меньше \(\lambda\).

Перекроем мысленно отверстие в преграде произвольной поверхностью \(S\). Разобьем эту поверхность на элементарные уча-

стки \(\mathrm{d} S\). По предположению Френеля каждый из этих участков становится источником вторичной сферической волны. Амплитуда вторичной световой волны, достигающей интересующей нас точки \(P\), должна быть пропорциональна амплитуде первичной волны, приходящей к элементу \(d S\), а также площади самого элемента \(\mathrm{d} S\), и обратно пропорциональна расстоянию \(r\) от элемента \(\mathrm{d} S\) до точки \(P\).

В дальнейшем мы будем рассматривать только такие случаи, которые позволяют в качестве поверхности \(S\) брать волновую поверхность падающей волны (это значительно облегчает расчеты). В этих случаях от каждого элемента \(d S\) волновой поверхности распространяющаяся сферическая волна вызывает в точке \(P\) колебание

\[
\mathrm{d} E=K \frac{a_{0} \mathrm{~d} S}{r} \cos (\omega t-k r) .
\]

Здесь \(a_{0}\) - величина, определяемая амплитудой световой волны в месте нахождения элемента \(\mathrm{d} S, k\) - волновое число ( \(k=2 \pi / \lambda\) ). Коэффициент \(K\) зависит от угла \(\vartheta\) между нормалью \(\mathbf{n}\) к элементу \(\mathrm{d} S\) и направлением от \(\mathrm{d} S\) к точке \(P\). Естественно предположить, что коэффициент \(K\) монотонно убывает с ростом угла Э. Многие практически важные дифракционные задачи можно, как мы увидим далее, решить при весьма общих предположениях относительно зависимости \(K(\vartheta)\), не уточняя конкретного вида этой зависимости.

Результирующее колебание в точке \(P\) может быть представлено как суперпозиция колебаний \(\mathrm{d} E\) от всех элементов \(\mathrm{d} S\) поверхности \(S\), а именно


\begin{equation*}
E=\int_{S} K(\vartheta) \frac{a_{0}}{r} \cos (\omega t-k r) \mathrm{d} S \tag{5.1}
\end{equation*}


Интеграл (5.1) выражает собой математическую формулировку принципа Гюйгенса-Френеля. Суть этого принципа в следующем:

для определения колебания в точке \(P\), лежащей перед некоторой поверхностью \(S\), надо найти колебания, приходящие в эту точку от всех элементов \(\mathrm{d} S\) поверхности \(S\) и затем сложить их с учетом амплитуд и фаз.

При этом предполагается, что все элементы поверхности \(S\) взаимно когерентны - это необходимое условие для интерференции вторичных волн.

Принцип Гюйгенса-Френеля можно представить в простой и наглядной форме с помощью векторной

\begin{figure}[h]
\begin{center}
  \includegraphics[alt={},max width=\textwidth]{950fefba-d83b-4c8f-96e7-0bd5a7eb90be-137_247_406_379_68}
\captionsetup{labelformat=empty}
\caption{Рис. 5.2}
\end{center}
\end{figure}

(фазовой) диаграммы (рис. 5.2). Использование подобных диаграмм в дальнейшем позволит значительно упростить многие рассуждения и расчеты. На этой диаграмме результирующая амплитуда - вектор \(\mathbf{E}_{m}\) - представлена как векторная сумма амплитуд dE колебаний в точке \(P\) от различных элементов \(d S\) поверхности \(S\) с учетом их фаз, т. е. углов между ними.

\section*{§ 5.2. Дифракция Френеля от круглого отверстия}
Зоны Френеля. Суммирование (интегрирование) амплитуд элементарных колебаний, приходящих в точку \(P\), вообще говоря, весьма сложно. Но в простейших случаях, обладающих определенной симметрией, интегрирование, как показал Френель, может быть заменено простым алгебраическим или графическим сложением (последнее особенно наглядно).

Суммирование амплитуд колебаний, приходящих от различных элементов волновой поверхности \(S\), Френель предложил делать с помощью разбиения поверхности \(S\) на зоны, конфигурация которых зависит от симметрии рассматриваемой задачи.

\begin{figure}[h]
\begin{center}
  \includegraphics[alt={},max width=\textwidth]{950fefba-d83b-4c8f-96e7-0bd5a7eb90be-137_406_701_1371_264}
\captionsetup{labelformat=empty}
\caption{Рис. 5.3}
\end{center}
\end{figure}

Пользуясь методом Френеля, определим амплитуду световых колебаний в точке \(P\) за круглым отверстием на его оси (рис. 5.3). Волновая поверхность \(S\), которой мы перекроем отверстие, симметрична относительно прямой \(P_{0} P\), поэтому ее наиболее целесообразно разбивать на кольцевые зоны с центром на оси отверстия. Эти зоны выбираем так, чтобы расстояния от краев каждой зоны до точки \(P\) отличались друг от друга на половину длины волны, \(\lambda / 2\). Это и есть зонь Френеля в данном случае.

\begin{figure}[h]
\begin{center}
  \includegraphics[alt={},max width=\textwidth]{950fefba-d83b-4c8f-96e7-0bd5a7eb90be-138_362_1018_520_106}
\captionsetup{labelformat=empty}
\caption{Рис. 5.4}
\end{center}
\end{figure}

Найдем внешний радиус \(m\)-й зоны Френеля, \(r_{m}\). С этой целью воспользуемся рис. 5.4 , из которого видно, что отрезок \(C O\) равен


\begin{equation*}
h_{a}+h_{b}=m \lambda / 2 . \tag{5.2}
\end{equation*}


Выразим \(h_{a}\) и \(h_{b}\) через \(r_{m}\) и соответствующие радиусы \(a\) и \(b+m \lambda / 2\). Согласно теореме Пифагора, \(r_{m}^{2}=a^{2}-\left(a-h_{a}\right)^{2}\). Преобразовав правую часть этого равенства как разность квадратов, получим \(r_{m}^{2}=\left(2 a-h_{a}\right) h_{a}\). Обычно мы будем иметь дело со случаями, когда \(h_{a} \ll 2 a\), поэтому предыдущее равенство можно записать так:


\begin{equation*}
h_{a}=r_{m}^{2} / 2 a . \tag{5.3}
\end{equation*}


Рассуждая аналогично для правой части рис. 5.4, получим следующее выражение: \(r_{m}^{2}=(b+m \lambda / 2)^{2}-\left(b+m \lambda / 2-h_{b}\right)^{2}=\left(2 b+m \lambda-h_{b}\right) h_{b}\). Пренебрегая здесь в последней скобке слагаемыми \(m \lambda\) и \(h_{b}\) по сравнению с \(2 b\), приходим к выводу, что


\begin{equation*}
h_{b}=r_{m}^{2} / 2 b . \tag{5.4}
\end{equation*}


Остается подставить (5.3) и (5.4) в исходную формулу (5.2), и мы получим, что внешний радиус \(m\)-й зоны Френеля


\begin{equation*}
r_{m}=\sqrt{m \lambda \frac{a b}{a+b}} . \tag{5.5}
\end{equation*}


Заметим, что если падающая нормально на данное отверстие волна плоская ( \(a \rightarrow \infty\) ), то


\begin{equation*}
r_{m}=\sqrt{m \lambda b} \tag{5.6}
\end{equation*}


Если же падающая волна сферическая и сходящаяся в точке, расположенной за отверстием на расстоянии \(a\) от него, то в знаменателе (5.5) должно быть \(a-b\), если \(b<a\), или \(b-a\), если \(b>a\).

Чтобы иметь некоторое представление о порядках величин, с которыми приходится иметь дело при дифракции света, рассмотрим следующий пример.

Пример. На круглое отверстие радиуса \(r=1,0\) мм нормально падает плоская световая волна с \(\lambda=0,50\) мкм. Определим число зон Френеля, которые укладываются в этом отверстии для точки наблюдения \(P\), расположенной на оси отверстия и отстоящей от него на расстояние \(b=80 \mathrm{~cm}\).\\
Поскольку падающая волна плоская, следует воспользоваться формулой (5.6), откуда находим

\[
m=\frac{r^{2}}{\lambda b}=\frac{\left(10^{-3}\right)^{2}}{0,5 \cdot 10^{-6} \cdot 0,8}=2,5
\]

Площади зон (при достаточно малых \(m\) ) \(\Delta S=\pi r_{m}^{2}-\pi r_{m-1}^{2}\), или


\begin{equation*}
\Delta S=\pi \lambda \frac{a b}{a+b}, \tag{5.7}
\end{equation*}


т. е. практически одинаковы. Но амплитуды колебаний, приходящих в точку \(P\) от этих зон, монотонно и слабо убывают из-за увеличения расстояния \(r\) до точки \(P\) от каждой следующей зоны и роста угла \(\vartheta\) между нормалью к элементам зоны и направлением на точку \(P\).

Фазы колебаний, возбуждаемых в точке \(P\) соседними зонами, отличаются на \(\pi\), поэтому векторы-амплитуды нечетных

зон противоположны по направлению векторам-амплитудам от четных зон. И результирующая амплитуда, а значит и интенсивность, зависит от того, четное или нечетное число \(m\) зон Френеля умещается в отверстии - для точки наблюдения \(P\). Если число зон нечетное, в точке \(P\) наблюдается максимум, если же число зон четное, то - минимум.

Число зон \(m\) в отверстии мы можем изменять. Например, для увеличения числа зон надо или расширить отверстие, или приблизить экран к нему, или то и другое вместе. Это непосредственно вытекает из формулы (5.5), если под \(r_{m}\) понимать радиус отверстия.

Спираль Френеля. Рассмотрим графический метод сложения амплитуд. В этом простом и наглядном методе волновую поверхность мысленно разбивают на весьма узкие кольцевые зоны. Амплитуду колебаний, создаваемых каждой из таких зон, изобразим вектором dA. Вследствие увеличения расстояния \(r\) и уменьшения коэффициента \(K\) амплитуда колебаний, создаваемых каждой следующей узкой кольцевой зоной, будет убывать по модулю и отставать по фазе от колебаний, создаваемых предыдущей зоной. Изобразив отставание по фазе поворотом каждого вектора dA против часовой стрелки на соответствующий угол, получим цепочку векторов, векторная сумма которых и есть результирующая амплитуда колебаний в точке \(P\).

На рис. 5.5, \(a\) показан результат действия 1-й зоны Френеля. Здесь амплитуда колебаний \(\mathrm{d} A_{N}\) от узкого кольца, прилегающего к границе 1-й зоны Френеля, отстает по фазе на \(\pi\) от амплитуды колебаний, приходящих в точку \(P\) из центра 1-й зоны - от \(\mathrm{d} A_{1}\), поэтому соответствующие этим амплитудам векторы взаимно противоположны по направлению.

\begin{figure}[h]
\begin{center}
  \includegraphics[alt={},max width=\textwidth]{950fefba-d83b-4c8f-96e7-0bd5a7eb90be-140_383_1018_1411_106}
\captionsetup{labelformat=empty}
\caption{Рис. 5.5}
\end{center}
\end{figure}

Продолжая построение, получим векторную диаграмму для результирующей амплитуды колебаний в точке \(P\) от действия

\begin{figure}[h]
\begin{center}
  \includegraphics[alt={},max width=\textwidth]{950fefba-d83b-4c8f-96e7-0bd5a7eb90be-141_289_279_246_72}
\captionsetup{labelformat=empty}
\caption{Рис. 5.6}
\end{center}
\end{figure}

первых двух зон Френеля (рис. 5.5, б), затем от первых трех зон Френеля (рис. 5.5, в) и т. д. Цепочка по мере увеличения числа узких кольцевых зон будет «закручиваться» в спираль, и в результате амплитуда от действия всех зон (всей волновой поверхности) будет равна \(A_{\infty}\) (рис. 5.6). Эту спираль назовем спиралью Френеля (в отличие от другой спирали, с которой мы встретимся в следующем параграфе).

Забегая вперед, отметим, что дифракция Френеля связана с действием лишь нескольких первых витков спирали (более подробно об этом в § 5.4).

Таким образом, амплитуда колебаний и интенсивность света в точке \(P\) (см. рис. 5.3) по мере увеличения радиуса отверстия в преграде изменяется не монотонно. Пока открывается первая зона Френеля, амплитуда в точке \(P\) увеличивается и достигает максимума при полностью открытой зоне (см. рис. 5.5, a). Но по мере открывания второй зоны Френеля амплитуда колебаний в точке \(P\) убывает, и при полностью открытых двух первых зонах уменьшается почти до нуля (см. рис. 5.5, б). Затем амплитуда увеличивается снова (рис. 5.5, в) и т. д.

То же самое будет наблюдаться, если вместо увеличения отверстия приближать к нему точку наблюдения \(P\) вдоль прямой \(P_{0} P\) (см. рис. 5.3). Это легко понять из данного рисунка: при этом число открываемых зон Френеля в отверстии преграды \(N\) будет увеличиваться.

На первый взгляд эти результаты, предсказанные на основе принципа Гюйгенса-Френеля, выглядят парадоксальными. Однако они хорошо подтверждаются опытом. В то же время согласно геометрической оптике интенсивность света в точке \(P\) не должна зависеть от радиуса отверстия.

Итак, амплитуда колебаний в точке \(P\) от полностью открытой волновой поверхности, согласно представлениям Френеля, равна \(A_{\infty}=A_{1} / 2\), т. е. интенсивность ( \(I \sim A^{2}\) ) в четыре раза меньше, чем при наличии преграды с круглым отверстием, открывающем только 1 -ю зону Френеля.

Особенно неожиданным в методе Френеля представляется тот удивительный вывод, что при отверстии в преграде, открывающем для точки \(P\) две зоны Френеля, интенсивность в этой точке падает практически до нуля, хотя световой поток через отверстие оказывается вдвое больше.

Пятно Пуассона. Не менее неожиданно и то, что за круглым непрозрачным диском в центре его геометрической тени интенсивность не равна нулю. Если диск перекрывает лишь несколько зон Френеля, то интенсивность в центре геометрической тени почти такая же, как при отсутствии диска. Это непосредственно следует из спирали Френеля (рис. 5.7), поскольку если диск закрывает, скажем, 1,5 зоны Френеля, то результирующий вектор \(\mathbf{A}_{\infty}\) при полностью открытой волновой поверхности можно представить как сумму двух векторов: \(\mathbf{A}_{\infty}=\mathbf{A}_{1,5}+\mathbf{A}_{\text {ост }}\). Так как первые

\begin{figure}[h]
\begin{center}
  \includegraphics[alt={},max width=\textwidth]{950fefba-d83b-4c8f-96e7-0bd5a7eb90be-142_362_350_506_808}
\captionsetup{labelformat=empty}
\caption{Рис. 5.7}
\end{center}
\end{figure}

полторы зоны закрыты, то остается только вектор \(\mathbf{A}_{\text {ост }}\) - от всех остальных зон. Этот вектор по модулю лишь немного меньше вектора \(\mathbf{A}_{\infty}\).

Это светлое пятно в центре геометрической тени называют пятном Пуассона. Рассматривая в свое время метод Френеля, Пуассон пришел к выводу, что в центре тени от диска должно быть светлое пятно, но счел этот вывод столь абсурдным, что выдвинул его как убедительное возражение против волновой теории, развиваемой Френелем. Однако это «абсурдное» предсказание было экспериментально подтверждено Арагоном. Волновая теория Френеля восторжествовала.

Зонная пластинка. Если в преграде открыть только нечетные зоны Френеля ( 1 -ю, 3 -ю,...), то векторы-амплитуды от этих зон будут сонаправлены и в сумме дадут вектор, во много раз превосходящий по модулю векторы \(\mathbf{A}_{\infty}\) и \(\mathbf{A}_{1}\). Такую систему называют зонной пластинкой. Аналогично можно изготовить зонную пластинку, где открыты только четные зоны Френеля.

Зонная пластинка, содержащая \(n\) открытых зон, создает в точке \(P\) интенсивность приблизительно в \(n^{2}\) раз бо́льшую, чем отверстие в первую зону Френеля.

Усиление интенсивности света зонной пластинкой эквивалентно фокусирующему действию линзы. Расстояния от зонной пластинки до источника \(P_{0}\) и его «изображения» \(P\) связаны таким же соотношением, как и соответствующие расстояния для линзы. Чтобы в этом убедиться, достаточно переписать формулу (5.5) в виде


\begin{equation*}
\frac{1}{a}+\frac{1}{b}=\frac{m \lambda}{r_{m}^{2}} \tag{5.8}
\end{equation*}


где выражение в правой части равенства можно рассматривать как \(1 / f, f\) - фокусное расстояние: \(f=r_{m}^{2} / m \lambda=r_{1}^{2} / \lambda\), поскольку \(r_{m}^{2} \sim m\). Но в отличие от линзы, зонная пластинка - система не таутохронная: колебания, приходящие в фокус \(F\) от соседних открытых зон, различаются по фазе на \(2 \pi\) (разность хода \(\lambda\) ). Кроме этого фокуса (основного), зонная пластинка имеет и другие, а именно те точки \(F^{\prime}\), в которые колебания от соседних открытых зон приходят с разностью хода \(2 \lambda\), \(3 \lambda\) и т. д. Эти другие фокусы оказываются более слабыми по сравнению с основным.

Интенсивность света в главном фокусе \(F\) зонной пластинки можно увеличить еще в четыре раза, если изменить на \(\pi\) фазы вторичных волн, исходящих из всех зон Френеля с четными (или нечетными) номерами. Тогда векторы-амплитуды от всех зон будут сонаправлены и результирующая амплитуда возрастет еще вдвое. Такая пластинка была изготовлена Вудом путем травления в соответствующих зонах тонкого лакового покрытия. Ее действие вполне эквивалентно действию линзы, так как в обоих случаях вторичные волны от всех точек волновой поверхности приходят в точку \(F\) в одинаковых фазах.

Дополнительные замечания. Они касаются как самой спирали Френеля в качестве рабочего инструмента, так и вида дифракционной картины в зависимости от радиуса отверстия.

\begin{enumerate}
  \item При решении некоторых вопросов, если дело ограничивается первым полувитком спирали Френеля, т. е. первой зоной, и мы не претендуем на особую точность результатов, то вполне достаточно первый виток принимать за окружность. Погрешность будет при этом для многих случаев несущественной.
\end{enumerate}

Пример. Плоская световая волна интенсивности \(I_{0}\) падает нормально на преграду, в которой для точки наблюдения \(P\) открыты только внутренняя половина 1 -й зоны Френеля и вторая (внешняя) половина 2 -й зоны Френеля. Найдем интенсивность света в точке \(P\).\\
Принимая первый виток спирали Френеля за окружность, построим соответствующие векторы \(\mathbf{A}_{1}\) и \(\mathbf{A}_{2}\) (рис. 5.8), и ответ очевиден: \(A=2 A_{\infty}\), откуда \(I=4 I_{0}\).

\begin{figure}[h]
\begin{center}
  \includegraphics[alt={},max width=\textwidth]{950fefba-d83b-4c8f-96e7-0bd5a7eb90be-144_336_348_231_808}
\captionsetup{labelformat=empty}
\caption{Рис. 5.8}
\end{center}
\end{figure}

\begin{enumerate}
  \setcounter{enumi}{1}
  \item Метод зон Френеля позволяет сравнительно просто найти интенсивность света только в точке \(P\), лежащей на оси круглого отверстия в преграде. Расчет же распределения интенсивности для всей дифракционной картины значительно сложнее. Вся картина обладает круговой симметрией и представляет собой чередующиеся светлые и темные кольца, плавно переходящие друг в друга.
\end{enumerate}

Если в отверстии преграды укладывается 1-я зона Френеля или ее часть, то интенсивность максимальна в центре картины (т. е. в точке \(P\) ) и монотонно убывает при удалении от точки \(P\). Если отверстие открывает две первые зоны Френеля, то в окрестности точки \(P\) возникает темное круглое пятно, а вокруг него - светлое кольцо. С увеличением числа \(m\) открытых зон в отверстии преграды увеличивается и число светлых и темных колец. На рис 5.9 показано распределение интенсивности \(I\) от расстояния \(r\) до центра дифракционной картины при различном числе \(m\) открытых зон Френеля. Когда же в отверстии укладывается большое число зон Френеля, интенсивность вблизи точки \(P\) оказывается почти равномерной и лишь у краев геометрической тени отверстия наблюдается чередование весьма узких светлых и темных кольцевых полос.

\begin{figure}[h]
\begin{center}
  \includegraphics[alt={},max width=\textwidth]{950fefba-d83b-4c8f-96e7-0bd5a7eb90be-144_220_1104_1567_60}
\captionsetup{labelformat=empty}
\caption{Рис. 5.9}
\end{center}
\end{figure}

Продемонстрируем на конкретном примере возможности спирали Френеля (см. рис. 5.6) при дифракции от некоторых объектов, не обладающих круговой симметрией.

\begin{figure}[h]
\begin{center}
  \includegraphics[alt={},max width=\textwidth]{950fefba-d83b-4c8f-96e7-0bd5a7eb90be-145_269_496_290_350}
\captionsetup{labelformat=empty}
\caption{Рис. 5.10}
\end{center}
\end{figure}

Пример. Плоская световая волна интенсивности \(I_{0}\) падает нормально на преграды, показанные на рис. 5.10. Найдем интенсивность света в точке \(P\), расположенной за преградами, если в случае а) открыто \(3 / 4\) волновой поверхности, а в случае б) закругленная часть преграды является границей 1 -й зоны Френеля для точки \(P\).

Прежде всего ясно, что в этих случаях дифракционная картина (чередующиеся светлые и темные полосы) не будет обладать круговой симметрией, она значительно сложнее. Но расчет для указанной точки \(P\) оказывается достаточно простым. Действительно, в случае а) от каждой зоны Френеля будет действовать только \(3 / 4\) ее части, следовательно, амплитуда в точке \(P\) будет \(A=(3 / 4) A_{\infty}\) и интенсивность \(I=(3 / 4)^{2} I_{0}\).\\
В случае же б) результирующая амплитуда \(A\) в точке \(P\) будет равна разности амплитуды от первой зоны Френеля \(A_{1}\) и амплитуды от всех остальных зон \((3 / 4) A_{\text {ост }}\), где \(A_{\text {ост }} \approx A_{\infty}\). Подчеркнем: именно разности, поскольку обеим амплитудам на спирали Френеля (см. рис. 5.6) соответствуют векторы, противоположно направленные, т. е. имеющие разность фаз \(\pi\). Итак, результирующая амплитуда в точке \(P\) равна

\[
A=A_{1}-(3 / 4) A_{\text {ост }}=2 A_{\infty}-(3 / 4) A_{\infty}=(5 / 4) A_{\infty} .
\]

Отсюда интенсивность \(I=(5 / 4)^{2} I_{0}\).\\
Замечания о методе Френеля. Вычисления, выполненные на основе принципа Гюйгенса-Френеля, дают, как показывает опыт, правильное распределение интенсивности при дифракции, т. е. позволяют найти правильное значение амплитуды результирующей волны в любой точке экрана, если размеры отверстий или препятствий не оказываются слишком малыми (сравнимыми с длиной волны \(\lambda\) ), другими словами, при не очень больших углах дифракции.

При этом, однако, в методе расчета Френеля есть принципиальные неясности. Главные из них заключаются в следующем.

\begin{enumerate}
  \item При вычислении результатов интерференции элементарных волн приходится считать, что амплитуда колебаний от элементов \(d S\) волновой поверхности зависит от угла \(\vartheta\) между нормалью к элементу \(d S\) и направлением на точку \(P\), для которой ведется расчет. Амплитуда максимальна при \(\vartheta=0\) и монотонно убывает до нуля при стремлении \(\vartheta ~ к ~ π / 2\), т. е. нет «обратной» волны. Это обстоятельство остается не обоснованным в теории Френеля.
  \item Расчет по методу Френеля дает неправильное значение фазы результирующего колебания. Для полностью открытой волновой поверхности она отличается на \(\pi / 2\) от действительной. Это видно из рис. 5.6. Направление спирали Френеля в ее начале дает в точке наблюдения фазу колебаний от центрального элемента первой зоны. Это и есть то значение фазы, которое соответствует действительности. Результирующий же вектор от полностью открытой волновой поверхности повернут на \(\pi / 2\) против часовой стрелки, т. е. отстает по фазе на \(\pi / 2\). Таким образом, постулат Френеля, правильно задавая амплитуды вспомогательных источников, неудачно определяет их фазы.
\end{enumerate}

Для большинства задач вопрос о фазе не имеет значения, ибо нас интересует интенсивность результирующей волны, которая пропорциональна квадрату амплитуды. Значение же последней метод Френеля дает правильное.

Итак, несмотря на некоторые недостатки, метод Френеля в вопросах расчета интенсивности волн для многих случаев является весьма плодотворным \({ }^{*}\).

\section*{§ 5.3. Дифракция Френеля от полуплоскости и щели}
В предыдущем параграфе мы рассматривали с помощью принципа Гюйгенса-Френеля дифракцию сферической волны от круглого отверстия. Осевая симметрия задачи подсказывала выбор конфигурации зон, на которые мы разбивали открытую часть волновой поверхности - в виде круговых колец. Теперь перейдем к случаю, когда волновая поверхность плоская и характер препятствия (полуплоскость, щель) предписывает разбивать открытую часть волновой поверхности на зоны в виде прямолинейных полосок.

\footnotetext{\begin{itemize}
  \item Впоследствии Кирхгоф дал более строгую формулировку принципу Гюйген-са-Френеля, устраняющую формально почти все недостатоки теории Френеля.
\end{itemize}
}\begin{figure}[h]
\begin{center}
  \includegraphics[alt={},max width=\textwidth]{950fefba-d83b-4c8f-96e7-0bd5a7eb90be-147_376_904_147_163}
\captionsetup{labelformat=empty}
\caption{Рис. 5.11}
\end{center}
\end{figure}

Дифракция от прямолинейного края полуплоскости. Пусть на экран Э падает нормально плоская монохроматическая волна длины \(\lambda\). Расположим перед экраном на расстоянии \(l\) от него непрозрачную полуплоскость \(N\) с прямолинейным краем (рис. 5.11). Если бы свет распространялся прямолинейно, то на экране Э мы наблюдали резкую тень от края этой полуплоскости (точка \(P_{0}\) на рисунке). В действительности же из-за волнового характера света на экране Э образуется сложная дифракционная картина.

Для расчета этой картины воспользуемся принципом Гюй-генса-Френеля. В данном случае в качестве интересующей нас волновой поверхности \(S\) возьмем ту открытую ее часть, которая продолжает непрозрачную полуплоскость. Соответствующие расчеты были проведены аналитически Френелем, получены результаты в виде так называемых интегралов Френеля, и задача, таким образом, была решена.

Мы не будем воспроизводить здесь этот расчет и ограничимся лишь интерпретацией его и полученного результата с помощью векторной диаграммы. Это наиболее простой и наглядный метод, открывающий к тому же весьма эффективные практические применения.

Из соображений симметрии ясно, что дифракционная картина на экране Э будет зависеть только от расстояния до границы геометрической тени - точки \(P_{0}\) на рис. 5.11, т. е. светлые и темные полосы должны быть параллельны прямолинейному краю \(K\) непрозрачной полуплоскости \(N\). Говоря далее об амплитуде колебаний в точке \(P\) на экране, мы будем иметь в виду, что это относится ко всем точкам прямой, проходящей через точку \(P\) и параллельной краю полуплоскости.

Сначала найдем амплитуду колебаний в точке \(P_{0}\), которая находится на краю геометрической тени (рис. 5.11, \(a\) ). Для этой точки (и только для нее) мы могли бы использовать разбиение открытой части волновой поверхности на полукольца - полузоны Френеля. Но поскольку нам предстоит определять амплитуду колебаний и в других точках экрана, то в соответствии с симметрией данной задачи разобьем мысленно открытую часть волновой поверхности \(S\) на весьма узкие одинаковой ширины прямолинейные полоски (зоны), параллельные краю полуплоскости.

Амплитуду колебаний, приходящих в точку \(P_{0}\) от первой зо-ны-полоски изобразим вектором \(\mathrm{dA}_{1}\) (рис. 5.12). Амплитуду колебаний от следующей полоски - вектором \(\mathrm{d} \mathbf{A}_{2}\), повернутым на очень небольшой угол против часовой стрелки, так как эти колебания проходят до точки \(P_{0}\) несколько большее расстояние и, значит, отстают по фазе.

В дальнейшем угол между соседними векторами-амплитудами становится все больше, поскольку запаздывание по фазе колебаний, приходящих в точку \(P_{0}\) от последующих зон-полосок растет все больше. Модули же векторов \(\mathrm{d} \mathbf{A}_{i}\) будут умень-

\begin{figure}[h]
\begin{center}
  \includegraphics[alt={},max width=\textwidth]{950fefba-d83b-4c8f-96e7-0bd5a7eb90be-148_469_328_681_828}
\captionsetup{labelformat=empty}
\caption{Рис. 5.12}
\end{center}
\end{figure}

шаться (из-за увеличения расстояния до \(P_{0}\) и угла \(\vartheta\) между нормалью к полоске и направлением на точку \(P_{0}\) ).

Результирующая амплитуда колебаний в точке \(P_{0}\) от достаточно широкой полосы волновой поверхности \(S\) изобразится суммой (цепочкой) векторов \(\mathrm{d} \mathbf{A}_{i}\) от всех укладывающихся на этой полосе элементарных зон-полосок. Это вектор \(\mathbf{A}\) на рис. 5.12.

Спираль Корню. В пределе, когда ширина полосы стремится к бесконечности, т. е. превращается в полуплоскость, и ширина каждой элементарной зоны-полоски стремится к нулю, цепочка векторов превращается в плавную кривую, являющуюся правой половиной спирали Корню (рис. 5.13). Эта спираль состоит из двух симметричных ветвей, закручивающихся вокруг фокусов \(F_{1}\) и \(F_{2}\). Ее левая половина описывает действие колебаний, приходящих в точку \(P_{0}\) от участков волновой поверхности (если бы

\begin{figure}[h]
\begin{center}
  \includegraphics[alt={},max width=\textwidth]{950fefba-d83b-4c8f-96e7-0bd5a7eb90be-149_577_659_159_69}
\captionsetup{labelformat=empty}
\caption{Рис. 5.13}
\end{center}
\end{figure}

они были открыты), лежащих левее края \(K\) непрозрачной полуплоскости (см. рис. 5.11, a).

Амплитуда колебаний в точке \(P_{0}\) от волновой поверхности, лежащей правее края \(K\) непрозрачной полуплоскости, изобразится вектором, проведенным из точки \(O\) в фокус \(F_{2}\) спирали Корню. Амплитуда же колебаний в точке \(P_{0}\) от полностью открытой волновой поверхности - вектором, проведенным из фокуса \(F_{1}\) в фокус \(F_{2}\).

Для нахождения вектора-амплитуды колебаний в точке \(P\), лежащей, например, правее точки \(P_{0}\) (см. рис. 5.11, б), от ка-кой-либо полосы волновой поверхности, лежащей между координатами \(x_{1}\) и \(x_{2}\), нужно построить вектор, который замыкает соответствующий этой полосе участок спирали Корню.

Это делается так. Каждой точке спирали Корню соответствует определенное значение некоторого параметра \(s\) (он пропорционален длине дуги спирали, отсчитываемой от точки \(O\) на рис. 5.13). Значения параметра указаны вдоль кривой. Из аналитического расчета следует, что параметр \(s\) связан с расстоянием \(x\), отсчитываемьлм от точки \(C\) до интересующей нас точки \(D\) волновой поверхности \(S\) (рис. 5.14) формулой

\begin{figure}[h]
\begin{center}
  \includegraphics[alt={},max width=\textwidth]{950fefba-d83b-4c8f-96e7-0bd5a7eb90be-149_323_436_1436_66}
\captionsetup{labelformat=empty}
\caption{Рис. 5.14}
\end{center}
\end{figure}


\begin{equation*}
s=x \sqrt{\frac{2}{l \lambda}}, \tag{5.9}
\end{equation*}


где \(\lambda\) - длина волны света, \(l\) - расстояние между экраном Э и волновой поверхностью \(S\), в плоскости которой расположено то или иное препятствие на пути световой волны.

Обратим внимание на то, что параметр \(s\) пропорционален расстоянию \(x\). Значит, \(x\) っ \(s\) っ длине дуги спирали Корню, отсчитываемой от точки \(O\) (см. рис. 5.13) в соответствующую сторону (вправо или влево).

Теперь покажем как с помощью спирали Корню получить распределение интенсивности света на экране вблизи края геометрической тени при дифракции плоской волны от прямолинейного края непрозрачной полуплоскости \(N\). Если точка \(P\) находится правее точки \(P_{0}\) (см. рис. 5.11, б), то правая часть волновой поверхности \(S\) (от точки \(C\) ) полностью открыта, и на спирали Корню амплитуда колебаний в точке \(P\) соответствует вектору \(D F_{2}\). Конец этого вектора находится в фокусе \(F_{2}\), а начало - точка \(D\) - в зависимости от положения точки \(P\). Когда \(P\) находится на краю геометрической тени (в точке \(P_{0}\) ), точка \(D\) совпадает с точкой \(O\) на спирали Корню (см. рис. 5.13), и век-тор-амплитуда соответствующих колебаний изобразится вектором \(O F_{2}\), равным половине вектора \(F_{1} F_{2}\) - от полностью открытой волновой поверхности \(S\). Поэтому интенсивность света в точке \(P_{0}\) в четыре раза меньше интенсивности \(I_{0}\) в отсутствие непрозрачной полуплоскости.

При перемещении точки \(P\) вправо от точки \(P_{0}\) точка \(D\) на спирали Корню (начало вектора \(D F_{2}\) ) будет перемещаться по левой ветви спирали, так как слева от точки \(C\) будут открываться все новые зоны-полоски. Это приводит к тому, что амплитуда и интенсивность в точке \(P\) при удалении ее от \(P_{0}\) будут последовательно проходить через максимумы и минимумы, различие между которыми постепенно уменьшается и интенсивность приближается к значению \(I_{0}\) (рис. 5.15).

При перемещении точки \(P\) влево от точки \(P_{0}\) - в область геометрической тени, точка \(D\) на спирали Корню перемещается

\begin{figure}[h]
\begin{center}
  \includegraphics[alt={},max width=\textwidth]{950fefba-d83b-4c8f-96e7-0bd5a7eb90be-150_327_632_1460_298}
\captionsetup{labelformat=empty}
\caption{Рис. 5.15}
\end{center}
\end{figure}

вправо от точки \(O\). Легко видеть, что длина вектора \(D F_{2}\), а значит и интенсивность, будет при этом монотонно уменьшаться до нуля (см. рис. 5.15).

Теперь покажем на конкретном примере как просто с помощью спирали Корню и формулы (5.9) решаются вопросы, связанные с распределением интенсивности при дифракции света от края непрозрачной полуплоскости.\\
Пример. Дифракцию плоской волны от края непрозрачной полуплоскости наблюдают на экране Э, отстоящем от полуплоскости \(N\) на расстояние \(l=100 \mathrm{~cm}\). Длина волны света \(\lambda=500\) нм. Найдем расстояние \(\Delta x\) между первыми двумя максимумами на экране Э и интенсивность первого максимума, если интенсивность падающего света равна \(I_{0}\).\\
Согласно формуле (5.9)

\[
\begin{gathered}
\Delta x=x_{2}-x_{1}=\left(s_{2}-s_{1}\right) \sqrt{l \lambda / 2}= \\
=(2,3-1,2) \sqrt{1 \cdot 500 \cdot 10^{-9} / 2}=5,5 \cdot 10^{-4} \mathrm{M}=0,55 \text { мм. }
\end{gathered}
\]

С помощью рис. 5.13 и линейки находим, что отношение амплитуды 1 -го максимума, т. е. расстояния между точками \(G\) и \(F_{2}\), к амплитуде падающего света \(F_{1} F_{2}\) равно \(\eta \approx 1,175\). Следовательно, интенсивность 1 -го максимума

\[
I_{1}=\eta^{2} I_{0}=1,37 I_{0} .
\]

Следует отметить, что обычно точка наблюдения \(P\) в лабораторных установках находится за непрозрачной полуплоскостью на расстоянии, не превышающем порядка одного метра. При этом для определения амплитуды результирующего колебания играет роль сравнительно небольшой участок волновой поверхности \(S\), лежащий вблизи края полуплоскости. В таких условиях край практически любого препятствия можно считать прямолинейным и для расчета дифракционной картины успешно пользоваться спиралью Корню.

Дифракция от щели. Таким же образом - с помощью спирали Корню и формулы (5.9) - можно рассчитать распределение интенсивности в дифракционной картине от бесконечно длинной прямолинейной щели. Сама дифракционная картина на экране имеет симметричный относительно середины вид чередующихся светлых и темных полос, параллельных щели (предполагается, что плоская световая волна падает на щель нормально).

\begin{figure}[h]
\begin{center}
  \includegraphics[alt={},max width=\textwidth]{950fefba-d83b-4c8f-96e7-0bd5a7eb90be-152_268_1099_166_66}
\captionsetup{labelformat=empty}
\caption{Рис. 5.16}
\end{center}
\end{figure}

С помощью той же спирали Корню легко убедиться в том, что при постепенном увеличении ширины щели интенсивность в середине дифракционной картины будет сначала иметь максимум, затем минимум, потом опять максимум и т. д. (рис. 5.16, \(a\), б, в). Таким образом, мы будем наблюдать при этом последовательное чередование максимумов и минимумов (в середине картины), разность между которыми будет постепенно уменьшаться, стремясь к интенсивности \(I_{0}\) падающего на щель света. Сама же дифракционная картина будет постепенно локализовываться только вблизи геометрической тени от краев щели.

Пример. На прямолинейную щель в непрозрачной преграде падает нормально плоская световая волна (рис. 5.17). Длина волны \(\lambda=0,60\) мкм. Расстояние от щели до экрана Э равно \(l=50 \mathrm{~cm}\). Найдем минимальную ширину \(b\) щели, при которой в точке \(P\) интенсивность будет максимальной.\\
Это означает, что амплитуда колебаний в точке \(P\) должна быть максимальной, т. е. соответствующей рис. 5.16, \(a\). При этом параметр \(s\) должен быть равным (см. рис. 5.13) \(s \approx 1,3\). Согласно формуле (5.9) этому значению параметра \(s\) отвечает расстояние \(x=s \sqrt{l \lambda / 2}=0,5\) мм. Значит, искомая ширина щели

\begin{figure}[h]
\begin{center}
  \includegraphics[alt={},max width=\textwidth]{950fefba-d83b-4c8f-96e7-0bd5a7eb90be-152_295_448_1179_715}
\captionsetup{labelformat=empty}
\caption{Рис. 5.17}
\end{center}
\end{figure}

\(b=2 x=1\) мм.

Отметим, что в отличие от спирали Френеля, которая давала возможность решать вопросы об интенсивности только в одной точке дифракционной картины, спираль Корню позволяет в ряде случаев находить распределение интенсивности во всех точках дифракционной картины.

\section*{§ 5.4. Дифракция Фраунгофера}
Дифракция Фраунгофера. Фраунгофер предложил иной способ наблюдения дифракции, получивший значительно большее практическое применение в оптике, поскольку приводит к более простым закономерностям (формулам). В этом способе на дифракционный объект (отверстие, щель и др.) направляют параллельный пучок света (плоскую волну) и дифракционную картину наблюдают на достаточно большом расстоянии, т. е. практически в параллельных лучах. Это и есть дифракция Фраунгофера или дифракция в параллельных лучах.

Есть критерий, позволяющий судить, с каким видом дифракции - френелевой или фраунгоферовой - мы имеем дело в каждом конкретном случае. Чтобы его получить, воспользуемся формулой (5.6), т. е. \(r_{m}=\sqrt{m \lambda b}\). Напомним, эта формула относится к случаю, когда на отверстие радиуса \(r_{m}\) падает нормально плоская световая волна, причем \(m\) означает число зон Френеля, которые укладываются в данном отверстии для точки наблюдения \(P\), отстоящей от отверстия на расстояние \(b\). Из этой формулы следует, что \(m=r_{m}^{2} / \lambda b\). Там же было отмечено, что характер дифракционной картины определяется только числом \(m\) открытых зон Френеля, и ничем другим. Значит, последнее выражение для \(m\) и можно взять в качестве интересующего нас параметра \(p\), заменив в этом выражении \(r_{m}\) на некоторый характерный размер \(h\) отверстия в преграде и \(b\) на \(l\).

Таким образом, безразмерный параметр \(p\) определяется как


\begin{equation*}
p=h^{2} / l \lambda, \tag{5.10}
\end{equation*}


где \(h\) - некоторый характерный размер: радиус или диаметр (это не существенно) круглого отверстия, или, например, ширина щели и т. п.

Значение именно этого безразмерного параметра и определяет характер дифракции:


\begin{align*}
& p \ll 1 \text { - дифракция Фраунгофера, } \\
& p \sim 1 \text { - дифракция Френеля, }  \tag{5.11}\\
& p \gg 1 \text { - приближение геометрической оптики. }
\end{align*}


При этом предполагается, что световая волна падает нормально на отверстие в преграде.

Пример. Выясним, с каким видом дифракции (френелевой или фраунгоферовой) мы имеем дело, если параллельный пучок света с длиной волны \(\lambda=0,6\) мкм падает нормально на круглое отверстие диаметром \(D=1,0\) мм, образуя дифракционную картину на экране, отстоящем от отверстия на расстояние \(l=50 \mathrm{~cm}\). В этом случае параметр

\[
p=\frac{D^{2}}{l \lambda}=\frac{\left(10^{-3}\right)^{2}}{0,5 \cdot 0,6 \cdot 10^{-6}} \approx 3 .
\]

Согласно критерию (5.11) это соответствует дифракции Френеля. И расчет дифракционной картины будет правильным, если его проводить по формулам дифракции Френеля.

\begin{figure}[h]
\begin{center}
  \includegraphics[alt={},max width=\textwidth]{950fefba-d83b-4c8f-96e7-0bd5a7eb90be-154_241_485_637_373}
\captionsetup{labelformat=empty}
\caption{Рис. 5.18}
\end{center}
\end{figure}

Практически дифракцию Фраунгофера наблюдают с помощью схемы, показанной на рис. 5.18. Точечный источник света \(S\) располагают в фокусе \(F\) линзы \(L_{1}\). Из линзы выходит параллельный пучок лучей, на пути которого находится некоторая преграда \(N\) с тем или иным отверстием. Дифрагированные лучи проходят линзу \(L_{2}\) и падают на экран Э, расположенный в фокальной плоскости линзы \(L_{2}\) (на фокусном расстоянии \(f\) ). Таким образом, в каждую точку экрана падают только те лучи, которые до линзы \(L_{2}\) были параллельны друг другу.

Вид дифракционной картины на экране зависит от формы и размеров отверстия и длины волны падающего света. Наша задача - найти распределение интенсивности в дифракционной картине. В общем случае произвольной формы отверстия решение этой задачи - процедура весьма трудная в техническом отношении и, вообще говоря, не представляет особого интереса.

Практически наибольший интерес имеют три случая:

\begin{enumerate}
  \item дифракция от круглого отверстия,
  \item дифракция от узкой прямолинейной щели,
  \item дифракция от регулярной системы щелей (дифракционная решетка).\\
Рассмотрим эти случаи подробнее.
\end{enumerate}

\section*{§ 5.5. Дифракция Фраунгофера от круглого отверстия}
Дифракция Фраунгофера от круглого отверстия представляет большой практический интерес, поскольку в оптических приборах оправы линз и объективов, а также диафрагмы имеют обычно круглую форму.

Дифракционную картину Фраунгофера от круглого отверстия в преграде \(N\) можно наблюдать на экране Э в фокальной плоскости линзы \(L\), направив на отверстие нормально плоскую световую волну (рис. 5.19). Эта картина имеет вид центрального светлого пятна, окруженного чередующимися темными и светлыми кольцами. Соответствующий расчет (он представляет большие трудности, и мы не будем его приводить) дает, что минимум первого темного кольца отстает от центра дифракционной картины на угловое расстояние


\begin{equation*}
\vartheta_{1}=1,22 \frac{\lambda}{D} \tag{5.12}
\end{equation*}


\begin{itemize}
  \item это при условии, что диаметр отверстия \(D \gg \lambda\), что чаще всего и бывает.
\end{itemize}

На рис. 5.20 показана зависимость интенсивности света \(I\) от углового радиуса \(\vartheta\) (угла дифракции), здесь 0 - центр дифракционной картины. Из формулы (5.12) следует, что размер дифракционной картины тем меньше, чем больше диаметр отверстия \(D\). Подавляющая часть светового потока ( \(84 \%\) ), проходящего через отверстие, попадает в область центрального светлого

\begin{figure}[h]
\begin{center}
  \includegraphics[alt={},max width=\textwidth]{950fefba-d83b-4c8f-96e7-0bd5a7eb90be-155_405_401_1378_138}
\captionsetup{labelformat=empty}
\caption{Рис. 5.19}
\end{center}
\end{figure}

\begin{figure}[h]
\begin{center}
  \includegraphics[alt={},max width=\textwidth]{950fefba-d83b-4c8f-96e7-0bd5a7eb90be-155_401_429_1379_693}
\captionsetup{labelformat=empty}
\caption{Рис. 5.20}
\end{center}
\end{figure}

пятна. Поэтому в первом приближении дифракционную картину можно считать состоящей из одного лишь светлого пятна с угловым радиусом \(\vartheta_{1}\), определяемым формулой (5.12). И центральное светлое пятно можно рассматривать как изображение удаленного точечного источника, уширенное дифракцией от краев круглого отверстия диаметра \(D\).\\
Пример. Найдем диаметр \(d\) в мм центрального светлого пятна на экране, если диаметр отверстия \(D=1,0 \mathrm{~mm}\), фокусное расстояние \(f=50\) см и длина волны света \(\lambda=0,5\) мкм. Экран находится в фокальной плоскости линзы.\\
Освещенность светлого центрального пятна по мере увеличения угла дифракции \(\vartheta\) монотонно падает (см. рис. 5.20) и при визуальном наблюдении по контрасту это пятно кажется меньше. Его «диаметр» \(d\) приближенно можно считать равным paduycy первого темного кольца, т. е. \(d \approx f \vartheta_{1}\). Воспользовавшись формулой (5.12), получим

\[
d \approx 1,22 \frac{\lambda f}{D}=1,22 \frac{0,5 \cdot 10^{-6} \cdot 0,5}{10^{-3}} \approx 0,3 \cdot 10^{-3} \mathrm{M}=0,3 \mathrm{MM} .
\]

Следует обратить внимание на то, что в центре фраунгоферовой дифракционной картины от круглого отверстия всегда образуется максимум (в отличие от френелевой дифракции, когда в центре может образоваться как максимум, так и минимум).

Дифракционная расходимость пучка. Полученные результаты можно использовать для оценки дифракционной расходимости пучков света, диаметр которых ограничен, например, в результате прохождения плоской световой волны через отверстие (или диафрагму). Мы отметили, что основная часть светового потока, проходящего через отверстие, приходится на центральный дифракционный максимум, поэтому его ширину можно принять в качестве оценки угловой расходимости \(\delta \theta\) пучка с первоначальным диаметром поперечного сечения \(D\) :


\begin{equation*}
\delta \theta \sim \lambda / D . \tag{5.13}
\end{equation*}


Это весьма важный вывод: принципиально (из-за дифракции) невозможно создать совершенно параллельный ограниченный в сечении пучок света. Этим уширением можно пренебречь лишь в тех случаях, когда оно мало по сравнению с исходной шириной пучка.

Пример. Оценим дифракционное уширение «параллельного» лазерного пучка с исходным диаметром \(D_{0}=2\) мм на расстоянии \(l=100\) м от лазера, если длина волны света \(\lambda=0,60\) мкм.\\
Из-за дифракционной расходимости интересующий нас диаметр пучка оказывается равным\\
\(D \approx l \delta \theta \approx l \lambda / D_{0}=100 \cdot 0,6 \cdot 10^{-6} / 2 \cdot 10^{-3}=30 \cdot 10^{-3} \mathrm{M}=30\) мм.\\
Как видно, дифракционное уширение весьма значительно.\\
Дифракция от множества отверстий. Важно отметить, что при фраунгоферовой дифракции распределение интенсивности в дифракционной картине определяется только направлением лучей, а не положением световых пучков. Это означает, что распределение интенсивности не изменится, если отверстие в преграде сместить в сторону без изменения его ориентации.

Особый интерес представляет ситуация, когда в преграде имеется большое число \(N\) одинаковых отверстий. Здесь возможны два случая:

\begin{enumerate}
  \item отверстия расположены хаотично, беспорядочно;
  \item отверстия расположены упорядоченно, регулярно.
\end{enumerate}

В первом случае фазовые соотношения между волнами, дифрагированными от отдельных отверстий, имеют случайный характер (волны оказываются некогерентными). Поэтому для каждого направления наблюдения происходит простое сложение интенсивностей волн, дифрагированных от всех отверстий. Распределение интенсивности в дифракционной картине от одного отверстия не зависит от его положения. От большого числа \(N\) отверстий получается такая же картина, но усиленная по интенсивности в \(N\) раз.

Этим приемом, кстати, пользуются, желая усилить освещенность дифракционной картины от «одного» отверстия, если источник света довольно слабый.

Во втором случае, напротив, волны, дифрагированные от соседних отверстий, имеют определенное значение разности фаз, и волны оказываются когерентными. Интерференция этих волн существенно изменяет дифракционную картину, образуя резкое увеличение интенсивности в некоторых направлениях. Это обстоятельство имеет большое практическое значение, и более подробно мы рассмотрим этот вопрос на примере дифракционной решетки в § 5.7.

Разрешающая способность объектива. Как мы сейчас увидим, соотношение (5.12) играет важную роль в этом вопросе. Вследствие волновой природы света изображение точки, даваемое линзой, имеет вид дифракционного кружка - результат дифракции света на оправе линзы. Так как свыше \(80 \%\) проходящего через линзу светового потока приходится на центральное светлое пятно, то в дальнейшем на окружающие его кольца можно не обращать внимания.

Теперь рассмотрим два одинаковых некогерентных точечных источника. Если расстояние между центрами их изображений мало по сравнению с размерами центральных светлых пятен, то результирующая картина практически не отличается от изображения одного точечного источника. И тогда говорят, что объектив не разрешает рассматриваемые точки.

Начиная с некоторого расстояния между центрами обоих светлых кружков, между ними появляется темный провал, и это будет восприниматься как раздельное изображение двух точек. Объектив, как говорят, будет разрешать эти точки.

Количественный критерий разрешающей способности может быть сформулирован по ряду причин лишь условно. Такой условный критерий был предложен Рэлеем. Согласно критерию Рэлея, два точечных некогерентных источника считаются разрешенными, если центр дифракционного пятна от одного из них совпадает с ближайшим к центру минимумом дифракционной картины от другого. Это соответствует минимальному угловому расстоянию между источниками, определяемому формулой (5.12):


\begin{equation*}
\vartheta_{\text {мин }}=1,22 \frac{\lambda}{D} . \tag{5.14}
\end{equation*}


Результирующая картина показана на рис. 5.21, где провал составляет около \(25 \%\) от максимума интенсивности.

\begin{figure}[h]
\begin{center}
  \includegraphics[alt={},max width=\textwidth]{950fefba-d83b-4c8f-96e7-0bd5a7eb90be-158_278_522_1520_354}
\captionsetup{labelformat=empty}
\caption{Рис. 5.21}
\end{center}
\end{figure}

Величину, обратную предельному углу \(\vartheta_{\text {мин }}\), называют разрешающей способностью (или разрешающей силой):


\begin{equation*}
R=\frac{1}{\vartheta_{\text {мин }}}=\frac{D}{1,22 \lambda} . \tag{5.15}
\end{equation*}


Рассматриваемый вопрос особенно важен в теории телескопа. Изображение звезды в фокальной плоскости объектива телескопа представляет собой лишь дифракционную картину, образуемую круглой оправой объектива. Как видно из (5.15), разрешающая способность объектива телескопа пропорциональна диаметру объектива. Телескоп с диаметром зеркала \(D=5\) м может обеспечить для света с \(\lambda=0,55\) мкм угловое разрешение

\[
\vartheta_{\text {мин }}=1,22 \frac{0,55 \cdot 10^{-6}}{5} \approx 1,3 \cdot 10^{-7} \approx 0,03 \text { угл. сек }
\]

и разрешающую способность \(R \sim 10^{7}\).\\
Увеличение разрешающей способности телескопов путем сооружения гигантских конструкций имеет естественный предел, обусловленный прочностью конструкционных материалов. Этот предел практически уже достигнут. Принципиально новые возможности повышения разрешающей способности связаны с идеей, суть которой состоит в построении большой оптической системы из множества зеркал сравнительно небольшого размера, которые образуют единую поверхность. Это позволяет существенно снизить массу составного зеркала-отражателя.

Глаз при рассматривании удаленных предметов действует в принципе так же, как объектив. Поэтому формулы (5.14) и (5.15) применимы и к глазу. Роль величины \(D\) играет диаметр \(d\) зрачка глаза. Полагая \(d \approx 4\) мм, \(\lambda=0,55\) мкм, находим, что разрешаемое угловое расстояние глазом

\[
\vartheta_{\text {мин }}=1,22 \lambda / d=1,67 \cdot 10^{-4}=35 \text { угл. сек. }
\]

Этот результат удивительно хорошо согласуется с физиологической оценкой разрешающей способности глаза, связанной со структурой его сетчатки.

\section*{§ 5.6. Дифракция Фраунгофера от щели*}
В отличие от дифракции Фраунгофера от круглого отверстия, расчет дифракции от узкой длинной щели с параллельными краями оказывается значительно проще, и мы воспроизведем его достаточно детально. Соответствующий расчет и здесь будет проведен с помощью принципа Гюйгенса-Френеля.

Рассмотрим случай, когда на щель ширины \(b\) падает нормально плоская световая волна (рис. 5.22). Разобьем мысленно эту щель - она же открытая часть волновой поверхности - на очень узкие одинаковые по ширине зоны-полоски, параллельные прямолинейным краям щели. Суммирование вторичных волн проведем с помощью векторной диаграммы.

\begin{figure}[h]
\begin{center}
  \includegraphics[alt={},max width=\textwidth]{950fefba-d83b-4c8f-96e7-0bd5a7eb90be-160_358_338_451_786}
\captionsetup{labelformat=empty}
\caption{Рис. 5.22}
\end{center}
\end{figure}

Колебания, приходящие в точку \(P\) от каждой такой зоны-полоски имеют одинаковую амплитуду \(\mathrm{d} A\), поскольку распространяются параллельно друг другу перед линзой и, значит, \(\mathrm{d} A\) не зависит от пройденного пути до точки \(P\) (напомним, что линза - система таутохронная). При этом разность фаз между колебаниями, приходящими в точку \(P\) от соседних зон-полосок, будет одинакова.

Отсюда следует, что при графическом изображении мы получим цепочку векторов \(\mathrm{dA}_{i}\), одинаковых по модулю и повернутых относительно друг друга на один и тот же угол (рис. 5.23, a). Результирующая амплитуда изобразится вектором \(\mathbf{A}\) - хордой дуги окружности с центром в точке \(C\).

Заметим, что для точки \(P_{0}\) эта цепочка образует прямую, что соответствует максимуму интенсивности.

Условие минимумов. Если разность хода крайних лучей (см. рис. 5.22) составляет \(\Delta=\lambda\), то их разность фаз \(\delta=2 \pi\), цепочка оказывается замкнутой и амплитуда результирующего колебания обращается в нуль (рис. 5.23, б). Это первый минимум дифракционной картины, представляющей собой симметричную

\footnotetext{\begin{itemize}
  \item От полуплоскости дифракция Фраунгофера ни при каких условиях не реализуется.
\end{itemize}
}\begin{figure}[h]
\begin{center}
  \includegraphics[alt={},max width=\textwidth]{950fefba-d83b-4c8f-96e7-0bd5a7eb90be-161_457_764_149_233}
\captionsetup{labelformat=empty}
\caption{Рис. 5.23}
\end{center}
\end{figure}

относительно середины систему чередующихся светлых и темных полос, параллельных щели.

Результирующая амплитуда обращается в нуль и тогда, когда разность фаз от крайних элементов щели равна \(2 \pi m\), где \(m=1,2, \ldots\) Цепочка при этом замыкается после \(m\) оборотов, практически не меняя своей длины \(A_{0}\), поскольку угол дифракции Э обычно достаточно мал.

Разность фаз \(\delta\) связана с разностью хода \(\Delta\) соотношением (4.4'), т. е.

\[
\delta=2 \pi \Delta / \lambda,
\]

где \(\lambda\) - длина волны света.\\
Так как \(\Delta=b \sin \vartheta\) (см. рис. 5.22) и в минимуме \(\delta=2 \pi m\), то из этих трех равенств следует условие для минимумов:


\begin{equation*}
b \sin \vartheta_{m}= \pm m \lambda, \quad m=1,2, \ldots \tag{5.16}
\end{equation*}


Заметим, что \(m \neq 0\), поскольку при \(m=0\) образуется максимум (цепочка векторов становится прямой). Из этой формулы видно, что уменьшение ширины \(b\) щели приводит к расширению дифракционной картины.

Распределение интенсивности. Найдем интенсивность \(I\) света на экране в зависимости от угла дифракции Э. Это легко сделать с помощью рис. 5.23, \(a\). Обозначив радиус цепочки-дуги через \(R\), запишем:


\begin{equation*}
A=2 R \sin (\delta / 2), A_{0}=R \delta . \tag{5.17}
\end{equation*}


Остается исключить \(R\) из этих двух равенств, и мы получим:


\begin{equation*}
A=A_{0} \frac{\sin (\delta / 2)}{\delta / 2} . \tag{5.18}
\end{equation*}


А так как \(I \sim A^{2}\), то искомая зависимость будет иметь вид:


\begin{equation*}
I=I_{0} \frac{\sin ^{2} \alpha}{\alpha^{2}}, \tag{5.19}
\end{equation*}


где \(\alpha=\delta / 2=\pi \Delta / \lambda=\pi b \sin \vartheta / \lambda\). График зависимости \(I\) от \(\sin \vartheta\) показан на рис. 5.24. Интенсивность второго максимума составляет около \(4 \%\) от интенсивности центрального, поэтому можно считать, что практически весь световой поток, проходящий через щель, сосредоточен в первом (центральном) максимуме, угловая полуширина которого равна \(\lambda / b\). Этот результат согласуется с формулой (5.13), определяющей дифракционную расходимость «параллельных» световых пучков ограниченного сечения.

\begin{figure}[h]
\begin{center}
  \includegraphics[alt={},max width=\textwidth]{950fefba-d83b-4c8f-96e7-0bd5a7eb90be-162_362_678_915_276}
\captionsetup{labelformat=empty}
\caption{Рис. 5.24}
\end{center}
\end{figure}

Отметим также, что в середине симметричной дифракционной картины, состоящей из чередующихся светлых и темных полос, при дифракции Фраунгофера всегда образуется максимум освещенности (в отличие от френелевой дифракции, где центральная полоса может быть как светлой, так и темной).

Если плоская световая волна падает на щель наклонно под углом \(\vartheta_{0}\) к нормали, то разность хода между колебаниями, распространяющимися от краев щели под углом Э к нормали, будет равна \(\Delta=b\left(\sin \vartheta-\sin \vartheta_{0}\right)\). Это при условии, что оба угла, Э и \(\vartheta_{0}\), отсчитываются от нормали в одну сторону - по или против часовой стрелки.

Условие дифракционных минимумов в данном случае принимает вид


\begin{equation*}
b\left(\sin \vartheta_{m}-\sin \vartheta_{0}\right)= \pm m \lambda \tag{5.20}
\end{equation*}


Центральный максимум ( \(m=0\) ) будет расположен под углом \(\vartheta_{m}=\vartheta_{0}\), т. е. в направлении падающей волны, и дифракционная картина будет несимметрична относительно центральной светлой полосы.

Теорема Бабине. Эта полезная теорема касается свойств так называемых дополнительньх экранов. Например, в нашем случае, когда в непрозрачной преграде имеется щель, то дополнительным к ней экраном будет непрозрачная прямолинейная полоска, ширина которой равна ширине щели.

Согласно теореме Бабине, фраунгоферовы дифракционные картины в фокальной плоскости объектива совершенно одинаковы от дополнительных экранов, за исключением самого фокуса. Эта теорема следует из того, что согласно принципу Гюйгенса-Френеля дифрагированные волны можно представить как сумму вторичных волн, исходящих из каждого элемента площади отверстий экрана. Пусть вектор-амплитуда дифрагированной в некотором направлении волны для данного экрана равна \(\mathbf{A}_{1}\), а для дополнительного к нему экрана в этом же направлении - \(\mathbf{A}_{2}\). В отсутствие обоих экранов амплитуды волн для всех направлений, кроме направления первоначальной волны, равны нулю. Следовательно, по принципу Гюйгенса-Френеля получим \(\mathbf{A}_{1}+\mathbf{A}_{2}=0\), откуда \(A_{1}=A_{2}\). Так как интенсивность \(I \sim A^{2}\), то тем самым теорема доказана.

Теорема Бабине позволяет упростить решение многих дифракционных задач, заменяя экраны дополнительными. Например, надо определить толщину человеческого волоса. Для этого получают от него фраунгоферову дифракционную картину, а затем такую же по размеpy картину - от щели, подбирая ее ширину. Найденная ширина щели и равна диаметру человеческого волоса.

\section*{§ 5.7. Дифракционная решетка}
Дифракционная решетка является важнейшим спектральным прибором, предназначенным для разложения света в спектр и измерения длин волн. Она представляет собой стеклянную или металлическую пластинку, на которых нанесено очень много (иногда до сотен тысяч) прямых равноотстоящих штрихов одинаковой конфигурации.

Рассмотрим простейшую идеализированную решетку, состоящую из одинаковых равноотстоящих щелей. Пусть ширина каждой щели равна \(b\), а период решетки - \(d\). В решетке реализуется многолучевая интерференция когерентных дифрагированных пучков света, исходящих из щелей решетки при ее освещении.

Дифракционную (точнее дифракционно-интерференционную) картину наблюдают по методу Фраунгофера, т. е. в параллельных лучах, а практически - в фокальной плоскости объектива (рис. 5.25, \(a\) ).

\begin{figure}[h]
\begin{center}
  \includegraphics[alt={},max width=\textwidth]{950fefba-d83b-4c8f-96e7-0bd5a7eb90be-164_395_1024_580_104}
\captionsetup{labelformat=empty}
\caption{Рис. 5.25}
\end{center}
\end{figure}

Пусть плоская монохроматическая световая волна падает на решетку нормально. Каждая из щелей в отдельности давала бы в фокальной плоскости объектива дифракционную картину, показанную на рис. 5.24. И такие картины от всех щелей в отсутствие когерентности точно накладывались бы друг на друга, независимо от их положения. Интенсивности при этом складывались бы, и мы получили бы при наличии \(N\) щелей дифракционную картину как от одной щели, но усиленную в \(N\) раз.

При освещении же решетки когерентным светом, световые волны от всех щелей интерферируют друг с другом, и дифракционная картина резко меняется. Мы будем наблюдать систему достаточно узких максимумов.

Главные максимумы. В середину дифракционно-интерференционной картины* когерентные колебания от всех щелей приходят в фазе. Это значит, что если амплитуда от одной щели равна \(A_{1}\), а число щелей в решетке \(N\), то результирующая

\footnotetext{\begin{itemize}
  \item Далее мы будем называть ее по традиции просто дифракционной.
\end{itemize}
}амплитуда \(A\) и соответствующая ей интенсивность \(I\) будут определяться формулами

\[
A=A_{1} N, \quad I=I_{1} N^{2} .
\]

Такой же результат получается и при углах дифракции Э, для которых оптическая разность хода \(\Delta\) колебаний от соседних щелей (см. рис. 5.25, б) равна целому числу длин волн:


\begin{equation*}
d \sin \vartheta_{m}= \pm m \lambda, \quad m=0,1,2, \ldots, \tag{5.21}
\end{equation*}


где знаки «±» следуют из симметрии дифракционной картины относительно нормали к решетке ( \(\vartheta_{0}=0\) ): при знаке «+» угол \(\vartheta_{m}>0\), а при знаке « угол \(\vartheta_{m}<0\).

В направлениях \(\vartheta_{m}\), определяемых этим уравнением, возникают максимумы, интенсивности которых в \(N^{2}\) раз превосходят интенсивность от каждой щели в том же направлении. Их называют главньмми максимумами \(m\)-го порядка, а уравнение (5.21) - условием главных максимумов. Именно главные максимумы и представляют особый практический интерес. Как мы увидим далее, они получаются тем более узкими и резкими, чем большее число \(N\) штрихов содержит решетка.

При наклонном падении плоской волны на решетку - под углом \(\vartheta_{0}\) к нормали (рис. \(5.26, a\) ) разность хода соответствующих лучей от двух соседних штрихов (щелей) равна \(\Delta=d\left(\sin \vartheta_{m}-\sin \vartheta_{0}\right)\), и направления \(\vartheta_{m}\) на главные фраунгоферовы максимумы определяются условием


\begin{equation*}
d\left(\sin \vartheta_{m}-\sin \vartheta_{0}\right)= \pm m \lambda \tag{5.22}
\end{equation*}


с учетом следующего правила знаков для углов \(\vartheta_{m}\) и \(\vartheta_{0}\) : они должны отсчитываться в одном направлении от нормали к решетке, например по часовой стрелке (см. рис. 5.26, \(a\), где \(\vartheta_{0}\) и \(\vartheta_{m}>0\) ).

\begin{figure}[h]
\begin{center}
  \includegraphics[alt={},max width=\textwidth]{950fefba-d83b-4c8f-96e7-0bd5a7eb90be-165_347_1076_1447_78}
\captionsetup{labelformat=empty}
\caption{Рис. 5.26}
\end{center}
\end{figure}

Это же условие (5.22) справедливо и для отражательной решетки, если углы \(\vartheta_{m}\) и \(\vartheta_{0}\) отсчитывать в противоположных направлениях от нормали (см. рис. 5.26, б, где \(\vartheta_{0}\) и \(\vartheta_{m}>0\) ).

Отметим попутно, что форма штрихов решетки не влияет на положение главных фраунгоферовых максимумов, и условия (5.21) и (5.22) являются универсальными.

Интерференционные минимумы. Для выяснения дальнейших деталей фраунгоферовой дифракционной картины воспользуемся векторной диаграммой, которая позволит легко найти и результирующую амплитуду \(A\) колебаний, приходящих в произвольную точку \(P\) фокальной плоскости объектива (см. рис. 5.25).

\begin{figure}[h]
\begin{center}
  \includegraphics[alt={},max width=\textwidth]{950fefba-d83b-4c8f-96e7-0bd5a7eb90be-166_404_606_655_312}
\captionsetup{labelformat=empty}
\caption{Рис. 5.27}
\end{center}
\end{figure}

Векторная диаграмма в нашем случае представляет собой цепочку векторов-амплитуд когерентных колебаний, приходящих в точку \(P\) от каждой из \(N\) щелей: \(A_{1}, A_{2}, \ldots, A_{N}\) (рис. 5.27). По модулю эти векторы одинаковы, и каждый следующий отстает от предыдущего (или опережает, это не существенно) по фазе на один и тот же угол \(\gamma\). Этот угол связан с оптической разностью хода \(\Delta\) соответствующих лучей от соседних щелей известным соотношением (4.4'), т. е. в нашем случае - при нормальном падении света на решетку


\begin{equation*}
\gamma=2 \pi \frac{\Delta}{\lambda}=2 \pi \frac{d \sin \vartheta}{\lambda}, \tag{5.23}
\end{equation*}


где \(d\) - период решетки (см. рис. 5.25, б).\\
Теперь проследим, как будет вести себя эта цепочка векторов (а значит и ее замыкающая \(A\) ) при удалении точки \(P\) от фокуса \(F\) (см. рис. \(5.25, a\) ), т. е. с ростом угла дифракции Э.

Ясно, что при этом будет увеличиваться разность фаз \(\gamma\) между колебаниями от соседних щелей, и цепочка векторов будет постепенно закручиваться. Первый раз она замкнется и вектор \(\mathbf{A}\) обратится в нуль, когда угол \(N \gamma\) станет равным \(2 \pi\) - это непосредственно видно из рис. 5.27, б.

При дальнейшем росте угла \(\vartheta\), а значит, разности фаз \(\gamma\) и \(N \gamma\), цепочка будет периодически то распрямляться (главные максимумы, \(A=\) макс), то замыкаться (интерференционные минимумы, \(A=0\) ). Последнее будет происходить при значениях угла \(N \gamma\) кратных \(2 \pi\) :


\begin{equation*}
N \gamma=2 \pi m^{\prime}, \tag{5.24}
\end{equation*}


где \(m^{\prime}\) принимает целочисленные значения, кроме \(0, N, 2 N, \ldots\), при которых цепочка распрямляется, и мы получаем главные максимумы.

Подставив в (5.24) значение \(\gamma\) из формулы (5.23), получим:


\begin{equation*}
d \sin \vartheta= \pm \frac{m^{\prime}}{N} \lambda \tag{5.25}
\end{equation*}


Это выражение представляет собой условие для интерференционных минимумов (при целочисленных значениях \(m^{\prime}\), кроме \(0, N, 2 N, .\).\() . Оно же содержит и условие (5.21) для главных мак-\) симумов (при \(m^{\prime}=0, N, 2 N, \ldots\) ). Между двумя соседними главными максимумами расположены \(N-1\) интерференционных минимумов. А между последними, в свою очередь, - добавочные максимумь, интенсивность которых при достаточно большом числе \(N\) штрихов решетки пренебрежимо мала (как мы увидим далее, она составляет не более \(5 \%\) от интенсивности главных максимумов).

В отличие от условия (5.21), которое дает только положения главных максимумов, соотношение (5.25) позволяет определить и их угловую ширину. В самом деле, при переходе от главного максимума к соседнему минимуму (рис. 5.28) \(m^{\prime}\) меняется на единицу, например от \(N\) до \(N+1\). Тогда при достаточно большом \(N\) угловую полуширину \(\delta \vartheta\) главного максимума 1 -го порядка можно найти, взяв дифференциал уравнения (5.25)

\begin{figure}[h]
\begin{center}
  \includegraphics[alt={},max width=\textwidth]{950fefba-d83b-4c8f-96e7-0bd5a7eb90be-168_342_501_157_365}
\captionsetup{labelformat=empty}
\caption{Рис. 5.28}
\end{center}
\end{figure}

с учетом того, что \(m^{\prime}\) при этом меняется на единицу ( \(\delta m^{\prime}=1\) ). Тогда \(d \cos \vartheta \delta \vartheta=\lambda / N\), откуда


\begin{equation*}
\delta \vartheta=\frac{\lambda}{N d \cos \vartheta}=\frac{\lambda}{h \cos \vartheta} . \tag{5.26}
\end{equation*}


Обращает на себя внимание тот факт, что \(\delta \vartheta\) зависит не от \(d\) и \(N\) в отдельности, а от их произведения, которое есть не что иное как ширина решетки \(h=N d\). С ростом угла дифракции \(\vartheta\) ширина главных максимумов увеличивается. Главные максимумы будут тем у́же, чем больше ширина решетки \(h\) и меньше угол дифракции Э.

Теперь выясним, что означает утверждение, например, «угловая ширина главного максимума \(\delta \vartheta\) мала». По сравнению с чем? Ответ достаточно очевидный: величину \(\delta \vartheta\) надо сравнивать с угловой шиной \(\Delta \vartheta\) между соседними главными максимумами. Если \(\delta \vartheta \ll \Delta \vartheta\), мы говорим, что главные максимумы узкие (резкие). Оценим отношение этих двух величин. Значение \(\delta \vartheta\) соответствует изменению \(m^{\prime}\) в (5.25) на единицу, но таких значений \(m^{\prime}\) между двумя соседними главными максимумами оказывается \(N\). Считая, что на каждый интервал \(\delta m^{\prime}=1\) приходится одно и то же значение \(\delta \vartheta\) (для оценки), приходим к выводу, что \(\delta \vartheta\) в \(N\) раз меньше, чем \(\Delta \vartheta\). Итак, резкость главных максимумов пропорциональна числу штрихов решетки (более точный расчет приводит к тому же результату).

Таким образом, с помощью условий (5.21) и (5.25) мы можем установить не только положения главных максимумов, но и их угловую ширину (резкость). Остается решить вопрос об их интенсивности. Рассмотрим его сначала качественно.

Прослеживая с помощью рис. 5.27, как будет вести себя векторная диаграмма по мере увеличения угла дифракции Э, мы оставили без внимания тот факт, что при этом каждый вектор цепочки по модулю будет уменьшаться, ибо он определяется дифракцией от каждой щели. В соответствии же с рис. 5.23 результирующий вектор при закручивании цепочки будет сначала уменьшаться и в дальнейшем вести себя аналогично тому, как показано на рис. 5.24.

Следовательно, кроме интерференционных минимумов, необходимо иметь в виду и дифракционные минимумы, определяемые условием (5.16), т. е.

\[
b \sin \vartheta_{m}= \pm m \lambda, \quad m=1,2, \ldots,
\]

где \(b\) - ширина каждой щели.\\
При этом условии все векторы цепочки обращаются в нуль, значит и результирующая интенсивность в этих направлениях всегда должна быть равна нулю. Даже в том случае, если этому направлению соответствует главный максимум \(m\)-го порядка.\\
Пример. Найдем отношение периода решетки к ширине щели, \(d / b\), при котором пропадает главный максимум 3 -го порядка.\\
В этом направлении - под углом дифракции \(\vartheta_{x}\) - должны выполняться два условия:

\[
d \sin \vartheta_{x}= \pm 3 \lambda, \quad b \sin \vartheta_{x}= \pm \lambda .
\]

Из отношения этих двух равенств находим: \(d / b=3\). Это значит, что в пределах основного дифракционного максимума мы будем наблюдать два главных максимума. Третий попадет на дифракционный минимум и, значит, пропадет.\\
Интенсивность главных максимумов. Распределение интенсивности в дифракционной картине проще всего получить с помощью векторной диаграммы (см. рис. 5.27). Из этой диаграммы видно, что результирующую амплитуду \(A\) при интерференции \(N\) волн можно записать так:

\[
A=2 R \sin (N \gamma / 2),
\]

где \(R\) - вспомогательный радиус дуги окружности, описанной вокруг данной цепочки векторов. Кроме того, амплитуда \(A_{1}\) колебаний от каждой щели, как видно из этого же рисунка,

\[
A_{1}=2 R \sin (\gamma / 2) .
\]

Исключив \(2 R\) из последних двух формул, получим:

\[
A=A_{1} \frac{\sin (N \gamma / 2)}{\sin (\gamma / 2)} .
\]

Учитывая, что \(A_{1}\) определяется формулой (5.18) и интенсивность \(I \sim A^{2}\), приходим к следующему (искомому) выражению:


\begin{equation*}
I=I_{0} \frac{\sin ^{2}(\delta / 2)}{(\delta / 2)^{2}} \cdot \frac{\sin ^{2}(N \gamma / 2)}{\sin ^{2}(\gamma / 2)}, \tag{5.27}
\end{equation*}


где, напомним,


\begin{equation*}
\delta=2 \pi b \sin \vartheta / \lambda, \quad \gamma=2 \pi d \sin \vartheta / \lambda . \tag{5.28}
\end{equation*}


Полученный результат (5.27) графически представлен на рис. 5.29 как зависимость интенсивности дифракционной картины от угла дифракции \(\vartheta\), точнее от \(\sin \vartheta\). Как видим, интерференция многих пучков привела к резкому перераспределению интенсивности света, обусловленному дифракцией от каждой щели.

\begin{figure}[h]
\begin{center}
  \includegraphics[alt={},max width=\textwidth]{950fefba-d83b-4c8f-96e7-0bd5a7eb90be-170_303_808_883_211}
\captionsetup{labelformat=empty}
\caption{Рис. 5.29}
\end{center}
\end{figure}

Первая дробь в выражении (5.27) представляет собой плавную функцию от \(\sin \vartheta\) (она показана пунктиром на рис. 5.29 и отражает дифракционое распределение интенсивности от каждой щели). Эта плавная функция модулирует многолучевую интерференционную картину от \(N\) щелей, которую описывает вторая дробь в формуле (5.27).

Практически наиболее важными являются главные максимумы, попадающие в центральный дифракционный максимум от каждой щели - они являются наиболее интенсивными*.

\footnotetext{\begin{itemize}
  \item Заметим, что распределение \(I(\vartheta)\), показанное на рис. 5.29 , справедливо для простой щелевой решетки. Если же коэффициент прозрачности (или отражения) меняется в решетке, например, по синусоидальному закону, то возникают главные максимумы только нулевого и первого порядков: \(m=0\) и \(m= \pm 1\).
\end{itemize}
}\section*{§ 5.8. Дифракционная решетка как спектральный прибор}
Из формулы (5.21), определяющей направления на главные фраунгоферовы максимумы, видно, что эти направления \(\vartheta_{m}\) зависят от длины световой волны \(\lambda\) (за исключением максимума нулевого порядка, \(m=0\) ). Поэтому решетка в каждом порядке \(m \neq 0\) разложит падающий на нее свет в спектр различных порядков. Причем наибольшее отклонение в каждом порядке испытывает красная часть спектра (более длинноволновая).

Основными характеристиками любого спектрального прибора являются угловая дисперсия, разрешающая способность и область дисперсии.

\begin{enumerate}
  \item Угловая дисперсия \(\boldsymbol{D}\) характеризует степень пространственного (углового) разделения волн с различными длинами \(\lambda\). По определению,
\end{enumerate}


\begin{equation*}
D=\mathrm{d} \vartheta / \mathrm{d} \lambda . \tag{5.29}
\end{equation*}


Дифференцируя формулу (5.21) при данном \(m\), находим \(\partial л я p e\) шетки \(d \cos \vartheta_{m} \mathrm{~d} \vartheta=m \mathrm{~d} \lambda\), откуда


\begin{equation*}
\frac{\mathrm{d} \vartheta}{\mathrm{~d} \lambda}=\frac{m}{d \cos \vartheta_{m}} . \tag{5.30}
\end{equation*}


Видно, что для заданного порядка \(m\) спектра угловая дисперсия тем больше, чем меньше период \(d\) решетки. Кроме того, \(\mathrm{d} \vartheta / \mathrm{d} \lambda\) растет с увеличением порядка \(m\) спектра.\\
2. Разрешающая способность \(\boldsymbol{R}\). По определению,


\begin{equation*}
R=\lambda / \delta \lambda, \tag{5.31}
\end{equation*}


где \(\delta \lambda\) - наименьшая разность длин волн спектральных линий, при которой эти линии воспринимаются еще раздельно, т. е. разрешаются. Величина \(\delta \lambda\) не может быть по ряду причин определена точно, а лишь ориентировочно (условно). Такой условный критерий был предложен Рэлеем (аналогично тому, как это было рассмотрено при решении вопроса о разрешающей способности объектива, см. § 5.5, в конце).

Согласно критерию Рэлея, спектральные линии с разными длинами волн, но одинаковой интенсивности, считаются разре-

шенными, если главный максимум одной спектральной линии совпадает с первым минимумом другой (рис. 5.30). В этом случае между двумя максимумами возникает провал, составляющий около \(20 \%\) от интенсивности в максимумах, и линии еще воспринимаются раздельно.

\begin{figure}[h]
\begin{center}
  \includegraphics[alt={},max width=\textwidth]{950fefba-d83b-4c8f-96e7-0bd5a7eb90be-172_334_422_153_720}
\captionsetup{labelformat=empty}
\caption{Рис. 5.30}
\end{center}
\end{figure}

Итак, согласно критерию Рэлея и формуле (5.25), необходимо, чтобы максимум \(m\)-го порядка ( \(m^{\prime}=m N\) ) линии с длиной волны \(\lambda+\delta \lambda\) (см. рис. 5.30) совпадал по направлению с первым минимумом линии \(\lambda\left(m^{\prime}=m N+1\right)\), т. е.

\[
d \sin \vartheta_{m}=m(\lambda+\delta \lambda)=\left(m+\frac{1}{N}\right) \lambda .
\]

Отсюда следует, что


\begin{equation*}
\lambda / \delta \lambda=m N . \tag{5.32}
\end{equation*}


Это и есть искомая формула для разрешающей способности дифракционной решетки. Данная формула дает верхний предел разрешающей способности. Она справедлива при выполнении следующих условий:

\begin{enumerate}
  \item Интенсивность обоих максимумов должна быть одинаковой.
  \item Расширение линий должно быть обусловлено только дифракцией.
  \item Необходимо, чтобы падающий на решетку свет имел ширину когерентности, превышающую размер решетки. Только в этом случае все \(N\) штрихов решетки будут «работать» согласованно (когерентно), и мы достигнем желаемого результата.\\
Для повышения разрешающей способности спектральных приборов можно, как показывает формула (5.32), либо увеличивать число \(N\) когерентных пучков, либо повышать порядок интерференции \(m\). Первое используется в дифракционных решетках (число \(N\) доходит до 200000 ), второе - в интерференцион-
\end{enumerate}

ных спектральных приборах (например, в интерферометре Фаб-ри-Перо число \(N\) интерферирующих волн невелико, порядка нескольких десятков, а порядки интерференции \(m \sim 10^{6}\) и более).\\
3. Область дисперсии \(\Delta \boldsymbol{\lambda}\) - это ширина спектрального интервала, при которой еще нет перекрытия спектров соседних порядков. Если спектры соседних порядков перекрываются, то спектральный аппарат становится непригодным для исследования соответствующего участка спектра.

Длинноволновый конец спектра \(m\)-го порядка совпадает с коротковолновым концом спектра \((m+1)\)-го порядка, если \(m(\lambda+\Delta \lambda)=(m+1) \lambda\), откуда следует, что область дисперсии


\begin{equation*}
\Delta \lambda=\lambda / m . \tag{5.33}
\end{equation*}


Значит, область дисперсии \(\Delta \lambda\) обратно пропорциональна порядку спектра \(m\). При работе со спектрами низких порядков (обычно второго или третьего) дифракционная решетка пригодна для исследования излучения, занимающего достаточно широкий спектральный интервал. В этом главное преимущество дифракционных решеток перед интерференционными спектральными приборами, например, интерферометром Фабри-Перо, у которого из-за высоких порядков \(m\) область дисперсии очень мала.

Еще о дифракционных решетках. Дифракционная решетка является одним из важнейших спектральных приборов, которому наука обязана многими фундаментальными открытиями. Спектр - это по существу код, который, будучи расшифрован с помощью того или иного математического аппарата, дает возможность получить ценнейшую информацию о свойствах атомов и внутриатомных процессов.

Для адекватного решения этой задачи спектр должен быть неискаженным и четко различимым - в этом суть той сложнейшей науч-но-технической проблемы, которую пришлось решить, чтобы наконец добиться получения высококачественных дифракционных решеток.

Технология изготовления дифракционных решеток в настоящее время доведена до высокой степени совершенства. Первые высококачественные отражательные решетки были созданы Роуландом* \({ }^{*}(1882)\). О технической сложности решаемой проблемы говорит хотя бы уже такой факт, что необходимая для этой цели делительная машина создавалась в течение 20 лет!

\footnotetext{\begin{itemize}
  \item Его дело продолжили Андерсен, Вуд и другие знаменитые экспериментаторы.
\end{itemize}
}Современные полностью автоматизированные делительные машины позволяют с помощью алмазного резца изготовлять решетки с почти строго эквидистантным расположением штрихов. Трудно даже представить, что алмазный резец при этом прочерчивает десятки километров, практически не изменяя свой профиль, - а это принципиально важно. Размеры уникальных решеток достигают \(40 \times 40 \mathrm{~cm}\) ! (Такие решетки используют в основном в астрофизике.)

В зависимости от области спектра решетки имеют различное число штрихов на 1 мм: от нескольких штрихов, начиная с инфракрасной области, до 3600 - для ультрафиолетовой. В видимой области спектра \(600-1200\) штрих/мм. Ясно, что обращение с гравированной поверхностью таких решеток требует предельной осторожности.

Вследствие высокой стоимости оригинальных гравированных решеток получили распространение реплики, т. е. отпечатки гравированных решеток на специальных пластмассах, покрытых тонким отражательным слоем. По качеству реплики почти не уступают оригиналам.

В 1970-х годах был разработан новый, голографический метод изготовления дифракционных решеток. В этом методе плоская подложка со светочувствительным слоем освещается двумя плоскими наклонными пучками когерентных лазерных излучений с определенной длиной волны. В области пересечения пучков образуется стационарная интерференционная картина с синусоидальным распределением интенсивности. После соответствующей обработки светочувствительного слоя получается качественная дифракционная решетка.

Отметим в заключение, что, кроме прозрачных и отражательных решеток, существуют еще и фазовые. Они влияют не на амплитуду световой волны, а вносят периодические изменения в ее фазу. По этой причине их и называют фазовыми. Примером фазовой решетки может служить пластмассовая кювета с прозрачной жидкостью, в которой возбуждена плоская стоячая ультразвуковая волна. Это приводит к периодическому изменению плотности жидкости, а значит ее показателя преломления и оптической разности хода. Такая структура меняет не амплитуду проходящего поперек волны света, а только фазу. Фазовые решетки также находят многочисленные практические применения.

Одномерная решетка вибраторов. Аналогично дифракционной решетке ведет себя в радиодиапазоне система из \(N\) параллельных друг другу вибраторов-антенн. Если они действуют синфазно, то нулевой (основной) максимум излучения направлен нормально к решетке в ее экваториальной плоскости. И здесь возникает интересная в практическом отношении возможность. Если создать режим, при котором колебания каждой следующей антенны будут, например, отставать по фазе от колебаний преды-

дущей на одну и ту же величину \(\varphi\), то нулевой максимум не будет совпадать с нормалью к решетке. Изменяя же φ во времени по определенному закону, мы получаем систему, у которой направление главного максимума будет изменяться в пространстве. Таким образом мы приходим к возможности радиолокационного обзора местности с помощью неподвижной системы антенн.

\section*{§ 5.9. Дифракция от пространственной решетки}
Пример пространственной дифракционной решетки - это кристаллическая решетка твердого тела. Частицы, образующие эту решетку, играют роль упорядоченно расположенных центров, когерентно рассеивающих падающую на них волну.

Рассмотрение дифракции от упорядоченных структур проще всего начать с дифракции монохроматического излучения от прямолинейной цепочки, состоящей из одинаковых равноотстоящих частиц (например, атомов). Пусть расстояние между соседними частицами (период структуры) равно \(d\) и параллельный пучок излучения с длиной волны \(\lambda\) падает на такую цепочку под углом скольжения \(\alpha_{0}\) (рис. 5.31). Разность хода между лучами 1 и 2 , рассеянными соседними частицами под углом \(\alpha\), равна, как вид-

\begin{figure}[h]
\begin{center}
  \includegraphics[alt={},max width=\textwidth]{950fefba-d83b-4c8f-96e7-0bd5a7eb90be-175_345_764_1081_233}
\captionsetup{labelformat=empty}
\caption{Рис. 5.31}
\end{center}
\end{figure}

но из этого рисунка, \(\Delta=A D-C B=d\left(\cos \alpha-\cos \alpha_{0}\right)\). Углы \(\alpha=\alpha_{m}\), под которыми образуются фраунгоферовы максимумы \(m\)-го порядка, определяются условием, при котором эта разность хода равна целому числу длин волн:

\[
d\left(\cos \alpha_{m}-\cos \alpha_{0}\right)= \pm m \lambda, \quad m=0,1,2, \ldots
\]

Условия Лауэ. Рассмотрим для простоты прямоугольную решетку, периоды которой вдоль осей координат \(X, Y, Z\), параллельных трем ребрам решетки, равны \(d_{1}, d_{2}, d_{3}\). Тогда при фраунгоферовой дифракции главные максимумы должны удовлетворять трем соотношениям, аналогичным только что приведенному для дифракционных максимумов при наклонном падении излучения на прямолинейную цепочку. Эти соотношения - условия Лауэ - имеют вид:


\begin{align*}
& d_{1}\left(\cos \alpha-\cos \alpha_{0}\right)= \pm m_{1} \lambda, \\
& d_{2}\left(\cos \beta-\cos \beta_{0}\right)= \pm m_{2} \lambda,  \tag{5.34}\\
& d_{3}\left(\cos \gamma-\cos \gamma_{0}\right)= \pm m_{3} \lambda,
\end{align*}


где \(\alpha_{0}, \beta_{0}, \gamma_{0}\) и \(\alpha, \beta, \gamma\) - углы между осями координат \(X, Y, Z\) и направлениями распространения падающего и дифрагированного пучков; \(m_{1}, m_{2}, m_{3}\) - целые числа, определяющие порядок максимума ( \(0,1,2, \ldots\) ).

Углы \(\alpha, \beta, \gamma\) не являются независимыми. В случае прямоугольной системы координат они связаны соотношением


\begin{equation*}
\cos ^{2} \alpha+\cos ^{2} \beta+\cos ^{2} \gamma=1 . \tag{5.35}
\end{equation*}


При одновременном выполнении этих четырех уравнений (а это оказывается возможным лишь для некоторых, вполне определенных длин волн \(\lambda\) ) каждому такому значению \(\lambda\) соответствует свой максимум (и направления \(\alpha, \beta, \gamma\) на него).

И еще одно важное обстоятельство. Если длина волны \(\lambda\) больше максимальной разности хода колебаний от соседних узлов решетки ( \(\lambda>2 d\) ), то должны отсутствовать все дифракционные максимумы, кроме нулевого. Излучение с такими длинами волн распространяется в среде, не испытывая дифракции (рассеяния).

Постоянные кристаллических решеток значительно меньше длины волны видимого света ( \(d \sim 0,5 \mathrm{Hм}, \lambda \sim 500 \mathrm{Hм}\) ), поэтому для видимого света кристаллы являются оптически однородными средами (не рассеивающими). Для рентгеновского же излучения, наоборот, кристаллы представляют естественные дифракционные решетки.

Формула Брэгга-Вульфа. Условия Лауэ (5.34) выражают физическую сущность явления дифракции на пространственной структуре. Но мы не будем более подробно останавливаться на анализе результатов, которые можно получить с помощью этих условий, поскольку существует более простой метод расчета дифракции рентгеновского излучения в кристалле. Этот метод был предложен Брэггом и Вульфом: дифракцию рентгеновского излучения в кристалле можно рассматривать как результат зеркального отражения от системы параллельных кристаллических плоскостей, т. е. плоскостей, в которых лежат узлы кристаллической решетки. Вторичные волны, отразившись от разных атомных плоскостей, когерентны и будут интерферировать между собой. Показатель преломления всех веществ для рентгеновских лучей близок к единице, поэтому разность хода двух волн, отразившихся зеркально от соседних кристаллических плоскостей, равна, как видно из рис. \(5.32, A B C=2 d \sin \alpha\), где \(d\) - межплоскостное расстояние, \(\alpha\) - угол скольжения. При этом направления, в которых возникают фраунгоферовы дифракционные максимумы, определяются формулой Брэгга-Вульфа:


\begin{equation*}
2 d \sin \alpha= \pm m \lambda, \quad m=1,2, \ldots \tag{5.36}
\end{equation*}


В кристалле можно провести множество систем атомных плоскостей в различных направлениях (рис. 5.33). Каждая система плоскостей может дать дифракционный максимум, если для нее будет выполнено условие (5.36). Однако эффективными являются только такие плоскости, в которых атомы расположены наиболее плотно.

Заметим, что формула (5.36) может быть получена как следствие условий Лауэ (5.34).

\begin{figure}[h]
\begin{center}
  \includegraphics[alt={},max width=\textwidth]{950fefba-d83b-4c8f-96e7-0bd5a7eb90be-177_319_649_1460_70}
\captionsetup{labelformat=empty}
\caption{Рис. 5.32}
\end{center}
\end{figure}

\begin{figure}[h]
\begin{center}
  \includegraphics[alt={},max width=\textwidth]{950fefba-d83b-4c8f-96e7-0bd5a7eb90be-177_313_422_1464_726}
\captionsetup{labelformat=empty}
\caption{Рис. 5.33}
\end{center}
\end{figure}

Применения дифракции рентгеновских лучей. Дифракция рентгеновских лучей от кристаллов получила развитие в двух направлениях: рентгеновская спектроскопия (исследование спектрального состава этого излучения) и рентгеноструктурный анализ (изучение структуры кристаллов).

Спектральный состав излучения, т. е. измерение его длин волн, можно определить с помощью формулы (5.36), найдя направления на максимумы при дифракции на кристалле с известной структурой.

В рентгеноструктурном анализе разработаны два метода:

\begin{enumerate}
  \item Метод Лауэ, в котором узкий пучок рентгеновского излучения направляется на исследуемый монокристалл. Для каждой системы кристаллических плоскостей в излучении находится длина волны, при которой выполняется условие (5.36). В результате на помещенной за кристаллом фотопластинке получается система пятен-максимумов, так называемая лауэграмма. Взаимное расположение пятен отражает симметрию кристалла. А по расстояниям между максимумами и их интенсивности можно расшифровать структуру данного кристалла.
  \item Метод Дебая-Шерера, в котором используется узкий пучок монохроматического рентгеновского излучениия и образец в виде поликристалла. Исследуемый кристалл предварительно измельчают в порошок (очень мелкие кристаллики), и из него прессуется образец в виде стерженька. В большом количестве беспорядочно ориентированных кристалликов найдется множество таких, для которых условие (5.36) окажется выполненным, и дифрагированный пучок будет образовывать конус направлений - свой для каждой системы межплоскостных расстояний \(d\) и порядка дифракции \(m\). Рентгенограмма образца, полученная по этому методу - дебайграмма - имеет вид системы концентрических колец. Ее расшифровка также позволяет определить структуру кристалла.
\end{enumerate}

\section*{§ 5.10. О голографии}
Голографией называют способ записи и последующего восстановления структуры световых волн, основанный на явлениях дифракции и интерференции когерентных световых пучков. В

отличие от обычной фотографии, голография - это принципиально новый метод получения объемных изображений предметов.

В этом методе регистрируется не оптическое изображение предмета, а интерференционная картина, возникающая при наложении световой волны, рассеянной предметом, и когерентной с ней опорной волны. Эта интерференционная картина фиксирует информацию о распределении не только амплитуд, но и фаз в предметной волне. Таким способом можно получить и зарегистрировать на фотопластинке значительно более полную информацию об объекте, нежели путем обычного фотографирования.

Идеи, лежащие в основе голографии, были высказаны и продемонстрированы на опыте Габором в 1947-48 гг. Для практической реализации голографии необходимы источники света с высокой степенью временно́й и пространственной когерентности. Поэтому широкое распространение она получила после создания лазеров.

Суть этого метода (голографии) можно пояснить с помощью рис. 5.34, \(a\). Фотопластинка \(\Phi\) регистрирует интерференционную картину, возникающую при наложении отраженной предметной волны 1 , рассеянной объектом \(A\), и когерентной с ней опорной волны 2. Волна 2 испускается тем же источником света, который освещает объект \(A\), и после отражения от зеркала 3 падает непосредственно на фотопластинку \(\Phi\). Интерференционную картину, зафиксированную на фотопластинке после ее проявления, называют голограммой. В отличие от обычной фотографии, голограмма представляет собой мелкий и замысловатый узор из чередующихся областей различного почернения фотоэмульсии.

\begin{figure}[h]
\begin{center}
  \includegraphics[alt={},max width=\textwidth]{950fefba-d83b-4c8f-96e7-0bd5a7eb90be-179_380_852_1407_189}
\captionsetup{labelformat=empty}
\caption{Рис. 5.34}
\end{center}
\end{figure}

Голограмма в закодированной форме содержит полную информацию об амплитудах и фазах рассеянной предметной волны.

Восстановление (декодирование) изображения предмета показано на рис. 5.34, б. Голограмму \(\Gamma\) просвечивают как диапозитив той же опорной волной 2 , которая использовалась для ее получения, причем при той же ориентации голограммы по отношению к опорной волне. Эта световая волна дифрагирует на голограмме, в результате чего наблюдаются два объемных изображения объекта. Мнимое изображение \(A^{\prime}\) находится в том месте, где был объект \(A\) при съемке. Оно видно сквозь голограмму как через окно. Действительное изображение \(A^{\prime \prime}\) расположено по другую сторону голограммы. Оно как бы висит в воздухе перед голограммой и является зеркальным изображением объекта.

Обычно пользуются мнимым изображением \(A^{\prime}\), которое по зрительному восприятию практически тождественно самому объекту.

Чтобы понять принцип голографии, рассмотрим один из простейших объектов - светящуюся прямую нить \(A\), которая испускает цилиндрическую волну. На рис. 5.35 (для наглядности он сильно деформирован) показана слева стадия записи голограммы; нить \(A\) перепендикулярна плоскости рисунка.

Пусть плоская опорная волна 2 падает нормально на фотопластинку. Предметная же волна 1 , когерентная с опорной, падает на пластинку так, что ее лучи наклонны (см. рис. 5.35, a). Колебания опорной волны достигают пластинки все в одной фазе. Колебания же предметной волны - с разными фазами.

\begin{figure}[h]
\begin{center}
  \includegraphics[alt={},max width=\textwidth]{950fefba-d83b-4c8f-96e7-0bd5a7eb90be-180_414_1024_1349_102}
\captionsetup{labelformat=empty}
\caption{Рис. 5.35}
\end{center}
\end{figure}

Ясно, что вследствие интерференции в тех местах, куда колебания придут в одной и той же фазе с колебаниями опорной волны, возникнут максимумы интенсивности, а значит и максимумы почернения. После химической обработки полученная голограмма будет иметь вид чередующихся светлых и темных прямых полос, представляющих по существу зонную пластинку Френеля, только с другой формой зон.

На стадии восстановления изображения голограмму освещают только той же опорной волной 2 (см. рис. 5.35, б). Голограмма для этой волны играет роль зонной пластинки: в результате дифракции максимально усиливать друг друга будут только те лучи (колебания), разность хода между которыми от соседних зон равна длине волны \(\lambda\). А здесь есть две возможности выполнения этого условия: или в направлении продолжения лучей падавшей предметной волны, или симметрично относительно нормали. В первом случае мы будем наблюдать мнимое изображение \(A^{\prime}\), во втором - действительное \(A^{\prime \prime}\). Наибольший интерес для голографии играет волна, дающая мнимое изображение предмета.

Отличительной особенностью этой зонной пластинки является то, что ее пропускательная способность изменяется в направлении, перпендикулярном интерференционным полосам (зонам) не скачком, а практически по синусоидальному закону (между соседними максимумами). Это приводит к тому, что возникающие интерференционные максимумы соответствуют разности хода между колебаниями от соседних полос (зон) только в одну длину волны \(\lambda\). Интенсивность максимумов более высокого порядка практически равна нулю.

Нетрудно заметить, что изменение положения нити \(A\) приводит к изменению не только интенсивности интерференционных максимумов, но и к расстоянию между ними. Расстояние же между максимумами характеризует определенную разность хода, или разность фаз.

Поэтому и говорят, что голограмма содержит информацию об амплитуде волны и ее фазе. Этого достаточно, чтобы восстановить предметную волну в том виде, в каком она была в действительности в месте расположения голограммы. Зафиксированная волна «оживает» с помощью голограммы и создает полную

иллюзию реальности наблюдаемых предметов. Изменяя положение глаза, можно видеть предмет в разных ракурсах и даже заглядывать за него.

Интерференционная картина в каждой точке голограммы определяется светом, рассеянным всеми точками объекта. Поэтому каждый участок голограммы содержит информацию обо всем объекте. Если голограмма случайно разбилась, то с помощью даже небольшого ее осколка можно восстановить изображение всего объекта. Отличие будет лишь в том, что уменьшается ее разрешающая способность - менее четко и ярко будет восстанавливаться изображение. С точки зрения надежности хранения информации голограмма значительно превосходит обычный фотонегатив.

Наконец, на одной фотопластинке можно последовательно записать несколько голограмм от разных объектов, причем изображение каждого объекта можно восстановить без помех со стороны других изображений. Это можно сделать, изменяя каждый раз, например, угол падения опорной волны.

В настоящее время голография представляет самостоятельный быстро развивающийся раздел науки и техники. Трудно даже перечислить области, где она нашла эффективное использование (включая и акустические голограммы).

\section*{Задачи}
5.1. Дифракция Френеля от круглого отверстия. Между точечным источником света и экраном находится диафрагма с круглым отверстием, радиус \(r\) которого можно изменять. Расстояния от диафрагмы до источника и экрана равны соответственно \(a\) и \(b\). Найти длину волны света, если максимум освещенности в центре дифракционной картины на экране наблюдается при радиусе отверстия \(r_{1}\) и следующий максимум - при \(r_{2}\left(r_{2}>r_{1}\right)\).\\
Решение. Максимумы соответствуют нечетному числу зон Френеля. Воспользовавшись формулой (5.5), запишем

\[
r_{2}^{2}-r_{1}^{2}=\left(m_{2}-m_{1}\right) \lambda \frac{a b}{a+b}
\]

Имея в виду, что в нашем случае \(m_{2}-m_{1}=2\), получим:

\[
\lambda=\frac{r_{2}^{2}-r_{1}^{2}}{2 a b}(a+b) .
\]

5.2. Плоская световая волна с \(\lambda=0,64\) мкм и интенсивностью \(I_{0}\) падает нормально на круглое отверстие радиуса \(r=1,2\) мм. Найти интенсивность в центре дифракционной картины на экране, отстоящем от отверстия на расстояние \(b=1,5 \mathrm{~m}\).

\begin{figure}[h]
\begin{center}
  \includegraphics[alt={},max width=\textwidth]{950fefba-d83b-4c8f-96e7-0bd5a7eb90be-183_315_313_327_132}
\captionsetup{labelformat=empty}
\caption{Рис. 5.36}
\end{center}
\end{figure}

Решение. Прежде всего вычислим число \(m\) зон Френеля, укладывающихся в данном отверстии. Согласно формуле (5.6)

\[
m=r^{2} / \lambda b=1,5 .
\]

Это значение \(m\) соответствует вектору \(\mathbf{A}\) на рис. 5.36, где приведена «действующая» часть первого витка спирали Френеля. Из этого рисунка сразу видно, что \(A \approx \sqrt{2} A_{0}\), а значит интенсивность \(I=2 I_{0}\).\\
5.3. Плоская световая волна с \(\lambda=0,60\) мкм падает нормально на достаточно большую стеклянную пластинку, на обратной стороне которой сделана круглая выемка (рис. 5.37). Для точки наблюдения \(P\) она представляет собой первые полторы зоны Френеля. Найти глубину \(h\) выемки, при которой интенсивность света в точке \(P\) будет:\\
a) максимальной;\\
б) минимальной.

Решение. Прежде всего изобразим на первом витке спирали Френеля интересующие нас векторы в отсутствие выемки (рис. 5.38), где изображенные векторы соответствуют амплитудам колебаний: от всей волновой поверхности ( \(\mathbf{A}_{\infty}\) ), от первых полутора зон Френеля ( \(\mathbf{A}_{1,5}\) ) и от всех остальных ( \(\mathbf{A}_{\text {ост }}\) ). Видно, что \(\mathbf{A}_{\infty}=\mathbf{A}_{1,5}+\mathbf{A}_{\text {ост }}\).

\begin{figure}[h]
\begin{center}
  \includegraphics[alt={},max width=\textwidth]{950fefba-d83b-4c8f-96e7-0bd5a7eb90be-183_410_372_1335_156}
\captionsetup{labelformat=empty}
\caption{Рис. 5.37}
\end{center}
\end{figure}

\begin{figure}[h]
\begin{center}
  \includegraphics[alt={},max width=\textwidth]{950fefba-d83b-4c8f-96e7-0bd5a7eb90be-183_320_314_1393_719}
\captionsetup{labelformat=empty}
\caption{Рис. 5.38}
\end{center}
\end{figure}

Теперь представим себе, что мы начали постепенно делать выемку - увеличивать \(h\). Это приведет к тому, что колебания, проходящие через выемку, начнут опережать по фазе, поскольку их оптический путь уменьшится на \(\Delta=h(n-1)\), что соответствует сдвигу по фазе на \(\delta=2 \pi \Delta / \lambda\) и повороту вектора \(\mathbf{A}_{1,5}\) на этот угол по часовой стрелке. Напомним, что отставание по фазе мы условились характеризовать поворотом против часовой стрелки, значит опережение - по часовой стрелке.

\begin{figure}[h]
\begin{center}
  \includegraphics[alt={},max width=\textwidth]{950fefba-d83b-4c8f-96e7-0bd5a7eb90be-184_318_924_492_153}
\captionsetup{labelformat=empty}
\caption{Рис. 5.39}
\end{center}
\end{figure}

a) Для получения максимума интенсивности, а значит и амплитуды, надо, чтобы вектор \(\mathbf{A}_{1,5}\) оказался сонаправленным с вектором \(\mathbf{A}_{\text {ост }}\). Для этого его следует повернуть, как показано на рис. 5.39, \(a\), на угол \(\delta=(3 / 4) \pi+2 \pi m\), где \(m=0,1, \ldots\) Итак, из условия \(\delta=2 \pi \Delta / \lambda\) получим

\[
\frac{3}{4} \pi+2 \pi m=2 \pi \frac{h(n-1)}{\lambda},
\]

откуда следует, что

\[
h=\frac{\lambda}{n-1}\left(m+\frac{3}{8}\right) .
\]

б) Для получения минимума нужно, чтобы вектор \(\mathbf{A}_{1,5}\) оказался противоположно направленным вектору \(\mathbf{A}_{\text {ост }}\). Из рис. 5.39 , б видно, что для этого его надо повернуть на угол \(\delta=(7 / 4) \pi+2 \pi m, m= =0,1,2, \ldots\) Следовательно,

\[
\frac{7}{4} \pi+2 \pi m=2 \pi \frac{h(n-1)}{\lambda},
\]

откуда искомая глубина выемки

\[
h=\frac{\lambda}{n-1}\left(m+\frac{7}{8}\right) .
\]

5.4. Дифракция Фраунгофера от щели. Плоская световая волна с \(\lambda=0,60\) мкм падает нормально на грань стеклянного клина с углом раствора \(\theta=15^{\circ}\) и показателем преломления \(n=1,5\). На противоположной непрозрачной грани клина имеется прозрачная щель шириной \(b=10\) мкм, параллельная ребру клина. Пренебрегая отражениями, найти:\\
a) угол \(\vartheta_{0}\) между направлением на центральный фраунгоферов максимум и направлением падающего света;\\
б) угловую ширину центрального максимума.

Решение. а) Для центрального максимума (максимума порядка \(m=0\) ) оптические пути всех лучей от одной пунктирной прямой до другой (рис. 5.40) должны быть одинаковы. Это значит,

\begin{figure}[h]
\begin{center}
  \includegraphics[alt={},max width=\textwidth]{950fefba-d83b-4c8f-96e7-0bd5a7eb90be-185_379_478_681_376}
\captionsetup{labelformat=empty}
\caption{Рис. 5.40}
\end{center}
\end{figure}

что оптическая длина ломаного отрезка \(A B\) не должна зависеть от \(x\), т. е.

\[
n \cdot x \sin \theta+(b-x) \sin \varphi=\text { const. }
\]

Сгруппируем слагаемые, содержащие \(x\), тогда

\[
(n \cdot \sin \theta-\sin \varphi) x+b \sin \varphi=\text { const. }
\]

Чтобы левая часть этого равенства не зависела от \(x\), выражение в скобках перед \(x\) должно быть равно нулю. Отсюда

\[
n \sin \theta=\sin \varphi .
\]

Так как \(\varphi=\theta+\vartheta_{0}\), то искомый угол

\[
\vartheta_{0}=\arcsin (n \sin \theta)-\theta \approx 8^{\circ} .
\]

б) Условие минимумов, ближайших к максимуму нулевого порядка, должно соответствовать разности хода колебаний от краев щели в одну длину волны, или согласно рис. 5.40

\[
b(\sin \varphi-n \sin \theta)= \pm \lambda .
\]

Отсюда при знаке «+» находим \(\varphi_{+}\), а при знаке «-» \(\varphi_{-}\), и значит

\[
\Delta \varphi=\varphi_{+}-\varphi_{-}=\vartheta_{+}-\vartheta_{-}=\Delta \vartheta=7,3^{\circ} .
\]

5.5. Дифракционая решетка. Свет с длиной волны \(\lambda=535\) нм падает нормально на прозрачную дифракционную решетку. Найти ее период, если один из фраунгоферовых максимумов возникает под углом дифракции \(\vartheta_{m}=35^{\circ}\) и наибольший порядок максимума равен пяти.

Решение. Запишем условия главных максимумов:


\begin{gather*}
d \sin \vartheta_{m}=m \lambda  \tag{1}\\
d \sin \vartheta_{\text {макс }}=5 \lambda \tag{2}
\end{gather*}


Из этих формул получим


\begin{equation*}
m=5 \frac{\sin \vartheta_{m}}{\sin \vartheta_{\text {макс }}} \tag{3}
\end{equation*}


Для вычисления \(m\) надо, чтобы значение \(\sin \vartheta_{\text {макс }}\) было равным как можно ближе к единице, но таким, чтобы \(m\) было при этом цельл. Это будет, как нетрудно установить, при \(\sin \vartheta_{\text {макс }}=0,955\). Тогда \(m=3\), и из формулы (1) найдем искомый период:

\[
d=\frac{m \lambda}{\sin \vartheta_{m}}=\frac{3 \cdot 0,535}{0,573} \mathrm{MKM}=2,8 \mathrm{мКм}
\]

5.6. Свет с длиной волны \(\lambda=0,53\) мкм падает на прозрачную дифракционную решетку с периодом \(d=1,50\) мкм. Найти угол с нормалью к решетке, под которым образуется максимум наибольшего порядка, если свет падает на решетку под углом \(\vartheta_{0}=60^{\circ}\) к нормали.\\
Решение. Максимуму наибольшего порядка должна отвечать максимальная разность хода \(\Delta\) между соответствующими лучами 1 и 2 от соседних щелей, как показано на рис. 5.41. Для этого надо, чтобы

\begin{figure}[h]
\begin{center}
  \includegraphics[alt={},max width=\textwidth]{950fefba-d83b-4c8f-96e7-0bd5a7eb90be-186_257_411_1459_752}
\captionsetup{labelformat=empty}
\caption{Рис. 5.41}
\end{center}
\end{figure}


\begin{equation*}
\Delta=A B+B C=d\left(\sin \vartheta_{0}+\sin \vartheta_{m}\right)=m \lambda . \tag{1}
\end{equation*}


Сначала найдем максимальное значение \(m\). Учитывая, что \(\vartheta_{m}\) не может превосходить \(\pi / 2\), положим \(\sin \vartheta_{m}=1\). Тогда


\begin{equation*}
m_{\text {макс }}=\left[\frac{d}{\lambda}\left(\sin \vartheta_{0}+1\right)\right]=[5,3]=5, \tag{2}
\end{equation*}


где квадратные скобки означают, что от полученного числового значения надо взять только целую часть.

Остается подставить \(m_{\text {макс }}\) в \((1)\), и мы получим:

\[
\sin \vartheta_{m}=m \lambda / d-\sin \vartheta_{0}=0,9 .
\]

Откуда \(\vartheta_{m}=64^{\circ}\).\\
5.7. Разрешающая способность дифракционной решетки. Свет, содержащий две спектральные линии одинаковой интенсивности и с длинами волн \(\lambda_{1}=600,00\) нм и \(\lambda_{2}=600,05\) нм падает нормально на дифракционную решетку шириной \(h=10,0\) мм. Найти угол Э, под которым эти линии окажутся на пределе разрешения (в соответствии с критерием Рэлея).

Решение. Из критерия Рэлея следует, что разрешающая способность \(\lambda / \delta \lambda=m N\). С другой стороны, условие главных фраунгоферовых максимумов утверждает, что \(d \sin \vartheta_{m}=m \lambda\). Из этих двух соотношений находим искомый угол:

\[
\sin \vartheta_{m}=\frac{\lambda^{2}}{h \delta \lambda}=0,72, \quad \vartheta_{m}=46^{\circ},
\]

где учтено, что ширина решетки \(h=N d\).\\
5.8. Разрешающая способность и угловая дисперсия решетки. Свет падает нормально на дифракционную решетку, ширина которой \(h=20\) мм. При достаточно малых углах дифракции, когда \(\cos \vartheta \approx 1\), угловая дисперсия решетки \(D=5,0\) угл. мин/нм. Найти (согласно критерию Рэлея) максимально возможную разрешающую способность решетки в этих условиях.

Решение. В соответствии с критерием Рэлея разрешающая способность (5.32) \(\lambda / \delta \lambda=m N\). Угловая же дисперсия (5.30) \(D=m / d\), поскольку в нашем случае \(\cos \vartheta \approx 1\). Из этих двух формул находим

\[
\frac{\lambda}{\delta \lambda}=\frac{m}{d} N d=D h=\frac{5 \cdot 10^{9}}{60 \cdot 57,3} 20 \cdot 10^{-3}=3 \cdot 10^{4} .
\]

5.9. Дифракция рентгеновского излучения. Узкий пучок рентгеновских лучей с длиной волны \(\lambda\) падает под углом скольжения \(\alpha=60^{\circ}\) на естественную грань монокристалла NaCl , плотность которого \(\rho=2,16 г / \mathrm{cm}^{3}\). При зеркальном отражении от этой грани образуется максимум второго порядка ( \(m=2\) ). Определить \(\lambda\).\\
Решение. Согласно формуле Брэгга-Вульфа (5.36)

\[
\lambda=2 d \sin \alpha / m .
\]

Как видно, решение задачи сводится к нахождению межплоскостного расстояния \(d\). Последнее равно ребру элементарной кубической ячейки, в узлах которой попеременно находятся атомы Na и Cl (рис. 5.42). На такую ячейку от каждого атома приходится \(1 / 8\) часть. Значит, на всю ячейку приходится \(4 \cdot(1 / 8)=1 / 2\) атома Na и \(1 / 2\) атома Cl , т. е. масса \(M\), приходящаяся на объем \(d^{3}\), равна половине массы \(M_{\text {м }}\) молеку-

\begin{figure}[h]
\begin{center}
  \includegraphics[alt={},max width=\textwidth]{950fefba-d83b-4c8f-96e7-0bd5a7eb90be-188_273_298_538_862}
\captionsetup{labelformat=empty}
\caption{Рис. 5.42}
\end{center}
\end{figure}

лы \(\mathrm{NaCl}: M=M_{\mathrm{M}} / 2\).

Плотность - это отношение массы, приходящейся на каждую ячейку, к ее объему: \(\rho=M / d^{3}\). Отсюда

\[
d=\sqrt[3]{M / \rho}=\sqrt[3]{M_{\mathrm{M}} / 2 \rho} .
\]

В результате получим:

\[
\lambda=\frac{2}{m}\left(\frac{M_{\mathrm{M}}}{2 \rho}\right)^{1 / 3} \cdot \sin \alpha=\frac{2}{2}\left(\frac{58,45}{6,025 \cdot 10^{23} \cdot 2 \cdot 2,16}\right)^{1 / 3} \cdot 0,866=244 \text { пм. }
\]

5.10. Дебайграмма. При прохождении узкого пучка рентгеновских лучей с длиной волны \(\lambda=17,8\) пм через поликристаллический образец, на экране, расположенном на расстоянии \(l=15\) см от образца, возникает система концентрических дифракционных ко-лец-максимумов. Определить радиус светлого кольца, соответствующего второму порядку отражения от системы плоскостей с межплоскостным расстоянием \(d=155\) пм.

P e ш e н и е. Согласно формуле Брэгга-Вульфа


\begin{equation*}
2 d \sin \alpha=m \lambda . \tag{1}
\end{equation*}


Угол скольжения \(\alpha\) найдем с помощью рис. 5.43, откуда видно, что


\begin{equation*}
r=l \operatorname{tg} 2 \alpha . \tag{2}
\end{equation*}


\begin{figure}[h]
\begin{center}
  \includegraphics[alt={},max width=\textwidth]{950fefba-d83b-4c8f-96e7-0bd5a7eb90be-189_327_606_150_313}
\captionsetup{labelformat=empty}
\caption{Рис. 5.43}
\end{center}
\end{figure}

Вычислим согласно (1) угол \(\alpha\) :

\[
\sin \alpha=\frac{m \lambda}{2 d}=0,1148, \quad \alpha=6^{\circ} 35^{\prime} .
\]

Это значение \(\alpha\) подставим в (2), и мы получим:

\[
r=l \cdot \operatorname{tg} 13^{\circ} 10^{\prime}=15 \cdot 0,234=3,5 \text { см. }
\]

\section*{Поляризация света}
\section*{§ 6.1. Общие сведения о поляризации}
Виды поляризации. Волну, в которой направление колебаний светового вектора \(\mathbf{E}\) упорядочено каким-либо образом, называют поляризованной. Если колебания вектора \(\mathbf{E}\) происходят только в одной плоскости, проходящей через луч, то мы имеем дело с плоско- (или линейно-) поляризованной волной. Плоскость, в которой колеблется вектор Е, называют плоскостъю поляризации* (плоскостью колебаний светового вектора).

Другой вид поляризации заключается в том, что вектор E вращается вокруг направления распространения волны одновременно изменяясь периодически по модулю. При этом конец вектора \(\mathbf{E}\) описывает эллипс (в каждой точке среды). Такую волну называют эллиптически-поляризованной. Или поляризованной по кругу, если конец вектора \(\mathbf{E}\) описывает окружность.

В зависимости от направления вектора \(\mathbf{E}\) различают правую и левую эллиптические (или круговые) поляризации. Если смотреть навстречу распространения волны, и вектор \(\mathbf{E}\) при этом поворачивается по часовой стрелке, то поляризацию называют правой, в противном случае (если против часовой стрелки) - левой.

Эллиптически-поляризованная - это наиболее общий вид поляризации волны, переходящий при определенных условиях в линейную и круговую поляризации.

Волну с эллиптической поляризацией всегда можно разложить (или представить) на две взаимно перпендикулярные ли-нейно-поляризованные волны с взаимно ортогональными плоскостями поляризации. Причем разность фаз этих двух волн сохраняется постоянной во времени. Такие волны, как мы знаем,

\footnotetext{\begin{itemize}
  \item Первоначально плоскость поляризации связывали с плоскостью, в которой колеблется вектор H (или B). Несмотря на то, что это устарело, следует быть внимательным, поскольку в некоторых учебниках до сих пор понятие «плоскость поляризации» используется в прежнем смысле.
\end{itemize}
}называют когерентными, в отличие от некогерентных, у которых разность фаз хаотически меняется во времени. Более подробно вопрос об эллиптической поляризации мы рассмотрим в § 6.3.

Естественный свет. В случае радиоволн создать линейно- или эллиптически-поляризованную волну не представляет никакого труда. Иначе обстоит дело с электромагнитными волнами оптического диапазона ввиду особой специфики физических процессов, лежащих в природе излучения таких волн (см. § 3.1).

В основном мы будем рассматривать обычные (не лазерные) источники оптического излучения (свечение раскаленных твердых тел, свечение возбужденных атомов газа). При использовании же лазеров мы должны по-иному подходить к рассмотрению многих оптических явлений в силу особых свойств их излучения: оно в высокой степени монохроматично и линейно поляризовано.

Несмотря на то, что световые волны и от обычных источников поперечны, они, как правило, не обнаруживают асимметрии по отношению к направлению распространения. Такой свет называют естественным. В естественном свете колебания вектора Е в любой (фиксированной) точке среды совершаются в разных направлениях, быстро и беспорядочно сменяя друг друга. Условно это изображают, как показано на рис. 6.1 слева (направление распространения света перпендикулярно плоскости рисунка).

Естественный свет можно представить как наложение (сумму) двух некогерентных плоскополяризованных волн с взаимно ортогональными плоскостями поляризации, что и показано на рис. 6.1 справа. Кстати, этим приемом мы будем пользоваться в дальнейшем неоднократно, чтобы значительно упростить анализ многих вопросов. Важно подчеркнуть, что ориентация этих двух взаимно ортогональных плоскополяризованных волн совершенно несущественна.

\begin{figure}[h]
\begin{center}
  \includegraphics[alt={},max width=\textwidth]{950fefba-d83b-4c8f-96e7-0bd5a7eb90be-191_228_550_1557_340}
\captionsetup{labelformat=empty}
\caption{Рис. 6.1}
\end{center}
\end{figure}

Поляризаторы. Из естественного света можно получить плоскополяризованный с помощью приборов, называемых поляризаторами. Эти приборы свободно пропускают колебания светового вектора, параллельные плоскости, которую мы будем называть плоскостью пропускания поляризатора. Колебания же, перпендикулярные к этой плоскости, задерживаются полностью или частично. В первом случае поляризатор является идеальньлм. В дальнейшем мы будем иметь в виду именно такие (идеальные) поляризаторы, если не будет каких-либо оговорок.

Степень поляризации. Помимо плоскополяризованного и естественного света существует еще «промежуточный» случай это частично-поляризованный свет. Частично-поляризованный свет, как и естественный, можно представить в виде наложения двух некогерентных плоскополяризованных волн с взаимно перпендикулярными плоскостями поляризации, но разными по интенсивности (рис. 6.2). Его также можно рассматривать как смесь (сумму) естественной ( \(e c m\) ) и плоскополяризованной (пол) составляющих, как показано на этом рисунке справа.

\begin{figure}[h]
\begin{center}
  \includegraphics[alt={},max width=\textwidth]{950fefba-d83b-4c8f-96e7-0bd5a7eb90be-192_234_614_955_309}
\captionsetup{labelformat=empty}
\caption{Рис.6.2}
\end{center}
\end{figure}

Из рис. 6.2 видно, что вертикальные колебания соответствуют максимальной интенсивности, горизонтальные - минимальной ( \(I_{\text {макс }}\) и \(I_{\text {мин }}\) ). Это можно обнаружить с помощью поляризатора.

Частично-поляризованный свет характеризуют степенью поляризации \(P\), которую определяют как


\begin{equation*}
P=\frac{I_{\text {макс }}-I_{\text {мин }}}{I_{\text {макс }}+I_{\text {мин }}}=\frac{I_{\text {пол }}}{I_{0}} . \tag{6.1}
\end{equation*}


Здесь \(I_{\text {пол }}\) - интенсивность поляризованной составляющей, \(I_{0}\) - полная интенсивность частично-поляризованного света: \(I_{0}=I_{\text {макс }}+I_{\text {мин }}\).

Для плоскополяризованного света ( \(I_{\text {пол }}=I_{0}\) ) степень поляризации \(P=1\), для естественного света ( \(I_{\text {пол }}=0\) ) \(P=0\). Это два крайних случая.

Заметим, что для эллиптически-поляризованного света понятие «степень поляризации», а значит и формула (6.1), не применимы.

Закон Малюса. Поляризаторы можно использовать и в качестве анализаторов - для определения характера и степени поляризации интересующего нас света.

\begin{figure}[h]
\begin{center}
  \includegraphics[alt={},max width=\textwidth]{950fefba-d83b-4c8f-96e7-0bd5a7eb90be-193_358_182_568_147}
\captionsetup{labelformat=empty}
\caption{Рис. 6.3}
\end{center}
\end{figure}

Пусть на анализатор падает линейно-поляризованный свет, вектор \(\mathbf{E}_{0}\) которого составляет угол \(\varphi\) с плоскостью пропускания \(P\) (рис. 6.3, где направление светового пучка перпендикулярно к плоскости рисунка). Анализатор пропускает только ту составляющую вектора \(\mathbf{E}_{0}\), которая параллельна плоскости пропускания \(P\), т. е. \(E=E_{0} \cos \varphi\). Интенсивность пропорциональна квадрату модуля светового вектора ( \(I \sim E^{2}\) ), поэтому интенсивность прошедшего света


\begin{equation*}
I=I_{0} \cos ^{2} \varphi, \tag{6.2}
\end{equation*}


где \(I_{0}\) - интенсивность падающего плоскополяризованного света. Это соотношение и выражает собой закон Малюса.\\
Пример. На систему, состоящую из двух поляризаторов, плоскости пропускания которых повернуты относительно друг друга на угол \(\varphi\), падает естественный свет интенсивности \(I_{0}\). Пренебрегая отражениями, найдем интенсивность \(I\) света, прошедшего сквозь эту систему.\\
Первый поляризатор пропускает, как легко сообразить, половину падающего на него света, т. е. \(I_{0} / 2\), но это будет уже плоскополяризованный свет. Второй поляризатор согласно закону Малюса (6.2) пропустит

\[
I=\left(I_{0} / 2\right) \cos ^{2} \varphi .
\]

Это и есть искомый результат.\\
О деполяризации. Иногда возникает обратная задача: поляризованный свет превратить в естественный, т. е. неполяризованный. Это

можно достигнуть следующим образом: приготовить слой из мелко истолченного стекла (порошка). Поляризованный свет, проходящий через такой слой, претерпевает многократные отражения, в результате чего его поляризация будет разрушена.

Такую же роль играет, например, калька (полупрозрачная восковая бумага). Достаточно поместить ее за поляризатором, чтобы с помощью вращения второго поляризатора (анализатора) убедиться, что калька почти полностью деполяризует свет, поляризованный первым поляризатором: интенсивность проходящего через второй поляризатор света практически не зависит от его угла поворота.

\section*{§ 6.2. Поляризация при отражении и преломлении}
Закон Брюстера. Если угол падения естественного света на границу раздела двух прозрачных диэлектриков отличен от нуля, то отраженный и преломленный пучки оказываются час-тично-поляризованными. В отраженном свете преобладают колебания вектора \(\mathbf{E}\), перпендикулярные к плоскости падения, а в преломленном свете - параллельные плоскости падения. Степень поляризации обеих волн (отраженной и преломленной) зависит от угла падения.

При некотором значении угла падения отраженный свет становится полностью поляризованным, и его плоскость поляризации (плоскость колебаний вектора E) оказывается перпендикулярной к плоскости падения. Этот угол \(\vartheta_{\text {Бр }}\) удовлетворяет следующему условию:


\begin{equation*}
\operatorname{tg} \vartheta_{\text {Бр }}=n_{2} / n_{1} . \tag{6.3}
\end{equation*}


Данное соотношение называют законом Брюстера, а угол \(\vartheta_{\text {Бр }}\) - углом Брюстера или углом полной поляризации. Здесь \(n_{2} / n_{1}\) - отношение показателей преломления второй среды и первой (рис. 6.4). Точками и черточками на отраженном и преломленном лучах этого рисунка показаны направления колебаний вектора Е.

\begin{figure}[h]
\begin{center}
  \includegraphics[alt={},max width=\textwidth]{950fefba-d83b-4c8f-96e7-0bd5a7eb90be-194_267_426_1448_740}
\captionsetup{labelformat=empty}
\caption{Рис. 6.4}
\end{center}
\end{figure}

Можно убедиться (см. задачу 3.4), что при падении света под углом Брюстера отраженный и преломленный лучи взаимно ортогональны.

При падении естественного света под углом Брюстера на границу раздела двух прозрачных диэлектриков преломленная волна становится частично-поляризованной, причем степень поляризации ее оказывается максимальной. В связи с этим вопросом (углом Брюстера и степенью поляризации) рассмотрим два примера.\\
Пример 1. Естественный свет падает под углом Брюстера на поверхность воды. При этом \(\rho\)-часть падающего света отражается. Найдем степень поляризации \(P\) преломленного света.

Пусть световой поток падающего света равен \(\Phi_{0}\). Степень поляризации преломленного света, по определению

\[
P=\frac{\Phi_{\text {пол }}}{\Phi_{\text {прел }}}=\frac{\rho \Phi_{0}}{\Phi_{0}-\rho \Phi_{0}}=\frac{\rho}{1-\rho},
\]

где \(\Phi_{\text {прел }}\) - световой поток преломленного света.\\
Пример 2. Частично-поляризованный свет падает под углом Брюстера на поверхность прозрачного диэлектрика так, что \(\rho\)-часть света отражается, и преломленный свет оказывается естественным. Найдем степень поляризации падающего света.\\
Обозначим падающий частично-поляризованный световой поток через \(\Phi_{0}\). Его степень поляризации можно представить следующим образом:

\[
P=\frac{\Phi_{\text {п }}}{\Phi_{\text {чп }}}=\frac{\rho \Phi_{0}}{\Phi_{0}}=\rho,
\]

где \(\Phi_{\text {п }}\) - поляризованная часть падающего светового потока, \(\Phi_{\text {чп }}\) - световой поток падающего частично-поляризованного света.

Пример 3. Естественный свет падает под углом Брюстера на поверхность прозрачной плоскопараллельной пластинки. Покажем, что преломленный свет падает на заднюю поверхность пластинки тоже под углом Брюстера.\\
Воспользуемся тем, что при падении света под углом Брюстера угол между отраженным и преломленным лучами должен быть прямым. На рис. 6.5 угол \(\vartheta_{1}=\vartheta_{\text {Бр }}\), и легко видеть,

\begin{displayquote}
что если \(\pi-\left(\vartheta_{1}+\vartheta_{2}\right)=\pi / 2\), то при отражении и от нижней плоскости пластинки угол между лучами \(2^{\prime}\) и \(2^{\prime \prime}\) будет тоже \(\pi / 2\), т. е. луч \(2^{\prime}\) будет также ли-нейно-поляризованным. На рисунке точками отмечено, что вектор Е в обоих отраженных лучах колеблется перпендикулярно плоскости падения.
\end{displayquote}

\begin{figure}[h]
\begin{center}
  \includegraphics[alt={},max width=\textwidth]{950fefba-d83b-4c8f-96e7-0bd5a7eb90be-196_338_414_161_746}
\captionsetup{labelformat=empty}
\caption{Рис. 6.5}
\end{center}
\end{figure}

О поляризации преломленного света. Степень поляризации преломленной волны при угле падения, равном углу Брюстера, достигает максимального значения, однако эта волна остается лишь частично-поляризованной.

Так как коэффициент отражения света в данном случае значительно меньше единицы (около 0,15 для границы раздела воздух - стекло), можно использовать преломленный свет, повышая его степень поляризации путем ряда последовательных отражений и преломлений. Это осуществляют с помощью стопы, состоящей из нескольких одинаковых и параллельных друг другу пластинок, установленных под углом Брюстера к падающему свету. При достаточно большом числе пластинок проходящий через эту систему свет будет практически полностью линей-но-поляризованным. И интенсивность прошедшего через такую стопу света (в отсутствие поглощения) будет равна половине интенсивности падающего на стопу естественного света.

Эта идея нашла высокоэффективное использование в газовых лазерах, где торцы разрядной трубки представляют собой плоскопараллельные стеклянные пластинки, расположенные под углом Брюстера к оси трубки (рис. 6.6). Поэтому излучение, распространяющееся вдоль оси трубки между зеркалами и поляризованное в плоскости падения на пластинки, многократно проходит сквозь них практически беспрепятственно, не ис-

\begin{figure}[h]
\begin{center}
  \includegraphics[alt={},max width=\textwidth]{950fefba-d83b-4c8f-96e7-0bd5a7eb90be-196_168_725_1603_252}
\captionsetup{labelformat=empty}
\caption{Рис. 6.6}
\end{center}
\end{figure}

пытывая отражения. В результате из лазера выходит луч, поляризованный в этой плоскости, что и показано на рисунке. Другая составляющая излучения, плоскость поляризации которой перпендикулярна плоскости падения, почти полностью удаляется из пучка благодаря отражениям.

О формулах Френеля. С помощью граничных условий для векторов \(\mathbf{E}\) и \(\mathbf{H}\) можно найти соотношения между амплитудами и фазами падающей, отраженной и преломленной волн - так называемые формуль Френеля. При необходимости с ними можно познакомиться во многих учебниках и справочниках.

Мы не будем выписывать эти формулы, поскольку для решения наших вопросов они нам не понадобятся. Важно отметить только, что с помощью этих формул можно показать, что при произвольном угле падения \(\vartheta_{1}\) (и соответствующем ему углу преломления \(\vartheta_{2}\) ) коэффициенты отражения линейно-поляризованного света, плоскость поляризации которого перпендикулярна плоскости падения ( \(\rho_{\perp}\) ) и параллельна ей ( \(\rho_{\|}\)), определяются следующими выражениями:


\begin{equation*}
\rho_{\perp}=\frac{I_{\perp}^{\prime}}{I_{\perp}}=\frac{\sin ^{2}\left(\vartheta_{1}-\vartheta_{2}\right)}{\sin ^{2}\left(\vartheta_{1}+\vartheta_{2}\right)}, \quad \rho_{\|}=\frac{I_{\|}^{\prime}}{I_{\|}}=\frac{\operatorname{tg}^{2}\left(\vartheta_{1}-\vartheta_{2}\right)}{\operatorname{tg}^{2}\left(\vartheta_{1}+\vartheta_{2}\right)} . \tag{6.4}
\end{equation*}


Из этих формул, кстати, видно, что при падении под углом Брюстера, когда \(\vartheta_{1}+\vartheta_{2}=\pi / 2\), и значит, \(\operatorname{tg}(\pi / 2) \rightarrow \infty\), коэффициент отражения \(\rho_{\|}\), т. е. отраженный свет будет полностью линей-но-поляризованным в плоскости, перпендикулярной плоскости падения.

Пример. На поверхность прозрачного диэлектрика с показателем преломления \(n\) падает под углом Брюстера линейно-поляризованный свет, плоскость поляризации которого\\
a) перпендикулярна плоскости падения;\\
б) лежит в плоскости падения.

Найдем коэффициенты отражения света в обоих случаях.\\
Прежде всего представим себе, как будут выглядеть отраженный и преломленный лучи в этих случаях (рис. 6.7). В случае a) согласно (6.4) коэффициент отражения \(\rho_{\perp}=\sin ^{2}\left(\vartheta_{1}-\vartheta_{2}\right)\), где \(\vartheta_{1}\) - угол Брюстера, определяемый формулой \(\operatorname{tg} \vartheta_{1}=n\), а угол \(\vartheta_{2}\) связан с углом \(\vartheta_{1}\) законом преломления: \(\sin \vartheta_{1}=n \sin \vartheta_{2}\). Определив \(\vartheta_{1}\) и \(\vartheta_{2}\), мы найдем и \(\rho_{\perp}\).

\begin{figure}[h]
\begin{center}
  \includegraphics[alt={},max width=\textwidth]{950fefba-d83b-4c8f-96e7-0bd5a7eb90be-198_388_1031_139_99}
\captionsetup{labelformat=empty}
\caption{Рис. 6.7}
\end{center}
\end{figure}

В случае же б) при падении под углом Брюстера тангенс в знаменателе (6.4) обращается в \(\infty\), поскольку \(\vartheta_{1}+\vartheta_{1}=\pi / 2\). Это значит, что в данном случае \(\rho_{\|}=0\), т. е. отраженный луч отсутствует (но это только при падении под углом Брюстера).

\section*{§ 6.3. Поляризация при двойном лучепреломлении}
Двойное лучепреломление. Почти все прозрачные кристаллические диэлектрики оптически анизотропны*, т. е. оптические свойства света при прохождении через них зависят от направления. Вследствие этого возникают явления, называемые двойным лучепреломлением. Оно заключается в том, что падающий на кристалл пучок света разделяется внутри кристалла на два пучка, распространяющиеся, вообще говоря, в различных направлениях и с разными скоростями.

Существуют кристаллы одноосные и двуосные. У одноосных кристаллов один из преломленных пучков подчиняется обычному закону преломления ( \(n_{1} \sin \vartheta_{1}=n_{2} \sin \vartheta_{2}\) ). Его называют обькновенным и обозначают буквой или индексом о. Другой пучок необыкновенный ( \(e\) ), он не подчиняется обычному закону преломления, и даже при нормальном падении светового пучка на поверхность кристалла необыкновенный пучок может отклоняться от нормали (рис. 6.8). И, как правило, необыкновенный луч не лежит в плоскости падения.

\footnotetext{\begin{itemize}
  \item Физическая природа анизотропии вещества связана с особенностями строения его молекул или особенностями самой кристаллической решетки, в узлах которой находятся атомы или ионы.
\end{itemize}
}\begin{figure}[h]
\begin{center}
  \includegraphics[alt={},max width=\textwidth]{950fefba-d83b-4c8f-96e7-0bd5a7eb90be-199_293_372_153_104}
\captionsetup{labelformat=empty}
\caption{Рис. 6.8}
\end{center}
\end{figure}

Наиболее сильно двойное лучепреломление выражено у таких одноосных кристаллов как кварц (кристаллический), исландский шпат и турмалин.

Далее мы ограничимся рассмотрением только одноосных кристаллов.\\
Одноосные кристаллы. У одноосных кристаллов имеется направление - оптическая ось \(O O^{\prime}\), вдоль которого обыкновенная и необыкновенная волны распространяются, не разделяясь пространственно и с одинаковой скоростью*.

Оптическая ось \(O O^{\prime}\) кристалла не является какой-то особой прямой линией. Она характеризует лишь избранное направление в кристалле и может быть проведена через произвольную точку кристалла.

Любую плоскость, проходящую через оптическую ось, называют главньмм сечением или главной плоскостью кристалла. Обычно пользуются главным сечением (плоскостью), проходящим через световой луч в кристалле.

Обыкновенная и необыкновенная волны (и лучи) линейно поляризованы. Колебания вектора \(\mathbf{E}\) в обыкновенной волне совершаются в направлении, перпендикулярном главному сечению кристалла для обыкновенного луча. Колебания же вектора \(\mathbf{E}\) в необыкновенной волне - в главном сечении кристалла для необыкновенного луча. Направления колебаний вектора \(\mathbf{E}\) (т. е. их плоскости поляризации) в обоих пучках показаны на рис. 6.8, где предполагается, что оба пучка и пересекающая их оптическая ось \(O O^{\prime}\) лежат в плоскости рисунка. Видно, что в данном случае плоскости поляризации обеих волн (о и \(e\) ) взаимно ортогональньl. Заметим, что это наблюдается практически при любой ориентации оптической оси, поскольку угол между обыкновенным и необыкновенным лучами достаточно мал.

Оба луча, вышедшие из кристалла, отличаются друг от друга только направлением поляризации, так что названия «обыкновенный» ( \(o\) ) и «необыкновенный» ( \(e\) ) имеют смысл только внутри кристалла.

\footnotetext{\begin{itemize}
  \item У двуосных кристаллов, например слюды, имеются два таких направления.
\end{itemize}
}Дихроизм. Существуют кристаллы, в которых один из лучей ( \(o\) или \(e\) ) поглощается сильнее другого. Это явление и называют дихроизмом. Очень сильный дихроизм присущ кристаллу турмалина (минералу сложного состава). В нем обыкновенный луч практически полностью поглощается на длине около 1 мм.

Явление дихроизма используют для изготовления поляризаторов в виде светофильтров, их называют поляроидами (герапатитовые и др.). Они представляют собой тонкую ( \(\sim 0,1 \mathrm{~mm}\) ) пленку, линейно поляризующую проходящий через нее свет.

О показателях преломления \(\boldsymbol{n}_{o}\) и \(\boldsymbol{n}_{e}\). Анизотропия кристалла по-разному отражается на скорости распространения обыкновенной и необыкновенной волн. В то время как скорость \(v_{o}\) обыкновенного луча не зависит от направления в кристалле, скорость \(v_{e}\) необыкновенного луча по мере отклонения его от направления оптической оси (вдоль которой скорости обоих лучей одинаковы) будет все больше отличаться, достигая максимального различия в направлении, перпендикулярном оптической оси.

Зависимость скорости необькновенного луча от направления связано с анизотропией кристалла, приводящей к тому, что диэлектрическая постоянная \(\varepsilon\), а значит и показатель преломления ( \(n=\sqrt{\varepsilon}\) ) оказываются разными для обыкновенного и необыкновенного лучей и существенно зависят от направления луча относительно оптической оси кристалла.

Одноосные кристаллы характеризуют показателем преломления обыкновенного луча \(n_{o}=c / v_{o}\) и показателем преломления необыкновенного луча, перпендикулярного оптической оси, \(n_{e}=c / v_{e}\). В таблицах приводят именно эти значения \(n_{o}\) и \(n_{e}\). Для света с длиной волны \(\lambda=0,55\) мкм эти значения таковы:

\begin{center}
\begin{tabular}{|l|l|l|l|}
\hline
кристалл & \(n_{o}\) & \(n_{e}\) & \(\left|n_{o}-n_{e}\right|\) \\
\hline
исландский шпат & 1,66 & 1,49 & 0,17 \\
кварц & 1,545 & 1,554 & 0,009 \\
\hline
\end{tabular}
\end{center}

Заметим, что значения \(n_{o}\) и \(n_{e}\) несколько зависят от \(\lambda\).

Пример. Имеются две пластинки, вырезанные параллельно оптической оси: одна - из исландского шпата толщиной \(h=1,0 \mathrm{~mm}\), другая - из кварца. Найдем толщину \(h^{\prime}\) кварцевой пластинки, чтобы для света с длиной волны 0,55 мкм она имела такую же разность хода обыкновенного и необыкновенного лучей. В чем при этом будет их различие?

Исходя из определения оптической разности хода, запишем

\[
h\left(n_{o}-n_{e}\right)=h^{\prime}\left(n_{e}^{\prime}-n_{o}^{\prime}\right) .
\]

Воспользовавшись данными из приведенной таблицы, найдем:

\[
h^{\prime}=h\left(\Delta n / \Delta n^{\prime}\right)=1 \cdot 0,17 / 0,009 \approx 19 \text { мм. }
\]

Различие этих двух пластинок, кроме толщины, в том, что при выходе из исландского шпата опережать будет необыкновенный луч ( \(e\) ), а при выходе из кварца - обыкновенный ( \(o\) ).\\
Поверхности лучевых скоростей. Тот факт, что скорость распространения обыкновенного луча в одноосном кристалле одинакова во всех напрвлениях, а скорость распространения необыкновенного луча зависит от направления, можно наглядно представить с помощью поверхностей лучевых скоростей (их называют еще и волновыми поверхностями). Для их построения из произвольной точки \(\boldsymbol{S}\) проводят во всевозможных направлениях лучи и откладывают на них отрезки, пропорциональные значениям лучевой скорости в этих направлениях. Множество концов отложенных отрезков образует замкнутую поверхность, которая для обыкновенной волны представляет собой сферу радиусом \(v_{o}\), а для необыкновенной волны - эллипсоид вращения с полуосями \(v_{o}\) и \(v_{e}\).

В зависимости от того, какая из скоростей, \(v_{o}\) или \(v_{e}\), больше, одноосные кристаллы подразделяют на положительнье и отрицательные (рис. 6.9, где \(O O^{\prime}\) - оптическая ось). У положительных кристаллов \(v_{o}>v_{e}\) (и значит, \(n_{o}<n_{e}\) ), а у отрицательных кристаллов \(v_{o}<v_{e}\) (т. е. \(n_{o}>n_{e}\) ). Из этого рисунка видно, что при распространении света вдоль оптической оси обе волны имеют одинаковую скорость, равную скорости обыкновенной волны.

Более подробное рассмотрение вопросов, связанных с оптической анизотропией кристаллов - это задача курса кристал-

\begin{figure}[h]
\begin{center}
  \includegraphics[alt={},max width=\textwidth]{950fefba-d83b-4c8f-96e7-0bd5a7eb90be-202_352_683_144_273}
\captionsetup{labelformat=empty}
\caption{Рис. 6.9}
\end{center}
\end{figure}

лооптики и выходит за рамки курса общей физики. Нас в дальнейшем будут интересовать только два практически важных частных случая, когда свет распространяется перпендикулярно и параллельно оптической оси.

\section*{§ 6.4. Суперпозиция поляризованных волн}
Двупреломляющая пластинка. Исследуем более подробно случай, представляющий наибольший практический интерес, когда свет распространяется в одноосном кристалле перпендикулярно оптической оси кристалла. При этом условии пространственного разделения обыкновенных и необыкновенных лучей не будет.

Пусть на кристаллическую пластинку, вырезанную параллельно оптической оси \(O O^{\prime}\) (рис. 6.10), падает нормально линей-но-поляризованный свет, плоскость поляризации которого составляет угол φ с оптической осью пластинки (обычно делают так, чтобы угол φ был равен \(45^{\circ}\) по причинам, которые станут ясными далее). В этом случае в кристалле в одном и том же на-

\begin{figure}[h]
\begin{center}
  \includegraphics[alt={},max width=\textwidth]{950fefba-d83b-4c8f-96e7-0bd5a7eb90be-202_295_421_1433_147}
\captionsetup{labelformat=empty}
\caption{Рис. 6.10}
\end{center}
\end{figure}

\begin{figure}[h]
\begin{center}
  \includegraphics[alt={},max width=\textwidth]{950fefba-d83b-4c8f-96e7-0bd5a7eb90be-202_404_434_1373_669}
\captionsetup{labelformat=empty}
\caption{Рис. 6.11}
\end{center}
\end{figure}

правлении (перпендикулярно оптической оси) будут распространяться две волны с разными скоростями ( \(v_{o}=c / n_{o}\) и \(v_{e}=c / n_{e}\) ), поляризованные взаимно ортогонально. Это показано на рис. 6.11, где свет распространяется перпендикулярно к плоскости рисунка. На этом рисунке \(P\) - плоскость поляризации падающего на пластинку света, \(\mathbf{E}\) - его вектор-амплитуда, \(O O^{\prime}\) оптическая ось кристалла, \(\mathbf{E}_{o}\) и \(\mathbf{E}_{e}\) - векторы-амплитуды обыкновенной и необыкновенной волн в кристаллической пластинке.

В зависимости от толщины \(h\) пластинки обе волны выходят из пластинки с той или иной разностью фаз \(\delta\), которая зависит от оптической разности хода данных волн, равной


\begin{equation*}
\Delta=h\left(n_{o}-n_{e}\right) . \tag{6.5}
\end{equation*}


Учитывая, что \(\delta=2 \pi \Delta / \lambda\), где \(\lambda\) - длина волны в вакууме, получим:


\begin{equation*}
\delta=2 \pi \frac{h\left(n_{o}-n_{e}\right)}{\lambda} . \tag{6.6}
\end{equation*}


Итак, из кристаллической пластинки выходят две взаимно ортогональные плоскополяризованные волны: одна поляризована перпендикулярно к главному сечению кристалла, другая - в плоскости этого сечения (см. рис. 6.10). Это значит, что в произвольной точке за кристаллической пластинкой соответствующие колебания светового вектора будут выглядеть так:


\begin{align*}
& E_{o x}=E_{o} \cos \omega t \\
& E_{e y}=E_{e} \cos (\omega t+\delta) \tag{6.7}
\end{align*}


Далее мы будем предполагать (но в эксперименте в этом надо убедиться), что оптическая разность хода обыкновенной и необыкновенной волн меньше длины когерентности (лучше, чтобы \(\Delta<l_{\text {ког }} / 2\) ), тогда обе волны можно считать когерентными* и их разность фаз \(\delta\) не зависящей от времени.

\footnotetext{\begin{itemize}
  \item Оценим толщину \(h\) пластинки, при которой это условие выполняется. Пусть \(\lambda \approx 0,60\) мкм и \(\Delta \lambda \approx 0,010\) мкм, тогда \(l_{\text {ког }} \approx \lambda^{2} / \Delta \lambda=36\) мкм и
\end{itemize}
}Характер поляризации результирующей волны будет зависеть от толщины данной кристаллической пластинки, т. е. в конечном счете от разности фаз \(\delta\). В общем случае уравнения (6.7) - это уравнение эллипса в параметрической форме. Следовательно, вышедший из кри-

\begin{figure}[h]
\begin{center}
  \includegraphics[alt={},max width=\textwidth]{950fefba-d83b-4c8f-96e7-0bd5a7eb90be-204_321_419_153_713}
\captionsetup{labelformat=empty}
\caption{Рис. 6.12}
\end{center}
\end{figure}

Таким образом, сложение двух взаимно ортогональных когерентных колебаний приводит в общем случае к эллиптической поляризации. Вид и ориентация эллипса зависят от отношения амплитуд ортогональных колебаний, \(\mathbf{E}_{o}\) и \(\mathbf{E}_{e}\), а также и от их разности фаз \(\delta\).

Справедливо и обратное утверждение: эллиптически-поляризованный свет всегда можно представить как сумму двух взаимно ортогональных когерентных колебаний.

В зависимости от значений амплитуд \(\mathbf{E}_{o}, \mathbf{E}_{e}\) и разности фаз \(\delta\) эллиптически-поляризованый свет может вырождаться в плоскополяризованный или поляризованный по кругу (подробнее этот вопрос мы рассмотрим ниже).

На рис. 6.11 ось \(Z\) направлена на нас, предполагается, что и световая волна распространяется в направлении оси \(Z\), т. е. тоже на нас. Глядя на рисунок, мы смотрим навстречу волне. Условились, что если при наблюдении навстречу волне вращение вектора \(\mathbf{E}\) в фиксированной точке пространства происходит по часовой стрелке, то такую волну называть правополяризованной, если же против часовой стрелки, то - левополяризованной. Рис. 6.12 соответствует правополяризованной волне.

Заметим, что направление поляризации (правая или левая) зависит от разности фаз \(\delta\), точнее от того, колебание по какой оси ( \(X\) или \(Y\) ) опережает - это при условии, что в формуле (6.7) \(\delta\) выбрано так, чтобы его значение было заключено между \(-\pi\) и \(+\pi\). Последнее можно сделать всегда, исключая из \(\delta\) целое число \(2 \pi\) - это ничего не меняет.

Если колебание опережает по оси \(Y(0<\delta<\pi)\), то сначала \(E_{y}\) и лишь затем \(E_{x}\) достигают максимальных значений. Следовательно, движение конца вектора \(\mathbf{E}\) будет происходить по часовой стрелке - поляризация будет правой.

Если же колебание по оси \(Y\) отстает ( \(0>\delta>-\pi\) ), то сначала \(E_{x}\) и лишь затем \(E_{y}\) достигают максимальных значений - движение конца вектора \(\mathbf{E}\) будет против часовой стрелки, значит поляризация левая.

В зависимости от толщины \(h\) данной кристаллической пластинки, вырезанной параллельно оптической оси, характер поляризации прошедшего через нее света будет отличаться друг от друга. Рассмотрим несколько частных случаев, представляющих наибольший практический интерес.

\begin{enumerate}
  \item Пластинка в четверть волньи (короче \(\lambda / 4\) ) - это пластинка, толщина \(h\) которой удовлетворяет условию
\end{enumerate}


\begin{equation*}
h\left|n_{o}-n_{e}\right|=m \frac{\lambda}{4}, \quad \text { где } m=1,3,5, \ldots, \tag{6.8}
\end{equation*}


т. е. при нечетных значениях \(m\). Эта пластинка согласно (6.6) вносит дополнительную разность фаз \(\delta=\pi / 2\) (точнее, \(m \pi / 2, m\) - нечетное) между проходящими через нее обыкновенной и необыкновенной волнами, поляризованными во взаимно перпендикулярных плоскостях. Эти плоскости определяют в пластинке два направления, называемые главными направлениями пластинки: одна параллельна оптической оси, другая - перпендикулярна ей.

\begin{figure}[h]
\begin{center}
  \includegraphics[alt={},max width=\textwidth]{950fefba-d83b-4c8f-96e7-0bd5a7eb90be-205_302_370_1258_68}
\captionsetup{labelformat=empty}
\caption{Рис. 6.13}
\end{center}
\end{figure}

При таких значениях \(\delta\) свет, прошедший через пластинку, будет в соответствии с формулой (6.7) эллиптически-поляризованным, причем эллипс будет приведен к осям \(X\) и \(Y\) (рис. 6.13, где ось \(O O^{\prime}\) - оптическая ось). Направления вращения вектора \(\mathbf{E}\) чередуются: если, например, при \(m=1\) по часовой стрелке, то при \(m=3\) против и т. д.\\
Если линейно-поляризованный свет падает на пластинку так, что угол между его плоскостью поляризации \(P\) и оптической осью пластинки \(\varphi=45^{\circ}\) (см. рис. 6.11), то амплитуды обыкновенной и необыкновенной волн будут одинаковы, и на выходе из пластин-

ки мы получим круго-поляризованный свет - эллипс вырождается в окружность (рис. 6.14).

Пластинки в \(\lambda / 4\) изготовляют обычно из кварца. Качество пластинки (насколько она соответствует четвертьволновой) можно проверить, исследуя с помощью поляроида (анализатора) выходящий из нее круго-поляризованный свет: враще-

\begin{figure}[h]
\begin{center}
  \includegraphics[alt={},max width=\textwidth]{950fefba-d83b-4c8f-96e7-0bd5a7eb90be-206_332_338_151_816}
\captionsetup{labelformat=empty}
\caption{Рис. 6.14}
\end{center}
\end{figure}

Пластинку в \(\lambda / 4\) можно использовать и для обратного превращения круго-поляризованного света в линейно-поляризованный. Плоскость поляризации выходящего света при этом составляет угол \(45^{\circ}\) с оптической осью пластинки.\\
2. Пластинка в польолньи (короче \(\lambda / 2\) ). Ее толщина \(h\) удовлетворяет условию


\begin{equation*}
h\left|n_{o}-n_{e}\right|=m \frac{\lambda}{2}, \quad \text { где } m=1,3,5, \ldots, \tag{6.9}
\end{equation*}


т. е. тоже при нечетных значениях \(m\). На выходе из такой пластинки между обыкновенной и необыкновенной волнами возникает согласно (6.6) дополнительная разность фаз \(\delta=\pi\) (точнее \(m \pi, m\) - нечетное). Это значит, что свет, вышедший из пластинки, остается линейно-поляризованным, однако направление колебаний вектора \(\mathbf{E}\) (плоскость поляризации) повернется на угол \(2 φ\) симметрично главному сечению пластинки (рис. 6.15). При \(\varphi=45^{\circ}\) такая пластинка «поворачивает» плоскость поляризации на \(90^{\circ}\), т. е. плоскость поляризации прошедшего через пластинку света будет ортогональна плоскости поляризации падающего света. Необходимость такого «поворота» нередко встречается в экспериментальных установках.

Следует также обратить внимание на то, что при четных значениях \(m(2,4, \ldots)\) в формуле (6.9) прошедший через пластинку свет остается линейно-поляризованным в той же плоскости, что и падающий свет. Т. е. такая пластинка ничего нового не вносит. Ее часто называют пластинкой в целую волну.

\begin{figure}[h]
\begin{center}
  \includegraphics[alt={},max width=\textwidth]{950fefba-d83b-4c8f-96e7-0bd5a7eb90be-207_301_875_153_127}
\captionsetup{labelformat=empty}
\caption{Рис. 6.15}
\end{center}
\end{figure}

Рис. 6.16\\
3. Компенсатор. Для анализа поляризованного света наряду с пластинкой \(\lambda / 4\) используют устройства, позволяющие скомпенсировать до нуля (или дополнить до \(\pi\) ) любую разность фаз между двумя взаимно ортогональными колебаниями. Это так называемые компенсаторы.

Простейший компенсатор состоит из двух кварцевых клиньев (рис.6.16). Вместе они образуют кристаллическую пластинку, оптическая ось которой параллельна ее граням (это показано штриховкой на рисунке). Один из клиньев можно перемещать относительно другого с помощью микрометрического винта, изменяя таким образом толщину компенсатора-пластинки, а значит и вносимую этой системой разность фаз между обыкновенной и необыкновенной волнами.

Если на компенсатор падает нормально линейно-поляризованный свет, плоскость которого составляет угол \(45^{\circ}\) с его оптической осью, то с увеличением толщины компенсатора растет и разность фаз взаимно ортогональных волн, проходящих через него. В результате мы будем наблюдать картину (рис. 6.17), показывающую как меняется характер поляризации прошедшего света с ростом разности фаз \(\delta\). Напомним, что согласно (6.7) значения \(\delta\) на рис. 6.17 определяют насколько (по фазе) колебания \(E_{y}\) опережают колебания \(E_{x}\) (опережают - в алгебраическом смысле).

\begin{figure}[h]
\begin{center}
  \includegraphics[alt={},max width=\textwidth]{950fefba-d83b-4c8f-96e7-0bd5a7eb90be-207_173_1098_1584_67}
\captionsetup{labelformat=empty}
\caption{Рис. 6.17}
\end{center}
\end{figure}

Из рис. 6.17 видно, что направление поляризации, как было отмечено ранее, периодически меняется. Если сначала было правое, т. е. по часовой стрелке (для наблюдателя), то затем с ростом толщины \(h\) компенсатора-пластинки, а значит и \(\delta\), оно переходит в левое, т. е. против часовой стрелки и т. д.

Анализ поляризованного света. В практике часто встречается необходимость выяснить, каков характер поляризации исследуемого света, как отличается характер поляризации одного света от другого. Рассмотрим несколько наиболее типичных случаев.

\begin{enumerate}
  \item Свет плоскополяризованный. В этом можно убедиться с помощью любого поляризатора: при вращении его плоскости пропускания вокруг направления пучка интенсивность проходящего света будет изменяться, и при некотором положении свет полностью гасится. Значит, исследуемый свет плоскополяризованный.
  \item Свет естественный и поляризованный по кругу. Как их различить? Для этого одного поляризатора недостаточно: в обоих случаях при вращении его плоскости пропускания вокруг напрвления пучка интенсивность проходящего света не меняется. Если же предварительно ввести пластинку \(\lambda / 4\), то поляризованный по кругу свет превращается в плоскополяризованный, поскольку эта пластинка вносит дополнительную разность фаз \(\pm \pi / 2\). Результирующая разность фаз окажется равной 0 или \(\pi\), и свет станет плоскополяризованным. Его можно погасить поляризатором. Если же свет естественный, то он останется таковым и после прохождения пластинки \(\lambda / 4\). В этом случае погасить свет не удастся: при любом положении плоскости пропускания поляризатора интенсивность прошедшего света будет одинаковой.
  \item Свет эллиптически-поляризованный и частично-поляризованный. Для их различия опять следует поместить в световом пучке пластинку \(\lambda / 4\), а за ней поляризатор. Если вращением пластинки вокруг направления пучка найдется такое положение, при котором свет, прошедший через нее, можно погасить, вращая поляризатор, то свет эллиптически-поляризованный. Если это сделать не удастся, то свет частично-поляризованный.
\end{enumerate}

Конечно, есть и более сложные случаи, но мы их рассматривать не будем, поскольку это связано лишь с техническим усложнением процедуры определения.\\
4. Право- и левополяризованный свет. Для отличия правой круговой поляризации от левой поступают аналогично: на пути светового пучка ставят пластинку \(\lambda / 4\), на которой указано направление колебаний, распространяющихся с бо́льшей скоростью, его называют \(\beta\)-направлением. Эти колебания опережают по фазе на \(\pi / 2\) колебания, ортогональные \(\beta\)-направлению.

\begin{figure}[h]
\begin{center}
  \includegraphics[alt={},max width=\textwidth]{950fefba-d83b-4c8f-96e7-0bd5a7eb90be-209_415_764_427_233}
\captionsetup{labelformat=empty}
\caption{Рис. 6.18}
\end{center}
\end{figure}

Допустим для определенности, что поляризация правая (рис. 6.18, \(a\), где оси \(X\) и \(Y\) выбраны произвольно), т. е. вращение вектора \(\mathbf{E}\) происходит по часовой стрелке (для наблюдателя). В этом случае перед пластинкой колебания \(E_{y}\) опережают колебания \(E_{x}\) по фазе на \(\pi / 2\) (это было объяснено на с. 204). Установив пластинку \(\lambda / 4\) в пучок, мы к имеющейся разности фаз \(\pi / 2\) добавим еще \(\pi / 2\). В результате получим, что разность фаз между ортогональными колебаниями окажется равной \(\pi\), а это значит, что свет станет плоскополяризованным, причем его плоскость поляризации будет повернута влево относительно \(\beta\)-направления пластинки \(\lambda / 4\) на угол \(45^{\circ}\).

Если же поляризация была левая (рис. 6.18, б), то колебания \(E_{y}\) отстают от колебаний \(E_{x}\) по фазе на \(\pi / 2\). Совместив \(\beta\)-направление нашей пластинки с произвольно выбранной осью \(Y\), мы компенсируем отставание на \(\pi / 2\), и в результате разность фаз окажется равной нулю, т. е. из пластинки будет выходить свет, плоскость поляризации которого повернута вправо относительно \(\beta\)-направления (рис. 6.18, б) тоже на угол \(45^{\circ}\).

Направление плоскости поляризации в обоих случаях легко найти с помощью поляризатора.

\section*{§ 6.5. Интерференция поляризованных волн}
Когерентность поляризованных волн. Если на кристаллическую пластинку, вырезанную параллельно оптической оси, нормально направить пучок естественного света, то из пластинки выйдут две волны с взаимно ортогональными плоскостями поляризации. Естественный свет - результат излучения различных независимых атомов источника света, испускающих отдельные некоррелированные друг с другом цуги волн. Эти цуги участвуют в образовании обыкновенной и необыкновенной волн в кристалле. Однако вклад каждого отдельного цуга в эти две волны, вообще говоря, не одинаков. Этот вклад больше в ту волну, плоскость поляризации которой составляет меньший угол с плоскостью поляризации данного цуга.

Другими словами, обыкновенная и необыкновенная волны в основном порождаются разными цугами, входящими в состав естественного света. Поэтому обыкновенная и необыкновенная волны, распространяющиеся в одноосном кристалле и выходящие из него (при падении естественного света), некогерентны.

Однако обе волны можно сделать когерентными, если на пути естественного света установить поляризатор \(P\) перед кристаллической пластинкой, причем так, чтобы плоскость его пропускания составляла некоторый угол \(\varphi \neq 0\) с оптической осью кристалла (обычно угол φ делают равным \(45^{\circ}\) ). В этом случае колебания каждого цуга разделяются между обыкновенной \(o\) и необыкновенной \(e\) волнами. Именно поэтому волны \(o\) и \(e\) оказываютя когерентными - необходимое условие для их интерференции.

Интерференция поляризованных волн. Сказанного еще недостаточно, если мы задались целью наблюдать интерференцию этих волн. Дело в том, что интерференция никогда не наблюдается, если складываемые волны поляризованы во взаимно перпендикулярных плоскостях. Выход простой: поставить на пути вышедшего из пластинки света еще один поляризатор. Он сведет два взаимно ортогональных когерентных колебания к одной плоскости. Интерференция будет обеспечена. Ее результат окажется в зависимости от оптической разности хода складываемых волн.

Итак, схема наблюдения интерференции поляризованных волн должна быть такой, как показано на рис. 6.19. Здесь \(S\) -

\begin{figure}[h]
\begin{center}
  \includegraphics[alt={},max width=\textwidth]{950fefba-d83b-4c8f-96e7-0bd5a7eb90be-211_180_322_153_161}
\captionsetup{labelformat=empty}
\caption{Рис. 6.19}
\end{center}
\end{figure}

обычный источник света, \(P\) поляризатор, \(K\) - кристаллическая одноосная пластинка, \(P^{\prime}\) второй поляризатор. Заметим, что если источник - лазер (он испускает уже плоскополяризованный свет), то необходимость в поляризаторе \(P\) отпадает.

Далее мы рассмотрим вопрос об интенсивности \(I^{\prime}\) света, прошедшего через эту систему в двух наиболее простых и практически важных случаях, связанных с взаимной ориентацией плоскостей пропускания поляризаторов \(P, P^{\prime}\) и оптической оси \(O O^{\prime}\). Но предварительно напомним, что картина интерференции бывает наиболее отчетливой, когда амплитуды складываемых волн одинаковы. В нашем случае это означает, что угол φ между плоскостью пропускания поляризатора \(P\) и оптической осью \(O O^{\prime}\) должен быть равным \(45^{\circ}\) (о чем уже говорилось). В дальнейшем, если не будет специальных оговорок, будет предполагаться именно это значение угла \(\varphi, \varphi=45^{\circ}\).

Теперь перейдем к рассмотрению двух частных случаев, когда плоскости пропускания обоих поляризаторов параллельны друг другу ( \(P^{\prime} \| P\) ) и взаимно перпендикулярны ( \(P^{\prime} \perp P\) ). В последнем случае говорят, что поляризаторы скрешеньι. Оба случая показаны на рис. 6.20 , где свет распространяется перпендикулярно плоскости рисунков.

\begin{figure}[h]
\begin{center}
  \includegraphics[alt={},max width=\textwidth]{950fefba-d83b-4c8f-96e7-0bd5a7eb90be-211_398_878_1206_181}
\captionsetup{labelformat=empty}
\caption{Рис. 6.20}
\end{center}
\end{figure}

\begin{enumerate}
  \item Случай \(\boldsymbol{P}^{\prime} \| \boldsymbol{P}\) (рис. 6.20, a). Здесь плоскополяризованная волна с амплитудой \(E\) (после поляризатора \(P\) ) разделяется пластинкой на обыкновенную и необыкновенную взаимно ортогональные волны с одинаковыми амплитудами \(E_{o}\) и \(E_{e}\). Затем ко-
\end{enumerate}

лебания этих волн приводятся поляризатором \(P^{\prime}\) к одной плоскости с одинаковыми амплитудами \(E_{o}^{\prime}\) и \(E_{e}^{\prime}\) :


\begin{equation*}
E_{o}^{\prime}=E_{e}^{\prime}=E / 2 . \tag{6.10}
\end{equation*}


Результат интерференции этих волн будет зависеть, как уже говорилось, от разности фаз \(\delta\), которую они приобретут в пластинке. С этой целью изобразим фазовую (векторную) диаграмму, показанную на рис. 6.21. Здесь пред-

\begin{figure}[h]
\begin{center}
  \includegraphics[alt={},max width=\textwidth]{950fefba-d83b-4c8f-96e7-0bd5a7eb90be-212_193_386_318_770}
\captionsetup{labelformat=empty}
\caption{Рис. 6.21}
\end{center}
\end{figure}

положено, что в кристаллической пластинке отстает по фазе на \(\delta\) обыкновенная волна (это не существенно, могло быть и наоборот). Нас интересует \(E^{\prime 2}\), поскольку именно эта величина определяет интенсивность прошедшей через поляризатор \(P^{\prime}\) волны. Из рис. 6.21 согласно теореме косинусов с учетом (6.10) следует, что


\begin{equation*}
E^{\prime 2}=2\left(\frac{E}{2}\right)^{2}+2\left(\frac{E}{2}\right)^{2} \cos \delta=E^{2} \frac{1+\cos \delta}{2}=E^{2} \cos ^{2} \frac{\delta}{2} \tag{6.11}
\end{equation*}


Таким образом, при \(P^{\prime} \| P\) интенсивность прошедшего света


\begin{equation*}
I_{\|}^{\prime}=I \cos ^{2} \frac{\delta}{2} \tag{6.12}
\end{equation*}


\begin{enumerate}
  \setcounter{enumi}{1}
  \item Случай \(\boldsymbol{P}^{\prime} \perp \boldsymbol{P}\) (рис. 6.20, б). Здесь следует отметить, что \(E_{o}^{\prime}=E_{e}^{\prime}\) при любых значениях угла \(\varphi\), но при \(\varphi=45^{\circ}\) обе амплитуды будут максимальны (в этом легко убедиться с помощью этого же рисунка), и, значит, результат интерференции будет выглядеть наиболее отчетливым. Так что и в этом случае оптимальным является \(\varphi=45^{\circ}\).
\end{enumerate}

Рис. 6.20, б достаточно ясно показывает, что происходит с проходящим светом в этом случае. Но здесь надо обратить внимание на тот факт, что векторы \(E_{o}^{\prime}\) и \(E_{e}^{\prime}\) направлены взаимно противоположно (даже при \(\delta \rightarrow 0\) ). Это наводит на мысль, что, кроме разности фаз \(\delta\), вносимой пластинкой, надо добавить еще \(\pi\), которая обусловлена скрещенным расположением поляризаторов (это можно строго доказать и математически).

Тогда в формуле (6.12) надо вместо \(\delta\) написать \(\delta+\pi\), и мы получим вместо косинуса синус. В результате


\begin{equation*}
I_{\perp}^{\prime}=I \sin ^{2} \frac{\delta}{2} . \tag{6.13}
\end{equation*}


Из формул (6.12) и (6.13) следует, что интенсивности \(I_{\|}^{\prime}\) и \(I_{\perp}^{\prime}\) оказываются «дополнительными»: в сумме они дают интенсивность \(I\) света, прошедшего через поляризатор \(P\).

Если свет монохроматический и толщина кристаллической пластинки всюду одинакова, мы получим на выходе равномерную освещенность без характерных для интерференционной картины чередующихся светлых и темных полос. Здесь интерференция проявляет себя в перераспределении световой энергии между взаимно ортогональными плоскостями. Действительно, если например при параллельных плоскостях пропускания поляризаторов мы получаем максимум освещенности, то достаточно повернуть поляризатор \(P^{\prime}\) на \(90^{\circ}\), и мы получим «дополнительную» освещенность: поле окажется темным. То же будет и наоборот.

Интенсивность выходящего из поляризатора \(P^{\prime}\) света можно изменять, изменяя разность фаз \(\delta\). Поскольку \(\delta\) определяется как


\begin{equation*}
\delta=2 \pi \frac{h\left|n_{o}-n_{e}\right|}{\lambda}, \tag{6.14}
\end{equation*}


то изменения \(\delta\) можно достигнуть либо меняя \(\lambda\) - это приводит к эффектному изменению окраски (т. е. максимумы пропускания будут соответствовать различным длинам волн), либо меняя толщину \(h\) пластинки. Последнее можно сделать, поставив вместо пластинки компенсатор (см. рис. 6.16).

Приведем сводную табличку (6.15), где указаны условия, при которых интенсивности \(I_{\|}^{\prime}\) и \(I_{\perp}^{\prime}\) достигают максимальных и минимальных значений:

\begin{center}
\begin{tabular}{|c|l|c|}
\hline
Разность хода \(\Delta\) & \(m \lambda\) & \(m^{\prime} \lambda / 2\) \\
\hline
Разность фаз \(\delta\) & \(2 \pi m\) & \(m^{\prime} \pi\) \\
\hline
\(I_{\|}^{\prime}=I \cos ^{2} \frac{\delta}{2}\) & макс & мин \\
\(I_{\perp}^{\prime}=I \sin ^{2} \frac{\delta}{2}\) & мин & макс \\
\hline
\end{tabular}
\end{center}

Здесь \(m=1,2,3, \ldots\), а \(m^{\prime}=1,3,5, \ldots\), т. е. нечетные.

В этой табличке достаточно запомнить результаты для \(I_{\|}^{\prime}\), а они сразу следуют из фазовой диаграммы (см. рис. 6.21). Результаты для \(I_{\perp}^{\prime}\) «дополнительные», т. е. противоположные.

Отметим, что во втором случае, когда поляризаторы скрещены ( \(P^{\prime} \perp P\) ), установка весьма чувствительна \(\kappa\) обнаружению анизотропии (двойного лучепреломления). Через два скрещенных поляризатора свет не проходит, и поле зрения оказывается темным. Если же между ними ввести какой-либо анизотропный кристалл, то даже при наличии слабой анизотропии система пропускает свет, и поле зрения просветляется. При этом надо проявлять осмотрительность: если случайно окажется, что оптическая ось кристалла будет параллельна или перпендикулярна плоскости пропускания поляризатора \(P\), поле зрения останется темным. Поэтому ориентацию кристалла между скрещенными поляризаторами \(P\) и \(P^{\prime}\), вообще говоря, надо менять во избежание этой случайности.

До сих пор мы рассматривали интерференцию в плоскопараллельной пластинке, где интерференция проявляла себя в изменении интенсивности равномерно освещенного поля зрения (после поляризатора \(P^{\prime}\) ). Но можно наблюдать интерференцию и в привычном виде чередующихся светлых и темных полос.

Пример. Поместим между двумя скрещенными поляризаторами кварцевый клин, оптическая ось которого параллельна его ребру и составляет угол \(45^{\circ}\) с плоскостями пропускания поляризаторов. Выясним, что мы будем наблюдать при прохождении монохроматического света через эту систему.\\
По мере увеличения толщины клина мы будем наблюдать переход от одной светлой полосы к другой, т. е. систему чередующихся светлых и темных полос, параллельных ребру клина (рис. 6.22). Каждая светлая полоса соответствует полуволновой пластинке, значит в этих местах происходит по-

\begin{figure}[h]
\begin{center}
  \includegraphics[alt={},max width=\textwidth]{950fefba-d83b-4c8f-96e7-0bd5a7eb90be-214_274_406_1298_758}
\captionsetup{labelformat=empty}
\caption{Рис. 6.22}
\end{center}
\end{figure}

ворот плоскости поляризации на \(90^{\circ}\), и свет проходит через поляризатор \(P^{\prime}\) (по условию \(P^{\prime} \perp P\) ). Переход к каждому следующему максимуму соответствует изменению оптической разности хода \(\Delta=\Delta h\left(n_{o}-n_{e}\right)\) на одну длину волны \(\lambda\). Это позволяет легко найти, например, угол между гранями клина.

\begin{displayquote}
Заметим, что в белом свете картина будет весьма красочной: она будет состоять из разных оттенков, периодически повторяющихся в пространстве вдоль клина.
\end{displayquote}

\section*{§ 6.6. Искусственное двойное лучепреломление}
Обычные прозрачные тела, не обладающие двойным лучепреломлением, тем не менее при определенном воздействии на них становятся двупреломляющими. Рассмотрим два наиболее характерных способа получения искусственного двойного лучепреломления.

Анизотропия при деформациях. При одностороннем сжатии или растяжении направление деформации становится выделенным и играет роль оптической оси. Тело становится анизотропным и двупреломляющим, разность показателей преломления которого


\begin{equation*}
n_{o}-n_{e}=k \sigma, \tag{6.16}
\end{equation*}


где \(\sigma\) - напряжение ( \(П а=\mathrm{H} / \mathrm{m}^{2}\) ), \(k\) - коэффициент, зависящий от свойств вещества.

Для наблюдения двойного лучепре-

\begin{figure}[h]
\begin{center}
  \includegraphics[alt={},max width=\textwidth]{950fefba-d83b-4c8f-96e7-0bd5a7eb90be-215_223_402_1044_68}
\captionsetup{labelformat=empty}
\caption{Рис. 6.23}
\end{center}
\end{figure}

ломления исследуемое тело помещают между скрещенными поляризаторами, плоскости пропускания которых составляют угол \(45^{\circ}\) с направлением деформации (рис. 6.23). Если тело имеет вид пластинки или кубика, то при увеличении напряжения наблюдают усиление и ослабление прошедшего света.

Если же тело имеет вид клина или другой более сложной формы, то в проходящем свете наблюдается картина в виде системы так или иначе расположенных полос с максимумами и минимумами освещенности. При изменении напряжения картина меняется. Этим пользуются при исследовании распределения напряжений в сложных телах (конструкциях): изготавливают геометрически подобную модель из подходящего прозрачного материала, подвергают ее нагрузке и по наблюдаемой между скрещенными поляризаторами картине судят о распределении внутренних напряжений. Этот метод значительно

упрощает весьма трудоемкую работу по расчету напряжений в новых конструкциях.

Заметим, что целлофановая пленка является двупреломляющей. Полиэтиленовые же пленки становятся двупреломляющими только в результате растяжения. Это можно легко проверить на опыте.

Анизотропия в электрическом поле. Возникновение двойного лучепреломления в жидкости и в аморфных телах под воздействием электрического поля - эффект Керра - нашел широкое применение как в науке, так и в технике эксперимента.

\begin{figure}[h]
\begin{center}
  \includegraphics[alt={},max width=\textwidth]{950fefba-d83b-4c8f-96e7-0bd5a7eb90be-216_212_699_621_266}
\captionsetup{labelformat=empty}
\caption{Рис. 6.24}
\end{center}
\end{figure}

Схема установки для наблюдения этого эффекта показана на рис. 6.24. Между двумя скрещенными поляризаторами \(P\) и \(P^{\prime}\) помещают ячейку Керра - исследуемую жидкость между обкладками конденсатора в кювете.

При создании электрического поля, напряженность \(\mathbf{E}\) которого составляет угол \(45^{\circ}\) с плоскостями пропускания поляризаторов, среда становится оптически анизотропной, двупреломляющей, оптическая ось которой совпадает с направлением вектора Е.

Возникающая разность показателей преломления обыкновенной и необыкновенной волн оказывается при этом равной следующему выражению:


\begin{equation*}
n_{e}-n_{o}=B \lambda E^{2}, \tag{6.17}
\end{equation*}


где \(\lambda\) - длина волны света, \(B\) - постоянная Керра*, имеющая особенно большое значение у нитробензола \(\left(2,2 \cdot 10^{-10} \mathrm{~cm} / \mathrm{B}^{2}\right)\).

\footnotetext{\begin{itemize}
  \item Для большинства веществ постоянная Керра \(B>0\), т. е. \(n_{e}>n_{o}\), что соответствует положительному кристаллу. Но есть и вещества, у которых \(B<0\), т. е. \(n_{e}<n_{o}\) (спирт, этиловый эфир).
\end{itemize}
}На пути \(l\) в конденсаторе между обыкновенной и необыкновенной волнами возникает разность фаз \(\delta=2 \pi \Delta / \lambda\), или с учетом (6.17) выражение для \(\delta\) принимает следующий вид:


\begin{equation*}
\delta=2 \pi B l E^{2} . \tag{6.18}
\end{equation*}


Изменение напряженности \(E\) электрического поля приводит к последовательным просветлениям и затемнениям поля зрения.

Пример. Ячейку Керра - конденсатор с исследуемой жидкостью (см. рис. 6.24) поместили между двумя поляризаторами, плоскости пропускания которых параллельны и составляют угол \(45^{\circ}\) с направлением электрического поля в конденсаторе. Длина конденсатора равна \(l\). Найдем постоянную Керра \(B\) данной жидкости, если минимальное значение напряженности электрического поля, при котором система не пропускает свет, равно \(E_{0}\). Плоскости пропускания поляризаторов \(P\) и \(P^{\prime}\) параллельны, значит при данном \(E_{0}\) жидкость в конденсаторе должна быть пластинкой в \(\lambda / 2\). Поэтому согласно (6.9) и (6.17) можно записать:

\[
l \cdot B \lambda E^{2}=m \lambda / 2,
\]

где значению \(E=E_{0}\) соответствует \(m=1\). Отсюда находим:

\[
B=\frac{1}{2 l E_{0}^{2}}
\]

Эффект Керра объясняется тем, что при включении электрического поля происходит поляризация молекул того же нитробензола и их выстраивание по полю. Это и создает анизотропию вещества с преимущественным направлением - оптической осью - вдоль электрического поля.

Наиболее важной особенностью эффекта Керра, обусловливающей его широкое применение, является весьма малая инерционность (до \(10^{-12}\) с!). Это, в частности, позволяет осуществить практически безинерционный оптический затвор, с помощью которого изучают весьма быстро протекающие процессы. Такой затвор представляет собой по существу «лупу времени».

Кроме того, данный эффект используют для создания сверхкоротких световых импульсов, что позволяет определять, например, скорость света на базе лабораторного стола ( \(\sim 3 м!\) ). Этот эффект используют для управления режимом работы лазеров с целью получения сверхкоротких импульсов огромной мощности и во многих других весьма тонких физических экспериментах.

Изменение оптических свойств кристалла под действием внешнего электрического поля называют электрооптическим эффектом Поккельса. В отличие от эффекта Керра, квадратичного по \(E\), эффект Поккельса зависит линейно от \(E\). Практически безинерционность и данного эффекта позволяет использовать его для создания быстродействующих оптических затворов и высокочастотных модуляторов света (с помощью так называемых пластинок кристалла \(K D P\) ).

\section*{§ 6.7. Вращение направления линейной поляризации}
Естественное вращение. Многие вещества, называемые оптически активными, обладают способностью поворачивать направление поляризации проходящего через них линейно-поляризованного света. Это кристаллические тела (кварц и др.), чистые жидкости (скипидар, никотин и др.) и растворы оптически активных веществ в неактивных растворителях (водные растворы сахара, винной кислоты и др.).

Если на оптически активное вещество падает плоскополяризованный свет, то прошедший через него свет оказывается тоже плоскополяризованным: поворотом анализатора его можно полностью погасить и установить при этом угол φ поворота плоскости поляризации.

В качестве примера возьмем кварцевую пластинку \(K\), вырезанную перпендикулярно оптической оси, и поместим ее между двумя скрещенными поляризаторами \(P\) и \(P^{\prime}\) (рис. 6.25). Система будет пропускать свет (чего не было бы в отсутствие пластинки \(K\) ). Повернув анализатор

\begin{figure}[h]
\begin{center}
  \includegraphics[alt={},max width=\textwidth]{950fefba-d83b-4c8f-96e7-0bd5a7eb90be-218_195_430_1528_730}
\captionsetup{labelformat=empty}
\caption{Рис. 6.25}
\end{center}
\end{figure}

\(P^{\prime}\) на некоторый угол \(\varphi\), обнаруживаем, что система перестала

пропускать свет. Это означает, что в кристалле вектор E повернулся на тот же угол и оказался перепендикулярным плоскости пропускания анализатора \(P^{\prime}\).

Опыт показывает, что все оптически активные вещества поворачивают плоскость поляризации падающего на них света на угол


\begin{equation*}
\varphi=\alpha l, \tag{6.19}
\end{equation*}


где \(l\) - толщина оптически активного слоя, \(\alpha\) - постоянная вращения. Эта постоянная имеет различное значение для разных веществ и, кроме того, сильно зависит от длины волны света. Так, для кварцевой пластинки толщиной в 1 мм углы поворота желтого и фиолетового света равны соответственно \(20^{\circ}\) и \(50^{\circ}\). В ультрафиолете еще больше: при \(\lambda=215\) нм угол \(\varphi=236^{\circ}\). Таким образом, поворот плоскости поляризации света кварцем - это сильный эффект, и его можно легко обнаружить.

В зависимости от направления вращения плоскости поляризации, оптически активные вещества подразделяют на право- и левовращающие, т. е. вращающие по или против часовой стрелки, если смотреть навстречу световому пучку.

Заметим, что все оптически активные вещества существуют в двух разновидностях - право- и левовращающие.

Вращательная способность кварца связана с его кристаллической структурой, так как плавленный кварц не обладает оптической активностью. Для оптически же активных жидкостей и аморфных тел эффект вращения обусловлен асимметрическим строением самих молекул.

\begin{figure}[h]
\begin{center}
  \includegraphics[alt={},max width=\textwidth]{950fefba-d83b-4c8f-96e7-0bd5a7eb90be-219_144_402_1359_68}
\captionsetup{labelformat=empty}
\caption{Рис. 6.26}
\end{center}
\end{figure}

Опыт показывает, что при изменении направления распространения света на противоположное поворот плоскости поляризации происходит в обратную сторону (рис. 6.26). Другими словами, направление вращения (правое или левое) «привязано» к направлению луча. Поэтому при прохождении света сквозь активную среду, отражении его от зеркала и вторичного прохождения через ту же среду назад направление линейной поляризации восстанавливается.

Измерение угла поворота плоскости поляризации лежит в основе методов определения концентрации оптически активных веществ. Этим пользуются, в частности, для определения концентрации сахара в производственных растворах и биологических объектах (кровь, моча).

Магнитное вращение. Способность поворачивать плоскость поляризации приобретают даже оптически неактивные вещества, если их поместить в продольное магнитное поле (эффект Фарадея). Схема установки для наблюдения этого эффекта состоит из соленоида с исследуемым веществом, который помещен между двумя скрещенными поляризаторами \(P\) и \(P^{\prime}\). Создание магнитного поля приводит к просветлению поля зрения. Поворотом плоскости пропускания анализатора \(P^{\prime}\) добиваются затемнения, и таким образом находят угол поворота плоскости поляризации:


\begin{equation*}
\varphi=V l H, \tag{6.20}
\end{equation*}


где \(V\), угл. мин/A - постоянная Верде (или магнитная вращательная способность), \(l\), м - длина пути света в веществе, \(H, \mathrm{~A} / \mathrm{m}\) - напряженность магнитного поля*. Постоянная Верде зависит от рода вещества, его физического состояния и длины волны света. Приведем значения этой постоянной для двух веществ при разных длинах волн:

\begin{center}
\begin{tabular}{|l|l|l|l|}
\hline
\multirow{2}{*}{Вещество} & \multicolumn{3}{|c|}{\(V\), угл. мин \(/ \mathrm{A}\)} \\
\hline
 & 656 нм & 589 нм & 486 нм \\
\hline
Вода & 0,013 & 0,016 & 0,025 \\
Сероуглерод & 0,040 & 0,052 & 0,084 \\
\hline
\end{tabular}
\end{center}

Направление вращения связано только с направлением магнитного поля Н. От направления луча направление вращения не зависит. Поэтому при отражении луча зеркалом и возвращении его в исходную точку поворот плоскости поляризации удваивается (в отличие от естественного вращения). Это свойство позволяет увеличить угол поворота удлинением пути света в образце за счет многократных отражений от посеребренных поверхностей образца.

\footnotetext{\begin{itemize}
  \item Иногда формулу (6.20) записывают иначе: \(\varphi=V l B\), где \(B\) - индукция магнитного поля и размерность постоянной Верде \(V\) в угл. мин/(Тл • м).
\end{itemize}
}Знак вращения условно считают, если смотреть вдоль магнитного поля (вдоль вектора Н). Для подавляющего большинства веществ вращение происходит вправо (т. е. правый винт относительно вектора Н). Такие вещества называют положительными. Встречаются однако и отрицательные вещества, вращающие влево (т. е. левый винт относительно вектора Н).

Пример. Узкий пучок плоскополяризованного света проходит, двукратно отражаясь, через правовращающее положительное вещество, находящееся в продольном магнитном поле (рис. 6.27). Найдем угол, на который повернется плоскость поляризации вышедшего пучка, если длина трубки с веществом равна \(l\), его постоянная вращения \(\alpha\), постоянная Верде \(V\) и напряженность магнитного поля \(H\).

\begin{figure}[h]
\begin{center}
  \includegraphics[alt={},max width=\textwidth]{950fefba-d83b-4c8f-96e7-0bd5a7eb90be-221_161_810_701_209}
\captionsetup{labelformat=empty}
\caption{Рис. 6.27}
\end{center}
\end{figure}

Если смотреть навстречу вышедшему пучку и положительное направление отсчитывать по часовой стрелке, то естественное вращение дает угол поворота \(\varphi_{\text {ест }}=\alpha l>0\), поскольку вещество правовращающее. Магнитное же вращение (вещество положительное) дает угол поворота \(\varphi_{H}=-V l H \cdot N\), где \(N\) - число прохождений пучка через вещество (на рисунке \(N=3\) ). В сумме угол поворота

\[
\varphi=\alpha l-V l H N .
\]

\begin{figure}[h]
\begin{center}
  \includegraphics[alt={},max width=\textwidth]{950fefba-d83b-4c8f-96e7-0bd5a7eb90be-221_334_328_1298_104}
\captionsetup{labelformat=empty}
\caption{Рис. 6.28}
\end{center}
\end{figure}

Тот факт, что направление вращения в магнитном поле связано только с направлением вектора H, позволяет осуществить так называемый оптический вентиль, который способен пропускать свет только в одном направлении. Сказанное поясняет рис. 6.28. Свет, прошедший после поляризатора \(P\) магнитное поле, поворачивает плоскость поляризации на \(\varphi=45^{\circ}\) и проходит через поляризатор \(P^{\prime}\). А обратно, пройдя \(P^{\prime}\) и повернувшись в ту же сторону на \(\varphi\), оказывается задержанным поляризатором \(P\).

Малая инерционность эффекта Фарадея ( \(\sim 10^{-9} \mathrm{c}\) ) позволяет использовать его для модуляции света, для создания оптического затвора и т. п.

\section*{Задачи}
6.1. Закон Малюса. Линейно-поляризованный световой пучок падает на поляризатор, плоскость пропускания которого вращается вокруг оси пучка с угловой скоростью \(\omega\). Найти световую энергию \(W\), проходящую через поляризатор за один оборот, если поток энергии в падающем пучке равен \(\Phi_{0}\).\\
Решение. Согласно закону Малюса \(\Phi=\Phi_{0} \cos ^{2} \omega t\). Тогда энергию, проходящую через поляризатор за один оборот, т. е. за период \(T=2 \pi / \omega\), определим следующим выражением:

\[
W=\int_{0}^{T} \Phi_{0} \cos ^{2} \omega t \mathrm{~d} t=\frac{\Phi_{0}}{\omega} \int_{0}^{2 \pi} \cos ^{2} \varphi \mathrm{~d} \varphi
\]

Здесь введено обозначение \(\varphi=\omega t\). Имея в виду, что

\[
\int_{0}^{2 \pi} \cos ^{2} \varphi \mathrm{~d} \varphi=\left\langle\cos ^{2} \varphi\right\rangle \cdot 2 \pi=\pi
\]

получим в результате \(W=\pi \Phi_{0} / \omega\).\\
6.2. Степень поляризации. На пути частично-поляризованного света поместили поляризатор. При повороте плоскости пропускания поляризатора из положения, соответствующего максимуму пропускания, на угол φ интенсивность прошедшего света уменьшилась в \(\eta\) раз. Найти степень поляризации падающего света.\\
Решение. Представим частично-поляризованный свет как сумму естественного и поляризованного (см. рис. 6.2). Тогда степень поляризации этого света


\begin{equation*}
P=\frac{I_{\text {п }}}{I_{\text {п }}+I_{e}}=\frac{1}{1+I_{e} / I_{\text {п }}}, \tag{1}
\end{equation*}


где \(I_{\mathrm{e}}\) и \(I_{\mathrm{n}}\) - интенсивности естественной и поляризованной составляющих. Найдем отношение \(I_{\mathrm{e}} / I_{п}\). Согласно условию и закону Малюса

\[
\eta=\frac{I_{\text {п }}+I_{\mathrm{e}} / 2}{I_{\text {п }} \cos ^{2} \varphi+I_{\mathrm{e}} / 2}
\]

Отсюда

\[
I_{\mathrm{e}} / I_{п}=2\left(1-\eta \cos ^{2} \varphi\right) /(\eta-1) .
\]

После подстановки последнего выражения в (1) получим:

\[
P=\frac{\eta-1}{1+\eta\left(1-2 \cos ^{2} \varphi\right)}=\frac{\eta-1}{1-\eta \cos 2 \varphi} .
\]

6.3. Естественный свет падает под некоторым углом на поверхность прозрачного диэлектрика. При этом \(\rho\)-часть светового потока отражается, имея степень поляризации \(P^{\prime}\). Найти степень поляризации преломленного света, \(P^{\prime \prime}\).\\
Решени е. По определению,


\begin{equation*}
P^{\prime}=\frac{\Phi_{\text {макс }}^{\prime}-\Phi_{\text {мин }}^{\prime}}{\rho \Phi_{0}}=\frac{\Phi_{\perp}^{\prime}-\Phi_{\|}^{\prime}}{\rho \Phi_{0}}, \tag{1}
\end{equation*}


где \(\Phi_{0}\) - падающий поток, \(\Phi_{\perp}^{\prime}\) и \(\Phi_{\|}^{\prime}\) - световые потоки, плоскость поляризации которых перепендикулярна и параллельна плоскости падения, причем ясно, что \(\Phi_{\perp}^{\prime}>\Phi_{\|}^{\prime}\) (рис. 6.29, где точками и

\begin{figure}[h]
\begin{center}
  \includegraphics[alt={},max width=\textwidth]{950fefba-d83b-4c8f-96e7-0bd5a7eb90be-223_280_415_776_135}
\captionsetup{labelformat=empty}
\caption{Рис. 6.29}
\end{center}
\end{figure}

черточками показаны преимущественные направления колебаний светового вектора Е). Следует обратить внимание на то, что в случае отражения не играет роли, брать интенсивность или поток, поскольку площади поперечных сечений падающего и зеркально отраженного пучков одинаковы (в отличие от преломленного пучка).\\
Запишем теперь выражение для степени поляризации преломленного пучка:


\begin{equation*}
P^{\prime \prime}=\frac{\Phi_{\|}^{\prime \prime}-\Phi_{\perp}^{\prime \prime}}{\Phi_{0}-\rho \Phi_{0}}=\frac{\left(\Phi_{0} / 2-\Phi_{\|}^{\prime}\right)-\left(\Phi_{0} / 2-\Phi_{\perp}^{\prime}\right)}{(1-\rho) \Phi_{0}}=\frac{\Phi_{\perp}^{\prime}-\Phi_{\|}^{\prime}}{(1-\rho) \Phi_{0}} . \tag{2}
\end{equation*}


Решив совместно уравнения (1) и (2), получим:

\[
P^{\prime \prime}=\frac{\rho}{1-\rho} P^{\prime}
\]

6.4. Коэффициент отражения. Определить с помощью формул (6.4) коэффициент отражения естественного света при нормальном падении на поверхность прозрачного диэлектрика с показателем преломления \(n\).\\
Решение. По определению, коэффициент отражения


\begin{equation*}
\rho=\frac{I_{\perp}^{\prime}+I_{\|}^{\prime}}{I_{\perp}+I_{\|}} \tag{1}
\end{equation*}


Теперь учтем, что \(I_{\perp}+I_{| |}=I_{0}\) - интенсивность падающего света, причем \(I_{\perp}=I_{| |}=I_{0} / 2\), так как падающий свет естественный. Поэтому формулу (1) можно записать так:


\begin{equation*}
\rho=\frac{I_{\perp}^{\prime}}{2 I_{\perp}}+\frac{I_{\|}^{\prime}}{2 I_{\|}} \tag{2}
\end{equation*}


Перепишем это соотношение с помощью формул (6.4) в следующем виде

\[
\rho=\frac{1}{2}\left(\rho_{\perp}+\rho_{\|}\right)=\frac{1}{2}\left[\frac{\sin ^{2}\left(\vartheta_{1}-\vartheta_{2}\right)}{\sin ^{2}\left(\vartheta_{1}+\vartheta_{2}\right)}+\frac{\operatorname{tg}^{2}\left(\vartheta_{1}-\vartheta_{2}\right)}{\operatorname{tg}^{2}\left(\vartheta_{1}+\vartheta_{2}\right)}\right] .
\]

При малых значениях \(\vartheta\) выражения в квадратных скобках упрощаются:


\begin{equation*}
\rho=\frac{1}{2}\left[\frac{\left(\vartheta_{1}-\vartheta_{2}\right)^{2}}{\left(\vartheta_{1}+\vartheta_{2}\right)^{2}}+\frac{\left(\vartheta_{1}-\vartheta_{2}\right)^{2}}{\left(\vartheta_{1}+\vartheta_{2}\right)^{2}}\right]=\left(\frac{\vartheta_{1}-\vartheta_{2}}{\vartheta_{1}+\vartheta_{2}}\right)^{2} . \tag{3}
\end{equation*}


Остается учесть закон преломления: при малых \(\vartheta\) имеем \(\vartheta_{1}=n \vartheta_{2}\). Подстановка этого равенства в (3) приводит к окончательному результату:

\[
\rho=\left(\frac{n-1}{n+1}\right)^{2}
\]

Этот результат полностью совпадает с формулой (3.12).\\
6.5. Найти с помощью соотношений (6.4) коэффициент отражения \(\rho\) естественного света, падающего под углом Брюстера на прозрачный диэлектрик с показателем преломления \(n\).\\
Решение. Пусть \(\vartheta_{1}\) и \(\vartheta_{2}\) - углы падения и преломления. Здесь \(\vartheta_{1}=\vartheta_{\text {Бр }}\), значит \(\vartheta_{1}+\vartheta_{2}=\pi / 2\) и в (6.4) отлично от нуля будет только \(\rho_{\perp}\). Падающий свет естественный, это значит, что у него \(I_{\perp}=I_{\|}\) и \(I_{\perp}=I_{0} / 2\), где \(I_{0}\) - интенсивность падающего света. Искомый коэффициент


\begin{equation*}
\rho=\frac{I_{\perp}^{\prime}}{I_{0}}=\frac{I_{\perp}^{\prime}}{2 I_{\perp}}=\frac{\rho_{\perp}}{2} \tag{1}
\end{equation*}


В нашем случае \(\sin \left(\vartheta_{1}+\vartheta_{2}\right)=1\), поэтому


\begin{equation*}
\rho_{\perp}=\sin ^{2}\left(\vartheta_{1}-\vartheta_{2}\right) . \tag{2}
\end{equation*}


Преобразуем (2) так, чтобы представить его через \(\operatorname{tg} \vartheta_{1}\), который по условию равен \(n\). Разность углов в скобках с учетом того, что \(\vartheta_{1}+\vartheta_{2}=\pi / 2\), запишем так:


\begin{equation*}
\vartheta_{1}-\vartheta_{2}=\vartheta_{1}-\left(\pi / 2-\vartheta_{1}\right)=2 \vartheta_{1}-\pi / 2 . \tag{3}
\end{equation*}


Тогда выражение (2) можно преобразовать следующим образом:


\begin{align*}
\sin ^{2}\left(\vartheta_{1}-\vartheta_{2}\right) & =\sin ^{2}\left(2 \vartheta_{1}-\pi / 2\right)=\cos ^{2} 2 \vartheta_{1}=\left(\cos ^{2} \vartheta_{1}-\sin ^{2} \vartheta_{1}\right)^{2}= \\
& =\left(1-2 \sin ^{2} \vartheta_{1}\right)^{2}=\left(1-2 \frac{1}{1+\operatorname{tg}^{2} \vartheta_{1}}\right)^{2} \tag{4}
\end{align*}


Заменив в (4) \(\operatorname{tg} \vartheta_{1}\) через \(n\) и учитывая (2), представим искомое \(\rho\), т. е. (1), в окончательном виде:


\begin{equation*}
\rho=\frac{1}{2}\left(\frac{n^{2}-1}{n^{2}+1}\right)^{2} . \tag{5}
\end{equation*}


Напомним, \(\rho\) - это та доля падающего света, которая при отражении под углом Брюстера превращается в полностью плоскополяризованный свет.\\
6.6. Пластинка в четверть волны. Кварцевая пластинка, вырезанная параллельно оптической оси, имеет толщину \(h=0,25\) мм и служит пластинкой в четверть волны для света с длиной волны \(\lambda= =0,53\) мкм. Для каких еще длин волн в области видимого спектра она будет также пластинкой \(\lambda / 4\) ? Считать для простоты, что для всех длин волн видимого спектра \(n_{e}-n_{o}=0,9 \cdot 10^{-2}\).

Решение. Для пластинки \(\lambda / 4\) согласно (6.8) \(h\left(n_{e}-n_{o}\right)=m \lambda / 4\), где \(m\) - нечетные числа ( \(1,3,5, \ldots\) ). Отсюда

\[
\lambda=\frac{4 h\left(n_{e}-n_{o}\right)}{m}=\frac{9}{m} \text { мкм. }
\]

После подстановки в это выражение (подбором) нечетных значений \(m\), соответствующих длинам волн в видимом диапазоне спектра, получим:

\begin{center}
\begin{tabular}{llllll}
\(m\) & \(=13\) & 15 & 17 & 19 & 21 \\
\(\lambda\), мкм & \(=0,69\) & 0,60 & 0,53 & 0,47 & 0,43 \\
\end{tabular}
\end{center}

6.7. Свет проходит через систему из двух скрещенных поляризаторов, между которыми расположена кварцевая пластинка. Ее оптиче-

ская ось составляет угол \(45^{\circ}\) с плоскостями пропускания поляризаторов. При какой минимальной толщине пластинки свет с длиной волны \(\lambda_{1}=643\) нм будет проходить сквозь эту систему с максимальной интенсивностью, а свет с \(\lambda_{2}=564\) нм будет практически задержан, если для обеих длин волн \(n_{e}-n_{o}=0,0090\) ?

Решение. Согласно (6.9) \(h=m \lambda / 2\left(n_{e}-n_{o}\right)\). Для максимума пропускания эта пластинка должна быть пластинкой \(\lambda / 2\), т. е. \(m_{1}\) должно быть нечетным, а для минимума - пластинкой в целую волну, т. е. \(m_{2}\) должно быть четным. Из приведенной формулы следует, что \(m_{1} \lambda_{1}=m_{2} \lambda_{2}\). Отсюда

\[
\frac{m_{2}}{m_{1}}=\frac{\lambda_{1}}{\lambda_{2}}=1,14 .
\]

Полученное значение 1,14 надо представить как отношение наименьших значений четного числа \(m_{2}\) к нечетному \(m_{1}\). Легко убедиться (например, подбором), что это будут 8 и 7, т. е. \(m_{2}=8\) и \(m_{1}=7\). Возвращаясь к первой формуле, находим

\[
h_{\text {мин }}=\frac{m_{1} \lambda_{1}}{2\left(n_{e}-n_{0}\right)}=0,25 \text { мм. }
\]

6.8. Естественное вращение плоскости поляризации. Естественный монохроматический свет падает на систему из двух скрещенных поляризаторов, между которыми находится кварцевая пластинка, вырезанная перпендикулярно оптической оси. Постоянная вращения пластинки для данной длины волны равна \(\alpha\). При какой минимальной толщине пластинки вся система будет пропускать \(\eta\)-часть интенсивности света, падающего на нее?

Решение. Обозначим плоскости пропускания поляризаторов через \(P\) и \(P^{\prime}\) (рис. 6.30). После первого поляризатора \(P\) интенсивность света станет равной \(I_{0} / 2\), где \(I_{0}\) - интенсивность падающего света. Плоскость поляризации этого света пластинка повернет на угол \(\varphi=\alpha h\), \(h\) - толщина пластинки.

Через поляризатор \(P^{\prime}\) пройдет (по Малю-

\begin{figure}[h]
\begin{center}
  \includegraphics[alt={},max width=\textwidth]{950fefba-d83b-4c8f-96e7-0bd5a7eb90be-226_326_376_1258_784}
\captionsetup{labelformat=empty}
\caption{Рис. 6.30}
\end{center}
\end{figure}

Значит, мы имеем


\begin{equation*}
\eta=\frac{\left(I_{0} / 2\right) \sin ^{2} \varphi}{I_{0}}=\frac{1}{2} \sin ^{2} \varphi . \tag{1}
\end{equation*}


Модуль проекции вектора \(\mathbf{E}\) на плоскость пропускания поляризатора \(P^{\prime}\), равный \(|E \sin \varphi|\), может иметь место при многих значениях угла \(\varphi\), а значит при многих значениях и \(h\) - толщины пластинки. Нас интересует минимальная толщина, обеспечивающая указанное условие. Эта толщина должна соответствовать минимальному значению \(\varphi_{\text {мин }}\), т. е. должно выполняться условие \(\varphi_{\text {мин }}<\pi / 2\). Значит,


\begin{equation*}
\varphi_{\text {мин }}=\alpha h_{\text {мин }} . \tag{2}
\end{equation*}


Подставив это значение \(\varphi\) в (1), получим

\[
h_{\text {мин }}=\frac{1}{\alpha} \operatorname{acrsin} \sqrt{2 \eta} .
\]

В заключение, полезно убедиться в том, что если бы в условии задачи не было требования о нахождении минимальной толщины пластинки, то угол \(\varphi\), удовлетворяющий пропусканию \(\eta\)-части света, был бы неоднозначным, а следовательно неоднозначной должна была бы быть и толщина пластинки. Нетрудно проверить с помощью рис. 6.30, что возможные значения φ были бы следующие:

\[
\varphi=\left\{\begin{array}{l}
\varphi_{\text {мин }}+\pi m, \\
\pi(m+1)-\varphi_{\text {мин }},
\end{array} \quad \text { где } m=0,1,2, \ldots\right.
\]

6.9. Эффект Керра. Ячейку Керра поместили между двумя скрещенными поляризаторами так, что направление напряженности Е электриченского поля образовало угол \(45^{\circ}\) с плоскостями пропускания поляризаторов. Конденсатор имеет длину \(l=10,0\) см и заполнен нитробензолом. На систему падает свет с длиной волны \(\lambda=0,50\) мкм. Имея в виду, что в этих условиях постоянная Керра \(B=2,2 \cdot 10^{-10} \mathrm{~cm} / \mathrm{B}^{2}\), определить:\\
а) минимальную напряженность \(E\) электрического поля в конденсаторе, при которой интенсивность света, прошедшего через эту систему, не будет зависеть от поворота заднего поляризатора;\\
б) число прерываний света ежесекундно, если на конденсатор подать синусоидальное напряжение с частотой \(v=10\) МГц и амплитудным значением напряженности \(E=50\) кВ/см.

Решение. а) Легко сообразить, что интенсивность прошедшего света не будет зависеть от поворота заднего поляризатора только в том случае, если свет поляризован по кругу, т. е. нитробензол ведет себя в этом случае как пластинка \(\lambda / 4\). Это значит, что согласно (6.8) и (6.17)

\[
l \cdot B \lambda E^{2}=m \lambda / 4,
\]

где \(m\) - нечетное, и по условию ( \(E\) должно быть минимальным) \(m=1\). Отсюда

\[
E_{\text {мин }}=\frac{1}{2 \sqrt{l B}}=10,6 \kappa \mathrm{~B} / \mathrm{cm} .
\]

б) Сначала найдем число прерываний за время, в течение которого \(E\) возрастает (рис. 6.31). Из условия

\[
l B \lambda E_{m}^{2}=m_{\text {макс }} \lambda
\]

найдем \(m_{\text {макс }}\). Справа в этой формуле записано целое число длин волн, поскольку поляризаторы скрещены, и в этом случае при целых значениях \(m\) интенсивность \(I_{\perp}^{\prime}=0\) согласно таб-

\begin{figure}[h]
\begin{center}
  \includegraphics[alt={},max width=\textwidth]{950fefba-d83b-4c8f-96e7-0bd5a7eb90be-228_379_421_375_739}
\captionsetup{labelformat=empty}
\caption{Рис. 6.31}
\end{center}
\end{figure}

\[
m_{\text {макс }}=[5,5]=5 .
\]

Квадратные скобки означают, что следует брать целое число от полученного значения. За период \(T\) таких прерываний будет (см. рис. 6.31)

\[
N=4 m_{\text {макс }}+2=22,
\]

где двойка соответствует тому, что при \(E=0\) система тоже не пропускает свет, а таких прерываний за период будет два.

Число прерываний за 1 с

\[
N_{1}=N v=2,2 \cdot 10^{8} \mathrm{c}^{-1} .
\]

6.10. Магнитное вращение. Некоторое вещество поместили в продольное магнитное поле соленоида, расположенного между двумя поляризаторами. Длина трубки с веществом \(l=30 \mathrm{~cm}\). Найти постоянную Верде, если при напряженности магнитного поля \(H=12,0 \kappa \mathrm{~A} / м\) угол поворота плоскости поляризации \(\varphi_{1}=+2^{\circ} 15^{\prime}\) для одного направления поля и \(\varphi_{2}=-1^{\circ} 06^{\prime}\) для противоположного направления.

Решение. Изобразив по двум значениям φ график зависимости \(\varphi(H)\) - он должен быть линейным (рис. 6.32), обнаруживаем, что прямая не проходит через точку 0 . Это означает, что,

\begin{figure}[h]
\begin{center}
  \includegraphics[alt={},max width=\textwidth]{950fefba-d83b-4c8f-96e7-0bd5a7eb90be-229_298_430_178_132}
\captionsetup{labelformat=empty}
\caption{Рис. 6.32}
\end{center}
\end{figure}

кроме магнитного вращения, вещество обладает и естественным вращением. Последнее можно исключить, если соотношение (6.20) записать для углов \(\varphi_{1}\) и \(\varphi_{2}\), а затем взять их разность:

\[
\varphi_{1}-\varphi_{2}=V l\left(H_{1}-H_{2}\right),
\]

где \(H_{2}=-H_{1}=H\), поэтому \(H_{1}-H_{2}= 2 H\). В результате

\[
V=\frac{\varphi_{1}-\varphi_{2}}{2 l H}=\frac{2^{\circ} 15^{\prime}-\left(-1^{\circ} 06^{\prime}\right)}{2 \cdot 0,3 \cdot 12 \cdot 10^{3}}=0,028 \text { угл. мин } / \text { А. }
\]

\section*{Взаимодействие света с веществом}
\section*{§ 7.1. Дисперсия света}
Дисперсия света - это явления, обусловленные зависимостью показателя преломления вещества от длины волны (или частоты):


\begin{equation*}
n=f(\lambda), \tag{7.1}
\end{equation*}


где \(\lambda\) - длина волны света в вакууме.\\
Производную \(\mathrm{d} n / \mathrm{d} \lambda\) называют Әисперсией вещества.\\
Для прозрачных бесцветных веществ график зависимости \(n(\lambda)\) в видимой части спектра имеет вид, показанный на рис. 7.1. Интервал длин волн, в котором \(\mathrm{d} n / \mathrm{d} \lambda<0\) (как на рисунке), соответствует нормальной дисперсии. Те же интервалы длин волн, где дисперсия вещества \(\mathrm{d} n / \mathrm{d} \lambda>0\), соответствуют аномальной дисперсии. На рис. 7.2 показан график зависимости \(n(\lambda)\) с участками нормальной и аномальной дисперсии. Заметим, что область аномальной дисперсии совпадает с полосой поглощения \(\varkappa(\lambda)\).

\begin{figure}[h]
\begin{center}
  \includegraphics[alt={},max width=\textwidth]{950fefba-d83b-4c8f-96e7-0bd5a7eb90be-230_308_340_1276_173}
\captionsetup{labelformat=empty}
\caption{Рис. 7.1}
\end{center}
\end{figure}

\begin{figure}[h]
\begin{center}
  \includegraphics[alt={},max width=\textwidth]{950fefba-d83b-4c8f-96e7-0bd5a7eb90be-230_308_471_1276_649}
\captionsetup{labelformat=empty}
\caption{Рис. 7.2}
\end{center}
\end{figure}

Все вещества в той или иной степени являются диспергирующими. Вакуум, как показали тщательные исследования, дисперсией не обладает.

Аналитический вид зависимости \(n(\lambda)\) в области нормальной дисперсии для не слишком больших интервалов длин волн может быть представлен приближенной формулой


\begin{equation*}
n=a+b / \lambda^{2}, \tag{7.2}
\end{equation*}


где \(a\) и \(b\) положительные постоянные, значения которых для каждого вещества определяются из опыта.

Пример. На рис. 7.1 и 7.2 изображены графики зависимости показателя преломления вещества от длины волны \(n(\lambda)\). Изобразим соответствующие графики зависимостей \(n(\omega)\), где \(\omega\) - циклическая частота света.

\begin{figure}[h]
\begin{center}
  \includegraphics[alt={},max width=\textwidth]{950fefba-d83b-4c8f-96e7-0bd5a7eb90be-231_309_345_661_169}
\captionsetup{labelformat=empty}
\caption{Рис. 7.3}
\end{center}
\end{figure}

\begin{figure}[h]
\begin{center}
  \includegraphics[alt={},max width=\textwidth]{950fefba-d83b-4c8f-96e7-0bd5a7eb90be-231_298_491_661_646}
\captionsetup{labelformat=empty}
\caption{Рис. 7.4}
\end{center}
\end{figure}

Поскольку \(\omega\) c \(1 / \lambda\), легко проверить, что графики \(n(\omega)\), соответствующие указанным рисункам, таковы, как показано на рис. 7.3 и 7.4. Причем, в случае графика, приведенного на рис. 7.3 , закон дисперсии в соответствии с формулой (7.2) принимает вид

\[
n=a+b^{\prime} \omega^{2},
\]

где постоянная \(b^{\prime}=b /(2 \pi c)^{2}\).

\section*{§ 7.2. Классическая теория дисперсии}
Дисперсию света можно объяснить на основе электромагнитной теории и электронной теории вещества. Строго говоря, движение (точнее - поведение) электронов в атоме подчиняется законам квантовой физики. Однако для качественного понимания дисперсии света достаточно ограничиться классическими представлениями, которые, как это ни удивительно, приводят к тем же результатам, что и квантовая теория.

Итак, поставим перед собой задачу объяснить ход зависимости \(n(\omega)\). Мы знаем, что в изотропной немагнитной среде \(n=\sqrt{\varepsilon}\). В свою очередь \(\varepsilon\) можно найти из соотношения \(\varepsilon=1+\varkappa\), где \(\varkappa\) диэлектрическая восприимчивость, которая является коэффициентом в соотношении \(\mathbf{P}=\varkappa \varepsilon_{0} \mathbf{E}, \mathbf{P}\) - поляризованность, т. е. дипольный момент единицы объема. Таким образом,


\begin{equation*}
\varepsilon=1+\frac{P_{x}(t)}{\varepsilon_{0} E_{x}(t)}, \tag{7.3}
\end{equation*}


где \(P_{x}\) - проекция вектора \(\mathbf{P}\) на ось \(X\), вдоль которой совершаются колебания вектора Е.

Известно, что \(P_{x}=n_{0} p_{x}\), где \(n_{0}\) - концентрация диполей, \(p_{x}\) - поекция дипольного момента отдельного диполя. В дальнейшем мы будем рассматривать простейшую модель вещества, состоящего из не взаимодействующих друг с другом атомов. Каждый атом представляет собой ядро, окруженное быстро движущимися электронами, которые в совокупности как бы «размазаны» по сферической симметричной области вокруг ядра. Поэтому принято говорить, что ядро с зарядом \(q\) окружено «электронным облаком» с зарядом \(-q\).

В отсутствие внешнего поля \(\mathbf{E}\) центр электронного облака совпадает с ядром, и дипольный момент атома равен нулю. При наличии же внешнего поля \(\mathbf{E}\) электронное облако смещается относительно практически неподвижного ядра, и возникает дипольный момент \(\mathbf{p}=q \mathbf{l}\), где \(q>0\), а \(\mathbf{l}\) - вектор, проведенный из центра «облака» к ядру. Проекция вектора \(\mathbf{p}\) на ось \(X\) равна


\begin{equation*}
p_{x}=q l_{x}=q(-x)=-q x \text {, } \tag{7.4}
\end{equation*}


здесь \(x\) - смещение центра «облака» из положения равновесия, т. е. относительно ядра. Заметим, что центр «облака» ведет себя как точечный заряд \(-q\).

С учетом (7.4) выражение (7.3) можно представить так:


\begin{equation*}
\varepsilon=1+\frac{n_{0}(-q x)}{\varepsilon_{0} E_{x}} . \tag{7.5}
\end{equation*}


Как видно, задача сводится к определению \(x(t)\) под действием \(E_{x}(t)\).

Для этого запишем уравнение движения электронного облака как


\begin{equation*}
m \ddot{x}=-k x-r \dot{x}+q E_{m} \cos \omega t, \tag{7.6}
\end{equation*}


где \(m\) - масса электронного облака, а справа записаны проекции на ось \(X\) квазиупругой силы, силы «сопротивления», обусловленной чем-то вроде «трения» облака о ядро, и вынуждающей силы со стороны гармонической электромагнитной волны частоты \(\omega\). Магнитной составляющей этой силы мы пренебрегаем, поскольку в нерелятивистском случае она ничтожно мала.

Разделив уравнение (7.6) на \(m\), приведем его к виду


\begin{equation*}
\ddot{x}+2 \beta \dot{x}+\omega_{0}^{2} x=f_{m} \cos \omega t, \tag{7.7}
\end{equation*}


где \(\omega_{0}^{2}=k / m, 2 \beta=r / m, f_{m}=q E_{m} / m\).\\
Для теории дисперсии имеет значение не общее, а только частное (установившееся) решение уравнения (7.7):


\begin{equation*}
x=a \cos (\omega t-\varphi), \tag{7.8}
\end{equation*}


где \(a\) - амплитуда колебаний, \(\varphi\) - разность фаз между смещением \(x\) и «силой» \(f_{m} \cos \omega t\). Подстановка этого решения в уравнение (7.7) позволяет с помощью векторной диаграммы найти значения амплитуды \(a\) и разности фаз \(\varphi\), а именно


\begin{equation*}
a=\frac{f_{m}}{\sqrt{\left(\omega_{0}^{2}-\omega^{2}\right)^{2}+4 \beta^{2} \omega^{2}}}, \quad \operatorname{tg} \varphi=\frac{2 \beta \omega}{\omega_{0}^{2}-\omega^{2}} \tag{7.9}
\end{equation*}


(решение уравнения (7.7) подробно рассматривается в теории колебаний).

Ограничимся простейшим случаем, когда \(2 \beta \omega \ll\left(\omega_{0}^{2}-\omega^{2}\right)\), т. е. когда вынуждающая частота (поля) не очень близка к собственной частоте \(\omega_{0}\) колебаний электронного облака и коэффициент \(\beta\), характеризующий затухание, достаточно мал. В этом случае, если \(\omega<\omega_{0}\), то


\begin{equation*}
x(t)=\frac{f_{m}}{\omega_{0}^{2}-\omega^{2}} \cos \omega t \tag{7.10}
\end{equation*}


Такой же результат будет и при \(\omega>\omega_{0}\), когда \(\varphi=\pi\).

Остается подставить (7.10) в (7.5) и учесть, что вынуждающая сила в (7.6) \(q E_{m} \cos \omega t=-q E_{x}\). В результате получим:


\begin{equation*}
\varepsilon=1+\frac{b}{\omega_{0}^{2}-\omega^{2}}, \tag{7.11}
\end{equation*}


где \(b=n_{0} q^{2} / \varepsilon_{0} m=N_{0} e^{2} / \varepsilon_{0} m_{e}, N_{0}\) - концентрация электронов (здесь учтено, что \(q=Z e, m=Z m_{e}\) и \(N_{0}=Z n_{0}, Z\) - число электронов в атоме).

Разрыв функции \(\varepsilon(\omega)\) при \(\omega=\omega_{0}\) и обращение ее в \(\pm \infty\) не имеют физического смысла, это получилось вследствие игнорирования затухания ( \(\beta \rightarrow 0\) ). Если же его учесть, то ход кривой будет иным (рис. 7.5) и достаточно хорошо подтверждается экспериментально (сравните с рис. 7.4). Зависимость \(\varkappa(\omega)\)

\begin{figure}[h]
\begin{center}
  \includegraphics[alt={},max width=\textwidth]{950fefba-d83b-4c8f-96e7-0bd5a7eb90be-234_229_406_548_754}
\captionsetup{labelformat=empty}
\caption{Рис. 7.5}
\end{center}
\end{figure}

Заметим, что собственных частот \(\omega_{0 i}\) может быть несколько в атоме, соответственно будет и несколько областей аномальной дисперсии. Кроме того, как видно из рис. 7.5, при \(\omega>\omega_{0}\) показатель преломления ( \(n=\sqrt{\varepsilon}\) ) будет меньше единицы, а это значит, что фазовая скорость электромагнитной волны \(v=c / n\) оказывается больше \(c\) ! Подобное имеет место в плазме, где \(\omega_{0}=0\) (электроны свободные), и для рентгеновского излучения ( \(\omega \gg \omega_{0}\) ). Никакого противоречия с теорией относительности здесь нет. Последняя утверждает, что скорость сигнала (импульса) не может превышать \(c\). Понятие же показателя преломления применимо к монохроматическим электромагнитным волнам, бесконечным в пространстве и во времени. Такие волны не могут служить для передачи сигнала, а кроме того, их в принципе невозможно осуществить.

Из выражения (7.11) вытекает и еще одно неожиданное следствие для случая, когда \(\omega_{0}=0\) (например, в той же плазме). При этом условии, когда частота электромагнитной волны \(\omega \leqslant \sqrt{b}\), оказывается, что диэлектрическая проницаемость \(\varepsilon(\omega) \leqslant 0\), а следовательно, показатель преломления для таких частот ( \(n=\sqrt{\varepsilon}\) ) становится мнимьлм, и его можно представить как \(n=i \varkappa\). Выясним, что это означает.

Запишем уравнение электромагнитной волны в комплексной форме:

\[
\hat{E}=E_{0} \mathrm{e}^{\mathrm{i}(k x-\omega t)},
\]

где \(k=2 \pi / \lambda, \lambda\) - длина волны в среде. Если длина волны в вакууме \(\lambda_{0}\), то \(\lambda=\lambda_{0} / n\), и

\[
k=\frac{2 \pi}{\lambda_{0}} n=k_{0} \cdot \mathrm{i} \varkappa
\]

где \(k_{0}=2 \pi / \lambda_{0}\) и \(n=\mathrm{i} \varkappa\). Подставив выражение для \(k\) в исходное уравнение волны \(\hat{E}(x, t)\), получим:

\[
\hat{E}=E_{0} \exp \left(-\varkappa k_{0} x\right) \mathrm{e}^{-\mathrm{i} \omega t},
\]

или для действительной части

\[
E=E_{0} \exp \left(-\varkappa k_{0} x\right) \cos \omega t
\]

Видно, что в рассматриваемом случае мы имеем стоячую волну, амплитуда которой экспоненциально затухает*. Фактически это означает, что излучение при \(\varepsilon<0\) не может пройти через плазму и происходит полное отражение его в пограничном слое. На этом, кстати, основан метод определения концентрации электронов в плазме.

Пример. При зондировании разреженной плазмы радиоволнами различных частот обнаружили, что радиоволны с частотами, меньшими, чем \(v_{0}=400\) МГц не проходят через плазму. Найдем концентрацию свободных электронов в этой плазме.\\
Радиоволны не проходят через плазму, а отражаются от нее, как мы выяснили, при мнимом показателе преломления, т. е. при значении диэлектрической проницаемости \(\varepsilon \leqslant 0\). Имея в виду (7.11) и учитывая, что для свободных электронов \(\omega_{0}=0\), получим:

\[
\varepsilon(\omega)=1-\frac{N_{0} e^{2}}{\varepsilon_{0} m_{e} \omega^{2}} \leqslant 0 .
\]

\footnotetext{\begin{itemize}
  \item В общем случае вводят комплексный показатель преломления \(\hat{n}=n+\mathrm{i} \varkappa\), где \(n\) определяет фазовую скорость волны \(v=c / n\), а мнимую часть \(\varkappa\) называют показателем затухания. Он характеризует затухание волны по мере ее распространения. Затухание не обязательно связано с поглощением электромагнитной волны, примером тому служит разобранный пример.
\end{itemize}
}Отсюда находим искомую концентрацию свободных электронов:

\[
N_{0}=\varepsilon_{0} m_{e} \omega^{2} / e^{2}=4 \pi^{2} \varepsilon_{0} v_{0}^{2} / e^{2}=2,0 \cdot 10^{9} \mathrm{~cm}^{-3} .
\]

\section*{§ 7.3. Групповая скорость}
Волновой пакет. Строго монохроматическая волна - это идеализация. Таких волн в природе нет. Любая реальная волна, согласно теореме Фурье, может быть представлена как суперпозиция монохроматических волн с различными амплитудами и частотами \(\omega\) в некотором интервале \(\Delta \omega\). Суперпозицию волн, мало отличающихся друг от друга по частотам ( \(\Delta \omega \ll \omega\) ), называют волновым пакетом или группой волн. Вид волнового пакета в некоторый момент времени показан на рис. 7.6. В его пределах монохроматические составляющие усиливают друг друга, вне пакета практически гасят друг друга.

\begin{figure}[h]
\begin{center}
  \includegraphics[alt={},max width=\textwidth]{950fefba-d83b-4c8f-96e7-0bd5a7eb90be-236_322_656_881_286}
\captionsetup{labelformat=empty}
\caption{Рис. 7.6}
\end{center}
\end{figure}

В вакууме все монохроматические волны, образующие пакет, распространяются, как уже было сказано выше, с одинаковой фазовой скоростью


\begin{equation*}
v=\omega / k, \tag{7.12}
\end{equation*}


где \(k\) - волновое число ( \(2 \pi / \lambda\) ). С такой же скоростью распространяется в вакууме и сам волновой пакет, не изменяя своей формы.

Групповая скорость. В диспергирующей же среде волновой пакет расплывается, поскольку скорости его монохроматических составляющих отличаются друг от друга, и понятие скорости такой волны требует уточнения.

Если дисперсия достаточно мала, расплывание волнового пакета происходит не слишком быстро. В этом случае волновому пакету можно приписать скорость \(u\), с которой перемещается его «центр тяжести». Это так называемая групповая скорость. Соответствующий расчет дает, что групповая скорость определяется как


\begin{equation*}
u=\mathrm{d} \omega / \mathrm{d} k . \tag{7.13}
\end{equation*}


Поясним эту формулу на примере суперпозиции двух волн с одинаковой амплитудой и несколько отличными друг от друга длинами волн (и частотами). На рис. \(7.7, a\) показано их относительное расположение в некоторый момент времени, а на рис. 7.7, б - результат их суперпозиции. Нас будет интересовать скорость, с которой перемещается место с максимальной амплитудой - это и будет скорость волнового пакета - групповая скорость. Определим ее величину.

\begin{figure}[h]
\begin{center}
  \includegraphics[alt={},max width=\textwidth]{950fefba-d83b-4c8f-96e7-0bd5a7eb90be-237_388_864_913_181}
\captionsetup{labelformat=empty}
\caption{Рис. 7.7}
\end{center}
\end{figure}

Пусть уравнения этих двух монохроматических волн имеют вид:

\[
\begin{gathered}
E_{1}=A \cos (\omega t-k x), \\
E_{2}=A \cos [(\omega+\mathrm{d} \omega) t-(k+\mathrm{d} k) x] .
\end{gathered}
\]

В результате их наложения образуется суммарная волна


\begin{equation*}
E=E_{1}+E_{2}=2 A \cos \frac{t \mathrm{~d} \omega-x \mathrm{~d} k}{2} \cos (\omega t-k x) . \tag{7.14}
\end{equation*}


Это выражение можно рассматривать как уравнение монохроматической волны, амплитуда которой меняется по закону

\[
A_{0}=\left|2 A \cos \frac{t \mathrm{~d} \omega-x \mathrm{~d} k}{2}\right| .
\]

Отсюда следует, что точки, соответствующие, например, максимуму амплитуды, движутся по закону

\[
t \mathrm{~d} \omega-x \mathrm{~d} k=0,
\]

откуда \(x=(\mathrm{d} \omega / \mathrm{d} k) t\). Величина в скобках и есть групповая скорость (7.13).

Выражение для групповой скорости можно представить в ином виде. Заменив \(\omega\) через \(v k\) согласно (7.12), получим:


\begin{equation*}
u=\frac{\mathrm{d}}{\mathrm{~d} k}(v k)=v+k \frac{\mathrm{~d} v}{\mathrm{~d} k} . \tag{7.15}
\end{equation*}


Так как \(k=2 \pi / \lambda\) и \(\mathrm{d} k=-\left(2 \pi / \lambda^{2}\right) \mathrm{d} \lambda\), где \(\lambda\) - длина волны в среде, то выражение (7.15) можно переписать так:


\begin{equation*}
u=v-\lambda \frac{\mathrm{d} v}{\mathrm{~d} \lambda} \tag{7.16}
\end{equation*}


Это так называемая формула Рэлея. В области нормальной дисперсии ( \(\mathrm{d} v / \mathrm{d} \lambda>0\) ) групповая скорость \(u\) оказывается меньше фазовой скорости \(v\). В отсутствие дисперсии \(\mathrm{d} v / \mathrm{d} \lambda=0\), и групповая скорость совпадает с фазовой (об этом уже говорилось).

Существует простой графический способ нахождения групповой скорости по кривой \(v(\lambda)\). Он показан на рис. 7.8. В слу-

\begin{figure}[h]
\begin{center}
  \includegraphics[alt={},max width=\textwidth]{950fefba-d83b-4c8f-96e7-0bd5a7eb90be-238_362_453_1395_389}
\captionsetup{labelformat=empty}
\caption{Рис. 7.8}
\end{center}
\end{figure}

чае группы волн роль играет только малый участок кривой \(v(\lambda)\) в узком диапазоне \(\Delta \lambda\) ( \(\Delta \lambda \ll \lambda\) ). Отрезок, который отсекает на оси ординат касательная к кривой \(v(\lambda)\), проведенная через точку \(A\), равен \(v-\lambda(\mathrm{d} v / \mathrm{d} \lambda)\), т. е. групповой скорости \(u\) при данной длине волны \(\lambda\).

Пример. Найдем выражение для групповой скорости в среде с известной зависимостью показателя преломления от частоты электромагнитной волны, \(n(\omega)\).\\
Будем исходить из определения групповой скорости - формулы (7.13). Принимая во внимание, что фазовая скорость \(v=\omega / k=c / n\), получим

\[
k=\omega n / c .
\]

Теперь возьмем производную \(\mathrm{d} k / \mathrm{d} \omega\) :

\[
\frac{\mathrm{d} k}{\mathrm{~d} \omega}=\frac{\mathrm{d}}{\mathrm{~d} \omega}\left(\frac{\omega n}{c}\right)=\frac{1}{c}\left(n+\omega \frac{\mathrm{d} n}{\mathrm{~d} \omega}\right) .
\]

Подстановка обратного значения этой величины, т. е. \(\mathrm{d} \omega / \mathrm{d} k\), в (7.13) приводит к искомому результату:

\[
u=\frac{\mathrm{d} \omega}{\mathrm{~d} k}=\frac{c}{n+\omega(\mathrm{d} n / \mathrm{d} \omega)}
\]

В некоторых случаях групповая скорость, вычисленная по приведенным выше формулам, оказывается больше \(c\) - скорости света в вакууме. Так будет, например, в области аномальной дисперсии. Это не противоречит теории относительности, ибо групповая скорость выражает скорость сигнала лишь тогда, когда волновой импульс в процессе распространения практически не изменяет своей формы. В области же аномальной дисперсии импульс сильно деформируется, и групповая скорость в таких условиях утрачивает определенное физическое содержание.

Групповая скорость и перенос энергии. Рассмотрим вопрос о скорости распространения энергии, переносимой электромагнитной волной. Прежде всего заметим, что фазовая скорость монохроматической волны не имеет ничего общего со скоростью переноса энергии. Фазовая скорость устанавливает только связь между фазами колебаний в различных точках пространства.

Строго монохроматическая волна не может служить для передачи сигнала, поскольку она не имеет ни начала, ни конца во

времени и пространстве. Поэтому распространение сигнала связано с перемещением изменений амплитуды. И в тех случаях, когда групповая скорость имеет смысл (т. е. электромагнитный импульс распространяется не расплываясь), она совпадает со скоростью переноса энергии.

Итак, в области, далекой от области сильного поглощения, скорость переноса энергии в группе волн совпадает с групповой скоростью.

Прямые измерения скорости света сводятся к измерению расстояния, проходимого световым сигналом (импульсом) за определенный промежуток времени. Этот метод практически дает групповую скорость. То же самое, как показывает подробный анализ, относится ко всем известным косвенным методам измерения скорости света. Фазовую же скорость (точнее, отношение фазовых скоростей в двух различных средах) можно определить по отношению показателей преломления, или воспользовавшись законом преломления.

\section*{§ 7.4. Поглощение света}
Закон Бугера. Прохождение световой волны через вещество сопровождается потерей энергии этой волны, затрачиваемой на возбуждение колебаний электронов (точнее, на изменение их состояния в атоме). Частично эта энергия возвращается излучению в виде вторичных волн, порождаемых колеблющимися электронами; частично же она переходит в другие формы энергии (во внутреннюю энергию вещества).

Поэтому интенсивность света при прохождении через обычное вещество уменьшается - свет поглощается в веществе. Поглощение света можно описать с энергетической точки зрения, не вникая в механизм взаимодействия света с атомами вещества, чем мы и воспользуемся.

Пусть через однородное вещество распространяется параллельный световой пучок. Выделим мысленно в этом веществе бесконечно тонкий плоский слой толщины \(\mathrm{d} x\) (рис. 7.9). При прохождении этого слоя интенсивность света уменьшится так, что ее убыль можно представить как \(-\mathrm{d} I\). Ясно, что эта величина будет пропор-

\begin{figure}[h]
\begin{center}
  \includegraphics[alt={},max width=\textwidth]{950fefba-d83b-4c8f-96e7-0bd5a7eb90be-240_261_229_1520_927}
\captionsetup{labelformat=empty}
\caption{Рис. 7.9}
\end{center}
\end{figure}

циональна интенсивности в данном поглощающем слое и его толщине \(\mathrm{d} x\), т.е.


\begin{equation*}
-\mathrm{d} I=x I \mathrm{~d} x, \tag{7.17}
\end{equation*}


где \(x\) - коэффициент поглощения, он характеризует поглощающие свойства вещества. Разделив переменные, получим: \(-\mathrm{d} I / I=\varkappa d x\).

После интегрирования в пределах от \(I_{0}\) до \(I\) и от 0 до \(x\) найдем: \(\ln \left(I / I_{0}\right)=-\varkappa x\), откуда


\begin{equation*}
I=I_{0} \mathrm{e}^{-\varkappa x} . \tag{7.18}
\end{equation*}


Это и есть закон Бугера. Таким образом, интенсивность света при прохождении однородного вещества уменьшается по экспоненциальному закону.

Заметим, что в случае точечного источника света, находящегося в однородной поглощающей среде, предыдущие рассуждения следует повторить, но только не для интенсивности \(I\), а для светового потока Ф. И в качестве исходного бесконечно тонкого слоя теперь следует выбрать сферический слой с радиусами от \(r\) до \(r+\mathrm{d} r\). В результате приходим к аналогичному (7.18) закону:


\begin{equation*}
\Phi=\Phi_{0} \mathrm{e}^{-\varkappa r}, \tag{7.19}
\end{equation*}


где \(\Phi_{0}\) - световая мощность источника (или его световой поток при \(r \rightarrow 0\) ).

\begin{figure}[h]
\begin{center}
  \includegraphics[alt={},max width=\textwidth]{950fefba-d83b-4c8f-96e7-0bd5a7eb90be-241_293_408_1258_68}
\captionsetup{labelformat=empty}
\caption{Рис. 7.10}
\end{center}
\end{figure}

О коэффициентах поглощения. Для всех веществ поглощение имеет селективный характер, т. е. коэффициент поглощения \(\varkappa\) зависит от длины волны света (в вакууме). Для жидких и твердых веществ зависимость \(\varkappa(\lambda)\) имеет вид, подобный изображенному на рис. 7.10. Т. е. сильное поглощение обнаруживается в достаточно широком интервале длин волн.\\
Совсем иначе ведет себя коэффициент \(\varkappa(\lambda)\) в случае газов или паров металлов при невысоком давлении. Здесь для всех длин волн коэффициент \(x \approx 0\), и лишь для очень узких спектральных интервалов \(\delta \lambda\) (порядка нескольких тысячных нм) обна-

руживает резкие максимумы (рис. 7.11). Эти максимумы соответствуют резонансным частотам колебаний электронов внутри атомов, которые практически не взаимодействуют друг с другом.

При повышении же давления максимумы поглощения все больше расширяются, и при высоких давлениях спектр \(\varkappa(\lambda)\) приближается к

\begin{figure}[h]
\begin{center}
  \includegraphics[alt={},max width=\textwidth]{950fefba-d83b-4c8f-96e7-0bd5a7eb90be-242_338_451_169_709}
\captionsetup{labelformat=empty}
\caption{Рис. 7.11}
\end{center}
\end{figure}

В заключение отметим, что можно создать такое состояние атомов вещества, при котором коэффициент \(\varkappa\) становится отрицательным, и прохождение света через вещество в таком (инверсионном, как говорят) состоянии сопровождается усилением его интенсивности. Именно это и осуществляется в лазерах.

\section*{§ 7.5. Рассеяние света}
Механизм рассеяния света. С классической точки зрения рассеяние света состоит в том, что световая волна, проходящая через вещество, вызывает колебания электронов в атомах (молекулах). Эти электроны возбуждают вторичные волны, распространяющиеся по всем направлениям. При этом вторичные волны оказываются когерентными между собой и поэтому интерферируют.

Теоретический расчет приводит к следующему выводу: в случае однородной среды вторичные волны полностью гасят друг друга во всех направлениях, кроме направления распространения первичной волны (это было показано в конце с. 175). В силу этого перераспределения света по направлениям, т. е. рассеяния света в однородной среде, не происходит.

Иначе обстоит дело при распространении света в неоднородной среде. В этом случае световые волны, дифрагируя на мелких неоднородностях среды, дают дифракционную картину в виде довольно равномерного распределения интенсивности по всем направлениям. Это явление и называют рассеянием свеma.

Примерами таких сред с явно выраженной оптической неоднородностью могут служить так называемые мутные среды. К их числу относятся аэрозоли (дым, туман), коллоидные растворы, матовые стекла и др., содержащие мелкие частицы, показатель преломления которых отличается от показателя преломления окружающей среды.

Закон Рэлея. Рассеяние света в мутных средах на неоднородностях, размеры которых малы по сравнению с длиной волны \(\lambda\), можно наблюдать, например, при прохождении яркого светового пучка через слой воздуха с мелкими частичками дыма или через сосуд с водой, в которую добавлено немного молока. Если мутную воду освещать пучком белого света, то при наблюдении сбоку - в рассеянном свете - среда кажется голубой, т. е. обнаруживается преобладание коротковолновой части спектра. В свете же, прошедшем сквозь достаточно толстый слой мутной среды, обнаруживается преобладание длинноволновой части спектра, и среда кажется красноватой.

Причина такого явления состоит в том, что электроны, совершающие вынужденные колебания в атомах электрически изотропной частицы малого размера (не более \(\sim 0,1 \lambda\) ), эквивалентны одному колеблющемуся диполю. Этот диполь колеблется с частотой падающей на него световой волны, и согласно (2.43) интенсивность излучаемого им света


\begin{equation*}
I \sim \omega^{4} \circlearrowleft \frac{1}{\lambda^{4}} . \tag{7.20}
\end{equation*}


Эту зависимость называют законом Рэлея. Из него следует, что коротковолновая часть спектра рассеивается значительно более интенсивно, нежели длинноволновая. Голубой свет, частота которого примерно в 1,5 раза больше частоты красного света, рассеивается почти в 5 раз интенсивнее, чем красный. Это и объясняет голубой цвет рассеянного света и красноватый - прошедшего.

Поляризация рассеянного света. При рассеянии естественного света в мутной среде зависимость интенсивности рассеянного света от угла рассеяния \(\vartheta\) имеет вид


\begin{equation*}
I=I_{0}\left(1+\cos ^{2} \vartheta\right), \tag{7.21}
\end{equation*}


где \(I_{0}\) - интенсивность света, рассеянного под прямым углом ( \(\vartheta=\pi / 2\) ) к направлению первичного светового пучка. Если молекулы рассеивающих частичек электрически изотропны (неполярные молекулы), то рассеиваемый свет оказывается час-тично-поляризованным и под углом \(\vartheta=\pi / 2\) - полностью поляризованным. В этом случае его плоскость поляризации (плоскость колебаний вектора E) перпендикулярна направлению первичного светового пучка.

Если размеры неоднородностей сравнимы с длиной волны света, то электроны в различных местах неоднородности колеблются уже не синфазно. Это усложняет явление рассеяния и приводит к другим закономерностям: закон Рэлея нарушается (интенсивность рассеянного света становится пропорциональной всего лишь квадрату частоты, \(I \sim \omega^{2} \sim 1 / \lambda^{2}\) ), и свет, рассеянный под углом \(\vartheta=\pi / 2\), оказывается поляризованным лишь частично.

Если же размеры неоднородностей значительно больше световой длины волны, то спектральный состав рассеянного света практически совпадает со спектральным составом первичного пучка. Этим объясняется, например, белый цвет облаков.

Молекулярное рассеяние. Даже тщательно очищенные от посторонних примесей и загрязнений жидкости и газы в некоторой степени рассеивают свет. М. Смолуховский (1908) выяснил, что причиной оптических неоднородностей в этом случае являются флуктуации плотности. Имеются в виду отклонения в пределах малых объемов плотности от ее среднего значения, возникающие в процессе хаотического теплового движения молекул среды. Рассеяние света, обусловленное этими флуктуациями плотности, называют молекулярным рассеянием.

Молекулярным рассеянием объясняется голубой цвет неба. Непрерывно возникающие в атмосфере флуктуации плотности в малых объемах приводят согласно закону Рэлея к тому, что синие и голубые составляющие солнечного света рассеиваются сильнее, чем желтые и красные. При восходе и заходе Солнца прямой солнечный свет проходит через большую толщу атмосферы, и при этом большая доля коротковолновой части спектра теряется на рассеяние. Из прямого света до поверхности Земли доходит преимущественно красная составляющая спект-

ра. Вот почему при восходе и заходе Солнце кажется красным. Аналогично объясняется и красный цвет зари.

Эффект, связанный с молекулярным рассеянием света, зависит от температуры: с ее ростом он увеличивается, и это подтверждает эксперимент.

Ослабление узкого светового пучка. В результате рассеяния интенсивность узкого светового пучка убывает в направлении распространения быстрее, чем в случае одного лишь поглощения. Поэтому для мутной среды в выражении (7.18) вместо коэффициента поглощения \(x\) должен стоять коэффициент ослабления


\begin{equation*}
\mu=x+x^{\prime}, \tag{7.22}
\end{equation*}


где \(\varkappa^{\prime}\) - коэффициент экстинкиии, связанный с рассеивающими свойствами среды. Тогда интенсивность пучка будет изменяться с проходимым расстоянием \(x\) как


\begin{equation*}
I=I_{0} \mathrm{e}^{-\mu x} \tag{7.23}
\end{equation*}


Еще раз отметим, что эта зависимость относится к узкому световому пучку.

\section*{Задачи}
7.1. Дисперсия света. Электромагнитная волна распространяется в разреженной плазме, концентрация свободных электронов которой равна \(N_{0}\). Пренебрегая взаимодействием волны с ионами плазмы, найти зависимость фазовой скорости волны от ее частоты \(\omega\).\\
Решение. В случае плазмы (электроны свободные) собственная частота колебаний электронов \(\omega_{0}=0\), поэтому согласно (7.11) диэлектрическая проницаемость

\[
\varepsilon=1-b / \omega^{2}
\]

где \(b=N_{0} e^{2} / \varepsilon_{0} m_{e}, m_{e}\) - масса электрона. Следовательно, фазовая скорость

\[
v=\frac{c}{\sqrt{\varepsilon}}=\frac{c}{\sqrt{1-b / \omega^{2}}}
\]

7.2. Найти концентрацию свободных электронов ионосферы, если для радиоволн с частотой \(v=100\) МГц ее показатель преломления \(n=0,90\).

Решение. Согласно (7.11) для плазмы ( \(\omega_{0}=0\) )


\begin{equation*}
n^{2}=1-b / \omega^{2}, \tag{1}
\end{equation*}


где \(b=N_{0} e^{2} / \varepsilon_{0} m_{e}, \omega=2 \pi \nu\). После подстановки выражений для \(b\) и \(\omega\) в (1) получим:

\[
N_{0}=\frac{4 \pi^{2} \varepsilon_{0} m_{e} v^{2}}{e^{2}}\left(1-n^{2}\right) \approx 2,4 \cdot 10^{7} \mathrm{~cm}^{-3} .
\]

7.3. Имея в виду, что для достаточно жестких рентгеновских лучей электроны вещества можно считать свободными, определить, насколько отличается от единицы показатель преломления графита для рентгеновского излучения с длиной волны \(\lambda=50\) пм (в вакууме).

Решение. Исходя из формулы (7.11) и учитывая, что в рассматриваемом случае \(\omega_{0}=0\), запишем:

\[
n^{2}=1-\frac{b}{\omega^{2}}=1-\frac{b \lambda^{2}}{4 \pi^{2} c^{2}}
\]

где \(b=N_{0} e^{2} / \varepsilon_{0} m_{e}, N_{0}=\left(N_{A} / A\right) \rho Z, \rho\) - плотность графита \(\left(1,6 г /\right.\) см \(\left.^{3}\right)\), \(Z\) - число электронов в атоме ( \(Z=6\) ). Искомое различие


\begin{equation*}
n-1=\sqrt{1-b / \omega^{2}}-1 \tag{1}
\end{equation*}


Вычислив значение величины \(b / \omega^{2}\), обнаруживаем, что она значительно меньше единицы, поэтому формулу (1) можно упростить:

\[
n-1=-b / 2 \omega^{2}=-b \lambda^{2} / 8 \pi^{2} c^{2}=-5,4 \cdot 10^{-7}
\]

где \(\omega=2 \pi c / \lambda\).\\
7.4. Групповая скорость. Найти зависимость между групповой \(u\) и фазовой \(v\) скоростями для следующих законов дисперсии:\\
а) \(v \sim k\);\\
б) \(v \sim 1 / \omega^{2}\).

Здесь \(k\) и \(\omega\) - волновое число и циклическая частота.\\
Решение. а) По определению, \(u=\mathrm{d} \omega / \mathrm{d} k\), где \(\omega=v k\). Тогда


\begin{equation*}
u=\frac{\mathrm{d}}{\mathrm{~d} k}(v k)=v+k \frac{\mathrm{~d} v}{\mathrm{~d} k} \tag{1}
\end{equation*}


Пусть \(v=a k\), где \(a\) - некоторая постоянная. В этом случае (1) примет вид

\[
u=v+a k=2 v .
\]

б) Пусть \(v=\alpha / \omega^{2}\), \(\alpha\) - некоторая постоянная. Тогда

\[
\frac{\mathrm{d} k}{\mathrm{~d} \omega}=\frac{\mathrm{d}}{\mathrm{~d} \omega}\left(\frac{\omega}{v}\right)=\frac{\mathrm{d}}{\mathrm{~d} \omega}\left(\frac{\omega^{3}}{\alpha}\right)=\frac{3}{v} .
\]

Поэтому групповая скорость

\[
u=\frac{\mathrm{d} \omega}{\mathrm{~d} k}=\frac{v}{3}
\]

7.5. Показатель преломления вещества для близких длин волн \(\lambda_{1}\) и \(\lambda_{2}\) (в вакууме) равен соответственно \(n_{1}\) и \(n_{2}\). Определить групповую скорость света в области данных длин волн.

Решение. Преобразуем в соответствии с условиями задачи выражение для групповой скорости (7.13) так, чтобы оно содержало \(n, \lambda\) и производную \(\mathrm{d} n / \mathrm{d} \lambda\). Для этого запишем


\begin{align*}
& u=\frac{\mathrm{d} \omega}{\mathrm{~d} k}=\frac{\mathrm{d}}{\mathrm{~d} k}(v k)=\frac{\mathrm{d}}{\mathrm{~d} k}\left(\frac{c k}{n}\right)=  \tag{1}\\
= & c \frac{n-k(\mathrm{~d} n / \mathrm{d} k)}{n^{2}}=\frac{c}{n}\left(1-\frac{k}{n} \frac{\mathrm{~d} n}{\mathrm{~d} k}\right) .
\end{align*}


Теперь учтем, что \(k=2 \pi / \lambda\) и \(\mathrm{d} k=-2 \pi \mathrm{~d} \lambda / \lambda^{2}\). После подстановки этих выражений в (1) получим:


\begin{equation*}
u=\frac{c}{n}\left(1+\frac{\lambda}{n} \frac{\mathrm{~d} n}{\mathrm{~d} \lambda}\right) \approx \frac{c}{n}\left(1+\frac{\langle\lambda\rangle}{\langle n\rangle} \frac{\Delta n}{\Delta \lambda}\right) \tag{2}
\end{equation*}


где \(\langle\lambda\rangle=\left(\lambda_{1}+\lambda_{2}\right) / 2,\langle n\rangle=\left(n_{1}+n_{2}\right) / 2, \Delta n=n_{2}-n_{1}, \Delta \lambda=\lambda_{2}-\lambda_{1}\).\\
7.6. В некоторой среде связь между групповой и фазовой скоростями электромагнитной волны имеет вид \(u v=c^{2}\), где \(c\) - скорость света в вакууме. Найти зависимость диэлектрической проницаемости этой среды от частоты волны, \(\varepsilon(\omega)\).

Решение. Исходим из выражения (7.13) для групповой скорости \(u=\mathrm{d} \omega / \mathrm{d} k\), где согласно (7.12) \(k=\omega / v=\omega n / c\). Учитывая, что \(n=\sqrt{\varepsilon}\), перепишем предыдущее соотношение для \(k\) так:

\[
k=\omega \sqrt{\varepsilon} / c
\]

Теперь найдем производную \(\mathrm{d} k / \mathrm{d} \omega\) :


\begin{equation*}
\frac{\mathrm{d} k}{\mathrm{~d} \omega}=\frac{1}{c}\left(\sqrt{\varepsilon}+\omega \frac{\mathrm{d} \varepsilon / \mathrm{d} \omega}{2 \sqrt{\varepsilon}}\right) . \tag{1}
\end{equation*}


Это выражение равно \(1 / u\), или


\begin{equation*}
\frac{1}{u}=\frac{v}{c^{2}}=\frac{1}{c \sqrt{\varepsilon}} \tag{2}
\end{equation*}


Приравняв (1) и (2), запишем:

\[
\sqrt{\varepsilon}+\frac{\omega}{2 \sqrt{\varepsilon}} \frac{d \varepsilon}{d \omega}=\frac{1}{\sqrt{\varepsilon}}
\]

Последнее уравнение можно упростить:

\[
\varepsilon+\frac{\omega}{2} \frac{\mathrm{~d} \varepsilon}{\mathrm{~d} \omega}=1
\]

Разделив переменные \(\varepsilon\) и \(\omega\), получим:

\[
\frac{\mathrm{d} \varepsilon}{\varepsilon-1}=-2 \frac{\mathrm{~d} \omega}{\omega}
\]

Интегрируем это уравнение:


\begin{equation*}
\ln (\varepsilon-1)=-2 \ln \omega+\ln A, \tag{3}
\end{equation*}


где \(A\) - произвольная постоянная. Потенцируя (3), находим: \(\varepsilon-1=A / \omega^{2}\), и окончательно

\[
\varepsilon=1-A / \omega^{2},
\]

где \(A\) - положительная постоянная, определяемая экспериментально.\\
7.7. Поглощение света. Из некоторого прозрачного вещества изготовили две пластинки: одну толщиной \(h_{1}\), другую толщиной \(h_{2}\). Введя поочередно эти пластинки перпендикулярно в пучок монохроматического света, обнаружили, что первая пластинка пропускает \(\tau_{1}\) светового потока, а вторая \(-\tau_{2}\). Пренебрегая вторичными отражениями, найти коэффициент поглощения \(\varkappa\) этого вещества.

Решение. В условии этой задачи описан обычный метод измерения коэффициента поглощения в случае, когда неизвестен коэффициент отражения от каждой поверхности пластинки. В этом случае поступают так. Сначала запишем выражения для

интенсивности света, прошедшего через первую и вторую пластинки:


\begin{align*}
& I_{1}=I_{0}(1-\rho)^{2} \exp \left(-\varkappa h_{1}\right), \\
& I_{2}=I_{0}(1-\rho)^{2} \exp \left(-\varkappa h_{2}\right), \tag{1}
\end{align*}


где \(I_{0}\) - интенсивность падающего света, \(\rho\) - неизвестный коэффициет отражения (он одинаков для обеих поверхностей пластинки). Имея в виду, что \(I_{1} / I_{0}=\tau_{1}\) и \(I_{2} / I_{0}=\tau_{2}\), найдем отношение обеих формул (1) и тем самым исключим неизвестное \(\rho\). В результате получим:

\[
\tau_{1} / \tau_{2}=\exp \left[\varkappa\left(h_{2}-h_{1}\right],\right.
\]

откуда, потенцируя, находим:

\[
\varkappa=\frac{\ln \left(\tau_{1} / \tau_{2}\right)}{h_{2}-h_{1}} .
\]

7.8. Монохроматический пучок света падает нормально на поверхность плоскопараллельной пластинки толщиной \(h\). Коэффициент поглощения вещества пластинки линейно изменяется вдоль нормали к ней от значения \(x_{1}\) до \(x_{2}\). Коэффициент отражения от каждой поверхности считать одинаковым и равным \(\rho\). Пренебрегая вторичными отражениями, найти коэффициент пропускания \(\tau\) для данной пластинки.\\
Решение. Выделим в пластинке бесконечно тонкий слой от \(x\) до \(x+\mathrm{d} x\), в пределах которого коэффициент поглощения равен \(x\). Убыль интенсивности света, прошедшего через этот слой, запишем как


\begin{equation*}
-\mathrm{d} I=\varkappa I(x) \mathrm{d} x . \tag{1}
\end{equation*}


В нашем случае коэффициент поглощения зависит от \(x\) линейно, а именно


\begin{equation*}
\varkappa=\varkappa_{1}+\frac{\varkappa_{2}-\varkappa_{1}}{h} x . \tag{2}
\end{equation*}


После подстановки (2) в (1) получим:

\[
-\frac{\mathrm{d} I}{I}=\varkappa_{1} \mathrm{~d} x+\frac{\varkappa_{2}-\varkappa_{1}}{h} x \mathrm{~d} x .
\]

Проинтегрировав это уравнение по \(x\) от 0 до \(h\) и по \(I\) от \(I_{0}(1-\rho)\) до \(I_{h}\) (рис. 7.12), найдем\\
\(-\ln \frac{I_{h}}{I_{0}(1-\rho)}=\varkappa_{1} h+\frac{\varkappa_{2}-\varkappa_{1}}{h} \frac{h^{2}}{2}\),\\
или после потенцирования

\begin{figure}[h]
\begin{center}
  \includegraphics[alt={},max width=\textwidth]{950fefba-d83b-4c8f-96e7-0bd5a7eb90be-250_260_450_193_712}
\captionsetup{labelformat=empty}
\caption{Рис. 7.12}
\end{center}
\end{figure}

\[
I_{h}=I_{0}(1-\rho) \exp \left(-h \frac{\varkappa_{1}+\varkappa_{2}}{2}\right) .
\]

После прохождения второй поверхности пластинки интенсивность света окажется \(I=I_{h}(1-\rho)\), см. рис. 7.12.\\
В результате искомый коэффициент пропускания

\[
\tau=\frac{I}{I_{0}}=(1-\rho)^{2} \exp \left(-h \frac{\varkappa_{1}+\varkappa_{2}}{2}\right) .
\]

7.9. Точечный монохроматический источник, испускающий световой поток \(\Phi_{0}\), находится в центре сферического слоя однородного вещества, внутренний радиус которого равен \(a\), наружный - \(b\). Коэффициент поглощения вещества слоя равен \(\varkappa\), коэффициент отражения каждой поверхности - \(\rho\). Пренебрегая вторичными отражениями, найти интенсивность света на выходе из этого слоя.

Решение. Записав убыль светового потока при прохождении бесконечно тонкого сферического слоя вещества толщиной от \(r\) до \(r+\mathrm{d} r\) (рис. 7.13), получим\\
\(-\mathrm{d} \Phi=\varkappa \Phi(r) \mathrm{d} r\), или \(-\mathrm{d} \Phi / \Phi=\varkappa \mathrm{d} r\).\\
Проинтегрировав последнее уравнение по \(r\) от \(a\) до \(b\) и по \(\Phi\) от \(\Phi_{a}=\Phi_{0}(1-\rho)\) до \(\Phi_{b}\),

\begin{figure}[h]
\begin{center}
  \includegraphics[alt={},max width=\textwidth]{950fefba-d83b-4c8f-96e7-0bd5a7eb90be-250_307_304_1161_836}
\captionsetup{labelformat=empty}
\caption{Рис. 7.13}
\end{center}
\end{figure}

получим, что перед выходом из сферического слоя поток


\begin{equation*}
\Phi_{b}=\Phi_{0}(1-\rho) \mathrm{e}^{-\varkappa(b-a)} . \tag{1}
\end{equation*}


Вышедший из сферического слоя поток


\begin{equation*}
\Phi=\Phi_{b}(1-\rho) . \tag{2}
\end{equation*}


Искомая интенсивность на выходе из этого слоя

\[
I=\frac{\Phi}{4 \pi b^{2}}=\frac{\Phi_{0}(1-\rho)^{2}}{4 \pi b^{2}} \mathrm{e}^{-\chi(b-a)}
\]

где учтены формулы (1) и (2).\\
7.10. Пучок естественного монохроматического света интенсивности \(I_{0}\) падает на систему из двух скрещенных поляризаторов, между которыми находится трубка с некоторой оптически неактивной жидкостью в продольном магнитном поле с индукцией \(B\). Длина трубки \(l\), коэффициент поглощения жидкости \(x\) и постоянная Верде \(V\). Пренебрегая отражениями на торцах трубки, найти интенсивность света, прошедшего через эту систему.

P е ш е н и е. После прохождения первого поляризатора \(P\) свет становится поляризованным с интенсивностью, равной \(I_{0} / 2\). До второго поляризатора \(P^{\prime}\) дойдет свет, интенсивность которого согласно закону Бугера (7.18) будет равна ( \(I_{0} / 2\) ) \(\mathrm{e}^{-\chi l}\).

\begin{figure}[h]
\begin{center}
  \includegraphics[alt={},max width=\textwidth]{950fefba-d83b-4c8f-96e7-0bd5a7eb90be-251_318_402_842_172}
\captionsetup{labelformat=empty}
\caption{Рис. 7.14}
\end{center}
\end{figure}

Кроме того, в магнитном поле произойдет поворот направления линейной поляризации на угол \(\varphi=V l B\), см. рис. 7.14. Поэтому поляризатор \(P^{\prime}\) пропустит согласно закону Малюса только ту часть интенсивности света, которая пропорциональна \(\cos ^{2}\left(90^{\circ}-\varphi\right)=\sin ^{2} \varphi\).\\
В результате интенсивность прошедшего света

\[
I=\left(I_{0} / 2\right) \mathrm{e}^{-x l} \sin (V l B) .
\]

\section*{Приложения}
\section*{1. Поведение плоской волны на границе двух диэлектриков}
Рассмотим случай, когда плоская электромагнитная волна падает на границу раздела двух однородных и изотропных диэлектриков с показателями преломления \(n_{1}\) и \(n_{2}\). Пусть волновой вектор \(\mathbf{k}_{1}\) характеризует падающую волну, а \(\mathbf{k}^{\prime}\) и \(\mathbf{k}^{\prime \prime}\) - отраженную и преломленную (рис. П.1). Из соображений симметрии следует, что в данном случае все три вектора лежат в одной плоскости - плоскости падения.

Найдем связь между модулями этих трех векторов, воспользовавшись условием для тангенциальных составляющих

\begin{figure}[h]
\begin{center}
  \includegraphics[alt={},max width=\textwidth]{950fefba-d83b-4c8f-96e7-0bd5a7eb90be-252_329_428_488_732}
\captionsetup{labelformat=empty}
\caption{Рис. П. 1}
\end{center}
\end{figure}


\begin{equation*}
E_{1 \tau}=E_{2 \tau}, \tag{1}
\end{equation*}


где орт \(\tau\) выберем вдоль оси \(X\), лежащей в плоскости падения. Рассмотрим падающую волну \(\mathbf{E}=\mathbf{E}_{m} \cos (\omega t-\mathbf{k r})\), колебания вектора \(\mathbf{E}\) которой происходят в плоскости, образующей произвольный угол с плоскостью падения. Тогда результирующее поле в первой среде


\begin{equation*}
\mathbf{E}_{1}=\mathbf{E}+\mathbf{E}^{\prime}=\mathbf{E}_{m} \cos (\omega t-\mathbf{k r})+\mathbf{E}_{m}^{\prime} \cos \left(\omega^{\prime} t-\mathbf{k}^{\prime} \mathbf{r}+\alpha^{\prime}\right), \tag{2}
\end{equation*}


а во второй среде


\begin{equation*}
\mathbf{E}_{2}=\mathbf{E}_{m}^{\prime \prime} \cos \left(\omega^{\prime \prime} t-\mathbf{k}^{\prime \prime} \mathbf{r}+\alpha^{\prime \prime}\right) . \tag{3}
\end{equation*}


Здесь радиус-вектор r всех трех волн взят относительно одной и той же точки, которую мы возьмем для простоты на границе раздела диэлектриков. Тогда для всех трех скалярных произведений проекция \(k_{z}=0\) и у границы раздела \(y=0\), поэтому


\begin{equation*}
\mathbf{k r}=k_{x} x+k_{y} y+k_{z} z=k_{x} x . \tag{4}
\end{equation*}


Согласно (1) тангенциальные составляющие у границы раздела должны быть одинаковы, и значит с учетом (4):


\begin{gather*}
E_{m \tau} \cos \left(\omega t-k_{x} x\right)+E_{m \tau}^{\prime} \cos \left(\omega^{\prime} t-k_{x}^{\prime} x+\alpha^{\prime}\right)= \\
=E_{m \tau}^{\prime \prime} \cos \left(\omega^{\prime \prime} t-k_{x}^{\prime \prime} x+\alpha^{\prime \prime}\right) . \tag{5}
\end{gather*}


Это равенство должно выполняться при любых \(t\), отсюда следует, что


\begin{equation*}
\omega=\omega^{\prime}=\omega^{\prime \prime} . \tag{6}
\end{equation*}


Оно должно выполняться и при любых \(x\), откуда следует, что


\begin{equation*}
k_{x}=k_{x}^{\prime}=k_{x}^{\prime \prime} . \tag{7}
\end{equation*}


Последнее условие можно записать с помощью рис. П. 1 так:

\[
k \sin \vartheta=k^{\prime} \sin \vartheta^{\prime}=k^{\prime \prime} \sin \vartheta^{\prime \prime},
\]

где \(k=k^{\prime}=\omega / v_{1}\) и \(k^{\prime \prime}=\omega / v_{2}\), поэтому

\[
\frac{\omega}{v_{1}} \sin \vartheta=\frac{\omega}{v_{1}} \sin \vartheta^{\prime}=\frac{\omega}{v_{2}} \sin \vartheta^{\prime \prime}
\]

Отсюда следует, что

\[
\vartheta^{\prime}=\vartheta \quad \text { и } \quad \frac{\sin \vartheta}{\sin \vartheta^{\prime \prime}}=\frac{v_{1}}{v_{2}}=\frac{n_{1}}{n_{2}},
\]

т.е. законы отражения и преломления.

\section*{2. Формула сферической преломляющей поверхности}
Установим связь между положениями источника \(S\) и его изображением \(S^{\prime}\), т. е. отрезками \(s\) и \(s^{\prime}\). Мы знаем, что для получения четкого изображения все оптические длины лучей, выходящих из точки \(S\) и сходящихся в точке \(S^{\prime}\), должны быть одинаковыми (стационарный случай принципа Ферма). С этой целью рассмотрим произвольный луч \(S N S^{\prime}\) (рис. П.2). Далее, проведем дуги окружностей из центров \(S\) и \(S^{\prime}\) радиусами \(S O\) и \(S^{\prime} N\). Мы должны убедиться в том, что оптические длины путей \(D N\) и \(O B\) одинаковы:


\begin{equation*}
n \cdot D N=n^{\prime} \cdot O B \tag{1}
\end{equation*}


Для параксиальных лучей \(D N \approx A O+O C\). Найдем последние два отрезка. Сначала \(A O\) : из рисунка видно, что

\[
r^{2}=(S D)^{2}-(S A)^{2}=(S D+S A)(S D-S A) \approx 2(-s) A O
\]

откуда \(A O=r^{2} /(-2 s)\).\\
Аналогично \(O C=r^{\prime 2} / 2 R\), и мы находим сумму \(A O+O C\), т. е. \(D N\) :

\[
D N=r^{2} /(-2 s)+r^{\prime 2} / 2 R
\]

\begin{figure}[h]
\begin{center}
  \includegraphics[alt={},max width=\textwidth]{950fefba-d83b-4c8f-96e7-0bd5a7eb90be-254_364_1094_155_70}
\captionsetup{labelformat=empty}
\caption{Рис. П. 2}
\end{center}
\end{figure}

В свою очередь

\[
O B=O C-B C=r^{\prime 2} / 2 R-r^{\prime 2} / 2 s^{\prime} .
\]

Подставив выражения \(D N\) и \(O B\) в формулу (1), получим


\begin{equation*}
n\left(\frac{r^{2}}{-2 s}+\frac{r^{\prime 2}}{2 R}\right)=n^{\prime}\left(\frac{r^{\prime 2}}{2 R}-\frac{r^{\prime 2}}{2 s^{\prime}}\right) . \tag{2}
\end{equation*}


Остается учесть, что в параксиальной области \(r^{\prime} \approx r\), поэтому их можно сократить в (2), и мы получим после перегруппировки

\[
\frac{n^{\prime}}{s^{\prime}}-\frac{n}{s}=\frac{n^{\prime}-n}{R} .
\]

\section*{3. Излучение Вавилова-Черенкова}
Известно, что заряженная частица, движущаяся в вакууме с постоянной скоростью, не излучает. Иначе обстоит дело при ее движении в веществе.

Пусть такая частица движется с постоянной скоростью \(\mathbf{v}_{0}\) в однородной прозрачной среде. Своим полем она возбуждает атомы и молекулы среды, и последние становятся центрами излучения электромагнитных волн.

При равномерном движении частицы эти волны оказываются когерентными и поэтому интерферируют между собой. Если скорость \(v_{0}\) частицы больше фазовой скорости \(v\) света в данной среде ( \(v_{0}>v\) ), то волны, исходящие от нее в различные моменты времени, в результате интерференции усиливают друг друга в определенном направлении, и мы будем наблюдать максимум интенсивности.

Действительно, максимум излучения будет наблюдаться под углом Э к направлению движения частицы (рис. П.3), если за время прохождения частицы от точки \(A\) до \(B\) свет, испущенный в точке \(A\), пройдет

\begin{figure}[h]
\begin{center}
  \includegraphics[alt={},max width=\textwidth]{950fefba-d83b-4c8f-96e7-0bd5a7eb90be-255_237_451_153_68}
\captionsetup{labelformat=empty}
\caption{Рис. П. 3}
\end{center}
\end{figure}

путь \(A C\). Тогда световые колебания в точках \(C\) и \(B\) будут когерентны и синфазны. Это же относится ко всем точкам отрезка \(C B\). И чтобы колебания усиливали друг друга, отрезок \(C B\) должен быть перпендикулярен к \(A C\). Таким образом, \(A C=A B \cos \vartheta\), или \(v=v_{0} \cos \vartheta\), где \(v\) - фазовая скорость ( \(v=c / n\) ). Отсюда следует, что угол Э, под которым будет испускаться излучение, определяется формулой


\begin{equation*}
\cos \vartheta=\frac{c / n}{v_{0}} . \tag{1}
\end{equation*}


Если это условие выполнено, то все колебания в направлении угла \(\vartheta\) будут распространяться в одной фазе, какова бы ни была длина отрезка \(A B\). В этом случае при интерференции волн произойдет их взаимное усиление.

\begin{figure}[h]
\begin{center}
  \includegraphics[alt={},max width=\textwidth]{950fefba-d83b-4c8f-96e7-0bd5a7eb90be-255_231_362_857_68}
\captionsetup{labelformat=empty}
\caption{Рис. П. 4}
\end{center}
\end{figure}

Итак, в направлении, определяемом условием (1), частица (точнее - среда, в которой она движется) будет излучать электромагнитные волны. В остальных же направлениях излучения не будет. Это наиболее характерное свойство данного излучения: оно испускается лишь вдоль образующих конуса, ось которого совпадает с направлением движения частицы (рис. П.4).\\
Такое излучение было обнаружено экспериментально в 1934 г. и получило название излучения Вавилова-Черенкова.

Излучение заряженной частицы приводит, конечно, к ее торможению. Но это торможение является не причиной излучения, а его следствием. Если бы к частице приложить силу, уравновешивающую тормозящие силы, то ее ускорение исчезнет, а излучение Вавилова-Черенкова останется. Именно так надо понимать утверждение, что заряженная частица, равномерно движущаяся в среде, излучает, если ее скорость больше фазовой скорости света в этой среде.

Излучение Вавилова-Черенкова нашло широкое применение в современной экспериментальной физике. На его основе созданы черенковские счетчики заряженных частиц, с помощью которых можно не только регистрировать эти частицы, но и определять величину и направление их скорости.

\section*{4. Единицы физических величин}
\begin{center}
\begin{tabular}{|l|l|}
\hline
А - ампер & Кл - кулон \\
\hline
Å - ангстрем & лм - люмен \\
\hline
В - вольт & мин - минута \\
\hline
Вб - вебер & Мкс - максвелл \\
\hline
Вт - ватт & H - ньютон \\
\hline
Гс - гаусс & Па - паскаль \\
\hline
Гц - герц & рад - радиан \\
\hline
Дж - джоуль & ср - стерадиан \\
\hline
дптр - диоптрия & Тл - тесла \\
\hline
К - кельвин & Ф - фарад \\
\hline
кд - кандела & Э - эрстед \\
\hline
\end{tabular}
\end{center}

\section*{5. Десятичные приставки к названиям единиц}
\begin{center}
\begin{tabular}{|l|l||}
\hline\hline
\(\mathrm{T}-\) тера, \(10^{12}\) & м - милли, \(10^{-3}\) \\
\(\Gamma-\) гига, \(10^{9}\) & мк - микро, \(10^{-6}\) \\
\(\mathrm{M}-\) мега, \(10^{6}\) & \(\mathrm{H}-\) нано, \(10^{-9}\) \\
\(\mathrm{~K}-\) кило, \(10^{3}\) & п - пико, \(10^{-12}\) \\
\(\mathrm{c}-\) санти, \(10^{-2}\) & \(\mathrm{q}-\) фемто, \(10^{-15}\) \\
\hline
\end{tabular}
\end{center}

\section*{6. Греческий алфавит}
\begin{center}
\begin{tabular}{|l|l|l|}
\hline
А, \(\alpha\) - альфа & I, ι - йота & Р, \(\rho\) - po \\
\hline
B, \(\beta\) - бета & \(\mathrm{K}, x\) - каппа & Σ, \(\sigma\) - сигма \\
\hline
Г, \(\gamma\) - гамма & \(\Lambda, \lambda\) - ламбда & T, \(\tau\) - тау \\
\hline
\(\Delta, \delta\) - дельта & М, μ - мю & Y, υ- ипсилон \\
\hline
Е, \(ε\) - эпсилон & \(\mathrm{N}, v\) - ню & \(\Phi, φ\) - фи \\
\hline
\(\mathrm{Z}, \zeta\) - дзета & Ξ, ξ - кси & \(\mathrm{X}, \chi\) - хи \\
\hline
H, η - эта & О, о - омикрон & \(\Psi, \Psi\) - пси \\
\hline
\(\Theta, \theta, \vartheta\) - тета & П, \(\pi\) - пи & \(\Omega, \omega\) - омега \\
\hline
\end{tabular}
\end{center}

\begin{enumerate}
  \setcounter{enumi}{6}
  \item Единицы величин в СИ и системе Гаусса
\end{enumerate}

\begin{center}
\begin{tabular}{|l|l|l|l|}
\hline
\multirow{2}{*}{Величина} & \multicolumn{2}{|c|}{Единица величины} & \multirow[t]{2}{*}{\begin{tabular}{l}
Отношение ед. СИ \\
ед. СГС \\
\end{tabular}} \\
\hline
 & СИ & СГС &  \\
\hline
Скорость & м/с & cm/c & \(10^{2}\) \\
\hline
Ускорение & м/с \({ }^{2}\) & см \(/ \mathrm{c}^{2}\) & \(10^{2}\) \\
\hline
Частота колебаний & Гц & Гц & 1 \\
\hline
Круговая частота & \(\mathrm{c}^{-1}\) & \(\mathrm{c}^{-1}\) & 1 \\
\hline
Угловая скорость & рад/с & рад/с & 1 \\
\hline
Сила & H & дин & \(10^{5}\) \\
\hline
Давление, напряжение & Па & дин/см \({ }^{2}\) & 10 \\
\hline
Импульс & кг•м/с & г.см/с & \(10^{5}\) \\
\hline
Энергия, работа & Дж & эрг & \(10^{7}\) \\
\hline
Мощность & Вт & эрг/с & \(10^{7}\) \\
\hline
Количество электричества & Кл & СГСЭ-ед. & \(3 \cdot 10^{9}\) \\
\hline
Потенциал & B/м & СГСЭ-ед. & 1/300 \\
\hline
Напряженность электрического поля & B/m & СГСЭ-ед. & \(1 /\left(3 \cdot 10^{4}\right)\) \\
\hline
Электрическое смещение & Кл/м \({ }^{2}\) & СГСЭ-ед. & \(12 \pi \cdot 10^{5}\) \\
\hline
Электрический момент диполя & Кл•м & СГСЭ-ед. & \(3 \cdot 10^{11}\) \\
\hline
Поляризованность & Кл \(/ \mathrm{M}^{2}\) & СГСЭ-ед. & \(3 \cdot 10^{5}\) \\
\hline
Емкость & Ф & см & \(9 \cdot 10^{11}\) \\
\hline
Сопротивление & Ом & СГСЭ-ед. & \(1 /\left(9 \cdot 10^{11}\right)\) \\
\hline
Магнитная индукция & Тл & Гс & \(10^{4}\) \\
\hline
Магнитный поток & Вб & Мкс & \(10^{8}\) \\
\hline
Напряженность магнитного поля & А/м & Э & \(4 \pi / 10^{3}\) \\
\hline
Индуктивность & Гн & см & \(10^{9}\) \\
\hline
Световой поток & лм & лм & 1 \\
\hline
Освещенность & лк & фот & \(10^{-4}\) \\
\hline
\end{tabular}
\end{center}

Примечание. Электрические и магнитные единицы в СГС даны здесь в гауссовой системе.

\section*{8. Основные формулы электродинамики в СИ и гауссовой системе}
\begin{center}
\begin{tabular}{|l|l|l|}
\hline
Наименование & СИ & Гауссова система \\
\hline
Напряженность поля точечного заряда & \(E=\frac{1}{4 \pi \varepsilon_{0}} \frac{q}{r^{2}}\) & \(E=\frac{q}{r^{2}}\) \\
\hline
Потенциал поля точечного заряда & \(\varphi=\frac{1}{4 \pi \varepsilon_{0}} \frac{q}{r}\) & \(\varphi=\frac{q}{r}\) \\
\hline
Связь между \(\mathbf{E}\) и φ & \multicolumn{2}{|c|}{\(\mathbf{E}=-\nabla \varphi\)} \\
\hline
Связь между \(\mathbf{P}\) и \(\mathbf{E}\) & \(\mathbf{P}=\varkappa \varepsilon_{0} \mathbf{E}\) & \(\mathbf{P}=-\varkappa \mathbf{E}\) \\
\hline
Связь между D и E & \(\mathbf{D}=\varepsilon \varepsilon_{0} \mathbf{E}\) & \(\mathbf{D}=\varepsilon \mathbf{E}\) \\
\hline
Плотность энергии электрического поля & \(w=\mathbf{E D} / 2\) & \(w=\mathbf{E D} / 8 \pi\) \\
\hline
Связь между В и Н & \(\mathbf{B}=\mu \mu_{0} \mathbf{H}\) & \(\mathbf{B}=\mu \mathbf{H}\) \\
\hline
Сила Лоренца & \(\mathbf{F}=q \mathbf{E}+q[\mathbf{v B}]\) & \(\mathbf{F}=q \mathbf{E}+\frac{q}{c}[\mathbf{v B}]\) \\
\hline
Плотность энергии магнитного поля & \(w=\mathbf{B H} / 2\) & \(w=\mathbf{B H} / 8 \pi\) \\
\hline
Уравнения Максвелла в дифференциальной форме & \( \begin{gathered} \nabla \times \mathbf{E}=-\frac{\partial \mathbf{B}}{\partial t} \\ \nabla \cdot \mathbf{B}=0 \\ \nabla \times \mathbf{H}=\mathbf{j}+\frac{\partial \mathbf{D}}{\partial t} \end{gathered} \) & \( \begin{gathered} \nabla \times \mathbf{E}=-\frac{1}{c} \frac{\partial \mathbf{B}}{\partial t} \\ \nabla \cdot \mathbf{B}=0 \\ \nabla \times \mathbf{H}=\frac{4 \pi}{c}\left(\mathbf{j}+\frac{1}{4 \pi} \frac{\partial \mathbf{D}}{\partial t}\right) \end{gathered} \) \\
\hline
Скорость электромагнитной волны в среде & \(v=\frac{1}{\sqrt{\varepsilon_{0} \mu_{0} \varepsilon \mu}}\) & \(v=\frac{c}{\sqrt{\varepsilon \mu}}\) \\
\hline
Связь между \(E\) и \(H\) в электромагнитной волне & \(E \sqrt{\varepsilon_{0} \varepsilon}=H \sqrt{\mu_{0} \mu}\) & \(E \sqrt{\varepsilon}=H \sqrt{\mu}\) \\
\hline
Плотность энергии электронного поля & \(w=\frac{\mathbf{E D}}{2}+\frac{\mathbf{B H}}{2}\) & \(w=\frac{\mathbf{E D}}{8 \pi}+\frac{\mathbf{B H}}{8 \pi}\) \\
\hline
Вектор Пойнтинга & \(\boldsymbol{\Pi}=[\mathbf{E H}]\) & \(\boldsymbol{\Pi}=\frac{c}{4 \pi}[\mathbf{E H}]\) \\
\hline
Мощность излучения диполя & \(P=\frac{\mu_{0}}{6 \pi c^{3}} \ddot{\mathbf{p}}^{2}\) & \(P=\frac{2}{3 c^{3}} \ddot{\mathbf{p}}^{2}\) \\
\hline
\end{tabular}
\end{center}

\section*{9. Некоторые физические константы}
\begin{center}
\begin{tabular}{|l|l|}
\hline
Скорость света в вакууме & \(c=299792458 \mathrm{~m} / \mathrm{c}\) (точно) \\
\hline
Постоянная Авогадро & \(N_{A}=6,022 \cdot 10^{23}\) моль \(^{-1}\) \\
\hline
Элементарный заряд & \(e=\left\{\begin{array}{l}1,602 \cdot 10^{-19} \text { Кл } \\ 4,803 \cdot 10^{-10} \text { СГСЭ }\end{array}\right.\) \\
\hline
Масса электрона & \(m_{e}=\left\{\begin{array}{l}0,911 \cdot 10^{-30} \text { кг } \\ 0,511 \text { МэВ }\end{array}\right.\) \\
\hline
Удельный заряд электрона & \(e / m_{e}=\left\{\begin{array}{l}1,76 \cdot 10^{11} \text { Кл } / \text { кг } \\ 5,27 \cdot 10^{17} \text { СГСЭ/г }\end{array}\right.\) \\
\hline
Электрическая постоянная & \( \begin{aligned} & \varepsilon_{0}=0,885 \cdot 10^{-11} \Phi / \mathrm{M} \\ & 1 / 4 \pi \varepsilon_{0}=9 \cdot 10^{9} \mathrm{M} / \Phi \end{aligned} \) \\
\hline
Магнитная постоянная & \( \begin{aligned} & \mu_{0}=1,257 \cdot 10^{-6} \Gamma \mathrm{H} / \mathrm{M} \\ & \mu_{0} / 4 \pi=10^{-7} \Gamma \mathrm{H} / \mathrm{M} \end{aligned} \) \\
\hline
\end{tabular}
\end{center}

\section*{Предметный указатель}
Амплитуда волны 8\\
Анализатор 192\\
Анализ поляризованного света 207

\begin{itemize}
  \item рентгеноструктурный 177
\end{itemize}

Анизотропия 197

\begin{itemize}
  \item при деформации 214
  \item в электрическом поле 215
\end{itemize}

Бел 30\\
Бизеркала Френеля 107\\
Билинза Бийе 109\\
Бипризма Френеля 105, 127\\
Вектор волновой 10

\begin{itemize}
  \item Пойнтинга 51
  \item световой 66
  \item Умова 23
\end{itemize}

Вентиль оптический 220\\
Вещества оптически активные 217

\begin{itemize}
  \item отрицательные и положительные 220
  \item право- или левовращающие 218
\end{itemize}

Волна гармоническая 8

\begin{itemize}
  \item левополяризованная 191
  \item опорная 179
  \item отраженная 69, 251
  \item падающая 69, 251
  \item плоскополяризованная 189
  \item правополяризованная 191
  \item предметная 178
  \item преломленная 251
  \item эллиптически-поляризованная 189\\
Волновая зона 58
  \item поверхность 9
\end{itemize}

Волновое число 9\\
Волновой вектор 10

\begin{itemize}
  \item пакет 235
\end{itemize}

Волны звуковые 29

\begin{itemize}
  \item когерентные 93
  \item линейно-поляризованные 189
  \item необыкновенные 197
  \item обыкновенные 197
  \item плоские 9
  \item поляризованные 189
  \item поперечные 7, 45
  \item продольные 7
  \item световые 67
  \item стоячие 25, 49
  \item сферические 11
  \item упругие 9
  \item цилиндрические 12
  \item электромагнитные 42
\end{itemize}

Восприимчивость диэлектрика 231\\
Вращение естественное 217

\begin{itemize}
  \item магнитное 219
\end{itemize}

Время когерентности 101\\
Высота звука 29

Голограмма 178\\
Голография 177\\
Громкость 29\\
Группа волн 235\\
Давление световое 53\\
Дебайграмма 177\\
Деполяризация 192\\
Деформация относительная 13\\
Децибел 30\\
Диаграмма векторная 92

\begin{itemize}
  \item фазовая 92
  \item направленности 59
\end{itemize}

Диоптрия 75\\
Диполь 58

Дипольный момент 58\\
Дисперсия 67, 229

\begin{itemize}
  \item аномальная 229
  \item вещества 229
  \item нормальная 229
  \item света 67, 229
  \item угловая 125, 170
\end{itemize}

Дифракция в параллельных лучах 152

\begin{itemize}
  \item от круглого отверстия 136, 154
  \item от множества отверстий 156
  \item от полуплоскости 145
  \item Фраунгофера 152
  \item 
  \begin{itemize}
    \item от щели 159
  \end{itemize}
  \item Френеля 140
  \item 
  \begin{itemize}
    \item от края полуплоскости 146
  \end{itemize}
  \item 
  \begin{itemize}
    \item от щели 150
  \end{itemize}
\end{itemize}

Дихроизм 199\\
Длина волны 8

\begin{itemize}
  \item когерентности 97, 98
  \item пути оптическая 72
\end{itemize}

Закон Брюстера 193

\begin{itemize}
  \item Бугера 240
  \item Гука 16
  \item Малюса 192
  \item отражения 71
  \item преломления 72
  \item Рэлея 242
  \item Снелла 72
\end{itemize}

Звук 29\\
Зеркало Ллойда 126\\
Зона волновая 58

\begin{itemize}
  \item интерференции 94
\end{itemize}

Зоны Френеля 136, 137

Излучение Вавилова-Черенкова 61, 253

\begin{itemize}
  \item дипольное 60
\end{itemize}

Изображение мнимое 75\\
Импульс волны электромагнитной 52\\
Интенсивность волны 23,67

\begin{itemize}
  \item главных максимумов 168
\end{itemize}

Интерференция волн 93

\begin{itemize}
  \item многолучевая 123
  \item поляризованных волн 209-213
\end{itemize}

Интерферометр звездный 103

\begin{itemize}
  \item Майкельсона 121
  \item Рэлея 126
  \item Фабри-Перо 124
\end{itemize}

Источник ламбертовский 83\\
Кандела 82\\
Когерентность 92

\begin{itemize}
  \item временная 97
  \item поляризованных волн 209
  \item пространственная 97
\end{itemize}

Кольца Ньютона 118\\
Колебания когерентные 92

\begin{itemize}
  \item собственные 28
  \item струны 28
\end{itemize}

Компенсатор 206\\
Коэффициент затухания волны 9

\begin{itemize}
  \item ослабления 244
  \item отражения 70
  \item поглощения 240
  \item пропускания 71
  \item зкстинкции 244
\end{itemize}

Кривая видности 80

\begin{itemize}
  \item относительной спектральной чувствительности 80\\
Кристалл одноосный 197, 198
  \item отрицательный 200
  \item положительный 200
\end{itemize}

Кристаллооптика 200\\
Критерий Рэлея 157, 170\\
Лазер 68\\
Лауэграмма 177\\
Линза тонкая 78\\
Луч 67

\begin{itemize}
  \item необыкновенный 197
  \item обыкновенный 197
  \item параксиальный 75
\end{itemize}

Лучепреломление двойное 197

Люкс 82\\
Люмен 81

Максимум белый 123\\
Максимумы главные 164

\begin{itemize}
  \item добавочные 166
\end{itemize}

Метод Дебая-Шерера 177

\begin{itemize}
  \item Лауэ 177
\end{itemize}

Минимумы интерференционные 165\\
Модуль Юнга 16

Направления пластинки главные 204

Обертоны 28\\
Область дисперсии 126, 172\\
Оператор Лапласа 15, 43\\
Оптика волновая 66

\begin{itemize}
  \item геометрическая 71
\end{itemize}

Опыт Юнга 97\\
Опыты Лебедева 53\\
Освещенность 82\\
Ось оптическая 74,189

\begin{itemize}
  \item 
  \begin{itemize}
    \item главная 79
  \end{itemize}
\end{itemize}

Относительная спектральная чувствительность 80\\
Отражение полное внутреннее 72\\
Пакет волновой 235\\
Параметр \(p 152\)\\
Период колебаний 8

\begin{itemize}
  \item решетки 163
  \item структуры 174
\end{itemize}

Плазма 234\\
Пластинка в полволны 205

\begin{itemize}
  \item 
  \begin{itemize}
    \item четверть волны 204
  \end{itemize}
  \item двупреломляющая 201
  \item зонная 141
  \item клиновидная 116
  \item кристалла KDP 217
\end{itemize}

Плоскости главные 78\\
Плоскость главная 198

\begin{itemize}
  \item падения 71
  \item поляризации 189
  \item пропускания поляризатоpa 191\\
Плотность импульса 52
  \item оптическая 67
  \item потока энергии 22, 51
\end{itemize}

Поверхности волновые 9 , 200

\begin{itemize}
  \item лучевых скоростей 200
\end{itemize}

Поглощение света 239\\
Показатель затухания 234

\begin{itemize}
  \item преломления 66
  \item 
  \begin{itemize}
    \item комплексный 234
  \end{itemize}
\end{itemize}

Полосы равного наклона 116, 122

\begin{itemize}
  \item равной толщины 117, 123
\end{itemize}

Поляризаторы 191\\
Поляризация волн 189

\begin{itemize}
  \item левая 204
  \item правая 204
\end{itemize}

Поляроид 199\\
Порог болевого ощущения 29

\begin{itemize}
  \item слышимости 29
\end{itemize}

Порядки спектра 158\\
Порядок интерференции 95\\
Постоянная Верде 219

\begin{itemize}
  \item вращения 218
  \item Керра 215
\end{itemize}

Потеря полуволны 112\\
Поток световой 81

\begin{itemize}
  \item энергии 81
\end{itemize}

Правило знаков 75,164\\
Преломление на сферической поверхности 74,75\\
Принцип Гюйгенса-Френеля 135

\begin{itemize}
  \item суперпозиции 25
  \item Ферма 73
\end{itemize}

Проницаемость диэлектрическая 42

\begin{itemize}
  \item магнитная 42
\end{itemize}

Просветление оптики 120\\
Пучности 25\\
Пятно Пуассона 141

Разность хода оптическая 95\\
Рассеяние света 241, 242

\begin{itemize}
  \item молекулярное 243
\end{itemize}

Расстояние межплоскостное 176

\begin{itemize}
  \item фокусное 76
\end{itemize}

Распределение интенсивности от щели 161

\begin{itemize}
  \item 
  \begin{itemize}
    \item от дифракционной решетки 169\\
Расходимость пучка дифракционная 155\\
Реплики 173\\
Решетка вибраторов 173
  \end{itemize}
  \item дифракционная 162
  \item 
  \begin{itemize}
    \item фазовая 173
  \end{itemize}
\end{itemize}

Свет естественный 67, 190

\begin{itemize}
  \item круго-поляризованный 189, 205
  \item линейно-поляризованный 68, 189
  \item плоскополяризованный 68, 189
  \item поляризованный 68, 189
  \item частично-поляризованный 191
  \item эллиптически-поляризованный 68, 189, 203\\
Светимость 82\\
Сечение главное 198\\
Сила оптическая 75
  \item света 81
\end{itemize}

Скорость групповая 236

\begin{itemize}
  \item звука в газе 20
  \item поперечной волны 17
  \item распространения энергии 239
  \item фазовая 10, 235
  \item электромагнитной волны 44
\end{itemize}

Спектр акустический 29

\begin{itemize}
  \item линейчатый 29
  \item сплошной 29
\end{itemize}

Спектроскопия рентгеновская 177\\
Спираль Корню 147, 148

\begin{itemize}
  \item Френеля 139, 140
\end{itemize}

Способность разрешающая\\
объектива 158

\begin{itemize}
  \item 
  \begin{itemize}
    \item решетки 170, 171
  \end{itemize}
  \item 
  \begin{itemize}
    \item спектрального прибора 171
  \end{itemize}
\end{itemize}

Среда мутная 242\\
Степень когерентности 97

\begin{itemize}
  \item поляризации 191
  \item монохроматичности 100
\end{itemize}

Стопа пластинок 195\\
Суперпозиция поляризованных волн 201\\
Счетчики черенковские 254

Таутохронность 73\\
Тембр 29\\
Теорема Бабине 162\\
Теория дисперсии 230\\
Точка узловая 91\\
Точки сопряженные 74

Увеличение линзы 80

\begin{itemize}
  \item системы 77
\end{itemize}

Угол Брюстера 193

\begin{itemize}
  \item полной поляризации 193
  \item предельный 72
  \item скольжения 174, 176
\end{itemize}

Узлы 25\\
Уравнение волновое 13, 15

\begin{itemize}
  \item волны 8
  \item 
  \begin{itemize}
    \item затухающей 9
  \end{itemize}
  \item 
  \begin{itemize}
    \item плоской 10
  \end{itemize}
  \item 
  \begin{itemize}
    \item стоячей 25, 49
  \end{itemize}
  \item 
  \begin{itemize}
    \item сферической 11
  \end{itemize}
  \item 
  \begin{itemize}
    \item цилиндрической 12
  \end{itemize}
\end{itemize}

Уравнения Максвелла 42\\
Условие главных максимумов 164

\begin{itemize}
  \item минимумов 160
\end{itemize}

Условия граничные 27, 69

\begin{itemize}
  \item Лауэ 175
\end{itemize}

Фаза волны 9\\
Флуктуации плотности 243

Фокус задний 76

\begin{itemize}
  \item передний 76
\end{itemize}

Фон 31\\
Формула Брэгга-Вульфа 176

\begin{itemize}
  \item Рэлея 237
  \item сферической поверхности 75, 252
  \item тонкой линзы 78
\end{itemize}

Формулы Френеля 196\\
Фотометрия 80\\
Цвета тонких пластинок 116\\
Цуг волн 68\\
Частоты собственные 28\\
Число волновое 9\\
Ширина зоны интерференции 106

\begin{itemize}
  \item интерференционной полосы 96
  \item когерентности 97, 101
\end{itemize}

Шкала интенсивности звука 30

Экраны дополнительные 162\\
Энергия волн упругих 20, 21

\begin{itemize}
  \item 
  \begin{itemize}
    \item электромагнитных 50
  \end{itemize}
\end{itemize}

Эффект Доплера для звука 33-36

\begin{itemize}
  \item 
  \begin{itemize}
    \item для электромагнитных волн 55
  \end{itemize}
  \item 
  \begin{itemize}
    \item поперечный 56
  \end{itemize}
  \item 
  \begin{itemize}
    \item продольный 55
  \end{itemize}
  \item запаздывания \(36,40,55,61\)
  \item Керра 215
  \item Поккельса 217
  \item Фарадея 219
\end{itemize}

Яркость 82, 83\\
Ячейка Керра 215

Минимальные системные требования определяются соответствующими требованиями программы Adobe Reader версии не ниже 11-й для операционных систем Windows, Mac OS, Android, iOS, Windows Phone u BlackBerry; экран 10"

Учебное электронное издание\\
Иродов Игорь Евгеньевич\\
ВОЛНОВЫЕ ПРОЦЕССЫ. ОСНОВНЫЕ ЗАКОНЫ\\
Ведущие редакторы Б. И. Копылов, Т. Г. Хохлова\\
Художник H. A. Новак\\
Технический редактор E. B. Денюкова\\
Компьютерная верстка: В. А. Носенко\\
Подписано к использованию 11.12.14.\\
Издательство «БИНОМ. Лаборатория знаний»\\
125167, Москва, проезд Аэропорта, д. 3\\
Телефон: (499) 157-5272\\
e-mail: \href{mailto:binom@Lbz.ru}{binom@Lbz.ru},\href{http://www.Lbz.ru}{http://www.Lbz.ru}

\begin{itemize}
  \item 
\end{itemize}

\begin{itemize}
  \item 
\end{itemize}

\begin{itemize}
  \item 
\end{itemize}

\begin{itemize}
  \item 
\end{itemize}

\begin{itemize}
  \item 
\end{itemize}

\begin{itemize}
  \item 
\end{itemize}

\begin{itemize}
  \item 
\end{itemize}

\begin{itemize}
  \item 
\end{itemize}

\begin{itemize}
  \item 
\end{itemize}

\begin{itemize}
  \item 
\end{itemize}

\begin{itemize}
  \item 
\end{itemize}

\begin{itemize}
  \item 
\end{itemize}

\begin{itemize}
  \item 
\end{itemize}

\begin{itemize}
  \item 
\end{itemize}

\begin{itemize}
  \item 
\end{itemize}

\begin{itemize}
  \item 
\end{itemize}

\begin{itemize}
  \item 
\end{itemize}

\begin{itemize}
  \item 
\end{itemize}

\begin{itemize}
  \item 
\end{itemize}


\end{document}