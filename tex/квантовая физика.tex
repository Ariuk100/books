% This LaTeX document needs to be compiled with XeLaTeX.
\documentclass[10pt]{article}
\usepackage[utf8]{inputenc}
\usepackage{ucharclasses}
\usepackage{graphicx}
\usepackage[export]{adjustbox}
\graphicspath{ {./images/} }
\usepackage{hyperref}
\hypersetup{colorlinks=true, linkcolor=blue, filecolor=magenta, urlcolor=cyan,}
\urlstyle{same}
\usepackage{amsmath}
\usepackage{amsfonts}
\usepackage{amssymb}
\usepackage[version=4]{mhchem}
\usepackage{stmaryrd}
\usepackage{caption}
\usepackage{multirow}
\usepackage[fallback]{xeCJK}
\usepackage{polyglossia}
\usepackage{fontspec}
\usepackage{newunicodechar}
\IfFontExistsTF{Noto Serif CJK JP}
{\setCJKmainfont{Noto Serif CJK JP}}
{\IfFontExistsTF{STSong}
  {\setCJKmainfont{STSong}}
  {\IfFontExistsTF{Droid Sans Fallback}
    {\setCJKmainfont{Droid Sans Fallback}}
    {\setCJKmainfont{SimSun}}
}}

\setmainlanguage{russian}
\setotherlanguages{belarusian, tamil}
\IfFontExistsTF{Noto Serif Tamil}
{\newfontfamily\tamilfont{Noto Serif Tamil}}
{\IfFontExistsTF{Tamil MN}
  {\newfontfamily\tamilfont{Tamil MN}}
  {\IfFontExistsTF{Lohit Tamil}
    {\newfontfamily\tamilfont{Lohit Tamil}}
    {\IfFontExistsTF{FreeSerif}
      {\newfontfamily\tamilfont{FreeSerif}}
      {\newfontfamily\tamilfont{Latha}}
}}}
\IfFontExistsTF{CMU Serif}
{\newfontfamily\lgcfont{CMU Serif}}
{\IfFontExistsTF{DejaVu Sans}
  {\newfontfamily\lgcfont{DejaVu Sans}}
  {\newfontfamily\lgcfont{Georgia}}
}
\setDefaultTransitions{\lgcfont}{}
\newfontfamily\cyrillicfont{CMU Serif}
\setDefaultTransitions{\cyrillicfont}{}
\setTransitionsFor{Tamil}{\tamilfont}{\lgcfont}

\title{КВАНТОВАЯ ФИЗИКА. ОСНОВНЫЕ ЗАКОНЫ \\
 Учебное пособие }

\author{Ведущие редакторы Б. И. Копылов, Т. Г. Хохлова\\
Художники Н. А. Лозинская, Н. А. Новак\\
Компьютерная верстка: В. А. Носенко\\
Подписано 11.02.14. Формат \(60 \times 90 / 16\).\\
Усл. печ. л. 16,00.\\
Издательство «БИНОМ. Лаборатория знаний»\\
125167, Москва, проезд Аэропорта, д. 3\\
Телефон: (499) 157-5272\\
e-mail: binom@Lbz.ru, http://www.Lbz.ru}
\date{}


%New command to display footnote whose markers will always be hidden
\let\svthefootnote\thefootnote
\newcommand\blfootnotetext[1]{%
  \let\thefootnote\relax\footnote{#1}%
  \addtocounter{footnote}{-1}%
  \let\thefootnote\svthefootnote%
}

%Overriding the \footnotetext command to hide the marker if its value is `0`
\let\svfootnotetext\footnotetext
\renewcommand\footnotetext[2][?]{%
  \if\relax#1\relax%
    \ifnum\value{footnote}=0\blfootnotetext{#2}\else\svfootnotetext{#2}\fi%
  \else%
    \if?#1\ifnum\value{footnote}=0\blfootnotetext{#2}\else\svfootnotetext{#2}\fi%
    \else\svfootnotetext[#1]{#2}\fi%
  \fi
}

\newunicodechar{⋅}{\ifmmode\cdot\else{$\cdot$}\fi}
\newunicodechar{×}{\ifmmode\times\else{$\times$}\fi}
\newunicodechar{ω}{\ifmmode\omega\else{$\omega$}\fi}
\newunicodechar{ε}{\ifmmode\varepsilon\else{$\varepsilon$}\fi}
\newunicodechar{𝜄}{\ifmmode\iota\else{$\iota$}\fi}
\newunicodechar{φ}{\ifmmode\varphi\else{$\varphi$}\fi}
\newunicodechar{Ψ}{\ifmmode\Psi\else{$\Psi$}\fi}
\newunicodechar{ψ}{\ifmmode\psi\else{$\psi$}\fi}
\newunicodechar{↓}{\ifmmode\downarrow\else{$\downarrow$}\fi}
\newunicodechar{↑}{\ifmmode\uparrow\else{$\uparrow$}\fi}
\newunicodechar{σ}{\ifmmode\sigma\else{$\sigma$}\fi}
\newunicodechar{𝜇}{\ifmmode\mu\else{$\mu$}\fi}
\newunicodechar{π}{\ifmmode\pi\else{$\pi$}\fi}
\newunicodechar{α}{\ifmmode\alpha\else{$\alpha$}\fi}
\newunicodechar{𝜃}{\ifmmode\theta\else{$\theta$}\fi}
\newunicodechar{υ}{\ifmmode\upsilon\else{$\upsilon$}\fi}
\newunicodechar{𝜂}{\ifmmode\eta\else{$\eta$}\fi}

\begin{document}
\maketitle
\captionsetup{singlelinecheck=false}
\section*{KBAHTOBAI Физика}
\section*{OCHOBHIS 3aKOHW}
\begin{center}
\includegraphics[max width=\textwidth, alt={}]{a57b3a57-78b5-4292-b0ab-4bd9f2ffcf56-001_697_767_1032_335}
\end{center}

\section*{И.Е. Иродов}
\section*{КВАНТОВАЯ ФИЗИКА}
\section*{основные законы}
6-е издание (электронное)

И83 Квантовая физика. Основные законы [Электронный ресурс] : учебное пособие / И. Е. Иродов. - 6-е изд. (эл.). М. : БИНОМ. Лаборатория знаний, 2014. - 256 с. : ил.

ISBN 978-5-9963-2302-9\\
Учебное пособие содержит теоретический и экспериментальный материал, относящийся к основным идеям квантовой физики, а также разбор многочисленных примеров и задач, где показано, как следует подходить к их решению. Задачи тесно связаны с основным текстом и часто являются его развитием и дополнением. Материал книги, насколько возможно, освобожден от излишней математизации - основной акцент перенесен на физическую сторону рассматриваемых явлений.

Для студентов физических и инженерно-технических специальностей вузов.

УДК 530.145(075)\\
ББК 22.31.я7

По вопросам приобретения обращаться:\\
«БИНОМ. Лаборатория знаний»\\
Телефон: (499) 157-5272\\
e-mail: \href{mailto:binom@Lbz.ru}{binom@Lbz.ru}, \href{http://www.Lbz.ru}{http://www.Lbz.ru}

\section*{Содержание}
Предисловие ..... 5\\
Введение ..... 7\\
Глава 1. Квантовые свойства электромагнитного излучения ..... 9\\
§ 1.1. Проблема теплового излучения ..... 9\\
§ 1.2. Фотоэффект ..... 12\\
§ 1.3. Тормозное рентгеновское излучение ..... 19\\
§ 1.4. Опыт Боте. Фотоны ..... 21\\
§ 1.5. Эффект Комптона ..... 24\\
Задачи ..... 29\\
Глава 2. Атом Резерфорда-Бора ..... 36\\
§ 2.1. Ядерная модель атома ..... 36\\
§ 2.2. Спектральные закономерности ..... 42\\
§ 2.3. Постулаты Бора. Опыты Франка и Герца ..... 44\\
§ 2.4. Боровская модель атома водорода ..... 47\\
Задачи ..... 53\\
Глава 3. Волновые свойства частиц ..... 60\\
§ 3.1. Гипотеза де-Бройля ..... 60\\
§ 3.2. Экспериментальные подтверждения гипотезы де- Бройля ..... 63\\
§ 3.3. Парадоксальное поведение микрочастиц ..... 69\\
§ 3.4. Принцип неопределенности ..... 73\\
Задачи ..... 79\\
Глава 4. Уравнение Шредингера. Квантование ..... 85\\
§ 4.1. Состояние частицы в квантовой теории ..... 85\\
§ 4.2. Уравнение Шредингера ..... 87\\
§ 4.3. Частица в прямоугольной яме ..... 90\\
§ 4.4. Квантовый гармонический осциллятор ..... 96\\
§ 4.5. Потенциальные барьеры ..... 100\\
Задачи ..... 103\\
Глава 5. Основы квантовой теории ..... 111\\
§ 5.1. Операторы физических величин ..... 111\\
§ 5.2. Основные постулаты квантовой теории ..... 113\\
§ 5.3. Квантование момента импульса ..... 118\\
§ 5.4. Ротатор ..... 123\\
Задачи ..... 125\\
Глава 6. Квантование атомов ..... 131\\
§ 6.1. Квантование атома водорода ..... 131\\
§ 6.2. Уровни и спектры щелочных металлов ..... 137\\
§ 6.3. Спин электрона ..... 141\\
§ 6.4. Механический момент многоэлектронного атома ..... 147\\
§ 6.5. Принцип Паули. Заполнение электронных обо- лочек ..... 150\\
§ 6.6. О периодической системе элементов Д. И. Менде- леева ..... 152\\
§ 6.7. Характеристические рентгеновские спектры ..... 156\\
Задачи ..... 161\\
Глава 7. Магнитные свойства атома ..... 167\\
§ 7.1. Магнитный момент атома ..... 167\\
§ 7.2. Эффекты Зеемана и Пашена-Бака ..... 171\\
§ 7.3. Электронный парамагнитный резонанс (ЭПР) ..... 176\\
Задачи ..... 178\\
Глава 8. Атомное ядро ..... 183\\
§ 8.1. Состав и характеристика атомного ядра ..... 183\\
§ 8.2. Масса и энергия связи ядра ..... 186\\
§ 8.3. Ядерные силы ..... 191\\
§ 8.4. Радиоактивность ..... 194\\
§ 8.5. Основные типы радиоактивности ..... 197\\
§ 8.6. Эффект Мессбауэра ..... 203\\
§ 8.7. Ядерные реакции ..... 207\\
Задачи ..... 216\\
Глава 9. Элементарные частицы ..... 223\\
§ 9.1. Введение ..... 223\\
§ 9.2. Систематика элементарных частиц ..... 225\\
§ 9.3. Античастицы ..... 227\\
§ 9.4. Законы сохранения ..... 229\\
§ 9.5. Четность ..... 233\\
§ 9.6. Изотопический спин ..... 235\\
§ 9.7. Кварковая модель адронов ..... 237\\
Задачи ..... 240\\
Приложения ..... 245

\begin{enumerate}
  \item Основные соотношения релятивистской динамики ..... 245
  \item Вывод формулы (2.1) ..... 246
  \item Соотношения между единицами некоторых величин ..... 247
  \item Формулы некоторых величин в гауссовой системе и в СИ ..... 248
  \item Массы легких нуклидов ..... 249
  \item Греческий алфавит ..... 249
  \item Некоторые физические константы ..... 250\\
Предметный указатель ..... 252
\end{enumerate}

\section*{Предисловие}
Основной замысел данной книги - органически совместить в одном учебном пособии изложение принципов теории и эксперимента с практикой решения задач. С этой целью в каждой главе сначала излагается теория соответствующего вопроса (с иллюстрацией на конкретных примерах), приводятся результаты наблюдений и эксперимента, а затем дается разбор ряда задач, где показывается, как, по мнению автора, следует подходить к их решению. Задачи тесно связаны с основным текстом, часто являются его развитием и дополнением, поэтому работа над ними должна проводиться параллельно с изучением основного материала.

При изложении теоретического материала автор стремился исключить из текста все второстепенное, с тем чтобы сконцентрировать внимание читателя на основных законах квантовой физики и, в частности, на вопросах наиболее трудных для понимания и восприятия. Стремление изложить основные идеи кратко, доступно и вместе с тем корректно побудило автора насколько возможно освободить материал от излишней математизации и формализма.

Изложение ведется в гауссовой системе (СГС). Это обусловлено главным образом тем, что в СИ многие формулы изучаемого круга явлений оказываются «загроможденными» коэффициентами и теряют свою простоту и наглядность. В Приложении дана сводка некоторых формул как в гауссовой системе, так и в СИ, а также приведены соотношения между единицами ряда величин в этих системах.

Курсивом выделены важнейшие положения и термины. Петит используется для материала повышенной трудности и относительно громоздких расчетов (этот материал при первом чтении можно безболезненно опустить), а также для примеров и задач.

Во \(2-м\) издании сделаны некоторые добавления и исправлены замеченные опечатки.

Книга как учебное пособие рассчитана на студентов физических и инженерно-технических специальностей.

Вопросы, связанные с квантовыми статистиками и их применениями (электронный, фотонный и фононный газы) рассматриваются в моей книге «Физика макросистем» - в соответствии с концепцией новой программы по физике Министерства образования РФ.\\
И. Иродов

\section*{Принятые обозначения}
Векторы обозначены жирным прямым шрифтом (например, \(\mathbf{v}, \mathbf{B}\) ). Та же буква светлым шрифтом и курсивом ( \(v, B\) ) означает модуль соответствующего вектора.

Средние величины отмечены угловыми скобками <>, например, \(\langle p\rangle,\langle K\rangle\).

Энергия частицы обозначена как\\
\(E\) - полная, \(K\) - кинетическая, \(U\) - потенциальная.\\
Системы отсчета:\\
Л-система - лабораторная система отсчета (она предполагается инерциальной),\\
Ц-система - система центра масс (или центра инерции) - система отсчета, движущаяся поступательно относительно инерциальной системы. Все величины в Ц-системе отмечены сверху значком \(\sim\) (тильда), например, \(\tilde{\mathbf{p}}, \tilde{E}\).\\
ɔ - знак пропорциональности;\\
\(\sim\) - величина порядка... ( \(r \sim 10^{-13}\) см).\\
Интегралы любой кратности обозначены одним-единственным знаком \(\int\) и различаются лишь обозначением элемента интегрирования: \(\mathrm{d} V\) - элемент объема, \(\mathrm{d} S\) - элемент поверхности.

Операторы физических величин обозначены латинскими буквами со «шляпками» (\^{}), например, \(\hat{p}, \hat{M}\).

\section*{Обозначения и названия единиц}
А - ампер\\
E - ангстрем\\
атм - атмосфера\\
б - барн\\
В - вольт\\
Вт- ватт\\
Гс - гаусс\\
Гц - герц

Дж - джоуль\\
дин - дина\\
K - кельвин\\
кг - килограмм\\
Кл - кулон\\
л - литр\\
м - метр\\
мин - минута

Ом - ом\\
Па - паскаль\\
H - ньютон\\
c - секунда\\
ср - стерадиан\\
Тл - тесла\\
ч - час\\
эВ - электроновольт

\section*{Десятичные приставки к названиям единиц}
Э - экса, \(10^{18}\)\\
\(\Pi\) - пета, \(10^{15}\)\\
T - тера, \(10^{12}\)\\
\(Г\) - гига, \(10^{9}\)

М - мега, \(10^{6}\)\\
к - кило, \(10^{3}\)\\
м - милли, \(10^{-3}\)\\
мк - микро, \(10^{-6}\)

н - нано, \(10^{-9}\)\\
п - пико, \(10^{-12}\)\\
ф - фемто, \(10^{-15}\)\\
a - атто, \(10^{-18}\)

\section*{Введение}
По мере развития физики как науки выяснилось, что понятия и принципы, возникшие на основе изучения макроскопических объектов, неприменимы или ограниченно применимы в области атомных масштабов. Здесь потребовались новые представления и законы, которые в конце концов и были найдены. Они составили основу новой так называемой квантовой физики.

Более того, теперь мы понимаем, что, строго говоря, существует только единая физика, квантовая по своей сущности. И говоря о классической физике, имеют в виду ту часть единой физики, в которой роль квантовых закономерностей пренебрежимо мала.

Парадоксальное, непредсказуемое поведение микрочастиц, с чем мы познакомимся в этой книге, в принципе не имеет аналога в классической физике. Понять, например, что микрочастица - это одновременно корпускула и волна, выходит за рамки наших представлений. Противоречие между корпускулярной и волновой формами материи на уровне мышления выступает как противоречие между дискретным и непрерывным. В физической реальности это противоречие снимается созданием квантовой теории, обобщающей опытные факты.

В этой теории под частицей подразумевается квантовый объект, у которого оба эти качества - дуализм волна-частица - находятся в единстве. При этом обнаруживается, что в определенных условиях квантовый объект ведет себя или как обычная частица или как волна. Но это крайние случаи. Говоря же, например, об электроне в атоме, мы под словом «электрон» понимаем квантовый объект. Ничего более детального о поведении микрочастицы квантовая теория не дает. Но это не надо понимать как неполноту, ограниченность квантовой теории. Просто такова сущность природы материи на микроуровне.

Мы увидим, что квантовые законы, обобщающие опытные факты, и соответствующие квантовые модели лишены привычного нам качества - наглядности. В этом главная трудность их понимания. Из-за нее, начиная знакомиться с квантовой теорией, обычно приходится преодолевать ощущение противоречивости ее законов здравому смыслу. Если же при этом не забывать, что квантовая теория опирается на опыт, такое ощущение постепенно сглаживается и адаптация к новым понятиям заметно облегчается.

И последнее. Как ни велики успехи квантовой теории, мы все же должны признать, что у нас нет детальной теории, которая могла бы объяснить все явления в нашем мире. Кроме того, необходимо иметь в виду, что законы классической физики остались непоколебимыми, если ограничить их область применения явлениями, для которых классическая физика была создана.

\section*{Квантовые свойства электромагнитного излучения}
\section*{§ 1.1. Проблема теплового излучения}
Тепловое излучение. Электромагнитное излучение, испускаемое источником, уносит с собой энергию. В зависимости от природы источника различают и виды излучения. Не будем их перечислять, поскольку нас интересует только одно излучение - тепловое, обусловленное нагреванием, т. е. подводом теплоты. Это излучение занимает особое место среди всех других видов излучения. В отличие от них тепловое излучение это единственный вид излучения, которое может находиться в термодинамическом равновесии с телами.

Чтобы составить себе представление о характере теплового излучения, рассмотрим несколько тел, нагретых до различной температуры и помещенных в замкнутую полость, стенки которой полностью отражают падающее на них излучение. Опыт показывает, что такая система в конечном счете приходит в состояние теплового равновесия, при котором температура всех тел становится одинаковой. Так происходит и в том случае, когда между телами в полости будет вакуум, и тела могут обмениваться энергией только путем испускания и поглощения электромагнитных волн. За любой промежуток времени испускаемая телами энергия становится равной поглощаемой энергии, и плотность энергии излучения в пространстве между телами достигает определенной величины, соответствующей установившейся температуре. Такое состояние излучения в полости остается неизменным во времени. Оно находится, как уже было сказано, в термодинамическом равновесии с телами, имеющими определенную температуру, и поэтому его называют равновесным или черным излучением.

Оказывается, плотность энергии равновесного излучения и его спектральный состав совершенно не зависят от размеров и формы полости и от свойств находящихся в ней тел. Характер равновесного излучения зависит только от температуры. Поэтому можно говорить о температуре самого излучения, считая ее

равной температуре тел, с которыми оно находится в тепловом равновесии. Равновесное излучение однородно, изотропно и неполяризовано.

Для экспериментального изучения спектрального состава равновесного излучения проделывают небольшое отверстие в стенке полости, поддерживаемой при определенной температуре. Выходящее наружу через отверстие излучение обладает таким же спектральным составом, что и внутри полости.

Распределение энергии по длинам волн \(\lambda\) или по частотам \(\omega\) характеризуют спектральной плотностью излучения \(u_{\lambda}\) или \(u_{\omega}\), так что величина \(u_{\lambda} \mathrm{d} \lambda\) дает энергию единицы объема излучения с длинами волн в интервале ( \(\lambda, \lambda+\mathrm{d} \lambda\) ), а \(u_{\omega} \mathrm{d} \omega-\mathrm{c}\) частотами в интервале ( \(\omega, \omega+\mathrm{d} \omega\) ).

В случае равновесного излучения спектральная плотность \(u_{\omega}\) (или \(u_{\lambda}\) ) представляет собой универсальную функцию только частоты (или длины волн) и температуры \(T\). Основная проблема теории теплового излучения и заключалась в нахождении этой функции.

Все попытки решить данную проблему с помощью классических представлений потерпели неудачу. Задача о равновесии излучения с простейшим примером излучающего тела - линейным гармоническим осциллятором приводила к абсурдному результату. Проблема теплового излучения зашла в тупик...

Открытие постоянной Планка. Это произошло в 1900 г. Планк получил формулу для спектральной плотности \(u_{\omega}(T)\) теплового излучения, хорошо согласующуюся с экспериментальными данными. Однако для этого ему пришлось ввести гипотезу, коренным образом противоречащую представлениям классической физики. Планк предположил, что энергия осциллятора может принимать не любые, а только вполне определенные дискретные значения \(\varepsilon_{n}\), пропорциональные некоторой элементарной порции - кванту энергии \(\varepsilon_{0}\). В связи с этим испускание и поглощение электромагнитного излучения осциллятором (веществом) осуществляется не непрерывно, а дискретно в виде отдельных квантов, величина которых пропорциональна частоте излучения:


\begin{equation*}
\varepsilon_{0}=\hbar \omega, \tag{1.1}
\end{equation*}


где коэффициент \(\hbar\) получил впоследствии название постоянной Планка*. Определенное из опыта значение \(\hbar\) равно

\[
\hbar=1,054 \cdot 10^{-27} \text { эрг ⋅ c }=0,659 \cdot 10^{-15} \text { эВ ⋅ c. }
\]

В физике есть величина, имеющая размерность «энергия × × время». Ее называют действием. Постоянная Планка имеет ту же размерность, поэтому ее иногда называют квантом действия. Заметим также, что размерность \(\hbar\) совпадает с размерностью момента импульса. Это совпадение, как мы увидим далее, не случайное.

Постоянная Планка была определена экспериментально не только с помощью законов теплового излучения, но и другими, более прямыми и точными методами. Значения \(\hbar\), полученные на основе разных физических явлений (тепловое излучение, фотоэффект, коротковолновая граница сплошного рентгеновского спектра и др.), хорошо согласуются друг с другом.

Постоянная Планка - это важнейшая универсальная константа, играющая в квантовой физике такую же фундаментальную роль, как скорость света в теории относительности. Открытие постоянной Планка и связанной с ней идеи квантования ознаменовало рождение новой, квантовой теории. Физику, как науку, стали подразделять на классическую (нерелятивистскую и релятивистскую) и квантовую, неразрывно связанную с фундаментальной константой \(\hbar\).

Итак, Планк доказал, что формулу для спектральной плотности энергии теплового излучения можно получить только в том случае, если допустить квантование энергии, противоречащее классическим представлениям.

Трудно было примириться с таким отказом от классических представлений, и Планк, совершив великое открытие, еще в течение нескольких лет пытался понять квантование энергии с позиций классической физики. Безуспешность этих попыток привела его к окончательному выводу, что в рамках классической теории природу теплового излучения понять невозможно.

\footnotetext{\begin{itemize}
  \item Собственно говоря, постоянной Планка называют коэффициент пропорциональности между \(\varepsilon_{0}\) и линейной частотой \(v, \varepsilon_{0}=h v\). Постоянная \(\hbar\) ( \(\hbar\) перечеркнутая) это постоянная Планка \(h\), деленная на \(2 \pi\). Числовое значение \(h\) равно \(h=6,62 \cdot 10^{-27}\) эрг ⋅ c \(=4,21 \cdot 10^{-15}\) эВ ⋅ c.
\end{itemize}
}\section*{§ 1.2. Фотоэффект}
Световые кванты. Квантовая гипотеза Планка была оценена по достоинству и получила дальнейшее развитие прежде всего в работах Эйнштейна. Он первый указал на то, что кроме теплового излучения существуют и другие явления, которые можно объяснить на основе квантовой гипотезы.

В 1905 г. Эйнштейн выдвинул гипотезу световых квантов. Он предположил, что дискретный характер присущ не только процессам испускания и поглощения света, но и самому свету. Гипотеза о корпускулярных свойствах света позволила объяснить результаты экспериментов по фотоэффекту, совершенно непонятные с позиций классической электромагнитной теории. Рассмотрим этот вопрос более подробно.

Фотоэлектрическим эффектом, или фотоэффектом называют испускание электронов вещест-

\begin{figure}[h]
\begin{center}
  \includegraphics[alt={},max width=\textwidth]{a57b3a57-78b5-4292-b0ab-4bd9f2ffcf56-013_358_402_843_68}
\captionsetup{labelformat=empty}
\caption{Рис. 1.1}
\end{center}
\end{figure}

вом под действием света. Исследование закономерностей фотоэффекта проводят на установке, схематически показанной на рис. 1.1. При освещении катода \(K\) монохроматическим светом через кварцевое окошко (пропускающее и ультрафиолетовые лучи) из катода вырываются фотоэлектроны, и в цепи возникает фототок, регистрируемый гальванометром \(G\). График зависимости фототока \(I\) от приложенного внешнего напряжения \(V\) между катодом и анодом \(A\) представлен на рис. 1.2. Этот график называют характеристикой фотоэлемента, т. е. того прибора, в котором наблюдают фото-

\begin{figure}[h]
\begin{center}
  \includegraphics[alt={},max width=\textwidth]{a57b3a57-78b5-4292-b0ab-4bd9f2ffcf56-013_277_350_1488_68}
\captionsetup{labelformat=empty}
\caption{Рис. 1.2}
\end{center}
\end{figure}

Многочисленными экспериментами были установлены три основные закономерности фотоэффекта:

\begin{enumerate}
  \item Фототок насыщения пропорционален падающему световому потоку (при одном и том же спектральном составе). Это значит, что число электронов, вырываемых светом ежесекундно, пропорционально мощности падающего света. Впервые это было установлено А. Г. Столетовым (1889).
  \item Для каждого металла существует максимальная длина волны света \(\lambda_{\mathrm{K}}\) (или минимальная частота \(\omega_{\mathrm{K}}\) ), при которой еще происходит вырывание электронов. Если длина волны превышает \(\lambda_{\mathrm{K}}\) - так называемую красную границу фотоэффекта, то испускание фотоэлектронов отсутствует даже при достаточно большой интенсивности падающего света*.
  \item Максимальная кинетическая энергия \(K\) фотоэлектронов линейно зависит от частоты \(\omega\) облучающего света (причем \(K_{\text {макс }}\) растет с увеличением \(\omega\) ) и не зависит от интенсивности света. Заметим, что максимальное значение кинетической энергии фотоэлектронов определяют по так называемой задерживающей разности потенциалов (этот вопрос рассмотрен ниже).
\end{enumerate}

С точки зрения классических волновых представлений сам факт вырывания электронов из металла неудивителен, так как падающая электромагнитная волна вызывает вынужденные колебания электронов в металле. Электрон, поглощая энергию, может накопить ее в количестве, достаточном для преодоления потенциального барьера, удерживающего электрон в металле, т. е. для совершения работы выхода. Если это так, то энергия фотоэлектронов должна зависеть от интенсивности света. Увеличение же интенсивности света приводит лишь к возрастанию числа фотоэлектронов.

Более того, резкое расхождение теории с опытом возникает при очень малой интенсивности света. По классической волновой теории фотоэффект в этих условиях должен протекать с заметным запаздыванием, поскольку требуется конечное время для накопления необходимой энергии. Однако опыт показывает, что фотоэффект появляется практически мгновенно, т. е. одно-

\footnotetext{\begin{itemize}
  \item При очень больших интенсивностях, например сфокусированное лазерное излучение, красная граница фотоэффекта исчезает.
\end{itemize}
}временно с началом освещения (промежуток времени между началом освещения и появлением фототока не превышает \(10^{-9} \mathrm{c}\) ).

Все трудности отпадают, если фотоэффект рассматривать на основе гипотезы Эйнштейна о световых квантах. В соответствии с этой гипотезой падающее монохроматическое излучение рассматривается как поток световых квантов - фотонов, энергия \(\varepsilon\) которых связана с частотой \(\omega\) соотношением


\begin{equation*}
\varepsilon=\hbar \omega . \tag{1.2}
\end{equation*}


При поглощении фотона его энергия целиком передается одному электрону. Таким образом, электрон приобретает кинетическую энергию не постепенно, а мгновенно. Этим и объясняется безынерционность фотоэффекта.

Формула Эйнштейна. Полученная электроном энергия ћω частично затрачивается на освобождение из металла. А остальная часть переходит в кинетическую энергию вылетевшего из металла фотоэлектрона. Минимальную энергию, необходимую для освобождения электрона из металла, т. е. для преодоления потенциального барьера, называют работой выхода \(A\). Следовательно, для фотоэлектронов с максимальной кинетической энергией \(K_{\text {макс }}\) закон сохранения энергии в элементарном акте поглощения фотона можно записать так:


\begin{equation*}
\hbar \omega=A+K_{\text {макс }} . \tag{1.3}
\end{equation*}


Эта формула впервые была получена Эйнштейном и носит его имя - формула Эйнштейна.

Пример. Отдаленный от других тел металлический шарик, работа выхода электрона с поверхности которого равна \(A\), освещают электромагнитным излучением с длиной волны \(\lambda\). Найдем, до какого минимального потенциала \(\varphi_{\text {мин }}\) зарядится шарик, испуская фотоэлектроны.

По мере испускания фотоэлектронов шарик будет заряжаться, т. е. приобретать положительный потенциал \(\varphi\), играющий роль задерживающего потенциала. Когда глубина потенциальной ямы, из которой должен «выбраться» фотоэлектрон, окажется равной его максимальной кинетической энергии, наступает равновесие, и мы можем в соответствии с (1.3) записать:

\[
e \varphi_{\text {макс }}=K_{\text {макс }}=\hbar \omega-A .
\]

Отсюда следует с учетом того, что \(\omega=2 \pi c / \lambda\),

\[
\varphi_{\text {макс }}=(2 \pi \hbar \mathrm{c} / \lambda-A) / e .
\]

Из последней формулы видно, что соотношение между \(\lambda\) и \(A\) должно быть таким, чтобы величина, стоящая в скобках, была положительной.

Вернемся к формуле Эйнштейна (1.3). Из нее автоматически вытекают следующие закономерности, находящиеся в строгом согласии с опытом.

\begin{enumerate}
  \item Максимальная кинетическая энергия фотоэлектронов линейно зависит от частоты падающего света и не зависит от его интенсивности. Интенсивность обусловливает только количество фотоэлектронов, но совершенно не влияет на их максимальную кинетическую энергию. Кстати отметим, что наклон прямой на графике \(K_{\text {макс }}(\omega)\), как видно из формулы (1.3), т. е. \(\mathrm{d} K_{\text {макс }} / \mathrm{d} \omega=\hbar\). На этом основан один из методов определения постоянной Планка.
  \item Существует низкочастотная граница - порог фотоэффек\(m a\), т. е. такая частота \(\omega_{0}\), ниже которой фотоэффект отсутствует. Эта частота согласно (1.3) соответствует равенству \(\hbar \omega_{0}=A . E c\) ли \(\omega<\omega_{0}\), то энергии фотона не хватает, чтобы электрон мог преодолеть потенциальный барьер «высотой» \(A\) и выбраться из металла. На этом основан один из методов определения работы выхода*.
\end{enumerate}

Частоте \(\omega_{0}\) соответствует красная граница фотоэффекта, длина волны которой \(\lambda_{\mathrm{K}}=2 \pi c / \omega_{0}\). Наличие такой границы совершенно непонятно с волновой точки зрения. Значения \(\lambda_{\mathrm{K}}\) для некоторых металлов приведены в табл. 1.1.

\begin{table}[h]
\begin{center}
\captionsetup{labelformat=empty}
\caption{Таблица 1.1}
\begin{tabular}{||c|c|c|c|c|c||}
\hline
Металл & Cs & Na & Zn & Ag & Pt \\
\hline
\(\lambda_{\mathrm{K}}\), мкм & 0,60 & 0,53 & 0,33 & 0,28 & 0,20 \\
\hline
\end{tabular}
\end{center}
\end{table}

\footnotetext{\begin{itemize}
  \item Работа выхода может быть определена экспериментально независимо от фотоэффекта, например, с помощью исследования термоэлектронной эмиссии. Эта работа зависит от ряда факторов и имеет порядок нескольких эВ.
\end{itemize}
}В справочной литературе наблюдается довольно большой разброс в значениях \(\lambda_{\mathrm{K}}\) для одних и тех же металлов. Поэтому к значениям \(\lambda_{\mathrm{K}}\) в табл. 1.1 следует относиться с определенной осторожностью.

Трудности эксперимента. Необходимо заметить, что получение точных результатов сильно затрудняют два обстоятельства:

\begin{enumerate}
  \item экспериментальная кривая \(I(V)\) в области \(V_{1}\) (см. рис. 1.2) подходит к оси \(V\) практически асимптотически, вследствие чего определение \(V_{1}\) довольно неопределенно;
  \item всю кривую \(I(V)\) смещает (влево или вправо) наличие так называемой контактной разности потенциалов, т. е. разности потенциалов, которая возникает между двумя различными металлами (а это приходится, как правило, делать, поскольку катод \(K\) и анод \(A\) изготовляют по необходимости из различных металлов). Причем известно, что контактная разность потенциалов между катодом и анодом не зависит от природы проводников, их соединяющих.\\
Неизбежное присутствие контактной разности потенциалов и трудность ее учета, а также ряд других экспериментальных затруднений и источников ошибок - все это привело к тому, что достаточно точное подтверждение уравнения Эйнштейна (1.3) было получено не сразу.
\end{enumerate}

Это уравнение было подтверждено в тщательных опытах Милликена (1916) и последующих исследователей, создавших установку, в которой катод \(K\) имел форму небольшого шарика, помещенного в центр сферической

\begin{figure}[h]
\begin{center}
  \includegraphics[alt={},max width=\textwidth]{a57b3a57-78b5-4292-b0ab-4bd9f2ffcf56-017_393_475_1296_68}
\captionsetup{labelformat=empty}
\caption{Рис. 1.3}
\end{center}
\end{figure}

обкладки - анода \(A\) (рис. 1.3). При такой конфигурации практически все электроны, вырванные светом из катода, попадают на анод и в отсутствие ускоряющей разности потенциалов. Кроме того, характеристика такого фотоэлемента \(I(V)\) спадает к нулю достаточно круто, и значение \(V_{1}\) (см. рис. 1.2) может быть определено с хорошей точностью.

Задерживающая разность потенциалов. Именно эта величина позволяет задержать фотоэлектроны, вылетающие из катода с максимальной кинетической энергией \(K_{\text {макс }}\), что и приводит к прекращению фототока. Если бы катод и анод фотоэлемента были изготовлены из одного и того же металла, то контактная разность потенциалов отсутствовала бы, и определение задерживающей разности потенциалов сводилось бы просто к измерению внешнего задерживающего напряжения, т. е. показаниям вольтметра \(V_{3}<0\) (рис. 1.4). Действительно, при \(V=0\) все фотоэлектроны вне зависимости от начальной скорости достигали бы анода,

\begin{figure}[h]
\begin{center}
  \includegraphics[alt={},max width=\textwidth]{a57b3a57-78b5-4292-b0ab-4bd9f2ffcf56-018_373_394_346_760}
\captionsetup{labelformat=empty}
\caption{Рис. 1.4}
\end{center}
\end{figure}

Определение задерживающей разности потенциалов усложняется, если катод и анод изготовлены из разных металлов (что обычно и бывает). В этом случае начинает играть заметную роль контактная разность потенциалов. Если она есть и, например, такова, что тормозит вылетающие из катода фотоэлектроны, то приходится прикладывать внешнее напряжение \(V\) (измеряемое вольтметром). И если это напряжение таково, что компенсирует тормозящую контактную разность потенциалов, то начало горизонтального участка (тока насыщения) - точка 2 на рис. 1.5 - сдвинется вправо, в сторону положительных значений показа-

\begin{figure}[h]
\begin{center}
  \includegraphics[alt={},max width=\textwidth]{a57b3a57-78b5-4292-b0ab-4bd9f2ffcf56-018_390_406_996_754}
\captionsetup{labelformat=empty}
\caption{Рис. 1.5}
\end{center}
\end{figure}

Таким образом, задерживающая разность потенциалов \(V_{3}\) будет равна (по модулю) сумме


\begin{equation*}
V_{3}=V_{2}+\left|V_{1}\right|=V_{2}-V_{1}, \tag{1.4}
\end{equation*}


как показано на рис. 1.5, где \(V_{1}<0\). Заметим, что, вообще говоря, \(V_{1}\) есть величина алгебраическая, она может иметь любой знак или равняться нулю.

Если контактная разность потенциалов не тормозит, а ускоряет фотоэлектроны, т. е. имеет противоположный знак, то характеристика фотоэлемента \(I(V)\) вместе с точкой 2 сместится влево. При этом выражение (1.4) для \(V_{3}\) остается, как легко убедиться, прежним, только в нем оба показания вольтметра ( \(V_{2}\) и \(V_{1}\) ) могут оказаться отрицательными, но их разность по-прежнему будет положительной и равной \(V_{3}\).

Итак, определив \(V_{3}\), мы тем самым находим максимальную кинетическую энергию фотоэлектронов \(-K_{\text {макс }}\) в формуле Эйнштейна (1.3):


\begin{equation*}
K_{\text {макс }}=e V_{\text {з }}=e\left(V_{2}-V_{1}\right) . \tag{1.5}
\end{equation*}


Отметим, что положение точки 2 на рис. 1.5, т. е. показание вольтметра \(V=V_{2}\), зависит только от контактной разности потенциалов, положение же точки 1 , т. е. показание \(V_{1}\) вольтметра - от частоты ω падающего света. Значит, и задерживающая разность потенциалов \(V_{3}\) тоже зависит от \(\omega\).

\begin{figure}[h]
\begin{center}
  \includegraphics[alt={},max width=\textwidth]{a57b3a57-78b5-4292-b0ab-4bd9f2ffcf56-019_414_396_895_66}
\captionsetup{labelformat=empty}
\caption{Рис. 1.6}
\end{center}
\end{figure}

Если построить экспериментальный график зависимости \(K_{\text {макс }}(\omega)\), то получается прямая (рис. 1.6), что является убедительным подтверждением формулы Эйнштейна (1.3).

Заметим, что точка пересечения прямой с осью абсцисс определяет частоту \(\omega_{0}\), соответствующую красной границе фотоэффекта, а точка пересечения продолжения прямой с осью ординат работу выхода \(A\). Если же на оси ординат откладывать \(V_{1}\) (показание вольтметра, при котором фототок обращается в нуль), то отмеченные две точки не будут соответствовать \(\omega_{0}\) и \(A\) (из-за наличия контактной разности потенциалов). К сожалению, это часто не учитывают, и полученные результаты сильно отличаются от действительных значений.

Пример. При последовательном освещении катода светом с частотой \(v=1,0 \cdot 10^{15}\) Гц и \(v^{\prime}=1,4 \cdot 10^{15}\) Гц показания вольтметра, при которых фототок прекращался, оказались \(V_{1}=-0,40 \mathrm{~B}\) и \(V_{1}^{\prime}=-2,0\) В (см. рис. 1.5). Найдем постоянную Планка.

Воспользовавшись уравнением Эйнштейна (I.3) и формулой (1.5), запишем:


\begin{align*}
& 2 \pi \hbar v^{\prime}=A+e\left(V_{2}-V_{1}^{\prime}\right),  \tag{1}\\
& 2 \pi \hbar v=A+e\left(V_{2}-V_{1}\right), \tag{2}
\end{align*}


где \(V_{1}{ }^{\prime}\) и \(V_{1}<0\). Чтобы избавиться от неизвестных \(A\) и \(V_{2}\), вычтем (2) из (1):

\[
2 \pi \hbar\left(v^{\prime}-v\right)=e\left(V_{1}-V_{1}^{\prime}\right) .
\]

Отсюда

\[
\hbar=\frac{e}{2 \pi} \frac{V_{1}-V_{1}^{\prime}}{v^{\prime}-v}=\frac{4,8 \cdot 10^{-10} \cdot 1,6 / 300}{2 \pi 0,4 \cdot 10^{15}}=1,0 \cdot 10^{-27} \text { эргЧс. }
\]

\section*{§ 1.3. Тормозное рентгеновское излучение}
Если энергия кванта \(\hbar \omega\) значительно превышает работу выхода \(A\), то уравнение Эйнштейна (1.3) принимает более простой вид:


\begin{equation*}
\hbar \omega=K_{\text {макс }} \cdot \tag{1.6}
\end{equation*}


Эту формулу можно интерпретировать и иначе: не как переход энергии светового кванта в кинетическую энергию электрона, а наоборот, как переход кинетической энергии электронов, ускоренных разностью потенциалов \(V\), в энергию квантов, возникающих при резком торможении электронов в металле. Тогда


\begin{equation*}
e V=\hbar \omega . \tag{1.7}
\end{equation*}


Именно такой процесс происходит в рентгеновской трубке. Она представляет собой вакуумный баллон, в котором находится нагреваемый током катод - источник термоэлектронов, и расположенный напротив анод, часто называемый антикатодом. Ускорение электронов осуществляется высоким напряжением \(V\), создаваемым между катодом и антикатодом.

Под действием напряжения \(V\) электроны разгоняются до энергии \(e V\). Попав в металлический антикатод, электроны резко тормозятся, вследствие чего и возникает так называемое тормозное рентгеновское излучение. Спектр этого излучения при разложении по длинам волн оказывается сплошным, как и

\begin{figure}[h]
\begin{center}
  \includegraphics[alt={},max width=\textwidth]{a57b3a57-78b5-4292-b0ab-4bd9f2ffcf56-021_364_445_145_68}
\captionsetup{labelformat=empty}
\caption{Рис. 1.7}
\end{center}
\end{figure}

спектр видимого белого света. На рис. 1.7 показаны экспериментальные кривые распределения интенсивности \(I_{\lambda}\) (т. е. \(\mathrm{d} I / \mathrm{d} \lambda\) ) по длинам волн \(\lambda\), полученные для разных значений ускоряющего напряжения \(V\) (они указаны на рисунке).

И здесь мы обнаруживаем наличие коротковолновой границь сплошного рентгеновского спектра. В целом процесс излучения при торможении электрона в металле антикатода весьма сложен, но существование коротковолновой границы с корпускулярной точки зрения имеет очень простое объяснение. Действительно, если излучение возникает за счет энергии, теряемой электроном при торможении, то величина кванта \(\hbar \omega\) не может быть больше энергии электрона \(e V\). Отсюда следует, что частота \(\omega\) излучения не может превышать значения \(\omega_{\text {макс }}=e V / \hbar\). Значит, длина волны излучения не может быть меньше, чем


\begin{equation*}
\lambda_{\text {мин }}=\frac{2 \pi \hbar c}{e V}=\frac{1,24}{V}, \tag{1.8}
\end{equation*}


где \(V\), кВ, а \(\lambda_{\text {мин }}\), нм.\\
Существование такой границы является одним из наиболее ярких проявлений квантовых свойств рентгеновского излучения. С позиции классической электромагнитной теории коротковолновой границы вообще не должно быть.

По измерению зависимости граничной частоты от ускоряющего напряжения можно с высокой точностью определить значение постоянной Планка. При этом получается хорошее согласие со значениями, найденными из теплового излучения и фотоэффекта, что экспериментально доказывает выполнение соотношения \(\varepsilon=\hbar \omega\) между энергией кванта и частотой для очень широкого диапазона спектра и указывает на универсальность данного соотношения.

Метод определения постоянной Планка, основанный на измерении коротковолновой границы тормозного рентгеновского излучения, является наиболее точным. Его называют методом изохромат. Этот метод заключается в том, что спектрометр для

рентгеновского излучения устанавливают так, чтобы в счетчик попадало излучение одной и той же определенной длины волны, и измеряют интенсивность \(I_{\lambda}\) в зависимости от приложенного к рентгеновской трубке напряжения \(V\). Уменьшая напряжение \(V\), получают зависимость интенсивности \(I_{\lambda}\) от напряжения \(V\). Эта зависимость для трех длин волн по-

\begin{figure}[h]
\begin{center}
  \includegraphics[alt={},max width=\textwidth]{a57b3a57-78b5-4292-b0ab-4bd9f2ffcf56-022_428_483_147_677}
\captionsetup{labelformat=empty}
\caption{Рис. 1.8}
\end{center}
\end{figure}


\begin{equation*}
\hbar=\frac{e \lambda}{2 \pi c} V_{0}, \tag{1.9}
\end{equation*}


где \(e\) - заряд электрона.

\section*{§ 1.4. Опыт Боте. Фотоны}
Опыт Боте наиболее непосредственно подтверждает гипотезу Эйнштейна о световых квантах. В этом опыте тонкую металлическую фольгу \(F\) устанавливали между двумя быстродействующими счетчиками \(G_{1}\) и \(G_{2}\) (рис. 1.9). Фольгу облучали слабьлм пучком рентгеновского излучения \(X\), под действием которого она сама становилась источником рентгеновского излучения. Вследствие весьма слабой интенсивности первичного пучка количе-

\begin{figure}[h]
\begin{center}
  \includegraphics[alt={},max width=\textwidth]{a57b3a57-78b5-4292-b0ab-4bd9f2ffcf56-022_327_422_1075_742}
\captionsetup{labelformat=empty}
\caption{Рис. 1.9}
\end{center}
\end{figure}

ство квантов, испускаемых фольгой, было достаточно мало.

Если бы энергия этого излучения распространялась в виде сферических волн, то оба счетчика должны были бы срабатывать одновременно. Опыт, однако, показал, что счетчики реагировали совершенно независимо друг от друга, и число совпадений не превышало ожидаемого числа случайных совпадений. Все происходило так, как если бы излучение фольги \(F\) распро-

странялось в виде отдельных квантов, которые могли попадать либо в один, либо в другой счетчик.

Это можно объяснить лишь тем, что в отдельных актах испускания возникают кванты излучения, т. е. частицы, летящие то в одном, то в другом направлении. Конечно, была принята предосторожность от того, что в результате первичного облучения фольга испускала и электроны. Для исключения этого окна счетчиков имели такую толщину, чтобы они были способны поглотить эти электроны и исключить их влияние на результаты опыта.

Итак, экспериментально было доказано существование особых электромагнитных квантов, или фотонов, как их впоследствии назвали.

Фотоны. Рассмотренные выше опыты и ряд других* со всей убедительностью подтвердили гипотезу Эйнштейна о световых квантах - фотонах.

Свет частоты \(\omega\) по Эйнштейну - это по существу поток фотонов с энергией \(\varepsilon=\hbar \omega\). Свет распространяется в вакууме со скоростью \(c\). Значит с такой же скоростью распространяются и фотоны. Согласно теории относительности полная энергия \(E\) любой частицы, движущейся со скоростью \(v\), определяется как


\begin{equation*}
E=m c^{2} / \sqrt{1-(v / c)^{2}} . \tag{1.10}
\end{equation*}


В случае фотона \(v=c\), и знаменатель этого выражения обращается в нуль. Для фотона, имеющего конечную энергию, это возможно лишь при условии \(m=0\).

Таким образом, мы имеем дело с частицей, масса покоя которой равна нулю.

Воспользовавшись связью между энергией \(E\) и импульсом \(p\) движущейся частицы, т. е.


\begin{equation*}
E^{2}-p^{2} c^{2}=m^{2} c^{4} \tag{1.11}
\end{equation*}


приходим к выводу, что фотон ( \(m=0\) ) обладает не только энергией \(E=\hbar \omega\), но и импульсом


\begin{equation*}
p=\hbar \omega / c . \tag{1.12}
\end{equation*}


\footnotetext{\begin{itemize}
  \item Например, опыт А. Ф. Иоффе и Н. И. Добронравова, аналогичный по идее опыту Боте, а также опыт С. И. Вавилова по обнаружению флуктуаций слабых потоков видимого света.
\end{itemize}
}Отношение \(\omega / c=2 \pi \nu / c=2 \pi / \lambda=k\), где \(k\) - волновое число, и тогда (1.12) примет вид \(p=\hbar k\).

Таким образом, фотон как частица обладает энергией и импульсом. Записав импульс в векторной форме, получим окончательно для энергии и импульса фотона следующие выражения:


\begin{equation*}
\varepsilon=\hbar \omega, \quad \mathbf{p}=\hbar \mathbf{k}, \tag{1.13}
\end{equation*}


где \(\mathbf{k}\) - волновой вектор, модуль которого \(k=2 \pi / \lambda\).\\
Частота \(\omega\) и волновой вектор k характеризуют волновые свойства монохроматического света, а энергия \(\varepsilon\) и импульс p корпускулярные.

Следует обратить внимание на то, что объект, с которым мы познакомились, фотон, как частица имеет весьма своеобразные свойства. У него отсутствует масса (покоя), и его единственное состояние - это движение с предельной скоростью \(c\), одинаковой во всех системах отсчета. Не существует системы отсчета, в которой он бы покоился. Фотон в состоянии покоя - понятие, лишенное физического смысла. Попытка остановить фотон или изменить направление его движения равносильны его уничтожению. Такое выражение, как «фотон рассеялся на такой-то частице» широко используют, но лишь постольку, поскольку это не противоречит рассмотрению некоторых явлений с энергетической точки зрения, и только.

Несмотря на эти «странности», фотон все же удобно рассматривать с тех же позиций, что и частицы, обладающие массой. При этом следует особо подчеркнуть, что фотон не похож на обычную частицу, лишь некоторые свойства фотона напоминают свойства частицы.

Корпускулярно-волновой дуализм. Из опытных фактов следует, что при взаимодействии с веществом свет обнаруживает корпускулярные свойства. Однако представление о свете как потоке классических корпускул несовместимо с классическими представлениями об электромагнитных волнах (которые подтверждаются в явлениях интерференции и дифракции).

Очевидно явное противоречие. Действительно, соотношения (1.13) связывают корпускулярные и волновые свойства света: левые части (ε и p) характеризуют фотон как частицу, правые же содержат \(\omega\) и \(\mathbf{k}\), что определяет их волновые свойства. Но именно

сосуществование этих свойств и не может быть логически непротиворечиво объяснено классической физикой. С точки зрения последней понятия частицы и волны исключают друг друга. Каким образом фотон-частица может иметь волновые свойства?

Представить себе такой объект, который совмещал бы несовместимое, - это выше возможностей нашего (классического) воображения. Опытные же факты вынуждают констатировать, что это так и есть, т. е. свет обнаруживает корпускулярно-волновой дуализм (двойственность). При этом фотон проявляет свои корпускулярно-волновые свойства в разных соотношениях: например, в области длинных волн - в основном волновые свойства, а в области коротких волн - корпускулярные.

Итак, фотон нельзя представить моделью, описываемой классическим образом. Он является квантовым объектом, который в принципе невозможно представить себе с помощью классических образов. Мы вынуждены признать, что при изучении явлений следует руководствоваться не тем, что доступно нашему воображению, а тем, что дают наблюдения и опыт.

Забегая вперед, отметим, что обычные корпускулы - электроны, нейтроны, атомы и др., как выяснилось в дальнейшем, обладают и волновыми свойствами. Опыты, вынуждающие нас принять это заключение, будут рассмотрены в главе 3. Поэтому обсуждение проблемы, как современная физика истолковывает корпускулярно-волновой дуализм, мы отложим до § 3.3, после того, как будут рассмотрены волновые свойства вещества.

\section*{§ 1.5. Эффект Комптона}
Опыты Комптона. Комптон (1923) открыл явление, в котором можно было наблюдать, что фотону присущи энергия и импульс. Результаты этого опыта - еще одно убедительное подтверждение гипотезы Эйнштейна о квантовой природе самого электромагнитного излучения.

Комптон исследовал рассеяние жесткого рентгеновского излучения на образцах, состоящих из легких атомов, таких как графит, парафин и др. Схема его установки показана на рис. 1.10.

Источником рентгеновского излучения служила рентгеновская трубка с молибденовым антикатодом. Диафрагмы \(D_{1}\) и \(D_{2}\)

выделяли узкий пучок монохроматического рентгеновского излучения, который падал затем на исследуемый образец \(O\). Для исследования спектрального состава рассеянного излучения оно после прохождения ряда диафрагм попадало на кристалл \(K\) рентгеновского спектрографа, а затем в

\begin{figure}[h]
\begin{center}
  \includegraphics[alt={},max width=\textwidth]{a57b3a57-78b5-4292-b0ab-4bd9f2ffcf56-026_309_438_141_726}
\captionsetup{labelformat=empty}
\caption{Рис. 1.10}
\end{center}
\end{figure}

счетчик \(C\) (или на фотопластинку).

Комптон обнаружил, что в рассеянном излучении, наряду с исходной длиной волны \(\lambda\), появляется смещенная линия с длиной волны \(\lambda^{\prime}>\lambda\). Это получило название комптоновского смещения, а само явление - эффекта Комптона.

Опыт показал, что наблюдаемое комптоновское смещение \(\lambda^{\prime}-\lambda\) не зависит от материала рассеивающего образца и длины волны \(\lambda\) падающего излучения, а определяется лишь углом \(\theta\) между направлениями рассеянного и падающего излучений (см. рис. 1.10). С увеличением угла интенсивность смещенной компоненты растет, а несмещенной - падает. Это показано на рис. 1.11, где представлены результаты измерений на графите при различных углах рассеяния для так называемой \(K_{\alpha}\)-линии молибдена, имеющей длину волны \(0,071 \mathrm{HM}\). Слева показана форма линии исходного излучения (т. е. спектральное распределение интенсивности по длинам волн). Правее - то же самое для рассеянного излучения при различных углах рассеяния.

\begin{figure}[h]
\begin{center}
  \includegraphics[alt={},max width=\textwidth]{a57b3a57-78b5-4292-b0ab-4bd9f2ffcf56-026_295_1002_1329_114}
\captionsetup{labelformat=empty}
\caption{Рис. 1.11}
\end{center}
\end{figure}

Теория эффекта Комптона. Классическая теория оказалась не в состоянии объяснить закономерности комптоновского рас-

сеяния и в первую очередь появление смещенной компоненты. Они были поняты только на основе квантовой теории. Комптон предположил, что рассеяние рентгеновского кванта с изменением длины волны надо рассматривать как результат одиночного акта столкновения его с электроном.

В атомах легких элементов, с которыми проводились опыты, энергия связи электрона с атомом мала по сравнению с энергией, передаваемой электрону рентгеновским квантом при столкновении. Это выполняется тем лучше, чем больше угол рассеяния. В легких атомах энергией связи электрона внутри атома можно пренебречь при всех углах рассеяния, т. е. все электроны можно считать свободными. Тогда одинаковость комптоновского смещения \(\lambda^{\prime}-\lambda\) для всех веществ сразу становится понятной. Действительно, ведь с самого начала предполагается, что рассеивающее вещество по существу состоит только из свободных электронов, т. е. индивидуальные особенности совсем не учитываются. Но это допустимо только для легких атомов. Для внутренних электронов тяжелых атомов такое представление не годится, что и подтверждает опыт.

Теперь рассмотрим столкновение фотона со свободным электроном с учетом того, что при этом должны соблюдаться законы сохранения энергии и импульса. Поскольку в результате столкновения электрон может стать релятивистским, этот процесс будем рассматривать на основе релятивистской динамики.

Пусть на первоначально покоившийся свободный электрон с энергией покоя \(m c^{2}\) падает фотон с энергией \(\varepsilon\) и импульсом \(\varepsilon / c\). После столкновения энергия фотона станет равной \(\varepsilon^{\prime}\), а энергия и импульс электрона отдачи \(E\) и \(p^{\prime}\). Согласно законам сохранения энергии и импульса системы фотон-электрон, запишем до и после столкновения следующие равенства:


\begin{gather*}
\varepsilon+m c^{2}=\varepsilon^{\prime}+E^{\prime}  \tag{1.14}\\
p^{\prime 2}=(\varepsilon / c)^{2}+\left(\varepsilon^{\prime} / c\right)^{2}-2\left(\varepsilon \varepsilon^{\prime} / c^{2}\right) \cos \theta \tag{1.15}
\end{gather*}


\begin{figure}[h]
\begin{center}
  \includegraphics[alt={},max width=\textwidth]{a57b3a57-78b5-4292-b0ab-4bd9f2ffcf56-027_194_376_1585_64}
\captionsetup{labelformat=empty}
\caption{Рис. 1.12}
\end{center}
\end{figure}

где второе равенство записано на основе теоремы косинусов для треугольника импульсов (рис. 1.12).

Имея в виду, что связь между энергией и импульсом релятивистского электрона согласно (П.3) имеет вид


\begin{equation*}
E^{\prime 2}-p^{\prime 2} c^{2}=m^{2} c^{4}, \tag{1.16}
\end{equation*}


найдем \(E^{\prime 2}\) из формулы (1.14) и \(p^{\prime 2} c^{2}\) из (1.15):


\begin{gather*}
E^{\prime 2}=\left(\varepsilon-\varepsilon^{\prime}+m c^{2}\right)^{2}=\varepsilon^{2}+\varepsilon^{\prime 2}+m^{2} c^{4}-2 \varepsilon \varepsilon^{\prime}+2 \varepsilon m c^{2}-2 \varepsilon^{\prime} m c^{2},  \tag{1.17}\\
\left(p^{\prime} c^{\prime}\right)^{2}=\varepsilon^{2}+\varepsilon^{\prime 2}-2 \varepsilon \varepsilon^{\prime} \cos \theta . \tag{1.18}
\end{gather*}


Вычтя в соответствии с (1.16) выражение (1.18) из (1.17) и приравняв полученный результат \(m^{2} c^{4}\), получим после сокращений:


\begin{equation*}
\varepsilon-\varepsilon^{\prime}=\frac{\varepsilon \varepsilon^{\prime}}{m c^{2}}(1-\cos \theta) . \tag{1.19}
\end{equation*}


Остается учесть, что \(\varepsilon=\hbar \omega, \varepsilon^{\prime}=\hbar \omega^{\prime}\), а также связь между \(\omega\) и \(\lambda\) ( \(\omega=2 \pi \mathrm{c} / \lambda\) ), и мы получим:


\begin{equation*}
\lambda^{\prime}-\lambda=\lambda_{C}(1-\cos \theta), \tag{1.20}
\end{equation*}


где \(\lambda_{C}\) - комптоновская длина волньι* частицы массы \(m\),


\begin{equation*}
\lambda_{C}=2 \pi \hbar / m c . \tag{1.21}
\end{equation*}


Для электрона \(\lambda_{C}=2,43 \cdot 10^{-10}\) см.\\
Универсальная постоянная \(\lambda_{C}\) является одной из важнейших атомных констант.

Анализ полученных результатов. Соотношение (1.20) очень хорошо согласуется с наблюдаемой на опыте зависимостью комптоновского смещения от угла рассеяния \(\theta\) (см. рис. 1.11). Уширение обеих компонент рассеянного излучения обусловлено движением электронов и атомов, на которых происходит рассеяние, т. е. эффектом Доплера.

Наличие несмещенной компоненты в рассеянном излучении обусловлено внутренними электронами атомов рассеивающего вещества. Их энергия связи, особенно в тяжелых атомах, сравнима с энергией рентгеновских фотонов, и, значит, такие электроны уже нельзя считать свободными. Обмен энергией и им-

\footnotetext{\begin{itemize}
  \item Комптоновской длиной волны называют также
\end{itemize}

\[
\lambda_{C}=\lambda_{C} / 2 \pi=\hbar / m c .
\]
}пульсом рентгеновского фотона происходит с атомом как целым. Масса же атома намного превышает массу электрона, поэтому комптоновское смещение фотонов, рассеянных на таких атомах, ничтожно, и их смещенная длина волны \(\lambda^{\prime}\) практически совпадает с длиной волны \(\lambda\) падающего излучения. Это, кстати, сразу видно из формул (1.20) и (1.21).

С ростом атомного номера относительное число связанных электронов увеличивается. Поэтому должно происходить возрастание интенсивности несмещенной компоненты по сравнению с интенсивностью смещенной. Это и наблюдается на опыте.

Кроме того, с ростом угла рассеяния \(\theta\) доля передаваемой электрону энергии возрастает. Отсюда следует, что при увеличении угла рассеяния \(\theta\) растет относительная доля электронов, которые можно считать свободными, а значит растет и отношение интенсивности смещенной компоненты к интенсивности несмещенной, что и показывает опыт.

Итак, чем больше энергия фотона, тем в меньшей степени проявляется связь электрона с атомом, тем больше электронов, которые можно считать свободными. Именно поэтому для наблюдения эффекта Комптона нужно использовать жесткое рентгеновское излучение. Вот почему эффект Комптона не наблюдается в видимой области спектра. Энергия соответствующих фотонов настолько мала, что даже внешние электроны атома не могут играть роль свободных.

В дальнейших опытах Комптона и других исследователей удалось зарегистрировать электроны отдачи и показать, что в элементарных актах рассеяния фотонов на электронах выполняются законы сохранения энергии и импульса.

\begin{figure}[h]
\begin{center}
  \includegraphics[alt={},max width=\textwidth]{a57b3a57-78b5-4292-b0ab-4bd9f2ffcf56-029_247_415_1389_68}
\captionsetup{labelformat=empty}
\caption{Рис. 1.13}
\end{center}
\end{figure}

В связи с этим упомянем, например, опыты Боте и Гейгера (1925), которые доказали, что электрон отдачи и рассеянный фотон появляются одновременно. Схема опыта показана на рис. 1.13, где \(X\) - источник рентгеновского излучения, \(P\) - рассеиватель, в котором под действием излучения происходит Комптон-эффект, \(\Phi\) и Э - счетчики рассеянных фотонов и электронов отдачи. Эти

счетчики установлены симметрично относительно рассеивателя \(P\) и включены в схему совпадений \(C\), т. е. в электрическую схему, которая позволяет регистрировать лишь те случаи, когда фотон и электрон в счетчиках \(\Phi\) и Э появляются одновременно.

В результате было установлено, что число одновременных регистраций фотона и электрона в счетчиках во много раз превосходит то число, которое можно было ожидать при случайном по времени появлении фотона и электрона. Так было доказано существование индивидуального столкновения фотона с электроном.

\section*{Задачи}
1.1. Фотоэффект. Найти работу выхода \(A\) с поверхности некоторого металла, если при поочередном освещении его электромагнитным излучением с длинами волн \(\lambda_{1}=0,35\) мкм и \(\lambda_{2}=0,54\) мкм максимальная скорость фотоэлектронов отличается в \(\eta=2,0\) раза.\\
Решение. Запишем уравнение Эйнштейна (1.3) для обеих длин волн (частот):


\begin{align*}
& \hbar \mathrm{w}_{1}=A+m v_{1}^{2} / 2, \\
& \hbar \mathrm{w}_{2}=A+m v_{2}^{2} / 2, \tag{*}
\end{align*}


где \(v_{1}\) и \(v_{2}\) - максимальные скорости фотоэлектронов, причем \(v_{1}>v_{2}\), поскольку в данном случае \(\omega_{1}>\omega_{2}\). Из уравнений (*) составим отношение

\[
\frac{\hbar \omega_{1}-A}{\hbar \omega_{2}-A}=\frac{v_{1}^{2}}{v_{2}^{2}}=\eta^{2}
\]

Из этого соотношения, учитывая, что \(\omega=2 \pi c / \lambda\), находим:

\[
A=\frac{2 \pi \hbar c}{\lambda_{2}} \frac{\eta^{2}-\lambda_{2} / \lambda_{1}}{\eta^{2}-1}=3,04 \cdot 10^{-12} \text { эрг }=1,9 \text { эВ. }
\]

1.2. При некоторой задерживающей разности потенциалов фототок с поверхности лития, освещаемого электромагнитным излучением с длиной волны \(\lambda_{0}\), прекращается. Изменив длину волны излучения в \(n=1,5\) раза, установили, что для прекращения фототока необходимо увеличить задерживающую разность потенциалов в \(\eta=2,0\) раза. Работа выхода электрона с поверхности лития \(A=2,39\) эВ. Вычислить \(\lambda_{0}\).

Решение. Запишем в соответствии с уравнением (1.3) и условием задачи два уравнения:

\[
e V_{\mathrm{S}}=\frac{\alpha}{\lambda_{0}}-A, \quad \eta e V_{3}=\frac{\alpha}{\lambda_{0} / n}-A,
\]

где \(\alpha=2 \pi \hbar c\). Разделив второе уравнение на первое, получим:

\[
\eta=\frac{\alpha n-A \lambda_{0}}{\alpha-A \lambda_{0}},
\]

откуда

\[
\lambda_{0}=\frac{\alpha(\eta-n)}{A(\eta-1)}=\frac{2 \pi \hbar c(\eta-n)}{A(\eta-1)}=0,26 \text { мкм. }
\]

1.3. Ток, возникающий в цепи вакуумного фотоэлемента при освещении цинкового электрода электромагнитным излучением с длиной волны \(\lambda=262\) нм, прекращается, когда внешняя разность потенциалов (показания вольтметра) достигает значения \(V_{1}=-1,5\) В. Имея в

\begin{figure}[h]
\begin{center}
  \includegraphics[alt={},max width=\textwidth]{a57b3a57-78b5-4292-b0ab-4bd9f2ffcf56-031_378_465_838_146}
\captionsetup{labelformat=empty}
\caption{Рис. 1.14}
\end{center}
\end{figure}

\[
\hbar \omega=A+e V_{3}=A+e\left(V_{2}-V_{1}\right),
\]

где \(V_{2}\) - искомая контактная разность потенциалов. Отсюда \(e V_{2}=2 \pi \hbar c / \lambda-A+e V_{1}=-0,5\) эВ. Значит (рис. 1.14)

\[
V_{2}=-0,5 \mathrm{~B} .
\]

1.4. Коротковолновая граница рентгеновского спектра. После увеличения напряжения на рентгеновской трубке в \(\eta=2,0\) раза первоначальная длина волны \(\lambda_{0}\) коротковолновой границы сплошного рентгеновского спектра изменилась на \(\Delta \lambda=50\) пм. Найти \(\lambda_{0}\). Решение. В данном случае «изменилась на \(\Delta \lambda\) » - это значит уменьшилась на такую величину. Поэтому согласно (1.8) можно записать:

\[
\lambda_{0}=\frac{a}{V_{1}}, \quad \lambda_{0}^{\prime}=\lambda_{0}-\Delta \lambda=\frac{a}{V_{2}},
\]

где \(V_{1}\) и \(V_{2}\) - напряжения на рентгеновской трубке, \(a\) - постоянная.

Разделив второе равенство на первое, получим:

\[
\frac{\lambda_{0}-\Delta \lambda}{\lambda_{0}}=\frac{V_{1}}{V_{2}}=\frac{1}{\eta} .
\]

Отсюда находим

\[
\lambda_{0}=\frac{\eta}{\eta-1} \Delta \lambda=0,10 \mathrm{HM} .
\]

1.5. Метод изохромат. В сплошном рентгеновском спектре интенсивность \(I\) излучения с длиной волны \(\lambda_{0}=50\) пм зависит следующим образом от напряжения \(V\) на рентгеновской трубке:

\begin{center}
\begin{tabular}{ccccc}
\(V\), кВ & 29 & 28 & 27 & 26 \\
\(I\), отн. ед. & 9,0 & 6,0 & 3,5 & 1,7 \\
\end{tabular}
\end{center}

Вычислить с помощью соответствующего графика постоянную Планка \(\hbar\).

Решение. Изобразим график зависимости \(I(V)\), экстраполируя его к нулю, как показано на рис. 1.15, находим \(V_{0}=25\) кВ. При этом напряжении излучение с длиной волны \(\lambda_{0}\) становится коротковолновой границей сплошного рентгеновского спектра. Значит, согласно (1.9)

\[
\hbar=\frac{\lambda_{0} e V_{0}}{2 \pi c}=1,06 \cdot 10^{-27} \text { эрг ⋅ с. }
\]

\begin{figure}[h]
\begin{center}
  \includegraphics[alt={},max width=\textwidth]{a57b3a57-78b5-4292-b0ab-4bd9f2ffcf56-032_288_384_915_778}
\captionsetup{labelformat=empty}
\caption{Рис. 1.15}
\end{center}
\end{figure}

1.6. Комптоновские электроны. Фотон с энергией \(\varepsilon\) рассеялся под углом \(\theta\) на покоившемся свободном электроне. Определить угол φ, под которым вылетел электрон отдачи относительно направления налетевшего фотона.

Решение. Из треугольника импульсов (рис. 1.16), выражающего собой за-

\begin{figure}[h]
\begin{center}
  \includegraphics[alt={},max width=\textwidth]{a57b3a57-78b5-4292-b0ab-4bd9f2ffcf56-032_285_398_1345_764}
\captionsetup{labelformat=empty}
\caption{Рис. 1.16}
\end{center}
\end{figure}

кон сохранения импульса, видно, что


\begin{equation*}
\operatorname{tg} \varphi=\frac{k^{\prime} \sin \theta}{k-k^{\prime} \cos \theta}=\frac{\sin \theta}{\lambda^{\prime} / \lambda-\cos \theta} \tag{1}
\end{equation*}


Согласно формуле (1.20), определяющей комптоновское смещение,


\begin{equation*}
\frac{\lambda^{\prime}}{\lambda}=1+\frac{\lambda_{C}}{\lambda}(1-\cos \theta) . \tag{2}
\end{equation*}


Подставив (2) в (1), получим после несложных преобразований:

\[
\operatorname{tg} \varphi=\frac{\sin \theta}{(1-\cos \theta)\left(1+\lambda_{C} / \lambda\right)}=\frac{\operatorname{ctg}(\theta / 2)}{1+\varepsilon / m c^{2}},
\]

где учтено, что \(\lambda_{C} / \lambda=2 \pi \hbar / m c \lambda=\varepsilon / m c^{2}\).\\
1.7. Эффект Комптона. При облучении вещества рентгеновским излучением с некоторой длиной волны \(\lambda\) обнаружили, что максимальная кинетическая энергия релятивистских электронов отдачи равна \(K_{m}\). Определить \(\lambda\).

Р e ш e н e . В соответствии с законами сохранения энергии и импульса имеем


\begin{equation*}
\varepsilon-\varepsilon^{\prime}=K_{m}, \quad \varepsilon / c+\varepsilon^{\prime} / c=p, \tag{1}
\end{equation*}


где \(\varepsilon\) и \(\varepsilon^{\prime}\) - энергия фотона до и после столкновения с электро-

\begin{figure}[h]
\begin{center}
  \includegraphics[alt={},max width=\textwidth]{a57b3a57-78b5-4292-b0ab-4bd9f2ffcf56-033_121_405_972_132}
\captionsetup{labelformat=empty}
\caption{Рис. 1.17}
\end{center}
\end{figure}

ном, \(p\) - его импульс отдачи. Во второй формуле учтено согласно условию задачи, что все три импульса должны быть коллинеарными (рис. 1.17), чтобы импульс \(p\) был максимальным.

Умножив все слагаемые второго из уравнений (1) на \(c\) и сложив после этого полученное выражение с первым уравнением, найдем


\begin{equation*}
2 \varepsilon=K_{m}+p c . \tag{2}
\end{equation*}


В релятивистской динамике связь между импульсом и кинетической энергией электрона легко получить с помощью инвариантного выражения \(E^{2}-p^{2} c^{2}=m^{2} c^{4}\), где \(E=m c^{2}+K\), откуда \(p c= =\sqrt{K_{m}\left(K_{m}+2 m c^{2}\right)}\). Тогда уравнение (2) примет вид


\begin{equation*}
2 \hbar \frac{2 \pi c}{\lambda}=K_{m}+\sqrt{K_{m}\left(K_{m}+2 m c^{2}\right)} . \tag{3}
\end{equation*}


Из последнего уравнения находим:


\begin{equation*}
\lambda=\frac{4 \pi \hbar c}{K_{m}\left(1+\sqrt{1+2 m c^{2} / K_{m}}\right)} . \tag{4}
\end{equation*}


Это выражение можно представить и в другом виде, умножив числитель и знаменатель на \(\sqrt{1+2 m c^{2} / K_{m}}-1\). Тогда

\[
\lambda=\frac{2 \pi \hbar}{m c}\left(\sqrt{1+\frac{2 m c^{2}}{K_{m}}}-1\right) .
\]

1.8. Обратный эффект Комптона. При столкновении с релятивистским электроном фотон рассеялся на угол \(\theta\), а электрон остановился. Найти комптоновское смещение длины волны рассеянного фотона.

Решение. Согласно закону сохранения импульса

\[
\hbar \mathbf{k}+\mathbf{p}=\hbar \mathbf{k}^{\prime}
\]

где \(\mathbf{k}\) и \(\mathbf{k}^{\prime}\) - волновые векторы первоначального и рассеянного фотонов, p - импульс электрона (рис. 1.18). Из этого рисунка согласно теореме косинусов имеем

Рис. 1.18


\begin{equation*}
p^{2} c^{2}=\varepsilon^{2}+\varepsilon^{\prime 2}-2 \varepsilon \varepsilon^{\prime} \cos \theta, \tag{1}
\end{equation*}


где учтено, что \(k=\omega / c, k^{\prime}=\omega^{\prime} / c\); ε и \(\varepsilon^{\prime}\) — энергия фотона до и после рассеяния.

На основании закона сохранения энергии запишем

\[
\varepsilon+E=\varepsilon^{\prime}+m c^{2},
\]

где \(E\) - полная энергия электрона, \(m\) - его масса покоя. Из этого равенства найдем \(E^{2}\) :


\begin{equation*}
E^{2}=\varepsilon^{2}+\varepsilon^{\prime 2}+m^{2} c^{4}-2 \varepsilon \varepsilon^{\prime}-2 \varepsilon m c^{2}+2 \varepsilon^{\prime} m c^{2} . \tag{2}
\end{equation*}


Теперь воспользуемся инвариантностью выражения \(E^{2}-p^{2} c^{2}\), которое равно \(m^{2} c^{4}\), а именно, вычтем (1) из (2). В результате после сокращений получим:


\begin{equation*}
\varepsilon \varepsilon^{\prime}(1-\cos \theta)=m c^{2}\left(\varepsilon^{\prime}-\varepsilon\right), \tag{3}
\end{equation*}


или

\[
\frac{\hbar}{m c^{2}}(1-\cos \theta)=\frac{1}{\omega}-\frac{1}{\omega^{\prime}}=\frac{\lambda-\lambda^{\prime}}{2 \pi c} .
\]

Из последнего выражения находим

\[
\lambda^{\prime}-\lambda=-\frac{2 \pi \hbar}{m c}(1-\cos \theta)<0,
\]

т. е. длина волны рассеянного фотона становится меньше и его энергия увеличивается.\\
1.9. Давление света. Плоский световой поток интенсивности \(I\), Вт \(/ \mathrm{m}^{2}\) освещает половину зеркальной сферической поверхности радиуса \(R\). Найти с помощью корпускулярных представлений силу светового давления, испытываемую сферой.

Решение. Для простоты будем считать падающий свет монохроматическим с частотой \(\omega\). Как это отразится на окончательном результате, мы увидим.

Сначала найдем силу \(\mathrm{d} F\), действующую на элементарное кольцо \(\mathrm{d} S\) (рис.1.19) в направлении оси \(X\). При зеркальном отражении каждый фотон передает поверхности импульс \(\Delta p_{x}\) (рис. 1.20):

\[
\Delta p_{x}=p-p_{x}=p-p \cos (\pi-2 \theta)=p(1+\cos 2 \theta)=2 p \cos ^{2} \theta,
\]

где \(p=\hbar \omega / c\).

\begin{figure}[h]
\begin{center}
  \includegraphics[alt={},max width=\textwidth]{a57b3a57-78b5-4292-b0ab-4bd9f2ffcf56-035_330_261_969_190}
\captionsetup{labelformat=empty}
\caption{Рис. 1.19}
\end{center}
\end{figure}

\begin{figure}[h]
\begin{center}
  \includegraphics[alt={},max width=\textwidth]{a57b3a57-78b5-4292-b0ab-4bd9f2ffcf56-035_290_458_1010_630}
\captionsetup{labelformat=empty}
\caption{Рис. 1.20}
\end{center}
\end{figure}

Число фотонов, падающих ежесекундно на элементарное кольцо \(d S\) (см. рис. 1.19), равно \(\mathrm{d} N=(I / \hbar \omega) \mathrm{d} S \cos \theta\), где \(\mathrm{d} S=2 \pi R \sin \theta \cdot R \mathrm{~d} \theta\). Тогда

\[
\mathrm{d} F=\Delta p_{x} \cdot \mathrm{~d} N=4 \pi R^{2}(I / c) \cos ^{3} \theta \sin \theta \mathrm{~d} \theta .
\]

Заметим, что частота света \(\omega\) сократилась, значит она не играет здесь роли.

Проинтегрировав последнее выражение по \(\theta\) от 0 до \(\pi / 2\), получим

\[
F=\pi R^{2} I / c .
\]

Интересно, что полученный результат в данном случае такой же, как и в случае абсолютно поглощающей поверхности. Кроме того, он в точности совпадает с результатом, полученным с помощью классических волновых представлений.\\
1.10. Эффект Доплера. Возбужденный атом, двигавшийся с нерелятивистской скоростью \(v\), испустил фотон под углом \(\theta\) к первоначальному направлению движения атома. Найти с помощью законов сохранения энергии и импульса относительное смещение частоты фотона, обусловленной отдачей атома.

Решение. Пусть «закрепленный» неподвижный атом при переходе из возбужденного состояния в нормальное испускает фотон с энергией ћω. Разность энергий указанных состояний атома равна \(\hbar \omega\) вне зависимости от того, покоится атом или движется. При испускании фотона свободно движущимся атомом импульс атома изменяется, поскольку испущенный фотон обладает импульсом. Изменяется и кинетическая энергия атома.

Согласно законам сохранения энергии и импульса (рис. 1.21),

\[
\begin{gathered}
p^{2} / 2 m+E^{*}=p^{\prime 2} / 2 m+\hbar \omega^{\prime} \\
p^{\prime 2}=p^{2}+p_{\phi}^{2}-2 p p_{\phi} \cos \theta
\end{gathered}
\]

Рис. 1.21

где \(E^{*}\) - энергия возбуждения атома, \(E^{*}=\hbar \omega\), а \(p_{\phi}=\hbar \omega^{\prime} / c\). Исключив из этих двух уравнений \(p^{\prime 2}\), получим:

\[
\omega^{\prime}-\omega=\omega^{\prime}\left(\frac{v}{c} \cos \theta-\frac{\hbar \omega^{\prime}}{2 m c^{2}}\right) .
\]

Учитывая, что энергия фотона \(\hbar \omega^{\prime} \ll 2 m c^{2}\) и \(\omega^{\prime}\) перед скобкой можно заменить на \(\omega\) (их разность весьма мала), приходим к следующему результату:

\[
\frac{\Delta \omega}{\omega}=\frac{v}{c} \cos \theta,
\]

где \(\Delta \omega=\omega^{\prime}-\omega\). Полученная формула совпадает с обычным нерелятивистским выражением для эффекта Доплера.

\section*{Атом Резерфорда - Бора}
\section*{§ 2.1. Ядерная модель атома}
Введение. В настоящее время мы знаем, что любой атом состоит из положительно заряженного ядра и окружающей его электронной оболочки. Размеры ядра менее \(10^{-12}\) см, размеры же самого атома, определяемые электронной оболочкой, порядка \(10^{-8} \mathrm{~cm}\), т. е. в десятки тысяч раз больше размеров ядра. При этом практически вся масса атома сосредоточена в ядре.

Если все это так, то атом должен быть в высокой степени прозрачным для пронизывающих его частиц. Экспериментальное доказательство изложенной модели атома было дано Резерфордом (1911) с помощью рассеяния \(\alpha\)-частиц (ядер атомов He) тонкой металлической фольгой.

Было обнаружено, что подавляющее число \(\alpha\)-частиц рассеивалось на небольшие углы (не больше \(\sim 3^{\circ}\) ). Вместе с тем наблюдались также отдельные \(\alpha\)-частицы, рассеянные на большие углы. Относительно последних Резерфорд сделал вывод, что такие частицы появляются в результате единичного акта их взаимодействия с ядром атома.

Исходя из предположений, что взаимодействие указанных \(\alpha\)-частиц с ядром является кулоновским, а заряд и масса ядра локализованы в очень малой области атома, Резерфорд разработал количественную теорию рассеяния \(\alpha\)-частиц и вывел формулу для распределения рассеянных \(\alpha\)-частиц в зависимости от угла отклонения \(\theta\). В своих рассуждениях Резерфорд принимал во внимание рассеяние \(\alpha\)-частиц только на ядрах, поскольку заметного отклонения \(\alpha\)-частиц электронами не может быть из-за того, что масса электронов на четыре порядка меньше

\begin{figure}[h]
\begin{center}
  \includegraphics[alt={},max width=\textwidth]{a57b3a57-78b5-4292-b0ab-4bd9f2ffcf56-037_301_461_1472_68}
\captionsetup{labelformat=empty}
\caption{Рис. 2.1}
\end{center}
\end{figure}

массы \(\alpha\)-частиц.

Когда \(\alpha\)-частица пролетает вблизи ядра, ее траектория представляет собой гиперболу, причем угол отклонения \(\alpha\)-частицы - угол \(\theta\) - равен углу между асимптотами гиперболы (рис. 2.1).

Для угла \(\theta\) было получено выражение


\begin{equation*}
\operatorname{tg} \frac{\theta}{2}=\frac{q q_{0}}{2 b K}, \tag{2.1}
\end{equation*}


где \(q\) и \(q_{0}\) - заряды налетающей частицы и ядра, \(b\) - прицельный параметр, т. е. расстояние от ядра до первоначального направления движения налетающей частицы, когда она находится вдали от ядра (см. рис. 2.1), \(K\) - кинетическая энергия частицы вдали от ядра.

Из формулы (2.1) видно, что чем меньше прицельный параметр \(b\), тем больше угол отклонения \(\theta\).

Вывод формулы (2.1) приведен в Приложении.\\
Формула Резерфорда. Непосредственная проверка формулы (2.1) экспериментально невозможна, поскольку мы не можем измерить прицельный параметр \(b\) налетающей частицы. Однако, следуя Резерфорду, мы можем положить формулу (2.1) в основу для следующих расчетов.

Рассмотрим тонкий слой рассеивающего вещества, настолько тонкий (фольга), чтобы каждая налетающая частица пучка претерпевала лишь однократное отклонение. Для отклонения в интервале углов ( \(\theta, \theta+d \theta\) ) прицельный параметр должен быть заключен в интервале ( \(b, b+\mathrm{d} b\) ). При этом значения \(\mathrm{d} \theta\) и \(\mathrm{d} b\) будут связаны определенным соотношением. Чтобы найти его, перепишем сначала (2.1) в виде


\begin{equation*}
b=\frac{q q_{0}}{2 K} \operatorname{ctg} \frac{\theta}{2}, \tag{2.2}
\end{equation*}


а затем возьмем дифференциал от этого выражения


\begin{equation*}
\mathrm{d} b=-\frac{q q_{0}}{2 K} \frac{\mathrm{~d} \theta}{2 \sin (\theta / 2)} . \tag{2.3}
\end{equation*}


Знак минус в этом выражении обусловлен тем, что знаки \(\mathrm{d} b\) и \(\mathrm{d} \theta\) взаимно противоположны. В дальнейшем существенным будет лишь модуль величин \(\mathrm{d} b\) и \(\mathrm{d} \theta\), поэтому знак минус в (2.3) мы не будем учитывать.

Пусть площадь поперечного сечения узкого пучка налетающих частиц равна \(S\). Тогда число ядер рассеивающего тон-

\begin{figure}[h]
\begin{center}
  \includegraphics[alt={},max width=\textwidth]{a57b3a57-78b5-4292-b0ab-4bd9f2ffcf56-039_281_283_141_70}
\captionsetup{labelformat=empty}
\caption{Рис. 2.2}
\end{center}
\end{figure}

кого слоя будет равно \(n S\), где \(n\) - число ядер (атомов) в расчете на единицу поверхности. При этом относительное число частиц, имеющих прицельный параметр \(b\) в интервале ( \(b, b+\mathrm{d} b\) ) и, значит, рассеянных в интервале углов ( \(\theta, \theta+d \theta\) ), будет равно (рис. 2.2)


\begin{equation*}
\frac{\mathrm{d} N}{N}=\frac{\mathrm{d} S}{S}=\frac{n S \cdot 2 \pi b \mathrm{~d} b}{S}=n \cdot 2 \pi b \mathrm{~d} b, \tag{2.4}
\end{equation*}


где \(\mathrm{d} S\) - суммарная площадь колец в сечении \(S\) пучка, \(\mathrm{d} N\) поток частии, рассеянных в интервале углов ( \(\theta, \theta+\mathrm{d} \theta\) ), и \(N\) - поток падающих частиц в пучке.

Подставив в (2.4) выражения для \(b\) и \(\mathrm{d} b\) из (2.2) и (2.3), получим:


\begin{equation*}
\frac{\mathrm{d} N}{N}=n\left(\frac{q q_{0}}{2 K}\right)^{2} 2 \pi \frac{\cos (\theta / 2) \mathrm{d} \theta}{2 \sin ^{3}(\theta / 2)} . \tag{2.5}
\end{equation*}


Умножим числитель и знаменатель правой части этого равенства на \(\sin (\theta / 2)\). Тогда


\begin{equation*}
\frac{\mathrm{d} N}{N}=n\left(\frac{q q_{0}}{2 K}\right)^{2} \frac{2 \pi \sin \theta \mathrm{~d} \theta}{4 \sin ^{4}(\theta / 2)}, \tag{2.6}
\end{equation*}


где выражение \(2 \pi \sin \theta \mathrm{~d} \theta\) - это телесный угол \(\mathrm{d} \Omega\), в пределах которого заключены углы рассеяния ( \(\theta, \theta+d \theta\) ). Поэтому (2.6) можно переписать так:


\begin{equation*}
\frac{\mathrm{d} N}{N}=n\left(\frac{q q_{0}}{4 K}\right)^{2} \frac{\mathrm{~d} \Omega}{\sin ^{4}(\theta / 2)} . \tag{2.7}
\end{equation*}


Это и есть формула Резерфорда. Она определяет относительное число частиц, рассеянных в телесном угле \(\mathrm{d} \Omega\) под углом \(\theta\) к первоначальному направлению их движения. Напомним, что в этой формуле \(n\) - число ядер на единицу поверхности рассеивающего слоя (фольги).

Если нас интересует относительное число \(\Delta N / N\) частиц, рассеянных в конечном интервале углов от \(\theta_{1}\) до \(\theta_{2}\), то выражение (2.7) надо проинтегрировать, учитывая, что \(\mathrm{d} \Omega= =2 \pi \sin \theta \mathrm{~d} \theta\). При этом следует иметь в виду, что для малых углов рассеяния (приблизительно меньших \(3^{\circ}\) ) формула Резерфорда не применима. Это связано с тем, что очень малым углам соответствуют большие значения прицельного параметра, выходящие за пределы атома, где сила уже не имеет кулоновского характера.

Заметим, что вопрос о нахождении относительного числа частиц, рассеянных в конечном интервале углов \(\theta\), может быть решен значительно проще (без интегрирования). Как именно, показано в нижеследующем примере.

Эффективное сечение. Формулу Резерфорда (2.7) можно представить в несколько ином виде, если ввести понятие \(\partial u \phi\) ференциального сечения \(\mathrm{d} \sigma\), равного площади кольца радиусом \(b\) и шириной \(\mathrm{d} b\) (см. рис. 2.2). Имея прицельные параметры в интервале ( \(b, b+\mathrm{d} b\) ), налетающие частицы отклоняются ядрами согласно (2.1) на углы в интервале ( \(\theta, \theta+\mathrm{d} \theta\) ). Поскольку


\begin{equation*}
\mathrm{d} \sigma=2 \pi b \mathrm{~d} b, \tag{2.8}
\end{equation*}


формулу (2.7) можно представить так:


\begin{equation*}
\frac{\mathrm{d} N}{N}=n \mathrm{~d} \sigma, \tag{2.9}
\end{equation*}


где дифференциальное эффективное сечение


\begin{equation*}
\mathrm{d} \sigma=\left(\frac{q q_{0}}{4 K}\right)^{2} \frac{2 \pi \sin \theta \mathrm{~d} \theta}{\sin ^{4}(\theta / 2)} . \tag{2.10}
\end{equation*}


Таким образом, формула (2.9) означает, что относительное число частиц, рассеянных в интервале углов ( \(\theta, \theta+d \theta\) ), равно произведению количества ядер на единицу поверхности фольги \((n)\) на соответствующее дифференциальное сечение (2.10).

Пример. Найдем относительное число \(\Delta N / N\) частиц, рассеянных в интервале углов от \(\theta_{1}\) до \(\theta_{2}\). Остальное предполагается заданным.

Величина \(\Delta N / N\) пропорциональна согласно (2.9) площади кольца, внутренний и внешний радиусы которого равны \(b_{1}\) и \(b_{2}\), т. е.


\begin{equation*}
\Delta N / N=n\left(\pi b_{2}^{2}-\pi b_{1}^{2}\right) . \tag{*}
\end{equation*}


Значения же \(b_{1}\) и \(b_{2}\) однозначно связаны с углами \(\theta_{1}\) и \(\theta_{2}\) формулой (2.1) или (2.2). Заменив параметр \(b\) в (*) выражением (2.2), получим:

\[
\frac{\Delta N}{N}=n \pi\left(\frac{q q_{0}}{2 K}\right)^{2}\left(\operatorname{ctg}^{2} \frac{\theta_{2}}{2}-\operatorname{ctg}^{2} \frac{\theta_{1}}{2}\right) .
\]

Вот и весь расчет. Практически так и следует поступать.\\
Проверка формулы Резерфорда. Формула (2.7) была подтверждена экспериментально. В качестве налетающих частиц использовали \(\alpha\)-частицы (их заряд \(q=2 e\) ) от радиоактивного источника. Кинетическая энергия \(\alpha\)-частиц была порядка нескольких МэВ.

Если зафиксировать телесный угол \(\mathrm{d} \Omega\), в котором подсчитывают рассеянные \(\alpha\)-частицы, и менять при этом угол \(\theta\) (рис. 2.3), то из формулы (2.7) следует, что


\begin{equation*}
\mathrm{d} N \cdot \sin ^{4}(\theta / 2)=\text { const. } \tag{2.11}
\end{equation*}


\begin{figure}[h]
\begin{center}
  \includegraphics[alt={},max width=\textwidth]{a57b3a57-78b5-4292-b0ab-4bd9f2ffcf56-041_283_415_1230_68}
\captionsetup{labelformat=empty}
\caption{Рис. 2.3}
\end{center}
\end{figure}

На опыте прежде всего было проверено соблюдение именно этого условия. Оказалось, что, несмотря на то, что каждый из сомножителей в левой части (2.11) изменялся (в процессе изменения угла \(\theta\) ) на три порядка, их произведение с хорошей точностью оставалось постоянным. Это означает, что формула (2.7) правильно описывает процесс рассеяния \(\alpha\)-частиц.\\
Опыты, подтверждающие формулу Резерфорда, могут рассматриваться как косвенное доказательство справедливости закона Кулона на весьма малых расстояниях (от \(10^{-12}\) до \(10^{-9} \mathrm{~cm}\) ).

Кроме того, они свидетельствуют в пользу предположения, что масса атома практически сосредоточена в очень малой его области - в ядре, размеры которого не превышают \(10^{-12}\) см.

Пример. Найдем расстояние, на которое приблизится \(\alpha\)-частица к неподвижному ядру атома золота, двигаясь точно по направлению к его центру. Порядковый номер атома золота \(Z=79\) и кинетическая энергия налетающей \(\alpha\)-частицы вдали от ядра \(K=5,7\) МэВ.

В момент остановки \(\alpha\)-частицы ее кинетическая энергия целиком переходит в потенциальную: \(K=2 Z e^{2} / r_{\text {мин }}\). Отсюда

\[
r_{\text {мин }}=\frac{2 Z e^{2}}{K}=\frac{2 \cdot 79\left(4,8 \cdot 10^{-10}\right)^{2}}{5,7 \cdot 1,6 \cdot 10^{-6}}=4 \cdot 10^{-12} \mathrm{~cm} .
\]

Из вышесказанного не следует, что закон Кулона справедлив на любых расстояниях между налетающей частицей и ядром. Опыты по рассеянию, например, протонов, ускоренных ускорителем, показали, что при достаточно больших энергиях наблюдаются резкие отступления от закона Кулона, когда прицельный параметр становится менее \(10^{-12}\) см. На таких расстояниях проявляют свое действие ядерные силь притяжения, значительно превосходящие кулоновские силы отталкивания.

Итак, результаты опытов по рассеянию \(\alpha\)-частиц говорят в пользу ядерной (планетарной) модели атома, предложенной Резерфордом. Однако эта модель оказалась в резком противоречии с законами классической электродинамики.

Предположение, что электроны движутся вокруг ядра по траекториям, подобно планетам вокруг Солнца, наталкивается на непреодолимую (с точки зрения классики) трудность. Двигаясь по искривленным траекториям, электрон испытывает ускорения, а значит неизбежно должен излучать электромагнитные волны. Этот процесс сопровождается потерей энергии, в результате чего электрон должен в конечном счете упасть на ядро. Время жизни такого атома оказывается порядка \(10^{-11}\) с (см. задачу 2.5). Этот результат красноречиво говорит о степени неустойчивости рассмотренной модели атома.

\section*{§ 2.2. Спектральные закономерности}
В первую очередь нас будут интересовать спектры, обусловленные излучением невзаимодействующих друг с другом атомов. Эти спектры состоят из отдельных узких спектральных линий, и их называют линейчатыми.

Наличие многих спектральных линий указывает на сложность внутреннего строения атома. Изучение атомных спектров послужило ключом к познанию внутренней структуры атомов. Прежде всего было замечено, что спектральные линии расположены не беспорядочно, а образуют серии линий. Изучая линейчатый спектр атомарного водорода, Бальмер (1885) установил следующую закономерность. В современных обозначениях она выглядит так*:


\begin{equation*}
\omega=R\left(\frac{1}{2^{2}}-\frac{1}{n^{2}}\right), \quad n=3,4,5, \ldots, \tag{2.12}
\end{equation*}


где \(\omega\) - циклическая частота, соответствующая каждой спектральной линии ( \(\omega=2 \pi c / \lambda\) ), \(R\) - постоянная Ридберга:


\begin{equation*}
R=2,07 \cdot 10^{16} \mathrm{c}^{-1} . \tag{2.13}
\end{equation*}


Формулу (2.12) называют формулой Бальмера, а соответствующую серию спектральных линий - серией Бальмера (рис. 2.4). Основные линии этой серии находятся в видимой части спектра.

Дальнейшие исследования спектра атомарного водорода показали, что

\begin{figure}[h]
\begin{center}
  \includegraphics[alt={},max width=\textwidth]{a57b3a57-78b5-4292-b0ab-4bd9f2ffcf56-043_168_386_1117_770}
\captionsetup{labelformat=empty}
\caption{Рис. 2.4}
\end{center}
\end{figure}

имеется еще несколько серий. В ультрафиолетовой части спек-

\footnotetext{\begin{itemize}
  \item В спектроскопии принято характеризовать спектральные линии не частотой, а так называемым волновым числом \(\bar{v}\) :
\end{itemize}

\[
\bar{v}=\frac{1}{\lambda}=\frac{\omega}{2 \pi c} \mathrm{~cm}^{-1},
\]

где \(\lambda\) - длина волны. Формула Бальмера, написанная для волнового числа \(\bar{v}\), имеет такой же вид, как (2.12):

\[
\bar{v}=\bar{R}\left(\frac{1}{2^{2}}-\frac{1}{n^{2}}\right)
\]

где постоянная Ридберга \(\bar{R}\) имеет значение

\[
\bar{R}=R / 2 \pi c=109737,31 \mathrm{~cm}^{-1} .
\]
}тра - серия Лаймана:


\begin{equation*}
\omega=R\left(\frac{1}{1^{2}}-\frac{1}{n^{2}}\right), \quad n=2,3,4, \ldots, \tag{2.14}
\end{equation*}


а в инфракрасной части спектра - серия Пашена:


\begin{equation*}
\omega=R\left(\frac{1}{3^{2}}-\frac{1}{n^{2}}\right), \quad n=4,5,6, \ldots, \tag{2.15}
\end{equation*}


а также серии Брэкета и Пфунда.\\
Все эти серии можно представить в виде обобщенной формуль Бальмера:


\begin{equation*}
\omega=R\left(\frac{1}{n_{0}^{2}}-\frac{1}{n^{2}}\right), \tag{2.16}
\end{equation*}


где \(n_{0}=1\) для серии Лаймана, \(n_{0}=2\) для серии Бальмера и т. д. При заданном \(n_{0}\) число \(n\) принимает все целочисленные значения, начиная с \(n_{0}+1\).

Максимальной длине волны серии Лаймана (2.14) отвечает \(n=2\), это \(\lambda_{\text {макс }}=2 \pi c / \omega_{\text {мин }}=8 \pi c / 3 R=121,6\) нм. Соответствующую спектральную линию называют резонансной линией водорода.

С ростом \(n\) частота линий в каждой серии стремится к предельному значению \(R / n_{0}^{2}\), которое называют границей серии (см. рис. 2.4). За границей серии спектр не обрывается, а становится сплошным. Это присуще не только всем сериям водорода, но и атомам других элементов.

Пример. Найдем спектральный интервал, в пределах которого расположены линии серии Бальмера атомарного водорода (в длинах волн).\\
Границы данного интервала - это головная линия серии, \(\lambda_{32}\), соответствующая \(n=3\) в формуле (2.12), и граница серии, \(\lambda_{\infty}\) ( \(n=\infty\) ). Имея в виду, что частота \(\omega\) связана с длиной волны \(\lambda\) как \(\omega=2 \pi c / \lambda\), получим

\[
\lambda_{32}=\frac{2 \pi c}{R(5 / 36)}=656 \text { нм, } \quad \lambda_{\infty}=\frac{2 \pi c}{R / 4}=365 \text { нм. }
\]

Таким образом, интересующая нас серия заключена в спектральном интервале от 365 до 656 нм, т. е. действительно, все основные линии ее расположены в видимой области спектра.

\section*{§ 2.3. Постулаты Бора. Опыты Франка и Герца}
Постулаты Бора. Абсолютная неустойчивость планетарной модели Резерфорда и вместе с тем удивительная закономерность атомных спектров, и в частности их дискретность, привели Н. Бора к необходимости сформулировать (1913) два важнейших постулата квантовой физики:

\begin{enumerate}
  \item Атом может длительное время* находиться только в определенных, так называемых стационарных состояниях, которые характеризуются дискретными значениями энергии \(E_{1}, E_{2}, E_{3}, \ldots\) В этих состояниях, вопреки классической электродинамике, атом не излучает.
  \item При переходе атома из стационарного состояния с большей энергией \(E_{2}\) в стационарное состояние с меньшей энергией \(E_{1}\) происходит излучение кванта света (фотона) с энергией \(\hbar \omega\) :
\end{enumerate}


\begin{equation*}
\hbar \omega=E_{2}-E_{1} . \tag{2.17}
\end{equation*}


Такое же соотношение выполняется и в случае поглощения, когда падающий фотон переводит атом с низшего энергетического уровня \(E_{1}\) на более высокий \(E_{2}\), а сам исчезает.

Соотношение (2.17) называют правилом частот Бора. Заметим, что переходы атома на более высокие энергетические уровни могут быть обусловлены и столкновением с другими атомами.

Таким образом, атом переходит из одного стационарного состояния в другое скачками (их называют квантовыми). Что происходит с атомом в процессе перехода - этот вопрос в теории Бора остается открытым.

Опыты Франка и Герца (1913). Эти опыты дали прямое доказательство дискретности атомных состояний. Идея опытов заключается в следующем. При неупругих столкновениях электрона с атомом происходит передача энергии от электрона атому. Если внутренняя энергия атома изменяется непрерыв-

\footnotetext{\begin{itemize}
  \item Это время порядка \(10^{-8}\) с. В макроскопическом масштабе такое время крайне мало. Однако в атомной шкале времени оно весьма велико, поскольку на много порядков превосходит период обращения электрона вокруг ядра, например, атома водорода, равный порядка \(10^{-16} \mathrm{c}\).
\end{itemize}
}но, то атому может быть передана любая порция энергии. Если же состояния атома дискретны, то его внутренняя энергия при столкновении с электроном должна изменяться также дискретно - на значения, равные разности внутренней энергии атома в стационарных состояниях.

Следовательно, при неупругом столкновении электрон может передать атому лишь определенные порции энергии. Измеряя их, можно определить значения внутренних энергий стационарных состояний атома.

Это и предстояло проверить экспериментально с помощью установки, схема которой показана на рис. 2.5. В баллоне с парами ртути под давлением порядка 1 мм рт. ст. ( \(\approx 130\) Па) имелись три электрода: \(K\) - катод, \(C\) - сетка и \(A\) - анод. Электроны, испускаемые горячим катодом вследствие термоэлектронной эмиссии, ускорялись разностью потенциалов \(V\) между катодом и сеткой. Величину \(V\)

\begin{figure}[h]
\begin{center}
  \includegraphics[alt={},max width=\textwidth]{a57b3a57-78b5-4292-b0ab-4bd9f2ffcf56-046_210_392_623_768}
\captionsetup{labelformat=empty}
\caption{Рис. 2.5}
\end{center}
\end{figure}

можно было плавно менять. Между сеткой и анодом создавалось слабое тормозящее поле с разностью потенциалов около \(0,5 \mathrm{~B}\).

Таким образом, если какой-то электрон проходит сквозь сетку с энергией, меньшей 0,5 эВ, то он не долетит до анода. Только те электроны, энергия которых при прохождении сетки больше 0,5 эВ, попадут на анод, образуя анодный ток \(I\), доступный измерению.

В опытах (см. рис. 2.5) исследовалась зависимость анодного тока \(I\) (гальванометром \(G\) ) от ускоряющего напряжения \(V\) (вольтметром \(V\) ). Полученные результаты представлены на рис. 2.6. Максимумы соответствуют значениям энергии \(E_{1}=4,9\) эВ, \(E_{2}=2 E_{1}, E_{3}=3 E_{1}\) и т. д.*

\begin{figure}[h]
\begin{center}
  \includegraphics[alt={},max width=\textwidth]{a57b3a57-78b5-4292-b0ab-4bd9f2ffcf56-046_313_418_1206_746}
\captionsetup{labelformat=empty}
\caption{Рис. 2.6}
\end{center}
\end{figure}

\footnotetext{\begin{itemize}
  \item Заметим, что кривая \(I(V)\) на рис. 2.6 имеет такой вид лишь в случае, если отсутствует внешняя контактная разность потенциалов \(\Delta \varphi\) между катодом и сеткой. Наличие же \(\Delta \varphi\) приводит к смещению кривой \(I(V)\) влево или вправо. Относительное же расположение максимумов зависит только от рода газа (атомов). Величину \(\Delta \varphi\) можно исключить, если энергию \(E_{1}\) измерять по «расстоянию» между соседними максимумами, что обычно и делают.
\end{itemize}
}Такой вид кривой объясняется тем, что атомы действительно могут поглощать лишь дискретные порции энергии, равные 4,9 эВ.

При энергии электронов, меньшей 4,9 эВ, их столкновения с атомами ртути могут быть только упругими (без изменения внутренней энергии атомов), и электроны достигают сетки с энергией, достаточной для преодоления тормозящей разности потенциалов между сеткой и анодом. Когда же ускоряющее напряжение \(V\) становится равным \(4,9 \mathrm{~B}\), электроны начинают испытывать вблизи сетки неупругие столкновения, отдавая атомам ртути всю энергию, и уже не смогут преодолеть тормозящую разность потенциалов в пространстве за сеткой. Значит, на анод \(A\) могут попасть только те электроны, которые не испытали неупругого столкновения. Поэтому, начиная с ускоряющего напряжения \(4,9 \mathrm{~B}\), анодный ток \(I\) будет уменьшаться.

При дальнейшем росте ускоряющего напряжения достаточное число электронов после неупругого столкновения успевает приобрести энергию, необходимую для преодоления тормозящего поля за сеткой. Начинается новое возрастание силы тока \(I\). Когда ускоряющее напряжение увеличится до \(9,8 \mathrm{~B}\), электроны после одного неупругого столкновения достигают сетки с энергией 4,9 эВ, достаточной для второго неупругого столкновения. При втором неупругом столкновении электроны теряют всю свою энергию и не достигают анода. Поэтому анодный ток \(I\) начинает опять уменьшаться (второй максимум на рис. 2.6). Аналогично объясняются и последующие максимумы.

Из результатов опытов следует, что разница внутренних энергий основного состояния атома ртути и ближайшего возбужденного состояния равна 4,9 эВ, что и доказывает дискретность внутренней энергии атома.

Аналогичные опыты были проведены в дальнейшем с атомами других газов. И для них были получены характерные разности потенциалов, их называют резонансными потенииалами или первыми потенциалами возбуждения. Резонансный потенциал соответствует переходу атома с основного состояния в ближайшее возбужденное. Для обнаружения более высоких возбужденных состояний была использована более совершенная методика, однако принцип исследования оставался тем же.

Итак, все опыты такого рода приводят к заключению, что состояния атомов изменяются лишь дискретно.

Опыты Франка и Герца подтверждают также и второй постулат Бора - правило частот. Оказывается, что при достижении ускоряющего напряжения \(4,9 \mathrm{~B}\) пары ртути начинают испускать ультрафиолетовое излучение с длиной волны 253,7 нм. Это излучение связано с переходом атомов ртути из первого возбужденного состояния в основное. Действительно, из условия (2.17) следует, что

\[
E_{2}-E_{1}=\frac{2 \pi c \hbar}{\lambda}=\frac{2 \pi \cdot 3 \cdot 10^{10} \cdot 1,054 \cdot 10^{-27}}{253,7 \cdot 10^{-7} \cdot 1,6 \cdot 10^{-12}}=4,9 \text { эВ. }
\]

Этот результат хорошо согласуется с предыдущими измерениями.

\section*{§ 2.4. Боровская модель атома водорода}
Чтобы получить согласие с результатами наблюдений, Бор предположил, что электрон в атоме водорода движется только по тем круговым орбитам*, для которых его момент импульса


\begin{equation*}
M=n \hbar, \quad n=1,2,3, \ldots, \tag{2.18}
\end{equation*}


где \(n\) - квантовые числа.\\
С помощью этого правила квантования можно найти радиусы круговых стационарных орбит водородоподобных систем ( \(\mathrm{H}, \mathrm{He}^{+}\), \(\left.\mathrm{Li}^{++} . ..\right)\)и соответствующие им энергии. Пусть заряд ядра водородоподобной системы равен \(Z e\). Масса ядра значительно больше массы электрона, поэтому ядро при движении электрона можно считать неподвижным. Следуя Бору, будем предполагать, что электрон движется вокруг ядра по окружности радиуса \(r\).

Согласно 2-му закону Ньютона


\begin{equation*}
m \frac{v^{2}}{r}=\frac{Z e^{2}}{r^{2}}, \tag{2.19}
\end{equation*}


\footnotetext{\begin{itemize}
  \item Позднее Зоммерфельд обобщил рассуждения Бора на эллиптические орбиты. Однако в настоящее время это потеряло значение, и мы оставим данный вопрос без внимания.
\end{itemize}
}где \(m\)-масса электрона. Отсюда кинетическая энергия электрона


\begin{equation*}
K=\frac{m v^{2}}{2}=\frac{Z e^{2}}{2 r}, \tag{2.20}
\end{equation*}


и полная энергия электрона в кулоновском поле ядра


\begin{equation*}
E=K+U=\frac{m v^{2}}{2}-\frac{Z e^{2}}{r}=-\frac{Z e^{2}}{2 r} . \tag{2.21}
\end{equation*}


Согласно правилу квантования (2.18), \(r m v=n \hbar\), откуда


\begin{equation*}
v=n \hbar / r m . \tag{2.22}
\end{equation*}


После подстановки (2.22) в (2.19) получим выражение для радиуса \(n\)-й стационарной орбиты:


\begin{equation*}
r_{n}=\frac{\hbar^{2}}{m e^{2}} \frac{n^{2}}{Z} \tag{2.23}
\end{equation*}


Радиус первой стационарной орбиты электрона в атоме водорода ( \(n=1, Z=1\) ) равен


\begin{equation*}
r_{1}=\hbar^{2} / m e^{2}=0,529 \cdot 10^{-8} \mathrm{~cm} \tag{2.24}
\end{equation*}


Его называют боровским радиусом.\\
Энергия \(E_{n}\) электрона на \(n\)-й стационарной орбите определяется формулой (2.21), в которой под \(r\) надо понимать (2.23). И мы приходим к следующему выражению для \(E_{n}\) :


\begin{equation*}
E_{n}=-\frac{m e^{4}}{2 \hbar^{2}} \frac{Z^{2}}{n^{2}} . \tag{2.25}
\end{equation*}


Эта формула описывает уровни энергии стационарных состояний электрона в водородоподобной системе. Для атома водорода схема энергетических уровней, соответствующих (2.25), показана на рис. 2.7. При \(n \rightarrow \infty\) уровни энергии сгущаются к своему предельному значению \(E_{\infty}=0\).

Состояние атома с наименьшей энергией ( \(n=1\) ) называют основным. Для атома водорода основному состоянию соответствует энергия \(E_{1}=-13,53\) эВ. Эта энергия (по модулю) является

энергией связи электрона в основном состоянии: \(E_{\text {св }}=E_{1}\). Именно такую энергию надо сообщить электрону в основном состоянии ( \(n=1\) ), чтобы удалить его из атома водорода. По этой причине ее называют еще и энергией ионизации:

\[
E_{\text {ион }}=E_{\text {св }}=13,6 \text { эВ. }
\]

Это значение, полученное из боровской теории атома, находится в хорошем согласии с результатами эксперимента.

Пример. Найдем скорость электрона на первой боровской орбите атома водорода.\\
Для этого в формулу (2.22) подста-

\begin{figure}[h]
\begin{center}
  \includegraphics[alt={},max width=\textwidth]{a57b3a57-78b5-4292-b0ab-4bd9f2ffcf56-050_592_370_153_790}
\captionsetup{labelformat=empty}
\caption{Рис. 2.7}
\end{center}
\end{figure}

\[
v_{1}=e^{2} / \hbar=2,2 \cdot 10^{6} \mathrm{~m} / \mathrm{c} .
\]

Спектральные серии водородоподобных систем. Согласно второму постулату Бора (2.17), определяющему энергию фотонов при переходе системы из одного стационарного состояния в другое, и формуле (2.24) имеем:


\begin{equation*}
\hbar \omega=E_{2}-E_{1}=\frac{m e^{4} Z^{2}}{2 \hbar^{2}}\left(\frac{1}{n_{1}^{2}}-\frac{1}{n_{2}^{2}}\right) . \tag{2.26}
\end{equation*}


Отсюда частота фотона


\begin{equation*}
\omega=\frac{m e^{4} Z^{2}}{2 \hbar^{3}}\left(\frac{1}{n_{1}^{2}}-\frac{1}{n_{2}^{2}}\right) . \tag{2.27}
\end{equation*}


Таким образом, мы пришли к обобщенной формуле Бальмера (2.16), установив при этом, от каких величин зависит постоянная Ридберга:


\begin{equation*}
R_{\infty}=\frac{m e^{4}}{2 \hbar^{3}} . \tag{2.28}
\end{equation*}


Подстановка в это выражение числовых значений \(m, e\) и \(\hbar\) дает величину, хорошо согласующуюся с экспериментальным значением постоянной Ридберга (2.13). Индекс \(\infty\) при \(R\) в (2.28) означает, что эта величина получена в предположении, что масса ядра весьма велика, и ядро при движении электрона неподвижно.

Учет конечности массы ядра приводит к тому, что массу \(m\) электрона в (2.28) следует заменить на приведенную массу \(\mu\) системы электрон-ядро: \(\mu=m M /(m+M)\), где \(M\) - масса ядра. Тогда


\begin{equation*}
R=\frac{R_{\infty}}{1+m / M} . \tag{2.29}
\end{equation*}


Как видим, постоянная Ридберга зависит и от массы ядра. Для атома водорода, ядром которого является протон, формула (2.29) дает значение, более точно совпадающее с экспериментальным.

Приведенная на рис. 2.7 система энергетических уровней помогает наглядно представить спектральные серии Лаймана, Бальмера и др. как группы переходов между соответствующими уровнями. Эти переходы изображены на рисунке вертикальными стрелками.

Систему энергетических уровней атома принято называть и иначе - системой термов. Терм \(T\) - это величина, определяемая согласно (2.16) и (2.25) как


\begin{equation*}
T_{n}=R / n^{2}=\left|E_{n}\right| / \hbar, \tag{2.30}
\end{equation*}


где \(R\) - постоянная Ридберга. В отличие от энергии \(E_{n}\), терм величина положительная, и чем ниже уровень, тем больше его значение. Терм имеет ту же размерность, что и частота \(\omega\), т. е. \(\mathrm{c}^{-1}\). Соответствующая частота фотона, испущенного при переходе атома из состояния с квантовым числом \(n_{1}\) в состояние с квантовым числом \(n_{2}\), определяется формулой


\begin{equation*}
\omega_{12}=T_{2}-T_{1}=\frac{R}{n_{2}^{2}}-\frac{R}{n_{1}^{2}} . \tag{2.31}
\end{equation*}


Формулы (2.15) и (2.28) позволяют записать выражение для энергии связи (энергии ионизации) водородоподобной системы в основном состоянии в более удобном виде:


\begin{equation*}
E_{\text {св }}=E_{\text {ион }}=\hbar R Z^{2} . \tag{2.32}
\end{equation*}


Пример. Найдем энергию связи электрона в основном состоянии водородоподобных ионов, в спектре испускания которых длина волны третьей линии серии Бальмера \(\lambda_{3}=108,5\) нм.

Искомая энергия определяется формулой (2.31). В данном случае \(Z\) неизвестно. Для его нахождения воспользуемся тем, что частота третьей линии серии Бальмера


\begin{equation*}
\omega_{3}=R Z^{2}\left(\frac{1}{2^{2}}-\frac{1}{(2+3)^{2}}\right)=\frac{21}{100} R Z^{2} . \tag{1}
\end{equation*}


Частота \(\omega_{3}\) связана с длиной волны \(\lambda_{3}\) формулой \(\omega_{3}=2 \pi c / \lambda_{3}\). Поэтому из (1) следует, что


\begin{equation*}
R Z^{2}=\frac{100}{21} \frac{2 \pi c}{\lambda_{3}} . \tag{2}
\end{equation*}


Таким образом, искомая энергия связи

\[
E_{\text {св }}=\frac{100}{21} \frac{\pi c \hbar}{\lambda_{3}}=54,4 \text { эВ. }
\]

Попутно из формулы (2) можно найти, что \(Z=2\), т.е. мы имеем дело с ионом \(\mathrm{He}^{+}\).

Магнитный момент атома водорода. Пусть электрон движется со скоростью \(v\) по орбите радиусом \(r\) (рис. 2.8). Через площадку, пересекающую орбиту электрона, переносится ежесекундно заряд \(e v\), где \(e\) - заряд электрона, \(v\) - число оборотов электрона вокруг ядра в секунду. Следовательно, движущийся по орбите электрон образует круговой ток \(I=e v\). Поскольку заряд электрона отрицателен, направление движения электрона противоположно направлению тока.

\begin{figure}[h]
\begin{center}
  \includegraphics[alt={},max width=\textwidth]{a57b3a57-78b5-4292-b0ab-4bd9f2ffcf56-052_265_219_1161_939}
\captionsetup{labelformat=empty}
\caption{Рис. 2.8}
\end{center}
\end{figure}

Магнитный момент такого тока (в гауссовой системе) по определению равен \(\mu=I S / c\), или \(\mu=e v \cdot \pi r^{2} / c\). Учитывая, что \(2 \pi r v=v\), перепишем предыдущее выражение в виде

\[
\mu=\frac{e r v}{2 c} .
\]

Остается учесть, что момент импульса электрона \(M=r m v\), и мы получим:


\begin{equation*}
\boldsymbol{\mu}=-\frac{e}{2 m c} \mathbf{M}, \tag{2.33}
\end{equation*}


где знак минус указывает, что направления обоих моментов, \(\boldsymbol{\mu}\) и \(\mathbf{M}\), взаимно противоположны.

Вектор М называют орбитальным моментом электрона. Он образует с направлением движения электрона правовинтовую систему (см. рис. 2.8).

Отношение магнитного момента частицы к ее механическому моменту называют гиромагнитным отношением. Для электрона оно равно


\begin{equation*}
\frac{\mu}{M}=\frac{e}{2 m c} . \tag{2.34}
\end{equation*}


Воспользовавшись боровским правилом квантования момента импульса, т. е. формулой (2.18), перепишем (2.33) в виде


\begin{equation*}
\mu=\mu_{\mathrm{B}} n, \quad n=1,2,3, \ldots, \tag{2.35}
\end{equation*}


где \(\mu_{\mathrm{E}}\) - это так называемый магнетон Бора:


\begin{equation*}
\mu_{\mathrm{b}}=\frac{e \hbar}{2 m c}=0,927 \cdot 10^{-20} \text { эрг } / \text { Гс. } \tag{2.36}
\end{equation*}


Таким образом, при движении электрона по первой боровской орбите ( \(n=1\) ) его магнитный момент равен одному магнетону Бора. В дальнейшем мы увидим, что это резко расходится с экспериментом, значит, полученный результат оказывается совершенно неверным. И тем не менее, мы привели формулы, связывающие магнитный момент с механическим, поскольку они послужат основой для получения правильных результатов (см. § 7.1).

Недостатки теории Бора. Теория Бора явилась крупным шагом в развитии теории атома, в понимании новых квантовых закономерностей, с которыми столкнулась физика при изучении явлений микромира. Эта теория отчетливо показала неприменимость законов классической физики для описания внутриатомных явлений.

Теория Бора стимулировала постановку многих экспериментов, принесших важные результаты. Даже в тех многочисленных случаях, когда теория не могла дать количественное объяснение явлений, два постулата Бора служили руководящей нитью при классификации и количественной интерпретации этих явлений.

Однако двух постулатов Бора недостаточно для построения полной теории. Они должны быть дополнены правилами квантования. Эти правила, достаточно искусственно введенные Бором для одноэлектронного атома, радикально проблемы не решили. Их не удалось распространить даже на простейший после водорода атом гелия, содержащий два электрона. Кроме того, теория Бора позволила вычислить только частоты спектральных линий, но не их интенсивность.

Основной же, принципиальный недостаток теории Бора это ее непоследовательность: она не была ни последовательно классической, ни последовательно квантовой. Эта теория принимала существование стационарных состояний атома, что совершенно непонятно с точки зрения классической физики. И вместе с тем к движению электронов в стационарных состояниях она применяла законы классической механики, хотя и считала неприменимой классическую электродинамику (поскольку нет излучения).

Итак, планетарную модель атома нельзя считать серьезной теорией. Она просто неверна. Тот факт, что эта модель приводит к очень хорошим результатам в случае атома водорода (при расчете некоторых величин), по существу случайный. Этот успех явился мощным толчком к развитию квантовой теории атома. Сам Бор рассматривал свою теорию как промежуточный этап в поисках верной теория. Такой последовательной теорией явилась квантовая физика.

\section*{Задачи}
2.1. Лобовое столкновение. На какое минимальное расстояние приблизится \(\alpha\)-частица с кинетической энергией \(K_{\alpha}\) к первоначально покоившемуся ядру \({ }^{7} \mathrm{Li}\) при лобовом столкновении?

Решение. Система \(\alpha\)-частица - ядро предполагается замкнутой, поэтому в процессе сближения будут сохраняться как ее им-

пульс, так и собственная механическая энергия. Отсюда для двух состояний - когда \(\alpha\)-частица далеко от ядра и в момент максимального сближения (система движется как единое целое), - можно записать:


\begin{equation*}
p_{\alpha}=p_{\alpha+\mathrm{Li}}, \quad K_{\alpha}=K_{\alpha+\mathrm{Li}}+\frac{q q_{0}}{r_{\text {мин }}} \tag{1}
\end{equation*}


где \(q\) и \(q_{0}\) - заряды \(\alpha\)-частицы и ядра атома лития. Имея в виду, что \(K=p^{2} / 2 m\), перепишем первое равенство в (1) через \(K\) :


\begin{equation*}
m_{\alpha} K_{\alpha}=\left(m_{\alpha}+m_{\mathrm{Li}}\right) K_{\alpha+\mathrm{Li}} . \tag{2}
\end{equation*}


Из последнего равенства находим \(K_{\alpha+\mathrm{Li}}\) и полученное выражение подставим во второе уравнение из (1). В результате:

\[
r_{\mathrm{MuH}}=\frac{q q_{0}}{K_{\alpha}}\left(1+\frac{m_{\alpha}}{m_{\mathrm{Li}}}\right) .
\]

2.2. Нелобовое столкновение. Альфа-частица с кинетической энергией \(K\) рассеялась под углом \(\theta\) на кулоновском поле неподвижного тяжелого ядра с зарядом \(Z e\). Найти минимальное расстояние, на которое она сблизилась с ядром в процессе движения.

Решение. Воспользуемся законами сохранения. Импульс \(\alpha\)-частицы не сохраняется, поскольку на нее все время действует кулоновская сила. Вместе с тем направление этой силы проходит через центр ядра, поэтому ее плечо относительно этого центра равно нулю, а значит равен нулю и момент силы. Отсюда следует, что момент импульса \(\alpha\)-частицы относительно центра ядра сохраняется. Вдали от ядра он был равен \(b m v\), где \(b\) и \(v\) - прицельный параметр и скорость \(\alpha\)-частицы вдали от ядра.

При наибольшем сближении с ядром скорость \(\mathbf{v} \alpha\)-частицы будет перпендикулярна ее радиусу-вектору относительно центра ядра, и момент импульса \(\alpha\)-частицы в этом положении будет равен \(r_{\text {мин }} \cdot m v^{\prime}\), где \(v^{\prime}\) - ее скорость. Таким образом, из сохранения момента импульса \(\alpha\)-частицы имеем:


\begin{equation*}
b v=r_{\text {мину }} v^{\prime} . \tag{1}
\end{equation*}


Поскольку \(\alpha\)-частица движется в кулоновском поле и сторонних сил нет, ее полная энергия в этом поле должна также сохраняться:


\begin{equation*}
K=K^{\prime}+q q_{0} / r_{\text {мин }} . \tag{2}
\end{equation*}


Здесь слева записана энергия \(\alpha\)-частицы вдали от ядра, а справа - при максимальном сближении с ядром.

Из формул (1) и (2) приходим к квадратному уравнению относительно \(r_{\text {мин }}\) :


\begin{equation*}
K r_{\text {мин }}^{2}-q q_{0} r_{\text {мин }}-b^{2} K=0 . \tag{3}
\end{equation*}


Решение этого уравнения:


\begin{equation*}
r_{\text {мин }}=\frac{q q_{0}+\sqrt{\left(q q_{0}\right)^{2}+4 b^{2} K^{2}}}{2 K}=\frac{q q_{0}}{2 K}\left(1+\sqrt{1+\operatorname{ctg}^{2} \theta}\right), \tag{4}
\end{equation*}


где учтено, что согласно (2.1) \(q q_{0} / 2 b K=\operatorname{tg}(\theta / 2)\). В окончательном виде (4) запишем так:

\[
r_{\mathrm{MuH}}=\frac{Z e^{2}}{K}\left(1+\csc \frac{\theta}{2}\right)
\]

2.3. Формула Резерфорда. Узкий пучок протонов, скорость которых \(v=6 \cdot 10^{6} \mathrm{~m} / \mathrm{c}\), падает нормально на серебряную ( \(Z=47\) ) фольгу толщиной \(d=1,0\) мкм. Найти вероятность рассеяния протонов в заднюю полусферу ( \(\theta>90^{\circ}\) ). Плотность серебра \(\rho=10,5 г / \mathrm{cm}^{3}\). Решение. Искомая вероятность \(P\) равна относительному числу протонов, рассеянных в заднюю полусферу:


\begin{equation*}
P=\Delta N / N=n \sigma, \tag{1}
\end{equation*}


где правая часть этой формулы записана согласно (2.9), причем \(\sigma\) - эффективное сечение, соответствующее рассеянию под углами \(\theta>90^{\circ}\). Это сечение \(\sigma=\pi b_{0}^{2}\), где \(b_{0}\) - прицельный параметр, при котором \(\theta=\theta_{0}=90^{\circ}\). Ясно, что все протоны с прицельным параметром, меньшим \(b_{0}\), рассеятся под углами \(\theta>\theta_{0}\). Используя формулу (2.1), получим:


\begin{equation*}
b_{0}=\frac{Z e^{2}}{m v^{2} \operatorname{tg}(\theta / 2)}=\frac{Z e^{2}}{m v^{2}} . \tag{2}
\end{equation*}


Теперь найдем выражение для \(\sigma\) как \(\pi b_{0}^{2}\) и учтем, что число ядер на единицу поверхности фольги \(n=n_{0} d\), где \(n_{0}\) - концентрация ядер (их число в единице объема). После подстановки полученных выражений в (1) находим, что

\[
P=\pi n_{0} d Z e^{2} / m v^{2}=0,006 .
\]

Здесь \(n_{0}=N_{A} \rho d / M, N_{A}\) - постоянная Авогадро, \(M\) - молярная масса серебра.\\
2.4. Узкий пучок \(\alpha\)-частиц с кинетической энергией \(K=0,60\) МэВ падает на золотую фольгу, содержащую \(n=1,1 \cdot 10^{19}\) ядер \(/ \mathrm{cm}^{2}\). Найти относительное число \(\alpha\)-частиц, рассеивающихся под углами \(\theta<\theta_{0}\), где \(\theta_{0}=20^{\circ}\).

Решение. Непосредственно использовать формулу Резерфорда для этого интервала углов мы не можем, поскольку для углов, меньших порядка \(3^{\circ}\) она, как было сказано ранее, несправедлива. Поэтому искомую величину представим так:


\begin{equation*}
\frac{\Delta N}{N}=1-n \sigma\left(\theta_{0}\right)=1-n \pi b_{0}^{2}, \tag{1}
\end{equation*}


где \(b_{0}\) - прицельный параметр, соответствующий углу рассеяния \(\theta_{0}\). Величину \(b_{0}\) находим с помощью формулы (2.1):


\begin{equation*}
b_{0}=\frac{Z e^{2}}{K \operatorname{tg}\left(\theta_{0} / 2\right)} \tag{2}
\end{equation*}


Подстановка (2) в (1) дает

\[
\frac{\Delta N}{N}=1-\frac{\pi n Z e^{4}}{K^{2} \operatorname{tg}^{2}\left(\theta_{0} / 2\right)}=0,6
\]

2.5. Классическое время жизни атома. Оценить промежуток времени \(\tau\), за который электрон, движущийся вокруг ядра атома водорода (протона) по окружности радиуса \(r_{0}=0,53 \cdot 10^{-8} \mathrm{~cm}\), упал бы на ядро из-за потери энергии на излучение.\\
Решение. Для простоты будем считать, что в любой момент падения на ядро электрон движется равномерно по окружности. Тогда, согласно 2 -му закону Ньютона, \(m v^{2} / r=e^{2} / r^{2}\), откуда кинетическая энергия


\begin{equation*}
K=m v^{2} / 2=e^{2} / 2 r, \tag{1}
\end{equation*}


и полная энергия электрона в поле ядра


\begin{equation*}
E=K+U=\frac{m v^{2}}{2}-\frac{e^{2}}{r}=-\frac{e^{2}}{2 r} . \tag{2}
\end{equation*}


В соответствии с классической электродинамикой, потеря энергии заряженной частицы на излучение в единицу времени опреде-

ляется формулой


\begin{equation*}
-\frac{\mathrm{d} E}{\mathrm{~d} t}=\frac{2 e^{2}}{3 c^{3}} \mathbf{a}^{2} \tag{3}
\end{equation*}


Учитывая (1) и (2), преобразуем (3) к виду

\[
-\frac{e^{2}}{2 r^{2}} \frac{\mathrm{~d} r}{\mathrm{~d} t}=\frac{2 e^{2}}{3 c^{3}}\left(\frac{e^{2}}{m r^{2}}\right)^{2}
\]

Разделив переменные \(r\) и \(t\), получим

\[
-r^{2} \mathrm{~d} r=\frac{4}{3} \frac{e^{4}}{m^{2} c^{3}} \mathrm{~d} t
\]

Остается проинтегрировать это уравнение по \(r\) от \(r_{0}\) до 0 и по \(t\) от 0 до \(\tau\). В результате получим:

\[
\tau=\frac{m^{2} c^{3} r_{0}{ }^{3}}{4 e^{4}}=\frac{\left(0,911 \cdot 10^{-27}\right)^{2}\left(3 \cdot 10^{10}\right)^{3}\left(0,53 \cdot 10^{-8}\right)^{3}}{4\left(4,8 \cdot 10^{-10}\right)^{4}}=1,3 \cdot 10^{-11} \mathrm{c}(!)
\]

2.6. Квантование. Частица массы \(m\) движется по круговой орбите в центрально-симметричном поле, где ее потенциальная энергия зависит от расстояния \(r\) до центра поля как \(U=\varkappa r^{2} / 2, \varkappa\) - постоянная. Найти с помощью боровского условия квантования возможные радиусы орбит и значения полной энергии частицы в данном поле.\\
Решение. Исходим из 2 -го закона Ньютона:


\begin{equation*}
m \frac{v^{2}}{r}=\frac{\partial U}{\partial r}=\varkappa r, \tag{1}
\end{equation*}


где справа написана проекция силы на нормаль \(\mathbf{n}\) к траектории. Согласно правилу квантования (2.18) имеем:


\begin{equation*}
r m v=\hbar n, \quad n=1,2, \ldots \tag{2}
\end{equation*}


Из этих двух уравнений находим возможные значения \(r\) :


\begin{equation*}
r_{n}=\sqrt{n \hbar / \sqrt{\varkappa m}} . \tag{3}
\end{equation*}


Возможные значения полной энергии

\[
E_{n}=\frac{m v^{2}}{2}+\frac{x r^{2}}{2}=n \hbar \sqrt{\varkappa / m},
\]

где приняты во внимание формулы (2) и (3).\\
2.7. Атом водорода. Покоившийся атом водорода испустил фотон, соответствующий головной линии серии Лаймана. Найти:\\
a) скорость отдачи, которую получил атом;\\
б) отношение кинетической энергии атома отдачи к энергии испущенного фотона.

Решение. а) В этом процессе атом приобрел импульс \(p\), равный импульсу вылетевшего из него фотона:


\begin{equation*}
p=\hbar \omega / c . \tag{*}
\end{equation*}


Кроме того, энергия возбуждения \(E^{*}\) атома распределилась между энергией фотона и кинетической энергией атома, испытавшего отдачу:

\[
E^{*}=\hbar \omega+p^{2} / 2 m,
\]

где \(\quad E^{*}=\hbar R\left(1-1 / 2^{2}\right)=(3 / 4) \hbar R\).\\
Из этих трех формул находим

\[
\frac{m}{2} v^{2}+m c v-\frac{3}{4} \hbar R=0,
\]

откуда следует, что скорость отдачи атома

\[
v=\frac{3}{4} \frac{\hbar R}{m c}=3,27 \mathrm{~m} / \mathrm{c},
\]

здесь \(m\) - масса атома.\\
б) Искомое отношение с учетом (*) равно

\[
\frac{K}{\hbar \omega}=\frac{p^{2} / 2 m}{p c}=\frac{p}{2 m c}=\frac{v}{2 c}=0,55 \cdot 10^{-8},
\]

т. е. оказывается величиной чрезвычайно малой, и поэтому энергией отдачи атома, как правило, пренебрегают.\\
2.8. Водородоподобная система. При каком наименьшем значении приращения внутренней энергии иона \(\mathrm{He}^{+}\), находящегося в основном состоянии, он смог бы испустить фотон, соответствующий головной линии серии Бальмера?\\
Р е ш е н и е. Из рис. 2.7 следует, что для этого ион необходимо возбудить на уровень с \(n=3\). Именно в этом случае может быть испущен указанный фотон (при переходе с \(n=3\) на \(n^{\prime}=2\) ).

Искомое приращение внутренней энергии согласно (2.25) и (2.27) равно

\[
E_{\text {мин }}=\hbar \omega_{13}=\hbar R Z^{2}\left(\frac{1}{1^{2}}-\frac{1}{3^{2}}\right)=\frac{8}{9} \hbar R Z^{2}=48,5 \text { эВ. }
\]

2.9. У какого водородоподобного иона разность длин волн между головными линиями серий Бальмера и Лаймана \(\Delta \lambda=59,3\) нм?

Решение. Запишем выражение для частот \(\omega\) этих линий. Согласно (2.26) и (2.27) имеем:

\[
\omega_{\mathrm{b}}=(5 / 36) R Z^{2}, \quad \omega_{\mathrm{J}}=(3 / 4) R Z^{2} .
\]

Из этих формул находим

\[
\Delta \lambda=\lambda_{\mathrm{S}}-\lambda_{\mathrm{J}}=2 \pi c\left(\frac{1}{\omega_{\mathrm{B}}}-\frac{1}{\omega_{\mathrm{J}}}\right)=\frac{176}{15} \frac{\pi c}{R Z^{2}},
\]

откуда

\[
Z=\sqrt{\frac{176}{15} \frac{\pi c}{R \Delta \lambda}}=3 .
\]

Это двукратно ионизированный атом лития, \(\mathrm{Li}^{++}\).\\
2.10. Энергия связи электрона в атоме Не равна \(E_{0}=24,6\) эВ. Найти минимальную энергию, необходимую для последовательного удаления обоих электронов из этого атома.\\
P е шен и е. На первый взгляд кажется, что это \(2 E_{0}\). Но это не так. После удаления первого электрона оставшийся оказывается в кулоновском поле ядра, а значит его энергия связи станет больше, и потребуется бо́льшая энергия для удаления второго электрона. Таким образом, искомая энергия

\[
E_{\text {мин }}=E_{0}+\hbar R Z^{2}=24,6+54,5=79 \text { эВ. }
\]

Здесь величина \(\hbar R Z^{2}\) - это энергия связи электрона в основном состоянии иона \(\mathrm{He}^{+}\).

\section*{Глава 3}
\section*{Волновые свойства частиц}
\section*{§ 3.1. Гипотеза де-Бройля}
Луи де-Бройль (1923) высказал и развил идею о том, что материальные частицы должны обладать и волновыми свойствами. К тому времени уже сложилась парадоксальная, но подтвержденная опытом, ситуация о свете: в одних явлениях (интерференция, дифракция и др.) свет проявляет себя как волны, в других явлениях с не меньшей убедительностью - как частицы. Это и побудило де-Бройля распространить подобный корпу-скулярно-волновой дуализм на частицы с массой покоя, отличной от нуля.

Если с такой частицей связана какая-то волна, можно ожидать, что она распространяется в направлении скорости \(\mathbf{v}\) частицы. О природе этой волны ничего определенного де-Бройлем не было высказано. Не будем и мы пока выяснять их природу, хотя сразу же подчеркнем, что эти волны не электромагнитные. Они имеют, как мы увидим далее, специфическую природу, для которой нет аналога в классической физике.

Итак, де-Бройль высказал гипотезу, что соотношение (1.12), относящееся к фотонам, имеет универсальный характер. Т. е. для всех частиц длина волны


\begin{equation*}
\lambda=\frac{h}{p}=\frac{2 \pi \hbar}{p} . \tag{3.1}
\end{equation*}


Эта формула получила название формуль Әe-Бройля, а \(\lambda\) - дебройлевской длинь волны частицы с импульсом \(p\).

Де-Бройль также предположил, что пучок частиц, падающих на двойную щель, должен за ними интерферировать.

Вторым, независимым от формулы (3.1), соотношением является связь между энергией \(E\) частицы и частотой \(\omega\) дебройлевской волны:


\begin{equation*}
E=\hbar \omega . \tag{3.2}
\end{equation*}


В принципе энергия \(E\) определена всегда с точностью до прибавления произвольной постоянной (в отличие от \(\Delta E\) ), следовательно частота \(\omega\) является принципиально ненаблюдаемой величиной (в отличие от дебройлевской длины волны).

С частотой \(\omega\) и волновым числом \(k\) связаны две скорости фазовая \(v_{\text {ф }}\) и групповая \(u\) :


\begin{equation*}
v_{\Phi}=\frac{\omega}{k} \quad \text { и } \quad u=\frac{\partial \omega}{\partial k} . \tag{3.3}
\end{equation*}


Умножив числитель и знаменатель обоих выражений на ћ, получим:


\begin{equation*}
\hbar \omega=E \quad \text { и } \quad \hbar k=2 \pi \hbar / \lambda=p, \tag{3.4}
\end{equation*}


где второе равенство написано на основании (3.1).\\
Ограничимся рассмотрением только нерелятивистского случая. Полагая \(E=p^{2} / 2 m\) (кинетическая энергия), перепишем соотношения (3.3) с помощью (3.4) в иной форме:


\begin{equation*}
v_{\phi}=\frac{E}{p}, \quad u=\frac{\partial}{\partial p}\left(\frac{p^{2}}{2 m}\right)=\frac{p}{m}=v . \tag{3.5}
\end{equation*}


Отсюда видно, что групповая скорость равна скорости частицы, т. е. является принципиально наблюдаемой величиной, в отличие от \(v_{\phi}\) - из-за неоднозначности \(E\).

Из первой формулы (3.5) следует, что фазовая скорость дебройлевских волн


\begin{equation*}
v_{\phi}=\frac{E}{\sqrt{2 m E}}=\sqrt{\frac{E}{2 m}}=\sqrt{\frac{\hbar \omega}{2 m}}, \tag{3.6}
\end{equation*}


т. е. зависит от частоты \(\omega\), а значит дебройлевские волны обладают дисперсией даже в вакууме. Далее будет показано, что в соответствии с современной физической интерпретацией фазовая скорость дебройлевских волн имеет чисто символическое значение, поскольку эта интерпретация относит их к числу принципиально ненаблюдаемых величин. Впрочем, сказанное видно и сразу, так как \(E\) в (3.6) определена, как уже говорилось, с точностью до прибавления произвольной постоянной.

Установление того факта, что согласно (3.5) групповая скорость дебройлевских волн равна скорости частицы, сыграло в свое время важную роль в развитии принципиальных основ квантовой физики, и в первую очередь в физической интерпретации дебройлевских волн. Сначала была сделана попытка рассматривать частицы как волновые пакеты весьма малой протяженности и таким образом решить парадокс двойственности свойств частиц. Однако подобная интерпретация оказалась ошибочной, так как все составляющие пакет гармонические волны распространяются с разными фазовыми скоростями. При наличии большой дисперсии, свойственной дебройлевским волнам даже в вакууме, волновой пакет «расплывается». Для частиц с массой порядка массы электрона пакет расплывается практически мгновенно, в то время как частица является стабильным образованием.

Таким образом, представление частицы в виде волнового пакета оказалось несостоятельным. Проблема двойственности свойств частиц требовала иного подхода к своему решению.

Вернемся к гипотезе де-Бройля. Выясним, в каких явлениях могут проявиться волновые свойства частиц, если они, эти свойства, действительно существуют. Мы знаем, что независимо от физической природы волн - это интерференция и дифракция. Непосредственно наблюдаемой величиной в них является длина волны. Во всех случаях дебройлевская длина волны определяется формулой (3.1). Проведем с помощью нее некоторые оценки.

Прежде всего убедимся, что гипотеза де-Бройля не противоречит понятиям макроскопической физики. Возьмем в качестве макроскопического объекта, например, пылинку, считая, что ее масса \(m=1\) мг и скорость \(v=1 \mathrm{~m}\) м/с Соответствующая ей дебройлевская длина волны


\begin{equation*}
\lambda=\frac{2 \pi \hbar}{\sqrt{2 m K}}=7 \cdot 10^{-20} \mathrm{~cm} . \tag{3.7}
\end{equation*}


Т. е. даже у такого небольшого макроскопического объекта как пылинка дебройлевская длина волны оказывается неизмеримо меньше размеров самого объекта. В таких условиях никакие волновые свойства, конечно, проявить себя не могут.

Иначе обстоит дело, например, у электрона с кинетической энергией \(K\) и импульсом \(p=\sqrt{2 m K}\). Его дебройлевская длина волны


\begin{equation*}
\lambda=\frac{2 \pi \hbar}{\sqrt{2 m K}}=\frac{1,22}{\sqrt{K}} \mathrm{HM}, \tag{3.8}
\end{equation*}


где \(K\) в эВ. При \(K=150\) эВ дебройлевская длина волны электрона равна согласно (3.8) \(\lambda \approx 0,1\) нм или \(\approx 1\) Е. Такой же порядок величины имеет постоянная кристаллической решетки. Поэтому, аналогично тому, как в случае рентгеновских лучей, кристаллическая структура может быть подходящей решеткой для получения дифракции дебройлевских волн электронов. Однако гипотеза де-Бройля представлялась настолько нереальной, что довольно долго не подвергалась экспериментальной проверке.

\section*{§ 3.2. Экспериментальные подтверждения гипотезы де-Бройля}
Опыты Дэвиссона и Джермера (1927). Идея их опытов заключалась в следующем. Если пучок электронов обладает волновыми свойствами, то можно ожидать, даже не зная механизма отражения этих волн, что их отражение от кристалла будет иметь такой же интерференционный характер, как у рентгеновских лучей.

В одной серии опытов Дэвиссона и Джермера для обнаружения дифракционных максимумов (если таковые есть) измерялись ускоряющее напряжение электронов и одновременно положение детектора \(D\) (счетчика отраженных электронов). В опыте использовался монокристалл никеля (кубической системы), сошлифованный так, как показано на рис. 3.1.

Если его повернуть вокруг вертикальной оси в положение, соответствующее рисунку, то в этом положении сошлифованная поверхность покрыта правильными рядами атомов, перпендикулярными к плоскости падения (плоскости рисунка), расстояние между которыми \(d=0,215\) нм.

\begin{figure}[h]
\begin{center}
  \includegraphics[alt={},max width=\textwidth]{a57b3a57-78b5-4292-b0ab-4bd9f2ffcf56-064_485_336_1276_824}
\captionsetup{labelformat=empty}
\caption{Рис. 3.1}
\end{center}
\end{figure}

\begin{figure}[h]
\begin{center}
  \includegraphics[alt={},max width=\textwidth]{a57b3a57-78b5-4292-b0ab-4bd9f2ffcf56-065_273_312_141_70}
\captionsetup{labelformat=empty}
\caption{Рис. 3.2}
\end{center}
\end{figure}

Детектор перемещали в плоскости падения, меняя угол \(\theta\). При угле \(\theta=50^{\circ}\) и ускоряющем напряжении \(V=54 \mathrm{~B}\) наблюдался особенно отчётливый максимум отраженных электронов, полярная диаграмма которого показана на рис. 3.2.

Этот максимум можно истолковать как интерференционный максимум первого порядка от плоской дифракционной решетки с указанным выше периодом в соответствии с формулой


\begin{equation*}
d \sin \theta=\lambda, \tag{3.9}
\end{equation*}


\begin{figure}[h]
\begin{center}
  \includegraphics[alt={},max width=\textwidth]{a57b3a57-78b5-4292-b0ab-4bd9f2ffcf56-065_383_429_697_68}
\captionsetup{labelformat=empty}
\caption{Рис. 3.3}
\end{center}
\end{figure}

что видно из рис. 3.3. На этом рисунке каждая жирная точка представляет собой проекцию цепочки атомов, расположенных на прямой, перпендикулярной плоскости рисунка. Период \(d\) может быть измерен независимо, например, по дифракции рентгеновских лучей.

Вычисленная по формуле (3.8) дебройлевская длина волны для \(V=54\) В равна 0,167 нм. Соответствующая же длина волны, найденная из формулы (3.9), равна 0,165 нм. Совпадение настолько хорошее, что полученный результат следует признать убедительным подтверждением гипотезы де-Бройля.

Другая серия опытов Дэвиссона и Джермера состояла в измерении интенсивности \(I\) отраженного электронного пучка при заданном угле падения, но при различных значениях ускоряющего напряжения \(V\).

Теоретически должны появиться при этом интерференционные максимумы отражения подобно отражению рентгеновских лучей от кристалла. От различных кристаллических плоскостей кристалла в результате дифракции падающего излучения на атомах исходят волны, как бы испытавшие зеркальное отражение от этих плоскостей. Данные волны при интерференции уси-

ливают друг друга, если выполняется условие Брэгга-Вульфа:


\begin{equation*}
2 d \sin \alpha=m \lambda, \quad m=1,2,3, \ldots, \tag{3.10}
\end{equation*}


где \(d\) - межплоскостное расстояние, \(\alpha\) - угол скольжения.\\
Напомним вывод этой формулы. Из рис. 3.4 видно, что разность хода двух волн, 1 и 2 , отразившихся зеркально от соседних атомных слоев, \(A B C=2 d \sin \alpha\). Следовательно, направления, в которых возникают интерференционные максимумы, определяются условием (3.10).

Теперь подставим в формулу (3.10) выра-

\begin{figure}[h]
\begin{center}
  \includegraphics[alt={},max width=\textwidth]{a57b3a57-78b5-4292-b0ab-4bd9f2ffcf56-066_249_313_367_851}
\captionsetup{labelformat=empty}
\caption{Рис. 3.4}
\end{center}
\end{figure}

жение (3.8) для дебройлевской длины волны. Поскольку значения \(\alpha\) и \(d\) экспериментаторы оставляли неизменными, то из формулы (3.10) следует, что


\begin{equation*}
\sqrt{V_{m}} \propto m, \tag{3.11}
\end{equation*}


т. е. значения \(\sqrt{V_{m}}\), при которых образуются максимумы отражения, должны быть пропорциональны целым числам \(m=1,2\), 3 , ..., другими словами, находиться на одинаковых расстояниях друг от друга.

Это и было проверено на опыте, результаты которого представлены на рис. 3.5 , где \(V\), В. Видно, что максимумы интенсивности \(I\) почти равноудалены друг от друга (такая же картина возникает и при дифракции рентгеновских лучей от кристаллов).

\begin{figure}[h]
\begin{center}
  \includegraphics[alt={},max width=\textwidth]{a57b3a57-78b5-4292-b0ab-4bd9f2ffcf56-066_390_754_1375_237}
\captionsetup{labelformat=empty}
\caption{Рис. 3.5}
\end{center}
\end{figure}

Полученные Дэвиссоном и Джермером результаты весьма убедительно подтверждают гипотезу де-Бройля. Заметим также, что в теоретическом отношении, как мы видели, анализ дифракции дебройлевских волн полностью совпадает с дифракцией рентгеновского излучения.

Итак, характер зависимости (3.11) экспериментально подтвердился, однако наблюдалось некоторое расхождение с предсказаниями теории. А именно, между положениями экспериментальных и теоретических максимумов (последние показаны стрелками на рис. 3.5) наблюдается систематическое расхождение, которое уменьшается с увеличением ускоряющего напряжения \(V\). Это расхождение, как выяснилось в дальнейшем, обусловлено тем, что при выводе формулы Брэгга-Вульфа не было учтено преломление дебройлевских волн.

О преломлении дебройлевских волн. Показатель преломления \(n\) дебройлевских волн, как и электромагнитных, определяется формулой


\begin{equation*}
n=v_{\mathrm{B}} / v_{\mathrm{c}}, \tag{3.12}
\end{equation*}


где \(v_{\text {в }}\) и \(v_{\text {с }}\) - фазовые скорости этих волн в вакууме и среде (кристалле). Выше (стр. 61) было отмечено, что фазовая скорость дебройлевокой волны - принципиально ненаблюдаемая величина. Поэтому формулу (3.12) следует преобразовать так, чтобы показатель преломления \(n\) можно было выразить через отношение измеряемых величин.

Это можно сделать следующим образом. По определению, фазовая скорость


\begin{equation*}
v=\omega / k, \tag{3.13}
\end{equation*}


где \(k\) - волновое число ( \(2 \pi / \lambda\) ). Считая аналогично фотонам, что частота \(\omega\) дебройлевских волн тоже не меняется при переходе границы раздела сред (если такое предположение несправедливо, то опыт неизбежно укажет на это), представим (3.12) с учетом (3.13) в виде


\begin{equation*}
n=\frac{k_{\mathrm{c}}}{k_{\mathrm{B}}}=\frac{\lambda_{\mathrm{B}}}{\lambda_{\mathrm{c}}} . \tag{3.14}
\end{equation*}


Попадая из вакуума в кристалл (металл), электроны оказываются в потенциальной яме. Здесь их кинетическая энергия \(K\) возрастает на «глубину» потенциальной ямы (рис. 3.6). Из формулы (3.8), где \(K=e V\), следует, что \(\lambda\) c \(1 / \sqrt{V}\). Поэтому выражение (3.14) можно переписать так:

\begin{figure}[h]
\begin{center}
  \includegraphics[alt={},max width=\textwidth]{a57b3a57-78b5-4292-b0ab-4bd9f2ffcf56-068_189_346_203_814}
\captionsetup{labelformat=empty}
\caption{Рис. 3.6}
\end{center}
\end{figure}


\begin{equation*}
n=\frac{\sqrt{V+V_{0}}}{\sqrt{V}}=\sqrt{1+\frac{V_{0}}{V}}, \tag{3.15}
\end{equation*}


где \(V_{0}\) - внутренний потенциал кристалла. Видно, что чем больше \(V\) (относительно \(V_{0}\) ), тем \(n\) ближе к единице. Таким образом, \(n\) проявляет себя особенно при малых \(V\), и формула Брэг-га-Вульфа принимает вид


\begin{equation*}
2 d \sqrt{n^{2}-\cos ^{2} \alpha}=m \lambda . \tag{3.I6}
\end{equation*}


Доказательство этой формулы приведено в решении задачи 3.7.\\
Убедимся, что формула Брэгга-Вульфа (3.16) с учетом преломления действительно объясняет положения максимумов интенсивности \(I(\sqrt{V})\) на рис. 3.5. Заменив в (3.16) \(n\) и \(\lambda\) согласно формулам (3.15) и (3.8) их выражениями через ускоряющую разность потенциалов \(V\), т.е.


\begin{equation*}
n^{2}=1+V_{0} / V, \quad \lambda=1,226 / \sqrt{V}, \text { нм, } \tag{3.17}
\end{equation*}


получим:


\begin{equation*}
\sqrt{\frac{V_{0}}{V}+\sin ^{2} \alpha}=\frac{m}{2 d} \frac{1,226}{\sqrt{V}} . \tag{3.18}
\end{equation*}


Теперь учтем, что распределение \(I(\sqrt{V})\) на рис. 3.5 получено для никеля при значениях \(V_{0}=15 \mathrm{~B}, d=0,203\) нм и \(\alpha=80^{\circ}\). Тогда (3.18) после несложных преобразований можно переписать так:


\begin{equation*}
\sqrt{V}=\frac{\sqrt{9,1 m^{2}-V_{0}}}{\sin \alpha} . \tag{3.19}
\end{equation*}


Вычислим по этой формуле значение \(\sqrt{V}\), например, для максимума третьего порядка ( \(m=3\) ), для которого расхождение с

формулой Брэгга-Вульфа (3.10) оказалось наибольшим:

\[
\sqrt{V}=\frac{\sqrt{9,1 \cdot 3^{2}-15}}{0,985}=8,3 \mathrm{~B}^{1 / 2} .
\]

Совпадение с действительным положением максимума 3 -го порядка не требует комментариев.

Итак, опыты Дэвиссона и Джермера следует признать блестящим подтверждением гипотезы де-Бройля.

Опыты Томсона и Тартаковского. В этих опытах пучок электронов пропускался через поликристаллическую фольгу (по методу Дебая при изучении дифракции рентгеновского излучения). Как и в случае рентгеновского излучения, на фотопластинке, расположенной за фольгой, наблюдалась система дифракционных колец. Сходство обеих картин поразительно. Подозрение, что система этих колец порождается не электронами, а вторичным рентгеновским излучением, возникающим в результате падения электронов на фольгу, легко рассеивается, если на пути рассеянных электронов создать магнитное поле (поднести постоянный магнит). Оно не влияет на рентгеновское излучение. Такого рода проверка показала, что интерференционная картина сразу же искажалась. Это однозначно свидетельствует, что мы имеем дело именно с электронами.\\
Г. Томсон осуществил опыты с быстрыми электронами (десятки кэВ), П. С. Тартаковский - со сравнительно медленньми электронами (до 1,7 кэВ).

Опыты с нейтронами и молекулами. Для успешного наблюдения дифракции волн на кристаллах необходимо, чтобы длина волны этих волн была сравнима с расстояниями между узлами кристаллической решетки. Поэтому для наблюдения дифракции тяжелых частиц необходимо пользоваться частицами с достаточно малыми скоростями. Соответствующие опыты по дифракции нейтронов и молекул при отражении от кристаллов были проделаны и также полностью подтвердили гипотезу де-Бройля в применении и к тяжелым частицам.

Благодаря этому было экспериментально доказано, что волновые свойства являются универсальным свойством всех частиц. Они не обусловлены какими-то особенностями внутреннего строения той или иной частицы, а отражают их общий закон движения.

Опыты с одиночными электронами. Описанные выше опыты выполнялись с использованием пучков частиц. Поэтому возникает естественный вопрос: наблюдаемые волновые свойства выражают свойства пучка частиц или отдельных частиц?

Чтобы ответить на этот вопрос, В. Фабрикант, Л. Биберман и Н. Сушкин осуществили в 1949 г. опыты, в которых применялись столь слабые пучки электронов, что каждый электрон проходил через кристалл заведомо поодиночке и каждый рассеянный электрон регистрировался фотопластинкой. При этом оказалось, что отдельные электроны попадали в различные точки фотопластинки совершенно беспорядочным на первый взгляд образом (рис.3.7, a). Между тем при достаточно длительной экспозиции на фотопластинке возникала дифракционная картина

\begin{figure}[h]
\begin{center}
\captionsetup{labelformat=empty}
\caption{a)}
  \includegraphics[alt={},max width=\textwidth]{a57b3a57-78b5-4292-b0ab-4bd9f2ffcf56-070_259_263_536_897}
\end{center}
\end{figure}

(рис. 3.7, б), абсолютно идентичная картине дифракции от обычного электронного пучка.\\
Так было доказано, что волновыми свойствами обладают и отдельные частицы.

Таким образом, мы имеем дело с микрообъектами, которые обладают одновременно как корпускулярными, так и волновыми свойствами. Это позволяет нам в дальней-

\begin{figure}[h]
\begin{center}
\captionsetup{labelformat=empty}
\caption{б)}
  \includegraphics[alt={},max width=\textwidth]{a57b3a57-78b5-4292-b0ab-4bd9f2ffcf56-070_254_257_834_903}
\end{center}
\end{figure}

Рис. 3.7\\
шем говорить об электронах, но выводы, к которым мы придем, имеют совершенно общий смысл и в равной степени применимы к любым частицам.

\section*{§ 3.3. Парадоксальное поведение микрочастиц}
Рассмотренные в предыдущем параграфе эксперименты вынуждают констатировать, что перед нами один из загадочнейших парадоксов: что означает утверждение «электрон - это одновременно частица и волна»?

Попытаемся разобраться в этом вопросе с помощью мысленного эксперимента, аналогичного опыту Юнга по изучению интерференции света (фотонов) от двух щелей. После прохождения пучка электронов через две щели на экране образуется система максимумов и минимумов, положение которых можно

рассчитать по формулам волновой оптики, если каждому электрону сопоставить дебройлевскую волну.

В явлении интерференции от двух щелей таится сама суть квантовой теории, поэтому уделим этому вопросу особое внимание.

Если мы имеем дело с фотонами, то парадокс (частица волна) можно устранить, предположив, что фотон в силу своей специфичности расщепляется на две части (на щелях), которые затем интерферируют.

А электроны? Они ведь никогда не расщепляются - это установлено совершенно достоверно. Электрон может пройти либо через щель 1 , либо че-

\begin{figure}[h]
\begin{center}
  \includegraphics[alt={},max width=\textwidth]{a57b3a57-78b5-4292-b0ab-4bd9f2ffcf56-071_289_574_665_64}
\captionsetup{labelformat=empty}
\caption{Рис. 3.8}
\end{center}
\end{figure}

рез щель 2. Следовательно, распределение их на экране Э должно быть суммой распределений 1 и 2 (рис. 3.8, \(a\) ) - оно показано пунктирной кривой.

Хотя логика в этих рассуждениях безупречна, такое распределение не осуществляется. Вместо этого мы наблюдаем совершенно иное распределение* (рис. 3.8, б).

Не есть ли это крушение чистой логики и здравого смысла? Ведь все выглядит так, как если бы \(100+100=0\) (в точке \(P\) ). В самом деле, когда открыта или щель 1 или щель 2 , то в точку \(P\) приходит, скажем, по 100 электронов в секунду, а если открыты обе щели, то ни одного!..

Более того, если сначала открыть щель 1 , а потом постепенно открывать щель 2 , увеличивая ее ширину, то по здравому смыслу число электронов, приходящих в точку \(P\) ежесекундно, должно расти от 100 до 200 . В действительности же - от 100 до нуля.

Если подобную процедуру повторить, регистрируя частицы, например, в точке \(O\) (см. рис. 3.8, б), то возникает не менее парадоксальный результат. По мере открывания щели 2 (при от-

\footnotetext{\begin{itemize}
  \item Это было установлено на эксперименте Мелленштадтом и Дюкером (1956), а также Йенсеном (1961).
\end{itemize}
}крытой щели 1) число частиц в точке \(O\) растет не до 200 в секунду, как следовало бы ожидать, а до 400 !

Как открывание щели 2 может повлиять на электроны, которые, казалось бы, проходят через щель 1? Т. е. дело обстоит так, что каждый электрон, проходя через какую-то щель, «чувствует» и соседнюю щель, корректируя свое поведение. Или подобно волне проходит сразу через обе щели (!?). Ведь иначе интерференционная картина не может возникнуть. Попытка все же определить, через какую щель проходит тот или иной электрон, приводит к разрушению интерференционной картины, но это уже совсем другой вопрос.

Какой же вывод? Единственный способ «объяснения» этих парадоксальных результатов заключается в создании математического формализма, совместимого с полученными результатами и всегда правильно предсказывающего наблюдаемые явления. Причем, разумеется, этот формализм должен быть внутренне непротиворечивым.

И такой формализм был создан. Он ставит в соответствие каждой частице некоторую комплексную \(n c u\)-функицю \(\Psi(\mathbf{r}, t)\). Формально она обладает свойствами классических волн, поэтому ее часто называют волновой функиией. Но более подробно об этой функции, ее физическом смысле и уравнении, которое управляет ее поведением в пространстве и времени, речь пойдет в следующей главе.

Возвращаясь к поведению электронов при прохождении через две щели, мы должны признать: тот факт, что в принципе нельзя ответить на вопрос, через какую щель проходит электрон (не разрушая интерференционной картины), несовместим с представлением о траектории. Таким образом, электронам, вообще говоря, нельзя приписать траектории.

Однако при определенных условиях, а именно когда дебройлевская длина волны микрочастицы становится очень малой и может оказаться много меньше, например, расстояния между щелями или атомных размеров, понятие траектории снова приобретает смысл. Рассмотрим этот вопрос более подробно и сформулируем более корректно условия, при которых можно пользоваться классической теорией.

Критерий классического описания. Подобно той роли, которую играет скорость света при решении вопроса о применимости ньютоновской (нерелятивистской) механики, существует критерий, показывающий в каких случаях можно ограничиться классическими представлениями. Этот критерий связан с постоянной Планка \(\hbar\).

Физическая размерность \(\hbar\) равна (энергия) × (время) или (импульс) × (длина) или (момент импульса). Величину с такой размерностью называют действием. Постоянная Планка является квантом действия.

Упомянутый критерий состоит в следующем. Если в данной физической системе значение некоторой характерной величины \(H\) с размерностью действия сравнимо с \(\hbar\), то поведение этой системы может быть описано только в рамках квантовой теории. Если же значение \(H\) очень велико по сравнению с \(\hbar\), то поведение системы с высокой точностью описывают законы классической физики.

Отметим, однако, что данный критерий имеет приближенный характер. Он указывает лишь, когда следует проявлять осторожность. Малость действия \(H\) не всегда свидетельствует о полной неприменимости классического подхода. Во многих случаях она может дать некоторое качественное представление о поведении системы, которое можно уточнить с помощью квантового подхода.

Величины макромира, имеющие размерность действия, в огромное число раз превышают квант действия \(\hbar\). Вот несколько примеров.

Пример 1. Небольшой маятник. Пусть средняя энергия его колебаний \(E \approx 1\) эрг, а период колебаний \(T \approx 1\) с. Величина с размерностью действия - это \(E \cdot T\). Отношение \(E T / \hbar \approx 10^{26}\).

Пример 2. Вращающееся тело с моментом инерции \(I=1\) г \(\cdot \mathrm{cm}^{2}\) и угловой скоростью \(\omega=1\) рад/с. Отношение момента импульса к кванту действия \(I \omega / \hbar \approx 10^{26}\).

Пример 3. Небольшой гармонический осциллятор. Пусть его масса \(m=1\) г, максимальная скорость \(v=1 \mathrm{~cm} / \mathrm{c}\) и максимальная амплитуда \(a=1 \mathrm{~cm}\). Тогда его максимальный импульс \(m v=1 г \cdot с м / с\). Величина \(a \cdot m v\) имеет размерность действия, и отношение \(a m v / \hbar \sim 10^{26}\).

Видно, что во всех трех случаях действие \(H \gg \hbar\), а это означает, что описание движения таких систем можно уверенно проводить в рамках классической физики.

Совсем иначе обстоит дело, когда действие \(H\) становится сравнимым с ћ. Здесь мы вступаем в область, где действуют совершенно другие законы - законьι квантовой физики. С этими законами нам и предстоит познакомиться.

\section*{§ 3.4. Принцип неопределенности}
В классической физике исчерпывающее описание состояния частицы определяется динамическими параметрами, такими как координаты, импульс, момент импульса, энергия и др. Однако реальное поведение микрочастиц показывает, что существует принципиальный предел точности, с которой подобные переменные могут быть указаны и измерены.

Соотношения неопределенностей. Глубокий анализ причин существования этого предела, который называют приниипом неопределенности, провел В. Гейзенберг (1927). Количественные соотношения, выражающие этот принцип в конкретных случаях, называют соотношениями неопределенностей.

Наиболее важными являются два соотношения неопределенностей.

Первое из них ограничивает точности одновременного измерения координат и соответствующих проекций импульса частицы. Для проекции, например, на ось \(X\) оно выглядит так*:


\begin{equation*}
\Delta x \cdot \Delta p_{x} \geqslant \hbar . \tag{3.20}
\end{equation*}


Второе соотношение устанавливает неопределенность измерения энергии, \(\Delta E\), за данный промежуток времени \(\Delta t\) :


\begin{equation*}
\Delta E \cdot \Delta t \geqslant \hbar . \tag{3.21}
\end{equation*}


\footnotetext{\begin{itemize}
  \item Заметим, что в точном соотношении неопределенностей под \(\Delta x\) и \(\Delta p_{x}\) должны пониматься среднеквадратичные отклонения от средних значений, а справа не \(h\) и не \(\hbar\), а \(\hbar / 2\). Мы не будем пользоваться точным соотношением, так как во всех принципиальных вопросах существенно знать лишь порядок величины \(\Delta x \cdot \Delta p_{x}\), а не ее точное значение.
\end{itemize}
}Поясним смысл этих двух соотношений. Первое из них утверждает, что если положение частицы, например, по оси \(X\) известно с неопределенностью \(\Delta x\), то в тот же момент проекцию импульса частицы на эту же ось можно измерить только с неопределенностью \(\Delta p_{x} \approx \hbar / \Delta x\). Заметим, что эти ограничения не касаются одновременного измерения координаты частицы по одной оси и проекции импульса - по другой: величины \(x\) и \(p_{y}, y\) и \(p_{z}\) и т. д. могут иметь одновременно точные значения.

Согласно второму соотношению (3.21) для измерения энергии с погрешностью \(\Delta E\) необходимо время, не меньшее, чем \(\Delta t \approx \hbar / \Delta E\). Примером может служить «размытие» энергетических уровней водородоподобных систем (кроме основного состояния). Это связано с тем, что время жизни во всех возбужденных состояниях этих систем порядка \(10^{-8}\) с. Размытие же уровней приводит к уширению спектральных линий ( естественное уширение), которое действительно наблюдается. Сказанное относится и к любой нестабильной системе. Если время жизни ее до распада порядка \(\tau\), то из-за конечности этого времени энергия системы имеет неустранимую неопределенность, не меньшую, чем \(\Delta E \approx \hbar / \tau\).

В дальнейшем будет показано, что во многих случаях умелое применение соотношений неопределенностей позволяет угадывать (или предсказывать) основные черты явлений.

О соотношении \(\Delta \boldsymbol{x} \cdot \Delta \boldsymbol{p}_{\boldsymbol{x}} \geqslant \hbar\). Обсудим более подробно смысл и возможности этого соотношения. Прежде всего обратим внимание на то, что оно определяет принципиальный предел неопределенностей \(\Delta x\) и \(\Delta p_{x}\), с которыми состояние частицы можно характеризовать классически, т. е. координатой \(x\) и проекцией импульса \(p_{x}\). Чем точнее \(x\), тем с меньшей точностью возможно установить \(p_{x}\), и наоборот.

Подчеркнем, что истинный смысл соотношения (3.20) отражает тот факт, что в природе объективно не существует состояний частицы с точно определенными значениями обеих переменных, \(x\) и \(p_{x}\). Вместе с тем мы вынуждены, поскольку измерения проводятся с помощью макроскопических приборов, приписывать частицам не свойственные им классические переменные. Издержки такого подхода и выражают соотношения неопределенностей.

После того, как выяснилась необходимость описывать поведение частиц волновыми функциями, соотношения неопределенностей возникают естественным образом - как математическое следствие теории.

Считая соотношение неопределенностей (3.20) универсальным, оценим, как бы оно сказалось на движении макроскопического тела. Возьмем очень маленький шарик массы \(m=1\) мг. Определим, например, с помощью микроскопа его положение с погрешностью \(\Delta x \approx 10^{-5} \mathrm{~cm}\) (она обусловлена разрешающей способностью микроскопа). Тогда неопределенность скорости шарика \(\Delta v=\Delta p / m \approx(\hbar / \Delta x) / m \sim 10^{-19} \mathrm{~cm} / \mathrm{c}\). Такая величина недоступна никакому измерению, а потому и отступление от классического описания совершенно несущественно. Другими словами, даже для такого маленького (но макроскопического) шарика понятие траектории применимо с высокой степенью точности.

Иначе ведет себя электрон в атоме. Грубая оценка показывает, что неопределенность скорости электрона, движущегося по боровской орбите атома водорода, сравнима с самой скоростью: \(\Delta v \approx v\). При таком положении представление о движении электрона по классической орбите теряет всякий смысл. И вообще, при движении микрочастиц в очень малых областях пространства понятие траектории оказывается несостоятельным.

Вместе с тем, при определенных условиях движение даже микрочастиц может рассматриваться классически, т. е. как движение по траектории. Так происходит, например, при движении заряженных частиц в электромагнитных полях (в электрон-но-лучевых трубках, ускорителях и др.). Эти движения можно рассматривать классически, поскольку для них ограничения, обусловленные соотношением неопределенностей, пренебрежимо малы по сравнению с самими величинами (координатами и импульсом).

Опыт со щелью. Соотношение неопределенностей (3.20) проявляет себя при любой попытке точного измерения положения или импульса микрочастицы. И каждый раз мы приходим к «неутешительному» результату: уточнение положения частицы приводит к увеличению неопределенности импульса, и наоборот. В качестве иллюстрации такой ситуации рассмотрим следующий пример.

Попытаемся определить координату \(x\) свободно движущейся с импульсом \(p\) частицы, поставив на ее пути щель шириной \(b\)

\begin{figure}[h]
\begin{center}
  \includegraphics[alt={},max width=\textwidth]{a57b3a57-78b5-4292-b0ab-4bd9f2ffcf56-077_247_499_252_68}
\captionsetup{labelformat=empty}
\caption{Рис. 3.9}
\end{center}
\end{figure}

(рис. 3.9). До прохождения частицы через щель ее проекция импульса \(p_{x}\) имеет точное значение: \(p_{x}=0\). Это значит, что \(\Delta p_{x}= =0\), но координата \(x\) частицы является совершенно неопределенной согласно (3.20).

Если частица пройдет сквозь щель, то в плоскости щели координата \(x\) будет зарегистрирована с неопределенностью \(\Delta x \approx b\). При этом вследствие дифракции с наибольшей вероятностью частица будет двигаться в пределах угла \(2 \theta\), где \(\theta\) - угол, соответствующий первому дифракционному минимуму. Он определяется условием, при котором разность хода волн от обоих краев щели будет равна \(\lambda\) (это доказывается в волновой оптике):

\[
b \sin \theta=\lambda .
\]

В результате дифракции возникает неопределенность значения \(p_{x}\) - проекции импульса, разброс которого

\[
\Delta p_{x} \approx p \sin \theta .
\]

Учитывая, что \(b \approx \Delta x\) и \(p=2 \pi \hbar / \lambda\), получим из двух предыдущих выражений:

\[
\Delta x \cdot \Delta p_{x} \approx p \lambda=2 \pi \hbar,
\]

что согласуется по порядку величины с (3.20).\\
Таким образом, попытка определить координату \(x\) частицы, действительно, привела к появлению неопределенности \(\Delta p_{x}\) в импульсе частицы.

Анализ многих ситуаций, связанных с измерениями, показывает, что измерения в квантовой области принципиально отличаются от классических измерений. В отличие от последних, в квантовой физике существует естественный предел точности измерений. Он в самой природе квантовых объектов и не может быть преодолен никаким совершенствованием приборов и методов измерений. Соотношение (3.20) и устанавливает один из та-

ких пределов. Взаимодействие между микрочастицей и макроскопическим измерительным прибором нельзя сделать сколь угодно малым. Измерение, например, координаты частицы неизбежно приводит к принципиально неустранимому и неконтролируемому искажению состояния микрочастицы, а значит и к неопределенности в значении импульса.

Некоторые выводы. Соотношение неопределенностей (3.20) является одним из фундаментальных положений квантовой теории. Одного этого соотношения достаточно, чтобы получить ряд важных результатов, в частности:

\begin{enumerate}
  \item Невозможно состояние, в котором частица находилась бы в состоянии покоя.
  \item При рассмотрении движения квантового объекта необходимо во многих случаях отказаться от самого понятия классической траектории.
  \item Часто теряет смысл деление полной энергии \(E\) частицы (как квантового объекта) на потенциальную \(U\) и кинетическую \(K\). В самом деле, первая, т. е. \(U\), зависит от координат, а вторая - от импульса. Эти же динамические переменные не могут иметь одновременно определенного значения.
\end{enumerate}

Размер атома водорода. Прежде чем рассмотреть важный пример, относящийся к атому водорода, остановимся на вопросе, который часто вызывает недоумение. Пусть частица «заперта» в одномерной области размером \(l\). При нахождении возможного значения минимальной энергии \(E_{\text {мин }}\) частицы мы обычно считаем, что импульс частицы по порядку величины равен его неопределенности, т. е. \(p \sim \Delta p\). На каком основании?

Чтобы понять, почему это так, представим себе, что частица в этой области имеет энергию \(E>E_{\text {мин }}\). Тогда ее импульс может быть представлен как \(p=\langle p\rangle+\Delta p\). Теперь начнем мысленно уменьшать энергию \(E\), а значит и импульс \(\langle p\rangle\). При этом \(\Delta p\) не меняется, поскольку \(\Delta p \approx \hbar / l\) согласно соотношению (3.20). Когда \(E\) станет равной \(E_{\text {мин }}\), величина \(\langle p\rangle\) обратится в нуль и останется только \(\Delta p\). Эту величину и принимают за \(p\). Теперь перейдем к атому водорода.

Оценим его размер и попытаемся понять, почему электрон не падает на ядро (как это можно объяснить с помощью соотношения неопределенностей).

Точное положение электрона в данном атоме запрещено принципом неопределенности: был бы бесконечно большой разброс в его импульсе. Поэтому для оценки наименьшей возможной энергии \(E_{\text {мин }}\) электрона в кулоновском поле ядра можно положить разброс расстояний электрона от ядра \(\Delta r \approx r\) и \(\Delta p \approx p\). Тогда согласно (3.20) \(p \approx \hbar / r\), и энергия \(E\) может быть представлена как


\begin{equation*}
E=\frac{p^{2}}{2 m}-\frac{e^{2}}{r} \approx \frac{\hbar}{2 m r^{2}}-\frac{e^{2}}{r} \tag{3.22}
\end{equation*}


Значение \(r\), при котором \(E=\) мин, можно найти, приравняв производную \(\mathrm{d} E / \mathrm{d} r\) к нулю:

\[
-\frac{\hbar^{2}}{m r^{3}}+\frac{e^{2}}{r^{2}}=0
\]

Отсюда следует, что


\begin{equation*}
r=\hbar^{2} / m e^{2} \tag{3.23}
\end{equation*}


Полученный результат полностью совпадает с боровским радиусом (2.23).

Подставив (3.23) в (3.22), мы найдем энергию \(E_{\text {мин }}\) :


\begin{equation*}
E_{\text {мин }}=-\frac{e^{2}}{2 r}=-\frac{m e^{4}}{2 \hbar^{2}}=-13,6 \text { эВ }, \tag{3.24}
\end{equation*}


что также совпадает с энергией основного состояния атома водорода (2.25).

Разумеется, совпадение наших грубых оценок с точными значениями \(r\) и \(E\) следует считать случайным. Важно лишь то, что получен верный порядок этих величин и что, основываясь на волновых представлениях, или принципе неопределенности, можно понять, почему атомный электрон не падает на ядро. Размер атома является результатом компромисса двух слагаемых энергии (3.22), имеющих противоположные знаки. Если увеличить отрицательное слагаемое (потенциальную энергию), уменьшив \(r\), то увеличится кинетическая энергия, и наоборот.

Таким образом, соотношение неопределенностей проявляет себя в атоме подобно силам отталкивания на малых расстояниях. В результате электрон находится в среднем на таком расстоянии от ядра, на котором действие этих сил отталкивания компенсируется силой кулоновского притяжения.

\section*{Задачи}
3.1. Волны де-Бройля. Какую энергию \(\Delta E\) необходимо сообщить нерелятивистскому электрону, чтобы его дебройлевская длина волны \(\lambda\) уменьшилась в \(n\) раз?

Решени е. Обозначим конечную дебройлевскую длину волны как \(\lambda^{\prime}\). Имея в виду, что согласно (3.1) \(\lambda\) っ \(1 / p\) s \(1 / \sqrt{K}\), запишем:

\[
n=\frac{\lambda}{\lambda^{\prime}}=\sqrt{\frac{K+\Delta E}{K}},
\]

где \(K\) - первоначальная кинетическая энергия электрона. Отсюда

\[
\Delta E=K\left(n^{2}-1\right)=\frac{2 \pi^{2} \hbar^{2}}{m \lambda^{2}}\left(n^{2}-1\right),
\]

где \(m\) - масса электрона.\\
3.2. Найти дебройлевскую длину волны протонов, если в однородном магнитном поле с индукцией \(B\) радиус кривизны их траектории - окружности - равен \(R\).

Решение. Согласно (3.1) для этого надо сначала определить импульс протона. Воспользуемся основным уравнением динамики:

\[
m \frac{v^{2}}{R}=e v B .
\]

Отсюда \(p=R e B\), и искомая длина волны

\[
\lambda=2 \pi \hbar / R e B .
\]

3.3. Нерелятивистская частица массы \(m_{1}\) с кинетической энергией \(K_{1}\) налетает на покоящуюся частицу массы \(m_{2}\). Найти дебройлевскую длину волны \(\tilde{\lambda}\) обеих частиц в системе их центра масс (Ц-системе). Решение. Искомая длина волны согласно (3.1) определяется как \(\tilde{\lambda}=2 \pi \hbar / \tilde{p}\), где \(\tilde{p}\) — импульс каждой частицы в Ц-системе. Напомним, что в Ц-системе импульсы обеих частиц равны по моду-

лю и противоположны по направлению. Итак, решение вопроса сводится к нахождению \(\tilde{p}\).

Для этого найдем сначала скорость \(\mathbf{v}_{C}\) Ц-системы. По определению,


\begin{equation*}
\mathrm{v}_{C}=\frac{m_{1} \mathbf{v}_{\mathbf{1}}+m_{2} \mathbf{v}_{\mathbf{2}}}{m_{1}+m_{2}} \tag{1}
\end{equation*}


В нашем случае \(\mathbf{v}_{2}=0\), следовательно


\begin{equation*}
\mathbf{v}_{C}=m_{1} \mathbf{v}_{1} /\left(m_{1}+m_{2}\right) . \tag{2}
\end{equation*}


Скорость частицы массы \(m_{1}\) в Ц-системе \(\tilde{\mathbf{v}}_{1}=\mathbf{v}_{1}-\mathbf{v}_{C}\), откуда следует с учетом (2), что

\[
\tilde{\mathbf{v}}_{1}=m_{1} \mathbf{v}_{1} /\left(m_{1}+m_{2}\right) .
\]

Импульс этой частицы в Ц-системе


\begin{equation*}
\tilde{p}_{1}=m_{1} \tilde{v}_{1}=\mu v_{1}, \tag{3}
\end{equation*}


где \(\mu\) - приведенная масса системы из двух частиц, т.е. \(\mu=m_{1} m_{2} /\left(m_{1}+m_{2}\right)\).\\
Подставив (3) в исходную формулу, найдем после несложных преобразований, что

\[
\tilde{\lambda}=\frac{2 \pi \hbar}{\sqrt{2 m_{1} K_{1}}}\left(1+\frac{m_{1}}{m_{2}}\right) .
\]

3.4. При каком значении кинетической энергии \(K\) дебройлевская длина волны \(\lambda\) релятивистского электрона равна его комптоновской длине волны \(\lambda_{C}\) ?

Решение. Исходим из равенства \(\lambda=\lambda_{C}\), где \(\lambda\) определяется формулой (3.1), а \(\lambda_{C}\) — формулой (1.21). Поэтому можно записать


\begin{equation*}
2 \pi \hbar / p=2 \pi \hbar / m c . \tag{1}
\end{equation*}


Из релятивистской динамики известно (П.5), что


\begin{equation*}
p c=\sqrt{K\left(K+2 m c^{2}\right)} . \tag{2}
\end{equation*}


Подставив (2) в (1), получим уравнение

\[
K^{2}+2 m c^{2} K-m^{2} c^{4}=0,
\]

решение которого

\[
K=(\sqrt{2}-1) m c^{2} .
\]

3.5. Параллельный пучок нерелятивистских электронов, ускоренных разностью потенциалов \(V\), падает нормально на диафрагму с двумя узкими щелями, расстояние между которыми \(d\). Определить расстояние между соседними максимумами интерференционной картины на экране, расположенном на расстоянии \(l\) от щелей ( \(l \gg d\) ).

Решение. Из волновой оптики известно, что искомое расстояние \(\Delta x\) (ширина интерференционной полосы) определяется формулой

\[
\Delta x=\lambda l / d .
\]

Подставив сюда вместо \(\lambda\) выражение (3.1) для дебройлевской длины волны, получим

\[
\Delta x=\frac{2 \pi \hbar l}{d \sqrt{2 m e V}}
\]

где учтено, что кинетическая энергия электронов \(K=e V\).\\
3.6. Узкий пучок нерелятивистских электронов с кинетической энергией \(K=180\) эВ падает нормально на поверхность монокристалла никеля. В направлении, составляющем угол \(\theta=55^{\circ}\) с направлением падающего пучка, наблюдается максимум отражения 4 -го порядка. Вычислить соответствующее значение межплоскостного расстояния \(d\). Преломления волн не учитывать.\\
Решение. Сначала изобразим схему (рис. 3.10), соответствующую условию задачи. Затем воспользуемся формулой Брэгга-Вульфа


\begin{equation*}
2 d \sin \alpha=n \lambda, \tag{1}
\end{equation*}


где \(\alpha\) - угол скольжения, который, как видно из рисунка, равен


\begin{equation*}
\alpha=\pi / 2-\theta / 2, \tag{2}
\end{equation*}


а \(\lambda\) - дебройлевская длина волны:


\begin{equation*}
\lambda=2 \pi \hbar / \sqrt{2 m K} . \tag{3}
\end{equation*}


\begin{figure}[h]
\begin{center}
  \includegraphics[alt={},max width=\textwidth]{a57b3a57-78b5-4292-b0ab-4bd9f2ffcf56-082_290_323_1472_839}
\captionsetup{labelformat=empty}
\caption{Рис. 3.10}
\end{center}
\end{figure}

После подстановки (2) и (3) в формулу (1) получим

\[
d=\frac{\pi \hbar n}{\sqrt{2 m K} \cos (\theta / 2)}=0,206 \mathrm{HM},
\]

где \(n=4\) - порядок интерференционного максимума.\\
3.7. Преломление волн де-Бройля. Показать, что с учетом преломления формула Брэгга-Вульфа имеет вид

\[
2 d \sqrt{n^{2}-\cos ^{2} \alpha}=m \lambda,
\]

где \(d\) - межплоскостное расстояние, \(n\) - показатель преломления кристалла для дебройлевских волн, \(m\) - порядок интерференционного максимума, \(\lambda\) - дебройлевская длина волны.

Решение. Рассмотрим две интерферирующие волны, представленные лучами \(1^{\prime}\) и \(2^{\prime}\) (рис. 3.11). Из-за преломления волн угол па-

\begin{figure}[h]
\begin{center}
  \includegraphics[alt={},max width=\textwidth]{a57b3a57-78b5-4292-b0ab-4bd9f2ffcf56-083_440_542_807_138}
\captionsetup{labelformat=empty}
\caption{Рис. 3.11}
\end{center}
\end{figure}

дения \(\theta\) не равен углу преломления \(\theta^{\prime}\). Запишем «оптическую» разность хода лучей 1 и 2. Как видно из рисунка, она равна


\begin{equation*}
\Delta=n(A B C)=n \cdot 2 d \cos \theta^{\prime} \tag{1}
\end{equation*}


(эта разность хода выделена на рисунке жирными отрезками).

С другой стороны, по закону преломления


\begin{equation*}
\sin \theta=n \sin \theta^{\prime} . \tag{2}
\end{equation*}


Условие образования интерференционного максимума - это \(\Delta=m \lambda\), где \(m=1,2, \ldots\) Запишем это условие с помощью (1) и (2) следующим образом:


\begin{equation*}
2 d n \cos \theta^{\prime}=2 d n \sqrt{1-\frac{\sin ^{2} \theta}{n^{2}}}=2 d \sqrt{n^{2}-\sin ^{2} \theta}=m \lambda . \tag{3}
\end{equation*}


Согласно рис. \(3.11 \sin \theta=\cos \alpha\), поэтому формулу (3) можно записать также в виде, представленном в условии задачи.\\
3.8. Соотношение неопределенностей. Убедиться, что измерение \(x\)-координаты микрочастицы с помощью микроскопа (рис. 3.12) вносит неопределенность в ее импульс \(\Delta p_{x}\) такую, что \(\Delta x \cdot \Delta p_{x}>\hbar\). Иметь в виду, что разрешение микроскопа, т.е. наименьшее разрешаемое расстояние \(d=\lambda / \sin \theta\), где \(\lambda\) длина световой волны.

Решение. У фотона, рассеянного на микрочастице и прошедшего через объектив \(O\), проекция импульса \(p_{x}\) не превышает, как видно из рисунка, значения \(p \sin \theta=\hbar k \sin \theta\), где \(k=2 \pi / \lambda\). Эта величина характеризует и

\begin{figure}[h]
\begin{center}
  \includegraphics[alt={},max width=\textwidth]{a57b3a57-78b5-4292-b0ab-4bd9f2ffcf56-084_321_315_249_844}
\captionsetup{labelformat=empty}
\caption{Рис. 3.12}
\end{center}
\end{figure}

неопределенность \(\Delta p_{x}\) фотона. Но при рассеянии фотона на микрочастице последняя испытывает отдачу, в результате чего ее импульс получит такую же неопределенность \(\Delta p_{x}\), как и фотон: \(\Delta p_{x} \approx \approx \hbar k \sin \theta\).

Имея, кроме того, в виду, что неопределенность координаты \(x\) микрочастицы \(\Delta x \approx d=\lambda / \sin \theta\), получим в результате:

\[
\Delta x \cdot \Delta p_{x} \approx \frac{\lambda}{\sin \theta} \frac{2 \pi \hbar}{\lambda} \sin \theta=2 \pi \hbar
\]

в чем и следовало убедиться.\\
3.9. Электрон находится в одномерной прямоугольной потенциальной яме с очень высокими «стенками». Ширина ямы \(l\). Оценить с помощью (3.20) силу давления электрона на стенки ямы при минимально возможной его энергии.

Решение. В данном случае \(\Delta x \approx l\). Кроме того, при минимальной энергии можно считать, что \(\Delta p_{x} \approx p\). Тогда согласно (3.20) \(p \approx \hbar / l\) и полная энергия электрона в яме (учитывая, что потенциальная энергия здесь равна нулю) определяется как

\[
E=K=\frac{p^{2}}{2 m} \approx \frac{\hbar^{2}}{2 m l^{2}} .
\]

Теперь представим себе, что одну из стенок ямы отодвинули на малое расстояние \(\mathrm{d} l\). Это означает, что сила \(F\), с которой электрон действует на эту стенку, совершила работу \(F \mathrm{~d} l\) за счет убыли энергии \(E\) :

\[
F \mathrm{~d} l=-\mathrm{d} E=\left(\hbar^{2} / m l^{3}\right) \mathrm{d} l .
\]

Отсюда искомая сила

\[
F=\hbar^{2} / m l^{3}
\]

3.10. Частица массы \(m\) движется в одномерном потенциальном поле, где ее потенциальная энергия \(U=x x^{2} / 2\) (гармонический осциллятор). Оценить с помощью (3.20) минимально возможную энергию \(E\) частицы в этом поле.\\
Решение. При \(E=\) мин можно считать, что \(p \approx \Delta p\) и \(x \approx \Delta x\). Тогда в соответствии с (3.20) \(p \approx \hbar / \Delta x \approx \hbar / x\), и мы можем записать выражение для полной энергии \(E\) как


\begin{equation*}
E=K+U=\frac{p^{2}}{2 m}+\frac{\varkappa x^{2}}{2} \approx \frac{\hbar^{2}}{2 m x^{2}}+\frac{\varkappa x^{2}}{2} \tag{1}
\end{equation*}


Из условия \(\mathrm{d} E / \mathrm{d} x=0\) находим значение \(x_{m}\), при котором \(E=\) мин:


\begin{equation*}
x_{m}=\hbar^{2} / m x \tag{2}
\end{equation*}


После подстановки (2) в (1) получим

\[
E_{\text {мин }} \approx \hbar \sqrt{\varkappa / m} .
\]

Точный расчет дает величину вдвое меньшую.

\section*{Уравнение Шредингера. Квантование}
\section*{§ 4.1. Состояние частицы в квантовой теории}
Рассмотрение вопроса о математическом формализме, адекватном парадоксальному поведению микрочастиц, мы начнем с выяснения принципов, на которых строится фундаментальная физическая теория. Проследим за содержанием этих принципов в классической и квантовой теории на простейшем примере движения нерелятивистской частицы в стационарном силовом поле.

Для этого должны быть определены:

\begin{enumerate}
  \item величины, задающие состояние частицы;
  \item уравнение движения, определяющее изменение состояния частицы во времени;
  \item физические величины, доступные измерению, и способ получения их значений в данном состоянии (это необходимо для сравнения выводов теории с экспериментом).\\
Будем предполагать, что читателю достаточно хорошо известно, как это делается в классической теории. Поэтому обратимся сразу к решению этих вопросов в квантовой теории.
\end{enumerate}

Для микрочастиц из-за соотношения неопределенностей классическое определение состояния частицы (координаты и импульс), вообще говоря, утрачивает смысл*. В соответствии с корпуску-лярно-волновым дуализмом в квантовой теории состояние частицы задается пси-функцией \(\Psi(\mathbf{r}, t)\), которая является комплексной величиной и формально обладает волновыми свойствами.

Понимание физического смысла пси-функции пришло после того, как выяснилось, что волновые свойства характерны для отдельных частиц. Этот факт можно истолковать по идее Борна (1926) так. Движение любой микрочастицы по отдельности подчиняется вероятностным законам. Распределение вероятности, характеризующее это движение, проявляется в результате регистрации достаточно большого числа частиц. Это распределение оказывается таким же, как распределение интенсивности

\footnotetext{\begin{itemize}
  \item Это относится и к понятию силь, которая по определению является функцией классического состояния.
\end{itemize}
}волны: там, где интенсивность волны больше, регистрируется и большее число частиц.

В квантовой теории постановка вопроса состоит не в точном предсказании событий, а в определении вероятностей этих событий. По значениям вероятностей согласно определенным правилам (см. ниже) можно найти средние случайных значений физических величин, которые и доступны эксперименту. Пси-функция \(\Psi(\mathbf{r}, t)\) и является той величиной, которая позволяет находить все вероятности.

Например, вероятность нахождения частицы в интересующем нас объеме \(\mathrm{d} V\) в момент \(t\) определяется как


\begin{equation*}
\mathrm{d} P=|\Psi|^{2} \mathrm{~d} V=\Psi \Psi^{*} \mathrm{~d} V, \tag{4.1}
\end{equation*}


где \(\Psi^{*}\) - комплексно-сопряженная функция. Отсюда плотность вероятности, т. е. вероятность нахождения частицы в единице объема,


\begin{equation*}
P=|\Psi|^{2}=\Psi \Psi^{*} . \tag{4.2}
\end{equation*}


Эта величина является экспериментально наблюдаемой, в то время как сама пси-функция, будучи комплексной, не доступна наблюдению. Напомним, что в классике величины, характеризующие состояние частицы, являются принципиально наблюдаемыми.

Пси-функция, вообще говоря, определяется с точностью до произвольного постоянного множителя. Это не влияет на состояние частицы, которое она описывает. И тем не менее пси-функцию выбирают так, чтобы она удовлетворяла условию нормировки*:


\begin{equation*}
\int|\Psi|^{2} \mathrm{~d} V=\int \Psi \Psi^{*} \mathrm{~d} V=1 \tag{4.3}
\end{equation*}


где интеграл берется по всему пространству или по той области, в которой Ψ отлична от нуля. Условие нормировки (4.3) означает, что во всей области, где \(\Psi \neq 0\), частица находится с досто-

\footnotetext{\begin{itemize}
  \item Условие (4.3) может оказаться невозможным, например, в случае, если Ψ-функция представляет собой плоскую волну де-Бройля, когда вероятность обнаружения частицы одинакова во всех точках пространства. Такие случаи следует рассматривать как идеализацию реальной ситуации, где частица находится в большой, но ограниченной области пространства, и тогда трудность устраняется.
\end{itemize}
}верностью. Пси-функцию, удовлетворяющую условию (4.3), называют нормированной.

Принцип суперпозиции. Итак, непосредственный физический смысл имеет не сама Ψ-функция, а квадрат ее модуля \(|\Psi|^{2}\) или ΨΨ*. И тем не менее в квантовой теории оперируют с Ψ-функцией, а не с экспериментально наблюдаемой величиной \(|\Psi|^{2}\). Это необходимо для истолкования волновых свойств микрочастиц - интерференции и дифракции. Ситуация здесь совершенно идентична той, какую мы имеем в волновой теории. В волновой теории принимается принцип суперпозиции самих волновых полей, а не их интенсивностей. Именно так вводятся в теорию явления интерференции и дифракции.

Подобным же образом в квантовой теории принимается как один из основных постулатов принцип суперпозиции пси-функций. Если у некоторой системы возможными являются состояния \(\Psi_{1}\) и \(\Psi_{2}\), то для нее существует также состояние


\begin{equation*}
\Psi=c_{1} \Psi_{1}+c_{2} \Psi_{2}, \tag{4.4}
\end{equation*}


где \(c_{1}\) и \(c_{2}\) - некоторые постоянные коэффициенты. Найдя таким образом \(\Psi\), можно далее определить и плотность вероятности ΨΨ* пребывания системы в этом состоянии.

Подтверждением принципа суперпозиции (4.4) является согласие с опытом вытекающих из него следствий.

\section*{§ 4.2. Уравнение Шредингера}
Поиск уравнения, управляющего изменениями состояния системы, т.е. ее Ψ-функции во времени успешно был завершен Э. Шредингером (1926). Это - основное уравнение нерелятивистской квантовой теории, уравнение Шредингера. Данное уравнение было именно найдено, оно является новым фундаментальным законом, который невозможно вывести из прежних представлений и теорий. Справедливость этого уравнения установлена тем, что все вытекающие из него следствия подтверждены экспериментом.

Сформулировав это уравнение, Шредингер сразу же применил его к атому водорода и получил для уровней энергии спектр, точно совпадающий со спектром по первоначальной теории Бора и соответственно - с результатами наблюдений.

Уравнение Шредингера играет в квантовой теории такую же роль, как основное уравнение динамики ( 2 -й закон Ньютона) в нерелятивистской механике.

Уравнение Шредингера имеет следующий вид:


\begin{equation*}
\mathrm{i} \hbar \frac{\partial \Psi}{\partial t}=-\frac{\hbar^{2}}{2 m} \nabla^{2} \Psi+U \Psi \tag{4.5}
\end{equation*}


где і - мнимая единица \((\sqrt{-1}), m\) - масса частицы, \(\nabla^{2}\) - оператор Лапласа, \(U\) - потенциальная энергия (мы ограничимся рассмотрением потенциальньх силовых полей, для которых функция \(U(\mathbf{r})\) не зависит явно от времени).

Обратим внимание на следующую особенность уравнения (4.5). В то время как, согласно интерпретации \(\Psi\)-функции, частица, как говорят, «размазана» в пространстве, потенциальная энергия \(U\) рассматривается в (4.5) как функция локализованной точечной частицы в силовом поле.

Стационарные состояния. Особую роль в квантовой теории играют стационарные состояния - состояния, в которых все наблюдаемые физические величины не меняются с течением времени. Сама Ψ-функция, как уже говорилось, принципиально ненаблюдаема. В стационарных состояниях она имеет вид


\begin{equation*}
\Psi(\mathbf{r}, t)=\psi(\mathbf{r}) \mathrm{e}^{-\mathrm{i} \omega t}, \quad \omega=E / \hbar, \tag{4.6}
\end{equation*}


где функция \(\psi(\mathbf{r})\) не зависит от времени, а выражение для частоты ω написано согласно (3.2).

При таком виде \(\Psi\)-функции плотность вероятности \(P\) остается постоянной. В самом деле,


\begin{equation*}
P=\Psi \Psi^{*}=\psi(\mathbf{r}) \cdot \psi^{*}(\mathbf{r}), \tag{4.7}
\end{equation*}


т. е. действительно, плотность вероятности \(P\) от времени не зависит.

Для нахождения функции \(\psi(\mathbf{r})\) в стационарных состояниях подставим выражение (4.6) в уравнение (4.5), и мы получим


\begin{equation*}
-\frac{\hbar^{2}}{2 m} \nabla^{2} \psi+U \psi=E \psi \tag{4.8}
\end{equation*}


Это уравнение называют уравнением Шредингера для стационарных состояний. В отличие от него, (4.5) называют временнь́м или общим уравнением Шредингера.

В дальнейшем мы будем иметь дело только с уравнением (4.8) и будем записывать его (как это обычно принято) в виде


\begin{equation*}
\nabla^{2} \psi+\frac{2 m}{\hbar^{2}}(E-U) \psi=0 . \tag{4.9}
\end{equation*}


Еще раз напомним, что потенциальная энергия - функция \(U(\mathbf{r})\)-здесь определяется классически, как если бы никакими волновыми свойствами частица не обладала.

Квантование. В отличие от первоначальной теории Бора, где квантование вводилось искусственно, в теории Шредингера оно возникает автоматически. Достаточно только учесть, что физический смысл имеют лишь те решения уравнения (4.9), которые удовлетворяют естественным или стандартным условиям. Эти условия состоят в том, что пси-функция \(\psi(\mathbf{r})\) должна быть конечной, однозначной, непрерывной и гладкой (т. е. без изломов) во всем пространстве, даже в тех точках (линиях, поверхностях), где потенциальная энергия \(U(\mathbf{r})\) терпит разрыв. Эти условия не представляют чего-нибудь особенного. Это обычные требования, накладываемые на искомое решение дифференциального уравнения.

Решения, удовлетворяющие этим условиям, оказываются возможными лишь при некоторых значениях энергии \(E\). Их называют собственными значениями, а функции \(\psi(\mathbf{r})\), являющиеся решениями уравнения (4.9) при этих значениях энергии, - собственными функциями, принадлежащими собственным значениям \(E\). В этом и состоит естественный и общий принцип квантования.

Собственные значения энергии \(E\) и принимаются за возможные значения энергии в соответствующих стационарных состояниях. Эти значения энергии \(E\) могут быть дискретными (квантованными) или непрерывными, образуя дискретный или непрерывный энергетический спектр.

В общем случае зависимости потенциальной энергии \(U(\mathbf{r})\) от координат, решение уравнения Шредингера представляет собой весьма громоздкую задачу. Но если мы все же нашли это

решение \(\psi(\mathbf{r})\), то в принципе мы можем найти не только распределение вероятности местонахождения частицы, но также вероятности собственных значений различных физических величин (например, энергии, импульса, момента импульса). Надо только знать способ, как извлечь значения этих величин из функции \(\psi(\mathbf{r})\). Но об этом в дальнейшем.

Заметим, что при более строгом рассмотрении стационарных состояний выясняется, что они вовсе не стационарные. Вместе с тем, решения уравнения Шредингера приводят к наличию строго стационарных состояний, в противоречии с известными экспериментальными фактами. Здесь проявляется очевидная ограниченность уравнений Шредингера: они не описывают радиационных переходов. Тем не менее, предсказываемые уравнением Шредингера стационарные состояния с хорошей точностью соответствуют почти стационарным состояниям. Об этом свидетельствует опыт.

Теперь перейдем к рассмотрению нескольких простейших случаев, на которых проиллюстрируем, что квантование - это, действительно, естественное следствие вышеприведенных условий, накладываемых на решения уравнения Шредингера. При этом никаких дополнительных предположений делать не требуется.

\section*{§ 4.3. Частица в прямоугольной яме}
Рассмотрим поведение частицы в одномерной прямоугольной потенциальной яме \(U(x)\), имеющей две различные конфигурации - два случая. Предполагается, что частица может двигаться только вдоль оси \(X\).

\begin{figure}[h]
\begin{center}
  \includegraphics[alt={},max width=\textwidth]{a57b3a57-78b5-4292-b0ab-4bd9f2ffcf56-091_348_471_1389_58}
\captionsetup{labelformat=empty}
\caption{Рис. 4.1}
\end{center}
\end{figure}

Случай 1. Он является самым простым: ширина ямы равна \(l\), стенки ямы бесконечно высокие (рис. 4.1, a). Потенциальная энергия в этом случае имеет следующие значения: она равна нулю в интервале ( \(0, l\) ) и обращается в бесконечность при \(x=0\) и \(x=l\).

Исходим из уравнения Шредингера (4.9). Для одномерного случая в пределах ямы (где \(U=0\) ) это уравнение упрощается:


\begin{equation*}
\frac{\partial^{2} \psi}{\partial x^{2}}+k^{2} \psi=0 \tag{4.10}
\end{equation*}


где введено обозначение


\begin{equation*}
k^{2}=2 m E / \hbar^{2} . \tag{4.11}
\end{equation*}


Общее решение уравнения (4.10) имеет вид


\begin{equation*}
\psi(x)=a \sin (k x+\alpha), \tag{4.12}
\end{equation*}


где \(a\) и \(\alpha\) - произвольные постоянные.\\
Теперь самое главное: мы должны потребовать от функции \(\psi(x)\), чтобы она удовлетворяла естественным (стандартным) условиям. Видно, что \(\psi(x)\) в виде (4.12) однозначна и конечна. Она должна быть еще и непрерывной, а именно, вне ямы частица быть не может, значит там \(\psi(x)=0\), и для непрерывности \(\psi\)-функции необходимо, чтобы при \(x=0\) и \(x=l\) функция (4.12) была бы равна нулю. Из условия

\[
\psi(0)=a \sin \alpha=0
\]

следует, что \(\alpha=0\). Из условия же

\[
\psi(l)=a \sin k l=0
\]

в свою очередь следует, что


\begin{equation*}
k l= \pm \pi n, \tag{4.13}
\end{equation*}


где \(n=1,2,3, \ldots\) ( \(n=0\) отпадает, так как при этом \(\psi=0\) - частицы вообще нет).

Подставив \(k\) из (4.13) в (4.11), получим


\begin{equation*}
E_{n}=\frac{\pi^{2} \hbar^{2}}{2 m l^{2}} n^{2}, \quad n=1,2,3, \ldots \tag{4.14}
\end{equation*}


Энергия оказалась квантованной и ее спектр - дискретный (рис. 4.1, б).

Итак, собственные значения \(E\) мы нашли - это (4.14). Теперь найдем соответствующие им собственные функции. Для этого подставим значения \(k\) из (4.13) в (4.12), где \(\alpha=0\), тогда

\[
\psi(x)=a \sin (n \pi x / l) .
\]

Для определения коэффициента \(a\) воспользуемся условием нормировки (4.3). В нашем случае оно примет вид

\[
a^{2} \int_{0}^{l} \sin ^{2} \frac{n \pi x}{l} \mathrm{~d} x=1
\]

На концах интервала ( \(0, l\) ) подынтегральная функция равна нулю, поэтому значение интеграла можно представить как произведение среднего значения квадрата синуса (а оно равно \(1 / 2\) ) на ширину ямы \(l\) :

\[
a^{2}(1 / 2) l=1,
\]

откуда \(a=\sqrt{2 / l}\).\\
Таким образом, собственные функции в данном случае имеют вид


\begin{equation*}
\psi_{n}(x)=\sqrt{2 / l} \sin (n \pi x / l), \quad n=1,2,3, \ldots \tag{4.15}
\end{equation*}


Графики нескольких собственных функций показаны на рис. 4.2 пунктирными линиями, а распределение плотности вероятности - сплошными. Из этих графиков видно, что в низшем энергетическом состоянии ( \(n=1\) ) с наибольшей вероятностью частицу можно обнаружить в середине ямы, а вероятность нахождения ее вблизи краев ямы весьма мала. Такое поведение частицы резко отличается от поведения классической частицы.

\begin{figure}[h]
\begin{center}
  \includegraphics[alt={},max width=\textwidth]{a57b3a57-78b5-4292-b0ab-4bd9f2ffcf56-093_173_971_1427_131}
\captionsetup{labelformat=empty}
\caption{Рис. 4.2}
\end{center}
\end{figure}

С увеличением же энергии (т. е. с ростом квантового числа \(n\) ) максимумы распределения \(\psi_{n}^{2}(x)\) располагаются все ближе друг к другу. При очень больших значениях \(n\) картина рас-

пределения \(\psi_{n}^{2}(x)\) практически «сливается» и представляется равномерным - частица начинает вести себя совсем «по-классически».

Внимательный читатель по-видимому заметил, что найденные нами собственные функции (4.15) удовлетворяют не всем естественным условиям: на границах ямы \(\psi\)-функции не гладкие, испытывают излом. Это обстоятельство является следствием того, что на границах ямы \(U \rightarrow \infty\), чего в реальном мире не бывает. При любом конечном разрыве потенциальной энергии \(\psi\)-функция все равно остается гладкой (об этом подробнее ниже).

Заметим также, что в отличие от классики минимальное значение энергии \(E\) частицы в яме согласно (4.14) не равно нулю. Это полностью согласуется с принципом неопределенности. Ведь у частицы в яме ограничена область возможных значений ее координаты, поэтому должен существовать разброс по импульсам, а значит, отлична от нуля и энергия.

Случай 2. Частица движется в одномерном потенциальном поле \(U(x)\), показанном на рис. 4.3. Уже этот случай связан с достаточно громоздкими математическими преобразованиями.

Если полная энергия частицы \(E<U_{0}\), то говорят, что частица находится в потенциальной яме, или в связанном состоянии. Будь частица классической, она не смогла бы при этом условии выйти за пределы

\begin{figure}[h]
\begin{center}
  \includegraphics[alt={},max width=\textwidth]{a57b3a57-78b5-4292-b0ab-4bd9f2ffcf56-094_382_354_863_808}
\captionsetup{labelformat=empty}
\caption{Рис. 4.3}
\end{center}
\end{figure}

Существенно иначе ведут себя частицы, подчиняющиеся квантовым законам. Чтобы выяснить, как именно, воспользуемся уравнением Шредингера (4.9) в одномерном виде. Поскольку функция \(U(x)\), как видно из рис. 4.3 , является ступенчатой, то удобно разбить область изменения \(x\) на два участка, (1) и (2), с постоянными значениями \(U\), получить решения для каждого участка, а затем «сшить» эти решения так, чтобы \(\psi\)-функция была непрерывной и гладкой.

Снабдим решения на участке 1 индексом 1 , а на участке 2 индексом 2. Теперь запишем уравнение Шредингера для этих двух участков:

\[
\begin{array}{ll}
\psi_{1}^{\prime \prime}+k^{2} \psi_{1}=0, & k^{2}=2 m E / \hbar^{2} \\
\psi_{2}^{\prime \prime}+\varkappa^{2} \psi_{2}=0, & \varkappa^{2}=2 m\left(U_{0}-E\right) / \hbar^{2} \tag{4.17}
\end{array}
\]

Общие решения этих уравнений имеют вид

\[
\begin{aligned}
& \psi_{1}(x)=a \sin (k x+\alpha) \\
& \psi_{2}(x)=b \mathrm{e}^{-\chi x}+c \mathrm{e}^{\chi x}
\end{aligned}
\]

Они должны удовлетворять естественным условиям. Из условия непрерывности \(\psi\)-функции, учитывая, что при \(x \leqslant 0 \quad \psi_{1} \equiv 0\), имеем \(\psi_{1}(0)=0\), откуда \(\alpha=0\). Из требования конечности \(\psi\)-функции следует, что коэффициент \(c=0\), поскольку экспонента с положительным показателем соответствует непрерывному росту вероятности обнаружения частицы в области 2 с увеличением глубины проникновения \(x\). И наконец, требование непрерывности и гладкости \(\psi\)-функции в точке \(x=l\) означает, что

\[
\psi_{1}(l)=\psi_{2}(l), \quad \psi_{1}^{\prime}(l)=\psi_{2}^{\prime}(l) .
\]

Отсюда мы приходим к трансцендентному уравнению


\begin{equation*}
\operatorname{tg} k l=-k / \varkappa, \tag{4.18}
\end{equation*}


которое удобнее представить через синус по формуле

\[
\sin \alpha=1 / \sqrt{1+\operatorname{ctg}^{2} \alpha}
\]

В результате получим


\begin{equation*}
\sin k l= \pm C k l \tag{4.19}
\end{equation*}


где \(C=\hbar / \sqrt{2 m l^{2} U_{0}}\).\\
Изобразив графики левой и правой частей этого уравнения (рис. 4.4), найдем точки пересечения прямой с синусоидой. При этом корни данного уравнения, отвечающие собственным значениям \(E\), будут соответствовать тем точкам пересечения, для которых \(\operatorname{tg} k l<0\) согласно (4.18). Это значит, что корни уравнения (4.19) должны находиться в четньιx четвертях окруж-

ности (эти участки оси абсцисс выделены на рисунке жирными отрезками).

Из рис. 4.4 видно, что корни уравнения (4.19), т. е. связанные состояния, существуют в такой яме не всегда. Пунктиром показано предельное поло-

\begin{figure}[h]
\begin{center}
  \includegraphics[alt={},max width=\textwidth]{a57b3a57-78b5-4292-b0ab-4bd9f2ffcf56-096_289_531_141_629}
\captionsetup{labelformat=empty}
\caption{Рис. 4.4}
\end{center}
\end{figure}

Например, первый уровень, как следует из этого рисунка, появляется при условии \(k l=\pi / 2\), когда \(C k l=1\), откуда \(E=U_{0}\). Второй уровень - при \(k l=(3 / 2) \pi\) и т. д.

Таким образом, в данной яме при \(E<U_{0}\) спектр собственных значений энергии \(E\) оказывается дискретным. Этим значениям соответствуют связанные состояния частицы и характеризующие эти состояния \(ψ\)-функции, одна из которых

\begin{figure}[h]
\begin{center}
  \includegraphics[alt={},max width=\textwidth]{a57b3a57-78b5-4292-b0ab-4bd9f2ffcf56-096_198_537_730_621}
\captionsetup{labelformat=empty}
\caption{Рис. 4.5}
\end{center}
\end{figure}

Следует еще раз отметить, что такая потенциальная яма, как показывает расчет и график на рис. 4.4 , может не содержать и ни одного уровня (это будет при условии \(l^{2} U_{0}<\pi^{2} \hbar^{2} / 8 m\) ). В этом случае движение частицы не локализовано в конечной области, ее движение, как говорят, инфинитно.

Нельзя не обратить внимания на тот удивительный (с точки зрения классики) факт, что частица, будучи в связанном состоянии, может оказаться и в области 2 (см. рис. 4.3), где ее полная энергия \(E<U_{0}\). Объясняется это тем, что равенство \(E=K+U\) в квантовой теории теряет смысл: кинетическая \(K\) и потенциальная \(U\) энергии в силу принципа неопределенности не могут одновременно принимать точные значения. В самом деле, \(U\) зависит от координат, а \(K\) - от импульса частицы. Поэтому не следует удивляться тому, что в некоторых местах полная энергия \(E<U\).

Отметим также, что с ростом, например, глубины ямы, т. е. \(U_{0}\), число уровней энергии \(E\) и связанных состояний будет увеличиваться, а вероятность обнаружения частицы в области 2

будет становиться все меньше, и при \(U_{0} \rightarrow \infty\) она обратится в нуль, \(\psi\)-функция в точке \(x=l\) приобретает излом (теряет гладкость), с чем мы и столкнулись в случае 1 и наблюдаем в данной яме в точке \(x=0\).

Уместно здесь коснуться вопроса о гладкости ψ-функции в месте конечного разрыва* функции \(U(x)\). Проинтегрируем уравнение Шредингера по малому интервалу координаты \(x\), внутри которого имеется скачок \(U(x)\), например, в точке \(x=0\). В результате получим

\[
\frac{\partial \psi}{\partial x}(+\delta)-\frac{\partial \psi}{\partial x}(-\delta)=-\int_{-\delta}^{+\delta} \frac{2 m}{\hbar^{2}}(E-U) \mathrm{d} x,
\]

где координату \(x\) берем в малом интервале \((-\delta,+\delta)\). Ввиду конечности скачка \(U(0)\) интервал при \(\delta \rightarrow 0\) тоже стремится к нулю. Отсюда и следует, что слева и справа от точки \(x=0\) производные \(\partial \psi / \partial x\) будут одинаковы, значит \(\psi\)-функция оказывается гладкой.

\section*{§ 4.4. Квантовый гармонический осциллятор}
Задача об уровнях энергии одномерного гармонического осциллятора является одной из наиболее важных задач о собственных значениях.

В квантовой теории понятие силы теряет смысл (см. сноску на стр. 85), поэтому квантовый гармонический осциллятор следует определить как поведение частицы массы \(m\) с потенциальной энергией \(U(x)\) такой же, как у классического осциллятора, а именно


\begin{equation*}
U=x x^{2} / 2, \tag{4.20}
\end{equation*}


где \(x\) - постоянная. Графиком функции (4.20) является парабола (рис. 4.6). Согласно классической механике осциллятор совершает гармонические колебания с циклической частотой

\footnotetext{\begin{itemize}
  \item Говоря о «разрыве», мы должны понимать этот термин не в математическом, а в физическом смысле: функция \(U(x)\) меняется от одного значения до другого в очень малой области пространства, испытывая по существу скачок. Именно поэтому в таком месте график \(U(x)\) изображают практически вертикальным отрезком.
\end{itemize}
}
\(\omega=\sqrt{\varkappa / m}\). В квантовой теории это равенство следует рассматривать просто как введение некоторой новой постоянной (и не более), однако, как будет видно в дальнейшем, это делается неспроста. Сейчас же, выразив в формуле (4.20) \(x\) через \(\omega\) и \(m\), получим

\begin{equation*}
U=\frac{m \omega^{2}}{2} x^{2} . \tag{4.21}
\end{equation*}


\begin{figure}[h]
\begin{center}
  \includegraphics[alt={},max width=\textwidth]{a57b3a57-78b5-4292-b0ab-4bd9f2ffcf56-098_321_275_145_875}
\captionsetup{labelformat=empty}
\caption{Рис. 4.6}
\end{center}
\end{figure}

Теперь обратимся к уравнению Шредингера (4.9), которое в нашем одномерном случае будет иметь вид


\begin{equation*}
\frac{\partial^{2} \psi}{\partial x^{2}}+\frac{2 m}{\hbar^{2}}\left(E-\frac{m \omega^{2}}{2} x^{2}\right) \psi=0 . \tag{4.22}
\end{equation*}


Нахождение решения этого уравнения, т. е. ψ-функции, является громоздкой математической задачей. Для нас главное не в этом. Оказывается, уравнение (4.22) имеет конечные, однозначные, непрерывные и гладкие решения (собственные функции) при собственных значениях \(E\), равных


\begin{equation*}
E_{v}=\left(v+\frac{1}{2}\right) \hbar \omega, \quad v=0,1,2, \ldots \tag{4.23}
\end{equation*}


Схема соответствующих энергетических уровней (4.23) дана на рис. 4.7. Видно, что эти уровни - эквидистантны, т. е. отстоят друг от друга на одинаковую величину. Минимальная энергия \(E_{0}=\hbar \omega / 2\), ее называют нулевой энергией.

То, что минимальная энергия квантового осциллятора не равна нулю (частица не может «лежать» в нижней точке параболической потенциальной ямы), связано с принципом неопределенности, как и в слу-

\begin{figure}[h]
\begin{center}
  \includegraphics[alt={},max width=\textwidth]{a57b3a57-78b5-4292-b0ab-4bd9f2ffcf56-098_370_338_1194_822}
\captionsetup{labelformat=empty}
\caption{Рис. 4.7}
\end{center}
\end{figure}

Наличие нулевой энергии подтверждается экспериментально.\\
Более детальный расчет, выходящий за рамки уравнения Шредингера, показывает, что для квантового осциллятора возможны переходы лишь между соседними «стационарными» уровнями, при которых квантовое число \(v\) изменяется на единицу:


\begin{equation*}
\Delta v= \pm 1 . \tag{4.24}
\end{equation*}


Это условие называют правилом отбора для квантового гармонического осциллятора.

При каждом из этих переходов испускается или поглощается фотон с энергией ћω, где \(\omega\) - его циклическая частота. Именно здесь введенная ранее постоянная \(\omega\) приобретает физический смысл. Говорить же, что в стационарных состояниях квантовый осциллятор испытывает колебания с частотой \(\omega\), это в принципе неверно. Дело обстоит совершенно иначе. Поясним это с помощью рис. 4.8, где приведены графики распределения плотности вероятности \(\psi^{2}(x)\) местоположения частицы при \(v=0\), 1,2 и при большом значении \(v\). Жирными отрезками на оси \(X\) показаны интервалы, на концах которых \(E=U\). Классическая частица при колебаниях за пределы интервала заходить не может. Квантовая же частица ведет себя совершенно не так. Она, как видно из рисунка, может быть обнаружена и вне пределов этих интервалов, где \(E<U\). И ни о каких колебаниях квантового осциллятора в стационарных состояниях речи быть не может. Мы можем говорить лишь о распределении плотности вероятности местоположения частицы. С ростом квантового числа квантовый осциллятор все больше становится классическим, у которого плотность вероятности плавно изменяется от минимума при \(x=0\) до бесконечности в точках поворота (где \(E=U\) ), т. е. совершенно противоположно тому, что мы имеем для квантового осциллятора, например, в состоянии с \(v=0\) (см. рис. 4.8).

\begin{figure}[h]
\begin{center}
  \includegraphics[alt={},max width=\textwidth]{a57b3a57-78b5-4292-b0ab-4bd9f2ffcf56-099_197_1100_1536_64}
\captionsetup{labelformat=empty}
\caption{Рис. 4.8}
\end{center}
\end{figure}

Колебания молекул. В атомной физике к осциллятору сводится задача о колебаниях молекул и многие другие важные задачи. Применим полученные выводы к колебаниям, например, двухатомных молекул.

На рис. 4.9 изображена потенциальная энергия \(U\) взаимодействия атомов в двухатомной молекуле (типа NaCl ) в зависимости от расстояния \(r\) между ядрами атомов. Из вида кривой \(U(r)\) следует, что атомы в молекуле могут совершать колебания относительно равновесного расстояния \(r_{0}\) между ядрами, и у молекулы, следовательно, должны суще-

\begin{figure}[h]
\begin{center}
  \includegraphics[alt={},max width=\textwidth]{a57b3a57-78b5-4292-b0ab-4bd9f2ffcf56-100_354_449_330_711}
\captionsetup{labelformat=empty}
\caption{Рис. 4.9}
\end{center}
\end{figure}

Приведем в качестве примера циклические частоты \(\omega\) некоторых двухатомных молекул:

\begin{center}
\begin{tabular}{cc||cc}
Молекула & \(\omega, 10^{14} \mathrm{c}^{-1}\) & Молекула & \(\omega, 10^{14} \mathrm{c}^{-1}\) \\
\hline
\(\mathrm{H}_{2}\) & 8,279 & HCl & 5,632 \\
\(\mathrm{O}_{2}\) & 2,997 & CO & 4,088 \\
\end{tabular}
\end{center}

Нижняя часть потенциальной кривой на рис. 4.9 совпадает с параболой (она изображена пунктиром), поэтому при малых колебаниях молекулы ведут себя как идеальные, гармонические осцилляторы, и их нижние колебательные уровни должны быть эквидистантны, как показано на рис. 4.10.

Наличие дискретных колебательных уровней

\begin{figure}[h]
\begin{center}
  \includegraphics[alt={},max width=\textwidth]{a57b3a57-78b5-4292-b0ab-4bd9f2ffcf56-100_321_249_1202_911}
\captionsetup{labelformat=empty}
\caption{Рис. 4.10}
\end{center}
\end{figure}

приводит к появлению в молекулярных спектрах линий, связанных с переходами между этими уровнями в соответствии с правилом отбора (4.24), и поэтому весь колебательный спектр должен состоять из одной линии (см. рис. 4.10). Впрочем при этом наблюдается не чисто колебательный, а так называемый колебательно-вращательный спектр (см. § 5.3).

Ангармоничность (отклонение от гармоничности), наступающая при увеличении интенсивности колебаний, приводит к тому, что с увеличением квантового числа \(v\) энергетические уровни сгущаются, и в формулу (4.23) необходимо вводить поправку на ангармоничность.

\section*{§ 4.5. Потенциальные барьеры}
Сначала рассмотрим простейший случай - прямоугольный потенциальный барьер, когда потенциальная энергия \(U\) зависит только от одной координаты \(x\), причем при \(x=0\) претерпевает скачок (рис. 4.11). У такого барьера

\begin{figure}[h]
\begin{center}
  \includegraphics[alt={},max width=\textwidth]{a57b3a57-78b5-4292-b0ab-4bd9f2ffcf56-101_281_314_657_68}
\captionsetup{labelformat=empty}
\caption{Рис. 4.11}
\end{center}
\end{figure}

\[
U(x)= \begin{cases}0 & \text { при } x<0,  \tag{4.25}\\ U_{0} & \text { при } x>0 .\end{cases}
\]

Пусть слева на границу барьера налетает с полной энергией \(E\) частица или поток частиц. На языке квантовой теории это означает, что на барьер слева «падает» дебройлевская волна


\begin{equation*}
\Psi(x, t)=a \mathrm{e}^{\mathrm{i}(k x-\omega t)} . \tag{4.26}
\end{equation*}


Чтобы удовлетворить граничным условиям для \(\Psi\) и \(\partial \Psi / \partial x\) при \(x=0\), должны существовать как прошедшая волна, так и отраженная. В этих трех волнах частота \(\omega\) одна и та же ( \(\omega=E / \hbar\) ), поэтому в дальнейших расчетах мы можем ограничиться только координатной частью этих волн, а именно \(\psi(x)\).

Наша задача: сначала найти амплитуды отраженной и падающей волн, а затем - коэффициенты отражения \(R\) и пропускания \(D\) для такого барьера. Исходим из уравнения Шредингера (4.9). В нашем случае оно имеет вид


\begin{equation*}
\psi_{x}^{\prime \prime}+k^{2} \psi=0, \quad k^{2}=2 m\left(E-U_{0}\right) / \hbar^{2} . \tag{4.27}
\end{equation*}


Здесь возможны два случая (см. рис. 4.11): \(E>U_{0}\) и \(E<U_{0}\).

\begin{enumerate}
  \item В случае \(\boldsymbol{E}>\boldsymbol{U}_{0}\) общее решение уравнения (4.27) имеет вид:
\end{enumerate}

\[
\begin{array}{ll}
\psi_{1}(x \leqslant 0)=a_{1} \exp \left(\mathrm{i} k_{1} x\right)+b_{1} \exp \left(-\mathrm{i} k_{1} x\right), & k_{1}=\sqrt{2 m E} / \hbar .  \tag{4.28}\\
\psi_{2}(x \geqslant 0)=a_{2} \exp \left(\mathrm{i} k_{2} x\right)+b_{2} \exp \left(-\mathrm{i} k_{2} x\right), & k_{2}=\sqrt{2 m\left(E-U_{0}\right)} / \hbar .
\end{array}
\]

Будем считать, что падающая волна характеризуется амплитудой \(a_{1}\), причем вещественной, а отраженная - амплитудой \(b_{1}\). В области \(x>0\) имеется только проходящая волна, поэтому \(b_{2}=0\). Из условия непрерывности \(\psi\) и \(\psi_{x}^{\prime}\) в точке \(x=0\) следует, что


\begin{align*}
& \psi_{1}(0)=\psi_{2}(0), \quad \text { или } \quad a_{1}+b_{1}=a_{2},  \tag{4.29}\\
& \psi_{1}^{\prime}(0)=\psi_{2}^{\prime}(0), \quad \text { или } \quad a_{1} k_{1}+b_{1} k_{1}=a_{2} k_{2} .
\end{align*}


Из совместного решения этих двух уравнений находим, что отношения амплитуд отраженной и прошедшей волн к амплитуде \(a_{1}\) падающей волны равны:


\begin{equation*}
\frac{b_{1}}{a_{1}}=\frac{k_{1}-k_{2}}{k_{1}+k_{2}}, \quad \frac{a_{2}}{a_{1}}=\frac{2 k_{1}}{k_{1}+k_{2}} . \tag{4.30}
\end{equation*}


Для определения интересующих нас коэффициентов \(R\) и \(D\) введем понятие плотности потока вероятности \(\mathcal{P}\). Скорость распространения вероятности такого потока просто совпадает с классической скоростью \(v\) частицы, и мы можем написать \(v=p / m=\hbar k / m\), поскольку согласно (3.1) \(p=\hbar k\). Таким образом,

\[
v \sim k,
\]

и плотность потока вероятности пропорциональна величине \(k \Psi \Psi^{*}\) :

\[
\mathcal{P} \sim k \Psi \Psi^{*} .
\]

В соответствии с видом Ψ-функции (4.26) для падающей, отраженной и прошедшей волн мы имеем

\[
\mathcal{P} \circlearrowleft k_{1} a_{1}^{2}, \quad \mathcal{P}^{\prime} \circlearrowleft k_{1} b_{1}{ }^{2}, \quad \mathcal{P} \prime \circ k_{2} a_{2}^{2} .
\]

Теперь можно записать выражения для коэффициентов отражения \(R\) и пропускания \(D\) :


\begin{equation*}
R=\frac{\mathcal{P}^{\prime}}{\mathcal{P}}=\left(\frac{b_{1}}{a_{1}}\right)=\left(\frac{k_{1}-k_{2}}{k_{1}+k_{2}}\right)^{2}, D=\frac{\mathcal{P}^{\prime \prime}}{\mathcal{P}}=\frac{k_{2}}{k_{1}}\left(\frac{a_{2}}{a_{1}}\right)^{2}=\frac{4 k_{1} k_{2}}{\left(k_{1}+k_{2}\right)^{2}} . \tag{4.31}
\end{equation*}


Отсюда следует, что \(R+D=1\), что и должно быть по определению. Кроме того, видно, что значения \(R\) и \(D\) не зависят от на-

правления движения частицы: слева направо на рис. 4.11 или наоборот.

Заметим, что в классическом случае \(R=0\) при \(E>U_{0}\).\\
2. В случае \(\boldsymbol{E}<\boldsymbol{U}_{0}\) формулы (4.30) остаются справедливыми. Однако \(k_{2}\) будет чисто мнимым согласно (4.28). При этом выражение (4.31) для коэффициента отражения следует записать так:


\begin{equation*}
R=\left|\frac{k_{1}-k_{2}}{k_{1}+k_{2}}\right|^{2} . \tag{4.32}
\end{equation*}


Здесь числитель и знаменатель - величины комплексно-сопряженные. Значит \(R=1\), т. е. отражение частиц будет полным. Но \(\psi\)-функция при \(x>0\) не обращается в нуль. В самом деле, полагая \(k_{2}=\mathrm{i} k\), где \(k=\sqrt{2 m\left(U_{0}-E\right)} / \hbar\), получим, что \(\psi_{2}\) c \(\mathrm{e}^{-k x}\) и плотность вероятности местоположения частицы


\begin{equation*}
P(x)=P(0) \mathrm{e}^{-2 k x} . \tag{4.33}
\end{equation*}


Видно, что с увеличением глубины проникновения \(x\) плотность вероятности \(P(x)\) убывает экспоненциально. Это убывание происходит тем быстрее, чем больше разность ( \(U_{0}-E\) ). Обычно глубину проникновения определяют как расстояние \(l\), на котором \(P(x)\) убывает в е раз. При этом в \((4.33) 2 k l=1\) и


\begin{equation*}
l=1 / 2 k=\hbar / \sqrt{8 m\left(U_{0}-E\right)} . \tag{4.34}
\end{equation*}


Можно убедиться, что для электрона при \(U_{0}-E \approx 10^{-3}\) эВ глубина проникновения \(l \approx 10^{-7}\) см.

Таким образом \(\psi\)-функция проникает в область \(x>0\), несмотря на то, что падающая волна отражается полностью.

В классической физике проникновение частиц под барьер запрещено, поскольку в этой области кинетическая энергия оказывается отрицательной, чего быть не может. Но мы уже знаем, что разделение полной энергии \(E\) на кинетическую и потенциальную не совместимо с соотношением неопределенностей (3.20), см. также стр. 95.

Туннельный эффект. Способность квантовых частиц в силу своих волновых свойств заходить под барьер приводит к так называемому туннельному эффекту. Он заключается в следующем. Если частица с энергией \(E\) налетает на некоторый потен-

циальный барьер \(U(x)\), то она с определенной вероятностью может пройти сквозь барьер как бы по туннелю, т. е. пройти область, где \(E<U\).

В качестве иллюстрации приведем результаты расчета плотности вероятности \(P(x)\) местоположения частицы, налетающей слева на простейший прямоугольный потенциальный барьер, показанный на рис. 4.12. Слева от барьера мы имеем падающую и отраженную волны, а за барьером - только прошедшую волну. Внутри барьера ψ-функция имеет не волновой характер, в резуль-

\begin{figure}[h]
\begin{center}
  \includegraphics[alt={},max width=\textwidth]{a57b3a57-78b5-4292-b0ab-4bd9f2ffcf56-104_306_303_294_855}
\captionsetup{labelformat=empty}
\caption{Рис. 4.12}
\end{center}
\end{figure}

тате чего \(P(x)\) убывает практически экспоненциально.

Соответствующий расчет показывает, что в случае потенциального барьера произвольной формы (рис. 4.13) вероятность прохождения частицы сквозь барьер, т. е. коэффициент прозрачности


\begin{equation*}
D \approx \exp \left(-\frac{2}{\hbar} \int_{x_{1}}^{x_{2}} \sqrt{2 m(U-E)} \mathrm{d} x\right) . \tag{4.35}
\end{equation*}


Это приближенное равенство, оно тем точнее, чем меньше ( \(U-E\) ) по сравнению с \(E\).

Туннельный эффект - специфически квантовое явление, не имеющее аналога в классической физике (где такого в принципе не может быть).

\begin{figure}[h]
\begin{center}
  \includegraphics[alt={},max width=\textwidth]{a57b3a57-78b5-4292-b0ab-4bd9f2ffcf56-104_253_410_1089_746}
\captionsetup{labelformat=empty}
\caption{Рис. 4.13}
\end{center}
\end{figure}

Этим эффектом объясняются многие физические явления; например, холодная эмиссия электронов из металлов, альфа-распад, спонтанное деление ядер и др.

\section*{Задачи}
4.1. Свободное движение частицы. Найти решение временного уравнения Шредингера (4.5) для свободной частицы массы \(m\), движущейся с импульсом \(p\) в положительном направлении оси \(X\).

Р е ш е н и е. В этом случае потенциальную энергию частицы можно считать равной нулю \(U(x)=0\), и уравнение (4.5) примет вид


\begin{equation*}
\mathrm{i} \frac{\partial \Psi}{\partial t}=-\frac{\hbar}{2 m} \frac{\partial^{2} \Psi}{\partial x^{2}} \tag{1}
\end{equation*}


Его решение будем искать методом разделения переменных, т. е. представим Ψ в виде произведения двух функций, одна из которых зависит только от \(x\), другая - только от \(t\) :


\begin{equation*}
\Psi(x, t)=\psi(x) \cdot f(t) . \tag{2}
\end{equation*}


Подставив (2) в (1), приходим к двум независимым уравнениям:


\begin{equation*}
\mathrm{i} \hbar \frac{\dot{f}}{f}=-\frac{\hbar^{2}}{2 m} \frac{\psi^{\prime \prime}}{\psi}, \tag{3}
\end{equation*}


где \(\dot{f}\) - производная по \(t, ψ^{\prime \prime}\) - вторая производная по \(x\). Так как обе части этого уравнения являются функциями независимых переменных \(t\) и \(x\), то равенство (3) возможно лишь в том случае, если обе его части равны одной и той же константе. Из сравнения выражения (3) с уравнением Шредингера (4.9) видно, что эта константа равна \(E\). Таким образом мы получаем два уравнения:


\begin{equation*}
\psi^{\prime \prime}+\frac{2 m}{\hbar^{2}} E \psi=0, \quad \dot{f}+\mathrm{i} \frac{\mathrm{E}}{\hbar} f=0 . \tag{4}
\end{equation*}


Их решения, как можно в этом убедиться непосредственной подстановкой, таковы:


\begin{gather*}
\psi(x) \sim \mathrm{e}^{ \pm \mathrm{i} k x}, \quad k=\sqrt{2 m E} / \hbar=p / \hbar, \\
f(t) \sim \mathrm{e}^{-\mathrm{i} \omega t}, \quad \omega=E / \hbar, \tag{5}
\end{gather*}


где значения \(k\) и \(\omega\) записаны в соответствии с постулатами корпу-скулярно-волнового дуализма.

В результате искомое решение согласно (2) будет иметь вид

\[
\Psi(x, t)=A e^{\mathrm{i}(k x-\omega t)}
\]

Это решение будет конечным лишь при \(E>0\), причем при любых значениях \(E\).

Именно такой вид имеет дебройлевская волна.\\
Плотность вероятности местоположения соответствующей частицы

\[
P(x)=\Psi \Psi^{*}=A A^{*}=\text { const. }
\]

Это означает равновероятность местонахождения такой частицы во всех точках пространства (оси \(X\) ). Данный вывод вполне согласуется с соотношением неопределенностей: при \(\Delta p_{x}=0 x \rightarrow \infty\), т. е. частица «размазана» равномерно по всему пространству.\\
4.2. Частица в прямоугольной яме с бесконечно высокими стенками. Частица находится в основном состоянии в одномерной прямоугольной потенциальной яме шириной \(l\) с абсолютно непроницаемыми стенками ( \(0<x<l\) ). Найти вероятность местонахождения частицы в интервале \((l / 3,2 l / 3)\).

Р е ш е н и е. Согласно (4.15) ψ-функция в основном состоянии \((n=1)\) это \(\psi=\sqrt{2 / l} \sin (\pi x / l)\). Искомая вероятность

\[
P=\int_{x_{1}}^{x_{2}} \psi^{2}(x) \mathrm{d} x=\frac{2}{\pi}\left(\frac{y}{2}-\frac{\sin 2 y}{4}\right)_{y_{1}}^{y_{2}}=\frac{1}{3}+\frac{\sqrt{3}}{2 \pi} \approx 0,61
\]

где введена новая переменная \(y=\pi x / l\).\\
4.3. Найти энергию \(E\) стационарного состояния частицы массы \(m\) в одномерной прямоугольной потенциальной яме шириной \(l\) с абсолютно непроницаемыми стенками, если на границе ямы ( \(x=0\) ) известно значение производной \(\partial \psi / \partial x\), т. е. \(\psi^{\prime}(0)\).

Решение. Известно, что \(\psi\)-функция \(n\)-го стационарного состояния определяется формулой (4.15). Взяв ее производную по \(x\) и положив затем \(x=0\), получим:

\[
\frac{\partial \psi}{\partial x}=\left.\sqrt{\frac{2}{l}} \frac{\pi n}{l} \cos \frac{\pi n x}{l}\right|_{x=0}=\frac{\pi \sqrt{2}}{l^{3 / 2}} n .
\]

Отсюда находим

\[
n=\frac{l^{3 / 2}}{\pi \sqrt{2}} \psi^{\prime}(0)
\]

Подставив это выражение в формулу (4.14) для энергии, имеем

\[
E=\frac{l \hbar^{2}}{4 m}\left[\psi^{\prime}(0)\right]^{2}
\]

4.4. Электрон находится в одномерной прямоугольной потенциальной яме с бесконечно высокими стенками. Ширина ямы равна \(l\) и такова, что энергетические уровни расположены весьма густо. Найти плотность этих уровней \(\mathrm{d} N / \mathrm{d} E\), т. е. их число на единичный интервал энергии, в зависимости от \(E\). Вычислить \(\mathrm{d} N / \mathrm{d} E\), если \(E=1,0\) эВ и \(l=1,0\) см.

Решение. Возьмем дифференциал натурального логарифма от выражения (4.14) для энергии \(E\) :

\[
\frac{\mathrm{d} E}{E}=2 \frac{\mathrm{~d} n}{n}
\]

Отсюда

\[
\frac{\mathrm{d} N}{\mathrm{~d} E}=\frac{\mathrm{d} n}{\mathrm{~d} E}=\frac{1}{2} \frac{n}{E}=\frac{l}{\pi \hbar} \sqrt{\frac{m}{2 E}},
\]

где \(n\) выражено через \(E\) с помощью (4.14), \(m\) - масса электрона. Для заданных значений \(E\) и \(l\)

\[
\mathrm{d} N / \mathrm{d} E=0,8 \cdot 10^{7} \text { уровней/эВ. }
\]

4.5. Частица массы \(m\) находится в основном состоянии в одномерной прямоугольной потенциальной яме с абсолютно непроницаемыми стенками. Максимальное значение плотности вероятности местонахождения частицы в этом состоянии равно \(P_{m}\). Найти ширину \(l\) ямы и энергию \(E\) частицы.

Решение. Воспользовавшись выражением (4.15) для ψ-функции, запишем плотность вероятности \(P(x)\) для основного состояния ( \(n=1\) ):

\[
P(x)=\psi^{2}=\frac{2}{l} \sin ^{2} \frac{\pi x}{l} .
\]

Эта величина максимальна в середине ямы, т. е. при \(x=l / 2\). Поэтому

\[
P_{m}=\frac{2}{l} \sin ^{2} \frac{\pi}{2}=\frac{2}{l}
\]

Отсюда находим \(l=2 / P_{m}\) и согласно (4.14)

\[
E=\frac{\pi^{2} \hbar^{2}}{8 m} P_{m}
\]

4.6. Частица массы \(m\) находится в двумерной прямоугольной потенциальной яме с абсолютно непроницаемыми стенками. Координаты \(x\) и \(y\) частицы находятся в интервалах соответственно ( \(0, a\) ) и ( \(0, b\) ), где \(a\) и \(b\) - стороны ямы. Найти возможные значения энергии \(E\) и нормированные \(\psi\)-функции частицы.

Решение. В этом случае уравнение Шредингера (4.9) имеет вид


\begin{equation*}
\frac{\partial^{2} \psi}{\partial x^{2}}+\frac{\partial^{2} \psi}{\partial y^{2}}+k^{2} \psi=0, \quad k^{2}=2 m E / \hbar^{2} \tag{1}
\end{equation*}


(в пределах ямы мы считаем \(U=0\) ).\\
На сторонах ямы ψ-функция должна обращаться в нуль, поскольку является непрерывной (за пределами ямы \(\psi=0\) ). Поэтому (ψ-функцию внутри ямы удобно искать сразу в виде произведения синусов


\begin{equation*}
\psi(x, y)=A \sin k_{1} x \cdot \sin k_{2} y, \tag{2}
\end{equation*}


так как на двух сторонах ( \(x=0\) и \(y=0\) ) автоматически \(\psi(x, 0)\) и \(\psi(0, y)\) равны нулю.

Возможные значения \(k_{1}\) и \(k_{2}\) найдем из условия обращения \(\psi\)-функции в нуль на противоположных сторонах ямы:

\[
\begin{array}{lll}
\psi(a, y)=0, & k_{1}= \pm \frac{\pi}{a} n_{1}, & n_{1}=1,2,3, \ldots \\
\psi(x, b)=0, & k_{2}= \pm \frac{\pi}{b} n_{2}, & n_{2}=1,2,3, \ldots \tag{3}
\end{array}
\]

После подстановки (2) в уравнение (1) получим \(k_{1}{ }^{2}+k_{2}{ }^{2}=k^{2}\), и, учитывая выражение для \(k^{2}\) в (1) и формулы (3) для \(k_{1}\) и \(k_{2}\), получим


\begin{equation*}
E_{n_{1} n_{2}}=\frac{\pi^{2} \hbar^{2}}{2 m}\left(\frac{n_{1}^{2}}{a^{2}}+\frac{n_{2}^{2}}{b^{2}}\right) \tag{4}
\end{equation*}


Постоянную \(A\) в (2) находим из условия нормировки

\[
\int_{0}^{a} \int_{0}^{b} \psi^{2}(x, y) \mathrm{d} x \mathrm{~d} y=1
\]

откуда следует, что \(A=\sqrt{4 / a b}=2 / \sqrt{a b}\). Следовательно, нормированная \(\psi\)-функция будет иметь вид

\[
\psi(x, y)=\frac{2}{\sqrt{a b}} \sin \left(\frac{\pi x}{a} n_{1}\right) \sin \left(\frac{\pi y}{b} n_{2}\right) .
\]

4.7. Частица массы \(m\) находится в двумерной квадратной потенциальной яме с абсолютно непроницаемыми стенками. Сторона ямы равна \(l\). Воспользовавшись результатами решения предыдущей задачи, найти значения энергии \(E\) для первых четырех уровней.

P е ш е н и е. В данном случае

\[
E=\frac{\pi^{2} \hbar^{2}}{2 m l^{2}}\left(n_{1}^{2}+n_{2}^{2}\right) .
\]

Задача сводится к подбору таких наименьших значений \(n_{1}\) и \(n_{2}\), при которых \(n_{1}^{2}+n_{2}^{2}\) имеет четыре наименьших значения. Составим табличку

\begin{center}
\begin{tabular}{c|c|c|c}
\(n_{1}\) & \(n_{2}\) & \(n_{1}^{2}+n_{2}^{2}\) & номер уровня \\
\hline
1 & 1 & 2 & 1 \\
1 & 2 & 5 & 2 \\
2 & 2 & 8 & 3 \\
1 & 3 & 10 & 4 \\
2 & 3 & 13 & 5 \\
1 & 4 & 17 & 6 и т. д. \\
\end{tabular}
\end{center}

Отсюда видно, что энергия \(E\) первых четырех уровней

\[
E=2,5,8 \text { и } 10 \text { единиц } \pi^{2} \hbar^{2} / 2 m l^{2} .
\]

4.8. Воспользовавшись условием и решением задачи 4.6 , найти число \(\mathrm{d} N\) состояний частицы в интервале энергии ( \(E, E+\mathrm{d} E\) ), полагая, что энергетические уровни расположены весьма густо.

P е ш е н и е. Каждому значению пары чисел \(n_{1}\) и \(n_{2}\) отвечает одно состояние частицы. Число состояний в интервале ( \(\mathrm{d} n_{1}, \mathrm{~d} n_{2}\) ) вблизи значений \(n_{1}\) и \(n_{2}\) равно

\[
\mathrm{d} N=\mathrm{d} n_{1} \cdot \mathrm{~d} n_{2} .
\]

Имея в виду уравнение \(k_{1}^{2}+k_{2}^{2}=k^{2}\), где \(k_{1}=n_{1} \pi / a, k_{2}=n_{2} \pi / b\), отложим на осях координат величины \(k_{1}\) и \(k_{2}\). Построим затем в этом « \(k\)-пространстве» окружность радиуса \(k\) с центром в начале координат. Точки, попадающие на эту окружность, соответствуют одному и тому же значению \(k\), а значит одной и той же энергии \(E\). Нас будет интересовать только \(1 / 4\) окружности, поскольку следует рассматривать лишь положительные значения \(k_{1}\) и \(k_{2}\) : отрица-

тельные значения не дают новых состояний, как видно из выражения для \(\psi\)-функции.

Число точек (состояний), заключенных между двумя окружностями с радиусами \(k\) и \(k+\mathrm{d} k\) в первой четверти (рис. 4.14) равно


\begin{equation*}
\mathrm{d} N=\int \mathrm{d} n_{1} \mathrm{~d} n_{2}=\int \frac{a b}{\pi^{2}} \mathrm{~d} k_{1} \mathrm{~d} k_{2}=\frac{1}{4} \frac{a b}{\pi^{2}} 2 \pi k \mathrm{~d} k \tag{*}
\end{equation*}


Имея в виду, что \(k^{2}=2 m E / \hbar^{2}\), получим \(2 k \mathrm{~d} k=2 m \mathrm{~d} E / \hbar^{2}\), и в результате подстановки в (*) найдем:

\[
\mathrm{d} N=\frac{1}{4} \frac{a b}{\pi^{2}} \pi 2 m \frac{\mathrm{~d} E}{\hbar^{2}}=\frac{a b}{2 \pi \hbar^{2}} \mathrm{~d} E
\]

Удивительно, что плотность состояний \(\mathrm{d} N / \mathrm{d} E\) в такой яме от \(E\) не зависит. Заметим, что в прямоугольной (не квадратной) яме расчет показывает: \(\mathrm{d} N / \mathrm{d} E \sim \sqrt{E}\).

\begin{figure}[h]
\begin{center}
  \includegraphics[alt={},max width=\textwidth]{a57b3a57-78b5-4292-b0ab-4bd9f2ffcf56-110_421_424_450_738}
\captionsetup{labelformat=empty}
\caption{Рис. 4.14}
\end{center}
\end{figure}

4.9. Частица массы \(m\) находится в одномерном потенциальном поле \(U(x)\) в стационарном состоянии \(\psi=A \exp \left(-\beta x^{2}\right)\), где \(A\) и \(\beta\) - постоянные ( \(\beta>0\) ). Найти энергию \(E\) частицы и вид функции \(U(x)\), если \(U(0)=0\).

Р ешени е. Сначала найдем вторую производную \(\psi(x)\) по \(x\) :

\[
\begin{gathered}
\psi^{\prime}=-2 A \beta x \exp \left(-\beta x^{2}\right), \\
\psi^{\prime \prime}=-2 A\left[\exp \left(-\beta x^{2}\right)+x \exp \left(-\beta x^{2}\right)(-2 \beta x)\right]=-2 A \beta\left(1-2 \beta x^{2}\right) \exp \left(-\beta x^{2}\right) .
\end{gathered}
\]

Теперь подставим \(\psi^{\prime \prime}\) и \(\psi\) в уравнение Шредингера:

\[
\psi^{\prime \prime}+\frac{2 m}{\hbar^{2}}(E-U) \psi=0
\]

После сокращения на экспоненту получим:


\begin{equation*}
-2 \beta+4 \beta^{2} x^{2}+\frac{2 m}{\hbar^{2}}(E-U)=0 \tag{1}
\end{equation*}


Полагая в этом равенстве \(x=0\) и соответственно \(U(0)=0\), имеем


\begin{equation*}
-2 \beta+\frac{2 m E}{\hbar^{2}}=0 \tag{2}
\end{equation*}


откуда \(E=\beta \hbar^{2} / m\).\\
Учитывая (2), находим из (1), что

\[
U(x)=\frac{2 \beta^{2} \hbar^{2}}{m} x^{2} .
\]

4.10. Прохождение частицы через порог. Частица массы \(m\) движется слева направо в потенциальном поле (рис. 4.15), которое в точке

\begin{figure}[h]
\begin{center}
  \includegraphics[alt={},max width=\textwidth]{a57b3a57-78b5-4292-b0ab-4bd9f2ffcf56-111_238_320_470_156}
\captionsetup{labelformat=empty}
\caption{Рис. 4.15}
\end{center}
\end{figure}

\(x=0\) испытывает скачок \(U_{0}\). При \(x<0\) энергия частицы равна \(E\). Найти коэффициент отражения \(R\), если \(E \ll U_{0}\).

Решение. Здесь следует повторить рассуждения, приведенные в § 4.5 для случая 1. Отличие заключается лишь в том, что в выражении для \(k_{2}(4.28)\) должно, как видно из рис. 4.15 , стоять не \(E-U_{0}\), а \(E+U_{0}\).\\
Таким образом, искомый коэффициент с учетом того, что \(k_{1} \ll k_{2}\), можно записать так:\\
\(R=\left(\frac{k_{1}-k_{2}}{k_{1}+k_{2}}\right)^{2} \approx\left(1-2 \frac{k_{1}}{k_{2}}\right)\left(1-2 \frac{k_{1}}{k_{2}}\right) \approx 1-4 \frac{k_{1}}{k_{2}}=1-4 \sqrt{E / U_{0}}\)\\
(здесь мы пренебрегли величиной \(k_{1} / k_{2}\) в квадрате).\\
Отсюда следует (чисто квантовый эффект), что чем меньше \(E\), тем ближе \(R\) к единице. С классической точки зрения это в принципе невозможно.

\section*{Основы квантовой теории}
\section*{§ 5.1. Операторы физических величин}
В предыдущей главе было показано, что состояние квантовой частицы определяется не координатами и импульсом, а заданием \(\psi\)-функции, вид которой зависит от конкретного потенциального поля. Кроме того, как выяснилось, \(\psi\)-функция, описывающая сама по себе распределение по координатам, определяет также распределение по импульсам и другим динамическим характеристикам частицы, таким как кинетическая энергия, момент импульса и др.

Таким образом, \(\psi\)-функция полностью определяет не только «положение» частицы, но и все ее динамические характеристики. Надо только знать рецепты, с помощью которых можно «извлечь» интересующую нас информацию из ψ-функции. К решению этой задачи мы и приступаем.

Средние значения физических величин. Понятие среднего значения различных физических величин является весьма важным в квантовой теории. Рассмотрим этот вопрос на конкретном примере - определим среднее значение координаты \(x\) частицы, если известна ее \(\psi\)-функция, которую мы ради простоты будем считать функцией только одной пространственной координаты \(x\).

Мы уже знаем, что \(|\psi(x)|^{2}\) или \(\psi(x) \psi^{*}(x)\) является плотностью вероятности найти частицу в окрестности координаты \(x\). Тогда вероятность местонахождения частицы в интервале \((x, x+\mathrm{d} x)\) есть \(\mathrm{d} P=\psi \psi^{*} \mathrm{~d} x\), и среднее значение \(x\) определяется как


\begin{equation*}
\langle x\rangle=\int x \psi \psi^{*} \mathrm{~d} x \tag{5.1}
\end{equation*}


где интегрирование проводится по интересующей нас области. При этом предполагается, что \(ψ\)-функция является нормированной в (5.1). т. е. удовлетворяет условию (4.3):

\[
\int_{-\infty}^{+\infty} \psi \psi^{*} \mathrm{~d} x=1
\]

И вообще, среднее значение любой функции координат \(f(x)\) определяется формулой, аналогичной (5.1), т. е.


\begin{equation*}
\langle f(x)\rangle=\int f(x) \psi \psi^{*} \mathrm{~d} x \tag{5.2}
\end{equation*}


Значительно сложнее задача о нахождении среднего значения проекции импульса \(p_{x}\) частицы, состояние которой задается определенной пси-функцией \(\psi(x)\). Весьма громоздкий расчет (выходящий за рамки данной книги) приводит к следующему результату:


\begin{equation*}
\left\langle p_{x}\right\rangle=\int \psi^{*}\left(-\mathrm{i} \hbar \frac{\partial \psi}{\partial x}\right) \mathrm{d} x \tag{5.3}
\end{equation*}


Для единообразия перепишем выражения (5.1) - (5.3) в такой форме:


\begin{align*}
& \langle x\rangle=\int \psi^{*} x \psi \mathrm{~d} x \\
& \langle f(x)\rangle=\int \psi^{*} f(x) \psi \mathrm{d} x  \tag{5.4}\\
& \left\langle p_{x}\right\rangle=\int \psi^{*}\left(-i \hbar \frac{\partial}{\partial x}\right) \psi \mathrm{d} x
\end{align*}


Запись средних значений этих величин именно в такой форме поможет нам в следующем параграфе сделать важный шаг в развитии адекватного математического формализма, выражающего специфические свойства микрочастиц.

Операторы. Оператором называют символическое обозначение математической операции, которую необходимо совершить с интересующей нас функцией. Примером оператора могут служить умножение на \(x\) или на какую-либо функцию \(f(x)\), дифференцирование по \(x\), т. е. \(\partial / \partial x, \partial^{2} / \partial x^{2}\) и т. д. Операторы принято обозначать буквами со «шляпкой», например \(\hat{Q}\), и его действие на некоторую функцию \(f(x)\) записывают как \(\hat{Q} f(x)\).

Некоторые свойства операторов. Операторы можно складывать: \(\hat{A}+\hat{B}\). Действие такого суммарного оператора на любую функцию \(f(x)\) дает результат \(\hat{A} f(x)+\hat{B} f(x)\).

Под произведением операторов \(\hat{A} \hat{B}\) понимают оператор, результат действия которого на любую функцию \(f(x)\) равен \(\hat{A}(\hat{B} f(x))\). Т. е. функция \(f(x)\) сначала подвергается действию оператора \(\hat{B}\), а затем полученный результат - действию оператора \(\hat{A}\).

Следует иметь в виду, что не всегда \(\hat{A} \hat{B}=\hat{B} \hat{A}\). Если такое равенство соблюдается, то говорят, что операторы \(\hat{A}\) и \(\hat{B}\) коммутируют друг с другом (коммутирующие операторы). В противном случае операторы некоммутирующие. Пример некоммутирующих операторов - это \(x\) и \(\partial / \partial x\). В самом деле,

\[
\left(x \frac{\partial}{\partial x}\right) f=x \frac{\partial f}{\partial x}, \quad\left(\frac{\partial}{\partial x} x\right) f=\frac{\partial}{\partial x}(x f)=1+x \frac{\partial f}{\partial x} .
\]

Следовательно,

\[
x \frac{\partial}{\partial x} \neq \frac{\partial}{\partial x} x .
\]

Сложение и умножение операторов производится по обычным алгебраическим правилам сложения и умножения чисел. Отличие лишь в том, что при умножении операторов не всегда можно переставлять порядок операторов-сомножителей: это зависит от того, коммутируют они или нет.

Оператор \(\hat{A}\) называют линейньлм, если для любых двух функций \(f_{1}\) и \(f_{2}\) и любых постоянных \(\alpha_{1}\) и \(\alpha_{2}\) выполняется соотношение

\[
\hat{A}\left(\alpha_{1} f_{1}+\alpha_{2} f_{2}\right)=\alpha_{1} \hat{A} f_{1}+\alpha_{2} \hat{A} f_{2} .
\]

Именно с линейностью операторов связан принцип суперпозиции состояний.

\section*{§ 5.2. Основные постулаты квантовой теории}
Общее утверждение квантовой теории заключается в том, что среднее значение любой физической величины \(Q\) находится по формуле


\begin{equation*}
\langle Q\rangle=\int \psi^{*} \hat{Q} \psi \mathrm{~d} x \tag{5.5}
\end{equation*}


где \(\hat{Q}\) - оператор физической величины \(Q\).\\
Сопоставив (5.5) с (5.4), приходим к выводу, что операторами величин \(x\) и \(p_{x}\) являются


\begin{equation*}
\hat{x}=x, \quad \hat{p}_{x}=-\mathrm{i} \hbar \frac{\partial}{\partial x} . \tag{5.6}
\end{equation*}


Аналогично для операторов \(\hat{y}, \hat{z}, \hat{p}_{y}, \hat{p}_{z}\).\\
Операторы \(\hat{x}\) и \(\hat{p}_{x}\) являются основными в квантовой теории.

Общее правило, позволяющее находить операторы других физических величин, таково:

формуль классической физики для связи между величинами в квантовой теории следует рассматривать как формульь, связывающие операторы этих величин.

Так, например, связь между квадратом импульса и квадратами его проекций в классической механике дается формулой

\[
p^{2}=p_{x}^{2}+p_{y}^{2}+p_{z}^{2} .
\]

Поэтому оператор квадрата импульса

\[
\hat{p}^{2}=\hat{p}_{x}^{2}+\hat{p}_{y}^{2}+\hat{p}_{z}^{2}=\left(-\mathrm{i} \hbar \frac{\partial}{\partial x}\right)^{2}+\left(-\mathrm{i} \hbar \frac{\partial}{\partial y}\right)^{2}+\left(-\mathrm{i} \hbar \frac{\partial}{\partial z}\right)^{2}
\]

В результате получим


\begin{equation*}
\hat{p}^{2}=-\hbar^{2}\left(\frac{\partial^{2}}{\partial x^{2}}+\frac{\partial^{2}}{\partial y^{2}}+\frac{\partial^{2}}{\partial z^{2}}\right)=-\hbar^{2} \nabla^{2} \tag{5.7}
\end{equation*}


где оператор \(\nabla^{2}\) - это лапласиан, т. е. выражение в круглых скобках.

Аналогично находим оператор кинетической энергии:


\begin{equation*}
\hat{K}=\frac{1}{2 m} \hat{p}^{2}=-\frac{\hbar^{2}}{2 m}\left(\frac{\partial^{2}}{\partial x^{2}}+\frac{\partial^{2}}{\partial y^{2}}+\frac{\partial^{2}}{\partial z^{2}}\right)=-\frac{\hbar^{2}}{2 m} \nabla^{2} \tag{5.8}
\end{equation*}


и оператор полной энергии частицы - гамильтониан (его обозначают \(\hat{H}\) ):


\begin{equation*}
\hat{H}=\hat{K}+\hat{U}=-\frac{\hbar^{2}}{2 m} \nabla^{2}+U \tag{5.9}
\end{equation*}


Зная выражения операторов \(\hat{p}^{2}, \hat{K}\) и \(\hat{H}\), можно найти средние значения \(\left\langle p^{2}\right\rangle,\langle K\rangle\) и \(\langle E\rangle\) по формуле (5.5), если известна \(ψ\)-функция частицы.

Пример. Найдем среднее значение кинетической анергии \(\langle K\rangle\) частицы в состоянии \(\psi(x)=\frac{1}{\sqrt{2 l}} \mathrm{e}^{\mathrm{i} k x}, k=p / \hbar\). Функция \(\psi(x)\) нормирована в интервале \(-l<x<l\), вне этого интервала \(\psi(x)=0\).

Согласно (5.8)

\[
\begin{gathered}
\langle K\rangle=\int \psi^{*} \hat{K} \psi \mathrm{~d} x=\frac{1}{2 l} \int_{-l}^{+l} \mathrm{e}^{-\mathrm{i} k x}\left(-\frac{\hbar^{2}}{2 m} \frac{\partial^{2}}{\partial x^{2}} \mathrm{e}^{\mathrm{i} k x}\right) \mathrm{d} x= \\
=\frac{1}{2 l} \int_{-l}^{+l} \frac{\hbar^{2}}{2 m} k^{2} \mathrm{~d} x=\frac{\hbar^{2} k^{2}}{2 m}=\frac{p^{2}}{2 m},
\end{gathered}
\]

как и должно быть. Но такой простой результат получается не всегда. Здесь это связано с тем, что в простой дебройлевской волне импульс и кинетическая энергия имеют вполне определенные значения. Средние значения данных величин совпадают с этими единственными их значениями.

Найдем, наконец, оператор момента импульса. Согласно классической механике

\[
\mathbf{M}=\mathbf{r} \times \mathbf{p}=\left|\begin{array}{lll}
\mathbf{i} & \mathbf{j} & \mathbf{k}  \tag{5.10}\\
x & y & z \\
p_{x} & p_{y} & p_{z}
\end{array}\right| .
\]

В соответствии с общим правилом оператор проекции момента импульса, например, на ось \(Z\) имеет вид:


\begin{equation*}
\hat{M}_{z}=x \hat{p}_{y}-y \hat{p}_{x}=x\left(-\mathrm{i} \hbar \frac{\partial}{\partial y}\right)-y\left(-\mathrm{i} \hbar \frac{\partial}{\partial x}\right)=-\mathrm{i} \hbar\left(x \frac{\partial}{\partial y}-y \frac{\partial}{\partial x}\right) . \tag{5.11}
\end{equation*}


В дальнейшем нам придется использовать этот оператор, но не в декартовой, а в сферической системе координат ( \(r, \theta, \varphi\) ). В этой системе оператор \(\hat{M}_{z}\), как показано в задаче 5.8 , имеет вид


\begin{equation*}
\hat{M}_{z}=-\mathrm{i} \hbar \frac{\partial}{\partial \varphi} . \tag{5.12}
\end{equation*}


Заметим, что вид этого оператора похож на вид оператора \(\hat{p}_{x}\).\\
Вернемся к оператору полной энергии (5.9). Найдем с помощью этого оператора связь между средними значениями полной, кинетической и потенциальной энергий:

\[
\langle E\rangle=\int \psi^{*}(\hat{K}+\hat{U}) \psi \mathrm{d} V=\int \psi^{*} \hat{K} \psi \mathrm{~d} V+\int \psi^{*} \hat{U} \psi \mathrm{~d} V .
\]

Это значит, что


\begin{equation*}
\langle E\rangle=\langle K\rangle+\langle U\rangle . \tag{5.13}
\end{equation*}


Полученное равенство не эквивалентно \(E=K+U\). Действительно, в силу соотношения неопределенностей величины \(K\) и \(U\) не могут одновременно иметь определенные значения, поскольку \(K\) зависит от импульса \(p\), а \(U\) - от координаты \(x\). Формула (5.13) показывает, однако, что классическая связь сохраняется между средними значениями \(E, K\) и \(U\).

Критерий наличия распределения. Установим связь между средним значением некоторой величины \(Q\) и средним значением его квадрата \(Q^{2}\), т.е. \(\langle Q\rangle\) и \(\left\langle Q^{2}\right\rangle\). Известно, что конкретное значение \(Q\) может быть представлено как


\begin{equation*}
Q=\langle Q\rangle+\Delta Q . \tag{5.14}
\end{equation*}


Найдем среднее значение \(\left\langle Q^{2}\right\rangle\) :

\[
\left\langle Q^{2}\right\rangle=\left\langle(\langle Q\rangle+\Delta Q)^{2}\right\rangle=\left\langle\langle Q\rangle^{2}+2\langle Q\rangle \Delta Q+(\Delta Q)^{2}\right\rangle .
\]

Здесь \(\langle Q\rangle\) и \(\langle Q\rangle^{2}\) - это просто числа. Их средние значения равны им самим. Среднее же \(\langle 2\langle Q\rangle \Delta Q\rangle=0\), поскольку \(\langle\Delta Q\rangle=0\). Отсюда следует, что


\begin{equation*}
\left\langle Q^{2}\right\rangle=\langle Q\rangle^{2}+\left\langle(\Delta Q)^{2}\right\rangle . \tag{5.15}
\end{equation*}


Среднее значение положительной величины \((\Delta Q)^{2}\) не может быть отрицательным и обращается в нуль в единственном случае, когда все \(\Delta Q=0\), т. е. когда нет никакого распределения, и наша величина \(Q\) имеет единственное значение (точно определена).\\
Формула (5.15) выражает собой критерий, позволяющий в каждом конкретном случае проверить, имеет ли интересующая нас величина распределение или имеет единственное значение. Для этого достаточно сравнить \(\left\langle Q^{2}\right\rangle\) с \(\langle Q\rangle^{2}\).

Собственные состояния. В предыдущей главе мы решали задачу о нахождении \(\psi\)-функций частицы в состояниях, где полная энергия имеет вполне определенные значения (в этом заключается суть квантования). Такие состояния и называют \(c o\) бственными.

Одним из основных постулатов квантовой теории является утверждение, что состояние, в котором физическая величина \(Q\) имеет определенное значение, описывается ψ-функиией, являющейся решением уравнения


\begin{equation*}
\hat{Q}_{\psi}=Q \psi, \tag{5.16}
\end{equation*}


где \(\hat{Q}\) - оператор физической величины \(Q\).\\
Убедимся, что это уравнение правильно решает поставленную задачу. Для этого найдем среднее значение \(Q\) в состоянии, которое описывается \(\psi\)-функцией, удовлетворяющей уравнению (5.16):

\[
\langle Q\rangle=\int \psi^{*} \hat{Q} \psi \mathrm{~d} V=\int \psi^{*} Q \psi \mathrm{~d} V=Q \int \psi^{*} \psi \mathrm{~d} V=Q .
\]

При нахождении \(\langle Q\rangle\) мы заменили в подынтегральном выражении \(\hat{Q} \psi\) на \(Q \psi\) в соответствии с (5.16) и учли условие нормировки \(ψ\)-функции. Полученный результат очевиден, поскольку других значений \(Q\) в этом состоянии нет.

Таким образом, \(\psi\)-функции, являющиеся решением уравнения (5.16), действительно описывают собственные состояния.

Уравнения (5.16), вообще говоря, являются уравнениями в частных производных. Согласно математике, для однозначного решения таких уравнений нужны дополнительные ограничения, например, граничные и начальные условия.

Условия же, которые накладывает квантовая теория на решения уравнения (5.16), имеют несколько иной характер: физический смысл могут иметь лишь такие решения, которые всюду конечные, однозначные, непрерывные и гладкие. Эти условия, как уже говорилось, называют естественньлми или стандартными.

Пример. Найдем с помощью уравнения (5.16) \(\psi\)-функцию состояния, в котором проекция импульса на ось \(X\) имеет определенное значение \(p_{x}\).\\
Для этого подставим в (5.16) в качестве оператора \(\hat{Q}\) оператор \(\hat{p}_{x}\) (5.6). Тогда

\[
-\mathrm{i} \hbar \frac{\partial \psi}{\partial x}=p_{x} \psi .
\]

Этому уравнению и всем необходимым условиям удовлетворяет функция

\[
\psi=\mathrm{e}^{\mathrm{i} k x}, \quad \text { где } k=p_{x} / \hbar,
\]

которая является координатной частью плоской волны де-Бройля.

Функции, являющиеся решением уравнения (5.16) и удовлетворяющие естественным условиям, называют собственными функииями оператора \(\hat{Q}\). Те значения \(Q\), при которых такие решения существуют, называют собственными значениями физической величины \(Q\). При этом набор собственных значений для оператора \(\hat{Q}\) определяет значения \(Q\), которые могут быть найдены из опыта при измерении данной физической величины.

Набор собственных значений физической величины \(Q\) иногда оказывается непрерывным, а иногда дискретным. Опыт показывает, что в последнем случае измеренные значения \(Q\) действительно оказываются дискретными и совпадают с собственными значениями \(Q\). Примером дискретности в микромире являются оптические спектры атомов, которые состоят из ряда отдельных тонких линий.

Уравнение (5.16) является обобщением правила квантования энергии, рассмотренного в предыдущей главе, на случай любых физических величин. Чтобы убедиться в этом, подставим (5.9) - оператор \(\hat{H}\) в (5.16):


\begin{equation*}
\left(-\frac{\hbar^{2}}{2 m} \nabla^{2}+U\right) \psi=E \psi \tag{5.17}
\end{equation*}


Это уравнение Шредингера (4.3) для стационарных состояний. Поэтому сокращенно его можно записать в символической форме


\begin{equation*}
\hat{H} \psi=E \psi \tag{5.18}
\end{equation*}


отличающейся от (5.16) только обозначениями.

\section*{§ 5.3. Квантование момента импульса}
Момент импульса. Момент импульса М является одной из важнейших характеристик движения. Его значение связано с тем, что М сохраняется, если система изолирована или движется в центральном силовом поле. Однако в квантовой теории момент импульса существенно отличается от классического. А именно, модуль момента импульса может быть задан сколь угодно точно только с одной из проекций, например, \(M_{z}\). Другие две проекции оказываются полностью неопределенными.

Это означает, что направление момента М в пространстве является неопределенным. Наглядно подобную ситуацию можно попытаться представить так: вектор \(\mathbf{M}\) как-то «размазан» по образующим конуса, ось которого совпадает с направлением координатной оси \(Z\) (рис. 5.1). В этом случае вполне определенное значение имеет лишь проекция \(M_{z}\). Другие две проекции, \(M_{x}\) и \(M_{y}\), оказываются

\begin{figure}[h]
\begin{center}
  \includegraphics[alt={},max width=\textwidth]{a57b3a57-78b5-4292-b0ab-4bd9f2ffcf56-120_321_289_141_871}
\captionsetup{labelformat=empty}
\caption{Рис. 5.1}
\end{center}
\end{figure}

полностью неопределенными.

Говоря в дальнейшем о «векторе» момента, мы будем иметь в виду именно такой квантовый смысл этой величины.

В этой главе мы ограничимся рассмотрением момента для одного электрона. В дальнейшем же по мере усложнения системы выясним, как это отразится на моменте системы (§ 6.4).

Модуль момента импульса. Начнем с квадрата момента. Согласно (5.13) для этого необходимо решить уравнение


\begin{equation*}
\hat{M}^{2} \psi=M^{2} \psi . \tag{5.19}
\end{equation*}


Оператор \(\hat{M}^{2}\) достаточно сложный, и решение этого уравнения является очень громоздким. Поэтому мы ограничимся приведением окончательных результатов, причем только для собственных значений данного оператора:


\begin{equation*}
M^{2}=l(l+1) \hbar^{2}, \quad l=0,1,2, \ldots, \tag{5.20}
\end{equation*}


где \(l\) - так называемое орбитальное (или азимутальное) квантовое число. Отсюда модуль момента


\begin{equation*}
M=\hbar \sqrt{l(l+1)}, \quad l=0,1,2, \ldots \tag{5.21}
\end{equation*}


Видно, что эта величина является дискретной (квантованной).\\
Следует отметить, что между классическим моментом импульса и соответствующим ему оператором имеется существенное различие. Классический момент \(\mathbf{r} \times \mathbf{p}\) зависит от выбора точки \(O\), относительно которой берется радиус-вектор r. Оператор же момента импульса не зависит от выбора точки \(O\) (в этом можно убедиться, записав проекции момента в сферических координатах).

Это значит, что оператор момента импульса зависит только от направления координатных осей. Поэтому его лучше называть оператором углового момента.

Не зависят от выбора точки \(O\) и собственные значения операторов квадрата и проекции углового момента, \(\hat{M}^{2}\) и \(\hat{M}_{z}\).

Проекция момента \(\boldsymbol{M}_{z}\). Поскольку в одном и том же состоянии проекции момента на два различных направления не могут иметь определенные значения, то избранное направление можно взять произвольно. Такое направление обычно принимают за ось \(Z\), так как в этом случае оператор \(\hat{M}_{z}\) дается более простой формулой (5.12).

Таким образом, для определения собственных значений и собственных функций этого оператора надо, согласно (5.16), peшить уравнение


\begin{equation*}
-\mathrm{i} \hbar \frac{\partial}{\partial \varphi} \psi=M_{z} \psi \tag{5.22}
\end{equation*}


Подстановка \(\psi=C \mathrm{e}^{\alpha φ}\) приводит после сокращения на общий множитель \(\mathrm{e}^{\alpha \varphi}\) к уравнению \(-\mathrm{i} \hbar \alpha=M_{z}\), из которого \(\alpha=\mathrm{i} M_{z} / \hbar\). Значит, решение уравнения (5.22) таково:


\begin{equation*}
\psi=C \mathrm{e}^{\mathrm{i} m \varphi}, \quad m=M_{z} / \hbar \tag{5.23}
\end{equation*}


Эта функция конечна, непрерывна и гладкая. Она должна быть и однозначной, ддя чего должно быть выполнено условие

\[
\psi(\varphi+2 \pi)=\psi(\varphi) .
\]

Данное условие выполняется только при целых значениях \(m\) в (5.23).

Следовательно, проекция углового момента на ось \(Z\) является кратной постоянной Планка:


\begin{equation*}
M_{z}=m \hbar, \quad m=0, \pm 1, \pm 2, \ldots \tag{5.24}
\end{equation*}


Поскольку ось \(Z\) выбирают произвольно, равенство (5.24) означает, что проекция углового момента на любое направление квантуется. Схематически это показано на рис. 5.2. Разумеется, подобные схемы не следует понимать буквально, ибо «вектор» \(\mathbf{M}\) принципиально не имеет определенных направлений в

пространстве. По причинам, которые выяснятся в дальнейшем (§ 7.1), число \(m\) называют магнитным квантовым числом.

С точки зрения квантовой теории волновая функция \(\psi_{l}\), соответствующая определенному квантовому числу \(l\), представляет собой суперпозицию состояний ( \(\psi_{l m}\)-функций), отличающихся друг от друга квантовым числом \(m\). Иначе говоря, состояние с заданным \(l\) является вырожденным по \(m\), причем кратность вырождения, т. е. число различных значений \(m\), как

\begin{figure}[h]
\begin{center}
  \includegraphics[alt={},max width=\textwidth]{a57b3a57-78b5-4292-b0ab-4bd9f2ffcf56-122_414_267_145_893}
\captionsetup{labelformat=empty}
\caption{Рис. 5.2}
\end{center}
\end{figure}

Проекция вектора не может быть больше модуля этого вектора, т. е. \(\left|M_{z}\right| \leqslant M\), поэтому в соответствии с (5.20) и (5.21) должно выполняться условие

\[
|m| \leqslant \sqrt{l(l+1)} .
\]

Отсюда следует, что максимальное значение \(|m|\) равно \(l\).\\
Мы видим, что при заданном \(l\) число \(m\) принимает \(2 l+1\) значений:

\[
l, \quad l-1, \quad \ldots, \quad 0, \quad \ldots, \quad-(l-1), \quad-l,
\]

образующих спектр величины \(M_{z}\). Заметим, что в квантовой теории при указании орбитального момента принято называть только \(l\), поскольку оно задает как модуль углового момента, так и все возможные значения его проекций на ось \(Z\). Так например, когда говорят, что орбитальный момент \(l=2\), то имеется в виду модуль \(M\) момента и спектр \(M_{z}\) :

\[
M=\hbar \sqrt{6}, \quad M_{z}=2 \hbar, 1 \hbar, 0,-1 \hbar,-2 \hbar .
\]

Итак, мы имеем:


\begin{align*}
& M=\hbar \sqrt{l(l+1),} \quad l=0,1,2, \ldots  \tag{5.25}\\
& M_{z}=\hbar m, \quad m=0, \pm 1, \pm 2, \ldots, \pm l \tag{5.26}
\end{align*}


Полученные результаты, определяющие возможные значения \(M\) и \(M_{z}\), называют пространственным квантованием. Для наглядности пространственное квантование обычно представляют графически (см. рис. 5.2).

Рассуждения, приведенные выше, можно провести и в обратном порядке: не от \(M к M_{z}\), а наоборот. При этом можно использовать довольно поучительный прием, познакомиться с которым имеет определенный смысл.

Итак, найдем зависимость \(M\) от числа \(l\). Для этого мысленно представим себе множество одинаковых частиц с одним и тем же моментом \(M\), но с разными значениями его проекции \(M_{z}\). Известно, что для средних значений справедливо равенство


\begin{equation*}
\left\langle M^{2}\right\rangle=\left\langle M_{x}^{2}\right\rangle+\left\langle M_{y}^{2}\right\rangle+\left\langle M_{z}^{2}\right\rangle \tag{5.27}
\end{equation*}


Левая часть этого равенства равна просто \(M^{2}\), а правая, в силу равновероятности всех проекций, может быть представлена как \(3\left\langle M_{z}^{2}\right\rangle\). Тогда (5.27) примет вид


\begin{equation*}
M^{2}=3\left\langle M_{z}^{2}\right\rangle . \tag{5.28}
\end{equation*}


Далее, согласно (5.21) при всяком значении \(l\) проекция \(M_{z}\) может принимать \(2 l+1\) различных значений. Поэтому среднее значение \(M_{z}^{2}\) равно


\begin{equation*}
\left\langle M_{z}^{2}\right\rangle=\hbar^{2}\left\langle m^{2}\right\rangle=\hbar^{2} \frac{\sum_{m=1}^{l} m^{2}}{2 l+1} \tag{5.29}
\end{equation*}


Из математики известно, что

\[
\sum_{m=1}^{l} m^{2}=\frac{l(l+1)(2 l+1)}{6} .
\]

Тогда формула (5.29) преобразуется к виду


\begin{equation*}
\left\langle M_{z}^{2}\right\rangle=\frac{\hbar^{2}}{3} l(l+1) \tag{5.30}
\end{equation*}


И наконец, после подстановки (5.30) в (5.28) получим


\begin{equation*}
M^{2}=\hbar^{2} l(l+1), \tag{5.31}
\end{equation*}


что и требовалось доказать.

\section*{§ 5.4. Ротатор}
В квантовой теории с моментом импульса \(M\) связан не только электрон, но и такой важный вопрос, как вращение молекул.

В классической механике кинетическая энергия вращающегося твердого тела определяется формулой \(E=M^{2} / 2 I\), где \(I\) момент инерции тела относительно соответствующей оси вращения.

Такая же формула справедлива и в квантовой теории, но только для связи между операторами:


\begin{equation*}
\hat{E}=\hat{M}^{2} / I . \tag{5.32}
\end{equation*}


Из этой формулы следует, что собственные значения оператора энергии, так же как и собственные значения оператора \(\hat{M}^{2}\), являются квантованными величинами. Согласно (5.21) имеем


\begin{equation*}
E_{r}=\frac{\hbar^{2}}{2 I} r(r+1), \quad r=0,1,2, \ldots, \tag{5.33}
\end{equation*}


где \(r\) - вращательное квантовое число (мы просто заменили \(l\) на \(r\), чтобы подчеркнуть, что это соотношение относится к вращению молекул).

Неизменяемую вращательную систему в квантовой физике называют ротатором. Формула (5.33) определяет его энергетические уровни, а значит и вращательные уровни молекулы. Из этой формулы следует, что расстояние между вращательными уровнями ротатора (молекулы) растет с увеличением квантового числа \(r\). В самом деле, интервал между уровнями \(r\) и \(r+1\)


\begin{equation*}
\Delta E=\frac{\hbar^{2}}{2 I}[(r+1)(r+2)-r(r+1)]=\frac{\hbar^{2}}{I}(r+1) . \tag{5.34}
\end{equation*}


Для вращательного квантового числа \(r\) действует правило отбора


\begin{equation*}
\Delta r= \pm 1 . \tag{5.35}
\end{equation*}


Поэтому частоты линий, испускаемых при переходах между вращательными уровнями, могут иметь значения, определяемые условием ћω \(=\Delta E\), откуда


\begin{equation*}
\omega=\frac{\hbar}{I}(r+1)=\omega_{1}(r+1), \quad \omega_{1}=\hbar / I, \tag{5.36}
\end{equation*}


где \(r\) - квантовое число уровня, на который происходит переход ( \(r=0,1,2, \ldots\) ).

\begin{figure}[h]
\begin{center}
  \includegraphics[alt={},max width=\textwidth]{a57b3a57-78b5-4292-b0ab-4bd9f2ffcf56-125_273_261_488_72}
\captionsetup{labelformat=empty}
\caption{Рис. 5.3}
\end{center}
\end{figure}

Заметим, что в случае двухатомной молекулы момент инерции \(I\) берется относительно оси \(O O\), проходящей через ее центр масс \(C\) и перпендикулярной прямой, проходящей через ядра атомов молекулы (рис. 5.3). Тогда (в этом полезно убедиться самостоятельно)


\begin{equation*}
I=\mu d^{2} \tag{5.37}
\end{equation*}


где \(d\) - расстояние между ядрами молекулы, \(\mu\) - ее приведенная масса, \(\mu=m_{1} m_{2} /\left(m_{1}+m_{2}\right), m_{1}\) и \(m_{2}\) - массы обоих атомов.

\begin{figure}[h]
\begin{center}
  \includegraphics[alt={},max width=\textwidth]{a57b3a57-78b5-4292-b0ab-4bd9f2ffcf56-125_370_237_939_68}
\captionsetup{labelformat=empty}
\caption{Рис. 5.4}
\end{center}
\end{figure}

Спектр вращательных уровней энергии и соответствующих спектральных линий изображен на рис. 5.4. Чисто вращательные спектры молекул находятся в далекой инфракрасной области и в области сантиметровых волн.

Измерив интервалы между линиями \(\Delta \omega=\omega_{1}\), можно определить момент инерции \(I\) молекулы по формуле (5.36) и, зная массы ядер, - расстояние \(d\) между ними. Приведем полученные таким образом значения \(I\) и \(d\) для некоторых двухатомных молекул.

\begin{center}
\begin{tabular}{ccc||ccc}
Молекула & \begin{tabular}{c}
\(I\), \\
\(10^{-40}{ }_{\Gamma \cdot \mathrm{cm}^{2}}\) \\
\end{tabular} & \begin{tabular}{c}
\(d\), \\
\(10^{-8} \mathrm{~cm}\) \\
\end{tabular} & Молекула & \begin{tabular}{c}
\(I\), \\
\(10^{-40}{ }_{\Gamma \cdot \mathrm{cm}^{2}}\) \\
\end{tabular} & \begin{tabular}{c}
\(d\), \\
\(10^{-8} \mathrm{~cm}\) \\
\end{tabular} \\
\hline
\(\mathrm{H}_{2}\) & 0,46 & 0,74 & HCl & 2,65 & 1,28 \\
\(\mathrm{O}_{2}\) & 19,0 & 1,20 & CO & 14,5 & 1,13 \\
\end{tabular}
\end{center}

Ранее (§ 4.4) было показано, что у молекул должны существовать колебательные уровни. Только что мы рассмотрели отдельно вращательнье уровни. В общем же случае молекулы колеблются и вращаются одновременно. Это приводит к возникновению так называемых колебательно-вращательных по-

лос, состоящих из весьма близких линий, расположенных симметрично относительно «линии» с частотой \(\omega_{0}\) и отстоящих друг от друга на \(\Delta \omega=\omega_{1}=\hbar / I\). Схема соответствующих уровней, переходов и расположения спектральных линий в полосе показана на рис. 5.5. В середине полосы интервал между соседними линиями вдвое больше, поскольку линия с частотой \(\omega_{0}\) не возникает из-за правила отбора (5.35), согласно которому \(\Delta r= \pm 1\).

\begin{figure}[h]
\begin{center}
  \includegraphics[alt={},max width=\textwidth]{a57b3a57-78b5-4292-b0ab-4bd9f2ffcf56-126_441_362_141_798}
\captionsetup{labelformat=empty}
\caption{Рис. 5.5}
\end{center}
\end{figure}

\section*{Задачи}
5.1. Проверить следующее операторное равенство:

\[
\left(1+\frac{\partial}{\partial x}\right)^{2}=1+2 \frac{\partial}{\partial x}+\frac{\partial^{2}}{\partial x^{2}} .
\]

Р еш ени е. Имея в виду, что \(\hat{Q}^{2} \psi=\hat{Q}(\hat{Q} \psi)\), запишем:

\[
\left(1+\frac{\partial}{\partial x}\right)\left(\psi+\frac{\partial \psi}{\partial x}\right)=\psi+\frac{\partial \psi}{\partial x}+\frac{\partial \psi}{\partial x}+\frac{\partial^{2} \psi}{\partial x^{2}}=\left(1+2 \frac{\partial}{\partial x}+\frac{\partial^{2}}{\partial x^{2}}\right) \psi .
\]

Равенство, таким образом, доказано.\\
5.2. Коммутативность операторов. Проверить, коммутируют ли операторы:\\
а) \(\hat{x}\) и \(\hat{p}_{x}\);\\
б) \(\hat{x}\) и \(\hat{p}_{y}\);\\
в) \(\hat{p}_{x}\) и \(\hat{p}_{y}\).

Решение. а) Вопрос сводится к установлению разности:\\
\(x \hat{p}_{x} \psi-\hat{p}_{x} x \psi=-\mathrm{i} \hbar\left(x \frac{\partial \psi}{\partial x}-\frac{\partial}{\partial x}(x \psi)\right)=-\mathrm{i} \hbar\left(x \frac{\partial \psi}{\partial x}-x \frac{\partial \psi}{\partial x}-\psi\right)=\mathrm{i} \hbar \psi\).\\
Следовательно, эти операторы взаимно не коммутируют.\\
б) \(x \hat{p}_{y} \psi-\hat{p}_{y} x \psi \circlearrowleft\left(x \frac{\partial \psi}{\partial y}-x \frac{\partial \psi}{\partial y}\right)=0\),\\
т. е. операторы коммутативны.\\
в) \(\hat{p}_{x} \hat{p}_{y} \psi-\hat{p}_{y} \hat{p}_{x} \psi \circlearrowleft\left(\frac{\partial}{\partial x} \frac{\partial \psi}{\partial y}-\frac{\partial}{\partial y} \frac{\partial \psi}{\partial x}\right)=\frac{\partial^{2} \psi}{\partial x \partial y}-\frac{\partial^{2} \psi}{\partial y \partial x}=0\).

Операторы коммутативны.\\
5.3. Собственные значения и собственные функции. Найти собственное значение оператора \(\hat{A}=-\frac{\partial^{2}}{\partial x^{2}}\), принадлежащее собственной функции \(\psi=C \sin 2 x, C\) - постоянная.\\
Решение. Согласно (5.16)


\begin{equation*}
-\frac{\partial^{2}}{\partial x^{2}} \psi=A \psi \tag{1}
\end{equation*}


Дважды продифференцировав функцию \(\psi\) по \(x\), получим


\begin{equation*}
-\frac{\partial}{\partial x}(2 \cos 2 x)=4 \sin 2 x \tag{2}
\end{equation*}


Из сопоставления (2) с (1) находим \(A=4\).\\
5.4. Найти собственные функции \(\psi\) и собственные значения оператора \(-\mathrm{i} \frac{\partial}{\partial x}\), если \(\psi(x)=\psi(x+a), a\) - постоянная.\\
Р е ш е ни е. На основании (5.16) запишем


\begin{equation*}
-\mathrm{i} \frac{\partial}{\partial x} \psi=\lambda \psi \tag{1}
\end{equation*}


откуда


\begin{equation*}
\frac{\partial \psi}{\psi}=\mathrm{i} \lambda \partial x \tag{2}
\end{equation*}


Проинтегрировав это уравнение, получим


\begin{equation*}
\ln \psi=\mathrm{i} \lambda x+C \tag{3}
\end{equation*}


где \(C\) - произвольная постоянная. Потенцируя (3), получим

\[
\psi=C \mathrm{e}^{\mathrm{i} \lambda x}
\]

По условию ( \(ψ\) - периодическая) следует, что

\[
\mathrm{e}^{\mathrm{i} \lambda x}=\mathrm{e}^{\mathrm{i} \lambda(x+a)}
\]

откуда

\[
\mathrm{e}^{\mathrm{i} \lambda a}=1, \quad \lambda a=2 \pi n, \quad n=0, \pm 1, \pm 2, \ldots
\]

В результате

\[
\psi=C \mathrm{e}^{\mathrm{i} \lambda x}, \quad \lambda=\frac{2 \pi n}{a}, \quad n=0, \pm 1, \pm 2, \ldots
\]

Постоянная \(C\) остается неопределенной.\\
5.5. Средние значения. В некоторый момент частица находится в состоянии, описываемом \(\psi\)-функцией, координатная часть которой \(\psi(x)=A \exp \left(\mathrm{i} k x-x^{2} / a^{2}\right)\), где \(A\) и \(a\) - неизвестные постоянные. Найти средние значения:\\
а) координаты \(x\); б) проекции импульса \(p_{x}\). P е ш е н и е. а) В соответствии с формулой (5.1)

\[
\langle x\rangle=\int x \psi \psi^{*} \mathrm{~d} x=A A^{*} \int_{-\infty}^{+\infty} x \exp \left(-2 x^{2} / a^{2}\right) \mathrm{d} x
\]

Поскольку подынтегральная функция нечетная, то интеграл равен нулю, значит и \(\langle x\rangle=0\).\\
б) Согласно (5.3) сначала найдем производную \(\partial \psi / \partial x\) :

\[
\frac{\partial \psi}{\partial x}=\psi(x) \cdot\left(\mathrm{i} k-2 x / a^{2}\right) .
\]

После подстановки этого выражения в (5.3) получим


\begin{equation*}
\left\langle p_{x}\right\rangle=-\mathrm{i} \hbar A A^{*} \int_{-\infty}^{+\infty}\left(\mathrm{i} k-2 x / a^{2}\right) \exp \left(-2 x^{2} / a^{2}\right) \mathrm{d} x \tag{1}
\end{equation*}


Из условия нормировки следует, что


\begin{equation*}
\int_{-\infty}^{+\infty} \psi \psi^{*} \mathrm{~d} x=A A^{*} \int_{-\infty}^{+\infty} \exp \left(-2 x^{2} / a^{2}\right) \mathrm{d} x=1 \tag{2}
\end{equation*}


Кроме того, интеграл (1) представляет собой разность двух интегралов. Второй из них равен нулю, так как подынтегральная функция его является нечетной. Остается первый интеграл:

\[
\left\langle p_{x}\right\rangle=\hbar k A A^{*} \int_{-\infty}^{+\infty} \exp \left(-2 x^{2} / a^{2}\right) \mathrm{d} x
\]

Учитывая (2), получим в результате

\[
\left\langle p_{x}\right\rangle=\hbar k .
\]

5.6. Частица находится в сферически-симметричном потенциальном поле в состоянии, описываемом нормированной пси-функцией

\[
\psi(r)=\frac{1}{\sqrt{2 \pi a}} \frac{\mathrm{e}^{-r / a}}{r},
\]

где \(r\) - расстояние от центра поля, \(a\) - постоянная. Найти \(\langle r\rangle\).

P е шен и е. В данном случае в формуле (5.1) под \(\mathrm{d} x\) надо понимать элемент объема \(\mathrm{d} V\). В качестве такового для упрощения расчета наиболее целесообразно взять сферический слой с радиусами \(r\) и \(r+\mathrm{d} r\). Для него \(\mathrm{d} V=4 \pi r^{2} \mathrm{~d} r\) и

\[
\langle r\rangle=\int r \psi^{2} 4 \pi r^{2} \mathrm{~d} r=\int_{0}^{\infty} \frac{\mathrm{e}^{-2 r / a}}{2 \pi a r^{2}} 4 \pi r^{3} \mathrm{~d} r=\frac{2}{a} \int_{0}^{\infty} e^{-2 r / a} r \mathrm{~d} r
\]

Введем новую переменную \(2 r / a=y\). Тогда предыдущее выражение примет вид

\[
\langle r\rangle=\frac{a}{2} \int_{0}^{\infty} \mathrm{e}^{-y} y \mathrm{~d} y
\]

Взяв интеграл по частям, находим, что он равен единице. Таким образом

\[
\langle r\rangle=a / 2
\]

5.7. Найти среднюю кинетическую энергию частицы в одномерной прямоугольной потенциальной яме с абсолютно непроницаемыми стенками ( \(0<x<l\) ), если частица находится в состоянии \(\psi(x)=A x(l-x)\).\\
Решение. Прежде всего найдем нормировочный коэффициент \(A\) :

\[
\int_{0}^{l} \psi^{2} \mathrm{~d} x=A^{2} \int_{0}^{l} x^{2}(l-x)^{2} \mathrm{~d} x=A^{2} l^{5} / 30
\]

Из условия нормировки полученный результат должен быть равен единице. Отсюда

\[
A^{2}=30 / l^{5}
\]

Средняя кинетическая энергия согласно (5.5) определяется как

\[
\langle K\rangle=\int_{0}^{l} \psi(\hat{K} \psi) \mathrm{d} x
\]

где выражение в круглых скобках можно представить с помощью (5.8) в виде

\[
\hat{K} \psi=-\frac{\hbar^{2}}{2 m} \frac{\partial^{2} \psi}{\partial x^{2}}=-\frac{\hbar^{2}}{2 m}(-2 A)
\]

После подстановки в выражение для \(\langle K\rangle\) и интегрирования получим:

\[
\langle K\rangle=5 \hbar^{2} / m l^{2}
\]

5.8. Оператор проекции момента \(\hat{\boldsymbol{M}}_{z}\). Показать, что в сферической системе координат оператор \(\hat{M}_{z}=-\mathrm{i} \hbar \frac{\partial}{\partial \varphi}\). Использовать формулы, связывающие декартовы и сферические координаты, а также выражение для оператора \(\hat{M}_{z}\) в декартовой системе координат.

Решение. Запишем с помощью рис. 5.6 связь между декартовыми и сферическими координатами:


\begin{align*}
& x=r \sin \theta \cos \varphi \\
& y=r \sin \theta \sin \varphi  \tag{1}\\
& z=r \cos \theta
\end{align*}


С помощью этих формул выразим частную производную по φ через производные по \(x, y, z\).


\begin{equation*}
\frac{\partial}{\partial \varphi}=\frac{\partial x}{\partial \varphi} \frac{\partial}{\partial x}+\frac{\partial y}{\partial \varphi} \frac{\partial}{\partial y}+\frac{\partial z}{\partial \varphi} \frac{\partial}{\partial z} . \tag{2}
\end{equation*}


Вычислив частные производные \(\partial x / \partial \varphi, \quad \partial y / \partial \varphi\) и \(\partial z / \partial \varphi\) формул (1), подставим результаты в (2) и получим

\begin{figure}[h]
\begin{center}
  \includegraphics[alt={},max width=\textwidth]{a57b3a57-78b5-4292-b0ab-4bd9f2ffcf56-130_395_392_624_755}
\captionsetup{labelformat=empty}
\caption{Рис. 5.6}
\end{center}
\end{figure}


\begin{equation*}
\frac{\partial}{\partial \varphi}=-r \sin \theta \sin \varphi \frac{\partial}{\partial x}+r \sin \theta \cos \varphi \frac{\partial}{\partial y}+0 . \tag{3}
\end{equation*}


Из сопоставления с (1) видим, что (3) можно переписать так:


\begin{equation*}
\frac{\partial}{\partial \varphi}=-y \frac{\partial}{\partial x}+x \frac{\partial}{\partial y} . \tag{4}
\end{equation*}


Правая часть этого равенства полностью совпадает с выражением в скобках формулы (5.11). Дальнейшее очевидно.\\
5.9. Вращательный спектр молекулы. Оценить, сколько линий содержит чисто вращательный спектр молекулы CO, момент инерции которой \(I=1,44 \cdot 10^{-39} г \cdot \mathrm{~cm}^{2}\) и собственная частота колебаний \(\omega=4,1 \cdot 10^{14} \mathrm{c}^{-1}\) ?

Решение. Искомое число линий должно быть равно числу вращательных уровней между нулевым и первым возбужденным колебательными уровнями ( \(v=0\) и \(v=1\) ), интервал между которыми согласно (4.23) равен ћω. Задача, таким образом, сводится к опре-

делению максимального вращательного квантового числа \(r\) уровня с энергией ћω. Учитывая (5.33), запишем

\[
\hbar \omega=\frac{\hbar^{2}}{2 I} r(r+1),
\]

откуда

\[
r^{2}+r-2 \omega I / \hbar=0
\]

Решение этого уравнения дает \(r_{\text {макс }}\) :

\[
r_{\text {Оањж }}=\frac{-1+\sqrt{1+4(2 \omega I / \hbar)}}{2} \approx 2 \omega I / \hbar=33 .
\]

Следовательно, чисто вращательный спектр данной молекулы содержит около 30 линий.\\
5.10. Колебательно-вращательная полоса. В середине колебательно-вращательной полосы спектра испускания молекул HCl , где отсутствует «нулевая» линия, запрещенная правилом отбора, интервал между соседними линиями равен \(\Delta \omega_{0}\). Найти расстояние между ядрами молекулы HCl .\\
P ешении е. Сначала найдем интервал \(\Delta E\) между соседними вращательными энергетическими уровнями. Согласно (5.34),

\[
\Delta E=\frac{\hbar^{2}}{I}(r+1)
\]

Соответствующая ему частота перехода

\[
\omega_{r}=\Delta E / \hbar=(r+1) \hbar / I .
\]

При переходе к соседней линии \(r\) меняется на единицу, согласно правилу отбора (5.35), и интервал между соседними линиями

\[
\Delta \omega=(\Delta r) \cdot \hbar / I=\hbar / I
\]

где \(\Delta r=1\). Остается учесть, что в середине колебательно-вращательной полосы этот интервал будет вдвое больше, а также выражение (5.37) для момента инерции молекулы. В результате получим \(\Delta \omega_{0}=2 \Delta \omega=2 \hbar / \mu d^{2}\), откуда

\[
d=\sqrt{2 \hbar / \mu \Delta \omega_{0}},
\]

где \(\mu\) - приведенная масса молекулы, \(\mu=m_{1} m_{2} /\left(m_{1}+m_{2}\right)\).

\section*{Квантование атомов}
\section*{§ 6.1. Квантование атома водорода}
Рассмотрим простейшую систему, состоящую из электрона \(e\), который движется в кулоновском поле ядра с зарядом \(Z e\). Такую систему называют водородоподобной. При \(Z=1\) это атом водорода, при \(Z=2\) - однократно ионизированный атом гелия - ион \(\mathrm{He}^{+}\), при \(Z=3\) - двукратно ионизированный атом лития - ион \(\mathrm{Li}^{++}\)и т. д.

Потенциальная энергия взаимодействия электрона с ядром в такой системе равна


\begin{equation*}
U(r)=-\frac{Z e^{2}}{r} \tag{6.1}
\end{equation*}


где \(r\) - расстояние между электроном и ядром, которое в первом приближении будем считать точечным (здесь и далее).

Уравнение Шредингера в этом случае имеет вид


\begin{equation*}
\nabla^{2} \psi+\frac{2 m}{\hbar^{2}}\left(E+\frac{Z e^{2}}{r}\right) \psi=0 \tag{6.2}
\end{equation*}


Поле (6.1), в котором движется электрон, является централь-но-симметричным, т. е. зависит только от \(r\). Поэтому решение уравнения (6.2) наиболее целесообразно проводить в сферической системе координат \(r, \theta, \varphi\), где оператор Лапласа \(\nabla^{2}\) имеет следующий вид:


\begin{equation*}
\nabla^{2}=\frac{\partial^{2}}{\partial r^{2}}+\frac{2}{r} \frac{\partial}{\partial r}+\frac{1}{r^{2} \sin \theta} \frac{\partial}{\partial \theta}\left(\sin \theta \frac{\partial}{\partial \theta}\right)+\frac{1}{r^{2} \sin ^{2} \theta} \frac{\partial^{2}}{\partial \varphi^{2}} . \tag{6.3}
\end{equation*}


Мы не будем воспроизводить здесь этапы решения уравнения (6.2), поскольку оно слишком громоздко (об этом красноречиво свидетельствует уже сам вид оператора Лапласа). Остановимся лишь на сути процесса решения и на анализе окончательных результатов.

Решение уравнения (6.2) проводят методом разделения переменных с учетом естественных требований, налагаемых на \(ψ\)-функцию: она должна быть однозначной, конечной, непрерывной и гладкой. В процессе решения обнаруживается, что этим требованиям можно удовлетворить при любы \(x\) положительных значениях энергии \(E\), но в области отрицательных значений \(E\) - только при дискретных значениях \(E\), а именно, если


\begin{equation*}
E_{n}=-\frac{m e^{2}}{2 \hbar^{2}} \frac{Z^{2}}{n^{2}}, \quad n=1,2,3, \ldots \tag{6.4}
\end{equation*}


Этот случай ( \(E<0\) ) для нас представляет особый интерес, поскольку он соответствует связанным состояниям электрона (электрону в атоме).

Таким образом, последовательное решение уравнения Шредингера приводит в случае \(E<0\) к формуле (6.4) для энергетических уровней - без использования каких-либо дополнительных постулатов (в отличие от первоначальной теории Бора). Кроме того, совпадение с формулой (2.25) означает, что мы пришли к той же самой системе энергетических уровней (см. рис. 2.7). Это же относится и к частотам излучения при переходах между уровнями. Поэтому повторять нет необходимости.

Различие в интерпретации относится только к состояниям электрона: в теории Бора это движение по стационарным орбитам, здесь же орбиты теряют физический смысл, их место занимают \(\psi\)-функции.

Собственные функции уравнения (6.2), т. е. \(\psi\)-функции, содержат, как выяснилось, три целочисленных параметра - \(n\), \(l, m\) :


\begin{equation*}
\psi=\psi_{n l m}(r, \theta, \varphi), \tag{6.5}
\end{equation*}


где \(n\) называют главным квантовым числом (это то же \(n\), что и в выражении для \(E_{n}\) ). Параметры же \(l\) и \(m\) - это орбитальное и магнитное квантовые числа, определяющие по формулам \((5.25)\) и (5.26) модуль момента импульса \(M\) и его проекцию \(M_{z}\).

В процессе решения выясняется, что решения, удовлетворяющие естественным условиям, получаются лишь при значениях \(l\), не превышающих \(n-1\). Таким образом, при данном \(n\) квантовое число \(l\) может принимать \(n\) значений:


\begin{equation*}
l=0, \quad 1, \quad 2, \quad \ldots, \quad n-1 \tag{6.6}
\end{equation*}


В свою очередь, при данном \(l\) квантовое число \(m\) согласно (5.26) может принимать \(2 l+1\) различных значений:


\begin{equation*}
m=0, \pm 1, \pm 2, \ldots, \pm l . \tag{6.7}
\end{equation*}


Энергия \(E_{n}\) электрона (6.4) зависит только от главного квантового числа \(n\). Отсюда следует, что каждому собственному значению \(E_{n}\) (кроме случая \(n=1\) ) соответствует несколько собственных функций \(\psi_{n l m}\), отличающихся значениями квантовых чисел \(l\) и \(m\). Это означает, что электрон может иметь одно и то же значение энергии, находясь в нескольких различных состояниях. Например, энергией \(E_{2}(n=2)\) обладают четыре состояния: \(\Psi_{200}, \Psi_{21-1}, \Psi_{210}, \Psi_{21+1}\).

Кратность вырождения. Состояния с одинаковой энергией называют вырожденными, а число различных состояний с определенным значением энергии \(E_{n}\) - кратностью вырождения данного энергетического уровня.

Кратность вырождения \(n\)-го уровня водородоподобной системы можно определить, учитывая число возможных значений \(l\) и \(m\). Каждому из \(n\) значений квантового числа \(l\) соответствует \(2 l+1\) значений \(m\). Поэтому полное число \(N\) различных состояний для данного \(n\) равно


\begin{equation*}
N=\sum_{l=0}^{n-1}(2 l+1)=1+3+5+\ldots+(2 n-1)=n^{2} . \tag{6.8}
\end{equation*}


Следовательно, кратность вырождения \(n\)-го энергетического уровня водородоподобных систем равна \(n^{2}\).

В действительности, как будет показано в дальнейшем (§ 6.3), это число надо удвоить из-за наличия собственного момента (спина) у электрона. Таким образом, кратность вырождения \(n\)-го энергетического уровня


\begin{equation*}
N=2 n^{2} . \tag{6.9}
\end{equation*}


Символы состояний. Различные состояния электрона в атоме принято обозначать малыми буквами латинского алфавита в зависимости от значения орбитального квантового числа \(l\) :

\begin{center}
\begin{tabular}{l|llllll}
Квантовое число \(l\) & 0 & 1 & 2 & 3 & 4 & 5 \\
\hline
Символ состояния & \(s\) & \(p\) & \(d\) & \(f\) & \(g\) & \(h\) \\
\hline
\end{tabular}
\end{center}

Принято говорить о \(s\)-состояниях (или \(s\)-электронах), \(p\)-состояниях (или \(p\)-электронах) и т. д.

Значение главного квантового числа \(n\) указывают перед символом состояния с данным \(l\). Например, электрон, имеющий главное квантовое число \(n=3\) и \(l=2\), обозначают символом \(3 d\) и т. д. Выпишем последовательно несколько состояний электрона:


\begin{equation*}
1 s ; \quad 2 s, 2 p ; \quad 3 s, 3 p, 3 d ; \quad \ldots \tag{6.11}
\end{equation*}


Собственные функции уравнения (6.2) представляют собой произведение двух функций, одна из которых зависит только от \(r\), а другая - только от углов \(\theta\) и φ:


\begin{equation*}
\psi_{n l m}(r, \theta, \varphi)=R_{n l}(r) \cdot Y_{l m}(\theta, \varphi), \tag{6.12}
\end{equation*}


где первый сомножитель зависит от квантовых чисел \(n\) и \(l\), второйже - от \(l\) и \(m\).

Функция \(Y_{l m}(\theta, \varphi)\) является собственной функцией оператора квадрата момента импульса \(\hat{M}^{2}\). Для \(s\)-состояний ( \(l=0\) ) эта функция является константой, так что \(\psi\)-функция вида \(\psi_{n 00}\) зависит только от \(r\). Вообще же


\begin{equation*}
Y_{l m}(\theta, \varphi)=\Theta_{l|m|}(\theta) \cdot \mathrm{e}^{\mathrm{i} m \varphi} . \tag{6.13}
\end{equation*}


В таблицах (6.1) и (6.2) приведен в качестве примера вид наиболее простых функций \(R_{n l}(r)\) и \(\Theta_{l|m|}(\theta)\) с точностью до нормировочных множителей.

\begin{table}[h]
\begin{center}
\captionsetup{labelformat=empty}
\caption{Таблица 6.1}
\begin{tabular}{||c|c|c||}
\hline\hline
Состояние & \(n, l\) & \(R(\rho)\) \\
\hline
\(1 s\) & 1,0 & \(\mathrm{e}^{-\rho}\) \\
\(2 s\) & 2,0 & \((2-\rho) \mathrm{e}^{-\rho / 2}\) \\
\(2 p\) & 2,1 & \(\rho \mathrm{e}^{-\rho / 2}\) \\
\hline\hline
\end{tabular}
\end{center}
\end{table}

\begin{table}[h]
\begin{center}
\captionsetup{labelformat=empty}
\caption{Таблица 6.2}
\begin{tabular}{||c|c|c||}
\hline\hline
Состояние & \(l,|m|\) & \(\Theta_{l|m|}(\theta)\) \\
\hline
\(s\) & 0,0 & 1 \\
\(p\) & 1,0 & \(\cos \theta\) \\
 & 1,1 & \(\sin \theta\) \\
\hline\hline
\end{tabular}
\end{center}
\end{table}

Здесь \(\rho=r / r_{1}, r_{1}\) - боровский радиус (2.24).\\
В соответствии с формулами (6.12) и (6.13) и этими таблицами представим, как выглядит, например, функция \(\psi_{211}\) :

\[
\psi_{211}=A r \mathrm{e}^{-r / 2 r_{1}} \sin \theta \cdot \mathrm{e}^{\mathrm{i} \varphi}
\]

где \(A\) - нормировочный коэффициент.

Распределение плотности вероятности. В квантовой теории нельзя говорить о траекториях электрона в атоме. Имеет смысл лишь состояние ( \(\psi\)-функция) и вероятность местонахождения электрона в том или ином месте в поле ядра. Для наглядности вводят представление об электронном облаке, плотность распределения которого в каждой точке пропорциональна плотности вероятности \(\mathrm{d} P / \mathrm{d} V\) местонахождения электрона в этой точке.

Плотность вероятности местонахождения электрона дается квадратом модуля волновой функции \(|\psi|^{2}\) или \(\psi \psi^{*}\). Ограничимся для простоты рассмотрением основного состояния электрона \(1 s\) атома водорода, которое является сферически-симметричным, т. е. его \(\psi\)-функция зависит только от \(r\) :


\begin{equation*}
\psi_{1 s} \sim \mathrm{e}^{-\alpha r}, \tag{6.14}
\end{equation*}


где \(\alpha=1 / r_{1}, r_{1}\) - боровский радиус.\\
Вероятность нахождения электрона в объеме \(\mathrm{d} V\), как мы знаем, равна \(|\psi|^{2} \mathrm{~d} V\). Возьмем в качестве элементарного объема \(\mathrm{d} V\) сферический слой толщиной \(\mathrm{d} r\) и радиусом \(r: \mathrm{d} V=4 \pi r^{2} \mathrm{~d} r\). Тогда вероятность \(\mathrm{d} P\) нахождения \(1 s\)-электрона в этом слое


\begin{equation*}
\mathrm{d} P=A r^{2} \psi^{2} \mathrm{~d} r, \tag{6.15}
\end{equation*}


где \(A\) - нормировочный коэффициент. Отсюда плотность вероятности \(\mathrm{d} P / \mathrm{d} r\), т. е. вероятность местонахождения электрона в сферическом слое единичной толщины вблизи радиуса \(r\) есть


\begin{equation*}
\mathrm{d} P / \mathrm{d} r=A r^{2} \mathrm{e}^{-2 \alpha r} \circlearrowleft r^{2} \mathrm{e}^{-2 \alpha r} . \tag{6.16}
\end{equation*}


Эту плотность вероятности не следует смешивать с плотностью вероятности \(\mathrm{d} P / \mathrm{d} V\), отнесенной к единице объема вблизи точки с радиусом-вектором \(\mathbf{r}\) и равной \(|\psi|^{2}\).

Видно, что (6.16) обращается в нуль при \(r \rightarrow 0\) и при \(r \rightarrow \infty\). Найдем значение \(r\), при котором (6.16) достигает максимума. Для этого продифференцируем (6.16) по \(r\) и приравняем нулю полученное выражение (после сокращения на экспоненту). В результате получим наиболее вероятное расстояние электрона от ядра:


\begin{equation*}
r_{m}=1 / \alpha=r_{1} . \tag{6.17}
\end{equation*}


Мы видим, что \(r_{m}\) в точности совпадает с радиусом первой боровской орбиты электрона в атоме водорода (2.24).

\begin{figure}[h]
\begin{center}
  \includegraphics[alt={},max width=\textwidth]{a57b3a57-78b5-4292-b0ab-4bd9f2ffcf56-137_354_457_246_72}
\captionsetup{labelformat=empty}
\caption{Рис. 6.1}
\end{center}
\end{figure}

На рис. 6.1 показаны графики зависимостей \(\psi(r), \psi^{2}(r)\) и \(\mathrm{d} P / \mathrm{d} r \sim r^{2} \psi^{2}\). Следует обратить внимание на то, что пространственное распределение в электронном облаке атома можно характеризовать либо квадратом модуля пси-функции \(|\psi(r)|^{2}\), либо величиной \(r^{2}|\psi(r)|^{2}\). Первое выражение определяет вероятность местонахождения электрона в единице объема, второе - в сферическом слое единичной толщины. Их графики существенно отличаются друг от друга, как видно из рисунка.

Заметим, что \(\psi_{1 s}(r)\) не является гладкой в точке \(r=0\). Это есть следствие того, что потенциальная энергия электрона при \(r \rightarrow 0\) обращается в бесконечность (в предположении, что ядро является точечным). Учет конечных размеров ядра устраняет этот дефект \(\psi\)-функции.

Состояние движения электрона в атоме не всегда имеет даже какой-то приближенный аналог. Например, во всех \(s\)-состояниях орбитальный момент электрона равен нулю ( \(l=0\) ). С классической точки зрения это соответствует движению электрона вдоль радиуса, т. е. электрон при своем движении должен был бы пересекать область, занятую ядром. Это в классике невозможно. В квантовой же теории состояние с нулевым орбитальным моментом существует - это \(s\)-состояния электрона, в которых распределение «плотности» электронного облака сфери-чески-симметрично. Итак, в основном \(1 s\)-состоянии угловой момент электрона, в отличие от теории Бора, равен нулю.

В заключение несколько слов о распределении электронного облака в других состояниях ( \(p, d, \ldots\) ). Здесь оно уже не сфериче-ски-симметрично и в сильной степени зависит от угла \(\theta\). Вместе с тем, выяснилось, что при усреднении по углу \(\theta\) остается зависимость \(\psi\)-функции только от \(r\), и максимумы распределения в состояниях с \(l=n-1\) ( т. е. наиболее вероятные расстояния электрона от ядра) приходятся на соответствующие боровские

орбиты. Это показано для трех состояний на рис. 6.2 , где на оси абсцисс длинными вертикальными отрезками отмечены радиусы соответствующих орбит в боровской теории атома водорода. Аналогия с теорией Бора на этом скромном (но любопытном) факте и исчерпывается.

\begin{figure}[h]
\begin{center}
  \includegraphics[alt={},max width=\textwidth]{a57b3a57-78b5-4292-b0ab-4bd9f2ffcf56-138_352_760_367_233}
\captionsetup{labelformat=empty}
\caption{Рис. 6.2}
\end{center}
\end{figure}

\section*{§ 6.2. Уровни и спектры щелочных металлов}
Спектры щелочных металлов. Спектры испускания атомов щелочных металлов, как и спектр атома водорода, состоят из множества спектральных линий. Кропотливая систематика этих спектральных линий позволила сгруппировать их в серии, каждая из которых связана с переходом возбужденного атома на какой-то определенный уровень. Для атомов лития это показано на рис. 6.3.

Схема уровней других щелочных металлов имеет аналогичную структуру.

Анализ полученных результатов позволил сопоставить их с весьма характерной структурой электронной оболочки атомов щелочных металлов. Если атом щелочного металла имеет всего \(Z\) электронов, то можно считать, что \(Z-1\) электронов вместе с ядром образуют сравнительно прочный остов, в электрическом поле которого движется внешний (валентный) электрон, довольно слабо связанный с остовом атома.

В некотором смысле атомы щелочных металлов являются водородоподобными, однако не полностью. Дело в том, что внешний электрон несколько деформирует электронный остов и тем самым искажает поле, в котором движется. В первом приближении поле остова можно рассматривать как суперпозицию поля точечного заряда \(+e\) и поля точечного диполя, распо-

\begin{figure}[h]
\begin{center}
  \includegraphics[alt={},max width=\textwidth]{a57b3a57-78b5-4292-b0ab-4bd9f2ffcf56-139_1527_872_141_179}
\captionsetup{labelformat=empty}
\caption{Рис. 6.3}
\end{center}
\end{figure}

ложенных в центре остова. При этом ось диполя направлена все время к внешнему электрону. Поэтому движение последнего происходит так, как если бы поле остова, несмотря на искажение, сохранялось сферически-симметричным.

Это позволяет представить потенциальную энергию внешнего электрона в поле такого остова как


\begin{equation*}
U(r)=-\frac{e^{2}}{r}-C \frac{e^{2}}{r^{2}}, \tag{6.18}
\end{equation*}


где \(C\) - некоторая постоянная.\\
Решение уравнения Шредингера для электрона с потенциальной энергией (6.18) приводит к тому, что теперь дозволенные значения энергии \(E\) в области \(E<0\) (для связанных состояний внешнего электрона) будут зависеть не только от главного квантового числа \(n\) (как в случае атома водорода), но и от орбитального квантового числа \(l\) :


\begin{equation*}
E_{n l}=-\frac{\hbar R}{\left(n+\sigma_{l}\right)^{2}}, \tag{6.19}
\end{equation*}


где \(\sigma_{l}\) - ридберговская поправка (или квантовый дефект), зависящая от \(l\). Заметим, что у лития (см. рис. 6.3) основным состоянием является \(2 s\), поскольку состояние с \(n=1\) уже занято двумя электронами, входящими в состав остова.

Энергетическому уровню (6.19) соответствует терм, имеющий согласно (2.30) вид


\begin{equation*}
T_{n l}=\frac{R}{\left(n+\sigma_{l}\right)^{2}} . \tag{6.20}
\end{equation*}


Зависимость энергии электрона от орбитального квантового числа \(l\) является принципиальным отличием уровней энергии атомов щелочных металлов от уровней энергии атома водорода. Эта зависимость означает, что в данном случае снимается вырождение по \(l\). Физически это связано с тем, что в атомах щелочных металлов внешний электрон находится в электрическом поле атомного остова. Заряд последнего не точечный, и распределение его несколько отличается от сферически-симметричного. Электрическое поле остова уже не кулоновское (не \(\sim 1 / r^{2}\) ). Благодаря этому и получается зависимость энергии\\
\(E\) электрона не только от \(n\), но и от \(l\). Иными словами, в данном случае вырождение по \(l\) снимается.

Исследование спектров ионов щелочных металлов показало, что момент импульса атомного остова (т. е. ядра и \(Z-1\) электронов) равен нулю. Следовательно, орбитальный момент атома щелочного металла оказывается равным моменту его внешнего электрона и определяется квантовым числом \(l\) (что и показано на рис. 6.3).

Правило отбора. Излучение (и поглощение) происходит в результате перехода внешнего электрона с одного уровня на другой. Однако не все переходы возможны. Возможны лишь те, при которых орбитальное квантовое число \(l\) внешнего электрона меняется на единицу:


\begin{equation*}
\Delta l= \pm 1 . \tag{6.21}
\end{equation*}


Это означает, что разрешенными являются переходы лишь между \(s\) - и \(p\)-состояниями, между \(p\) - и \(d\)-состояниями и т. д. (см. рис. 6.3). Заметим еще, что главное квантовое число \(n\) может изменяться на любое целое число.

Пример. Определим, сколько спектральных линий будет испускать источник, состоящий из атомов лития, которые постоянно возбуждают на уровень \(4 s\).\\
Согласно правилу отбора (6.21) таких переходов будет шесть (см. рис. 6.3), это:

\[
4 s \rightarrow 3 p, \quad 4 s \rightarrow 2 p, \quad 3 p \rightarrow 3 s, \quad 3 p \rightarrow 2 s, \quad 3 s \rightarrow 2 p, \quad 2 p \rightarrow 2 s .
\]

С точки зрения квантовой теории правила отбора связаны с вероятностью перехода из одного квантового состояния в другое. Оказывается, вероятность переходов, не разрешенных правилами отбора, практически равна нулю.

Спектральные серии. С учетом правила отбора (6.21) легко понять, что в спектрах атомов лития наблюдаются следующие серии:

\begin{center}
\begin{tabular}{lll}
главная: & \(\omega=2 s-n p\), & \(n=2,3,4, \ldots\) \\
резкая: & \(\omega=2 p-n s\), & \(n=3,4,5, \ldots\) \\
диффузная: & \(\omega=2 p-n d\), & \(n=3,4,5, \ldots\) \\
\end{tabular}
\end{center}

Здесь \(\omega\) - частота испускаемой линии. Спектральная линия главной серии, соответствующая переходу \(2 s-2 p\), является самой интенсивной. Эту линию называют резонансной.

Серии (6.22) записаны в символической форме. В явном виде их записывают как разности, двух соответствующих термов. Например, для главной серии лития:


\begin{equation*}
\omega=\frac{R}{\left(2+\sigma_{s}\right)^{2}}-\frac{R}{\left(n+\sigma_{p}\right)^{2}}, \quad n=2,3,4, \ldots \tag{6.23}
\end{equation*}


Аналогично и для других серий. Заметим, что ридберговские поправки в пределах каждой серии практически постоянные, но меняются от серии к серии.

Тонкая структура спектральных линий. Исследование спектральных линий атомов щелочных металлов приборами с большой разрешающей способностью обнаружило, что эти линии являются двойными (дублетами), т. е. образуют тонкую струкmypy.

Спектральные линии, состоящие из нескольких компонент, называют мультиплетами. Число компонент в мультиплете различных атомов может быть равно двум (дублеты), трем (триплеты), четырем (квартеты) и т. д. В частности, спектральные линии могут быть и одиночными (синглеты).

Тонкая структура, т. е. расщепление спектральных линий, очевидно, вызвана расщеплением самих энергетических уровней (термов). Вместе с тем, это никак не следует из решения уравнения Шредингера. В чем же причина такого загадочного расщепления? Ответ на этот вопрос - в следующем параграфе.

\section*{§ 6.3. Спин электрона}
Гипотеза спина. Тонкая структура спектральных линий, т. е. их расщепление, как было сказано в конце предыдущего параграфа, является следствием расщепления самих энергетических уровней. Это был первый экспериментальный факт, побудивший Гаудсмита и Уленбека (1925) выдвинуть гипотезу о наличии у электрона собственного момента, названного спином. В дальнейшем эта гипотеза была подтверждена и рядом других весьма убедительных экспериментальных фактов.

Гипотеза спина сразу открыла возможность простого объяснения большого числа экспериментальных фактов, некоторые из которых мы рассмотрим далее.

Спин - существенно квантовая величина, не имеющая классического аналога. Он ничего общего не имеет с представлением о вращающейся частице, как первоначально предполагали (отсюда и название).

Спин характеризует внутреннее свойство электрона подобно массе и заряду. Выяснилось, что спин является свойством одновременно квантовым и релятивистским*. В отличие от орбитального момента, спин всегда сохраняется (как внутреннее свойство).

Спин электрона определяется по общим законам квантовой теории. Аналогично орбитальному моменту, определенные значения в одном и том же состоянии могут иметь квадрат спина \(M_{s}^{2}\) (а значит и модуль спина \(M_{s}\) ) и одна из его проекций \(M_{s z}\) на произвольно выбранную ось \(Z\) :


\begin{equation*}
M_{s}=\hbar \sqrt{s(s+1)}, \quad s=1 / 2, \tag{6.24}
\end{equation*}


где \(s\) - спиновое квантовое число, и


\begin{equation*}
M_{s z}=\hbar m_{s}, \quad m_{s}= \pm s=+1 / 2 \text { и }-1 / 2 . \tag{6.25}
\end{equation*}


Значение \(s=1 / 2\) получено из следующих соображений. Аналогично орбитальному моменту число возможных значений проекции \(m_{s}\), соответствующих данному значению \(s\), равно \(2 s+\) 1. Экспериментально было установлено, что это число для электрона равно двум, т. е. \(2 s+1=2\), откуда \(s=1 / 2\).

Отметим, что спином обладает подавляющее большинство частиц. Например, у протона и нейтрона \(s=1 / 2\), а у фотона \(s=1\).

Поскольку спин электрона \(s=1 / 2\), а его проекции \(m_{s}\) равны \(1 / 2\) и \(-1 / 2\), то становится понятным, почему кратность вырождения \(n\)-го энергетического уровня атома водорода равна не \(n^{2}\), а \(2 n^{2}\). Впрочем, это скорее кратность вырождения не \(n\)-го уровня, а суммарная кратность вырождения двух подуровней, соответствующих квантовому числу \(n\).

\footnotetext{\begin{itemize}
  \item Дирак (1928) показал, что спин электрона автоматически содержится в его теории электрона, основанной на релятивистском волновом уравнении.
\end{itemize}
}Полный момент импульса электрона. С механическими моментами (орбитальным и спиновым) связаны магнитные моменты. В результате их взаимодействия происходит сложение моментов - возникает полный момент импульса электрона. Символически это записывают так: \(\mathbf{M}_{j}=\mathbf{M}_{l}+\mathbf{M}_{s}\), где \(j\) - квантовое число полного момента.

Правила сложения угловых моментов в квантовой теории не зависят от того, являются ли моменты орбитальными или спиновыми. Поэтому полный момент электрона \(M_{j}\) определяется формулой, аналогичной формулам для орбитального и спинового моментов, а именно


\begin{equation*}
M_{j}=\hbar \sqrt{j(j+1)}, \quad j=l+s=l \pm 1 / 2 . \tag{6.26}
\end{equation*}


Таким образом, квантовое число \(j\) является полуцельлм, поскольку \(l\) - целое, причем, если \(l=0\), то \(j=s=1 / 2\). Кроме того, \(j\) всегда положительно.

В связи со знаками \(\pm\) перед спином \(s\) в (6.26) условно принято говорить, что спиновый момент либо «сонаправлен» с орбитальным моментом (знак +), либо они взаимно противоположны «по направлению» (знак -).

Возможные проекции момента (6.26) на ось \(Z\) определяются как


\begin{equation*}
M_{j z}=\hbar m_{j}, \quad m_{j}=j, \quad j-1, \quad j-2, \quad \ldots, \quad-j, \tag{6.27}
\end{equation*}


т. е. при данном \(j\) возможны \(2 j+1\) квантовых состояний, отличающихся значениями \(m_{j}\). Например, при \(l=1\)

\[
\begin{array}{cccc}
j_{1}=1+1 / 2=3 / 2, & m_{j}=3 / 2, \quad 1 / 2, \quad-1 / 2, \quad-3 / 2, \\
j_{2}=1-1 / 2=1 / 2, & m_{j}=1 / 2, \quad-1 / 2 .
\end{array}
\]

Если же \(l=0\), то весь момент импульса чисто спиновый.\\
Общие результаты. Выпишем собственные значения угловых моментов (орбитального, спинового и полного) и их проекций на ось \(Z\) в одной таблице (табл. 6.3), чтобы обратить внимание на их однотипность и облегчить запоминание.

\begin{table}[h]
\begin{center}
\captionsetup{labelformat=empty}
\caption{Таблица 6.3}
\begin{tabular}{||lc||}
\hline
\begin{tabular}{l}
\(M_{l}=\hbar \sqrt{l(l+1)}\), \\
\(M_{l z}=\hbar m_{l}\), \\
\end{tabular} & \begin{tabular}{l}
\(l=0,1,2, \ldots\) \\
\(m_{l}=0, \pm 1, \pm 2, \ldots, \pm l\). \\
\end{tabular} \\
\hline
\begin{tabular}{l}
\(M_{s}=\hbar \sqrt{s(s+1)}\), \\
\(M_{s z}=\hbar m_{s}\), \\
\end{tabular} & \begin{tabular}{l}
\(s=1 / 2\), \\
\(m_{s}=+1 / 2,-1 / 2\). \\
\end{tabular} \\
\hline
\(M_{j}=\hbar \sqrt{j(j+1)}\), & \(j=l \pm s=l \pm 1 / 2\), \\
\(M_{j z}=\hbar m_{j}\), & \(m_{j}=j, j-1, \ldots,-j\). \\
\hline\hline
\end{tabular}
\end{center}
\end{table}

В дальнейшем на эти формулы мы будем неоднократно ссылаться.

Тонкая структура. Рассмотрим на примере атома лития, как с помощью спина можно объяснить дублетную структуру линий спектра. Вследствие того, что момент атомного остова равен нулю (см. стр. 140), момент атома лития равен моменту внешнего (валентного) электрона. Момент же этого электрона равен сумме орбитального момента и спинового. Полный момент данного электрона согласно (6.30) определяется квантовым числом \(j\) :

\[
j=l \pm 1 / 2,
\]

где \(l\) и \(1 / 2\) - орбитальное и спиновое квантовые числа. Причем, в случае \(l=0\) квантовое число \(j\) имеет только одно значение: \(j=1 / 2\).

Мы уже знаем, что моменты \(\mathbf{M}_{l}\) и \(\mathbf{M}_{s}\) взаимодействуют друг с другом. Энергия этого взаимодействия зависит от взаимной «ориентации» орбитального и спинового моментов, что и приводит к расщеплению энергетических уровней.

Таким образом, каждый уровень (терм) ряда \(P(l=1)\) расщепляется на два подуровня с \(j=1 / 2\) и \(3 / 2\), каждый уровень ряда \(D(l=2)\) - на подуровни с \(j=3 / 2\) и \(5 / 2\) и т. д. Исключение составляют уровни ряда \(S(l=0)\), которым соответствует только одно значение \(j=1 / 2\); поэтому уровни этого ряда не расщепляются (остаются синглетными).

Итак, каждый ряд уровней, кроме \(S\)-ряда, имеет дублетную структуру. Уровни (термы) принято обозначать символом, определяющим значения квантовых чисел \(l, s\) и \(j\), т. е. по существу полностью «структуру» углового момента электрона. Символически это записывают так:


\begin{equation*}
{ }^{v}(L)_{j}, \tag{6.31}
\end{equation*}


где \(L\) - символ состояния, определяемого квантовым числом \(l\) - в соответствии с (6.10), только большими латинскими буквами: \(S, P, D\) и т. д.; \(v\) - так называемая мультиплетность, она связана со спином: \(v=2 s+1\).

Выпишем несколько первых рядов термов атома щелочных металлов:


\begin{equation*}
{ }^{2} S_{1 / 2} ; \quad{ }^{2} P_{1 / 2},{ }^{2} P_{3 / 2} ; \quad{ }^{2} D_{3 / 2},{ }^{2} D_{5 / 2} ; \quad \ldots \tag{6.32}
\end{equation*}


Для атомов щелочных металлов дублетное расщепление очень мало (по сравнению с расстояниями между «основными» уровнями).

Величина тонкого расщепления уровней для легких атомов не более \(10^{-5}\) эВ. Для тяжелых же может достигать десятых долей эВ (это уже трудно назвать тонким расщеплением). Для сравнения приведем разность между двумя уровнями на рис. 6.3, которая равна \(\sim 2\) эВ.

Правила отбора для \(j\). Для квантового числа \(j\) действует правило отбора, согласно которому возможны только те переходы между уровнями, при которых


\begin{equation*}
\Delta j=0, \pm 1 . \tag{6.33}
\end{equation*}


Тонкая структура спектральных линий была обнаружена экспериментально и у атома водорода. Но расщепление уровней атома водорода оказалось слишком мало. и поэтому чаще всего им просто пренебрегают (за исключением очень тонких исследований).

Закономерности тонкой структуры. Поясним происхождение тонкой структуры спектральных линий, например, лития, в трех случаях.

\begin{figure}[h]
\begin{center}
  \includegraphics[alt={},max width=\textwidth]{a57b3a57-78b5-4292-b0ab-4bd9f2ffcf56-147_241_201_141_70}
\captionsetup{labelformat=empty}
\caption{щих дублетов.}
\end{center}
\end{figure}

\begin{figure}[h]
\begin{center}
  \includegraphics[alt={},max width=\textwidth]{a57b3a57-78b5-4292-b0ab-4bd9f2ffcf56-147_237_203_459_68}
\captionsetup{labelformat=empty}
\caption{Рис. 6.5}
\end{center}
\end{figure}

Главная серия. В результате переходов с близко отстоящих друг от друга подуровней \(p\)-термов на один и тот же уровень \(2 s\) возникают две близко расположенные линии, т. е. дублет (рис. 6.4). Расщепление различных \(p\)-термов различно, отсюда и наблюдаемое различие расщепления соответствую-

Резкая серия. Переходы с \(s\)-уровней на \(2 p\)-уровень (рис. 6.5) приводит к одному и тому же расщеплению линий этой серии, поскольку у всех линий оно обусловлено расщеплением одного и того же уровня \(2 p\).

Диффузная серия. Вследствие переходов с \(d\)-уровней на \(2 p\)-уровень (рис. 6.6) - спектральные

\begin{figure}[h]
\begin{center}
  \includegraphics[alt={},max width=\textwidth]{a57b3a57-78b5-4292-b0ab-4bd9f2ffcf56-147_261_221_766_68}
\captionsetup{labelformat=empty}
\caption{Рис. 6.6}
\end{center}
\end{figure}

линии оказываются триплетами, так как переходы, в которых квантовое число \(j\) меняется на 2 , запрещено правилом отбора (6.33). Таковым является переход \(d_{5 / 2}-2 p_{1 / 2}\), изображенный пунктиром. Расщепление \(d\)-уровней значительно меньше расщепления \(2 p\)-уровня. Поэтому компоненты триплета не всегда разрешаются, а сами линии получаются размытыми (отсюда и название серии).\\
Таким образом, тонкая структура уровней и спектральных линий атомов щелочных металлов обусловлены спином электрона, или, что то же, спин-орбитальньлм взаимодействием.

В заключение рассмотрим пример, с решением которого нередко возникают затруднения.\\
Пример. У атомов некоторого щелочного металла головная линия резкой серии с длиной волны \(\lambda\) представляет собой дублет, разность длин волн которого \(\Delta \lambda\). Найдем величину расщепления в частотах \(\omega\) следующих линий этой серии.\\
Поскольку все линии резкой серии обусловлены переходом с синглетных \(s\)-уровней на один и тот же расщепленный нижний \(p\)-уровень, то разность энергий переходов будет одинакова в каждом дублете. Значит одинаковым будет и расщепление \(\Delta \omega\). В нашем случае \(\Delta \lambda \ll \lambda\), поэтому, учитывая связь \(\omega=2 \pi c / \lambda\), можно записать:

\[
\Delta \omega=\frac{2 \pi c}{\lambda^{2}} \Delta \lambda .
\]

\section*{§ 6.4. Механический момент многоэлектронного атома}
Сложение угловых моментов. Как показывает расчет (который мы опускаем), суммарный орбитальный момент системы определяется выражением


\begin{equation*}
M_{L}=\hbar \sqrt{L(L+1)}, \tag{6.34}
\end{equation*}


где \(L\) - орбитальное квантовое число результирующего момента. В случае системы из двух частиц с орбитальными моментами \(l_{1}\) и \(l_{2}\) квантовое число \(L\) - целое, положительное - может иметь следующие значения:


\begin{equation*}
L=\left(l_{1}+l_{2}\right), \quad\left(l_{1}+l_{2}-1\right), \quad \ldots, \quad\left|l_{1}-l_{2}\right| . \tag{6.35}
\end{equation*}


Отсюда следует, что \(L\) (а значит и результирующий момент) может иметь \(2 l_{1}+1\) или \(2 l_{2}+1\) различных значений (нужно взять меньшее из двух значений \(l\) ). Это легко проверить; например, для \(l_{1}=2 \quad l_{2}=3\) получаем \(2 \cdot 2+1=5\) разных значений \(L: 5,4\), 3, 2, 1.

Если система состоит не из двух, а из многих частиц, то квантовое число \(L\), определяющее результирующий орбитальный момент, находится путем последовательного применения правила (6.35), но мы не будем на этом останавливаться, поскольку в дальнейшем это не понадобится.

Проекция результирующего орбитального момента на некоторое направление \(Z\) определяется аналогично (6.28):


\begin{equation*}
M_{z}=\hbar m_{L}, \quad m_{L}=0, \pm 1, \pm 2, \ldots, \pm L . \tag{6.36}
\end{equation*}


Подобным же образом определяется и суммарный спиновый момент системы:


\begin{equation*}
M_{S}=\hbar \sqrt{S(S+1)}, \tag{6.37}
\end{equation*}


где квантовое число \(S\) результирующего спинового момента может быть целым или полуцелым - в зависимости от числа частиц - четного или нечетного. Если число \(N\) частиц четное, то \(S=N s, N s-1, \ldots, 0\), где \(s=1 / 2\), т. е. в этом случае \(S\) - целые числа. Например, при \(N=4\) число \(S\) может быть равно \(2,1,0\).

Если же число \(N\) частиц нечетное, то \(S\) принимает все полуцелые значения от \(N s\) до \(s\), где \(s=1 / 2\). Например, при \(N=5\) возможные значения \(S\) равны \(5 / 2,3 / 2\) и \(1 / 2\).

Типы связи. В многоэлектронном атоме каждый электрон можно характеризовать орбитальным и спиновым моментами. Возникает естественный вопрос: чему равен полный механический момент атома? Ответ на этот вопрос зависит от того, какие моменты взаимодействуют друг с другом сильнее: орбитальные, спиновые или спин-орбитальные.

Оказывается, наиболее важной и распространенной является так называемая нормальная связь, или связь Рессель-Саундерса. Эта связь заключается в том, что орбитальные моменты электронов взаимодействуют между собой сильнее, чем со спиновыми моментами. Аналогично ведут себя и спиновые моменты. Вследствие этого все орбитальные моменты складываются в результирующий орбитальный момент \(M_{L}\), а спиновые - в результирующий спиновый момент \(M_{S}\). А затем взаимодействие \(M_{L}\) и \(M_{S}\) определяет суммарный момент \(M_{J}\) атома:


\begin{equation*}
M_{J}=\hbar \sqrt{J(J+1)}, \tag{6.38}
\end{equation*}


где квантовое число \(J\) полного момента может иметь одно из следующих значений:

\[
J=L+S, \quad L+S-1, \quad \ldots, \quad|L-S| .
\]

Значит, \(J\) будет целым, если \(S\) целое ( т. е. при четном числе электронов) или полуцелым, если \(S\) полуцелое (при нечетном числе электронов). Так например,

\begin{center}
\begin{tabular}{c|c|c}
\(L\) & \(S\) & Возможные значения \(J\) \\
\hline
2 & 1 & \(3,2,1\) \\
2 & \(3 / 2\) & \(7 / 2,5 / 2,3 / 2,1 / 2\) \\
\end{tabular}
\end{center}

Такой вид связи, как правило, присущ легким и не слишком тяжелым атомам.

Однако нормальная связь является не единственно возможной. Это только один из крайних случаев связи. Другой край-

ний случай - так называемая \(j-j\) связь, когда спин-орбитальное взаимодействие у каждого электрона оказывается основным. В этом случае суммарный момент атома \(\mathbf{M}_{J}=\sum \mathbf{M}_{j}\), т. е. равен сумме отдельных спин-орбитальных моментов \(\mathbf{M}_{j}\).

Такая связь встречается у тяжелых атомов, но достаточно редко. В основном же осуществляются более сложные промежуточные виды связи. Но мы их затрагивать не будем, ограничившись в основном только нормальной связью, наиболее важной и чаще встречающейся.

Спектральные обозначения. В случае нормальной связи термы принято обозначать символами, подобными (6.31):


\begin{equation*}
{ }^{v}(L)_{J}, \tag{6.39}
\end{equation*}


где \(v=2 S+1\) - мультиплетность, \(J\) - квантовое число полного момента. Отличие с обозначением (6.31) лишь в том, что малые буквы \(s\) и \(j\) заменены на соответствующие большие \(S\) и \(J\).

Приведем примеры термов систем с двумя электронами. Здесь возможны два случая: \(S=0\) (спины электронов противоположны) и \(S=1\) (спины сонаправлены).

В первом случае \(J=L\) и \(2 S+1=1\), т. е. все термы - синглеты. Во втором случае \(2 S+1=3\), т. е. все три терма - триплеты. Причем во втором случае возможны три значения \(J: L+1, L\) и \(|L-1|\). Сказанное сведено для наглядности в таблицы 6.4 и 6.5.

Таблица 6.4

Таблица 6.5

\begin{center}
\begin{tabular}{||c|c|c|c||}
\hline\hline
\(S\) & \(L\) & \(J\) & Синглеты \\
\hline
0 & 0 & 0 & \({ }^{1} S_{0}\) \\
0 & 1 & 1 & \({ }^{1} P_{1}\) \\
0 & 2 & 2 & \({ }^{1} D_{2}\) \\
\hline\hline
\end{tabular}
\end{center}

\begin{center}
\begin{tabular}{||c|c|c|cc||}
\hline\hline
S & \(L\) & \(J\) & \multicolumn{2}{|c|}{Триплеты} \\
\hline
1 & 0 & 1,0 & \({ }^{3} S_{1}\) & \({ }^{3} S_{0}\) \\
1 & 1 & \(2,1,0\) & \({ }^{3} P_{2}\) & \({ }^{3} P_{1}\) \\
\({ }^{3} P_{0}\) &  &  &  &  \\
1 & 2 & \(3,2,1\) & \({ }^{3} D_{3}\) & \({ }^{3} D_{2}\) \\
\({ }^{3} D_{1}\) &  &  &  &  \\
\hline\hline
\end{tabular}
\end{center}

Следует отметить, что мультиплетность \(v\) дает количество подуровней только в случае \(S<L\) (в случае же \(S>L\), число подуровней равно \(2 L+1\) ).

Правила отбора. При рассмотрении внешнего электрона в атомах щелочных металлов было отмечено, что не все переходы

между термами возможны. Возможны только те, которые подчиняются правилам отбора (6.21) и (6.33).

При переходе к сложным атомам правила отбора необходимо уточнить. Эмпирически было установлено, что при нормальной связи правила отбора для квантовых чисел \(L, S\) и \(J\) таковы:


\begin{align*}
& \Delta L=0, \pm 1  \tag{6.40}\\
& \Delta S=0  \tag{6.41}\\
& \Delta J=0, \pm 1 \tag{6.42}
\end{align*}


При этом, однако, переход \(J=0 \rightarrow J=0\) запрещен.\\
Указанные правила отбора обоснованы квантовой теорией и не всегда являются достаточно жесткими (впрочем, эти случаи мы рассматривать не будем). Напомним, суть этих правил в том, что только при таких изменениях квантовых чисел \(L, S, J\) вероятность переходов является существенной.

\section*{§ 6.5. Принцип Паули. Заполнение электронных оболочек}
В предыдущих параграфах этой главы мы выяснили, что электроны в атомах могут находиться в различных состояниях, которым соответствуют разные наборы четверки квантовых чисел \(n, l, m_{l}, m_{s}\) или \(n, l, j, m_{j}\).

Пусть атом находится в невозбужденном состоянии. Выясним, в каких состояниях при этом могут находиться его электроны.

На первый взгляд представляется, что все электроны должны заполнить уровень с наименьшей возможной энергией. Опыт же показывает, что это не так.

По мере увеличения порядкового номера \(Z\) атома происходит последовательное строго определенное заполнение электронных уровней атома. Объяснение такого порядка заполнения уровней нашел Паули (1940). Это было великое открытие, названное впоследствии принципом Паули:

в любом квантовом состоянии может находиться не более одного электрона.\\
Поэтому каждый следующий электрон невозбужденного атома должен занимать самый глубокий из еще незаполненных

уровней. Тщательная проверка явилась убедительным подтверждением принципа Паули.

Другими словами, в атоме (и в любой квантовой системе) не может быть электронов с одинаковыми значениями всех четырех квантовых чисел.

Именно принцип Паули объяснил, почему электроны в атомах оказываются не все на самом нижнем дозволенном энергетическом уровне.

В § 6.3 было показано, что данному значению \(n\) соответствует \(2 n^{2}\) состояний, отличающихся друг от друга значениями квантовых чисел \(l, m_{l}, m_{s}\). Совокупность электронов атома с одинаковыми значениями квантового числа \(n\) образует так называемую оболочку. В соответствии со значением \(n\) оболочки обозначают большими буквами латинского алфавита следующим образом:

\begin{center}
\begin{tabular}{l|c|c|c|c|c|c}
Значение \(n\) & 1 & 2 & 3 & 4 & 5 & 6 \\
\hline
Оболочка & \(K\) & \(L\) & \(M\) & \(N\) & \(O\) & \(P\) \\
\hline
\end{tabular}
\end{center}

Оболочки подразделяют на подоболочки*, отличающиеся квантовым числом \(l\). Различные состояния в подоболочке отличаются значениями квантовых чисел \(m_{l}\) и \(m_{s}\). Число состояний в подоболочке равно \(2(2 l+1)\). Подоболочки обозначают или большой латинской буквой с числовым индексом ( \(K, L_{1}, L_{2}, \ldots\) ) или в виде

\[
1 s ; \quad 2 s, 2 p ; \quad 3 s, 3 p, 3 d ; \ldots,
\]

где цифра означает квантовое число \(n\), т. е. принадлежность к соответствующей оболочке ( \(K, L, M, \ldots\) ).

Возможные состояния электронов в атоме и их распределение по оболочкам и подоболочкам показано в табл. 6.6 , в которой вместо обозначений \(m_{s}=+1 / 2\) и \(-1 / 2\) использованы для наглядности стрелки \(\uparrow\) и ↓ . Видно, что число возможных состояний в \(K, L, M, \ldots\) оболочках равно соответственно \(2,8,18, \ldots\), т. е. равно \(2 n^{2}\).

\footnotetext{\begin{itemize}
  \item Некоторые авторы оболочки называют слоями, а подоболочки - оболочками, поэтому надо быть внимательным к принятой терминологии.
\end{itemize}
}\begin{table}[h]
\begin{center}
\captionsetup{labelformat=empty}
\caption{Таблица 6.6}
\begin{tabular}{|l|l|l|l|l|l|l|l|l|l|l|l|l|}
\hline
Оболочка & \(K\) & \multicolumn{3}{|c|}{\(L\)} & \multicolumn{8}{|c|}{\(M\)} \\
\hline
Подоболочка ( \(n, l\) ) & \(1 s\) & \(2 s\) &  & \(2 p\) & \(3 s\) & \multicolumn{3}{|c|}{\(3 p\)} & \multicolumn{4}{|c|}{\(3 d\)} \\
\hline
\(m_{l} m_{s}\) & 0 ↑↓ & 0 ↑ ↓ & +1 ↑↓ & 0 ↑ ↓ & 0 ↑ ↓ & +1 ↑ ↓ & 0 ↑ ↓ & \(-1\) ↑ ↓ & +2 ↑ ↓ & +1 ↑ ↓ & 0 ↑ ↓ & -1 -2 ↑↓ \(\uparrow \downarrow\) \\
\hline
Число электронов & 2 & 2 &  & 6 & 2 &  & 6 &  & \multicolumn{4}{|c|}{10} \\
\hline
\end{tabular}
\end{center}
\end{table}

Полностью заполненные оболочки и подоболочки имеют \(L=0\) и \(S=0\), значит и \(J=0\). Например, \(3 d\)-подоболочка: суммарное квантовое число \(m_{L}=\sum m_{l}=0\), и это единственное значение, поэтому \(L=0\). Аналогично относительно спина. Значит, действительно, \(J=0\).

Это важный результат: при определенных квантовых числах \(L\) и \(S\) атома заполненные подоболочки можно не принимать во внимание (мы это уже использовали в случае атомов щелочных металлов).

Пример. Выясним, у атома какого элемента заполнены \(K, L\) и \(M\)-оболочки, \(4 s\)-подоболочка и наполовину \(4 p\)-подоболочка.

Решение этого вопроса сводится к нахождению атомного номера \(Z\), который равен числу электронов в атоме. В каждой оболочке находится \(2 n^{2}\) электронов. Значит, в заполненных \(K, L, M\)-оболочках содержится

\[
2+8+18=28 \text { электронов. }
\]

В \(4 s\)-подоболочке - два электрона и в наполовину заполненной \(4 p\)-подоболочке три электрона. Таким образом, всего электронов \(28+2+3=33\). Это и есть \(Z\), что соответствует атому As.

\section*{§ 6.6. О периодической системе элементов Д. И. Менделеева}
Понимание периодической системы элементов основано на идее об оболочечной структуре электронного облака атома. Процесс застройки первых 22-х элементов периодической системы представлен в таблице 6.7.

\begin{table}[h]
\begin{center}
\captionsetup{labelformat=empty}
\caption{Таблица 6.7}
\begin{tabular}{|l|l|l|l|l|l|l|l|l|l|}
\hline
\multirow{2}{*}{Элемент \(Z\)} & \(K\) & \multicolumn{2}{|c|}{L} & \multicolumn{3}{|c|}{M} & \multicolumn{2}{|c|}{N} & \multirow{2}{*}{Основной терм} \\
\hline
 & \(1 s\) & \(2 s\) & \(2 p\) & \(3 s\) & \(3 p\) & \(3 d\) & \(4 s\) & \(4 p\) &  \\
\hline
1 H & 1 & - & - & - & - & - & - & - & \begin{tabular}{l}
\({ }^{2} S_{1 / 2}\) \\
\({ }^{1} S_{0}\) \\
\end{tabular} \\
\hline
2 He & 2 & - & - & - & - & - & - & - &  \\
\hline
3 Li & 2 & 1 & - & - & - & - & - & - & \({ }^{2} S_{1 / 2}\) \\
\hline
4 Be & 2 & 2 & - & - & - & - & - & - & \({ }^{1} S_{0}\) \\
\hline
5 B & 2 & 2 & 1 & - & - & - & - & - & \({ }^{2} P_{1 / 2}\) \\
\hline
6 C & 2 & 2 & 2 & - & - & - & - & - & \({ }^{3} P_{0}\) \\
\hline
7 N & 2 & 2 & 3 & - & - & - & - & - & \({ }^{4} S_{3 / 2}\) \\
\hline
8 O & 2 & 2 & 4 & - & - & - & - & - & \({ }^{3} P_{2}\) \\
\hline
9 F & 2 & 2 & 5 & - & - & - & - & - & \({ }^{2} P_{3 / 2}\) \\
\hline
10 Ne & 2 & 2 & 6 & - & - & - & - & - & \({ }^{1} S_{0}\) \\
\hline
11 Na & 2 & 2 & 6 & 1 & - & - & - & - & \({ }^{2} S_{1 / 2}\) \\
\hline
12 Mg & 2 & 2 & 6 & 2 & - & - & - & - & \({ }^{1} S_{0}\) \\
\hline
13 Al & 2 & 2 & 6 & 2 & 1 & - & - & - & \({ }^{2} P_{1 / 2}\) \\
\hline
14 Si & 2 & 2 & 6 & 2 & 2 & - & - & - & \({ }^{3} P_{0}\) \\
\hline
15 P & 2 & 2 & 6 & 2 & 3 & - & - & - & \({ }^{4} S_{3 / 2}\) \\
\hline
16 S & 2 & 2 & 6 & 2 & 4 & - & - & - & \({ }^{3} P_{2}\) \\
\hline
17 Cl & 2 & 2 & 6 & 2 & 5 & - & - & - & \({ }^{2} P_{3 / 2}\) \\
\hline
18 Ar & 2 & 2 & 6 & 2 & 6 & - & - & - & \({ }^{1} S_{0}\) \\
\hline
19 K & 2 & 2 & 6 & 2 & 6 & - & 1 & - & \({ }^{2} S_{1 / 2}\) \\
\hline
20 Ca & 2 & 2 & 6 & 2 & 6 & - & 2 & - & \({ }^{1} S_{0}\) \\
\hline
21 Sc & 2 & 2 & 6 & 2 & 6 & 1 & 2 & - & \({ }^{2} D_{3 / 2}\) \\
\hline
22 Ti & 2 & 2 & 6 & 2 & 6 & 2 & 2 & - & \({ }^{3} F_{2}\) \\
\hline
\end{tabular}
\end{center}
\end{table}

Каждый следующий атом получается из предыдущего добавлением заряда ядра на единицу ( \(e\) ) и добавлением одного электрона, который помещают в разрешенное принципом Паули состояние с наименьшей энергией. Так, третий элемент (литий) имеет, кроме заполненной \(K\)-оболочки, один электрон в подоболочке \(2 s\). Этот электрон связан с ядром слабее других и является внешним (валентным, оптическим). Основное состояние этого электрона характеризуется значением \(n=2\). Этим, кстати, и объясняется, почему на схеме уровней атома лития (см. рис. 6.3) основной уровень помечен цифрой 2 .

\section*{Некоторые комментарии к табл. 6.7.}
\begin{enumerate}
  \item Распределение электронов по состояниям называют электронной конфигурацией. Их обозначают символически, например, так:
\end{enumerate}

\[
1 s^{2} 2 s^{2} 2 p^{6} 3 s .
\]

Это означает, что в атоме имеются два \(1 s\)-электрона, два \(2 s\)-электрона, шесть \(2 p\)-электронов и один \(3 s\)-электрон. Из таблицы 6.7 видно, что это - электронная конфигурация атома Na.\\
2. Оболочку (или подоболочку), полностью заполненную электронами, называют замкнутой. В предыдущем параграфе мы установили, что все три квантовых числа \((L, S, J)\) у замкнутых оболочек (и подоболочек) равны нулю. Основными термами таких оболочек являются \({ }^{1} \mathrm{~S}_{0}\). В таблице 6.7 это у атомов \(\mathrm{He}, \mathrm{Be}\), \(\mathrm{Ne}, \mathrm{Mg}\) и др.

Электроны в каждой подоболочке называют эквивалентны\(м и\), у них одинаковые значения \(n\) и \(l\).\\
3. Вплоть да атома калия K последовательность заполнения оболочек и подоболочек имеет «идеальный» характер. Первый «сбой» происходит с атомом K: внешний электрон занимает, вместо \(3 d\)-состояния, \(4 s\). Подобное - не единственный случай в периодической системе, и связано это с тем, что такие конфигурации оказываются более выгодными в энергетическом отношении (расчет это полностью подтвердил).\\
4. Наблюдаемая периодичность химических и ряда физических свойств атомов объясняется поведением внешних валентных электронов. Выяснилось, что эта периодичность связана с определенной периодичностью электронной конфигурации атомов, в частности, с конфигурацией внешних электронов.\\
5. В правой колонке табл. 6.7 приведены основные термы атомов. Для первых четырех атомов определение основного состояния не вызывает трудности - для этого достаточно принципа Паули. Но уже для бора В возникает неопределенность: одному \(p\)-электрону соответствует \(l=1\) и \(s=1 / 2\), откуда \(j=3 / 2\) или \(1 / 2\), т. е. два состояния: \(P_{3 / 2}\) и \(P_{1 / 2}\). Какое из них является основным, можно решить лишь с помощью правил Хунда.

Правила Хунда. Это полуэмпирические правила, относящиеся к системе эквивалентных электронов (у них \(n\) и \(l\) одинаковы), т. е. для электронов, находящихся в одной подоболочке. Этих правил два:

\begin{enumerate}
  \item Минимальной энергией данной электронной конфигурации обладает терм с наибольшим возможным значением спина \(S\) и с наибольшим возможным при таком \(S\) значении \(L\).
  \item При этом квантовое число\\
\(J=\left\{\begin{array}{l}|L-S|, \text { если подоболочка заполнена менее, чем наполовину, } \\ L+S \text { в остальных случаях. }\end{array}\right.\)\\
Применим эти правила к \(p\)-оболочке. В ней всего могут находиться \(2(2 l+1)=6\) электронов. Возьмем, например, атом кислорода О (у него электронная конфигурация, как видно из табл. 6.7 , имеет вид \(1 s^{2} 2 s^{2} p^{4}\) ), т. е. \(p\)-подоболочка заполнена не полностью. Изобразим состояние с различными значениями \(m_{l}\). Для \(p\)-подоболочки это будут \(+1 / 2,0\) и \(-1 / 2\), т.е. три ячейки:
\end{enumerate}

\begin{center}
\begin{tabular}{c|c|c|c}
\(m_{l}\) & \(+1 / 2\) & 0 & \(-1 / 2\) \\
\hline
\(m_{s}\) & \(\uparrow \downarrow\) & \(\uparrow\) & \(\uparrow\) \\
\hline
\end{tabular}
\end{center}

Затем будем заполнять эти состояния (ячейки) электронами. У каждого электрона \(m_{s}=+1 / 2\) или \(-1 / 2\). Ддя наглядности эти значения \(m_{s}\) будем, как и раньше, изображать стрелками \(\uparrow\) и ↓ соответственно.

Начнем с заполнения ячеек спинами ↑ (таких в каждой ячейке может быть не более одного согласно принципу Паули). Оставшийся четвертый электрон со спином ↓ надо поместить в такую ячейку, \(m_{l}\) которой максимально. Этим самым мы обеспечиваем максимальные значения \(m_{S}\) и \(m_{L}\) :

\[
m_{S}=\sum m_{s}=1, \quad m_{L}=\sum m_{l}=1 .
\]

Но максимальные значения \(m_{S}\) и \(m_{L}\) равны \(S\) и \(L\), т. е. \(S=1\) и \(L=1\).

В данном случае подоболочка заполнена более, чем наполовину, поэтому согласно второму правилу Хунда

\[
J=L+S=2
\]

Таким образом, основной терм данной конфигурации это \({ }^{3} P_{2}\).\\
Вернемся к атому бора В. У него в незаполненной \(p\)-подоболочке только один электрон. Легко сообразить, что в этом случае максимальные значения \(m_{S}=1 / 2\) и \(m_{L}=+1\), а значит, \(L=1\) и \(S=1 / 2\). Подоболочка заполнена менее, чем наполовину, поэтому \(J=|L-S|=1 / 2\). И мы приходим к тому, что основным термом является \({ }^{2} P_{1 / 2}\).

Полезно убедиться самостоятельно (с помощью правил Хунда) в справедливости распределения \(p\)-электронов по ячейкам \(m_{l}\) для конфигураций \(p^{2}, p^{3}\) и \(p^{5}\), приведенных в нижеследующих табличках, и соответствующего каждому из них основного терма:

\begin{figure}[h]
\begin{center}
\captionsetup{labelformat=empty}
\caption{1) \(p^{2}\)}
  \includegraphics[alt={},max width=\textwidth]{a57b3a57-78b5-4292-b0ab-4bd9f2ffcf56-157_116_243_693_135}
\end{center}
\end{figure}

\begin{figure}[h]
\begin{center}
\captionsetup{labelformat=empty}
\caption{2) \(p^{3}\)}
  \includegraphics[alt={},max width=\textwidth]{a57b3a57-78b5-4292-b0ab-4bd9f2ffcf56-157_116_271_693_494}
\end{center}
\end{figure}

\begin{figure}[h]
\begin{center}
\captionsetup{labelformat=empty}
\caption{3) \(p^{5}\)}
  \includegraphics[alt={},max width=\textwidth]{a57b3a57-78b5-4292-b0ab-4bd9f2ffcf56-157_116_289_693_859}
\end{center}
\end{figure}

Рассмотрим в качестве примера обратную задачу.\\
Пример. Найдем с помощью правил Хунда число электронов в единственной незаполненной подоболочке атома, основной терм которого \({ }^{3} F_{2}\).\\
Символ \(F\) означает, что \(L=3\). Спиновое число находим из мультиплетности: \(3=2 S+1\), откуда \(S=1\). Поскольку \(J=2\), то оно может быть представлено только как \(J=L-S\), а это значит, согласно второму правилу Хунда, что подоболочка \(d\) (ей отвечает \(L=3\) ) заполнена менее, чем наполовину, и только таким способом:

\begin{center}
\begin{tabular}{c|c|c|c|c|c}
\(m_{l}\) & +2 & +1 & 0 & -1 & -2 \\
\hline
\(m_{s}\) & \(\uparrow\) & \(\uparrow\) &  &  &  \\
\hline
\end{tabular}
\end{center}

Ей соответствует электронная конфигурация \(d^{2}\). Например, атом титана \(\mathrm{Ti}\left(1 s^{2} 2 s^{2} p^{6} 3 s^{2} p^{6} d^{2} 4 s\right)\).

\section*{§ 6.7. Характеристические рентгеновские спектры}
Рентгеновские спектры, возникающие при бомбардировке электронами антикатода рентгеновской трубки, бывают двух видов: сплошные и линейчатые. Сплошные спектры возникают при торможении быстрых электронов в веществе антикатода и явля-

ются обычным тормозным излучением электронов (см. § 1.2). Вид этих спектров не зависит от материала антикатода.

При повышении напряжения на трубке наряду со сплошным спектром появляется линейчатый. Он состоит из отдельных линий и зависит от материала антикатода. Каждый элемент обладает своим, характерным для него линейчатым спектром. Поэтому такие спектры называют характеристическими.

С увеличением напряжения на рентгеновской трубке коротковолновая граница сплошного спектра смещается (см. § 1.2), линии же характеристического спектра становятся лишь более интенсивными, не меняя своего расположения.

\section*{Особенности характеристических спектров.}
\begin{enumerate}
  \item В отличие от оптических линейчатых спектров с их сложностью и разнообразием, рентгеновские характеристические спектры различных элементов отличаются простотой и однообразием. С ростом атомного номера \(Z\) элемента они монотонно смещаются в коротковолновую сторону.
  \item Характеристические спектры разных элементов имеют сходный характер (однотипны) и не меняются, если интересующий нас элемент находится в соединении с другими. Это можно объяснить лишь тем, что характеристические спектры возникают при переходах электронов во внутренних частях атома, частях, имеющих сходное строение.
  \item Характеристические спектры состоят из нескольких серий: \(K, L, M, \ldots\) Каждая серия - из небольшого числа линий: \(K_{\alpha}, K_{\beta}\), \(K_{\gamma}, \ldots L_{\alpha}, L_{\beta}, L_{\gamma}, \ldots\) и т. д. в порядке убывания длины волны \(\lambda\).
\end{enumerate}

Анализ характеристических спектров привел к пониманию, что атомам присуща система рентгеновских термов \(K, L, M, \ldots\) (рис. 6.7). На этом же рисунке показана схема возникновения характеристических спектров. Возбуждение атома возникает при

\begin{figure}[h]
\begin{center}
  \includegraphics[alt={},max width=\textwidth]{a57b3a57-78b5-4292-b0ab-4bd9f2ffcf56-158_523_499_1218_661}
\captionsetup{labelformat=empty}
\caption{Рис. 6.7}
\end{center}
\end{figure}

удалении одного из внутренних электронов (под действием электронов или фотонов достаточно большой энергии). Если вырывается один из двух электронов \(K\)-уровня ( \(n=1\) ), то освободившееся место может быть занято электроном из какого-либо более высокого уровня: \(L, M, N\), и т. д. В результате возникает \(K\)-серия. Подобным же образом возникают и другие серии: \(L, M, \ldots\)

Серия \(K\), как видно из рис. 6.7 , непременно сопровождается появлением и остальных серий, поскольку при испускании ее линий освобождаются электроны на уровнях \(L, M\) и др., которые в свою очередь будут заполняться электронами с более высоких уровней.

Закон Мозли. Мозли (1913) экспериментально установил закон, согласно которому частота \(\omega K_{\alpha}\)-линий зависит от атомного номера \(Z\) элемента как


\begin{equation*}
\omega_{K_{\alpha}}=\frac{3}{4} R(Z-\sigma)^{2}, \tag{6.43}
\end{equation*}


где \(R\) - постоянная Ридберга \(\left(2,07 \cdot 10^{16} \mathrm{c}^{-1}\right), \sigma\) - постоянная, практически равная единице ( \(\sigma \approx 1\) ) для легких элементов. Этот закон сыграл в свое время важную роль при уточнении расположения элементов в периодической системе.

Пример. Вычислим разность \(\Delta E\) энергий связи \(K\) - и \(L\)-электронов ванадия ( \(Z=23\) ).

Достаточно обратиться к рис. 6.7, и мы увидим, что искомая разность энергий связи равна просто энергии перехода между \(L\) - и \(K\)-уровнями, которая связана с частотой \(K_{\alpha}\)-линии, т. е. с законом Мозли. Таким образом,

\[
\Delta E=\hbar \omega_{K_{\alpha}}=\frac{3}{4} \hbar R(Z-1)^{2}=5 \text { кэВ. }
\]

Закон Мозли достаточно точно выполняется для легких элементов. Для тяжелых же элементов поправка σ значительно отличается от единицы; например, для олова ( 0,29 ), цезия \((0,00)\) и вольфрама \((-2,1)\).

С одной стороны ясно, что частоты, фигурирующие в законе Мозли, обусловлены переходами между соответствующими рентгеновскими термами. С другой стороны, попытки представить значение \(\omega\) как разность термов вида \(T=R(Z-\sigma)^{2} / n^{2}\) следует признать неудачными: уж очень сильно отличаются поправки\\
\(\sigma\) для разных термов. В чем здесь дело - это предстоит еще выяснить, но сконструировать закон Мозли (6.43) с помощью такого вида термов не удается.

Особенности спектра поглощения. Коэффициент поглощения μ вещества при прохождении через него рентгеновского излучения вообще возрастает с увеличением длины волны. Однако при некотором значении \(\lambda_{K}\) он резко падает, а затем начинает снова плавно возрастать (рис. 6.8, где показаны и линии испускания).

Такая особенность поглощения веществом рентгеновского излуче-

\begin{figure}[h]
\begin{center}
  \includegraphics[alt={},max width=\textwidth]{a57b3a57-78b5-4292-b0ab-4bd9f2ffcf56-160_301_445_387_713}
\captionsetup{labelformat=empty}
\caption{Рис. 6.8}
\end{center}
\end{figure}

Итак, например, \(K\)-край полосы поглощения ( \(\lambda_{K}\) ) связан с прекращением возбуждения \(K\)-уровня. Это значит, что длина волны \(\lambda_{K}\) характеризует энергию связи \(K\)-электрона ( \(E_{K}\) ):

\[
E_{K}=\hbar \omega_{K}=\frac{2 \pi \hbar c}{\lambda_{K}}=\frac{1,24}{\lambda_{K}(\mathrm{HM})} \text { кэВ. }
\]

Значения длин волн, соответствующих \(K\) - и \(L\)-краям поглощения для разных элементов, определены экспериментально и представлены в виде специальных таблиц.

Тонкая структура рентгеновских спектров. Более детальный анализ характеристических спектров привел к уточнению структуры рентгеновских термов (рис. 6.9). \(K\)-терм остается одиночным. \(L\)-терм оказался тройным, \(M\)-терм - пятикратным.

Поясним причину расщепления.\\
Прежде всего отметим, что мы встречаемся здесь со случаем \(j j\)-связи, которая осуществляется в глубинных слоях тяжелых атомов.

У \(K\)-оболочки \(n=1\), значит каждый электрон имеет \(l=0\), \(s=1 / 2\) и \(j=1 / 2\) (это единственное значение).

У \(L\)-оболочки \(n=2\), каждый электрон имеет \(l=0\) или 1. При \(l=0 j=1 / 2\), а при \(l=1\) согласно (6.30) \(j=1 / 2\) и \(3 / 2\). Итак, мы имеем здесь три подуровня в точном соответствии с кратностью \(L\)-края полосы поглощения. А именно, при \(n=2\) ( \(L\)-оболочка)

\begin{center}
\begin{tabular}{c|c|c}
\(l\) & 0 & 1 \\
\hline
\(j\) & \(1 / 2\) & \(1 / 2,3 / 2\) \\
\hline
\end{tabular}
\end{center}

Аналогично для \(M\)-оболочки и т. д. (см. рис. 6.9).\\
Кроме того, необходимо учесть, что возможны только те переходы между термами, которые подчиняются правилу отбора:


\begin{equation*}
\Delta l= \pm 1, \quad \Delta j=0, \pm 1 . \tag{6.44}
\end{equation*}


Теперь должны быть понятными изображенные на рис. 6.9

\begin{figure}[h]
\begin{center}
  \includegraphics[alt={},max width=\textwidth]{a57b3a57-78b5-4292-b0ab-4bd9f2ffcf56-161_487_459_1290_88}
\captionsetup{labelformat=empty}
\caption{ис. 6.9}
\end{center}
\end{figure}

переходы: только они удовлетворяют этим правилам отбора. Мы видим, что линии \(K\)-серии имеют дублетную структуру. Компоненты дублетов обозначают индексами \(\alpha_{1}, \alpha_{2} ; \beta_{1}, \beta_{2}\) и т. д. Например, \(K_{\alpha}\)-линия представляет собой дублет \(K_{\alpha 1}\) и \(K_{\alpha 2}\).

Серия \(L\) и другие имеют более сложную мультиплетную структуpy.

\section*{Задачи}
6.1. Атом водорода. Электрон атома водорода находится в стационарном состоянии, описываемом волновой функцией \(\psi(r)=A \mathrm{e}^{-\alpha r}\), где \(A\) и \(\alpha\) - некоторые постоянные. Найти энергию \(E\) электрона и постоянную \(\alpha\).

Решение. В данном случае уравнение Шредингера (6.4) будет иметь вид


\begin{equation*}
\frac{\partial^{2} \psi}{\partial r^{2}}+\frac{2}{r} \frac{\partial \psi}{\partial r}+\frac{2 m}{\hbar^{2}}(E-U) \psi=0 . \tag{1}
\end{equation*}


Вычислив первую и вторую производные \(\psi\)-функции по \(r\), подставим их выражения в (1) и сгруппируем следующим образом:


\begin{equation*}
\left(\alpha^{2}+\frac{2 m E}{\hbar^{2}}\right)-\left(\frac{2 m e^{2}}{\hbar^{2}}\right) \frac{1}{r}=0 . \tag{2}
\end{equation*}


Из этого соотношения видно, что равенство его нулю при любых значениях \(r\) возможно лишь в том случае, когда обе скобки по отдельности равны нулю. Отсюда

\[
E=-\frac{\alpha^{2} \hbar^{2}}{2 m}, \quad \alpha=\frac{m e^{2}}{\hbar^{2}} .
\]

6.2. Найти средний электростатический потенциал \(\varphi_{0}\), создаваемый электроном в центре атома водорода, если электрон находится в основном состоянии, описываемом нормированной \(\psi\)-функцией \(\psi(r)=A \exp \left(-r / r_{1}\right)\), где \(A=1 / \sqrt{\pi r_{1}{ }^{3}}, r_{1}\) - первый боровский радиус.

Решение. Заряд электрона в \(1 s\)-состоянии - это заряд сфери-чески-симметричного электронного облака, плотность которого \(\rho=-e \psi^{2}(r)\). Выделим мысленно тонкий сферический слой с радиусами \(r\) и \(r+\mathrm{d} r\). Полный заряд этого слоя \(\mathrm{d} q=\rho \cdot 4 \pi r^{2} \mathrm{~d} r\) создает в центре атома потенциал \(\mathrm{d} \varphi=\mathrm{d} q / r\). Проинтегрировав это выражение по \(r\) от 0 до \(\infty\), найдем:

\[
\varphi_{0}=\int \mathrm{d} \varphi=\int_{0}^{\infty} \frac{\rho}{r} 4 \pi r^{2} \mathrm{~d} r .
\]

Приведя последний интеграл к табличному виду, получим

\[
\varphi_{0}=-\frac{e}{r_{1}} \int_{0}^{\infty} \mathrm{e}^{-x} x \mathrm{~d} x=-\frac{e}{r_{1}}
\]

Этот интеграл берется по частям или прямо из таблиц, он равен единице.\\
6.3. Найти наиболее вероятное расстояние электрона от ядра атома водорода в состоянии \(2 p\).\\
Решение. В этом состоянии электронное облако не является сферически-симметричным. Согласно формуле (6.12)

\[
\psi=R(r) \cdot Y(\theta, \varphi) .
\]

Найдем сначала вероятность местонахождения электрона в элементе объема \(\mathrm{d} V\) вблизи некоторой точки пространства \(r, \theta, \varphi\) :


\begin{equation*}
\mathrm{d} P=\psi^{2} \mathrm{~d} V=A R^{2}(r) Y^{2}(\theta, \varphi) r^{2} \mathrm{~d} r \mathrm{~d} \Omega \tag{1}
\end{equation*}


\begin{figure}[h]
\begin{center}
  \includegraphics[alt={},max width=\textwidth]{a57b3a57-78b5-4292-b0ab-4bd9f2ffcf56-163_248_263_692_136}
\captionsetup{labelformat=empty}
\caption{Рис. 6.10}
\end{center}
\end{figure}

где объем \(\mathrm{d} V=\mathrm{d} S \mathrm{~d} r\) (рис. 6.10). Поскольку \(\mathrm{d} \Omega=\mathrm{d} S / r^{2}\), то \(\mathrm{d} V=r^{2} \mathrm{~d} \Omega \cdot \mathrm{~d} r\).

Теперь найдем вероятность пребывания электрона в сферическом слое с радиусами \(r\) и \(r+\mathrm{d} r\). Для этого надо проинтегрировать (1) по телесному углу \(\mathrm{d} \Omega\) :


\begin{equation*}
\mathrm{d} P=A R^{2}(r) r^{2} \mathrm{~d} r \int Y(\theta, \varphi) \mathrm{d} \Omega . \tag{2}
\end{equation*}


Интеграл в (2) не зависит от \(r\), это некоторое число. Значит, можно записать, что


\begin{equation*}
\mathrm{d} P=B R^{2}(r) r^{2} \mathrm{~d} r \tag{3}
\end{equation*}


и плотность вероятности в расчете на единицу толщины слоя:


\begin{equation*}
\mathrm{d} P / \mathrm{d} r=B r^{2} R^{2}(r), \tag{4}
\end{equation*}


где \(B\) - некоторая постоянная. Теперь введем вместо \(r\) новую переменную \(\rho=r / r_{1}\) и согласно табл. 6.1 перепишем \(R(\rho)\) в явном виде. В результате (4) примет вид


\begin{equation*}
\mathrm{d} P / \mathrm{d} r \sim \rho^{2} R^{2}(\rho)=\rho^{4} \mathrm{e}^{-\rho} . \tag{5}
\end{equation*}


Функция \(f(\rho)=\rho^{4} \mathrm{e}^{-\rho}\) имеет максимум. Найдем значение \(\rho\), при котором он будет наблюдаться. Для этого продифференцируем \(f(\rho)\) по \(\rho\) и полученный результат приравняем к нулю. В результате получим

\[
\rho_{\text {вер }}=4 .
\]

Так как \(\rho\) - это расстояние \(r\) в единицах \(r_{1}\), то найденное значение \(\rho_{\text {вер }}\) соответствует второму боровскому радиусу согласно (2.23).\\
6.4. Атомы щелочных металлов. Найти ридберговскую поправку \(3 P\)-терма атома Na , первый потенциал возбуждения которого \(\varphi_{1}=2,10\) В, а энергия связи валентного электрона в основном \(3 S\)-состоянии \(E_{0}=5,14\) эВ.

Р е ш е н и е. Воспользуемся формулой (6.19), которая содержит интересующую нас поправку. В этой формуле левую часть (энергию \(3 P\)-состояния) можно представить согласно рис. 6.11 как

\[
E_{3 P}=-\left(E_{0}-e \varphi_{1}\right)=-(5,14-2,10) \vartheta \mathrm{B}
\]

(имея в виду, что энергия уровня равна с об-

\begin{figure}[h]
\begin{center}
  \includegraphics[alt={},max width=\textwidth]{a57b3a57-78b5-4292-b0ab-4bd9f2ffcf56-164_321_312_407_850}
\captionsetup{labelformat=empty}
\caption{Рис. 6.11}
\end{center}
\end{figure}

ратным знаком энергии связи на этом уровне). Таким образом, формула (6.19) примет вид

\[
E_{0}-e \varphi_{1}=\frac{\hbar R}{\left(3+\sigma_{p}\right)^{2}}
\]

откуда

\[
\sigma_{p}=\sqrt{\frac{\hbar R}{E_{0}-e \varphi_{1}}}-3=-0,88 .
\]

6.5. Найти энергию связи валентного электрона в основном состоянии атома лития, если известно, что длины волн головной линии резкой серии и ее коротковолновой границы равны соответственно \(\lambda_{1}=813 \mathrm{HM}\) и \(\lambda_{K}=350 \mathrm{HM}\).

Р е ш е н и е. Согласно (6.19) энергия связи электрона в \(2 s\)-состоянии равна


\begin{equation*}
E_{\text {св }}=\left|E_{2 s}\right|=\frac{\hbar R}{\left(2+\sigma_{s}\right)^{2}} \text {. } \tag{1}
\end{equation*}


Задача сводится к нахождению поправки \(\sigma_{s}\). Воспользовавшись рис. 6.3 и формулой (6.19), запишем разность энергий


\begin{equation*}
\hbar \omega_{K}-\hbar \omega_{1}=\frac{\hbar R}{\left(3+\sigma_{s}\right)^{2}} \tag{2}
\end{equation*}


Имея в виду, что \(\omega=2 \pi c / \lambda\), перепишем (2) так:


\begin{equation*}
\sigma_{s}=\sqrt{\frac{R \lambda_{1} \lambda_{K}}{2 \pi c\left(\lambda_{1}-\lambda_{K}\right)}}-3 . \tag{3}
\end{equation*}


Остается подставить (3) в (1), и мы получим:

\[
E_{\text {жв }}=\frac{\hbar R}{\left(\sqrt{R \lambda_{1} \lambda_{K} / 2 \pi c \Delta \lambda}-1\right)^{2}}=5,3 \text { эВ, }
\]

где \(\Delta \lambda=\lambda_{1}-\lambda_{K}\).\\
6.6. Момент импульса. Найти максимально возможный полный механический момент и соответствующий спектральный символ терма атома в состоянии с электронной конфигурацией \(1 s^{2} 2 p 3 d\).

Решение. Максимальный момент будет складываться из максимальных орбитального и спинового моментов. Это относится и к соответствующим квантовым числам:

\[
L_{\text {макс }}=1+2=3, \quad S_{\text {макс }}=1 / 2+1 / 2=1 .
\]

В результате получим:

\[
J_{\text {макс }}=3+1=4, \quad M_{\text {макс }}=\hbar \sqrt{20}, \quad{ }^{3} F_{4} .
\]

6.7. Определить спектральный символ терма атома, мультиплетность \(v\) которого равна пяти, кратность вырождения по квантовому числу \(J\) - семи и значение орбитального квантового числа равно максимально возможному в этих условиях.

Р еш е н и е. Из мультиплетности \(v=2 S+1\) находим \(S=2\), а из кратности вырождения \((2 J+1)\) имеем \(J=3\). Далее, мы знаем, что вообще говоря, если известны \(L\) и \(S\), то квантовое число \(J\) в формуле (6.38) может принимать значения (через единицу) от \(L+S\) до \(|L-\mathrm{S}|\). Отсюда видно, что значениям \(J=3, S=2\) и требованию, чтобы \(L\) было максимальным, отвечает условие

\[
J=L-S, \quad \text { откуда } L=J+S=5 .
\]

Спектральный символ этого состояния \({ }^{5} H_{3}\).\\
6.8. Написать спектральный символ терма, кратность вырождения которого по \(J\) равна семи, и квантовые числа \(L\) и \(S\) связаны соотношением \(L=3 S\).

Р е ш е н и е. Из условия, что кратность вырождения по \(J\), т. е. \(2 J+1=7\), находим \(J=3\). Отсюда следует, что \(S\) может быть только целым числом: \(1,2, \ldots\) Соответственно \(L\) равно 3,6 , …

При \(S=1\) и \(L=3\) одно из значений \(J\) будет равно 3 . Если же взять другую пару: \(S=2\) и \(L=6\), то значение \(J=3\) из них получить невозможно. То же и при больших значениях \(S\).

Остается \(S=1, L=3, J=3\). Соответствующий спектральный символ \({ }^{3} F_{3}\).\\
6.9. Правила Хунда. Найти кратность вырождения основного терма атома, электронная конфигурация единственной незаполненной подоболочки которого \(d^{6}\).

Решение. Символ \(d\) соответствует \(l=2\). Составим табличку распределения электронов по значениям квантового числа \(m_{l}\), стараясь в соответствии с правилами Хунда, чтобы суммарное \(S\) было максимальным и чтобы при этом и \(L\) было максимальным (точнее следует говорить сначала о максимальных значениях \(m_{S}\) и \(m_{L}\) ). Из приведенной таблички

\begin{center}
\begin{tabular}{c|c|c|c|c|c}
\(m_{l}\) & +2 & +1 & 0 & -1 & -2 \\
\hline
\(m_{s}\) & \(\uparrow \downarrow\) & \(\uparrow\) & \(\uparrow\) & \(\uparrow\) & \(\uparrow\) \\
\hline
\end{tabular}
\end{center}

видно, что максимальная сумма \(m_{S}=2\), значит и \(S=2\). Кроме того, максимальное значение \(m_{L}=2\), значит и \(L=2\). Так как подоболочка заполнена более, чем наполовину, то по второму правилу Хунда \(J=L+S=4\).

Итак, основной терм этой конфигурации \({ }^{5} D_{4}\) и его кратность вырождения (число различных \(m_{J}\) ) определяется как \(2 J+1\), т. е. девять.\\
6.10. Рентгеновские спектры. Найти порядковый номер \(Z\) легкого элемента, у которого в спектре поглощения рентгеновского излучения разность частот \(K\) - и \(L\)-краев поглощения равна \(\Delta \omega=6,85 \cdot 10^{18} \mathrm{c}^{-1}\). Решени е. По существу \(\hbar \Delta \omega\) - это разность энергий связи электрона на \(K\) - и \(L\)-уровнях, частота перехода между которыми (см. рис. 6.7) определяется законом Мозли (6.43). Таким образом, из равенства

\[
\frac{3}{4} R(Z-1)^{2}=\Delta \omega
\]

найдем:

\[
Z=1+\sqrt{4 \Delta \omega / 3 R}=22, \text { т. е. титан. }
\]

6.11. Найти энергию связи \(K\)-электрона ванадия ( \(Z=23\) ), для которого длина волны \(L\)-края полосы поглощения равна \(\lambda_{L}\).\\
Решение. С помощью схемы на рис. 6.7 можно записать, что искомая энергия связи

\[
E_{K}=\hbar \omega_{L}+\hbar \omega_{K \alpha}
\]

где \(\omega_{L}=2 \pi c / \lambda_{L}\) и \(\omega_{K \alpha}\) - частота, определяемая законом Мозли (6.43). В результате

\[
E_{K}=\hbar\left[\frac{2 \pi c}{\lambda_{L}}+\frac{3}{4} R(Z-1)^{2}\right]
\]

\section*{Магнитные свойства атома}
\section*{§ 7.1. Магнитный момент атома}
Орбитальный магнитный момент. Ранее неоднократно отмечалось, что с механическим моментом \(M\) атома связан магнитный момент \(\mu\). В § 2.3 было получено классическое выражение (2.33) для связи \(\mu\) с \(M\), обусловленной орбитальным движением электрона в атоме водорода. В квантовой теории величины \(\mu\) и \(M\) следует заменить операторами \(\hat{\mu}\) и \(\hat{M}\) :


\begin{equation*}
\hat{\mu}=-\frac{e}{2 m c} \hat{M}, \quad \hat{\mu}_{z}=-\frac{e}{2 m c} \hat{M}_{z} . \tag{7.1}
\end{equation*}


Отсюда следует, что изучение свойств магнитного момента электрона сводится к изучению свойств операторов \(\hat{\mu}\) и \(\hat{\mu}_{z}\). А так как операторы \(\hat{\mu}\) и \(\hat{M}, \hat{\mu}_{z}\) и \(\hat{M}_{z}\) отличаются друг от друга только постоянным множителем, то их свойства совершенно аналогичны: магнитный и механический моменты квантуются по одинаковым правилам.

В стационарном состоянии определенные значения могут иметь только модуль магнитного момента \(\mu_{L}\) и одна из его проекций на произвольную ось \(Z\). Имея в виду (7.1), а также (6.34) и (6.36), запишем собственные значения операторов \(\hat{\mu}\) и \(\hat{\mu}_{z}\) :


\begin{align*}
& \mu_{L}=-\mu_{\mathrm{B}} \sqrt{L(L+1)}, \quad L=0,1,2, \ldots  \tag{7.2}\\
& \mu_{L z}=-\mu_{\mathrm{B}} m_{L}, \quad m_{L}=0, \pm 1, \pm 2, \ldots, \pm L, \tag{7.3}
\end{align*}


где \(\mu_{\mathrm{C}}\) - магнетон Бора (2.36): \(\mu_{\mathrm{E}}=e \hbar / 2 m c\). Он играет роль кванта магнитного момента (точнее его проекции \(\mu_{z}\) ).

В заключение отметим, что

\begin{enumerate}
  \item отношение магнитного момента к механическому, т. е.
\end{enumerate}


\begin{equation*}
\mu / M=e / 2 m c, \tag{7.4}
\end{equation*}


называют гиромагнитным отношением;\\
2) знак минус в вышеприведенных формулах указывает на то, что «векторы» \(\boldsymbol{\mu}\) и \(\mathbf{M}\) взаимно противоположны по направлению (в классическом смысле понятия «векторов»).

Опыты Штерна и Герлаха. Наличие у атомов магнитных моментов и их квантование было доказано экспериментально Штерном и Герлахом (1921). В их опытах пучок атомов пропускался сквозь сильно неоднородное поперечное магнитное поле (рис. 7.1, \(a\) ). Необходимая степень неоднородности поля достигалась с помощью специальной формы полюсных наконечников \(N\) и \(S\) электромагнита (рис. 7.1, б). После прохождения магнитного поля пучок атомов попадал на фотопластинку \(P\) и оставлял на ней след.

\begin{figure}[h]
\begin{center}
  \includegraphics[alt={},max width=\textwidth]{a57b3a57-78b5-4292-b0ab-4bd9f2ffcf56-169_342_1040_536_98}
\captionsetup{labelformat=empty}
\caption{Рис. 7.1}
\end{center}
\end{figure}

Если атомы обладают магнитным моментом, то согласно электродинамике на них будет действовать сила, проекция которой на ось \(Z\) (см. рис. 7.1, б)


\begin{equation*}
F_{z}=\mu_{z} \frac{\partial B_{z}}{\partial z}, \tag{7.5}
\end{equation*}


где \(\mu_{z}\) - проекция магнитного момента атома на ось \(Z\). Из этой формулы видно, что для получения необходимого эффекта при малых значениях \(\mu_{z}\) нужно обеспечить достаточно большую неоднородность поля, т. е. \(\partial B_{z} / \partial z\). Это и достигалось с помощью указанной формы полюсных наконечников.

В отсутствие магнитного поля след пучка на фотопластинке \(P\) имел вид одной полоски ( \(z=0\) ). При включении же магнитного поля наблюдалось расщепление пучка (рис. 7.1, в), что являлось следствием квантования проекции магнитного момента \(\mu_{z}\) в формуле (7.5): \(\mu_{z}\) может принимать только ряд дискретных значений. В опытах обнаружилось также, что для разных атомов число компонент, на которые расщеплялся пучок, было или нечетным, или четным.

Анализ полученных результатов показал, что нечетное число компонент возникает у атомов, обладающих только орбитальным механическим моментом \(M_{L}\), тогда магнитное поле снимает вырождение по \(L\) и число компонент (значений \(m_{L}\) ) будет равно \(2 L+1\), т. е. нечетным.

Если же момент атома является суммой орбитального и спинового, т. е. определяется квантовым числом \(J\), то число компонент будет равно \(2 J+1\), и в зависимости от того, полуцелым или целым будет значение \(J\), число компонент будет соответственно четным или нечетным.

В частности, при пропускании атомов водорода или серебра пучок расщеплялся на две компоненты, что в свое время явилось полной неожиданностью, поскольку в основном состоянии их орбитальные моменты равны нулю (а спиновые моменты были еще неизвестны), и пучок не должен был расщепляться.

Но вскоре объяснение было найдено: эти атомы обладают спиновым моментом ( \(s=1 / 2\) ), и число \(2 s+1\) компонент \(m_{s}\) в полном соответствии с опытом равно двум.

Спиновый магнитный момент. Зная степень неоднородности магнитного поля, т. е. \(\partial B_{z} / \partial z\), Штерн и Герлах по величине расщепления пучка на фотопластинке рассчитали значение проекции спинового магнитного момента на направление магнитного поля, \(\mu_{B}\). Выяснилось, что \(\mu_{B}\) равен одному магнетону Бора. Этот результат сначала также оказался неожиданным, поскольку приводит к гиромагнитному отношению вдвое превышающему (7.4), связывающему орбитальные моменты. В связи с этим говорят, что спин обладает удвоенным магнетизмом.

Итак, спиновый магнитный момент и его проекция на произвольную ось \(Z\) определяются как


\begin{align*}
& \mu_{S}=-2 \mu_{\mathrm{b}} \sqrt{S(S+1)}  \tag{7.6}\\
& \mu_{S z}=-2 \mu_{\mathrm{b}} m_{S}, \quad m_{S}=S, S-1, \ldots,-S . \tag{7.7}
\end{align*}


При \(S=1 / 2 \quad m_{S}=+1 / 2\) и \(-1 / 2\).\\
Принято говорить, что спиновый магнитный момент электрона равен одному магнетону Бора. Такая терминология обусловлена тем, что при измерении магнитного момента мы обычно измеряем его проекцию, а она как раз и равна одному \(\mu_{\mathrm{b}}\).

Опыты Штерна и Герлаха явились еще одним убедительным доказательством наличия у электрона спина*.

Полный магнитный момент атома. Вследствие удвоенного магнетизма спина гиромагнитное отношение полных моментов \(\mu / M_{J}\) оказывается значительно более сложным. Оно зависит от квантовых чисел \(L, S\) и \(J\). Соответствующий расчет, проводимый в квантовой теории, позволил найти магнитный момент \(\mu\) и его проекцию на ось \(Z\) :


\begin{align*}
& \mu=-\mu_{\mathrm{B}} g \sqrt{J(J+1)},  \tag{7.8}\\
& \mu_{z}=-\mu_{\mathrm{B}} g m_{J}, \quad m_{J}=J, J-1, \ldots,-J, \tag{7.9}
\end{align*}


где \(g\) - множитель (или фактор) Ланде:


\begin{equation*}
g=\frac{3}{2}+\frac{S(S+1)-L(L+1)}{2 J(J+1)} \tag{7.10}
\end{equation*}


В частности, в синглетных состояниях ( \(S=0\) ) \(J=L, g=1\), и мы приходим к формулам (7.2) и (7.3). А при \(L=0(J=S, g=2)\) к формулам (7.6) и (7.7).

Отметим также некоторые «экзотические» случаи. Например:

\begin{enumerate}
  \item в состоянии \({ }^{3} P_{0} \quad g=0 / 0\); эта неопределенность не должна смущать, поскольку при \(J=0\) механический момент равен нулю, а значит, отсутствует и магнитный момент;
  \item в состоянии \({ }^{4} D_{1 / 2} \quad g=0\), т. е. механический момент есть, а магнитный отсутствует;
  \item в состоянии \({ }^{6} F_{1 / 2} \quad g=-2 / 3\), а это значит, что в данном состоянии знак минус в формулах (7.8) и (7.9) исчезает. На языке классики это означает, что «векторы» \(\boldsymbol{\mu}\) и \(\mathbf{M}\) «сонаправлены» (не взаимно противоположны).
\end{enumerate}

\footnotetext{\begin{itemize}
  \item Помимо этих опытов следует упомянуть и о так называемых магнитомеханических явления \(x\) - опытах Эйнштейна и де Хааса, а также опыте Барнетта. И в этих опытах было обнаружено, что гиромагнитное отношение спиновых моментов тоже вдвое больше отношения орбитальных.
\end{itemize}
}
4) в состоянии \({ }^{5} P_{1} g=5 / 2\), т.е. фактор Ланде в некоторых состояниях может быть и больше двух (вопреки утверждению некоторых авторов).\\
Случаи 2) и 3), когда \(g=0\) и \(g<0\), представляют собой чисто квантовые эффекты, не имеющие аналогов в классической физике.

\section*{§ 7.2. Эффекты Зеемана и Пашена-Бака}
Эффект Зеемана. При помещении источника в магнитное поле его спектральные линии испытывают расщепление. Это и есть эффект Зеемана (1896).

Расщепление линий связано с расщеплением самих энергетических уровней, поскольку атом, обладающий магнитным моментом, приобретает в магнитном поле дополнительную энергию


\begin{equation*}
\Delta E=-\mu_{B} B, \tag{7.11}
\end{equation*}


где \(\mu_{B}\) - проекция полного магнитного момента атома на направление поля В. Имея в виду формулу (7.9), запишем выражение для энергии каждого подуровня:


\begin{equation*}
E=E_{0}+\Delta E=E_{0}+\mu_{\mathrm{L}} g B m_{J}, \quad m_{J}=J, J-1, \ldots,-J, \tag{7.12}
\end{equation*}


где \(E_{0}\) - энергия уровня в отсутствие магнитного поля.\\
Отсюда следует, что уровни с квантовым числом \(J\) расщепляются в магнитном поле на \(2 J+1\) равноотстоящих друг от друга подуровней, причем величина расщепления зависит от множителя Ланде \(g\), т. е. интервалы \(\delta E\) между соседними подуровнями пропорциональны \(g: \delta E \sim g\). Таким образом, магнитное поле в результате расщепления уровней снимает вырождение по \(m_{J}\).

Кроме этого, необходимо учесть, что возможны только такие переходы между подуровнями, принадлежащими разным уровням, при которых выполняются следующие правила отбора для квантового числа \(m_{J}\) :


\begin{equation*}
\Delta m_{J}=0, \pm 1 . \tag{7.13}
\end{equation*}


Формулы (7.12) и (7.13) составляют основу для понимания эффекта Зеемана.

Отметим попутно, что компоненты, сответствующие \(\Delta m_{J}=0\), называют \(\pi\)-компонентами, а \(\Delta m_{J}= \pm 1-\sigma\)-компонентами. При наблюдении перпендикулярно магнитному полю присутствуют и \(\pi\) - и \(\sigma\)-компоненты. При наблюдении же вдоль магнитного поля \(\pi\)-компоненты исчезают, остаются только σ-компоненты.

Частоты \(\omega\) зеемановских компонент спектральной линии с частотой \(\omega_{0}\) определяются формулой

\[
\omega=\frac{E_{2}+\Delta E_{2}}{\hbar}-\frac{E_{1}+\Delta E_{1}}{\hbar}=\frac{E_{2}-E_{1}}{\hbar}+\frac{\Delta E_{2}-\Delta E_{1}}{\hbar}=\omega_{0}+\Delta \omega_{0} .
\]

Согласно (7.12), \(\Delta \omega\) - зеемановское смещение (относительно несмещенной линии):


\begin{equation*}
\Delta \omega=\left(m_{2} g_{2}-m_{1} g_{1}\right) \delta \omega_{0} \tag{7.14}
\end{equation*}


где величина \(\delta \omega_{0}=\mu_{\mathrm{b}} B / \hbar\), ее называют лоренцевым смещением.\\
Простой эффект Зеемана. Так называют эффект, в котором спектральная линия расщепляется на \(m p u\) компоненты (при наблюдении перпендикулярно магнитному полю). Простой эффект присущ спектральньлм линиям, не имеющим тонкой структуры. Эти линии возникают при переходах между синглетными уровнями ( \(S=0, J=L, m_{J}=m_{L}, g=1\) ). Поэтому формула (7.14) принимает вид


\begin{equation*}
\Delta \omega=\Delta m_{L} \cdot \delta \omega_{0} \tag{7.15}
\end{equation*}


где \(\Delta m_{L}=0, \pm 1\), т. е. возникают, действительно, три компоненты, зеемановское смещение которых


\begin{equation*}
\Delta \omega=\delta \omega_{0}, \quad 0, \quad-\delta \omega_{0} . \tag{7.16}
\end{equation*}


На рис. 7.2 показано расщепление уровней для перехода \({ }^{1} P \rightarrow{ }^{1} S\).

В отсутствие поля (слева) наблюдается одна линия частоты \(\omega_{0}\). При включении поля возникают три эеемановские компоненты в соответствии с (7.16).

Более сложный случай показан на рис. 7.3 для перехода \({ }^{1} D \rightarrow{ }^{1} P\). Однако и здесь, если внимательно следить за переходами с помощью правил отбора (7.13), возникают тоже только

\begin{figure}[h]
\begin{center}
  \includegraphics[alt={},max width=\textwidth]{a57b3a57-78b5-4292-b0ab-4bd9f2ffcf56-174_540_447_139_92}
\captionsetup{labelformat=empty}
\caption{Рис. 7.2}
\end{center}
\end{figure}

\begin{figure}[h]
\begin{center}
  \includegraphics[alt={},max width=\textwidth]{a57b3a57-78b5-4292-b0ab-4bd9f2ffcf56-174_544_504_141_639}
\captionsetup{labelformat=empty}
\caption{Рис. 7.3}
\end{center}
\end{figure}

три зеемановские компоненты (7.16). Соответствующие им переходы показаны на этом рисунке справа.

Пример. Оценим в длинах волн расщепление \(\delta \lambda\) спектральной линии \(\lambda=550\) нм в случае простого эффекта Зеемана в магнитном поле с индукцией \(B=10^{4}\) Гс (1 Тл).

Так как \(\lambda=2 \pi c / \omega\), то

\[
\delta \lambda=\frac{2 \pi c}{\omega^{2}} \delta \omega_{0} .
\]

Полагая, что \(\delta \omega_{0}\) - лоренцево смещение, равное согласно (7.14) \(\mu_{\mathrm{b}} B / \hbar\), получим:

\[
\begin{aligned}
\delta \lambda=\lambda^{2} \frac{\mu_{\mathrm{b}} B}{2 \pi c \hbar} & =\left(5,5 \cdot 10^{-5}\right)^{2} \frac{0,927 \cdot 10^{-20} \cdot 10^{4}}{2 \pi \cdot 3 \cdot 10^{10} \cdot 1,054 \cdot 10^{-27}}= \\
& =1,4 \cdot 10^{-9} \mathrm{~cm}=0,014 \mathrm{HM} .
\end{aligned}
\]

Как видим, расщепление весьма мало даже при значительной для лабораторных условий индукции магнитного ноля. Поэтому для обнаружения такого расщепления используют приборы с высокой разрешающей способностью, типа интерферометров Фабри-Перо.

О поляризации зеемановских компонент. В заключение отметим, что \(\pi\) - и σ-компоненты оказываются поляризованными. При наблюдении перпендикулярно магнитному полю \(\mathbf{B}\), как показано на рис. \(7.4, a\), все три компоненты поляризованы линейно: у \(\pi\)-компоненты колебания \(\mathbf{E}\)-вектора направлены вдоль \(\mathbf{B}\), а у \(\sigma\)-компонент - перпендикулярно \(\mathbf{B}\).

\begin{figure}[h]
\begin{center}
  \includegraphics[alt={},max width=\textwidth]{a57b3a57-78b5-4292-b0ab-4bd9f2ffcf56-175_403_293_141_58}
\captionsetup{labelformat=empty}
\caption{Рис. 7.4}
\end{center}
\end{figure}

При наблюдении же вдоль магнитного поля В (см. рис. 7.4, б) π-компонента отсутствует (исчезает), а \(\sigma\)-компоненты поляризованы по кругу в противоположных относительно друг друга направлениях.

Сложный эффект Зеемана. Так называют эффект, когда спектральная линия от источника, находящегося в магнитном поле, расщепляется на число компонент более трех*. Это связано с зависимостью расщепления самих уровней от множителя Ланде \(g\), как видно из (7.12), т. е. в конечном счете с наличием cnuна электрона и его удвоенным магнетизмом.\\
При объяснении сложного эффекта Зеемана будем исходить из предположения, что имеет место нормальная связь \(L-S\) (связь Рессель-Саундерса, см. § 6.4). Это подтверждается экспериментально.

Более подробный характер расщепления уровней (естественное и зеемановское) и возможные переходы между ними показаны на рис. 7.5. Слева на этом рисунке показано естественное

\begin{figure}[h]
\begin{center}
  \includegraphics[alt={},max width=\textwidth]{a57b3a57-78b5-4292-b0ab-4bd9f2ffcf56-175_701_901_962_164}
\captionsetup{labelformat=empty}
\caption{Рис. 7.5}
\end{center}
\end{figure}

\footnotetext{\begin{itemize}
  \item Число зеемановских компонент при сложном эффекте может достигать нескольких десятков.
\end{itemize}
}расщепление (тонкая структура, компоненты \(\lambda_{1}\) и \(\lambda_{2}\) ) в отсутствие магнитного поля. Справа - зеемановское расщепление в магнитном поле и возможные по правилу отбора (7.13) переходы. Заметим, что при наличии магнитного поля первоначальная линия в данном случае отсутствует. Вместо линии \({ }^{2} P_{1 / 2} \rightarrow{ }^{2} S_{1 / 2}\) появляются четыре зеемановских компоненты, смещения которых \(\Delta \omega=( \pm 2 / 3, \pm 4 / 3) \delta \omega_{0}\). Вместо же линии \({ }^{2} P_{3 / 2} \rightarrow{ }^{2} S_{1 / 2}\) появляются шесть зеемановских компонент, смещения которых \(\Delta \omega=( \pm 1 / 3, \pm 3 / 3, \pm 5 / 3) \delta \omega_{0}\).

Сложный эффект Зеемана наблюдается в слабом магнитном ноле, когда зеемановское расщепление спектральных линий мало по сравнению с интервалом между компонентами тонкой структуры* (т. е. по сравнению с разностью \(\lambda_{1}-\lambda_{2}\) на рис. 7.5).

Какой эффект Зеемана в слабом магнитном поле (простой или сложный) будет испытывать данная спектральная линия - сразу ответить на этот вопрос не всегда возможно.

Пример. Отнесем этот вопрос к линиям, обусловленным переходами:\\
a) \({ }^{2} D_{5 / 2} \rightarrow{ }^{2} P_{3 / 2}\);\\
б) \({ }^{5} I_{5} \rightarrow{ }^{5} H_{4}\).

Прежде всего необходимо проверить, равны или нет множители Ланде в состояниях, между которыми происходят переходы. Можно убедиться с помощью (7.10), что в случае\\
a) \(g_{1} \neq g_{2}\), поэтому эффект Зеемана сложный;\\
б) \(g_{1}=g_{2}\), значит - простой.

Эффект Пашена-Бака. В сильном магнитном поле (другой крайний случай) связь между моментами \(M_{L}\) и \(M_{S}\) разрывается, и они ведут себя по отношению к магнитному полю независимо друг от друга. В этом случае дополнительная энергия, связанная с их магнитными моментами, определяется как


\begin{equation*}
\Delta E=\mu_{\mathrm{B}} B m_{L}+2 \mu_{\mathrm{B}} B m_{S}=\mu_{\mathrm{B}} B\left(m_{L}+2 m_{S}\right) . \tag{7.17}
\end{equation*}


Дозволенные переходы соответствуют правилам отбора


\begin{equation*}
\Delta m_{L}=0, \pm 1, \quad \Delta m_{S}=0 \tag{7.18}
\end{equation*}


\footnotetext{\begin{itemize}
  \item Заметим в связи с этим, что для одиночных линий (синглетов) указанное условие никогда не может выполняться. Для таких линий всякое магнитное поле является сильным, и наблюдаемый на них эффект всегда простой.
\end{itemize}
}В результате возникает нормальный зеемановский триплет, схематически показанный на рис. 7.6.

\begin{figure}[h]
\begin{center}
  \includegraphics[alt={},max width=\textwidth]{a57b3a57-78b5-4292-b0ab-4bd9f2ffcf56-177_600_981_252_123}
\captionsetup{labelformat=empty}
\caption{Рис. 7.6}
\end{center}
\end{figure}

Если в сильном магнитном поле магнитное расщепление линий оказывается больше тонкого расщепления, то это значит, что мы наблюдаем эффект Пашена-Бака.

Таким образом, увеличивая индукцию \(B\) магнитного поля, мы будем наблюдать сначала тонкое расщепление линий (при \(B \approx 0\) ), затем сложный эффект Зеемана (множество компонент) и наконец при сильном поле - простой эффект (триплет).

Мы рассмотрели крайние случаи. Наиболее сложной оказывается картина расщепления спектральных линий в промежуточных случаях (полях).

\section*{§ 7.3. Электронный парамагнитный резонанс (ЭПР)}
Выше было установлено, что у парамагнитного атома в магнитном поле каждый уровень с квантовым числом \(J\) расщепляется на \(2 J+1\) подуровней (число возможных \(m_{J}\) ). Пря этом интервал между соседними подуровнями, как следует из (7.12), равен \(\delta E=\mu_{\mathrm{b}} g B\), ибо \(\left|\delta m_{J}\right|=1\).

Если на атом, находящийся в таком состоянии, направить электромагнитную волну с частотой \(\omega\), удовлетворяющей условию


\begin{equation*}
\hbar \omega=\delta E=\mu_{\mathrm{B}} g B, \tag{7.19}
\end{equation*}


то под действием магнитной составляющей падающей волны в согласии с правилом отбора (7.13) будут происходить переходы атома между соседними подуровнями, принадлежащими одному и тому же уровню. Это явление, связанное с вынужденными переходами, впервые наблюдал Е. К. Завойский (1944). Оно и получило название электронного парамагнитного резонанса* (ЭПР) в связи с тем, что имеет резонансный характер: переходы возникают при строго определенной частоте падающей волны.

Оценим с помощью (7.19) резонансную частоту \(v\) при типичной для лабораторных условий индукции магнитного поля \(B=10^{4}\) Гс (1 Тл). Полагая \(g \sim 1\), получим:

\[
v=\frac{\omega}{2 \pi}=\frac{\mu \cdot B}{2 \pi \hbar} \approx 10^{10} \text { Гц, }
\]

что соответствует длине волны порядка нескольких сантиметров. Значит, резонансные частоты находятся в радиодиапазоне. Их называют СВЧ.

Необходимо заметить, что при резонансе энергия передается не только от поля к атому, но и в обратном направлении - при переходе атома с более высоких зеемановских подуровней на более низкие. Однако при тепловом равновесии число атомов с меньшей энергией превышает число атомов с большей энергией. Поэтому переходы, увеличивающие энергию атомов, преобладают над переходами в обратном направлении. Так что в результате парамагнетик поглощает энергию радиочастотного поля и нагревается.

Эксперименты с ЭПР дают возможность из условия резонанса (7.19) определить одну из величин \(-g, B, \omega_{\text {рез }}\) - по известным остальным величинам. Например, измерив с высокой точностью индукцию поля \(B\) и \(\omega_{\text {рез }}\), с помощью ЭПР можно найти

\footnotetext{\begin{itemize}
  \item Отметим, что сначала наблюдался ядерный магнитный резонанс, Раби (1938).
\end{itemize}
}значение \(g\), а затем и магнитный момент атома в состоянии с квантовым числом \(J\).

В жидкостях и кристаллах атомы не являются изолированными, так как взаимодействуют с другими атомами. Это взаимодействие приводит к тому, что интервалы между соседними подуровнями зеемановского расщепления оказываются различными, и линии ЭПР имеют конечную ширину.

Для исследований ЭПР применяют приборы, называемые paдиоспектроскопами. В них частота ω поддерживается постоянной, а изменяется в широких пределах индукция магнитного поля \(B\), создаваемого электромагнитом \(N S\) (рис. 7.7).

\begin{figure}[h]
\begin{center}
  \includegraphics[alt={},max width=\textwidth]{a57b3a57-78b5-4292-b0ab-4bd9f2ffcf56-179_170_415_647_68}
\captionsetup{labelformat=empty}
\caption{Рис. 7.7}
\end{center}
\end{figure}

Небольшой образец \(A\) помещают в объемный резонатор \(R\), настроенный на длину волны \(\lambda \sim 3 \mathrm{~cm}\). Радиоволны такой длины, создаваемые генератором \(\Gamma\), подводятся к резонатору \(R\) через волновод \(V\) (трубку с проводящими стенками). После частичного поглощения в образце \(A\) они поступают тоже через волновод \(V\) на детектор \(D\).

В ходе эксперимента плавно изменяется магнитное поле, создаваемое электромагнитом. При значении индукции \(B\), удовлетворяющем условию (7.19), наблюдается интенсивное (резонансное) поглощение волны образцом.

Следует отметить, что ЭПР - это один из самых простых, но не единственный метод радиоспектроскопии, которая позволяет раскрывать многие тонкие особенности строения вещества. Для радиоспектроскопических методов характерна весьма высокая разрешающая способность, в сотни тысяч раз превышающая разрешающую способность оптических методов.

\section*{Задачи}
7.1. Магнитный момент атома. Вычислить модуль магнитного момента атома в состоянии с квантовыми числами \(S=1, L=2\) и фактором Ланде \(g=4 / 3\).\\
Решение. Магнитный момент атома определяется формулой (7.8). Чтобы его вычислить, надо знать \(J\). Воспользуемся выраже-

нием (7.10). После подстановки в него данных из условия задачи получим:

\[
J^{2}+J-12=0
\]

откуда \(J=3\). Модуль искомого магнитного момента

\[
\mu=(8 / \sqrt{3}) \mu_{\mathrm{E}} .
\]

7.2. Максимальное значение проекции магнитного момента атома в состоянии \(D_{2}\) равно четырем магнетонам Бора. Определить мультиплетность \(v\) этого состояния.

Решение. Воспользуемся формулой (7.9). Из условия

\[
\mu_{z \text { макс }}=g J \mu_{\mathrm{E}}=4 \mu_{\mathrm{B}}
\]

находим \(g=2\). Зная, что \(L=2\) и \(J=2\), определим с помощью (7.10) квантовое число \(S\) :

\[
S^{2}+S-12=0, \quad S=3
\]

Отсюда мультиплетность \(v=2 S+1=7\).\\
7.3. Написать спектральный символ терма атома, у которого \(S=2\), полный момент \(M_{J}=\hbar \sqrt{2}\), а магнитный момент равен нулю.

Решение. При наличии механического момента магнитный момент может быть равен нулю только потому, что множитель Ланде \(g=0\). Распишем это условие, учитывая, что из выражения для \(M_{J}\) подкоренное число \(2=J(J+1)\), откуда \(J=1\). Итак, из условия, что \(g(7.10)\) равно нулю, приходим к уравнению

\[
L^{2}+L-12=0
\]

откуда \(L=3\). Соответствующий спектральный символ \({ }^{5} F_{1}\).\\
7.4. Найти с помощью правил Хунда магнитный момент основного состояния атома, единственная незаполненная подоболочка которого заполнена ровно наполовину пятью электронами.

Решение. Всего в подоболочке имеется \(2(2 l+1)\) состояний. Из условия \(2 l+1=5\) находим \(l=2\), значит и \(m_{l \text { макс }}=2\). Это \(d\)-подоболочка. Составим для нее табличку заполнения:

\begin{center}
\begin{tabular}{c|c|c|c|c|c}
\(m_{l}\) & +2 & +1 & 0 & -1 & -2 \\
\hline
\(m_{s}\) & \(\uparrow\) & \(\uparrow\) & \(\uparrow\) & \(\uparrow\) & \(\uparrow\) \\
\hline
\end{tabular}
\end{center}

Электроны расположены именно так, чтобы по первому правилу Хунда суммарный спин был максимален. Итак, \(S=5 / 2, L=0\). По второму правилу Хунда \(J=L+S=5 / 2\). Основной терм \({ }^{6} S_{5 / 2}\). Множитель Ланде для этого состояния \(g=2\), и магнитный момент

\[
\mu=\mu_{\mathrm{B}} g \sqrt{J(J+1)}=\sqrt{35} \mu_{\mathrm{B}} .
\]

7.5. Опыт Штерна и Герлаха. Узкий пучок атомов ванадия в основном состоянии \({ }^{4} F_{3 / 2}\) проходит через поперечное резко неоднородное

\begin{figure}[h]
\begin{center}
  \includegraphics[alt={},max width=\textwidth]{a57b3a57-78b5-4292-b0ab-4bd9f2ffcf56-181_198_458_524_135}
\captionsetup{labelformat=empty}
\caption{Рис. 7.8}
\end{center}
\end{figure}

магнитное поле и попадает на экран Э (рис. 7.8). Найти расстояние \(\Delta z\) между крайними компонентами расщепленного пучка на экране, если известны расстояния \(l\), градиент магнитного поля \(\partial B / \partial z\) и кинетическая энергия \(K\) атомов.

P e ш e н и е. Смещение на экране определяется формулой


\begin{equation*}
\delta z=\frac{a_{z} t_{1}^{2}}{2}+v_{z} t_{2}, \tag{1}
\end{equation*}


где \(t_{1}\) и \(t_{2}\) - времена движения атома в магнитном поле и между магнитом и экраном. В нашем случае \(t_{1}=t_{2}=t\). Кроме того, \(v_{z}=a_{z} t\), поэтому (1) перепишем так:


\begin{equation*}
\delta z=\frac{3}{2} a_{z} t^{2} \tag{2}
\end{equation*}


Теперь учтем, что

\[
a_{z}=\frac{F}{m}=\frac{\mu_{z}}{m} \frac{\partial B}{\partial z}, \quad \mu_{z}=g J \mu_{\mathrm{E}},
\]

где множитель \(g\) согласно (7.10) равен \(2 / 5\). Время \(t=l / v, v\) - скорость атомов, \(v=\sqrt{2 K / m}\). После подстановки этих выражений в (2) получим

\[
\Delta z=2 \delta z=\frac{3}{2} g J \mu_{\mathrm{B}} \frac{\partial B}{\partial z} \frac{l^{2}}{K} .
\]

7.6. Эффект Зеемана. На сколько подуровней расщепятся в слабом магнитном поле термы:\\
а) \({ }^{2} F_{5 / 2}\) и б) \({ }^{4} D_{1 / 2}\) ?

Решение. Это зависит от числа различных \(m_{J}\) в формуле (7.9), а оно равно, как мы знаем, \(2 J+1\). Но это не всегда так. Ведь проекция магнитного момента (7.9) зависит не только от \(m_{J}\), но и от фактора \(g\). А вдруг \(g=0\) ? Проверим.

В случае а) \(g=6 / 7\), поэтому данный терм расщепится на \(2 J+1=6\) подуровней.

В случае же б) \(g=0\), поэтому \(\mu=0\), т. е. второй терм не расщепится совсем. На первый взгляд это выглядит довольно неожиданно. Но теперь мы убедились, что без предварительной проверки значения фактора \(g\) (не равен ли он нулю), ответ на поставленный вопрос может оказаться неверным.\\
7.7. Сложный эффект Зеемана. Некоторая спектральная линия, обусловленная переходом в \({ }^{2} S_{1 / 2}\)-состояние, расщепилась в слабом магнитном поле на шесть компонент. Определить спектральный символ исходного терма.

Решение. Согласно правилам отбора ( \(\Delta S=0, \Delta L= \pm 1\) и \(\Delta J=0, \pm 1\) ) можно написать \({ }^{2} P_{J} \rightarrow{ }^{2} S_{1 / 2}\). Неопределенным осталось только квантовое число \(J\). Из правила отбора для \(\Delta J\) следует, что \(J\) может иметь два значения: \(1 / 2\) или \(3 / 2\). При \(J=1 / 2\) возникают четыре компоненты, а при \(J=3 / 2\) - шесть (см. рис. 7.5). Следует обратить внимание на то, что во втором случае образуются именно шесть, а не три компоненты. Это обусловлено тем, что значения фактора Ланде у термов \({ }^{2} P_{3 / 2}\) и \({ }^{2} S_{1 / 2}\) разные ( \(4 / 3\) и 2 ). Таким образом, символ исходного терма \({ }^{2} P_{3 / 2}\).\\
7.8. Одну и ту же спектральную линию, испытывающую сложный эффект Зеемана, наблюдают в направлении 1 , а также в направлении 2 - после отражения от зеркала 3 (рис. 7.9). Сколько компонент будет наблюдаться в обоих направлениях, если спектральная линия обусловлена переходом

\begin{figure}[h]
\begin{center}
  \includegraphics[alt={},max width=\textwidth]{a57b3a57-78b5-4292-b0ab-4bd9f2ffcf56-182_225_317_1236_840}
\captionsetup{labelformat=empty}
\caption{Рис. 7.9}
\end{center}
\end{figure}

\({ }^{3} P_{2} \rightarrow{ }^{3} S_{1}\) ?

Решение. Изобразим возможные переходы между расщепленными в магнитном поле термами (рис. 7.10). Верхние подуровни расположены более тесно, чем нижние, поскольку их множители Ланде равны соответственно \(3 / 2\) и 2 . Поэтому все изображенные согласно правилу отбора (7.13) переходы различны, и в направлении 1 мы будем наблюдать девять компонент. В направлении же 2 наблюдается излучение, не перпендикулярное магнитному полю,

\begin{figure}[h]
\begin{center}
  \includegraphics[alt={},max width=\textwidth]{a57b3a57-78b5-4292-b0ab-4bd9f2ffcf56-183_491_454_136_130}
\captionsetup{labelformat=empty}
\caption{Рис. 7.10}
\end{center}
\end{figure}

а вдоль него. Поэтому \(\pi\)-компоненты исчезают, остаются только \(\sigma\)-компоненты, их шесть.\\
7.9. ЭПР. Найти магнитный момент атомов никеля (в состоянии \({ }^{3} F\) ), которые обнаруживают резонансное поглощение энергии при одновременном воздействии постоянного магнитного поля с индукцией \(B= =2,00\) кГс и перпендикулярного к нему переменного поля \(B_{v}\) с частотой \(v=3,50\) ГГц.

Решение. Согласно (7.19) при резонансе

\[
\hbar \cdot 2 \pi v=\delta E_{\text {рез }}=\mu_{\mathrm{b}} g B
\]

Отсюда находим фактор Ланде \(g=1,25\). Затем с помощью формулы (7.10) и данных в условии задачи ( \(L=3, S=1\) ) определим квантовое число \(J\) :

\[
J^{2}+J-20=0, \quad \text { откуда } J=4 .
\]

В результате получим

\[
\mu=\mu_{\mathrm{B}} g \sqrt{J(J+1)}=5,6 \mu_{\mathrm{B}} .
\]

\section*{Глава 8}
\section*{Атомное ядро}
\section*{§ 8.1. Состав и характеристика атомного ядра}
Состав ядра. Экспериментально установлено, что атомное ядро состоит из протонов и нейтронов. Эти частицы называют нуклонами.

Протон ( \(p\) ) обладает положительным зарядом \(e\) и массой

\[
m_{p}=\left\{\begin{array}{l}
1836,15 m_{e} \\
1,00759 \text { а.е.м. } \\
938,28 \text { МэВ, }
\end{array}\right.
\]

где \(m_{e}\) - масса электрона, а.е.м. - атомная единица массьг. Здесь же приведено значение массы протона и в энергетических единицах (как принято в ядерной физике).

Протон имеет спин \(s=1 / 2\) и собственный магнитный момент

\[
\mu_{p}=2,793 \mu_{\Omega},
\]

где \(\mu_{\text {я }}\) - ядерный магнетон (единица, в которой измеряют магнитные моменты нуклонов):

\[
\mu_{\text {я }}=\frac{e \hbar}{2 m_{p} c}=5,05 \cdot 10^{-24} \text { эрг } / \text { Гс. }
\]

Ядерный магнетон в 1836 раз меньше магнетона Бора, т. е. собственный магнитный момент протона в 660 раз меньше магнитного момента электрона.

Нейтрон \((n)\). Его электрический заряд равен нулю, а масса близка к массе протона:

\[
m_{n}=\left\{\begin{array}{l}
1838,68 m_{e} \\
1,00898 \text { а.е.м. } \\
939,55 \text { МэВ, }
\end{array}\right.
\]

что на \(0,14 \%\) или \(2,5 m_{e}\) больше массы протона.

\footnotetext{\begin{itemize}
  \item Атомная единица массы равна \(1 / 12\) массы нейтрального атома \({ }^{12}\) С, т. е. 1 а.е.м. \(=1,66 \cdot 10^{-24}\) г или 931,50 МэВ.
\end{itemize}
}Спин нейтрона \(s=1 / 2\) и, несмотря на отсутствие электрического заряда, нейтрон имеет магнитный момент

\[
\mu_{n}=-1,91 \mu_{\Omega} .
\]

Знак минус означает, что «направления» спина и магнитного момента у нейтрона взаимно противоположны.

В свободном состоянии нейтрон нестабилен и самопроизвольно распадается, превращаясь в протон и испуская электрон и еще одну частицу, нейтрино* ( \(v\) ):


\begin{equation*}
n \rightarrow p+e+v . \tag{8.1}
\end{equation*}


Период полураспада (время, за которое распадается половина первоначального количества нейтронов) равно примерно 12 мин.

Характеристики атомного ядра. Основными величинами, характеризующими атомное ядро, являются зарядовое \(Z\) и массовое \(A\) числа. Число \(Z\) равно количеству протонов в ядре и определяет его электрический заряд \(Z e\). Его также называют атомным номером. Массовое число \(A\) определяет число нуклонов в ядре. Число же нейтронов в ядре \(N=A-Z\).

Символически эти характеристики ядра обозначают так:


\begin{equation*}
{ }_{Z}^{A} X \tag{8.2}
\end{equation*}


где под \(X\) имеется в виду химический символ элемента, которому принадлежит данное ядро, например, \({ }_{1}^{2} \mathrm{H},{ }_{2}^{4} \mathrm{He},{ }_{92}^{238} \mathrm{U}\) и т. д. Поскольку \(Z\) определяется местом элемента в периодической системе, число \(Z\) в символическом обозначении (8.2) часто не указывают.

В соответствии с общепринятой терминологией конкретные атомы с данным числом протонов и нейтронов в ядре принято называть нуклидами. Нуклиды с одинаковым числом протонов (т. е. принадлежащие одному химическому элементу) называют изотопами.

Атомы изотопов обладают практически очень близкими фи-зико-химическими свойствами. Это связано с тем, что на строение электронной оболочки атома ядро влияет в основном толь-

\footnotetext{\begin{itemize}
  \item В дальнейшем вопрос об этой частице будет уточнен.
\end{itemize}
}ко своим электрическим полем. У изотопов же эти поля одинаковы, за исключением некоторых случаев. Сильнее всего это различие у трех нуклидов: \({ }_{1}^{1} \mathrm{H},{ }_{1}^{2} \mathrm{H}\) и \({ }_{1}^{3} \mathrm{H}\), ядра которых также существенно отличаются друг от друга. Поэтому этим трем нуклидам присвоены разные названия - соответственно обычный водород, дейтерий и тритий, а ядра дейтерия и трития дейтрон \((d)\) и тритон \((t)\).

У разных атомов число изотопов различно, среди них имеются стабильные и радиоактивные.

Размеры ядер. У атомного ядра (как и у всякой квантовой системы) нет четко определенной границы. В экспериментах по рассеянию электронов и нуклонов на ядрах установлено, что в каждом ядре имеется внутренняя область, в которой плотность \(\rho\) ядерного вещества практически постоянна, и поверхностный слой, где эта плотность падает до нуля. Типичное распределение концентрации нуклонов в зависимости от расстояния до центра ядра, т. е. \(n(r)\) показано на рис. 8.1, где \(r_{0}\) - радиус ядра - расстояние от

\begin{figure}[h]
\begin{center}
  \includegraphics[alt={},max width=\textwidth]{a57b3a57-78b5-4292-b0ab-4bd9f2ffcf56-186_301_408_730_750}
\captionsetup{labelformat=empty}
\caption{Рис. 8.1}
\end{center}
\end{figure}

В первом приближении ядро можно считать сферическим радиуса


\begin{equation*}
r_{0}=(1,2 \div 1,3) A^{1 / 3} \text { фм, } \tag{8.3}
\end{equation*}


где 1 фм \(=10^{-13}\) см. Из этой формулы вытекает важный вывод: масса ядра, определяемая массовым числом \(A\), пропорциональна его объему \(V\), поскольку \(V \sim r_{0}^{3}\) ο \(A\). Следовательно, плотность вещества во всех ядрах примерно одинакова и, как показывает расчет, равна \(\rho \approx 2 \cdot 10^{14} г / \mathrm{cm}^{3}\) - величина, весьма впечатляющая!

Спин ядра I. Он слагается из спинов нуклонов и их орбитальных моментов. Напомним, что при определении спина называют одно число, которым указывается максимальная проекция спина на произвольную ось \(Z\).

Спин нуклона равен \(1 / 2\), поэтому спин \(I\) ядра может быть как целым, так и полуцелым - в зависимости от числа нуклонов, четного или нечетного.

В основных состояниях всех стабильных ядер \(I \leqslant 9 / 2\). Это указывает на то, что моменты импульса большинства нуклонов в ядре взаимно компенсируют друг друга, располагаясь «антипараллельно». У всех ядер с четными числами протонов и нейтронов спин основного состояния \(I=0\).

Со спином ядра связан магнитный момент. Взаимодействие магнитных моментов электронов и ядра приводит к дополнительному расщеплению энергетических уровней, в результате чего линии тонкой структуры испытывают в свою очередь расщепление - наблюдается так называемая сверхтонкая структура спектральных линий. Соответствующее расщепление весьма мало (порядка нескольких тысячных нм), и его можно наблюдать лишь с помощью спектральных приборов очень высокой разрешающей способности (например, интерферометрами Фаб-ри-Перо).

\section*{§ 8.2. Масса и энергия связи ядра}
Масса ядра не является аддитивной величиной: она не равна сумме масс образующих ядро нуклонов. Причиной является сильное взаимодействие нуклонов в ядре. Из-за этого взаимодействия для полного разделения ядра на отдельные свободные нуклоны необходимо произвести минимальную работу, которая и определяет энергию связи ядра \(E_{\text {св }}\). Наоборот, при образовании ядра из свободных нуклонов эта энергия выделяется (в виде, например, электромагнитного излучения).

\begin{figure}[h]
\begin{center}
  \includegraphics[alt={},max width=\textwidth]{a57b3a57-78b5-4292-b0ab-4bd9f2ffcf56-187_227_342_1425_68}
\captionsetup{labelformat=empty}
\caption{Рис. 8.2}
\end{center}
\end{figure}

Известно, что энергия покоя частицы связана с ее массой как \(E_{0}=m c^{2}\). Значит, энергия покоя ядра меньше суммы энергий покоя свободных нуклонов, входящих в состав данного ядра (рис. 8.2), и мы имеем


\begin{equation*}
E_{\text {св }}=\sum m_{N}-m_{\text {я }}, \tag{8.4}
\end{equation*}


где \(\sum m_{N}\) - сумма масс нуклонов, \(m_{\text {я }}\) - масса ядра. Здесь, как и в дальнейшем, массы частиц выражены в энергетических единицах. Более детально (8.4) записывают так:


\begin{equation*}
E_{\text {св }}=Z m_{p}+N m_{n}-m_{\text {я }}, \tag{8.5}
\end{equation*}


где \(Z\) и \(N\) - число протонов и нейтронов в ядре, причем

\[
Z+N=A .
\]

Формула (8.5) неудобна для практических расчетов, поскольку в таблицах приводятся массы не ядер, а массы нуклидов, т. е. атомов \(m_{\text {я }}\). Учитывая это обстоятельство, поступим так. Соотношение (8.5) практически не изменится, если заменить массу протона массой нуклида \({ }^{1} \mathrm{H}\left(m_{\mathrm{H}}\right)\), а массу ядра \(m_{\text {я }}\) массой соответствующего нуклида ( \(m_{\mathrm{a}}\) ). Другими словами, в выражении (8.5) мы добавляем \(Z\) электронов и столько же их вычитаем, пренебрегая при этом ничтожной по сравнению с массой ядра энергией связи электрона с ядром.

И тогда формулу (8.5) можно записать в виде


\begin{equation*}
E_{\text {св }}=Z m_{\mathrm{H}}+N m_{n}-m_{\mathrm{a}} . \tag{8.6}
\end{equation*}


Более того, для упрощения расчетов вводят понятие дефект массы \(\Delta\) как разность между массой (в а.е.м.) и массовым числом \(A\) ядра или нуклона: \(\Delta=m-A\). Тогда


\begin{equation*}
m_{\mathrm{H}}=1+\Delta_{\mathrm{H}}, \quad m_{n}=1+\Delta_{n}, \quad m_{\mathrm{a}}=A+\Delta_{\mathrm{a}}, \tag{8.7}
\end{equation*}


и формулу (8.6) можно представить в виде


\begin{equation*}
E_{\mathrm{cB}}=Z \Delta_{\mathrm{H}}+N \Delta_{n}-\Delta_{\mathrm{a}}, \tag{8.8}
\end{equation*}


где \(N=A-Z\). Соответственно и в таблицах приводят не массы нуклидов, а их дефекты масс, как это показано (в качестве примера) в табл. 8.1.

\begin{table}[h]
\begin{center}
\captionsetup{labelformat=empty}
\caption{Таблица 8.1}
\begin{tabular}{||c|c|c||}
\hline\hline
\(Z\) & Нуклид & \begin{tabular}{c}
Дефект массы \\
\(\Delta=m-A\), а.е.м. \\
\end{tabular} \\
\hline
0 & \(n\) & 0,008665 \\
1 & H & 0,007825 \\
 & \({ }^{2} \mathrm{H}\) & 0,014102 \\
\multirow{2}{*}{2} & \({ }^{3} \mathrm{H}\) & 0,016049 \\
 & \({ }^{3} \mathrm{He}\) & 0,016030 \\
 & \({ }^{4} \mathrm{He}\) & 0,002604 \\
\hline
\end{tabular}
\end{center}
\end{table}

Более обширная таблица дефектов масс \(\Delta\) приведена в Приложении, из которой видно, что \(\Delta\) может быть как положительным, так и отрицательным. За «начало отсчета» принят нуклид \({ }^{12} \mathrm{C}\), дефект массы которого \(\Delta=0\).

Удельная энергия связи. Так называют энергию связи, приходящуюся в среднем на один нуклон, т. е. \(E_{\text {св }} / A\). Эта величина характеризует меру прочности ядра: чем больше \(E_{\text {св }} / A\), тем ядро прочнее.

Пример. Вычислим с помощью табл. 8.1 удельную энергию связи в ядре \({ }^{4} \mathrm{He}\).\\
Воспользовавшись формулой (8.8), запишем:\\
\(E_{\text {св }}=2 \cdot 0,007825+2 \cdot 0,008665-0.002604=0,030376\) а.е.м.\\
Учитывая, что 1 а.е.м. соответствует энергии 931,5 МэВ, получим:

\[
E_{\text {св }}=28,3 \text { МэВ и } E_{\text {св }} / A=7,1 \text { МэВ. }
\]

Для сравнения: энергия связи электронов в атомах порядка 10 эВ, что по существу пренебрежимо мало с величиной удельной энергии связи ядра.

Аналогично (8.8) имеет вид формула для расщепления ядра массы \(m\), например, на две частицы с массами \(m_{1}\) и \(m_{2}\). Необходимая для этого работа равна энергии связи \(E_{\text {св }}\) этих частиц в исходном ядре. Она определяется (рис. 8.3) как


\begin{equation*}
E_{\text {св }}=\Delta_{1}+\Delta_{2}-\Delta_{\mathrm{a}}, \tag{8.9}
\end{equation*}


где все три слагаемых справа - это дефекты масс соответствующих данным ядрам нуклидов (в а.е.м. или МэВ). Еще раз отметим, что используя дефекты масс вместо самих масс, мы заметно упрощаем процедуру расчета.

\begin{figure}[h]
\begin{center}
  \includegraphics[alt={},max width=\textwidth]{a57b3a57-78b5-4292-b0ab-4bd9f2ffcf56-190_176_293_141_867}
\captionsetup{labelformat=empty}
\caption{Рис. 8.3}
\end{center}
\end{figure}

Число протонов и нейтронов в процессе расщепления ядра не меняется, поэтому в выражении \(E_{\text {св }}= =\left(m_{1}+m_{2}\right)-m\), где \(m_{1}\) и \(m_{2}\) - массы ядер, на которые расщепилось исходное ядро массы \(m\); массы (и энергии покоя) протонов и нейтронов сокращаются и остается только со знаком минус энергия связи этих трех ядер. В результате имеем


\begin{equation*}
E_{\text {св }}=E_{\text {св я }}-\left(E_{\text {св } 1}+E_{\text {св } 2}\right), \tag{8.10}
\end{equation*}


где \(E_{\text {св }}\) - энергия связи исходного ядра.\\
Формулы, подобные (8.9) и (8.10), широко используют в ядерной физике при анализе тех или иных ядерных реакций.

Интересно сравнить полученную в предыдущем примере удельную энергию связи у ядра \({ }^{4} \mathrm{He}\) с энергией связи, скажем, одного нейтрона в этом же ядре (т. е. с работой, которую необходимо затратить для извлечения одного нейтрона из этого ядра).

В этом процессе нуклид \({ }^{4} \mathrm{He}\) превращается в нуклид \({ }^{3} \mathrm{He}\), и мы, воспользовавшись формулой (8.9) и табл. 8.1, запишем:

\[
E_{\text {св }}=\Delta_{1}+\Delta_{3}-\Delta_{4}=0,02125 \text { а.е.м. }=19,8 \text { МэВ, }
\]

где \(\Delta_{1}, \Delta_{3}, \Delta_{4}\) - это дефекты масс нуклидов \({ }^{1} \mathrm{H},{ }^{3} \mathrm{He}\) и \({ }^{4} \mathrm{He}\).\\
Отличие полученного результата от удельной энергии связи \((7,1 \mathrm{M} \ni \mathrm{B})\) весьма разительное. Но дело в том, что это разные по своей сути величины. Можно, конечно, продолжать бы этот процесс: из ядра \({ }^{3}\) Не извлечь, например, протон, т. е. найти энергию связи протона с этим ядром. Получим нуклид \({ }^{3} \mathrm{H}\). Из него извлечь последовательно сначала один, затем другой нейтрон. И мы обнаружим, что суммарная работа всех этих процессов, т. е. сумма соответствующих энергий связи, окажется, как и должно быть, равной энергии связи ядра \({ }^{4} \mathrm{He}\), т. е. 28,3 МэВ.

Вернемся к удельной энергии связи \(E_{\text {св }} / A\). Эта величина зависит от массового числа \(A\). График соответствующей зависимости показан на рис 8.4. Анализ вида этого графика дает существенную информацию о свойствах ядер и даже о характере ядерных сил между нуклонами.

\begin{figure}[h]
\begin{center}
  \includegraphics[alt={},max width=\textwidth]{a57b3a57-78b5-4292-b0ab-4bd9f2ffcf56-191_388_503_163_68}
\captionsetup{labelformat=empty}
\caption{Рис. 8.4}
\end{center}
\end{figure}

В грубом приближении можно считать, что удельная энергия связи ядер почти не зависит от массового числа \(A\) и равна примерно 8 МэВ. Приближенная независимость удельной энергии связи от \(A\) означает, что ядерные силы обладают свойством насьщения. Оно заключается в том, что каждый нуклон взаимодействует только с ограниченным числом соседних нуклонов. Иначе бы удельная энергия связи линейно зависела от \(A\) (если бы каждый нуклон взаимодействовал со всеми остальными, то энергия этого взаимодействия была бы пропорциональна \(A-1\) ). Благодаря насыщению ядерных сил плотность ядерного вещества внутри ядра однородна. Именно поэтому линейный размер ядра с массовым числом \(A\) пропорционален \(A^{1 / 3}\) в соответствии с (8.3).

Отсюда также следует, что ядерные силы являются короткодействующими с радиусом порядка среднего расстояния между нуклонами в ядре ( \(\sim 10^{-13} \mathrm{~cm}\) ).

Наиболее прочными являются ядра с массовыми числами \(A \sim 50 \div 60\), т. е. элементов от \(\operatorname{Cr}\) до Zn . Удельная энергия связи этих ядер достигает 8,7 МэВ на нуклон. Как с ростом, так и с уменьшением \(A\) удельная энергия связи уменьшается, и тяжелым ядрам становится энергетически выгодным делиться, образуя при этом более легкие (и прочные) ядра, а легким ядрам, наоборот, выгодно сливаться друг с другом, образуя более тяжелые ядра.

В обоих случаях выделяется энергия. Например, при делении ядра \({ }^{235} \mathrm{U}\) - около 200 МэВ (в основном в виде кинетической энергии разлетающихся под действием кулоновских сил отталкивания осколков). А при слиянии дейтрона с тритоном ( \(d+t=\alpha+n\) ) происходит синтез \(\alpha\)-частиц - ядер нуклида \({ }^{4} \mathrm{He}\) - с выделением энергии 17,6 МэВ. В первом случае выделяемую энергию называют атомной, во втором - термоядерной. На единицу массы во втором случае выделяется в пять раз больше энергии, чем в первом, поэтому проблема управляемого термоядерного синтеза считается особо важной.

\section*{§ 8.3. Ядерные силы}
Особенности ядерных сил. Огромная энергия связи нуклонов в ядрах (по сравнению с энергией связи электронов в атоме) означает, что между нуклонами действуют мощные ядерные силы притяжения, по сравнению с которыми электромагнитные силы отталкивания в сотни раз слабее.

Отличительными особенностями ядерных сил являются следующие.

\begin{enumerate}
  \item Эти силы являются короткодействующими с радиусом действия \(\sim 10^{-13} \mathrm{~cm}\). На существенно меньших расстояниях притяжение нуклонов сменяется их отталкиванием.
  \item Они обладают зарядовой независимостью, что проявляется в одинаковости сил взаимодействия нуклонов \(n-n, p-p, n-p\).
  \item Эти силы не являются центральными. Их, образно говоря, нельзя представить направленными вдоль прямой, проходящей через центры взаимодействующих нуклонов. Нецентральность связана с тем фактом, что эти силы зависят от ориентации спинов нуклонов.
  \item Обладают свойством насьшения: каждый нуклон в ядре взаимодействует с ограниченным числом ближайших нуклонов. Это проявляется практически в независимости удельной энергии связи от массового числа \(A\).
\end{enumerate}

Механизм взаимодействия нуклонов. Согласно классической физике взаимодействие между частицами осуществляется посредством силовых полей. Так, покоящийся электрический заряд создает вокруг себя электрическое поле, которое воздействует на другой заряд с некоторой силой.

Квантовая физика не изменила такое представление, но учла квантовые свойства самого поля: всякому полю должна соответствовать определенная частица - квант поля, которая и является переносчиком взаимодействия. Одна из взаимодействующих частиц испускает квант поля, другая его поглощает. В этом и состоит механизм взаимодействия частиц. Существенно, что обмен частицами лежит в основе вообще всех взаимодействий частиц и является фундаментальным квантовым свойством природы (например, электромагнитные взаимодействия осуществляются путем обмена фотонами).

При взаимодействии нуклонов квантами поля являются \(\pi\)-мезоны, существование которых было предсказано Юкавой (1935). По его оценке эти частицы занимали промежуточное положение по массе между электроном и нуклоном. И такие частицы были экспериментально обнаружены.

Квантовая природа подобных процессов взаимодействия заключается в том, что они могут происходить только благодаря соотношению неопределенностей. По классическим законам такие процессы идти не могут в связи с нарушением закона сохранения энергии. Ясно, что, например, покоившийся свободный нейтрон не может самопроизвольно превратиться в нейтрон \(+\pi\)-мезон, суммарная масса которых больше массы нейтрона.

Квантовая теория этот запрет устраняет. Согласно ей энергия состояния системы, существующего время \(\Delta t\), оказывается определенной лишь с неопределенностью \(\Delta E\), удовлетворяющей соотношению \(\Delta E \cdot \Delta t \sim \hbar\). Из этого соотношения следует, что энергия системы может претерпевать отклонения \(\Delta E\), длительность которых не должна превышать величины \(\Delta t \approx \hbar / \Delta E\).

В этом случае нарушение закона сохранения энергии при испускании \(\pi\)-мезона обнаружить нельзя.

Согласно соотношению неопределенностей энергия-время испущенный \(\pi\)-мезон с энергией \(m_{\pi} c^{2}\) (а это есть величина \(\Delta E\) ) может существовать только конечное время, которое не больше, чем


\begin{equation*}
\tau_{\pi} \approx \hbar / m_{\pi} c^{2} . \tag{8.11}
\end{equation*}


По истечении этого времени \(\pi\)-мезон поглощается испустившим его нуклоном. Расстояние, на которое \(\pi\)-мезон удаляется от нуклона, при этом составляет


\begin{equation*}
l \approx c \tau_{\pi} \approx \hbar / m_{\pi} c \tag{8.12}
\end{equation*}


что равно комптоновской длине волны \(\pi\)-мезона \(\lambda_{C}=\lambda_{C} / 2 \pi\).\\
Частицы, испускание и поглощение которых происходит с кажущимся нарушением закона сохранения энергии, называют виртуальными.

Если поблизости от нуклона нет других частиц, то все испущенные нуклоном виртуальные π-мезоны поглощаются этим же нуклоном. В этом случае говорят, что одиночный нуклон

всегда окружен так называемой «мезонной шубой». Это облако виртуальных \(\pi\)-мезонов, которые безостановочно испускаются и поглощаются нуклоном, удаляясь от него в среднем на расстояние \(l\) не более, чем комптоновская длина волны (8.12).

Когда два нуклона сближаются и их мезонные шубы начинают соприкасаться, создаются условия для обмена виртуальными мезонами - возникает ядерное взаимодействие. В этом и состоит механизм взаимодействия нуклонов. Мы видим, что радиус действия ядерных сил имеет порядок комптоновской длины волны (8.12). Из опыта известно, что этот радиус порядка \(10^{-13} \mathrm{~cm}\), что позволяет с помощью (8.12) оценить массу \(\pi\)-мезона: \(m_{\pi} \sim 270 m_{e}\).

Зависимость радиуса действия ядерных сил от массы виртуальных частиц - переносчиков взаимодействия - это фундаментальный квантовый закон. Именно этим законом определяется дальнодействие электромагнитных сил, поскольку кванты электромагнитного поля - виртуальные фотоны являются безмассовыми частицами, которые могут иметь сколь угодно малую энергию.

Если нуклону передать энергию не меньше, чем энергия покоя \(\pi\)-мезона, то один или несколько виртуальных мезонов могут быть превращены в обычные \(\pi\)-мезоны, существующие независимо от нуклона. Это происходит, например, при столкновении нуклонов достаточно высоких энергий.

Модели ядер. К настоящему времени еще нет последовательно законченной теории ядра, которая бы единым образом объясняла все его свойства. И связано это в основном с двумя трудностями:

\begin{enumerate}
  \item недостаточность наших знаний о силах взаимодействия нуклонов в ядре и
  \item каждое атомное ядро - это квантовая система многих сильно взаимодействующих частиц; задача же многих тел в квантовой теории чрезвычайно трудна и громоздка. До сих пор не найдены способы ее решения.\\
Поэтому в теории атомного ядра очень важную роль играют модели, достаточно хорошо описывающие определенную совокупность ядерных свойств и допускающие сравнительно простую математическую трактовку. При этом каждая модель об-
\end{enumerate}

ладает, естественно, ограниченными возможностями и не претендует на полное описание ядра.

Ограничимся кратким рассмотрением двух моделей ядра: капельной и оболочечной.

Капельная модель. Эта простейшая модель была предложена М. Борном (1936). В ней атомное ядро рассматривается как капля заряженной несжимаемой жидкости с очень высокой плотностыо ( \(\sim 10^{14} г / \mathrm{cm}^{3}\) ). Капельная модель позволила вывести полуэмпирическую формулу для энергии связи ядра и помогла объяснить ряд других явлений, в частности процесс деления тяжелых ядер.

Оболочечная модель. Эта модель, предложенная Геп-перт-Майер и Йенсоном (1950), является более реалистичной. В данной модели считается, что каждый нуклон движется в усредненном поле остальных нуклонов ядра. В соответствии с этим имеются дискретные энергетические уровни, заполненные нуклонами с учетом принципа Паули. Эти уровни группируются в оболочки, в каждой из которых может находиться определенное число нуклонов. Полностью заполненные оболочки образуют особо устойчивые структуры. Таковыми являются ядра, имеющие, в соответствии с опытом, число протонов, либо нейтронов (либо оба эти числа) \(2,8,20,28,50,82,126\). Эти числа и соответствующие им ядра называют магическими.

Кроме предсказания магических чисел, эта модель позволила объяснить спины основных и возбужденных состояний ядер, а также их магнитные моменты.

\section*{§ 8.4. Радиоактивность}
Радиоактивность заключается в самопроизвольном (спонтанном) распаде ядер с испусканием одной или нескольких частиц. Такие ядра и соответствующие им нуклиды называют радиоактивными (в отличие от стабильных ядер). Радиоактивное ядро называют материнским, а ядра, образующиеся в результате распада, - дочерними.

Необходимое условие радиоактивного распада заключается в том, что масса исходного ядра должна превышать сумму масс продуктов распада. Поэтому каждый радиоактивный распад происходит с выделением энергии.

Радиоактивность подразделяют на естественную и искусственную. Первая относится к радиоактивным ядрам, существующим в природных условиях, вторая - к ядрам, полученным посредством ядерных реакций в лабораторных условиях. Принципиально они не отличаются друг от друга.\\
\(К\) основным типам радиоактивности относятся \(\alpha\)-, \(\beta\) - и \(\gamma\)-распады. Прежде чем характеризовать их более подробно, рассмотрим общий для всех видов радиоактивности закон протекания этих процессов во времени.

Основной закон радиоактивного распада. Одинаковые ядра претерпевают распад за различные времена, предсказать которые заранее нельзя. Поэтому можно считать, что число ядер, распадающихся за малый промежуток времени \(\mathrm{d} t\), пропорционально как числу \(N\) имеющихся ядер в этот момент, так и \(\mathrm{d} t\) :


\begin{equation*}
-\mathrm{d} N=\lambda N \mathrm{~d} t, \tag{8.13}
\end{equation*}


где \(-\mathrm{d} N\) - убыль числа ядер за время \(\mathrm{d} t\) (это и есть число распавшихся ядер за промежуток \(\mathrm{d} t\) ), \(\lambda\) - постоянная распада, величина, характерная для каждого радиоактивного препарата.

Интегрирование уравнения (8.13) дает


\begin{equation*}
N=N_{0} \mathrm{e}^{-\lambda t}, \tag{8.14}
\end{equation*}


где \(N_{0}\) - число ядер в момент \(t=0, N\) - число нераспавшихся ядер к моменту \(t\). Соотношение (8.14) и называют основным законом радиоактивного распада. Как видно, число \(N\) еще не распавшихся ядер убывает со временем экспоненциально.

Интенсивность радиоактивного распада характеризуют числом ядер, распадающихся в единицу времени. Из (8.13) видно, что эта величина \(|\mathrm{d} N / \mathrm{d} t|=\lambda N\). Ее называют активностью \(A\). Таким образом, активность


\begin{equation*}
A=\lambda N . \tag{8.15}
\end{equation*}


Ее измеряют в беккерелях (Бк), 1 Бк \(=1\) распад/с; а также в кюри (Ки), 1 Ки \(=3,7 \cdot 10^{10}\) Бк.

Активность в расчете на единицу массы радиоактивного препарата называют удельной активностью.

Вернемся к формуле (8.14). Наряду с постоянной \(\lambda\) и активностью \(A\) процесс радиоактивного распада характеризуют еще двумя величинами: периодом полураспада \(T\) и средним временем жизни ч ядра.

Период полураспада \(T\) - это время, за которое распадается половина первоначального количества ядер. Оно определяется условием \(N_{0} / 2=N_{0} \mathrm{e}^{-\lambda T}\), откуда


\begin{equation*}
T=\ln 2 / \lambda=0,693 / \lambda \tag{8.16}
\end{equation*}


Среднее время жизни ч. Число ядер \(\delta N(t)\), испытавших распад за промежуток времени ( \(t, t+\mathrm{d} t\) ), определяется правой частью выражения (8.13): \(\delta N(t)=\lambda N \mathrm{~d} t\). Время жизни каждого из этих ядер равно \(t\). Значит сумма времен жизни всех \(N_{0}\) имевшихся первоначально ядер определяется интегрированием выражения \(t \delta N(t)\) по времени от 0 до \(\infty\). Разделив сумму времен жизни всех \(N_{0}\) ядер на \(N_{0}\), мы и найдем среднее время жизни \(\tau\) рассматриваемого ядра:

\[
\tau=\frac{1}{N_{0}} \int t \delta N(t)=\frac{1}{N_{0}} \int_{0}^{\infty} t \lambda N(t) \mathrm{d} t
\]

Остается подставить сюда выражение (8.14) для \(N(t)\) и выполнить интегрирование по частям, после чего мы получим:


\begin{equation*}
\tau=1 / \lambda . \tag{8.17}
\end{equation*}


Заметим, что \(\tau\) равно, как следует из (8.14) промежутку времени, за которое первоначальное количество ядер уменьшается в е раз.

Сравнивая (8.16) и (8.17), видим, что период полураспада \(T\) и среднее время жизни \(\tau\) имеют один и тот же порядок и связаны между собой формулой


\begin{equation*}
T=\tau \ln 2=0,693 \tau . \tag{8.18}
\end{equation*}


В заключение рассмотрим пример на активность и среднее время жизни.

Пример. Найдем среднее время жизни радионуклида \({ }^{55} \mathrm{Co}\), если его активность уменьшается на \(\eta=4 \%\) за время \(t_{0}=60\) мин.\\
Активность \(A\) уменьшается со временем по тому же закону (8.14), что и число радиоактивных ядер, ибо \(A=\lambda N=A_{0} \mathrm{e}^{-\lambda t}\). В

нашем случае \(\eta=\left(A_{0}-A\right) / A_{0}=1-\mathrm{e}^{-\lambda t_{0}}\). Отсюда


\begin{equation*}
\ln (1-\eta)=-\lambda t_{0} . \tag{*}
\end{equation*}


Согласно (8.17) \(\lambda=1 / \tau\). Поэтому из формулы (*) следует, что

\[
\tau=-t_{0} / \ln (1-\eta) \approx t_{0} / \eta=1 \text { ч } / 0,04=25 \text { ч. }
\]

\section*{§ 8.5. Основные типы радиоактивности}
К основным типам радиоактивности относятся альфа-, бетаи гамма-распады. Рассмотрим более подробно их специфические особенности.

Альфа-распад. В этом случае происходит самопроизвольное испускание ядром \(\alpha\)-частицы (ядра нуклида \({ }^{4} \mathrm{He}\) ), и это происходит по схеме


\begin{equation*}
{ }_{Z}^{A} \mathrm{X} \rightarrow{ }_{Z-2}^{A-4} \mathrm{Y}+{ }_{2}^{4} \mathrm{He}, \tag{8.19}
\end{equation*}


где X - символ материнского ядра, Y - дочернего.\\
Установлено, что \(\alpha\)-частицы испускают только тяжелые ядра. Кинетическая энергия, с которой \(\alpha\)-частицы вылетают из распадающегося ядра, порядка нескольких МэВ. В воздухе при нормальном давлении пробег \(\alpha\)-частиц составляет несколько сантиметров (их энергия расходуется на образование ионов на своем пути).

Кинетическая энергия \(\alpha\)-частиц возникает за счет избытка энергии покоя материнского ядра над суммой энергий покоя дочернего ядра и \(\alpha\)-частицы. Эта избыточная энергия распределяется между \(\alpha\)-частицей и дочерним ядром в отношении, обратно пропорциональном их массам (в соответствии с законом сохранения импульса).

Пример. Покоившееся ядро \({ }^{213} \mathrm{Po}\) испустило \(\alpha\)-частицу с кинетической энергией \(K_{\alpha}=8,34\) МэВ. При этом дочернее ядро оказалось непосредственно в основном состоянии. Найдем суммарную энергию \(Q\), освобождающуюся в этом процессе (ее называют энергией \(\alpha\)-распада).

Искомая энергия \(Q=K_{\alpha}+K_{\text {д }}\), где \(K_{\text {д }}\) - кинетическая энергия дочернего ядра. Чтобы ее найти, воспользуемся законом сохранения импульса, \(p_{\alpha}=p_{\text {д }}\), и соотношением \(K_{\text {д }}=p_{\text {д }}^{2} / 2 m_{\text {д }}\). Из последних двух формул следует, что \(m_{\alpha} K_{\alpha}=m_{\mu} K_{\text {д }}\).

Значение \(K_{\text {д }}\) из этого равенства подставим в выражение для \(Q\) и в результате получим

\[
Q=K_{\alpha}\left(1+m_{\alpha} / m_{\text {д }}\right)=(213 / 209) K_{\alpha}=8,50 \text { МэВ. }
\]

Следует заметить, что относительная доля энергии, приобретаемой дочерним ядром, мала - порядка \(2 \%\), в чем можно убедиться, вычислив дробь в скобках последней формулы.

Чаще всего радиоактивный препарат испускает несколько

\begin{figure}[h]
\begin{center}
  \includegraphics[alt={},max width=\textwidth]{a57b3a57-78b5-4292-b0ab-4bd9f2ffcf56-199_275_330_510_68}
\captionsetup{labelformat=empty}
\caption{Рис. 8.5}
\end{center}
\end{figure}

моноэнергетических групп \(\alpha\)-частиц, отличающихся по энергиям. Это объясняется тем, что дочернее ядро Y может возникать не только в основном, но и в возбужденных состояниях (рис. 8.5, где для простоты показан только один возбужденный уровень). Наиболее интенсивной является группа \(\alpha\)-частиц, обусловленная переходом непосредственно в основное состояние. Распады, идущие через возбужденные уровни дочернего ядра, сопровождаются испусканием \(\gamma\)-квантов.

Альфа-частица возникает только в момент радиоактивного

\begin{figure}[h]
\begin{center}
  \includegraphics[alt={},max width=\textwidth]{a57b3a57-78b5-4292-b0ab-4bd9f2ffcf56-199_360_364_1048_68}
\captionsetup{labelformat=empty}
\caption{Рис. 8.6}
\end{center}
\end{figure}

Преодоление \(\alpha\)-частицей потенциального барьера в данных условиях происходит благодаря туннельному эффекту (§ 4.5). Квантовая теория, учитывая волновые свойства \(\alpha\)-частицы, «позволяет» ей с определенной вероятностью проникать сквозь такой барьер. Соответствующий расчет хорошо подтверждается результатами измерений.

Бета-распад. Так называют самопроизвольный процесс, в котором исходное ядро превращается в другое ядро с тем же массовым числом \(A\), но с зарядовым числом \(Z\), отличающимся от исходного на \(\pm 1\). Это связано с тем, что \(\beta\)-распад сопровожда-

ется испусканием электрона (позитрона) или его захватом из оболочки атома. Различают три разновидности \(\beta\)-распада:

\begin{enumerate}
  \item электронный \(\beta^{-}\)-распад, в котором ядро испускает электрон и его зарядовое число \(Z\) становится \(Z+1\);
  \item позитронный \(\beta^{+}\)-распад, в котором ядро испускает позитрон и его зарядовое число \(Z\) становится \(Z-1\);
  \item \(K\)-захват, в котором ядро захватывает один из электронов электронной оболочки атома (обычно из \(K\)-оболочки) и его зарядовое число \(Z\) становится равным \(Z-1\). На освободившееся место в \(K\)-оболочке переходит электрон с другой оболочки, и поэтому \(K\)-захват всегда сопровождается характеристическим рентгеновским излучением.\\
Энергия \(\boldsymbol{\beta}\)-распада. Выясним, как определяется энергия \(Q\), освобождающаяся при \(\beta^{-}\)-распаде, \(\beta^{+}\)-распаде и \(K\)-захвате, если известны массы материнского и дочернего атомов ( \(M_{\text {м }}\) и \(M_{\text {д }}\) ), а также масса электрона \(m_{e}\).
\end{enumerate}

При \(\beta^{-}\)-распаде ядро с порядковым номером \(Z\) распадается по схеме


\begin{equation*}
M(Z) \rightarrow M(Z+1)+m_{e} \tag{8.20}
\end{equation*}


где \(M\) - это масса ядра. Однако в таблицах всегда приводятся массы атомов (нуклидов). Чтобы перейти в (8.20) к массам атомов, добавим к обеим частям этого соотношения по \(Z\) электронов, т. е. массу \(Z m_{e}\), причем следующим образом:


\begin{equation*}
Z m_{e}=(Z+1) m_{e}-m_{e}, \tag{8.21}
\end{equation*}


В результате получим соотношение для масс нуклидов:


\begin{equation*}
M_{\mathrm{M}}(Z) \rightarrow M_{\text {д }}(Z+1) . \tag{8.22}
\end{equation*}


Отсюда энергия \(\beta^{-}\)-распада


\begin{equation*}
Q=M_{\mathrm{M}}-M_{\text {д }}, \tag{8.23}
\end{equation*}


где, напомним, массы берутся в энергетических единицах (МэВ).\\
Аналогично для \(\beta^{+}\)-распада:

\[
+\begin{align*}
& M(Z) \rightarrow M(Z-1)+m_{e}  \tag{8.24}\\
& \frac{Z m_{e}=(Z-1) m_{e}+m_{e}}{M_{\mathrm{M}} \rightarrow M_{\text {д }}+2 m_{e}}
\end{align*}
\]

Следовательно, при \(\beta^{+}\)-распаде


\begin{equation*}
Q=M_{\mathrm{M}}-M_{\text {д }}+2 m_{e} . \tag{8.25}
\end{equation*}


Наконец, в случае \(K\)-захвата

\[
+\begin{gather*}
M(Z)+m_{e} \rightarrow M(Z-1) \\
Z m_{e}-m_{e}=(Z-1) m_{e}  \tag{8.26}\\
\hline M_{\mathrm{M}}(Z) \rightarrow M_{\text {д }}(Z-1),
\end{gather*}
\]

и энергия, выделяющаяся при \(K\)-захвате,


\begin{equation*}
Q=M_{\mathrm{M}}-M_{\text {д }}, \tag{8.27}
\end{equation*}


что совпадает с выражением (8.23) для энергии \(Q\) при \(\beta^{-}\)-распаде.\\
Следует отметить, что формулы (8.23), (8.25) и (8.27) определяют одновременно и условия энергетической возможности этих трех процессов: необходимо, чтобы выполнялось условие \(Q>0\).

Распределение электронов по энергиям. Общим свойством всех \(\beta\)-спектров является их плавность и наличие у каждого спектра предельной кинетической энергии \(K_{\text {макс }}\), на которой

\begin{figure}[h]
\begin{center}
  \includegraphics[alt={},max width=\textwidth]{a57b3a57-78b5-4292-b0ab-4bd9f2ffcf56-201_344_527_1056_68}
\captionsetup{labelformat=empty}
\caption{Рис. 8.7}
\end{center}
\end{figure}

По какой же причине возникают электроны с энергией \(K<K_{\text {макс }}\) ? Спектр регистрирует около \(1 / 3\) выделяемой при распаде энергии. Куда исчезают остальные \(2 / 3\) ? Не есть ли это нарушение закона сохранения энергии? Возникшую в свое время «проблему \(\beta\)-распада» решил Паули (1930), предположивший, что вместе с электроном испускается электрически нейтральная частица, неуловимая вследствие очень большой проникающей способности. Ее назвали нейтрино \(v\).

Тогда становится понятным, что энергия, выделяемая при распаде, распределяется между электроном и нейтрино в самых разных пропорциях, и мы получаем изображенный на рис. 8.7 спектр.

Имеется еще одно важное обстоятельство в пользу гипотезы о существовании нейтрино - это необходимость сохранения момента импульса в реакции распада. Дело в том, что отличительной чертой \(\beta\)-распада является превращение в ядре нейтрона в протон, и наоборот. Поэтому можно сказать, что \(\beta\)-распад есть не внутриядерный процесс, а внутринуклонный процесс. В связи с этим указанные выше три разновидности \(\beta\)-распада обусловлены следующими превращениями нуклонов в ядре*:

\[
\begin{array}{ll}
n \rightarrow p+e^{-}+v & \left(\beta^{-} \text {-распад }\right), \\
p \rightarrow n+e^{+}+v & \left(\beta^{+} \text {-распад }\right), \\
e^{-}+p \rightarrow n+v & (K \text {-захват }) .
\end{array}
\]

Известно, что спин нейтрона, протона и электрона одинаков и равен \(1 / 2\). Если бы, например, нейтрон распадался только как \(n \rightarrow p+e^{-}\), то суммарный спин возникающих частиц согласно квантовым законам сложения моментов был бы равен 1 либо 0 , что отличается от спина исходной частицы. Таким образом, участие в \(\beta\)-распаде еще одной частицы диктуется и законом сохранения момента, причем эта частица должна обладать спином \(1 / 2\) (или \(3 / 2\) ). Сейчас установлено, что спин нейтрино равен \(1 / 2\).

Наблюдать нейтрино непосредственно очень сложно. Это обусловлено тем, что их электрический заряд равен нулю, масса (если она есть) чрезвычайно мала, фантастически мало и эффективное сечение взаимодействия их с ядрами. Согласно теоретическим оценкам средняя длина свободного пробега нейтрино с энергией \(1 \mathrm{M} \ni \mathrm{B}\) в воде порядка \(10^{16}\) км (или 100 световых лет!). Это значительно превышает размеры звезд. Такие нейтрино свободно пронизывают Солнце, а тем более Землю.

\footnotetext{\begin{itemize}
  \item Пока нет необходимости уточнять, о какой именно «неуловимой» частице идет речь, мы будем пользоваться только термином «нейтрино». В следующей же главе мы увидим, что есть разные нейтрино и, кроме них, частицы, называемые антинейтрино.
\end{itemize}
}Чтобы зарегистрировать процесс захвата нейтрино, необходимо иметь огромные плотности потока их. Это стало возможным только после создания ядерных реакторов, которые и были использованы как мощные источники нейтрино.

Непосредственное экспериментальное доказательство существования нейтрино было получено в 1956 г.

Гамма-распад. Этот вид распада заключается в испускании возбужденным ядром при переходе его в нормальное состояние \(\gamma\)-квантов, энергия которых варьируется в пределах от 10 кэВ до 5 МэВ. Существенно, что спектр испускаемых \(\gamma\)-квантов дискретный, так как дискретны энергетические уровни самих ядер.

Свободный нуклон испускать \(\gamma\)-квант не может, ибо в противном случае было бы нарушено одновременное выполнение законов сохранения энергии и импульса (в этом полезно убедиться самостоятельно). Между тем такой процесс возможен и действительно происходит внутри ядра, поскольку испущенный (или поглощенный) \(\gamma\)-квант может обмениваться импульсом не только с порождающим его нуклоном, но и с остальными нуклонами ядра. Таким образом, в отличие от \(\beta\)-распада,

\begin{figure}[h]
\begin{center}
  \includegraphics[alt={},max width=\textwidth]{a57b3a57-78b5-4292-b0ab-4bd9f2ffcf56-203_267_330_996_68}
\captionsetup{labelformat=empty}
\caption{Рис. 8.8}
\end{center}
\end{figure}

\(\gamma\)-распад - процесс внутриядерный, а не внутринуклонный.

Возбужденные ядра образуются при \(\beta\)-распаде в случае, если распад материнского ядра X в основное состояние дочернего ядра Y запрещен. Тогда дочернее ядро Y оказывается в одном из возбужденных состояний, переход из которого в основное состояние и сопровождается испусканием \(\gamma\)-квантов (рис. 8.8).

Возбужденное ядро может перейти в основное состояние и другим путем, путем непосредственной передачи энергии возбуждения одному из атомных электронов, например, в \(K\)-оболочке. Этот процесс, конкурирующий с \(\beta\)-распадом, называют внутренней конверсией электронов.

Очевидно, что электроны внутренней конверсии моноэнергетичны. Это и позволяет отличить их от электронов, испускаемых при \(\beta\)-распаде, спектр которых, как мы знаем, непрерывный.

Внутренняя конверсия сопровождается рентгеновским излучением, возникающим при переходе электрона с вышележащих оболочек на место, освобожденное электроном внутренней конверсии.

Пример. Возбужденное ядро \({ }^{81} \mathrm{Se}\) с энергией возбуждения \(E^{*}=103\) кэВ переходит в основное состояние, испуская или \(\gamma\)-квант, или конверсионный электрон с \(K\)-оболочки атома. Энергия связи \(K\)-электрона \(E_{K}=12,7\) кэВ. Найдем скорость \(v\) отдачи ядра в обоих случаях.

В первом случае импульс ядра отдачи \(m v=\hbar \omega / c\). Здесь \(\hbar \omega \approx E^{*}\), поскольку энергия отдачи тяжелого ядра пренебрежимо мала. Тогда

\[
v=\frac{E^{*}}{m c^{2}} c=\frac{0,103}{81 \cdot 931,5} 3 \cdot 10^{8}=0,41 \kappa м / с .
\]

Во втором случае следует воспользоваться релятивистским соотношением (П.5) \(p c=\sqrt{K_{e}\left(K_{e}+2 m_{e} c^{2}\right)}\), где импульс ядра \(p=m v, K_{e}=E^{*}-E_{K}=90,3\) кэВ. Тогда

\[
v=c \frac{\sqrt{K_{e}\left(K_{e}+2 m_{e} c^{2}\right)}}{m c^{2}}=1,26 \mathrm{Km} / \mathrm{c} .
\]

\section*{§ 8.6. Эффект Мессбауэра}
Известно, что атомы наиболее интенсивно поглощают свет частоты, соответствующей переходу из основного состояния атома в ближайшее к нему возбужденное состояние. Это явление называют резонансным поглощением. Другими словами, фотоны, испущенные атомом при переходе из первого возбужденного состояния в основное, без всяких проблем поглощаются такими же атомами, поскольку их частоты практически совпадают.

Иначе обстоит дело в случае излучения \(\gamma\)-квантов ядрами. Энергия и импульс \(\gamma\)-кванта во много раз больше, чем у фотона видимого света, поэтому значительно больше и энергия отдачи. Представим себе два одинаковых первоначально покоящихся ядра, одно из которых находится в основном состоянии, другое - в возбужденном с энергией возбуждения \(E^{*}\). Переходя в основное состояние, возбужденное ядро испускает \(\gamma\)-квант с энер-

гией \(\hbar \omega\) и импульсом \(\hbar \omega / c\), удовлетворяющими законам сохранения:


\begin{equation*}
E^{*}=\hbar \omega+K, \quad \hbar \omega / c=p \tag{8.28}
\end{equation*}


где \(K\) - энергия отдачи ядра. Из этих уравнений следует, что


\begin{equation*}
K=\frac{p^{2}}{2 m}=\frac{(\hbar \omega)^{2}}{2 m c^{2}}, \tag{8.29}
\end{equation*}


здесь \(m\) - масса ядра.\\
Согласно первой из формул (8.28) энергия \(\gamma\)-кванта ћω сдвинута относительно энергии \(E^{*}\) ядерного перехода на величину \(K\) - энергию отдачи ядра. Поэтому \(\gamma\)-квант сможет поглотиться другим ядром только при условии, что сдвиг*


\begin{equation*}
K<\Gamma, \tag{8.30}
\end{equation*}


где \(\Gamma\) - ширина возбужденного уровня \(E^{*}\).\\
Выясним, насколько выполняется соотношение (8.30). Например, ядро \({ }^{57} \mathrm{Fe}\) при переходе из первого возбужденного состояния испускает \(\gamma\)-квант с энергией \(\hbar \omega \approx 14\) кэВ. При этом его энергия испытывает сдвиг на величину

\[
K=\frac{(\hbar \omega)^{2}}{2 m c^{2}}=\frac{\left(14 \cdot 10^{-3}\right)^{2}}{2 \cdot 57 \cdot 931,5}=2 \cdot 10^{-9} \mathrm{M} \ni \mathrm{~B}=2 \cdot 10^{-3} \text { эВ. }
\]

Ширина же Г первого возбужденного уровня, время жизни которого \(\tau \sim 10^{-7} \mathrm{c}\), согласно соотношению неопределенностей \(\Delta E \cdot \Delta t \sim \hbar\) равна


\begin{equation*}
\Gamma \approx \hbar / \tau \approx 10^{-8} \ni \text { В. } \tag{8.31}
\end{equation*}


Таким образом, сдвиг \(K\) не меньше Г, а наоборот, больше на пять порядков, что далеко перекрывает возможность резонансного поглощения.

\footnotetext{\begin{itemize}
  \item Точнее, надо было бы написать \(2 K<\Gamma\), поскольку ядро, находившееся в основном состоянии, тоже испытывает такую же отдачу \(K\) при поглощении \(\gamma\)-кванта. Но, как будет видно чуть ниже, это не существенно.
\end{itemize}
}И тем не менее явление резонансного поглощения \(\gamma\)-лучей было обнаружено Мессбауэром (1958) на изотопе \({ }^{191} \mathrm{Ir}\). Это оказалось возможным только с ядрами, входящими в состав кристалла. При этом существует вероятность испускания \(\gamma\)-кванта ядром с отдачей, которую воспринимает не ядро, а весь кристалл в целом, не меняя своего внутреннего состояния (т. е. без возбуждения колебаний решетки). Масса кристалла несопоставимо велика по сравнению с массой отдельного ядра, поэтому энергия отдачи кристалла практически равна нулю. В результате частота испущенного \(\gamma\)-кванта не смещается относительно резонансного значения, и этот \(\gamma\)-квант может быть поглощен другим таким же ядром, тоже входящим в состав кристалла.

В этом заключается суть эффекта Мессбауэра: испускание и поглощение \(\gamma\)-квантов без отдачи, т. е. резонансное. Этот эффект удается наблюдать только при очень низких температурах, но иногда и при комнатных температурах (в случае с Fe).

Эффект Мессбауэра наблюдают так. Источник \(\gamma\)-излучения приводят в движение с небольшой скоростью \(v\) навстречу поглотителю или в обратном направлении. При этом измеряют скорость счета \(\gamma\)-квантов за поглотителем. Если \(v \neq 0\), то резонанс нарушается: линии испускания и поглощения сдвигаются относительно друг друга за счет эффекта Доплера. При \(v=0\) наблюдается резонансное поглощение \(\gamma\)-квантов, что показано на

\begin{figure}[h]
\begin{center}
  \includegraphics[alt={},max width=\textwidth]{a57b3a57-78b5-4292-b0ab-4bd9f2ffcf56-206_339_519_964_645}
\captionsetup{labelformat=empty}
\caption{Рис. 8.9}
\end{center}
\end{figure}

Благодаря очень малому отношению ширины \(\Gamma\) возбужденных ядерных уровней к энергии возбуждения \(E^{*}\left(\Gamma / E^{*} \sim 10^{-12} \div 10^{-16}\right)\) эффект Мессбауэра дает уникальный метод измерения ничтожных изменений энергии, которые не могут быть измерены никаким другим методом.

В частности, с помощью этого эффекта удалось обнаружить в лабораторных условиях гравитационное смещение спектральных линий (уменьшение частоты фотона при удалении его от источника тяготения). Для этого надо было измерить отно-

сительное изменение энергии фотона порядка \(10^{-15}\) на базе около \(20 м\), что впервые и проделали Паунд и Ребка (1960).

Рассмотрим этот вопрос более подробно. Найдем относительное уменьшение частоты \(\gamma\)-кванта (гравитационное смещение) при удалении его от поверхности Земли на \(l \approx 20 \mathrm{~m}\).

Считая, что \(\gamma\)-квант ведет себя подобно частице, обладающей гравитационной массой \(\hbar \omega / c^{2}\), запишем, что приращение энергии \(\gamma\)-кванта на пути \(\mathrm{d} r\) равно работе гравитационной силы \(F_{r}\) на этом пути:


\begin{equation*}
\mathrm{d}(\hbar \omega)=F_{r} \mathrm{~d} r=-\gamma \frac{\left(\hbar \omega / c^{2}\right) M_{3}}{r^{2}} \mathrm{~d} r, \tag{1}
\end{equation*}


где \(\gamma\) - гравитационная постоянная, \(M_{3}\) - масса Земли; знак минус связан с тем, что проекция силы \(F_{r}<0\). Разделив в (1) переменные \(\omega\) и \(r\), получим:


\begin{equation*}
\frac{\mathrm{d} \omega}{\omega}=-\frac{\gamma M_{\mathrm{E}}}{c^{2}} \frac{\mathrm{~d} r}{r^{2}} . \tag{2}
\end{equation*}


Проинтегрируем это уравнение по частоте от \(\omega_{0}\) до \(\omega\) и по \(r\) от радиуса Земли \(R\) до \(R+l\) :


\begin{equation*}
\ln \frac{\omega}{\omega_{0}}=\frac{\gamma M_{\mathrm{E}}}{c^{2}}\left(\frac{1}{R+l}-\frac{1}{r}\right)=-\frac{\gamma M_{\mathrm{E}}}{R(R+l)} \frac{l}{c^{2}} \approx-g \frac{l}{c^{2}}, \tag{3}
\end{equation*}


где \(g\) - напряженность гравитационного поля ( \(g=\gamma M_{3} / R^{2}\) ). Здесь учтено, что \(l \ll R\). Из (3) следует:

\[
\omega=\omega_{0} \mathrm{e}^{-g l / c^{2}} \approx \omega_{0}\left(1-g l / c^{2}\right)
\]

поскольку \(g l / c^{2} \ll 1\). Видно, что частота \(\gamma\)-кванта с удалением от поверхности Земли уменьшается.

Искомое относительное изменение частоты \(\gamma\)-кванта

\[
\frac{\Delta \omega}{\omega_{0}}=\frac{\omega_{0}-\omega}{\omega_{0}} \approx \frac{g l}{c^{2}}=\frac{9,8 \cdot 20}{\left(3 \cdot 10^{8}\right)^{2}} \approx 2 \cdot 10^{-15} .
\]

Несмотря на чрезвычайную малость этого смещения (сдвиг составлял сотую часть ширины линии), его удалось измерить с достаточной степенью точности и тем самым экспериментально в лабораторных условиях подтвердить наличие гравитационного (красного) смещения.

\section*{§ 8.7. Ядерные реакции}
Ядерная реакция - это процесс сильного взаимодействия атомного ядра с элементарной частицей или с другим ядром, процесс, сопровождающийся преобразованием ядер. Это взаимодействие возникает благодаря действию ядерных сил при сближении частиц до расстояний порядка \(10^{-13} \mathrm{~cm}\).

Отметим, что именно ядерные реакции дают наиболее широкую информацию о свойствах ядер. Поэтому изучение ядерных реакций является самой главной задачей ядерной физики.

Наиболее распространенным типом ядерной реакции является взаимодействие частицы \(a\) с ядром \(X\), в результате чего образуется частица \(b\) и ядро \(Y\). Это записывают символически так:

\[
a+X \rightarrow Y+b
\]

или в сокращенном виде


\begin{equation*}
X(a, b) Y . \tag{8.32}
\end{equation*}


Роль частиц \(a\) и \(b\) чаще всего выполняют нейтрон \(n\), протон \(p\), дейтрон \(d\), α-частица и \(\gamma\)-квант. Говоря, что (8.32) есть ядерная реакция, мы подразумеваем, что частица \(b\) не тождественна частице \(a\). В противном случае этот процесс называют рассеянием.

Частицы, рождающиеся в результате ядерной реакции (8.32), могут быть не только \(b\) и \(Y\), но вместе с ними и другие \(b^{\prime}, Y^{\prime}\). В этом случае говорят, что ядерная реакция имеет несколько каналов, причем различным каналам соответствуют различные вероятности.

Выход ядерной реакции. В ядерной физике вероятность взаимодействия принято характеризовать с помощью эффективного сечения σ. Наглядно σ интерпретируется как площадь сечения ядра \(X\), попадая в которую налетающая частица вызывает реакцию.

Если мишень из ядер \(X\) настолько тонкая, что ядра не перекрывают друг друга, то относительная доля площади \(S\) мишени, перекрытая ядрами \(X\), равна \(\sigma n S / S=\sigma n\), где \(n\) - число ядер на единицу площади мишени. И мы можем сказать, что относительное число \(\Delta N / N\) частиц \(a\), вызвавших ядерную реак-

цию (или, другими словами, вероятность \(P\), что частица \(a\) вызовет ядерную реакцию), определяется как


\begin{equation*}
\frac{\Delta N}{N}=P=\sigma n . \tag{8.33}
\end{equation*}


Эту величину называют выходом ядерной реакции


\begin{equation*}
w=\Delta N / N . \tag{8.34}
\end{equation*}


Именно \(w\) является непосредственно измеряемой величиной. А зная \(w\) и \(n\), можно найти и \(\sigma\) с помощью (8.33).

Заметим, что если мишень не тонкая, то выражение для \(w\) усложняется:

\[
w=\frac{\Delta N}{N}=1-\mathrm{e}^{-\sigma n} .
\]

Мы не будем углубляться в дальнейшие детали, но на одно обстоятельство следует обратить внимание. Геометрическое сечение ядра имеет порядок \(10^{-24} \mathrm{~cm}^{2}\). Эту величину принимают за единицу ядерных сечений и называют барном (б),

\[
1 \text { барн }=10^{-24} \mathrm{~cm}^{2} .
\]

\begin{figure}[h]
\begin{center}
  \includegraphics[alt={},max width=\textwidth]{a57b3a57-78b5-4292-b0ab-4bd9f2ffcf56-209_423_483_1028_68}
\captionsetup{labelformat=empty}
\caption{Рис. 8.10}
\end{center}
\end{figure}

Из-за волновых и квантовых свойств частиц сечение σ может оказаться в тех или иных случаях как меньше геометрического сечения, так и больше (причем иногда весьма значительно). Это зависит как от самих взаимодействующих частиц, так и от кинетической энергии налетающей частицы \(a\). В качестве примера на рис. 8.10 приведена кривая зависимости сечения захвата нейтрона ядром \({ }^{238} \mathrm{U}\) от кинетической энергии \(K\) нейтрона.

Типы ядерных реакций. Установлено, что реакции, вызываемые не очень быстрыми частицами, протекают в два этапа. Первый этап - это захват налетающей частицы \(a\) ядром \(X\) с образованием составного (или промежуточного) ядра. При этом энергия частицы \(a\) быстро перераспределяется между всеми

нуклонами ядра, и составное ядро оказывается в возбужденном состоянии. В этом состоянии ядро пребывает до тех пор, пока в результате внутренних флуктуаций на одной из частиц (которая может состоять и из нескольких нуклонов) не сконцентрируется энергия, достаточная для вылета ее из ядра.

Такой механизм протекания ядерной реакции был предложен Н. Бором (1936) и впоследствии подтвержден экспериментально. Эти реакции иногда записывают с указанием составного ядра \(C\), как например


\begin{equation*}
a+X \rightarrow C^{*} \rightarrow Y+b, \tag{8.35}
\end{equation*}


где звездочка у \(C\) указывает на то, что ядро \(C^{*}\) возникает в возбужденном состоянии.

Составное ядро \(C^{*}\) существует достаточно долго - по сравнению с «ядерным временем», т. е. временем пролета нуклона с энергией порядка \(1 \mathrm{M} \ni \mathrm{B}\left(v \approx 10^{9} \mathrm{~cm} / \mathrm{c}\right)\) расстояния, равного диаметру ядра. Ядерное время \(\tau_{\text {я }} \approx 10^{-21}\) с. Время же жизни составного ядра в возбужденном состоянии \(\sim 10^{-14}\) с. Т. е. в ядерном масштабе составное ядро живет действительно очень долго. За это время все следы истории его образования исчезают. Поэтому распад составного ядра - вторая стадия реакции - протекает независимо от способа образования составного ядра.

Реакции, вызываемые быстрыми частицами с энергией, превышающей десятки МэВ, протекают без образования составного ядра. И ядерная реакция, как правило, является прямой. В этом случае налетающая частица непосредственно передает свою энергию какой-то частице внутри ядра, например, одному нуклону, дейтрону, \(\alpha\)-частице и т. д., в результате чего эта частица вылетает из ядра.

Типичная реакция прямого взаимодействия - это реакция срыва, когда налетающей частицей является, например, дейтрон. При попадании одного из нуклонов дейтрона в область действия ядерных сил он будет захвачен ядром, в то время как другой нуклон дейтрона окажется вне зоны действия ядерных сил и пролетит мимо ядра. Символически реакцию срыва записывают как ( \(d, n\) ) или ( \(d, p\) ).

При бомбардировке ядер сильно взаимодействующими частицами с очень высокой энергией (от нескольких сотен МэВ и

выше) ядра могут «взрываться», распадаясь на множество мелких осколков. При регистрации такие взрывы оставляют след в виде многолучевых звезд.

Энергия реакции. Принято говорить, что ядерные реакции могут происходить как с выделением, так и с поглощением энергии. Это надо понимать так. Пусть \(E_{0}\) и \(E_{0}^{\prime}\) - суммы энергий покоя исходных частиц и продуктов реакции. Полная энергия в реакции сохраняется, т. е.


\begin{equation*}
E_{0}+K=E_{0}^{\prime}+K^{\prime}, \tag{8.36}
\end{equation*}


где \(K\) и \(K^{\prime}\) - суммарные кинетические энергии исходных частиц и продуктов реакции. Из этого равенства следует, что убыль суммарной энергии покоя ( \(E_{0}-E_{0}^{\prime}\) ) равна приращению суммарной кинетической энергии ( \(K^{\prime}-K\) ) и наоборот. Эти величины и называют энергией реакции \(Q\) :


\begin{equation*}
Q=E_{0}-E_{0}^{\prime}=K^{\prime}-K . \tag{8.37}
\end{equation*}


Реакции с \(Q>0\) называют экзоэнергетическими (с выделением энергии, кинетической), реакции же с \(Q<0\) - эндоэнергетическими.

Часто ядерную реакцию с учетом \(Q\) записывают так:


\begin{equation*}
A(a, b) B+Q . \tag{8.38}
\end{equation*}


Для расчетов формулу (8.37) удобнее представить в другом виде - через массы или, еще лучше, - через дефекты масс \(\Delta\) нуклидов (если пользоваться таблицами). Тогда

\[
Q=\left\{\begin{array}{l}
\left(m_{a}+m_{A}\right)-\left(m_{b}+m_{B}\right),  \tag{8.39}\\
\left(\Delta_{a}+\Delta_{A}\right)-\left(\Delta_{b}+\Delta_{\mathrm{B}}\right) .
\end{array}\right.
\]

Пример. Найдем кинетическую энергию \(\alpha\)-частицы, образующейся в реакции

\[
{ }^{10} \mathrm{~B}(n, \alpha){ }^{7} \mathrm{Li}
\]

при взаимодействии весьма медленных нейтронов с покоящимися ядрами нуклида \({ }^{10} \mathrm{~B}\), если энергия реакции \(Q=+2,8\) МэВ.

Пренебрегая по условию энергией и импульсом нейтрона, запишем

\[
Q=K_{\alpha}+K_{\mathrm{Li}}, \quad p_{\alpha}=p_{\mathrm{Li}} .
\]

Из второго равенства следует, что \(m_{\alpha} K_{\alpha}=m_{\mathrm{Li}} K_{\mathrm{Li}}\). Тогда

\[
Q=K_{\alpha}+\frac{m_{\alpha}}{m_{\mathrm{Li}}} K_{\alpha} \quad \text { и } \quad K_{\alpha}=\frac{Q}{1+m_{\alpha} / m_{\mathrm{Li}}}=\frac{7}{11} Q \approx 1,8 \mathrm{M} \vartheta \text { В. }
\]

Энергетическая схема ядерной реакции. Подавляющее большинство экспериментальных исследований выполняется в лабораторной системе отсчета (Л -системе), где мишень покоится. В теоретических же расчетах удобнее система иентра масс или система центра инериии (Ц-система), в которой суммарный импульс сталкивающихся (и образующихся) частиц равен нулю. Результаты, полученные в Ц-системе, при необходимости можно пересчитать в Л-систему.

Отметим попутно, что в экспериментальных исследованиях под энергией частицы всегда понимают ее кинетическую энергию \(K\); для безмассовых частиц \(K\) совпадает с энергией частицы. Этому будем следовать и мы в дальнейшем (для краткости).

Приведем без вывода основные соотношения, определяющие для системы из двух частиц с массами \(m\) и \(M\) импульс \(\tilde{p}\) каждой частицы и суммарную кинетическую энергию \(\tilde{K}\) обеих частиц в Ц-системе:


\begin{equation*}
\tilde{p}=\mu v_{\mathrm{oTH}}, \quad \tilde{K}=\frac{\tilde{p}^{2}}{2 \mu}, \quad \mu=\frac{m M}{m+M}, \tag{8.40}
\end{equation*}


где \(\mu\) - приведенная масса системы, \(v_{\text {отн }}\) - относительная скорость частиц \(\left|\mathbf{v}_{m}-\mathbf{v}_{M}\right|\). Заметим, что эта скорость одинакова в \(Л\) - и Ц-системах.

Чаще всего мы будем иметь дело с ядерной реакцией

\[
M\left(m, m^{\prime}\right) M^{\prime}+Q,
\]

где \(m\) - масса налетающей частицы, \(M\) - масса покоящегося ядра мишени. В этом случае связь между \(\tilde{K}\) и энергией \(K_{m}\) налетающей частицы определяется согласно (8.40) как


\begin{equation*}
\tilde{K}=\frac{\mu v_{\text {отн }}}{2}=\frac{\mu}{m} K_{m} . \tag{8.41}
\end{equation*}


Из условия \(\tilde{K}^{\prime}-\tilde{K}=Q\) следует, что \(\tilde{p}^{\prime 2} / 2 \mu^{\prime}-(\mu / m) K_{m}=Q\), где \(\mu^{\prime}\) - приведенная масса продуктов реакции. Отсюда


\begin{equation*}
\tilde{p}^{\prime}=\sqrt{2 \mu^{\prime}\left(\frac{\mu}{m} K_{m}+Q\right)} . \tag{8.42}
\end{equation*}


Это значит, что зная энергию \(K_{m}\) налетающей частицы и энергию реакции \(Q\), мы можем определить импульс \(\tilde{p}^{\prime}\) каждой частицы, возникшей после реакции, а также их суммарную кинетическую энергию \(\tilde{K}^{\prime}\). И наоборот, зная \(\tilde{p}^{\prime}\) и \(Q\), можно определить \(K_{m}\).

Из механики известно, что кинетическая энергия \(K\) системы частиц может быть представлена как


\begin{equation*}
K=\tilde{K}+K_{c}, \tag{8.43}
\end{equation*}


где \(\tilde{K}\) - кинетическая энергия этой системы частиц в Ц-системе, а \(K_{C}\) - кинетическая энергия, связанная с движением системы как целого, т. е. с движением центра масс \(C\) системы. Энергия \(K_{C}\) сохраняется и в реакции не участвует, поэтому формулу (8.37) мы можем представить в виде


\begin{equation*}
Q=K^{\prime}-K=\tilde{K}^{\prime}-\tilde{K} \tag{8.44}
\end{equation*}


Изобразим для наглядности схему ядерной реакции в энергетической шкале в Ц-системе для двух случаев:

\begin{enumerate}
  \item \(Q>0\), реакция экзоэнергетическая (рис. 8.11),
  \item \(Q<0\), реакция эндоэнергетическая (рис. 8.12).
\end{enumerate}

\begin{figure}[h]
\begin{center}
  \includegraphics[alt={},max width=\textwidth]{a57b3a57-78b5-4292-b0ab-4bd9f2ffcf56-213_332_416_1280_137}
\captionsetup{labelformat=empty}
\caption{Рис. 8.11}
\end{center}
\end{figure}

\begin{figure}[h]
\begin{center}
  \includegraphics[alt={},max width=\textwidth]{a57b3a57-78b5-4292-b0ab-4bd9f2ffcf56-213_330_415_1282_677}
\captionsetup{labelformat=empty}
\caption{Рис. 8.12}
\end{center}
\end{figure}

Из этих рисунков видно, что, во-первых, всякая реакция, обратная экзоэнергетической, будет эндоэнергетической. Приме-

ром может служить реакция

\[
p+{ }^{7} \mathrm{Li} \rightarrow \alpha+\alpha+17,3 \text { МэВ, }
\]

а обратная реакция

\[
\alpha+\alpha \rightarrow p+{ }^{7} \mathrm{Li}-17,3 \text { МэВ. }
\]

Во-вторых, экзоэнергетическая реакция может идти при сколь угодно малой энергии сталкивающихся частиц (если нет ка-ких-либо запретов на ту или иную реакцию). Эндоэнергетическая же реакция может идти только тогда, когда суммарная энергия \(\tilde{K}\) сталкивающихся частиц (в Ц-системе) превосходит некоторое минимальное значение, которое называют порогом реакции.

Порог реакции. Существенно отметить, что порог реакции, т. е. минимальная энергия \(K_{\text {пор }}\) налетающей частицы измеряется всегда в Л -системе, где ядра мишени покоятся.

Найдем выражение для \(K_{\text {пор }}\) налетающей частицы. Этот вопрос наиболее просто решается в Ц-системе, где ясно (см. рис. 8.12), что суммарная кинетическая энергия \(\tilde{K}\) частиц до столкновения во всяком случае должна быть не меньше \(|Q|\), т. е. \(\tilde{K} \geqslant|Q|\).

Отсюда следует, что существует минимальное значение \(\tilde{K}_{\text {мин }}=|Q|\), при котором кинетическая энергия системы целиком пойдет на создание покоящихся в Ц-системе частиц \(m^{\prime} u M^{\prime}\).

Теперь перейдем в Л -систему. Так как в Ц-системе при \(\tilde{K}_{\text {мин }}\) образовавшиеся частицы \(m^{\prime} u M^{\prime}\) покоятся, то это значит, что в \(Л\)-системе при соответствующем значении пороговой энергии \(K_{\text {пор }}\) налетающей частицы обе частицы, \(m^{\prime} u M^{\prime}\), после образования будут двигаться как единое целое, причем с суммарным импульсом, равным импульсу \(p\) налетающей частицы, и кинетической энергией \(p^{2} / 2(m+M)\). Поэтому

\[
K_{\text {пор }}=|Q|+p^{2} / 2(m+M) .
\]

А так как \(K_{\text {пор }}=p^{2} / 2 m\), то, исключив \(p^{2}\) из этих двух уравнений, получим


\begin{equation*}
K_{\text {пор }}=\frac{m+M}{M}|Q| . \tag{8.45}
\end{equation*}


Это и есть пороговая кинетическая энергия налетающей частицы \(m\), начиная с которой данная эндоэнергетическая реакция становится энергетически возможной.

В ядерной физике обычно можно ограничиться нерелятивистской формулой (8.45). Но в процессах с участием релятивистских частиц следует исходить из инвариантности выражения (П.3):


\begin{equation*}
E^{2}-p^{2} c^{2}=m^{2} c^{4}=\mathrm{inv} \tag{8.46}
\end{equation*}


где \(E\) - полная энергия системы.\\
Например, в случае расщепления атомного ядра массы \(m\) под действием \(\gamma\)-кванта при пороговом значении его энергии \(\varepsilon_{\text {пор }}\) мы имеем

\[
\left(\varepsilon_{\text {пор }}+m c^{2}\right)^{2}-\varepsilon_{\text {пор }}=\left(m_{1}+m_{2}+\ldots\right)^{2} c^{4}
\]

Здесь левая часть равенства записана в Л - системе, а правая - в Ц-системе, где образовавшиеся частицы покоятся (при \(\varepsilon_{\text {пор }}\) ). Из этого равенства получим


\begin{align*}
& \varepsilon_{\text {пор }}=\frac{\left(m_{1}+m_{2}+\ldots\right)^{2}-m^{2}}{2 m} c^{2}= \\
& =\frac{\left(m_{1}+m_{2}+\ldots-m\right)\left(m_{1}+m_{2}+\ldots+m\right)}{2 m} c^{2} . \tag{8.47}
\end{align*}


Из последних двух скобок в числителе первая представляет собой энергию эндоэнергетической реакции \(|Q|\), а вторая равна \(|Q|+2 m c^{2}\). В результате (8.47) примет вид


\begin{equation*}
\varepsilon_{\text {пор }}=|Q|\left(1+\frac{|Q|}{2 m c^{2}}\right) . \tag{8.48}
\end{equation*}


Это и есть выражение для пороговой энергии \(\gamma\)-кванта в случае эндоэнергетической реакции, энергия \(Q\) которой известна.

Уровни возбуждения ядра. Приведем в заключение энергетическую схему ядерной реакции, проходящей через составное ядро \(M^{*}\) :

\[
m+M \rightarrow M^{*} \rightarrow m^{\prime}+M^{\prime}+Q .
\]

Эта схема показана на рис. 8.13. Здесь \(m+M\) и \(m^{\prime}+M^{\prime}-\) суммы масс частиц до и после реакции, \(\tilde{K}\) и \(\tilde{K}^{\prime}\) - суммарные энер-

гии частиц в Ц-системе, \(E^{*}\) энергия возбуждения составного ядра, \(Q\) - энергия реакции. На рисунке показаны также уровни составного ядра, они обозначены цифрами 1, 2, 3, ...

Варьируя энергию налетающей частицы, т. е. \(\widetilde{K}\) в Ц-системе, можно обнаружить, что выход \(w\) ядерной реакции вблизи каждого уровня плавно меняется, проходя через максимум (рис. 8.14). Из этого следует, что сами энергетические уровни «размыты». Пусть ширина кривой \(w(E)\) на половине «высоты» равна Г. Эта величина представляет собой неопределенность энергии соответствующего уровня. Из соотношения неопределенностей для энергии и времени

\begin{figure}[h]
\begin{center}
  \includegraphics[alt={},max width=\textwidth]{a57b3a57-78b5-4292-b0ab-4bd9f2ffcf56-216_487_481_141_679}
\captionsetup{labelformat=empty}
\caption{Рис. 8.13}
\end{center}
\end{figure}

\begin{figure}[h]
\begin{center}
  \includegraphics[alt={},max width=\textwidth]{a57b3a57-78b5-4292-b0ab-4bd9f2ffcf56-216_235_382_697_782}
\captionsetup{labelformat=empty}
\caption{Рис. 8.14}
\end{center}
\end{figure}

\[
\Gamma \cdot \tau \geqslant \hbar
\]

можно оценить время жизни \(\tau\) данного конкретного уровня, т. е. время пребывания составного ядра в данном возбужденном состоянии. Разным уровням соответствуют, вообще говоря, разные значения \(\Gamma\) и \(\tau\).

Эту главу закончим рассмотрением примера на нахождение энергетических уровней ядра.

Пример. При облучении мишени из углерода дейтронами возбуждается ядерная реакция

\[
{ }^{13} \mathrm{C}(d, n){ }^{14} \mathrm{~N},
\]

выход \(w\) которой имеет максимумы при следующих значениях энергии \(K_{d}\) дейтронов: \(0,60,0,90,1,55\) и 1,80 МэВ. Найдем энергии \(E^{*}\) соответствующих

\begin{figure}[h]
\begin{center}
  \includegraphics[alt={},max width=\textwidth]{a57b3a57-78b5-4292-b0ab-4bd9f2ffcf56-216_352_337_1317_830}
\captionsetup{labelformat=empty}
\caption{Рис. 8.15}
\end{center}
\end{figure}

уровней составного ядра, через которые идет данная реакция, если энергия связи дейтрона в со-

ставном ядре \({ }^{15} \mathrm{~N}\) равна \(E_{\text {св }}=16,16\) МэВ.\\
Из рис. 8.15 видно, что


\begin{equation*}
E^{*}=E_{\mathrm{cB}}+\tilde{K}=E_{\mathrm{cB}}+\frac{\mu}{m_{d}} K_{d} \tag{1}
\end{equation*}


где учтено, что суммарная энергия \(\widetilde{K}\) исходных частиц (С и \(d\) ) определяется формулой (8.41). В результате получим:

\[
E^{*}=E_{\text {св }}+\frac{m_{\mathrm{C}}}{m_{\mathrm{C}}+m_{d}} K_{d}=E_{\text {св }}+\frac{13}{15} K_{d}
\]

При указанных значениях \(K_{d}\) получим соответственно 16,68 , \(16,94,17,50\) и 17,72 МэВ.

\section*{Задачи}
8.1. Энергия связи ядра. Найти энергию связи ядра, которое имеет одинаковое число протонов и нейтронов, а радиус в \(\eta=1,5\) раза меньший радиуса ядра \({ }^{27} \mathrm{Al}\).\\
Решение. Сначала найдем массовое число \(A\) искомого ядра. Согласно формуле (8.3) \(A=\left(r / r_{\mathrm{Al}}\right)^{3} A_{\mathrm{Al}}=27 / \eta^{3}=8\). Это отвечает ядру \({ }^{8}\) Ве. Его энергия связи в соответствии с (8.8) и табл. 8.1 равна

\[
\begin{gathered}
E_{\mathrm{cb}}=4\left(\Delta_{\mathrm{H}}+\Delta_{n}\right)-\Delta_{\mathrm{Be}}=4(0,007825+0,008665)-0,005308= \\
=0,060652 \text { а.е.м. }=56,5 \text { МэВ. }
\end{gathered}
\]

8.2. Закон радиоактивного распада. Радионуклид \(X\) образуется с постоянной скоростью так, что ежесекундно возникает \(q\) радиоактивных ядер. Постоянная распада этих ядер равна \(\lambda\). Считая, что в момент \(t=0\) число данных ядер \(N(0)=0\), найти закон накопления их со временем, т. е. \(N(t)\).\\
P е ш е н и е. За промежуток времени \(\mathrm{d} t\) приращение \(\mathrm{d} N\) числа ядер нуклида \(X\) определяется уравнением

\[
\mathrm{d} N=q \mathrm{~d} t-\lambda N \mathrm{~d} t=(q-\lambda N) \mathrm{d} t
\]

Приведем это выражение к виду, удобному для интегрирования:

\[
\frac{\mathrm{d} N}{q-\lambda N}=\mathrm{d} t
\]

Интегрирование последнего уравнения по \(N\) и \(t\) с учетом начального условия \(N(0)=0\) дает

\[
-\frac{1}{\lambda} \ln \frac{q-\lambda N}{q}=t .
\]

Потенцируя, получим в результате

\[
N(t)=\frac{q}{\lambda}\left(1-\mathrm{e}^{-\lambda t}\right) .
\]

Видно, что с течением времени число ядер \(X\) асимптотически приближается к \(N_{m}=q / \lambda\).\\
8.3. При радиоактивном распаде ядер нуклида \(X_{1}\) образуется радионуклид \(X_{2}\). Их постоянные распада равны \(\lambda_{1}\) и \(\lambda_{2}\). Полагая, что в момент \(t=0\) препарат содержал только нуклид \(X_{1}\) в количестве \(N_{10}\), определить:\\
a) количество ядер нуклида \(X_{2}\) как функцию времени, \(N_{2}(t)\);\\
б) момент \(t_{m}\), когда количество ядер нуклида \(X_{2}\) достигает максимума.

Решение. а) В данном случае изменения во времени количеств \(N_{1}\) и \(N_{2}\) ядер обоих радионуклидов будут описываться следующими уравнениями:


\begin{equation*}
-\mathrm{d} N_{1} / \mathrm{d} t=\lambda_{1} N_{1}, \quad \mathrm{~d} N_{2} / \mathrm{d} t=\lambda_{1} N_{1}-\lambda_{2} N_{2} . \tag{1}
\end{equation*}


Первое из этих уравнений совпадает с (8.13) и имеет решение (8.14). т. е. \(N_{1}=N_{10} \mathrm{e}^{-\lambda_{1} t}\). Второе же уравнение описывает приращение в единицу времени количества ядер радионуклида \(X_{2}\). Оно происходит за счет распада ядер \(X_{1}\) в количестве \(\lambda_{1} N_{1}\) и убыли дочерних ядер из-за их распада ( \(-\lambda_{2} N_{2}\) ). Перепишем второе уравнение из (1) в виде


\begin{equation*}
\dot{N}_{2}+\lambda_{2} N_{2}=\lambda_{1} N_{10} \mathrm{e}^{-\lambda_{1} t} . \tag{2}
\end{equation*}


Решение этого неоднородного уравнения ищем как сумму решения однородного уравнения и частного решения неоднородного:


\begin{equation*}
N_{2}(t)=A \mathrm{e}^{-\lambda_{2} t}+B \mathrm{e}^{-\lambda_{1} t} . \tag{3}
\end{equation*}


Из начального условия \(N_{2}(0)=0\) находим \(0=A+B\), поэтому (3) можно переписать в виде


\begin{equation*}
N_{2}(t)=A\left(\mathrm{e}^{-\lambda_{2} t}-\mathrm{e}^{-\lambda_{1} t}\right) . \tag{4}
\end{equation*}


После подстановки (4) в (2) найдем: \(A=N_{10} \lambda_{1} /\left(\lambda_{1}-\lambda_{2}\right)\). Тогда (4) примет вид

\[
N_{2}(t)=N_{10} \frac{\lambda_{1}}{\lambda_{2}-\lambda_{1}}\left(\mathrm{e}^{-\lambda_{1} t}-\mathrm{e}^{-\lambda_{2} t}\right) .
\]

б) Взяв производную \(\mathrm{d} N_{2} / \mathrm{d} t\) и приравняв ее нулю, найдем:

\[
t_{m}=\frac{\ln \left(\lambda_{2} / \lambda_{1}\right)}{\lambda_{2}-\lambda_{1}}
\]

8.4. Альфа-распад. Распад покоящихся ядер \({ }^{210} \mathrm{Po}\) происходит из основ-\\
\includegraphics[max width=\textwidth, alt={}, center]{a57b3a57-78b5-4292-b0ab-4bd9f2ffcf56-219_322_163_563_139}\\
ного состояния и сопровождается испусканием двух групп \(\alpha\)-частиц: основной с энергией \(K_{\alpha}=5,30 \mathrm{M}\) ЭВ и слабой (по интенсивности) с энергией \(K_{\alpha}^{\prime}=4,50\) МэВ. Найти энергию \(\alpha\)-распада этих ядер и энергию \(\gamma\)-квантов, испускаемых дочерними ядрами.

Решение. Из условия следует, что дочерние ядра возникают не только в основном состоянии, но и в возбужденном (рис. 8.16).

Рис. 8.16 Из сохранения импульса имеем


\begin{equation*}
p_{\alpha}=p_{\text {д }}, \quad \text { или } \quad m_{\alpha} K_{\alpha}=m_{\text {д }} K_{\text {д }}, \tag{1}
\end{equation*}


где индексом «д» отмечено дочернее ядро ( \({ }^{206} \mathrm{~Pb}\) ). Энергия же \(\alpha\)-распада с учетом (1) равна


\begin{equation*}
Q=K_{\alpha}+K_{\text {д }}=K_{\alpha}\left(1+\frac{m_{\alpha}}{m_{\Gamma}}\right) \tag{2}
\end{equation*}


Аналогичное выражение запишем для \(Q^{\prime}\), когда дочернее ядро возникает в возбужденном состоянии:


\begin{equation*}
Q^{\prime}=K_{\alpha}^{\prime}\left(1+\frac{m_{\alpha}}{m_{\Gamma}}\right) . \tag{3}
\end{equation*}


Из формул (2) и (3) получим

\[
\hbar \omega=Q-Q^{\prime}=\left(K_{\alpha}-K_{\alpha}^{\prime}\right)\left(1+\frac{m_{\alpha}}{m_{\Gamma}}\right)=0,80 \frac{210}{206}=0,815 \text { МэВ. }
\]

8.5. Бета-распад. Неподвижное ядро \({ }^{6} \mathrm{He}\) испытывает \(\beta\)-распад, в результате которого дочернее ядро оказалось непосредственно в основном состоянии. Энергия распада \(Q=3,50\) МэВ. Под каким углом к

направлению вылета электрона испущено нейтрино, если электрон с энергией \(K_{e}=0,60\) МэВ вылетел под прямым углом к направлению движения ядра отдачи?

Решение. Сначала изобразим треугольник импульсов (рис. 8.17), где \(p_{\text {д }}\) - импульс отдачи дочернего ядра. Из рисунка видно, что


\begin{equation*}
\cos \alpha=p_{e} / p_{v} \tag{1}
\end{equation*}


Согласно (П.5) \(p_{e} c=\sqrt{K_{e}\left(K_{e}+2 m_{e} c^{2}\right)}\) и

\begin{figure}[h]
\begin{center}
  \includegraphics[alt={},max width=\textwidth]{a57b3a57-78b5-4292-b0ab-4bd9f2ffcf56-220_396_218_141_944}
\captionsetup{labelformat=empty}
\caption{Рис. 8.17}
\end{center}
\end{figure}

\(p_{v} c=\varepsilon_{v}=Q-K_{e}-K_{\text {я }}\).\\
В последнем равенстве кинетической энергией ядра отдачи можно пренебречь. В самом деле, из соотношения \(K=p^{2} / 2 m\), принимая во внимание, что импульсы всех трех частиц по порядку величины одинаковы, а масса ядра отдачи значительно превосходит массу электрона, следует: \(K_{\text {я }} \ll K_{e}\).

Таким образом,

\[
\cos \alpha=\frac{\sqrt{K_{e}\left(K_{e}+2 m_{e} c^{2}\right)}}{Q-K_{e}}=0,34 .
\]

Отсюда \(\alpha=70^{\circ}\) и \(\theta=110^{\circ}\).\\
8.6. Эффективное сечение реакции. Какова должна быть толщина кадмиевой пластинки, чтобы поток тепловых нейтронов при прохождении через нее уменьшался в \(\eta=100\) раз? Сечение поглощения нейтрона ядром атома кадмия \(\sigma_{a}=2,54\) кб, плотность кадмия \(\rho=8,65 г / \mathrm{cm}^{3}\).

Решении е. Выделим мысленно бесконечно тонкий плоский слой кадмия, перпендикулярный потоку нейтронов (рис. 8.18). Пусть на 1 см \(^{2}\) этого слоя ежесекундно падает \(N\) нейтронов.

\begin{figure}[h]
\begin{center}
  \includegraphics[alt={},max width=\textwidth]{a57b3a57-78b5-4292-b0ab-4bd9f2ffcf56-220_269_237_1224_919}
\captionsetup{labelformat=empty}
\caption{Рис. 8.18}
\end{center}
\end{figure}

Тогда можно утверждать, что убыль числа \(N\) равна


\begin{equation*}
-\mathrm{d} N=N \sigma n_{0} \mathrm{~d} x \tag{1}
\end{equation*}


где \(n_{0}\) - количество ядер в единице объема. Разделив переменные \(N\) и \(x\) в формуле (1), проинтегрируем полученное выражение.

В результате

\[
\ln \left(N / N_{0}\right)=-\sigma n_{0} d,
\]

откуда искомая толщина \(d\) равна

\[
d=\frac{\ln \left(N_{0} / N\right)}{\sigma n_{0}}=\frac{\ln \eta}{\sigma n_{0}}=0,40 \mathrm{mM} .
\]

Здесь \(n_{0}=N_{A} \rho / M, N_{A}\) - постоянная Авогадро, \(M\) - атомная масса кадмия (112,4 г/моль).\\
8.7. Энергия реакции. Найти энергию \(Q\) реакции

\[
{ }^{7} \mathrm{Li}(p, \alpha){ }^{4} \mathrm{He},
\]

если известно, что энергии связи ядер \({ }^{7} \mathrm{Li}\) и \({ }^{4} \mathrm{He}\) равны соответственно \(E_{\mathrm{Li}}=39,2 \mathrm{M} \ni \mathrm{B}\) и \(E_{\mathrm{He}}=28,2 \mathrm{M} \ni\) В.

Решение. Энергия реакции согласно (8.39) равна


\begin{equation*}
Q=\left(m_{\mathrm{Li}}+m_{\mathrm{H}}\right)-2 m_{\mathrm{He}} \tag{1}
\end{equation*}


где \(m\) - массы ядер - выражены в энергетических единицах. Представим эти массы в соответствии с формулой (8.5) как

\[
\begin{aligned}
m_{\mathrm{Li}} & =3 m_{p}+4 m_{n}-E_{\mathrm{Li}}, \\
m_{\mathrm{He}} & =2 m_{p}+2 m_{n}-E_{\mathrm{He}} .
\end{aligned}
\]

После подстановки этих выражений в (1) и сокращения всех \(m_{p}\) и \(m_{n}\) получим

\[
Q=2 E_{\mathrm{He}}-E_{\mathrm{Li}}=17,2 \text { МэВ. }
\]

\section*{8.8. Найти энергию реакции}
\[
{ }^{14} \mathrm{~N}(\alpha, p){ }^{17} \mathrm{O},
\]

если энергия налетающей \(\alpha\)-частицы \(K_{\alpha}=4,00 \mathrm{M} Э \mathrm{~B}\), и протон, вылетевший под углом \(\theta=60^{\circ}\) к направлению движения \(\alpha\)-частицы,

\begin{figure}[h]
\begin{center}
  \includegraphics[alt={},max width=\textwidth]{a57b3a57-78b5-4292-b0ab-4bd9f2ffcf56-221_196_358_1449_138}
\captionsetup{labelformat=empty}
\caption{Рис. 8.19}
\end{center}
\end{figure}

имеет энергию \(K_{p}=2,08\) МэВ.\\
Решение. Исходим из того, что импульс и полная энергия системы в этом процессе сохраняются. С помощью рис. 8.19 и теоремы косинусов запишем:


\begin{equation*}
p_{\mathrm{O}}^{2}=p_{\alpha}^{2}+p_{p}^{2}-2 p_{\alpha} p_{p} \cos \theta \tag{1}
\end{equation*}


Кроме того, энергия реакции \(Q\) согласно (8.37)


\begin{equation*}
Q=\left(K_{p}+K_{\mathrm{O}}\right)-K_{\alpha} . \tag{2}
\end{equation*}


Имея в виду, что \(K=p^{2} / 2 m\), решим совместно уравнения (1) и (2), исключив из них \(p_{\mathrm{O}}\) и \(K_{\mathrm{O}}\). В результате получим

\[
Q=(1+a) K_{p}-(1-b) K_{\alpha}-2 \cos \theta \sqrt{a b K_{p} K_{\alpha}},
\]

где \(a=m_{p} / m_{\mathrm{O}}, b=m_{\alpha} / m_{\mathrm{O}}\), или

\[
Q=\frac{18}{17} K_{p}-\frac{13}{17} K_{\alpha}-2 \cos \theta \sqrt{\frac{18 \cdot 13}{17^{2}} K_{p} K_{\alpha}}=-1,2 \text { МэВ. }
\]

8.9. Порог реакции. Литиевую мишень облучают пучком протонов с кинетической энергией в \(\eta=1,50\) раза превышающей пороговое значение реакции

\[
{ }^{7} \mathrm{Li}(p, n){ }^{7} \mathrm{Be}+Q,
\]

где \(Q=-1,65\) МэВ. Найти кинетическую энергию нейтронов, вылетающих под прямым углом к пучку протонов.

Решен и е. Из сохранения импульса

\begin{figure}[h]
\begin{center}
  \includegraphics[alt={},max width=\textwidth]{a57b3a57-78b5-4292-b0ab-4bd9f2ffcf56-222_197_296_830_864}
\captionsetup{labelformat=empty}
\caption{Рис. 8.20}
\end{center}
\end{figure}

(рис. 8.20) и энергии следует:


\begin{equation*}
p_{n}^{2}+p_{p}^{2}=p_{\mathrm{Be}}^{2} \quad \text { или } \quad m K_{n}+m K_{p}=m_{\mathrm{Be}} K_{\mathrm{Be}} \text {, } \tag{1}
\end{equation*}


где считаем \(m_{n}=m_{p}=m\). Согласно (8.37) можно записать второе уравнение:


\begin{equation*}
Q=\left(K_{n}+K_{\mathrm{Be}}\right)-K_{p} \quad \text { или } \quad K_{p}-|Q|=K_{n}+K_{\mathrm{Be}} . \tag{2}
\end{equation*}


Решив совместно уравнения (1) и (2) с учетом того, что \(K_{p}=\eta K_{p \text { пор }}\), получим:


\begin{equation*}
K_{n}\left(m+m_{\mathrm{Be}}\right)=\left(m_{\mathrm{Be}}-m\right) \eta K_{p \text { пор }}-m_{\mathrm{Be}}|Q| . \tag{3}
\end{equation*}


Остается учесть, что согласно (8.45)

\[
K_{p_{\text {пор }}}=\frac{m+m_{\mathrm{Li}}}{m_{\mathrm{Li}}}|Q| .
\]

Тогда из (3) следует, что

\[
K_{n}=\left(\frac{m_{\mathrm{Be}}-m}{m_{\mathrm{Li}}} \eta-\frac{m_{\mathrm{Be}}}{m+m_{\mathrm{Be}}}\right)|Q|=\left(\frac{6}{7} \eta-\frac{7}{8}\right)|Q|=0,68 \text { МэВ. }
\]

8.10. Энергетические уровни ядра. Борную мишень облучают пучком дейтронов с энергией \(K_{d}=1,50\) МэВ. В результате реакции \({ }^{10} \mathrm{~B}(d, p){ }^{11} \mathrm{~B}\) под прямым углом к пучку дейтронов испускаются протоны с энергиями \(K_{p}=7,64,5,51\) и 4,98 МэВ. Найти энергию \(E^{*}\) уровней возбужденных ядер \({ }^{11} \mathrm{~B}\), которые отвечают этим значениям энергии.

Рис. 8.21

Решение. Из сохранения импульса (рис. 8.21) следует, что


\begin{equation*}
p_{d}^{2}=p_{\mathrm{B}}^{2}-p_{p}^{2} \quad \text { или } \quad m_{d} K_{d}=m_{\mathrm{B}} K_{\mathrm{B}}-m_{p} K_{p} . \tag{1}
\end{equation*}


Здесь и далее индекс «В» относится к ядру \({ }^{11} \mathrm{~B}\).\\
Теперь запишем баланс энергии, учитывая, что ядро \({ }^{11} \mathrm{~B}\) возникает в возбужденном состоянии:


\begin{equation*}
K_{d}+Q=K_{p}+K_{\mathrm{B}}+E^{*} \tag{2}
\end{equation*}


где \(Q\) - энергия реакции (номинальная), определяемая формулой (8.39), здесь \(Q=9,23\) МэВ; \(E^{*}\) - энергия возбуждения ядра \({ }^{11} \mathrm{~B}\).

Решив совместно уравнения (1) и (2) путем исключения \(K_{\mathrm{B}}\), получим:

\[
E^{*}=Q+\left(1-\frac{m_{d}}{m_{\mathrm{B}}}\right) K_{d}-\left(1+\frac{m_{p}}{m_{\mathrm{B}}}\right) K_{p}=Q+\frac{9}{11} K_{d}-\frac{12}{11} K_{p} .
\]

При указанных значениях \(K_{p}\) получим соответственно

\[
E^{*}=2,12,4,45 \text { и } 5,03 \text { МэВ. }
\]

\section*{Глава 9}
\section*{Элементарные частицы}
\section*{§ 9.1. Введение}
Определения. Вообще говоря, элементарными (по смыслу) следует называть микрочастицы, относительно которых нет доказательств, что они являются составными. Это электроны, протоны, нейтроны и многие другие частицы. Впрочем, ситуация с определением элементарности усложнилась после того, как выяснилось, что многие из этих частиц имеют внутреннюю структуру.

Несмотря на последнее обстоятельство, за этими частицами сохранили название элементарных. И это в какой-то степени оправдано: во всех наблюдавшихся до сих пор явлениях каждая такая частица ведет себя как единое целое. Они могут рождаться и превращаться друг в друга, но не расщепляться на ка-кие-то составляющие.

Поэтому теперь в ядерной физике под термином «элементарные частицы» понимается общее название для всех субатомных частиц, отличных от атомов и атомных ядер.

Итак, частицы, которые мы называем элементарными, ведут себя как единое целое и обладают способностью к рождению и взаимопревращению. Например, распад нейтрона:


\begin{equation*}
n \rightarrow p+e^{-}+v, \tag{9.1}
\end{equation*}


где нейтрон превращается в протон, электрон и нейтрино*. Продукты распада нейтрона возникают только в самом этом процессе. До распада их не было совсем, и они не входили в состав нейтрона.

Для элементарных частиц весьма характерна их многочисленность. В настоящее время открыто несколько сотен частиц, подавляющее большинство которых нестабильно.

Источниками заряженных частиц высоких энергий являются в основном ускорители. Вместе с детекторами они позволяют ис-

\footnotetext{\begin{itemize}
  \item См. сноску на с. 201.
\end{itemize}
}следовать процессы, в которых образуются и взаимодействуют различные элементарные частицы. Вот почему физику элементарных частиц часто называют также физикой высоких энергий. В связи с этим мы будем часто использовать соотношения релятивистской динамики. Эти соотношения приведены в Приложении.

Фундаментальные взаимодействия. В природе существуют четыре типа фундаментальных взаимодействий: сильное, электромагнитное, слабое и гравитационное. Эти взаимодействия отличаются интенсивностью процессов, вызываемых среди элементарных частиц.

Об интенсивности взаимодействий можно судить по скорости (или степени вероятности) процессов, вызываемых ими. Обычно для сравнения берут скорости процессов при энергиях сталкивающихся частиц около 1 ГэВ (такая энергия характерна для физики элементарных частиц). Сравнительные характеристики этих четырех типов взаимодействия приведены в табл. 9.1, где указаны интенсивности взаимодействий по сравнению с сильным, принятым за единицу, а также длительность процессов и радиус действия соответствующих сил.

\begin{table}[h]
\begin{center}
\captionsetup{labelformat=empty}
\caption{Таблица 9.1}
\begin{tabular}{|l|l|l|l|}
\hline
Взаимодействие & Интенсивность & Длительность процессов, с & Радиус действия, см \\
\hline
Сильное & 1 & \(10^{-23}\) & \(10^{-13}\) \\
\hline
Электромагнитное & \(10^{-2}\) & \(10^{-20}\) & \(\infty\) \\
\hline
Слабое & \(10^{-14}\) & \(10^{-9}\) & \(10^{-16}\) \\
\hline
Гравитационное & \(10^{-31}\) & - & \(\infty\) \\
\hline
\end{tabular}
\end{center}
\end{table}

Остановимся более подробно на характеристике этих взаимодействий.

\begin{enumerate}
  \item Сильные взаимодействия удерживают нуклоны в атомных ядрах, они же присущи большинству адронов (протон, нейтрон, гипероны, мезоны и др.). Эти взаимодействия короткодействующие: на расстояниях свыше \(10^{-13} \mathrm{~cm}\) они прекращаются, вследствие чего сильные взаимодействия не способны создавать структуры макроскопических размеров.
  \item Электромагнитные взаимодействия осуществляются через электромагнитное поле. Они значительно слабее сильных взаимодействий, однако из-за дальнодействия электромагнитные силы во многих случаях оказываются главными. Именно эти силы вызывают разлет осколков, которые образуются при делении атомных ядер. Эти силы ответственны за все электрические и магнитные явления, наблюдаемые нами в различных формах их проявления: оптических, механических, тепловых, химических и т. д.
  \item Слабые взаимодействия весьма малы по сравнению с сильными и электромагнитными. Слабые взаимодействия являются универсальными: они присутствуют во всех взаимодействиях.
  \item Гравитационные взаимодействия самые слабые. Они универсальны. Но для элементарных частиц эти взаимодействия никакого значения не имеют, поэтому современная физика элементарных частиц - это физика без гравитации. В связи с этим в дальнейшем под фундаментальными мы будем понимать только сильные, электромагнитные и слабые взаимодействия.
\end{enumerate}

Практически все элементарные частицы являются нестабильными (за исключением фотона, электрона и трех нейтрино). Время жизни таких частиц варьируется в пределах от \(10^{-18}\) до \(10^{-11}\) с (у так называемых резонансов еще меньше). Но в некоторых случаях оно оказывается весьма продолжительным: например, среднее время жизни свободного нейтрона составляет 11,7 мин.

\section*{§ 9.2. Систематика элементарных частиц}
Бозоны и фермионы. Все частицы (включая и неэлементарные и так называемые квазичастицы) подразделяют на бозоньι и фермионы. Бозоны - это частицы с нулевым или целочисленным спином (фотон, мезоны и др.). Фермионы же - это частицы с полуцелым спином (электрон, мюон, таон, нейтрино, протон, нейтрон и др.).

Время жизни \(\tau\). Практически все элементарные частицы, как уже говорилось, являются нестабильными, распадаясь на другие частицы. По времени жизни различают стабильные,

квазистабильные и так называемые резонансы. Резонансами называют частицы, распадающиеся за счет сильного взаимодействия с временем жизни \(\sim 10^{-23}\) с. Нестабильные частицы с временем жизни, превышающим \(10^{-20} \mathrm{c}\), распадаются за счет электромагнитного или слабого взаимодействия. По сравнению с характерным ядерным временем ( \(10^{-23} \mathrm{c}\) ) время \(10^{-20}\) следует считать большим. По этой причине их и называют квазистабильными. Стабильными же частицами ( \(\tau \rightarrow \infty\) ) являются только фотон, электрон, протон и нейтрино.

Переносчики взаимодействия. Это особая группа элементарных частиц, в которую входят фотоньι (переносчики электромагнитного взаимодействия), родственные им \(W\) - и \(Z\)-бозонь (переносчики слабого взаимодействия), так называемые глюоны (переносчики сильного взаимодействия) и гипотетические гравитонью.

Все остальные частицы подразделяют по характеру взаимодействий, в которых они участвуют, на лептонь и адронь.

Лептоны. Это частицы, не участвующие в сильных взаимодействиях и имеющие спин \(1 / 2\). К ним относятся электроны, мюоны, таоны и соответствующие им нейтрино. Лептоны принимают участие в слабых взаимодействиях. За исключением нейтрино, лептоны участвуют и в электромагнитных взаимодействиях.

Все лептоны можно отнести к истинно элементарным частицам, поскольку у них, в отличие от адронов, не обнаружена внутренняя структура.

Адроны. Так называют элементарные частицы, участвующие в сильных взаимодействиях. Как правило, они участвуют и в электромагнитном, и в слабом взаимодействиях. Эти частицы образуют самую многочисленную группу частиц (свыше 400).

Адроны подразделяют на мезоны и барионы.\\
Мезоньи - это адроны с нулевым или целочисленным спином ( т. е. бозоны). К ним относятся \(\pi\)-, \(K\) - и \(\eta\)-мезоны, а также множество мезонных резонансов, т. е. мезонов с временем жизни \(\sim 10^{-23}\) с.

Барионьι - это адроны с полуцелым спином ( т. е. фермионы) и массами, не меньшими массы протона. К ним относятся

нуклонь (протоны и нейтроны), гиперонь и множество барионных резонансов. За исключением протона, все барионы нестабильны. Нестабильные барионы с массами, большими массы протона, и большим временем жизни (сравнительно с ядерным \(\sim 10^{-23} \mathrm{c}\) ) называют гиперонами. Это гипероны \(\Lambda, \Sigma, \Xi\) и \(\Omega\). Все гипероны имеют спин \(1 / 2\), за исключением \(\Omega\), спин которого \(3 / 2\). За время \(\tau \sim 10^{-10} \div 10^{-19}\) с они распадаются на нуклоны и легкие частицы ( \(\pi\)-мезоны, электроны, нейтрино, \(\gamma\)-кванты).

Сведем для наглядности основную систематику элементарных частиц в табл. 9.2.

\begin{table}[h]
\begin{center}
\captionsetup{labelformat=empty}
\caption{Таблица 9.2}
\begin{tabular}{|l|l|l|l|l|}
\hline
\multirow{3}{*}{Фотоны} & \multirow{3}{*}{Лептоны} & \multicolumn{3}{|c|}{Адроны} \\
\hline
 &  & \multirow{2}{*}{Мезоны} & \multicolumn{2}{|c|}{Барионы} \\
\hline
 &  &  & Нуклоны & Гипероны \\
\hline
\(\gamma\) & \(e, \mu, \tau, v\) & \(\pi, K, \eta\) и резонансы & \(p, n\) & \(\Lambda, \Sigma, \Xi, \Omega\) и резонансы \\
\hline
\end{tabular}
\end{center}
\end{table}

Более подробно классификация элементарных частиц приведена в таблице на заднем форзаце. Пояснения к некоторым характеристикам частиц в этой таблице будут даны в дальйшем по мере надобности.

\section*{§ 9.3. Античастицы}
Частицы и античастицы. Существование античастиц является универсальным свойством элементарных частиц. Каждой частице соответствует своя античастица: например, электрону \(e^{-}\)- позитрон \(e^{+}\), протону \(p^{+}\)- антипротон \(p^{-}\), нейтрону \(n\) антинейтрон \(\tilde{n}\) и т. д. Позитрон и антипротон отличаются от электрона и протона прежде всего знаком электрического заряда. Антинейтрон отличается от нейтрона знаком магнитного момента.

В общем случае античастица отличается от частицы только знаками так называемых зарядов (электрического, барионного, лептонного, странности), с которыми связаны определенные законы сохранения (подробнее об этом в следующем параграфе).

Такие же характеристики как масса, спин, время жизни у них одинаковы.

В некоторых случаях античастица совпадает со своей частицей, т. е. все свойства частицы и античастицы одинаковы. Такие частицы называют истинно нейтральньлми. К ним относятся, например, фотон \(\gamma\), \(\pi^{\circ}\)-мезон и \(\eta^{\circ}\)-мезон.

Понятия частицы и античастицы относительны. Электрон считают частицей, а позитрон - античастицей только потому, что во Вселенной преобладают именно электроны, а позитроны более экзотические частицы. Условившись считать электрон и протон частицами, далее с помощью законов сохранения можно однозначно установить, чем является каждая элементарная частица - частицей или античастицей (см. пример в следующем параграфе).

Аннигиляция и рождение пар. При встрече электрона с позитроном происходит их аннигиляиия, т. е. превращение их в \(\gamma\)-кванты, например так:

\[
e^{-}+e^{+} \rightarrow \gamma+\gamma .
\]

Заметим, что один \(\gamma\)-квант при этом излучиться не может: в этом случае нарушался бы закон сохранения импульса. Это легко понять, если рассмотреть процесс в Ц-системе, где суммарный импульс электрона и позитрона равен нулю.

Существует процесс, обратный аннигиляции, - рождение пар: \(\gamma\)-квант может породить пару \(e^{+} e^{-}\). Для этого необходимо, чтобы энергия \(\gamma\)-кванта была не меньше собственной энергии пары \(2 m_{e} c^{2}\). Этот процесс может происходить только в поле атомного ядра, иначе нарушался бы закон сохранения импульса. В самом деле, в Ц-системе суммарный импульс образовавшейся пары был бы равен нулю, тогда как импульс породившего ее \(\gamma\)-кванта отличен от нуля. При наличии атомного ядра импульс \(\gamma\)-кванта будет восприниматься ядром без нарушения закона сохранения импульса.

Пример. Определим наименьшую энергию \(\gamma\)-кванта, \(\varepsilon_{\text {мин }}\), при которой возможно рождение пары электрон-позитрон на покоившемся протоне:

\[
\gamma+p \rightarrow p+e^{-}+e^{+} .
\]

Воспользуемся инвариантностью выражения (П.3'), т. е. \(E^{2}-p^{2}=m^{2}\), записав левую часть равенства в Л -системе, а правую - в Ц-системе:

\[
\left(\varepsilon_{\text {Мин }}+m_{p}\right)^{2}-\varepsilon_{\text {Мин }}^{2}=\left(m_{p}+2 m_{e}\right)^{2},
\]

где учтено, что \(p_{\text {мин }}=\varepsilon_{\text {мин }}\), и тот факт, что все три частицы в \(Ц\)-системе должны покоиться при \(\varepsilon=\varepsilon_{\text {мин }}\). После раскрытия скобок и сокращения соответствующих слагаемых получим:

\[
\varepsilon_{\text {мин }}=2 m_{e}\left(1+m_{e} / m_{p}\right) \approx 2 m_{e}=1,02 \text { МэВ. }
\]

Аннигилируют не только электрон с позитроном, но и любая другая частица со своей античастицей. Однако при аннигиляции тяжелых частиц и античастиц возникают преимущественно \(\pi\)-мезоны (доля \(\gamma\)-квантов весьма мала). Это обусловлено проявлением различных типов взаимодействий: аннигиляция электрона с позитроном вызывается электромагнитным взаимодействием, тогда как аннигиляция более тяжелых частиц - адронов - сильным взаимодействием.

\section*{§ 9.4. Законы сохранения}
Роль законов сохранения. Законы сохранения играют особо важную роль в физике элементарных частиц. Это обусловлено следующими двумя обстоятельствами.

\begin{enumerate}
  \item Они не только ограничивают последствия различных взаимодействий, но определяют также все возможности этих последствий, и поэтому отличаются высокой степенью предсказательности.
  \item В этой области открытие законов сохранения опережает создание последовательной теории. Многие законы сохранения для элементарных частиц уже установлены из опыта, а соответствующие фундаментальные законы их поведения еще неизвестны. Поэтому законы сохранения играют здесь главенствующую роль и позволяют анализировать процессы, механизм которых еще не раскрыт.
\end{enumerate}

Для элементарных частиц выполняется гораздо больше законов сохранения, чем для макроскопических процессов. Все эти законы подразделяются на точные и приближенные. Точные законы сохранения выполняются во всех фундаментальных взаимодействиях, а приближенные - только в некоторых.

Точными являются законы сохранения энергии, импульса и момента импульса. Точными являются и законы сохранения всех зарядов (речь о них ниже). Происхождение этих законов пока не установлено. Ясно только одно: каждый из этих зарядов характеризует некое внутреннее свойство частицы.

Необходимость введения зарядов (кроме электрического) было продиктовано многочисленными экспериментальными фактами, объяснить которые оказалось возможным только при допущении, что существуют заряды неэлектрической природы, которые также сохраняются.

Установлено пять зарядов: электрический \(Q\), барионный \(B\) и три лептонных \(, L_{e}, L_{\mu}\) и \(L_{\tau}\). У всех элементарных частиц эти заряды имеют только целочисленные значения (заряд \(Q\) - это число единиц элементарного заряда).

Барионный заряд. Если барионам и антибарионам приписать барионный заряд \(B\) такой, что

\[
B=\left\{\begin{array}{l}
+1 \text { для барионов (нуклонов и гиперонов), } \\
-1 \text { для антибарионов, }
\end{array}\right.
\]

а всем остальным частицам - барионный заряд \(B=0\), то для всех процессов с участием барионов и антибарионов суммарный барионный заряд будет сохраняться. Это и называют законом сохранения барионного заряда.

Барионный заряд, как и все другие заряды, аддитивен: для сложной системы частиц заряд каждого вида равен сумме зарядов того же вида всех частиц системы. Например, барионный заряд ядра атома равен сумме всех барионных зарядов нуклонов данного ядра. Другими словами, барионный заряд ядра равен его массовому числу \(A\).

Согласно закону сохранения барионного заряда частицы с \(B=+1\) или -1 не распадаются только на частицы с \(B=0\). Например, протон \(p\) не может превратиться в позитрон \(e^{+}\)и фотон \(\gamma\), хотя это не запрещено законами сохранения энергии, импульса, момента и электрического заряда. Запрет на это превращение связан с нарушением закона сохранения барионного заряда \(B:\) у протона \(B=+1\), а у позитрона и \(\gamma\)-кванта \(B=0\). Если бы такое превращение было возможно, то это неизбежно привело бы к аннигиляции атомов вещества, так как образовавшиеся позитроны аннигилировали бы с атомными электронами.

Из того же закона следует, что антибарион может рождаться только в паре со своим барионом. Например, антипротон рождается в реакции

\[
p+p \rightarrow p+p+p+\tilde{p} .
\]

Могут возникнуть и два антипротона, но тогда появятся и два новых протона.

Лептонные заряды. Существуют три вида лептонных зарядов: электронный \(L_{e}\) (для \(e\) и \(v_{e}\) ), мюонный \(L_{\mu}\) (для \(\mu\) и \(v_{\mu}\) ) и таонный \(L_{\tau}\) (для \(\tau\) и \(v_{\tau}\) ). Здесь \(v_{e}, v_{\mu}, v_{\tau}\) - электронное, мюонное и таонное нейтрино. Из эксперимента следует, что это разные нейтрино.

С помощью лептонных зарядов легко интерпретируется установленный экспериментально закон, согласно которому в замкнутой системе при любых процессах разность между числом лептонов и антилептонов сохраняется (это же относится и к барионам).

Условились считать, что

\[
L_{e}=L_{\mu}=L_{\tau}=\left\{\begin{array}{l}
+1 \text { для лептонов }\left(e^{-}, v_{e} ; \mu^{-}, v_{\mu} ; \tau^{-}, v_{\tau}\right) ; \\
-1 \text { для антилептонов }\left(e^{+}, \tilde{v}_{e} ; \mu^{+}, \tilde{v}_{\mu} ; \tau^{+}, \tilde{v}_{\tau}\right) .
\end{array}\right.
\]

Для всех остальных элементарных частиц лептонные заряды принимаются равными нулю.

Закон сохранения лептонного заряда требует, чтобы при распаде, например, нейтрона


\begin{equation*}
n \rightarrow p+e^{-}+\tilde{v}_{e} \tag{9.2}
\end{equation*}


вместе с электроном рождалось электронное антинейтрино, так как суммарный лептонный заряд этих двух частиц равен нулю. Тем самым мы уточнили выражения (8.1) и на стр. 201: в них вместо \(v\) должно фигурировать \(\tilde{v}_{e}\). Вместе с тем из этого превращения (9.2) следует, что поскольку протон \(p\) - частица \((B=+1)\), то частицей является и нейтрон \(n\) (тоже \(B=+1\) ).

Законом сохранения лептонного заряда объясняется невозможность следующих процессов:


\begin{equation*}
v_{e}+p \nrightarrow e^{+}+n, \quad v_{\mu}+p \nrightarrow \mu^{+}+n, \tag{9.3}
\end{equation*}


хотя другими законами сохранения они разрешены. Процессы же


\begin{equation*}
\tilde{v}_{e}+p \rightarrow e^{+}+n, \quad \widetilde{v}_{\mu}+p \rightarrow \mu^{+}+n, \tag{9.4}
\end{equation*}


удовлетворяющие закону сохранения лептонного заряда, наблюдали экспериментально.

Эти два примера показывают, что нейтрино (как электронное, так и мюонное) не тождественны своим античастицам. После того, как было установлено, что \(v_{e}\) и \(v_{\mu}\) - разные частицы, и были введены разные лептонные заряды \(L_{e}\) и \(L_{\mu}\). Аналогично обстояло дело и с введением таонного лептонного заряда \(L_{\tau}\).

Странность \(S\). Было обнаружено, что гипероны рождаются при столкновениях адронов высоких энергий. Значит их рождение связано с сильным взаимодействием, и время жизни гиперонов должно быть порядка \(10^{-23}\) с (время, характерное для процессов, обусловленных сильным взаимодействием). На опыте же было найдено, что их время жизни в \(10^{13}\) раз больше. Такое поведение гиперонов представлялось странным.

Оказалось также, что гипероны в этих процессах рождаются не поодиночке, а только парами. Например, при столкновении протонов:


\begin{equation*}
p+p \rightarrow p+\Lambda^{0}+K^{+}, \tag{9.5}
\end{equation*}


причем \(\Lambda^{0}\)-гиперон появляется только совместно с \(K^{+}\)-мезоном или с \(\Sigma^{+}\)-гипероном, но никогда не появляется вместе с \(K^{-}\)-мезоном или \(\Sigma^{-}\)-гипероном.

Гипероны и \(K\)-мезоны назвали странными частицами. После рождения эти частицы медленно и независимо друг от друга распадаются за счет слабого взаимодействия.

Для количественного описания парного рождения и медленного распада странных частиц было введено квантовое число \(S\) - странность. Поведение странных частиц можно объяснить, если считать, что частицы \(\Lambda^{0}, \Sigma\) и \(K^{-}\)имеют странность \(S=-1\), частицы \(\Xi-S=-2\) и \(\Omega^{-}\)-гиперон \(-S=-3\). У соответствующих античастиц странность одинакова по модулю, но противоположна по знаку.

При этом странность в сильных и электромагнитных взаимодействиях сохраняется, а в слабых может меняться на \(\pm 1\).

Представим сведения о барионных зарядах \(B\) и странности \(S\) адронов в табл. 9.3. Для соответствующих античастиц \(B\) и \(S\) имеют противоположные знаки.

\begin{table}[h]
\begin{center}
\captionsetup{labelformat=empty}
\caption{Таблица 9.3}
\begin{tabular}{|l|l|l|l|l|l|l|l|}
\hline
\multirow{3}{*}{Заряды} & \multirow{2}{*}{Мезоны} & \multicolumn{6}{|c|}{Барионы} \\
\hline
 &  & \multicolumn{2}{|r|}{нуклоны} & \multicolumn{4}{|c|}{гипероны} \\
\hline
 & \(\pi \quad K^{+} \eta\) & \(p\) & \(n\) & \(\Lambda\) & \(\Sigma\) & \(\Xi\) & \(\Omega\) \\
\hline
B & 000 & +1 & +1 & +1 & +1 & +1 & +1 \\
\hline
S & 0 +1 0 & 0 & 0 & -1 & -1 & -2 & -3 \\
\hline
\end{tabular}
\end{center}
\end{table}

В реакции (9.5) протоны, будучи обычными частицами, странностью не обладают, их \(S=0\). Таким образом, \(0+0 \rightarrow 0-1+1\), т. е. странность при рождении пары странных частиц сохраняется. Распады же странных частиц на обычные (у которых \(S=0\) ) происходит с нарушением закона сохранения странности. Этим нарушением и объясняется медленность распада странных частиц.

Шарм (очарование) \(\boldsymbol{C}\) и красота (прелесть) \(\boldsymbol{b}\). Эти квантовые числа являются аналогами квантового числа странности \(S\). Они сохраняются только в сильных и электромагнитных взаимодействиях. Поскольку квантовые числа \(\boldsymbol{C}\) и \(\boldsymbol{b}\) присущи немногим, причем экзотическим, частицам ( \(D\) - и \(F\)-мезоны, \(\Lambda_{C}{ }^{-}\), \(\Lambda_{b}\)-барионы), мы этим и ограничимся.

\section*{§ 9.5. Четность}
Понятие четности возникает в связи с операцией инверсии. Мы знаем, что состояние микрочастицы описывается в квантовой теории Ψ-функцией. Выясним, как может вести себя эта функция при так называемой пространственной инверсии, т. е. при переходе к координатам \(x^{\prime}, y^{\prime}, z^{\prime}\), связанным с \(x, y, z\) как

\[
x^{\prime}=-x, \quad y^{\prime}=-y, \quad z^{\prime}=-z, \quad \text { или } \quad \mathbf{r}^{\prime}=-\mathbf{r} .
\]

Такие преобразования, как видно из рис. 9.1, представляют собой переход от правовинтовой системы координат к левовинтовой, и наоборот. Другими словами, пространственная инверсия состоит из зеркального

\begin{figure}[h]
\begin{center}
  \includegraphics[alt={},max width=\textwidth]{a57b3a57-78b5-4292-b0ab-4bd9f2ffcf56-235_372_548_139_68}
\captionsetup{labelformat=empty}
\caption{Рис. 9.1}
\end{center}
\end{figure}

отражения относительно плоскости, проходящей через начало координат \(O\) (на рисунке - это \(X Y\)-плоскость), и последующего поворота на \(180^{\circ}\) вокруг оси, перпендикулярной этой плоскости (на рисунке - это ось \(Z\) ). В результате правый «винт» (n) превращается в левый ( \(\mathbf{n}^{\prime}\) ).

Особенность пространственной инверсии обусловлена зеркальным отражением. В связи с этим ее часто называют зеркальным отражением.

Можно показать, что при пространственной инверсии в любой момент времени Ψ-функция или не изменяется совсем, или у нее изменяется только знак. В первом случае состояние, описываемое функцией \(\Psi(r, t)\), называют четным, во втором - нечетным.

Поведение Ψ-функции при инверсии зависит от внутренних свойств частиц, описываемых этой функцией. Говорят, что частица обладает соответственно положительной или отрицательной внутренней четностью ( \(P=+1\) или \(P=-1\) ).

Внутренняя четность характеризует именно внутреннее свойство частицы наряду с такими величинами как масса, электрический заряд и спин. Четными являются, например, электроны, протоны и нейтроны. \(К\) нечетным относятся, например, \(\pi\)-мезоны.

Отметим, что четность, как величина сугубо квантового происхождения, не имеет классического аналога.

Четность является важной физической величиной благодаря симметрии трех фундаментальных взаимодействий (сильного, электромагнитного и гравитационного) по отношению к зеркальным отражениям. Все фундаментальные взаимодействия (за исключением слабого) происходят одинаково как в физических системах, так и в системах, являющихся их зеркальными копиями.

Это называют также инвариантностью фундаментальных взаимодействий (за исключением слабого) относительно пространственной инверсии. Данную симметрию выражает закон сохранения четности:

Четность квантового состояния не зависит от времени при условии, что влияние слабых взаимодействий пренебрежимо мало.\\
Существенно отметить, что как зеркальная симметрия, так и выражающий ее закон сохранения четности справедливы с точностью до эффектов, обусловленных слабыми взаимодействиями. Но из-за «слабости» последних ими можно пренебречь для подавляющего большин-

ства микрочастиц, и в этих случаях четность является достаточно добротным квантовым числом.

Однако в процессах, где основным эффектом являются слабые взаимодействия (например, \(\beta\)-распад ядер, или процесс \(K^{0} \rightarrow \pi^{+}+\pi^{-}\)), наблюдается несохранение четности, т. е. физическое неравноправие левого и правого по отношению к этим взаимодействиям.

Теоретически это предсказали Ц. Ли и Ч. Янг (1956). Они также предложили идею опыта, который был осуществлен Ву Цзянь-сун и ее сотрудниками. Суть идеи в следующем: если правое и левое в природе неразличимы, то при \(\beta\)-распаде вылет электронов в направлении спина ядра и в противоположном направлении должен быть равновероятен.

Действительно, при зеркальном отражении ядра направление его «вращения», т. е. направление спина, изменится на противоположное (рис. 9.2, где направление спина показано двойной стрелкой). Если ядро испускает электроны с равной вероятностью в обоих направлениях (рис. 9.2, \(a\) ), то зеркальное отражение ядро (спин) - электроны будет неотличимо от самой системы (они лишь повернуты относительно друг друга на \(180^{\circ}\) вокруг оси, перпендикулярной зеркалу 3). Если же электроны испускаются преимущественно в одном направлении (рис. 9.2, б), то «левое» и «правое» становятся различимыми.

В опыте Ву спины ядер \(\beta\)-активного кобальта \({ }^{60} \mathrm{Co}\) ориентировались с помощью маг-

\begin{figure}[h]
\begin{center}
  \includegraphics[alt={},max width=\textwidth]{a57b3a57-78b5-4292-b0ab-4bd9f2ffcf56-236_505_388_639_776}
\captionsetup{labelformat=empty}
\caption{Рис. 9.2}
\end{center}
\end{figure}

нитного поля в одном направлении. Оказалось, что электроны испускаются преимущественно в направлении, противоположном направлению ядерных спинов. Так была доказана экспериментально неравноправность правого и левого при слабых взаимодействиях (которыми обусловлен \(\beta\)-распад).

\section*{§ 9.6. Изотопический спин}
Оказывается, что сильно взаимодействующие частицы (адроны), весьма близкие по своим физическим свойствам, можно разбить на группы, называемые изотопическими мультиплетами (дублеты, триплеты и т. д.). В каждом мультиплете частицы одинаковым образом участвуют в сильных взаимодействиях, имеют примерно равные массы и одинаковые барионный заряд, спин, внутреннюю четность, странность, отличаясь друг

от друга электрическим зарядом. В отсутствии электромагнитных и слабых взаимодействий все свойства таких частиц были бы одинаковыми.

Эту по существу независимость от электрических зарядов называют изотопической (или зарядовой) независимостью сильных взаимодействий. Так, протон и нейтрон объединяют в изотопический дублет. Эти две частицы рассматриваются как различные квантовые состояния одной и той же частицы - нуклона. Изотопические триплеты - это, например, ( \(\pi^{-}, \pi^{0}, \pi^{+}\)) и ( \(\Sigma^{-}, \Sigma^{0}, \Sigma^{+}\)). Cy ществуют и одиночные частицы, не входящие в мультиплеты, их называют синглетами ( \(\eta\)-мезон, \(\Lambda\) - и \(\Omega\)-гипероны).

По аналогии с обычным спином каждому зарядовому мультиплету приписывают определенное значение изотопического спина (короче изоспина) \(T\). Значение \(T\) выбирают так, чтобы \(2 T+1\) было равно числу частиц в мультиплете. Отдельным частицам мультиплета приписывают различные значения \(T_{z}\) — проекиии изоспина на ось \(Z\) в воображаемом изотопическом пространстве. Причем частице с большим электрическим зарядом - большее значение \(T_{z}\). Например, для нуклонов \(T=1 / 2\), у протона \(T_{z}=+1 / 2\), у нейтрона \(T_{z}=-1 / 2\); для \(\pi\)-мезонов* \(T=1\), тогда для \(\pi^{+}, \pi^{0}, \pi^{-}\)соответственно \(T_{z}\) равно \(+1,0,-1\).

С изоспином связан закон сохранения. При сильных взаимодействиях сохраняется как изоспин \(T\), так и его проекция. При электромагнитных - только \(T_{z}\), сам же изоспин \(T\) не сохраняется. Слабые взаимодействия протекают как правило с изменением изоспина \(T\).

Понятие изоспина оказалось весьма плодотворным. На основании изотопической инвариантности удается предсказать существование, массу и заряд новых частиц. Именно так были предсказаны существование и свойства частиц \(\pi^{0}, \Sigma^{0}, \Xi^{0}\) по известным \(\pi^{ \pm}, \Sigma^{ \pm}\)и \(\Xi^{-}\). В заключение заметим, что понятие изоспина плодотворно используется не только по отношению к элементарным частицам, но и к атомным ядрам.

\footnotetext{\begin{itemize}
  \item Тот факт, что в случае \(\pi\)-мезонов в одном зарядовом мультиплете объединяются частица ( \(\pi^{+}\)) и античастица ( \(\pi^{-}\)), объясняется тем, что частицы, входящие в мультиплет, должны отличаться только величиной или знаком электрического заряда. Все остальные величины частиц мультиплета должны быть одинаковыми, что и имеет место для всех трех компонент данного триплета.
\end{itemize}
}\section*{§ 9.7. Кварковая модель адронов}
Кварки. Большое разнообразие адронов заставило усомниться в их «элементарности» и побудило к поиску более фундаментальных, первичных частиц, из которых они могли бы быть построены. В настоящее время внутренняя структура не обнаружена только у фотона и лептонов. А составной характер адронов уже доказан (теоретически и подтвержден экспериментально).

Первоначально гипотеза о том, что все адроны построены из частиц, названных кварками, была выдвинута Гелл-Манном и Цвейгом в 1964 г. На основе кварковой гипотезы была не только понята структура уже известных адронов, но и предсказано существование новых. Ниже кварковая модель адронов будет представлена в своем современном виде.

К настоящему времени установлено существование пяти типов (или ароматов) кварков: \(u, d, s, c, b\). Все кварки имеют спин \(1 / 2\) и барионный заряд \(B=1 / 3\). Остальные свойства этих частиц (т. е. соответствующих квантовых чисел) приведены в табл. 9.4.

\begin{table}[h]
\begin{center}
\captionsetup{labelformat=empty}
\caption{Таблица 9.4}
\begin{tabular}{|l|l|l|l|l|}
\hline
Кварк & Электрический заряд \(Q\) & Странность \(S\) & Шарм (очарование) C & Красота \(b\) \\
\hline
\(u\) & 2/3 & 0 & 0 & 0 \\
\hline
\(d\) & -1/3 & 0 & 0 & 0 \\
\hline
\(s\) & -1/3 & -1 & 0 & 0 \\
\hline
\(c\) & \(2 / 3\) & 0 & 1 & 0 \\
\hline
\(b\) & -1/3 & 0 & 0 & 1 \\
\hline
\end{tabular}
\end{center}
\end{table}

Таким образом, кварки разительно отличаются от всех известных до сих пор частиц дробностью своих зарядов \(Q\) и \(B\).

Кварк \(s\) является носителем странности, \(c\) - шарма (очарования), \(b\) - красоты.

Соответствующие антикварки отличаются от кварков знаками зарядов \(Q, B, S, C\) и \(b\).

Сравнивая заряды кварков с зарядами мезонов и барионов, мы приходим к выводу, что каждый мезон является парой кварк-антикварк, а каждый барион состоит из трех кварков. Действительно, только кварк-антикварк имеет \(B=0\) и только три кварка образуют частицу с полуцелым спином и барионным зарядом \(B=1\). В табл. 9.5 приведен кварковый состав некоторых адронов, спин которых указан в скобках. «Ориентация» спинов кварков и антикварков здесь показаны условно стрелками.

\begin{table}[h]
\begin{center}
\captionsetup{labelformat=empty}
\caption{Таблица 9.5}
\begin{tabular}{||c|c|c|c|c|c||}
\hline
Частицы & \(\pi^{+}(0)\) & \(\pi^{-}(0)\) & \(p(1 / 2)\) & \(n(1 / 2)\) & \(\Omega^{-}(3 / 2)\) \\
\hline
Состав & \(u \tilde{d}(\uparrow \downarrow)\) & \(\tilde{u} d(\uparrow \downarrow)\) & \(u u d(\uparrow \downarrow \uparrow)\) & \(u d d(\uparrow \downarrow \uparrow)\) & \(s s s(\uparrow \uparrow \uparrow)\) \\
\hline
\end{tabular}
\end{center}
\end{table}

Заметим попутно, что истинно нейтральный \(\pi^{0}\)-мезон состоит из таких же кварка и антикварка. Но он представляет собой суперпозицию состояний \(u \tilde{u}\) и \(d \tilde{d}\), находясь с равной вероятностью в одном или другом состоянии.

Из табл. 9.5 видно, что \(\Omega^{-}\)-гиперон состоит из трех \(s\)-кварков с параллельными спинами (подобная ситуация имеет место и в случае некоторых других адронов). Это оказывается несовместимым с принципом Паули, который запрещает одинаковым частицам с полуцелым спином находиться в одном и том же состоянии.

Чтобы устранить это противоречие, было выдвинуто предположение о наличии у кварков некой внутренней степени свободы, из-за которой кварки одного типа (аромата) могут отличаться друг от друга. Эту степень свободы назвали цветом.

Каждый тип (аромат) кварка характеризуют тремя цветами: красный, зеленый и голубой. Их смесь бесцветна. Цвет каждого антикварка считается дополнительным цвету кварка, так что пара кварк-антикварк также бесцветна.

Противоречие с принципом Паули было устранено с помощью принципа бесцветности адронов. Этот принцип разрешает возможными только те сочетания кварков разных цветов, смесь которых бесцветна. Так, согласно этому принципу \(\Omega^{-}\)-гиперон состоит из трех \(s\)-кварков разных цветов, смесь которых

бесцветна. Поскольку кварки разных цветов неодинаковы, то кварковая структура \(\Omega^{-}\)-гиперона не будет противоречить принципу Паули. Антикваркам присвоили антицвета, каждый из которых является дополнительным к своему цвету, так что комбинации цвет-антицвет считаются бесцветными.

По современным представлениям сильные взаимодействия осуществляются путем обмена между кварками безмассовыми частицами - глюонами. Глюоны являются квантами поля, которое кварки создают и которое на них же и воздействует. Кроме того, они еще являются и переносчиками цвета. Поэтому при испускании и поглощении глюонов цвет кварков изменяется, но их аромат при этом сохраняется. Например, и-кварк не превращается в \(s\)-кварк.

Таким образом, согласно модели цветных кварков, последние, не нарушая бесцветности адронов, беспрестанно изменяют в них свою окраску.

Успешная классификация адронов на основе кварковой модели - это веский аргумент в ее пользу. То же следует сказать о об опытах по прямому просвечиванию нуклонов и других адронов электронами высоких энергий. Анализ полученных результатов привел к заключению, что внутри адронов электроны рассеиваютоя на точенных частицах с электрическими зарядами \(+2 / 3\) и \(-1 / 3\), причем эти частицы (кварки) ведут себя как бесструктурные точечные элементы.

Необычное поведение кварков. Многочисленные поиски свободных кварков оказались безуспешными. По-видимому, в свободном состоянии кварки не существуют, и это свидетельствует о необычных свойствах сил взаимодействия между кварками. А именно, согласно одной из гипотез сила взаимодействия между кварками не убывает с увеличением расстояния между ними, чем кварки резко отличаются от всех других частиц. Поэтому при неубывающей с расстоянием силе, связывающей кварки в адроне, нужно затратить неограниченно большую энергию, чтобы вырвать кварк из адрона. Такое поведение кварков обусловлено тем, что все глюоны, которые кварки испускают, сосредотачиваются только вблизи прямой, проходящей через кварки, образуя узкую трубку глюонного поля. Так как при этом глюонное поле «не рассеивается» в окружающем пространстве, то

глюоны также не вылетают из адронов, и поэтому их также невозможно зарегистрировать.

Кварк, получивший энергию в результате столкновения с электроном, не вылетает наружу из адрона, а затрачивает ее на образование кварк-антикварковых пар, т. е. на образование новых адронов, в основном мезонов.

Не исключена и другая причина ненаблюдаемости кварков в свободном состоянии - возможно их очень большие массы. Это значит, что их энергия связи в адронах весьма велика и оказывается недоступной для современных ускорителей.

Все это следует рассматривать пока только как предположения, и не более. Проблема ждет своего разрешения.

В любом случае в настоящее время считают, что истинно элементарными или фундаментальными частицами являются фотон, лептоны и кварки.

\section*{Задачи}
Внимание! В задачах 9.1-9.6 использованы сокращенные обозначения, приведенные в Приложении 1 (например, \(p\) и \(m\) - это сокращенные записи величин \(p c\) и \(m c^{2}\) ).\\
9.1. Релятивистские соотношения. Определить кинетическую энергию \(K\) релятивистской частицы массы \(m\) с импульсом \(p\).

Р е ш ен и е. Из инвариантности выражения \(E^{2}-p^{2}=m^{2}\), где \(E=m+K\), находим

\[
K^{2}+2 m K-p^{2}=0,
\]

корень которого

\[
K=-m+\sqrt{m^{2}+p^{2}}=m\left(\sqrt{1+(p / m)^{2}-1}\right)
\]

9.2. Замедление времени. Релятивистский \(\pi\)-мезон с кинетической энергией \(K\) пролетает от места рождения до распада в среднем расстояние \(l\). Найти собственное время жизни \(\tau_{0}\) этих мезонов.\\
Решение. Известно, что время жизни частицы в лабораторной системе отсчета и ее собственное время \(\tau_{0}\) связаны соотношением \(\tau=\tau_{0} / \sqrt{1-\beta^{2}}\), где \(\beta=v / c\) и \(\tau=l / v\). Тогда


\begin{equation*}
\tau_{0}=\frac{l}{\beta c} \sqrt{1-\beta^{2}}=\frac{l}{c} \sqrt{\frac{1}{\beta^{2}}-1} . \tag{*}
\end{equation*}


Выразим \(\beta^{2}\) через \(K\). Из формулы для полной энергии частицы

\[
E=m+K=\frac{m}{\sqrt{1-\beta^{2}}}
\]

Отсюда

\[
\beta^{2}=1-\left(\frac{m}{m+K}\right)^{2} .
\]

После подстановки этого выражения в (*) получим:

\[
\tau_{0}=\frac{l}{c} \frac{m}{\sqrt{K(K+2 m)}} .
\]

9.3. Распад частиц. Остановившийся \(\pi\)-мезон распался на мюон и антинейтрино. Найти кинетическую энергию мюона.

Решение. Энергия распада \(Q=m_{\pi}-m_{\mu}\), где учтено, что масса антинейтрино равна нулю. Энергия \(Q\) - это суммарная кинетическая энергия мюона и антинейтрино:


\begin{equation*}
Q=K_{\mu}+E_{v} \tag{1}
\end{equation*}


Кроме того, суммарный импульс системы равен нулю, а это значит, что


\begin{equation*}
p_{\mu}=p_{v} . \tag{2}
\end{equation*}


Из этих формул следует:

\[
Q=K_{\mu}+p_{v}=K_{\mu}+p_{\mu}=K_{\mu}+\sqrt{K_{\mu}\left(K_{\mu}+2 m_{\mu}\right)},
\]

откуда \(K_{\mu}=Q^{2} /\left(Q+m_{\mu}\right)\), или с учетом (1)

\[
K_{\mu}=\left(m_{\pi}-m_{\mu}\right)^{2} / 2 m_{\pi} .
\]

9.4. \(\Sigma\)-гиперон с кинетической энергией \(K_{\Sigma}\) распался на лету на нейтральную частицу и \(\pi\)-мезон, который вылетел с энергией \(K_{\pi}\) под прямым углом к первоначальному направлению движения \(\Sigma\)-гиперона. Определить энергию покоя нейтральной частицы (обозначим ее индексом \(x\) ).\\
Решение. Исходим из сохранения импульса и полной энергии в этом процессе:


\begin{equation*}
\mathbf{p}_{\Sigma}=\mathbf{p}_{x}+\mathbf{p}_{\pi}, \quad E_{x}=E_{\Sigma}-E_{\pi} \tag{1}
\end{equation*}


Так как угол между векторами \(\mathbf{p}_{\pi}\) и \(\mathbf{p}_{\Sigma}\) прямой, то по теореме Пифагора


\begin{equation*}
p_{x}^{2}=p_{\Sigma}^{2}+p_{\pi}^{2} . \tag{2}
\end{equation*}


Кроме того, возведя в квадрат второе из равенств (1), запишем


\begin{equation*}
E_{x}^{2}=E_{\Sigma}^{2}-2 E_{\Sigma} E_{\pi}+E_{\pi}^{2} \tag{3}
\end{equation*}


Теперь, имея в виду, что \(E^{2}-p^{2}=m^{2}\) согласно (П. 3'), вычтем (2) из (3). В результате получим:

\[
m_{x}=\sqrt{m_{\Sigma}^{2}+m_{\pi}^{2}-2\left(m_{\Sigma}+K_{\Sigma}\right)\left(m_{\pi}+K_{\pi}\right)} .
\]

9.5. Аннигиляция частиц. Релятивистский позитрон с кинетической энергией \(K_{e}\) налетает на покоящийся свободный электрон. В результате аннигиляции возникают два \(\gamma\)-кванта с одинаковыми энергиями. Определить угол \(\theta\) между направлениями их разлета.

Решение. При одинаковых энергиях \(\gamma\)-квантов треугольник

\begin{figure}[h]
\begin{center}
  \includegraphics[alt={},max width=\textwidth]{a57b3a57-78b5-4292-b0ab-4bd9f2ffcf56-243_239_355_892_135}
\captionsetup{labelformat=empty}
\caption{Рис. 9.3}
\end{center}
\end{figure}

импульсов данного процесса будет равносторонним (рис. 9.3). По теореме косинусов


\begin{equation*}
p_{e}^{2}=2 p_{\gamma}^{2}+2 p_{\gamma}^{2} \cos \theta=2 p_{\gamma}^{2}(1+\cos \theta) . \tag{1}
\end{equation*}


Кроме того, из равенства полных энергий до и после аннигиляции следует:


\begin{equation*}
K_{e}+2 m_{e}=2 E_{\gamma}=2 p_{\gamma} . \tag{2}
\end{equation*}


Выразим импульс \(p_{e}\) через \(K_{e}\). Согласно (П.5')


\begin{equation*}
p_{e}^{2}=K_{e}\left(K_{e}+2 m_{e}\right) . \tag{3}
\end{equation*}


Подставим затем в исходную формулу (1) выражения для \(p_{\gamma}\) из (2), а также (3). Тогда


\begin{equation*}
K_{e}\left(K_{e}+2 m_{e}\right)=\left(K_{e}+2 m_{e}\right)^{2} \frac{1+\cos \theta}{2} . \tag{4}
\end{equation*}


Учитывая, что \(1+\cos \theta=2 \cos ^{2}(\theta / 2)\), получим в результате

\[
\cos \frac{\theta}{2}=\frac{1}{\sqrt{1+2 m_{e} / K_{e}}} .
\]

9.6. Энергетический порог реакции. Релятивистская частица массы \(m\) в результате столкновения с покоившейся частицей массы \(M\) возбуждает реакцию рождения новых частиц:

\[
m+M \rightarrow m_{1}+m_{2}+\ldots,
\]

где справа записаны массы возникающих частиц. Воспользовавшись инвариантностью величины \(E^{2}-p^{2}\), получить формулу для пороговой кинетической энергии налетающей частицы.

Решение. Из инвариантности указанной величины получим


\begin{equation*}
\left(K_{\text {пор }}+m+M\right)^{2}-K_{\text {пор }}\left(K_{\text {пор }}+2 m\right)=\left(m_{1}+m_{2}+\ldots\right)^{2}, \tag{*}
\end{equation*}


где левая часть равенства записана в Л-системе, а правая - в Ц-системе. Здесь учтено, что при пороговом значении \(K_{\text {пор }}\) образовавшиеся частицы покоятся (в Ц-системе).

Раскрыв скобки в левой части равенства (*) и произведя сокращения, придем к формуле

\[
(m+M)^{2}+2 M K_{\text {пор }}=\left(m_{1}+m_{2}+\ldots\right)^{2} .
\]

Отсюда искомое выражение

\[
K_{\text {пор }}=\frac{\left(m_{1}+m_{2}+\ldots\right)^{2}-(m+M)^{2}}{2 M} .
\]

Для расчетов числитель удобнее преобразовать (как разность квадратов).\\
9.7. Лептонные и барионные заряды. Выяснить с помощью закона сохранения этих зарядов, возможны ли следующие процессы:

\begin{enumerate}
  \item \(n \rightarrow p+e^{-}+v_{e}\),
  \item \(\tilde{v}_{\mu}+p \rightarrow n+\mu^{+}\),
  \item \(\mu^{+} \rightarrow e^{+}+\tilde{v}_{e}+v_{\mu}\),
  \item \(K^{+} \rightarrow \mu^{+}+v_{\mu}+\pi^{0}\),
  \item \(\pi^{-}+n \rightarrow K^{-}+K^{0}\),
  \item \(K^{-}+p \rightarrow \Sigma^{+}+\pi^{-}\).
\end{enumerate}

Решение. Невозможны следующие процессы: (1), так как не сохраняется лептонный заряд ( \(0 \neq 0+1+1\) ); (3), поскольку не сохраняются ни электронный, ни мюонный лептонные заряды; (5), так как не сохраняется барионный заряд ( \(+1 \neq 0+0\) ).\\
9.8. Странность. Какие из приведенных ниже процессов запрещены законом сохранения странности:

\begin{enumerate}
  \item \(\pi^{-}+p \rightarrow \Lambda+K^{0}\),
  \item \(\pi^{-}+p \rightarrow K^{-}+\Sigma^{+}\),
  \item \(\tilde{p}+p \rightarrow \tilde{\Sigma}^{0}+\tilde{K}^{0}+n\),
  \item \(\tilde{p}+n \rightarrow \Lambda+\tilde{\Sigma}^{+}\),
  \item \(\Sigma^{-}+p \rightarrow \Lambda+n\),
  \item \(\pi^{-}+n \rightarrow \Xi^{-}+K^{+}+K^{-}\)?
\end{enumerate}

Решение. Подставим значения странности \(S\) последовательно во все процессы:

\begin{enumerate}
  \item \(0+0 \rightarrow-1+1\),
  \item \(0+0 \rightarrow-1-1\),
  \item \(0+0 \rightarrow+1-1+0\),
  \item \(0+0 \rightarrow-1+1\),
  \item \(-1+0 \rightarrow-1+0\),
  \item \(0+0 \rightarrow-2+1-1\).
\end{enumerate}

Видно, что суммарное значение странности не сохраняется только в процессах (2) и (6). По этой причине они запрещены.\\
9.9. Какие каналы приведенных ниже распадов запрещены и по какой причине:\\
a) \(\Sigma^{-} \xrightarrow{n+\pi^{-}} \begin{aligned} & \lambda+\pi^{-}\end{aligned}\)\\
б) \(\begin{aligned} & \Xi^{-} \\ & \longrightarrow \Lambda+2 \pi^{-} \\ & \longrightarrow\end{aligned} \quad\) п \(\begin{aligned} & p+p+\pi^{-}\end{aligned}\)

Решение. а) Запрещен канал (2) - энергетически: \(m_{\Sigma}<m_{\Lambda}+m_{\pi}\); б) запрещен канал (1) - законом сохранения странности \(S\) :

\[
-2 \nrightarrow 0+0+0, \quad \text { т. е. }|\Delta S|=2 .
\]

9.10. Кварки. Установить с помощью табл. 9.4 кварковый состав \(K^{+}\)-мезона, а также гиперонов \(\Lambda^{0}\) и \(\Omega^{-}\).\\
Решение. Мезоны должны состоять из кварка и антикварка, поскольку их барионный заряд \(B=0\). В случае \(K^{+}\)-мезона \(Q=1\), \(B=0, S=+1\). Это возможно лишь в случае \(K^{+}(u \widetilde{s})\).\\
У гиперонов барионный заряд \(B=1\), странность \(\Lambda\)-гиперона \(S=-1\), а у \(\Omega^{-}\)-гиперона \(S=-3\). Каждый кварк имеет \(B=1 / 3\), значит эти гипероны должны состоять из трех кварков. Кроме того, у \(\Lambda^{0}\)-гиперона \(Q=0\). Это возможно лишь в случае \(\Lambda^{0}(u d s)\).\\
У \(\Omega^{-}\)-гиперона \(Q=-1, S=-3\). Это возможно реализовать только с помощью трех кварков: \(\Omega^{-}(s s s)\).

\section*{1. Основные соотношения релятивистской динамики}
Релятивистский импульс частицы:


\begin{equation*}
p=\frac{m v}{\sqrt{1-(v / c)^{2}}} . \tag{П.1}
\end{equation*}


Связь полной энергии \(E\) с кинетической \(K\) :


\begin{equation*}
E=m c^{2}+K . \tag{П.2}
\end{equation*}


Соотношение между полной энергией \(E\) и импульсом \(p\) :


\begin{align*}
& E^{2}-p^{2} c^{2}=m^{2} c^{4}=\operatorname{inv},  \tag{П.З}\\
& p=E v / c^{2},  \tag{П.4}\\
& p c=\sqrt{K\left(K+2 m c^{2}\right)} . \tag{П.5}
\end{align*}


Эти соотношения в настоящее время принято записывать в более компактном виде, используя следующие сокращенные обозначения:

\begin{enumerate}
  \item величины \(m c^{2}\) и \(p c\) обозначают просто как \(m\) и \(p\), выражая их в энергетических единицах (например, в МэВ);
  \item все скорости выражают в единицах скорости света и обозначают \(\beta\) :
\end{enumerate}

\[
\beta=v / c .
\]

Эти обозначения резко упрощают вид самих формул, а также все преобразования и расчеты. Приведем предыдущие формулы в этих обозначениях:


\begin{align*}
& p=\frac{m \beta}{\sqrt{1-\beta^{2}}},  \tag{П.1́}\\
& E=m+K,  \tag{П.2'}\\
& E^{2}-p^{2}=m^{2}=\mathrm{inv},  \tag{П.3'}\\
& p=E \beta,  \tag{П.4́}\\
& p=\sqrt{K(K+2 m)} . \tag{П.5'}
\end{align*}


\section*{2. Вывод формулы (2.1)}
Из закона сохранения энергии следует, что модуль импульса рассеянной частицы остается таким же, как и до рассеяния, поскольку ядро, на котором происходит рассеяние, мы считаем неподвижным (из-за большой массы). Отсюда модуль приращения импульса рассеянной частицы (рис. П.1, б):


\begin{equation*}
|\Delta \mathbf{p}|=2 p_{0} \sin (\theta / 2) . \tag{П.6}
\end{equation*}


\begin{figure}[h]
\begin{center}
  \includegraphics[alt={},max width=\textwidth]{a57b3a57-78b5-4292-b0ab-4bd9f2ffcf56-247_346_860_516_185}
\captionsetup{labelformat=empty}
\caption{Рис. П. 1}
\end{center}
\end{figure}

С другой стороны, из рис. П.1, а следует, что


\begin{equation*}
|\Delta \mathbf{p}|=\int F_{n} \mathrm{~d} t=\int \frac{q q_{0}}{r^{2}} \cos \alpha \cdot \mathrm{~d} t, \tag{П.7}
\end{equation*}


где \(F_{n}\) - проекция кулоновской силы на направление \(\Delta \mathbf{p}\) (или \(\mathbf{n}\) ), действующей на налетающую частицу ( \(q\) ) со стороны ядра \(\left(q_{0}\right)\). Перепишем интеграл (П.7) в иной форме, учитывая, что согласно рис. П.1, \(a\) углы \(\alpha, \varphi, \theta\) связаны соотношением \(\alpha+\varphi= =(\pi-\theta) / 2\), откуда

\[
\alpha=\frac{\pi}{2}-\frac{\theta}{2}-\varphi .
\]

Тогда \(\cos \alpha=\sin (\varphi+θ / 2)\) и интеграл (П.7) после замены \(\mathrm{d} t=\mathrm{d} \varphi / \dot{\varphi}\) можно представить в виде


\begin{equation*}
|\Delta \mathbf{p}|=\int \frac{q q_{0}}{r^{2}} \frac{\sin (\varphi+\theta / 2) \mathrm{d} \varphi}{\dot{\varphi}} . \tag{П.8}
\end{equation*}


Замена \(\mathrm{d} t\) сделана для того, чтобы в знаменателе получить величину \(r^{2} \dot{\varphi}\). Она связана с моментом импульса \(M_{z}\) относительно оси \(Z\), проходящей через ядро \(q_{0}\) и перпендикулярной плос-

кости рисунка. Действительно, \(M_{z}=m r v_{\varphi}=m r^{2} \dot{\varphi}\), где \(v_{\varphi}\) проекция скорости частицы на орт \(\mathbf{e}_{\varphi}\), перпендикулярный ради-усу-вектору \(\mathbf{r}\). Момент силы, действующий на налетающую частицу (относительно ядра \(q_{0}\) ) все время равен нулю. Поэтому момент \(M_{z}\) сохраняется и равен своему первоначальному значению \(b p\), т. е. \(r^{2} \dot{\varphi}=b v_{0}\).

Теперь проинтегрируем (П.8) по φ от 0 до \(\pi-\theta\). В результате получим:


\begin{equation*}
|\Delta \mathbf{p}|=\frac{q q_{0}}{b v_{0}} 2 \cos \frac{\theta}{2} \tag{П.9}
\end{equation*}


Из сопоставления (П.9) с (П.6) получаем искомое соотношение:


\begin{equation*}
\operatorname{tg} \frac{\theta}{2}=\frac{q q_{0}}{2 b K} . \tag{П.10}
\end{equation*}


где, напомним, \(b\) - прицельный параметр, \(K\) - кинетическая энергия налетающей частицы вдали от ядра.

\begin{enumerate}
  \setcounter{enumi}{2}
  \item Соотношения между единицами некоторых величин
\end{enumerate}

\begin{center}
\begin{tabular}{|l|l|l|l|}
\hline
Величина & СИ & Гауссова система (СГС) & Отношение ВГ. ИЁ ВГ. И\^{}И \\
\hline
Сила \(F\) & H & дин & \(10^{5}\) \\
\hline
Работа \(A\), энергия \(E\) & Дж & эрг & \(10^{7}\) \\
\hline
Импульс \(p\) & кг•м/с & г.см/с & \(10^{5}\) \\
\hline
Момент импульса \(M\) & Дж•с & эрг•с & \(10^{7}\) \\
\hline
Заряд \(q\) & Кл & ед.СГСЭ & \(3 \cdot 10^{9}\) \\
\hline
Потенциал φ & B & ед.СГСЭ & 1/300 \\
\hline
Напряженность поля \(E\) & B/m & ед.СГСЭ & \(1 /\left(3 \cdot 10^{4}\right)\) \\
\hline
Электрический момент \(p_{e}\) & Кл.м & ед.СГСЭ & \(3 \cdot 10^{11}\) \\
\hline
Сила тока I & A & ед.СГСЭ & \(3 \cdot 10^{9}\) \\
\hline
Магнитная индукция \(B\) & Тл & Гс & \(10^{4}\) \\
\hline
Магнитный момент \(p_{\mathrm{m}}\) & \(\mathrm{A} \cdot \mathrm{m}^{2}\) & ед.СГСМ & \(10^{3}\) \\
\hline
\end{tabular}
\end{center}

\section*{Внесистемные единицы}
\begin{center}
\begin{tabular}{||l|c||}
\hline\hline
1 год \(=3,11 \cdot 10^{7} \mathrm{c}\) & 1 а.е. м. \(=\left\{\begin{array}{l}1,66 \cdot 10^{-24} \text { г } \\ 1,66 \cdot 10^{-27} \text { кг } \\ 931,50 \text { МэВ }\end{array}\right.\) \\
1 А (ангстрем) \(=10^{-8} \mathrm{~cm}\) & 1 эВ \(=\left\{\begin{array}{l}1,6 \cdot 10^{-12} \text { эрг } \\ 1,6 \cdot 10^{-19} \text { Дж }\end{array}\right.\) \\
\hline
\end{tabular}
\end{center}

\section*{4. Формулы некоторых величин в гауссовой системе и в СИ}
\begin{center}
\begin{tabular}{|l|l|l|}
\hline
Величина & Гауссова система & СИ \\
\hline
Закон Кулона & \(F=\frac{q_{1} q_{2}}{r^{2}}\) & \(\frac{1}{4 \pi \varepsilon_{0}} \frac{q_{1} q_{2}}{r^{2}}\) \\
\hline
Потенциал поля точечного заряда \(q\) & \(\varphi=\frac{q}{r}\) & \(\frac{1}{4 \pi \varepsilon_{0}} \frac{q}{r}\) \\
\hline
Магнитный момент контура с током & \(p_{\mathrm{m}}=\frac{1}{c} I S\) & IS \\
\hline
Сила Лоренца & \(\mathbf{F}=q[\mathbf{v B}]\) & \(\frac{q}{c}\) [ vB] \\
\hline
Постоянная Ридберга \(R, \mathrm{c}^{-1}\) & \(R=\frac{m e^{4}}{2 \hbar^{3}}\) & \(\left(\frac{1}{4 \pi \varepsilon_{0}}\right)^{2} \frac{m e^{4}}{2 \hbar^{3}}\) \\
\hline
\(R^{\prime}=\frac{R}{2 \pi c}, \mathrm{~cm}^{-1}\) & \(R^{\prime}=\frac{m e^{4}}{4 \pi c \hbar^{3}}\) & \(\left(\frac{1}{4 \pi \varepsilon_{0}}\right)^{2} \frac{m e^{4}}{4 \pi c \hbar^{3}}\) \\
\hline
Первый боровский радиус & \(r_{1}=\frac{\hbar^{2}}{m e^{2}}\) & \(4 \pi \varepsilon_{0} \frac{\hbar^{2}}{m e^{2}}\) \\
\hline
Энергия связи электрона в атоме водорода & \(E_{\text {св }}=\frac{m e^{4}}{2 \hbar^{2}}\) & \(\left(\frac{1}{4 \pi \varepsilon_{0}}\right)^{2} \frac{m e^{4}}{4 \hbar^{2}}\) \\
\hline
Магнетон Бора & \(\mu_{\mathrm{E}}=\frac{e \hbar}{2 m_{e} c}\) & \(\frac{e \hbar}{2 m_{e}}\) \\
\hline
Гиромагнитное отношение & \(g=\frac{e}{2 m c}\) & \(\frac{e}{2 m}\) \\
\hline
\end{tabular}
\end{center}

\section*{5. Массы легких нуклидов}
\begin{center}
\begin{tabular}{|l|l|l|l|l|l|}
\hline
Z & Нуклид & Избыток массы нуклида \(M-A\), а.е.м. & \(Z\) & Нуклид & Избыток массы нуклида \(M-A\), а.е.м. \\
\hline
0 & \(n\) & 0,00867 & 6 & \({ }^{11} \mathrm{C}\) & 0,01143 \\
\hline
1 & \({ }^{1} \mathrm{H}\) & 0,00783 &  & \({ }^{12} \mathrm{C}\) & 0 \\
\hline
 & \({ }^{2} \mathrm{H}\) & 0,01410 &  & \({ }^{13} \mathrm{C}\) & 0,00335 \\
\hline
 & \({ }^{3} \mathrm{H}\) & 0,01605 & 7 & \({ }^{13} \mathrm{~N}\) & 0,00574 \\
\hline
2 & \({ }^{3} \mathrm{He}\) & 0,01603 &  & \({ }^{14} \mathrm{~N}\) & 0,00307 \\
\hline
 & \({ }^{4} \mathrm{He}\) & 0,00260 &  & \({ }^{15} \mathrm{~N}\) & 0,00011 \\
\hline
3 & \({ }^{6} \mathrm{Li}\) & 0,01513 & 8 & \({ }^{15} \mathrm{O}\) & 0,00307 \\
\hline
 & \({ }^{7} \mathrm{Li}\) & 0,01601 &  & \({ }^{16} \mathrm{O}\) & -0,00509 \\
\hline
4 & \({ }^{7} \mathrm{Be}\) & 0,01693 &  & \({ }^{17} \mathrm{O}\) & -0,00087 \\
\hline
 & \({ }^{8} \mathrm{Be}\) & 0,00531 & 9 & \({ }^{19} \mathrm{~F}\) & -0,00160 \\
\hline
 & \({ }^{9} \mathrm{Be}\) & 0,01219 & 10 & \({ }^{20} \mathrm{Ne}\) & -0,00756 \\
\hline
 & \({ }^{10} \mathrm{Be}\) & 0,01354 & 11 & \({ }^{23} \mathrm{Na}\) & -0,01023 \\
\hline
5 & \({ }^{10} \mathrm{~B}\) & 0,01294 &  & \({ }^{24} \mathrm{Na}\) & -0,00903 \\
\hline
 & \({ }^{11} \mathrm{~B}\) & 0,00930 & 12 & \({ }^{24} \mathrm{Mg}\) & -0,01496 \\
\hline
\multicolumn{6}{|l|}{Примечание. Здесь \(M\) - масса нуклида в а.е.м., \(A\) - массовое число.} \\
\hline
\end{tabular}
\end{center}

\section*{6. Греческий алфавит}
\begin{center}
\begin{tabular}{|l|l|l|}
\hline
А, \(\alpha\) - альфа & I, ι - йота & P, \(\rho\) - po \\
\hline
B, \(\beta\) - бета & K, \(x\) - каппа & \(\Sigma, \sigma\) - сигма \\
\hline
Г, \(\gamma\) - гамма & \(\Lambda, \lambda\) - ламбда & T, \(\tau\) - тау \\
\hline
\(\Delta, \delta\) - дельта & М, μ - мю & Y, υ- ипсилон \\
\hline
Ε, \(ε\) - эпсилон & \(\mathrm{N}, v\) - ню & \(\Phi, φ\) - фи \\
\hline
\(\mathrm{Z}, \zeta\) - дзета & \(\Xi, \xi\) - кси & X, \(\chi\) - хи \\
\hline
Η, η - эта & О, о - омикрон & \(\Psi, \Psi\) - пси \\
\hline
\(\Theta, \theta, \vartheta\) - тета & П, π - пи & \(\Omega, \omega\) - омега \\
\hline
\end{tabular}
\end{center}

\section*{7. Некоторые физические константы}
\begin{center}
\begin{tabular}{|l|l|}
\hline
Скорость света в вакууме & \(c=2,998 \cdot 10^{8} \mathrm{~m} / \mathrm{c}\) \\
\hline
Гравитационная постоянная & \(\gamma=\left\{\begin{array}{l}6,67 \cdot 10^{-8} \mathrm{~cm}^{3} /\left(г \cdot \mathrm{c}^{2}\right) \\ 6,67 \cdot 10^{-11} \mathrm{M}^{3} /\left(к г \cdot \mathrm{c}^{2}\right)\end{array}\right.\) \\
\hline
Постоянная Авогадро & \(N_{A}=6,022 \cdot 10^{23}\) моль \(^{-1}\) \\
\hline
Универсальная газовая постоянная & \(R=\left\{\begin{array}{l}8,314 \cdot 10^{7} \text { эрг / (К • моль) } \\ 8,314 \text { Дж / (К • моль) }\end{array}\right.\) \\
\hline
Постоянная Больцмана & \(k=\left\{\begin{array}{l}1,3807 \cdot 10^{-16} \text { эрг } / \text { К } \\ 1,3807 \cdot 10^{-23} \text { Дж /К } \\ 0,8617 \cdot 10^{-4} \text { эВ /К }\end{array}\right.\) \\
\hline
Элементарный заряд & \(e=\left\{\begin{array}{l}4,803 \cdot 10^{-10} \text { СГСЭ } \\ 1,602 \cdot 10^{-19} \text { Кл }\end{array}\right.\) \\
\hline
Масса электрона & \(m_{e}=\left\{\begin{array}{l}0,911 \cdot 10^{-27} \text { г } \\ 5,486 \cdot 10^{-4} \text { а.е. м. } \\ 0,511 \text { МэВ }\end{array}\right.\) \\
\hline
Удельный заряд электрона & \(\frac{e}{m_{e}}=\left\{\begin{array}{l}5,27 \cdot 10^{17} \text { СГСЭ } \\ 1,76 \cdot 10^{11} \text { Кл / кг }\end{array}\right.\) \\
\hline
Масса протона & \(m_{p}=\left\{\begin{array}{l}1,6726 \cdot 10^{-24} \text { г } \\ 1,007276 \text { а. е. м. } \\ 938,28 \text { МэВ }\end{array}\right.\) \\
\hline
Удельный заряд протона & \(\frac{e}{m_{p}}=\left\{\begin{array}{l}2,87 \cdot 10^{14} \text { СГСЭ } \\ 0,959 \cdot 10^{8} \text { Кл } / \text { кг }\end{array}\right.\) \\
\hline
Постоянная Планка & \(\hbar=\left\{\begin{array}{l}1,0546 \cdot 10^{-27} \text { эрг • с } \\ 1,0546 \cdot 10^{-34} \text { Дж • с } \\ 0,6582 \cdot 10^{-15} \quad \text { эВ • с }\end{array}\right.\) \\
\hline
Постоянная Ридберга & \( \begin{aligned} & R=2,067 \cdot 10^{16} \mathrm{c}^{-1} \\ & R^{\prime}=R / 2 \pi c=1,097 \cdot 10^{5} \mathrm{~cm}^{-1} \end{aligned} \) \\
\hline
Первый боровский радиус & \(r_{1}=0,529 \cdot 10^{-8} \mathrm{~cm}\) \\
\hline
Энергия связи электрона в атоме водорода & \(E=13,56\) эВ \\
\hline
Комптоновская длина волны электрона & \(\lambda_{C}=2,426 \cdot 10^{-10} \mathrm{~cm}\) \\
\hline
Классический радиус электрона & \(r_{e}=2,82 \cdot 10^{-13} \mathrm{~cm}\) \\
\hline
\end{tabular}
\end{center}

\begin{table}[h]
\begin{center}
\captionsetup{labelformat=empty}
\caption{продолжение табл. 9}
\begin{tabular}{|l|l|}
\hline
Магнетон Бора & \(\mu \cdot=\left\{\begin{array}{l}0,9274 \cdot 10^{-20} \text { эрг /Гс } \\ 0,9274 \cdot 10^{-23} \text { Дж /Тл } \\ 0,5788 \cdot 10^{-8} \text { эВ /Гс }\end{array}\right.\) \(\mu_{\text {я }}=\left\{\begin{array}{l}5,051 \cdot 10^{-24} \text { эрг /Гс } \\ 5,051 \cdot 10^{-27} \text { Дж /Тл } \\ 3,152 \cdot 10^{-12} \text { эВ /Гс }\end{array}\right.\) \\
\hline
Магнитный момент протона нейтрона & \( \begin{aligned} & \mu_{p}=2,7928 \mu_{\Omega} \\ & \mu_{n}=-1,913 \mu_{\Omega} \end{aligned} \) \\
\hline
Электрическая постоянная & \( \begin{aligned} & \varepsilon_{0}=0,885 \cdot 10^{-11} \Phi / \mathrm{M} \\ & 1 / 4 \pi \varepsilon_{0}=9 \cdot 10^{9} \mathrm{M} / \Phi \end{aligned} \) \\
\hline
Магнитная постоянная & \( \begin{aligned} & \mu_{0}=1,257 \cdot 10^{-6} \Gamma \mathrm{H} / \mathrm{M} \\ & \mu_{0} / 4 \pi=10^{-7} \Gamma \mathrm{H} / \mathrm{M} \end{aligned} \) \\
\hline
\end{tabular}
\end{center}
\end{table}

\section*{Предметный указатель}
Адроны 226\\
Активность 195

\begin{itemize}
  \item удельная 195
\end{itemize}

Альфа-распад 197\\
Аннигиляция 228\\
Антинейтрон 227\\
Антипротон 227\\
Античастицы 227\\
Аромат кварка 237\\
Барионы 226\\
Барн 208\\
Барьер потенциальный 103\\
Беккерель 195\\
Бета-распад 199

\begin{itemize}
  \item спектр 200
\end{itemize}

Бозоны 225\\
Взаимодействия фундаментальные 224

\begin{itemize}
  \item спин-орбитальные 146
\end{itemize}

Волна де-Бройля 60\\
Время жизни ядра среднее 196\\
Время жизни элементарных частиц 225

\begin{itemize}
  \item ядерное 209
\end{itemize}

Выход ядерной реакции 207, 208

Гамма-распад 202\\
Гипероны 227\\
Гипотеза де-Бройля 60

\begin{itemize}
  \item спина 141
  \item Юкавы 192
\end{itemize}

Глюоны 226, 239\\
Гравитоны 226\\
Граница коротковолновая 20

\begin{itemize}
  \item красная фотоэффекта 13, 15
  \item серии 43
\end{itemize}

Действие 11, 72\\
Дейтерий 185\\
Дейтрон 185

Дефект массы 187\\
Дисперсия дебройлевских волн 61\\
Дифракция рентгеновских лучей 65

\begin{itemize}
  \item электронов 64,65
\end{itemize}

Длина волны дебройлевская 60

\begin{itemize}
  \item 
  \begin{itemize}
    \item комптоновская 27
  \end{itemize}
\end{itemize}

Дуализм корпускулярноволновой 24,60

Закон Мозли 158

\begin{itemize}
  \item радиоактивного распада 195
\end{itemize}

Закономерности спектральные 42

\begin{itemize}
  \item тонкой структуры 145,146
\end{itemize}

Закон сохранения барионного заряда 230

\begin{itemize}
  \item 
  \begin{itemize}
    \item изоспина 236
  \end{itemize}
  \item 
  \begin{itemize}
    \item лептонного заряда 231
  \end{itemize}
  \item 
  \begin{itemize}
    \item странности 232
  \end{itemize}
  \item 
  \begin{itemize}
    \item четности 234
  \end{itemize}
\end{itemize}

Законы сохранения точные и приближенные 229\\
Заряд барионный 230

\begin{itemize}
  \item лептонный 231
  \item электрический 230
\end{itemize}

Значения собственные 89,118

\begin{itemize}
  \item физических величин средние 111
\end{itemize}

Излучение равновесное 9

\begin{itemize}
  \item рентгеновское тормозное 19
  \item 
  \begin{itemize}
    \item характеристическое 157
  \end{itemize}
\end{itemize}

Изоспин 236\\
Изотопы 184\\
Инверсия 233

Канал ядерной реакции 207\\
Квантование 89

\begin{itemize}
  \item атома водорода 131
  \item пространственное 122
\end{itemize}

Кварки 237\\
\(K\)-захват 199

\begin{itemize}
  \item серия 157
\end{itemize}

Классификация взаимодействий 224, 225\\
Колебания молекул 99\\
Коммутативность операторов 113\\
Конверсия внутренняя 202\\
Конфигурация электронная 154\\
Коэффициент прозрачности 103\\
Край полосы поглощения 157\\
Красота (прелесть) 233, 237\\
Кратность вырождения 133\\
Критерий классического описания 72

\begin{itemize}
  \item наличия распределения 116
\end{itemize}

Кюри 195

Лептоны 226

Магнетон Бора 52

\begin{itemize}
  \item ядерный 183
\end{itemize}

Масса нейтрона 183

\begin{itemize}
  \item приведенная 50
  \item протона 183
\end{itemize}

Мезоны 226\\
Метод задерживающего поля 45

\begin{itemize}
  \item изохромат 20
\end{itemize}

Механизм взаимодействия нуклонов 191\\
Множитель (фактор) Ланде 170\\
Модели ядер 193\\
Модель адронов кварковая 237

\begin{itemize}
  \item атома водорода боровская 47
  \item 
  \begin{itemize}
    \item ядерная 36
  \end{itemize}
  \item цветных кварков 239
\end{itemize}

Молекула двухатомная 99, 123, 124\\
Момент импульса 115,118

\begin{itemize}
  \item магнитный 51, 167
  \item орбитальный 52,147
  \item полный 143, 148
  \item спиновый 142, 147
  \item магнитный полный 170
  \item 
  \begin{itemize}
    \item орбитальный 167
  \end{itemize}
  \item 
  \begin{itemize}
    \item спиновый 169
  \end{itemize}
\end{itemize}

Мультиплетность 145\\
Мультиплеты спектральные 141

\begin{itemize}
  \item изотопические 235
\end{itemize}

Насыщение ядерных сил 190, 191\\
Нейтрино 184, 200

\begin{itemize}
  \item мюонное 231
  \item таонное 231
  \item электронное 231
\end{itemize}

Нейтрон 184\\
Несохранение четности 235\\
Нуклиды 184\\
Нуклоны 183, 227

Обозначения спектральные 149\\
Оболочка 151

\begin{itemize}
  \item замкнутая 152
  \item заполненная 152
  \item ядерная 194
\end{itemize}

Оператор 112

\begin{itemize}
  \item квадрата импульса 114
  \item кинетической энергии 114
  \item координаты 113
\end{itemize}

Оператор линейный 113

\begin{itemize}
  \item момента импульса 115
  \item полной энергии (гамильтониан) 114
  \item проекции импульса 113
  \item 
  \begin{itemize}
    \item момента импульса 115, 120
  \end{itemize}
  \item углового момента 120
\end{itemize}

Операторы коммутирующие 113\\
Опыт Боте 21

\begin{itemize}
  \item Боте и Гейгера 28
  \item By 235
  \item Дэвиссона и Джермера 63
  \item Комптона 24
  \item Паунда и Ребки 206
  \item Резерфорда 36
  \item Томсона и Тартаковского 68
  \item Фабриканта, Бибермана, Сушкина 69
  \item Франка и Герца 45
  \item Штерна и Герлаха 168
\end{itemize}

Особенности ядерных сил 191\\
Остов атома 137, 140\\
Осциллятор квантовый 96,97

Отношение гиромагнитное 52, 167

Параметр прицельный 37\\
Переносчики взаимодействия 226\\
Период полураспада 196\\
Плотность вероятности 86

\begin{itemize}
  \item излучения спектральная 10
  \item потока вероятности 101
\end{itemize}

Поглощение резонансное \(\gamma\) лучей\\
205\\
Подоболочка 151\\
Полосы колебательновращательные 124\\
Поправка ридберговская 139\\
Порог реакции 213

\begin{itemize}
  \item фотоэффекта 15
\end{itemize}

Постоянная Планка 11

\begin{itemize}
  \item распада 195
  \item Ридберга 42
\end{itemize}

Постулаты Бора 44

\begin{itemize}
  \item квантовой теории основные 113\\
Потенциал внутренний металла 67\\
Потенциалы резонансные 47\\
Правила отбора квантового числа вращательного 123
  \item 
  \begin{itemize}
    \item 
    \begin{itemize}
      \item 
      \begin{itemize}
        \item колебательного \(v 98\)
      \end{itemize}
    \end{itemize}
  \end{itemize}
  \item 
  \begin{itemize}
    \item 
    \begin{itemize}
      \item 
      \begin{itemize}
        \item орбитального \(l 140\)
      \end{itemize}
    \end{itemize}
  \end{itemize}
  \item 
  \begin{itemize}
    \item 
    \begin{itemize}
      \item 
      \begin{itemize}
        \item полного момента \(j 145\)
      \end{itemize}
    \end{itemize}
  \end{itemize}
  \item 
  \begin{itemize}
    \item квантовых чисел \(L, S, J 150\)\\
---- магнитных \(m_{L}, m_{S}, m_{J}\) 171,176
  \end{itemize}
  \item Хунда 155
\end{itemize}

Правило квантования Бора 47

\begin{itemize}
  \item частот Бора 44
\end{itemize}

Преломление дебройлевских волн 66, 82\\
Принцип бесцветности адронов 238

\begin{itemize}
  \item неопределенности 73
  \item Паули 150, 151
  \item суперпозиции 87
\end{itemize}

Проекция изоспина 236

\begin{itemize}
  \item момента импульса 115,120
\end{itemize}

Протон 183\\
Прохождение частицы через порог 100, 101\\
Пси-функция 71,85

\begin{itemize}
  \item нормированная 87
\end{itemize}

Работа выхода 14\\
Радиоактивность 194\\
Радиоспектроскопия 178\\
Радиус боровский 48,134

\begin{itemize}
  \item ядра 185
\end{itemize}

Размер атома водорода 77\\
Разность потенциалов задерживающая 13, 17

\begin{itemize}
  \item 
  \begin{itemize}
    \item контактная 16
  \end{itemize}
\end{itemize}

Распад электронный 199

\begin{itemize}
  \item позитронный 199
\end{itemize}

Рассеяние альфа-частиц 36\\
Реакция ядерная 207

\begin{itemize}
  \item 
  \begin{itemize}
    \item прямая (срыва) 209
  \end{itemize}
  \item 
  \begin{itemize}
    \item через составное ядро 208
  \end{itemize}
  \item 
  \begin{itemize}
    \item экзоэнергетическая 210
  \end{itemize}
  \item 
  \begin{itemize}
    \item эндоэнергетическая 210
  \end{itemize}
\end{itemize}

Резонанс электронный парамагнитный (ЭПР) 177\\
Резонансы барионные 227

\begin{itemize}
  \item мезонные 226
\end{itemize}

Рождение пар 228\\
Ротатор 123\\
Связь нормальная 148

\begin{itemize}
  \item Рессель-Саундерса 148, 174
  \item \(j\) - \(j 159\)
\end{itemize}

Серии спектральные 42,49\\
Серия Бальмера 42

\begin{itemize}
  \item Лаймана 43
  \item Пашена 43
\end{itemize}

Серии щелочных металлов 140 , 141\\
Сечение дифференциальное 39

\begin{itemize}
  \item эффективное 39, 207
\end{itemize}

Силы взаимодействия короткодействующие 190

\begin{itemize}
  \item 
  \begin{itemize}
    \item ядерные 41, 191
  \end{itemize}
\end{itemize}

Символы состояний 133, 149

\begin{itemize}
  \item термов 145
\end{itemize}

Скорость дебройлевских волн групповая 61

\begin{itemize}
  \item 
  \begin{itemize}
    \item 
    \begin{itemize}
      \item фазовая 61
    \end{itemize}
  \end{itemize}
\end{itemize}

Сложение угловых моментов 147\\
Смещение гравитационное 205

\begin{itemize}
  \item зеемановское 172
  \item комптоновское 25
  \item лоренцево 172
\end{itemize}

Соотношения неопределенностей 73, 74\\
Состояние вырожденное 133

\begin{itemize}
  \item нечетное, четное 234
  \item основное 48
  \item стационарное 44,88
  \item частицы 85
\end{itemize}

Состояния собственные 116\\
Спектр вращательных уровней 124

\begin{itemize}
  \item колебательно-вращательный 99, 124
  \item линейчатый 42
  \item молекулы вращательный 124
  \item поглощения рентгеновского излучения 159
  \item характеристический 157
\end{itemize}

Спин 141

\begin{itemize}
  \item изотопический (изоспин) 236
  \item ядра 185
\end{itemize}

Столкновение лобовое 53

\begin{itemize}
  \item нелобовое 54
\end{itemize}

Странность 232\\
Структура сверхтонкая спектральных линий 186

\begin{itemize}
  \item тонкая 144
  \item 
  \begin{itemize}
    \item рентгеновских спектров 160
  \end{itemize}
  \item 
  \begin{itemize}
    \item спектральных линий 141
  \end{itemize}
  \item 
  \begin{itemize}
    \item уровней щелочных металлов 141, 144, 145\\
Схема уровней атома водорода 49
  \end{itemize}
  \item 
  \begin{itemize}
    \item 
    \begin{itemize}
      \item лития 138
    \end{itemize}
  \end{itemize}
  \item ядерной реакции энергетическая 212
\end{itemize}

Теория Бора 47, 48

\begin{itemize}
  \item фотоэффекта Эйнштейна 14
  \item эффекта Комптона 20
\end{itemize}

Термы 50\\
Типы радиоактивности 197\\
Триплеты 141, 146

Тритий 185\\
Тритон 185

Уравнение Шредингера временное 88,89

\begin{itemize}
  \item 
  \begin{itemize}
    \item стационарное 88,89
  \end{itemize}
\end{itemize}

Уровни возбуждения ядра 214

\begin{itemize}
  \item вращательные 123
  \item колебательные 99
\end{itemize}

Условие нормировки 86\\
Условия естественные (стандартные) 89,117\\
Усреднение физических величин 111\\
Уширение линии естественное 74

Фактор (множитель) Ланде 170\\
Фермионы 225\\
Формула Бальмера 42

\begin{itemize}
  \item 
  \begin{itemize}
    \item обобщенная 43
  \end{itemize}
  \item Брэгга- Вульфа 65
  \item де-Бройля 60
  \item Резерфорда 38
  \item Эйнштейна для фотоэффекта 14\\
Фотон 22, 226\\
Фотоэффект 12\\
Функция волновая 71
  \item собственная 89,118
\end{itemize}

Характеристика атомного ядра 184

Цвет кварка 238\\
\(Ц\)-система (отсчета) 6, 211

Частицы истинно нейтральные 228

\begin{itemize}
  \item 
  \begin{itemize}
    \item элементарные 223, 226, 240
  \end{itemize}
  \item виртуальные 192
  \item квазистабильные 226
  \item стабильные 225, 226
  \item странные 232
\end{itemize}

Чётность 233

\begin{itemize}
  \item внутренняя 234
\end{itemize}

Числа магические 194\\
Число волновое 42

\begin{itemize}
  \item квантовое главное 132
  \item 
  \begin{itemize}
    \item вращательное 123
  \end{itemize}
  \item 
  \begin{itemize}
    \item зарядовое 184
  \end{itemize}
  \item 
  \begin{itemize}
    \item магнитное 121, 132
  \end{itemize}
  \item 
  \begin{itemize}
    \item орбитальное (азимутальное) 119, 132
  \end{itemize}
  \item 
  \begin{itemize}
    \item полного момента 143, 148
  \end{itemize}
  \item 
  \begin{itemize}
    \item спиновое 142
  \end{itemize}
  \item массовое 184
\end{itemize}

Шарм (очарование) 233, 237\\
Ширина уровней \(74,204,215\)\\
Электроны эквивалентные 154\\
Энергия альфа-распада 197

\begin{itemize}
  \item бета-распада 199
  \item возбуждения ядра 215
\end{itemize}

Энергия вращательная 123

\begin{itemize}
  \item квантового осциллятора 97
  \item нулевая 97
  \item связи атома водорода 49
  \item 
  \begin{itemize}
    \item ядра 186
  \end{itemize}
  \item 
  \begin{itemize}
    \item 
    \begin{itemize}
      \item удельная 188
    \end{itemize}
  \end{itemize}
  \item ядерной реакции 210
\end{itemize}

Эффект Доплера 35, 205

\begin{itemize}
  \item Зеемана 171
  \item 
  \begin{itemize}
    \item простой 172
  \end{itemize}
  \item 
  \begin{itemize}
    \item сложный 174
  \end{itemize}
  \item Комптона 25
  \item 
  \begin{itemize}
    \item обратный 33
  \end{itemize}
  \item Мессбауэра 205
  \item Пашена- Бака 175
  \item туннельный 102, 198
\end{itemize}

Ядра магические 194\\
Ядро дочернее 194

\begin{itemize}
  \item материнское 194
  \item промежуточное (составное) 208
  \item радиоактивное 194
\end{itemize}

\begin{center}
\begin{tabular}{|l|l|l|l|l|l|l|l|l|l|l|}
\hline
\multirow[t]{2}{*}{\begin{tabular}{l}
㙜员 \\
1 \\
\end{tabular}} & I & \multicolumn{5}{|l|}{ПЕРИОДИЧЕСКАЯ СИСТЕМА ЭЛЕМЕНТов Д.И.МЕНДЕЛЕЕВА} & vii & \multicolumn{3}{|c|}{VIII} \\
\hline
 & (H) & II & III & IV & v & vi & \({ }_{\text {водогод }}{ }^{1} \mathrm{H}\) & \({ }_{\text {гети }} 2 \mathrm{He}\) & \multicolumn{2}{|c|}{\multirow{3}{*}{}} \\
\hline
2 & " & ,арог18 верилими & \({ }^{\text {гор }}{ }^{5}\) В & улкеод 6 С & 1,2,01 & \({ }_{\text {килсород }}{ }^{8}\) О & атор \({ }^{9}\) F & 18,98930 &  &  \\
\hline
3 & " & 24.395 малг & лломнтне \({ }_{2,69854}\) & \({ }_{13}\) Al & крентий \({ }_{28,966}\) & 000000 зая.376 & корор \({ }^{2}\) Cl & 23,453 &  &  \\
\hline
4 & || 19 & 20 Ca & 21Sc & 22 Ti ткан & \({ }_{23} \mathbf{V}\) " &  & \({ }_{25} \mathrm{Mn}\) 54,9350 MMARAMELL & \({ }_{26}^{26}\) Fe 55,877 & \({ }_{21} \mathrm{CO}_{8}\) & \begin{tabular}{l}
59,32 \\
58,932 \\
\end{tabular} \\
\hline
" & '' & \({ }_{29} \mathrm{Cu}\) меко \({ }^{63,346}\) & \({ }_{30} \mathrm{Zn} { }_{6,38}\) & \({ }_{11} \mathrm{Ga}\) галиии 69,735 & \({ }_{32} \mathrm{Ge}\) геммний т. 2,9 & \({ }_{33}\) As мышьк 7,4,216 & \({ }_{\text {вrom }}{ }^{35} \mathbf{B r}\) & 36 Kr \({ }_{8,30}\) &  &  \\
\hline
\multirow[t]{2}{*}{\includegraphics[max width=\textwidth, alt={}]{a57b3a57-78b5-4292-b0ab-4bd9f2ffcf56-258_54_77_771_124}
} & oRb \({ }_{8,5,47}\) РУбый & \({ }_{38}\) Sr 8,622 стоонций & \({ }_{39} \mathbf{Y}\) \includegraphics[max width=\textwidth, alt={}]{a57b3a57-78b5-4292-b0ab-4bd9f2ffcf56-258_22_57_811_644}
 & \({ }^{40}\) Zr 9,1,2 பирконий & \({ }^{41}\) Nb 92,906 ннобий & \({ }_{42}\) Mo 95,94 мотицант & \({ }_{43}\) Tc яვ,966 текении & \({ }_{44}\) Ru T0,0,7 Рртений & \({ }_{45} \mathbf{R h}\) 112,2965 Родиий & \({ }_{46}\) Pd 10.64 плитации \\
\hline
 & 47 Ag & \({ }^{48}\) Cd & \begin{tabular}{l}
\({ }^{49}\) In \\
\includegraphics[max width=\textwidth, alt={}]{a57b3a57-78b5-4292-b0ab-4bd9f2ffcf56-258_49_183_900_599}
 \\
\end{tabular} & \({ }_{50}^{50} \mathbf{S n}\) отвов \({ }^{511 s .99}\) & 51 Sb стремм 412175 & 52 Te телитр 172,60 & под 53 I G.9.945 & \({ }_{54}\) Xe &  &  \\
\hline
" & \begin{tabular}{l}
7111 \({ }_{55}\)/129.95 \\
Cs \(_{\text {и⿻廿一力八 }}\) \\
\end{tabular} & \(\omega_{n}^{*}\) &  & \({ }_{12}\) Hf твля геонний & \({ }_{13}\) Ta \(\mathbf{T a}_{\text {тантал }}\) таям & 14 W 18,355 вольворми & \({ }_{75} \mathbf{R e} { }_{156,27}\) & \begin{tabular}{l}
\({ }_{76}\) Os \\
1902 \\
\end{tabular} & \({ }_{n}\) Ir 12,22 & \(78{ }^{7 \text { pt }}{ }_{\text {плини }}\) 195,09 \\
\hline
 & " & \({ }_{56}\) Ba 13,3.3 & \begin{tabular}{l}
g \\
80Hg \\
\end{tabular} & 81 T1 талики 203,37 & \({ }_{82} \mathbf{P b}\) атней 2002 & \({ }^{84}\) PO попоннии[ грот] & \({ }^{85} \mathbf{~ A t}\) actat[214] & 66Rn & радон \({ }^{[122]} { }^{1221}\) &  \\
\hline
7 & » 87 Fr 1223 ФРАЧий & 88 Ra 260.0254 ралии & 89AC** рат)Актитий & \begin{tabular}{l}
\({ }_{104} \mathbf{K u}\) \\
\includegraphics[max width=\textwidth, alt={}]{a57b3a57-78b5-4292-b0ab-4bd9f2ffcf56-258_42_194_1202_793}
 \\
\end{tabular} & 105 & \({ }^{10}\) ов &  &  & HOMEP АТОмная масса & 81 T1 таллат \\
\hline
\end{tabular}
\end{center}

*Л А H T АН О и Д Ы

\begin{center}
\begin{tabular}{|l|l|l|l|l|l|l|l|l|l|l|l|l|l|}
\hline
58 Ce 140,12 & 59 Pr 140,9077 & 60 Nd 144,24 & 61Pm ПРОМЕТИЙ [145] & 62Sm 150,4 &  & 63 Eu 151,96 & вы Tb 158,9254 ТЕРБИЙ & \({ }_{16230}^{6620}\) Dy & \({ }^{67}\) HO & 68 Er 1626 &  &  & \({ }_{69} \mathbf{T m}{ }_{70} \mathbf{Y b}{ }_{11} \mathbf{L u}\) 168,9342 ТУЛИЙ ИТТЕРБИЙ ЛютЕЦи წ. 7 \\
\hline
 &  &  &  &  &  &  &  &  &  &  &  &  &  \\
\hline
\multicolumn{14}{|c|}{**А к Т ин О и д ы} \\
\hline
\includegraphics[max width=\textwidth, alt={}]{a57b3a57-78b5-4292-b0ab-4bd9f2ffcf56-258_65_142_1533_100}
 & \includegraphics[max width=\textwidth, alt={}]{a57b3a57-78b5-4292-b0ab-4bd9f2ffcf56-258_65_134_1533_257}
 & \includegraphics[max width=\textwidth, alt={}]{a57b3a57-78b5-4292-b0ab-4bd9f2ffcf56-258_74_114_1533_406}
 & \({ }_{93} \mathbf{N}\) 237,0482 & \({ }_{94}\) Purroma [244] & 95 Amm [243] & g6CM \({ }^{24]}\) & \begin{tabular}{l}
\(]_{[2 \pi]}\) \\
Bk БЕРКЛИЙ \\
\end{tabular} & 98ССС. КАЛ [251] & gg ES \({ }_{1234}^{14}\) & \({ }_{100}\) Fm \({ }_{12371}\) & оентии & 112(No) \({ }_{1259}{ }^{(1065)}\) & 103(Lr) Soperetchang \\
\hline
232,0381 & 231,0359 & " 238.09 &  &  &  &  &  &  &  &  &  &  &  \\
\hline
\end{tabular}
\end{center}

\section*{Таблица элементарных частиц}
\begin{center}
\begin{tabular}{|l|l|l|l|l|l|l|l|l|l|l|l|l|l|l|}
\hline
\multicolumn{2}{|c|}{\multirow{2}{*}{Частица}} & \multirow{2}{*}{Символ*} & \multirow[t]{2}{*}{Macca, МэВ} & \multicolumn{5}{|c|}{Заряды} & \multirow{2}{*}{Спин, четность \(I^{P}\)} & \multicolumn{2}{|c|}{Изоспин} & \multirow{2}{*}{Странность S} & \multirow{2}{*}{Среднее время жизни \(\tau\), с} & \multirow{2}{*}{Основные схемы распада частицы} \\
\hline
 &  &  &  & \(Q\) & \(L_{e}\) & \(L_{\mu}\) & \(L_{\tau}\) & B &  & \(T\) & \(T_{z}\) &  &  &  \\
\hline
\multicolumn{2}{|c|}{Фотон} & \(\gamma\) & 0 & 0 & 0 & 0 & 0 & 0 & 1 & - & - & - &  & - \\
\hline
\multirow{6}{*}{Лептоны} & Нейтрино & \(v_{e} \quad \tilde{v}_{e}\) & 0 & 0 & +1 & 0 & 0 & 0 & 1/2 & - & - & - &  & - \\
\hline
 &  & \(v_{\mu} \quad \tilde{v}_{\mu}\) & 0 & 0 & 0 & +1 & 0 & 0 & 1/2 & - & - & - &  & - \\
\hline
 &  & \(v_{\tau} \quad \tilde{v}_{\tau}\) & 0 & 0 & 0 & 0 & +1 & 0 & 1/2 & - & - & - &  & - \\
\hline
 & Электрон & \(e^{-} \quad e^{+}\) & 0,511 & -1 & +1 & 0 & 0 & 0 & \(1 / 2^{+}\) & - & - & - &  & - \\
\hline
 & Мюoн & \(\mu^{-} \quad \mu^{+}\) & 105,66 & -1 & 0 & +1 & 0 & 0 & \(1 / 2^{+}\) & - & - & - & \(2,2 \cdot 10^{-6}\) & \(e^{-} v_{\mu} \tilde{v}_{e}\) \\
\hline
 & Таон & \(\tau^{-} \quad \tau^{+}\) & 1782 & 0 & 0 & 0 & +1 & 0 & \(1 / 2^{+}\) & - & - & - & \(3,5 \cdot 10^{-12}\) & \(\mu^{-} \tilde{v}_{\mu} v_{\tau}, \quad e^{-} \tilde{v}_{e} v_{\tau}\) \\
\hline
\multirow{6}{*}{Мезоны} & Пи-мезоны & \(\pi^{0}\) & 135,0 & 0 & \multicolumn{3}{|c|}{0} & 0 & \(0^{-}\) & 1 & 0 & 0 & \(0,8 \cdot 10^{-16}\) & \(\gamma \gamma\) \\
\hline
 &  & \(\pi^{+} \quad \pi^{-}\) & 139,6 & -1 & \multicolumn{3}{|c|}{0} & 0 & \(0^{-}\) & 1 & \(+1\) & 0 & \(2,6 \cdot 10^{-8}\) & \(\mu^{+} v_{\mu}\) \\
\hline
 & Ка-мезоны & \(K^{+} \quad K^{-}\) & 493,8 & +1 & \multicolumn{3}{|c|}{0} & 0 & \(0^{-}\) & 1/2 & +1/2 & +1 & \(1,2 \cdot 10^{-8}\) & \(\mu^{+} v_{\mu}, \pi^{+} \pi^{0}\) \\
\hline
 &  & \(K^{0} \quad \tilde{K}^{0}\) & 497,8 & 0 & \multicolumn{3}{|c|}{0} & 0 & \(0^{-}\) & 1/2 & -1/2 & -1 & \(5,8 \cdot 10^{-8}\) & \(\pi \mu v_{\mu}\) \\
\hline
 & Эта-мезон & \(\eta\) & 549 & 0 & \multicolumn{3}{|c|}{0} & 0 & \(0^{-}\) & 0 & 0 & 0 & \(2,4 \cdot 10^{-19}\) & \(\pi^{+} \pi^{-}\) \\
\hline
 & Фи-мезон & \(\varphi\) & 1019 & 0 & \multicolumn{3}{|c|}{0} & 0 & 1 & 0 & 0 & 0 & \(\sim 10^{-19}\) & \(\pi^{+} \pi^{-}\) \\
\hline
\multirow{9}{*}{Барионы} & Протон & \(p \quad \tilde{p}\) & 938,26 & +1 & \multicolumn{3}{|c|}{0} & +1 & \(1 / 2^{+}\) & 1/2 & +1/2 & 0 &  & - \\
\hline
 & Нейтрон & \(n \quad \tilde{n}\) & 939,55 & 0 & \multicolumn{3}{|c|}{0} & +1 & \(1 / 2^{+}\) & 1/2 & -1/2 & 0 & \(0,93 \cdot 10^{3}\) & \(p \mathrm{e}^{-\tilde{v}}{ }_{e}\) \\
\hline
 & Ламбда-гиперон & \(\Lambda \quad \tilde{\Lambda}\) & 1115,6 & 0 & \multicolumn{3}{|c|}{0} & +1 & \(1 / 2^{+}\) & 0 & 0 & -1 & \(2,5 \cdot 10^{-10}\) & \(p \pi^{-}, n \pi^{0}\) \\
\hline
 & Сигма-гиперонь & \(\Sigma^{+} \quad \tilde{\Sigma}^{+}\) & 1189,4 & +1 & \multicolumn{3}{|c|}{0} & +1 & \(1 / 2^{+}\) & 1 & +1 & -1 & \(0,8 \cdot 10^{-10}\) & \(p \pi^{0}, n \pi^{+}\) \\
\hline
 & \multirow{2}{*}{} & \(\Sigma^{0} \quad \tilde{\Sigma}^{0}\) & 1198,5 & 0 & \multicolumn{3}{|c|}{0} & +1 & \(1 / 2^{+}\) & 1 & 0 & -1 & \(<1 \cdot 10^{-14}\) & \(\Lambda \gamma\) \\
\hline
 &  & \(\Sigma^{-} \quad \tilde{\Sigma}^{-}\) & 1197,4 & -1 & \multicolumn{3}{|c|}{0} & +1 & \(1 / 2^{+}\) & 1 & -1 & -1 & \(1,5 \cdot 10^{-10}\) & \(n \pi^{-}\) \\
\hline
 & Кси-гипероны & \(\Xi^{0} \quad \tilde{\Xi}^{0}\) & 1314,9 & 0 & \multicolumn{3}{|c|}{0} & +1 & \(1 / 2^{+}\) & 1/2 & +1/2 & -2 & \(3 \cdot 10^{-10}\) & \(\Lambda \pi^{0}\) \\
\hline
 &  & \(\Xi^{-} \quad \tilde{\Xi}^{-}\) & 1321,3 & -1 & \multicolumn{3}{|c|}{0} & +1 & \(1 / 2^{+}\) & 1/2 & -1/2 & -2 & \(1,7 \cdot 10^{-10}\) & \(\Lambda \pi^{-}\) \\
\hline
 & Омега-гиперон & \(\Omega^{-} \quad \tilde{\Omega}^{-}\) & 1672 & -1 & \multicolumn{3}{|c|}{0} & +1 & \(3 / 2^{+}\) & 0 & 0 & -3 & \(1,3 \cdot 10^{-10}\) & \(\Lambda K^{-}, \Xi^{-} \pi^{0}\) \\
\hline
\end{tabular}
\end{center}

\footnotetext{\begin{itemize}
  \item Справа указаны символы соответствующих античастиц
\end{itemize}
}ПРИМЕЧАНИЕ. Античастицы имеют тождественные с частицей значения массы, времени жизни, спина, четности и изоспина, но противоположные по знаку значения электрического \(Q\), лептонного \(L\) и барионного \(B\) зарядов, проекции изоспина \(T_{z}\) и странности \(S\).

Учебное электронное издание\\
Иродов Игорь Евгеньевич

Минимальные системные требования определяются соответствующими требованиями программы Adobe Reader версии не ниже 10-й для операционных систем Windows, Android, iOS, Windows Phone и BlackBerry

\begin{itemize}
  \item 
\end{itemize}

\begin{itemize}
  \item 
\end{itemize}

\begin{itemize}
  \item 
\end{itemize}

\begin{itemize}
  \item 
\end{itemize}

\begin{itemize}
  \item 
\end{itemize}

\begin{itemize}
  \item 
\end{itemize}

\begin{itemize}
  \item 
\end{itemize}

\begin{itemize}
  \item 
\end{itemize}

\begin{itemize}
  \item 
\end{itemize}

\begin{itemize}
  \item 
\end{itemize}

\begin{itemize}
  \item 
\end{itemize}

\begin{itemize}
  \item 
\end{itemize}

\begin{itemize}
  \item 
\end{itemize}

\begin{itemize}
  \item 
\end{itemize}

\begin{itemize}
  \item 
\end{itemize}

\begin{itemize}
  \item 
\end{itemize}

\begin{itemize}
  \item 
\end{itemize}

\begin{itemize}
  \item 
\end{itemize}

\begin{itemize}
  \item 
\end{itemize}

\begin{itemize}
  \item 
\end{itemize}

\begin{itemize}
  \item 
\end{itemize}

\begin{itemize}
  \item 
\end{itemize}

\begin{itemize}
  \item 
\end{itemize}

\begin{itemize}
  \item 
\end{itemize}

\begin{itemize}
  \item 
\end{itemize}

\begin{itemize}
  \item 
\end{itemize}

\begin{itemize}
  \item 
\end{itemize}


\end{document}